Климентий Аркадиевич Тимирязев


ЛУИ ПАСТЕР

1822--1895

I

Теория и практика,  чистая наука и прикладная  наука...Как часто, чуть
не  на  каждом шагу,  приходится  слышать  это сопоставление,  причем,
если указывающий  на него  полагает, что  его устами  гласит житейская
или  государственная  мудрость,   то  почти  непременно  высказывается
за   превосходство  практического   знания  перед   теоретическим,  за
преимущество прикладной науки перед чистой. А если это будет моралист,
то он еще почтет своим долгом сделать внушение теоретику, эгоистически
изучающему предметы, не имеющие прямого, непосредственного отношения к
общему благу.

И вот перед  нами --- картина, до сих пор  невиданная. Сходит в могилу
простой  ученый,  и  люди,  ---  не  только  ему  близкие,  не  только
земляки,  но представители  всех стран  и народов,  всех толков,  всех
степеней развития, правительства и частные лица, --- соперничают между
собой в стремлении отдать  успокоившемуся работнику последнюю почесть,
выразить  чувства  безграничной,  неподдельной  признательности.  Если
когда-нибудь слова:  "благодарное человечество своему  благодетелю" не
звучали риторической  фразой, то,  конечно, на  могиле Луи  Пастера. А
между  тем  вся деятельность  этого  человека,  словом и  делом,  была
одним  сплошным опровержением  этого  ходячего  мнения о  преимуществе
практического знания перед теоретическим.

Уже одного этого достаточно для того, чтобы задуматься над тем уроком,
который можно извлечь из жизни этого гениального человека.

Жизнь ученого  заключается в  его трудах. О  трудах Пастера  так часто
рассказывали,  в общих  чертах они  так доступны  всеобщему пониманию,
что, я полагаю,  нет образованного человека, который не имел  бы о них
хоть приблизительного представления, и  потому я буду, по возможности,
краток и попытаюсь, не  придерживаясь строго хронологического порядка,
проследить логическую нить,  проходящую через все его  главные труды и
сообщающую  всей  его  деятельности совершенно  исключительную  печать
целостности и  единства. Найдется немного людей,  к итогу деятельности
которых  можно  было  бы  так уместно  применить  удачное  французское
выражение: l'oeuvre.  То, что  потомство назовет l'oeuvre  de Pasteur,
было,  действительно, как  бы  одним  слитным, непрерывным  творческим
актом, имеющим единство и прочность монолита.

Луи  Пастер,   как  известно,  первоначально  составил   себе  громкую
известность в  научном мире,  благодаря своим исследованиям  в области
химической  кристаллографии;   эти  исследования  открыли   ему  двери
Французской  Академии  наук, где  до  конца  своей жизни  он  числился
по  отделу минералогии,  несмотря  на  то, что  уже  почти с  половины
пятидесятых годов  вступил в  совершенно иную область.  можно сказать,
почти созданной  им новой  науки --- микробиологии  Все первоначальные
его  исследования группировались  вокруг  одной  центральной идеи  ---
зависимости   между  известными   оптическими  свойствами   химических
тел  и  их  кристаллической  формой. На  этом  основании  его  считают
родоначальником  гораздо   позднее  явившегося   крайне  плодотворного
направления   химии,  так   называемой   стереохимии,   ---  химии   в
пространстве,  объясняющей  химические  факты  не  одним  качественным
и  количественным  составом  тел,  но   и  группировкой  их  атомов  в
пространстве.

Эти исследования, между прочим,  заставили Пастера остановить внимание
на  одном  факте,  определившем   всю  его  последующую  деятельность,
сделавшую  его   имя  достоянием   уже  не   одних  ученых,   а  всего
образованного  и необразованного  мира.  Исследуя  раствор смеси  двух
весьма между  собой сходных, но отличающихся  по своим кристаллическим
формам органических кислот,  он заметил, что, разводя  в этом растворе
плесневый грибок, он мог разрушить одну кислоту, сохраняя другую. Этот
факт  взаимодействия между  микроскопическим  организмом  и средой,  в
которой  он развивается,  послужил  исходным  пунктом всего  стройного
здания экспериментальной микробиологии. Наблюдение это привело Пастера
к  изучению явлений  так  называемого брожения.  Немного, может  быть,
найдется в науке слов, которыми  в былое время так злоупотребляли, как
этим словом  "брожение"; почти все,  касающееся жизни и  организмов, а
также всевозможные  превращения веществ приурочивались к  брожениям, а
вызывающие  их тела  именовались ферментами.  Некоторым алхимикам  сам
философский камень представлялся чем-то в роде фермента.

В  исходе   первой  половины  прошлого  столетия   большинство  ученых
склонялось к мнению,  высказанному еще в тридцатых  годах Либихом, что
брожения -  это химические  явления, вызываемые в  самых разнообразных
телах  разлагающимися   белковыми  веществами.   Атомы  разлагающегося
белкового  вещества   приходят  в  какое-то  движение;   это  движение
сообщается  другим веществам,  раскачивает,  расшатывает  их атомы,  и
вещества их разлагаются. Это представление Либиха о каком-то невидимом
и неведомом движении, в своей  простоте, должно быть, заключало в себе
что-нибудь  очень  привлекательное,  так  как даже  много  лет  спустя
немецкий ботаник Негели выступил  с своим учением, существенно сходным
с учением Либиха, и увлек многих ботаников.

Против  этого-то воззрения  Либиха  вооружился Пастер.  Он выступил  с
теорией, что  все процессы  брожения-не простые химические  явления, а
результаты воздействия  на бродящие  тела микроскопически  малых живых
существ --- микроорганизмов. В целом  ряде работ он провел свою мысль,
применяя ее  к самым  разнообразным случаям  брожения: молочнокислому,
маслянокислому,  спиртовому,  уксусному,  и везде  деятельным  началом
оказывалось живое  существо --- дрожжевой грибок  или бактерия. Тщетно
пускал в  ход Либих  свое необычайное  остроумие и  диалектику, Пастер
теснил его  по всей линии  своими блестящими опытами,  не допускавшими
двух  толкований. Укажем,  в  виде примера,  хотя бы  на  тот опыт,  в
котором он доказал, что разлагающееся  белковое вещество не может быть
причиной  брожения, так  как  брожение обнаруживается  и в  отсутствие
всякого белкового вещества, --- этот  классический его опыт, в котором
дрожжевой грибок питался на счет сахара, золы и аммиачной соли.

Итак,  все  самые  разнообразные  случаи брожения  сводятся  к  одному
осязательному,  реальному   явлению  ---   развитию  микроскопического
организма.  Но   сами  эти   организмы  ---   откуда  они   берутся  и
действительно ли они представляют истинную причину, а не сопутствующее
явление? Проникают ли они в  бродящие вещества извне или зарождаются в
них  или  из  них?  Пастер сталкивается,  таким  образом,  с  вопросом
еще  более  широким  и  темным,  чем  самое  брожение:  с  вопросом  о
происхождении  простейших микроскопических  организмов. Интерес  этого
вопроса  как   раз  в  это  время   возбуждался  исследованиями  Пуше,
доказывавшего существование  самозарождения - generatio  spontanea ---
различных  микроскопических организмов.  Нигде, быть  может, так  ясно
не  обнаруживается характер  естествознания в  половине девятнадцатого
века  в сравнении  с  тем,  чем оно  было  в половине  восемнадцатого,
как  в   отношении  науки   к  этому  вековому   вопросу.  Сопоставьте
звучные,  округленные  периоды,  в  которых   за  сто  лет  Бюффон  не
стесняясь размежевывал весь мир  между существами самозарождающимися и
рождающимися  от родителей;  сопоставьте эти  беспочвенные рассуждения
с  той строгой,  исключительно  экспериментальной  почвой, на  которую
поставлен был вопрос в классическом  исследовании Пастера, и вы вполне
оцените, какие громадные успехи сделал научный метод, научная логика.

В    результате    этого   исследования,    произвольное    зарождение
микроорганизмов вычеркивается из числа возможных предположений. Везде,
где  наблюдается микроорганизм,  он  занесен  извне. Оказывается,  что
вполне во власти человека не  только вызвать, но и предотвратить любое
из этих явлений брожения: стоит произвести посев или воспрепятствовать
самосеву  этих простейших  из  наших культурных  или сорных  растений.
Культурными мы можем  считать те из них, которые человек,  сам того не
подозревая, с незапамятных времен  разводил для того, чтобы превращать
сусло в спирт, спирт --- в уксус;  сорными мы можем считать те из них,
которые, проникая против нашей воли, изменяют течение этих процессов и
дают нам продукты не того качества, какое мы желаем. Как успешно вести
культуру  этих  невидимых существ,  как  бороться  с ними,  когда  они
являются  такими же  невидимыми  сорными  растениями? Пастер  задается
этими  вопросами  по  отношению  к производствам,  в  которых  процесс
брожения играет важную роль, и в  своих знаменитых Etudes sur ie vin *
и  особенно Etudes  sur  la  biere **  дает  рациональную теорию  этих
производств и  научает, как разводить необходимые  микроорганизмы, как
вести борьбу с вредными. Кто не слыхал о так называемой "пастеризации"
вин ---  процессе,который ограждает  их от порчи,  от целого  ряда так
называемых "болезней"?

* Исследования над вин * Исследования над пив                      ом.

Мы  произнесли   слово,  с   которым  непрерывно  будет   связана  вся
дальнейшая  деятельность   Пастера.  Если  известными   мерами  борьбы
против  микроорганизмов   мы  можем  оградить  от   болезни  вино,  то
не  представляет  ли  это  учение ключа  к  другой,  неизмеримо  более
плодотворной борьбе  --- к  борьбе с  настоящими болезнями  животных и
человека?  Если  нет  произвольного  зарождения, то,  может  быть,  не
существует  и  произвольного  заражения. Эти  бичи  человечества,  эти
заразы, передающиеся от одного организма к другому, охватывающие целые
местности, разносящиеся  в ширь и  в даль, --- не  будут ли это  те же
невидимые  существа, а  результат их  действия, болезненные  изменения
в  организмах  животных  и  человека,  -  только  процессы,  подобные,
аналогические брожению? Этот невидимый,  но всюду проникающий заразный
яд,  --- не  потому ли  он  страшен, что  он  живой, что  он растет  и
размножается?

Пастер  останавливает   свое  внимание   не  сразу  на   человеке  или
каком-нибудь крупном животном;  он начинает с объекта,  в применении к
которому строго научная постановка была гораздо легче осуществима.

Юг Франции  страдал в то  время от бедствия,  грозившего окончательным
разорением целым местностям. Какая-то эпидемия истребляла шелковичного
червя. Пастеру  представлялся случай изучить явление  заразной болезни
на сравнительно простом, легко подчиняющемся строго экспериментальному
исследованию организме, к тому  же находившемся в неограниченном числе
экземпляров. Тем не менее, потребовались годы упорного труда для того,
чтобы изучить  болезни, ---  их оказалось  целых две,  --- во  всех их
подробностях, проследить  пути заражения  и наследственной  передачи и
найти  средство если  не прямой  борьбы  с эпидемией,  то, по  крайней
мере,  обеспечения промышленности  здоровой греной.  Болезни оказались
паразитарными, а пути заражения и передачи были путями распространения
микроорганизмов.  Факт  существования  эпидемической  болезни,  вполне
объясняемой присутствием микроскопического паразита и исчезающей с его
удалением, был, таким образом, поставлен вне сомнения.

Тогда  Пастер переходит  уже  к  крупным животным  и  для этого  сразу
избирает  одну   из  самых  страшных,  почти   безусловно  смертельных
болезней, поражающих рогатый  скот, а порой и  человека, --- сибирскую
язву. Выделив  из крови  зараженного животного  паразита, оказавшегося
бациллом,  он  культивирует его  в  других  жидкостях, вне  организма,
вводит  эти  культуры  в  организм   здоровой  коровы  и  вызывает  ее
заражение. Пастер,  таким образом, поставил вне  сомнения паразитарный
характер этой заразы.

Он показал далее, что  этот бацилл, благодаря способности образовывать
особые  органы размножения  ---  споры,  упорно сопротивляется  целому
ряду  условий,  убивающих  вегетативные  формы,  чем  ввел  совершенно
новый ряд  соображений в  учение об  источниках заразности  и способах
обезвреживания подозрительных  предметов. Он  показал, как  эти споры,
подобно  семенам высших  растений, могут  сохраняться годами  в земле,
где  были зарыты  трупы, как  земляными червями  они могут  выноситься
на   поверхность  почвы,   вызывая  новый   взрыв  эпидемии.   Попутно
показал  он,  как  в  опытах  над сибирской  язвой  можно  смешать  ее
особенный,  специфический бацилл  с  другими,  не менее  смертоносными
микроорганизмами ---  бациллами гнилокровия.  Словом, он  пролил целые
потоки  света на  вопросы  о механизме  заражения, скрытом  состоянии,
новом  возникновении и  распространении  такой  типической и  страшной
заразы, какова сибирская язва.

Остановимся только на одном из  опытов, едва ли не самом поразительном
из  этого длинного  ряда. Пастер  заметил, что  курам без  вреда можно
делать прививку  этой заразы,  смертельной для более  крупных животных
и  человека,  и вскоре  нашел  поразительно  простое объяснение  этому
любопытному факту. Температура крови  у птиц выше температуры животных
и человека, погибающих от сибирской язвы. Эта температура уже близка к
той,  при которой  бацилл не  может более  развиваться. Представлялось
вероятным, что  курица не  заражается потому,  что при  температуре ее
крови бацилл сибирской  язвы не может размножаться.  Но Пастер никогда
не довольствовался вероятным объяснением; он признавал значение только
за полной несомненностью.  Он взял курицу, привил ей  сибирскую язву и
поставил  ее ногами  в холодную  воду,  так что  температура ее  крови
понизилась  до 37--38.  На другой  день она  была мертва,  и кровь  ее
переполнена бациллами.  Но Пастеру и  этого показалось мало;  он берет
другую курицу, заражает,  охлаждает до тех пор, пока  в ней появляются
несомненные признаки  заразы; тогда  он ей  дает отогреться,  и курица
остается живой и  невредимой. Очевидно, жизнь и смерть в  его руках, и
он  распределяет их  с  такой  уверенностью, как  будто  имеет дело  с
каким-нибудь простейшим физическим опытом.

Но курице предстояло сыграть и не  такую еще роль в деятельности этого
гениального экспериментатора  и в том  перевороте в науке и  в будущих
судьбах  человечестма, который  он готовил  в тиши  своей лаборатории.
Куры не  заражаются сибирской  язвой, но  болеют другими  болезнями, в
том  числе  одной,  носящей  название  куриной  холеры.  При  изучении
этой-то  болезни Пастер  встретился  с фактом,  который определил  все
направление его дальнейшей деятельности.  По остроумному замечанию его
биографа, "это  была одна из  тех счастливых случайностей,  на которые
наталкиваются  те именно  ученые,  которые все  делают,  чтобы на  них
наткнуться". Микроорганизм  куриной холеры  можно также  разводить вне
организма курицы, например, в бульоне, и ничтожной капли этого бульона
достаточно, чтобы заразить и убить  курицу. Каплей этого бульона можно
заразить  новое количество  бульона,  каплей этого  бульона еще  новое
количество, и  так хоть до  ста раз, ---  сотая разводка будет  так же
ядовита, как первая,  но под условием, чтобы  между каждым последующим
заражением  проходило  не  более  суток.  Это  ---  приготовление  так
называемого постоянного яда, virus fixe.

Но  вот  однажды  Пастер,  желая  привить  курице  холеру  и  не  имея
под  рукой  сведшей культуры,  взял  простоявшую  несколько времени  в
пробирке,  заткнутой ватой.  Привитый яд  оказался уже  несмертельным,
курица поболела  и выздоровела. Пастер  повторял, умножал опыты,  и из
них  выяснилась  возможность,  по  желанию,  ослаблять  яд  заразы  во
всех  желаемых степенях  от безусловной  смертельности до  безусловной
безвредности. И  средство, опять крайне  простое, состояло и  том, что
культуру и бульоне оставляли более или менее продолжительное время при
доступе воздуха; чем долее она стояла, тем безвреднее становился яд.

Наоборот,  если взять  смертельно ядовитый  бульон и  сохранить его  в
запаянном  стеклянном  сосуде,  время  не оказывает  действия  на  его
ядовитость.  Это  ослабление,  притупление заразы,  ---  l'attenuation
des  virus,  --- конечно,  величайшее  из  открытий Пастера.  Из  него
непосредственно вытекают  все остальные. Пастер давно  задумывался над
фактом, что заразные  болезни, вообще говоря, не  повторяются, а также
над  возможностью  посредством  прививки  оспы  оградить  человека  от
естественной  оспы. Почему  бы не  распространить этой  прививки и  на
все  заразные болезни?  Теперь  представился  первый случай  проверить
возможность  этого   обобщения.  В  его  власти   привить  курам  этот
ослабленный яд,  virus attenue, вызывающий только  слабое расстройство
организма,и вслед за тем неизмененный яд, virus fixe, в его безусловно
смертельной форме. Опыт блистательно  оправдал ожидание: куры, которым
предварительно была привита зараза  ослабленная, оказались затем почти
нечувствительными  к заразе  смертельной. Прививка  оказалась приемом,
распространимым на заразные болезни вообще.

Здесь  необходимо тотчас  же оттенить,  подчеркнуть коренное  различие
между открытием Дженера, открывшим прививку оспы, и открытием Пастера.
"Если Дженер открыл отдельный факт,  --- говорит профессор Гранше, ---
то Пастер  открыл общий метод"  --- метод, применимый ко  всем случаям
и  вполне  подчинивший  яд  заразы  власти  человека.  Возьмите  самый
ядовитый микроорганизм известной  заразной болезни, ослабьте культурой
его  ядовитость до  желаемой  степени, привейте  его  животному, и  вы
обеспечите его от  заражения этой болезнью. В первый  раз была открыта
тайна  превращать, по  желанию, смертельный  яд в  противоядие. Пастер
предложил  назвать все  такие прививки  противоядия, по  примеру оспы,
вакциной.

Вооруженный этим  бесценным методом.  Пастер возвращается  к сибирской
язве,  но на  этот  раз  уже не  затем,  чтобы  ее изучать,  объяснять
пути  ее распространения,  а затем,  чтобы  вступить с  ней в  борьбу.
Но  здесь с  первых  же шагов  встречается непреодолимое  препятствие.
Прием,  выработанный над  заразой  куриной  холеры, оказывается  здесь
неприменимым. Если  оставить несколько дней культуру  бацилл сибирской
язвы, то они образуют споры, а эти споры сохраняют свою первоначальную
ядовитость. Но  Пастер был  не из  тех людей,  которые останавливаются
перед препятствием. Вскоре он нашел  исход. При температуре 42--43 эти
бациллы  уже  не производят  спор,  но  еще  размножаются, а  если  их
заставить  развиваться  в той  же  среде  и  при доступе  воздуха,  то
заразительность  их ослабевает,  притупляется. Уже  на втором  примере
путем многочисленных  лабораторных опытов  убедился Пастер  в верности
своей  теории: зараза  не  представляет из  себя  чего-то всегда  себе
равного; напротив, это нечто такое, ядовитость чего можно, по желанию,
понижать  и,  прививая этот  притупленный  яд,  оберегать организм  от
заражения его более грозной, смертельной формой.

Только теперь решился Пастер покинуть свою лабораторию, выйти на улицу
или, вернее, в поле и явить сомневающейся толпе знаменье своей научной
мощи.  Это был  его навеки  знаменитый опыт  в местечке  Пулье-ле-Фор,
весной 1881 года. Получив в свое распоряжение стадо овец в 50 штук, он
сделал  25  из  них  несколько  предварительных  прививок  ослабленной
заразы.

31  мая,  в присутствии  многочисленных  и  в большинстве  недоверчиво
настроенных  зрителей, он  привил всем  50 овцам  сибирскую язву  в ее
самой  смертельной форме  и  пригласил  всех присутствующих  вернуться
через  48 часов,  объявив вперед,  что  25 животных  они застанут  уже
мертвыми,  а 25  других целыми  и  невредимыми. Даже  друзья его  были
испуганы его самоуверенностью. Но пророчество исполнилось буквально.

Собравшимся в Пулье-ле-Фор 2 июня представилась такая картина: 22 овцы
лежали  мертвыми,  две умерли  у  них  на  глазах,  а третья  к  ночи;
остальные 25 были живы и здоровы.

Скептицизм врагов, опасения друзей уступили место взрыву безграничного
восторга. И действительно, с тех пор, что свет стоит, конечно, не было
видано ничего  подобного. Представим  себе, что когда-нибудь  в темные
века,  предшествовавшие той  заре,  которая  занялась над  обновленным
человечеством в шестнадцатом веке,  какой-нибудь человек в одежде мага
или кудесника объявил, что простым прикосновением к живому существу он
может  по желанию  или спасти  его или  обречь на  быструю мучительную
смерть,  а  ведь  на  то,  чтобы  скрыть  в  рукаве  небольшой  шприц,
потребовалось  бы  немного  ловкости,  и  можно  легко  понять,  какое
впечатление произвело бы это чудо на окружающих. Но современный маг не
прятал своего шприца в широких складках своей одежды, и разочарованные
охотники до чудесного, поговорив несколько дней об этом действительном
чуде девятнадцатого века, вернулись  к своему столоверчению, вызыванию
духов  и знахарству.  Прививка  сибирской язвы  стала таким  заурядным
делом, что теперь, без малого через пятнадцать лет *, уже никого более
не удивляет.

* Лекция читана в 1895 году.

Пастер  тем временем  шел вперед  по раз  намеченному пути.  Уже давно
желал он проверить истинность своего  учения не на червяке, курице или
овце,  а на  самом царе  природы. И  для этого  он снова  избрал самую
ужасную, самую безнадежную из болезней,  одна мысль о которой приводит
в содрогание,  - бешенство. Разъяснить причину  водобоязни, условия ее
передачи, --- словом,  повторить то же, что уже было  сделано в других
случаях, --- вот с чего приходилось снова начать.

Но с  первого же шага,  и в первый  раз, из рук  Пастера выскользнула,
оборвалась та путеводная нить, которая неизменно вела его по лабиринту
этих темных явлений.  Микроба бешенства не оказалось,  несмотря на все
поиски;  не  найден  он,  кажется,  и  до сих  пор  *.  Но  Пастер  не
останавливается  перед этим  препятствием,  которое  в глазах  всякого
другого ученого  могло бы  показаться непреодолимым.  Путеводной нитью
впредь ему будет уже не  присутствие микроба, а столько раз испытанный
экспериментальный  метод: найти  вместилище заразы  в организме  и это
нечто подвергнуть опытному исследованию, пока не найдутся условия, при
которых ослабляется его ядовитость.

*  Ныне  общеизвестно,  что   болезнетворным  началом  здесь  является
фильтрующийся вирус --- V.V.

После долгих исследований обнаружилось, что главным вместилищем заразы
должно  считать нервную  систему, мозг  головной и  спинной и  нервные
стволы.  Кусочек  нервной  ткани,  разведенный  бульоном  н  введенный
посредством  шприца, вызывает  неминуемое заражение.  Но как  ослабить
его  ядовитость,  пока  не  найдено микроба,  который  можно  было  бы
культивировать?

После  долгих поисков  Пастер  находит это  средство. Стоит  тщательно
отпрепарировать мозг  зараженного животного, подвергнуть его  со всеми
необходимыми  предосторожностями  высушиванию  и, по  мере  высыхания,
он  будет  утрачивать  свои  заразительные  свойства,  пока  их  вовсе
не   утратит.  Привитый   собакам   этот  ослабленный   яд  делал   их
невосприимчивыми к яду сильнейшему и к непосредственному укусу бешеным
животным.  Как  и  в  сибирской язве,  предварительная  прививка  была
осуществлена.  Но  прежде чем  применить  ее  к человеку,  нужно  было
сделать  еще  один  шаг,  совершенно новый  и  в  экспериментальном  и
даже  в  логическом  отношении.  До  сих  пор  шла  речь  о  прививках
предохранительных, предупреждающих заражение  и ему предшествующих. Но
разве можно было  бы применить ее ко  всем людям, как в  оспе, и ждать
последствий?

Случайность быть укушенным бешеным животным, по счастью, так мала, что
едва ли можно было  рассчитывать на такую смелую предусмотрительность.
А привить  себе ослабленный  яд бешенства и  затем дать  себя искусать
бешеной собаке, --- у кого же  достало бы на то самоотвержения? Пастер
нашел  и на  этот раз  совершенно новый,  смелый, поистине  гениальный
прием,  --- прием  также  предохранительной, но  не предшествующей,  а
последующей прививки.

На возможность такого приема наводило открытие нового свойства заразы.
Переводя  заразу бешенства  из одного  кролика в  другого, Пастер  мог
заметить, что скрытый,  инкубационный период заразы мог  более и более
сокращаться; наконец, он  был им сведен на семь дней.  Так как у людей
скрытое  состояние  длится  не  менее  месяца  или  шести  недель,  то
можно было  надеяться вдогонку  этому медленному, но  смертельному яду
послать  яд ослабленный,  но с  более быстрым  течением заражения.  Он
опередит этот сильный  яд и подготовит организм к  его приему, сделает
этот  организм  неуязвимым.  Проверенная на  собаках,  эта  гениальная
мысль  оказалась таким  же верным  средством борьбы,  как и  прививки,
предшествующие заражению. Открыто было средство уже не предупреждения,
не охранения, а прямого излечения от самой страшной из зараз.

Тогда наступил  самый решительный, самый торжественный  момент в жизни
Пастера, --- момент,  когда ему пришлось доказать  уверенность в своем
учении,  рискнув  применить открытое  им  излечение  уже на  человеке.
Рассказывать  ли  драматические  подробности двух  первых  опытов  над
маленьким Мейстером и подростком Жюпилем?  Они, я полагаю, еще свежи в
нашей  памяти.  Торжество Пастера  было  полное.  Первые пациенты,  им
спасенные, были  так жестоко искусаны бешеной  собакой, что, производя
над ними  опыт, Пастер,  казалось, мог бы  успокоить себя  мыслью, что
делает эксперимент  над людьми,  фактически обреченными на  смерть. Но
только  близкие  к нему  люди  знали,  какой  ценой было  куплено  это
торжество.  Какие  подъемы  надежды, сменявшиеся  приступами  мрачного
уныния, какие  томительные дни  и мучительные, бессонные  ночи перенес
этот уже  немолодой, истощенный  трудами и  болезнями человек  между 4
июля,  когда профессор  Гранше, вооружившись  правацевским шприцем,  в
первый раз привил живому человеческому  существу яд бешенства, на этот
раз превращенный в противоядие, и 26 октября, когда Пастер, выждав все
сроки  возможной инкубации,  в  своей обычной  скромной форме  сообщил
Академии, что излечение от бешенства  --- уже совершившийся факт. Всем
памятен тот взрыв всеобщего восторга,  который пронесся из края в край
образованного мира при слухе, что самая страшная из болезней побеждена
наукой.

Это  было высшей  точкой  научной деятельности  Пастера  и его  славы.
Имя  его  стало  достоянием  всех  людей, как  ценящих  науку,  так  и
равнодушных  к  ней.  Выражением всеобщего  увлечения  его  открытиями
явилась международная подписка на  постройку достойной его лаборатории
--- этого знаменитого Пастеровского института, которому суждено играть
такую роль в будущих судьбах созданной Пастером новой науки.

Нужно  ли подводить  итог, нужно  ли указывать  на строгое  логическое
развитие  этого  стройного  учения,  выражающегося  четырьмя  словами,
которым соответствуют четыре последовательные ступени развития одной и
той же мысли: брожение, зараза, ее предупреждение и врачевание?

II

Я нарочно пытался изобразить  эту удивительную деятельность в возможно
сжатой,  почти   схематической  форме,   чтобы  выдвинуть   вперед  ее
поразительное  единство и  естественное развитие,  но такой  умышленно
сжатый очерк всегда грешит с двух сторон *.

* Тем, кто пожелал бы познакомиться с деятельностью Пастера подробнее,
можно  рекомендовать  две  брошюры:  Дюкло.  ---  Пастер.  Брожение  и
самозарождение. Москва, 1897 г.; Дюкло. --- Пастер. Заразные болезни и
их  прививка. Москва,  1898 г.  Дюкло ---  ученик и  преемник Пастера.
Перевод под моей редакцией.

Во-первых,  дело представляется  как будто  очень простым;  за блеском
успеха  остается  невидимым   почти  невероятный,  колоссальный  труд,
остается скрытым  тот на каждом шагу  проявляющийся, неистощимый запас
изобретательности  и находчивости,  преодолевающей  все препятствия  и
превращающей  длинную  вереницу  исследований в  какое-то  непрерывное
победоносное  шествие. С  другой стороны,  целое научное  направление,
полувековые  плоды   пауки  являются   как  бы   исключительным  делом
одного  человека,  деятельность  которого  представляется  чем-то  уже
сверхчеловеческим. Будь все то, что мы перечислили, завоеванием одного
человека, зародись  все эти мысли в  одной голове, перед нами  было бы
явление, которому трудно  подобрать аналогию. Беспристрастный историк,
- и в этом он последует прежде всего примеру самого Пастера, с крайней
добросовестностью разыскивавшего и  указавшего своих предшестпенников,
- беспристрастный историк, конечно,  отметит, что учение о зависимости
брожения от  микроорганизмов, определенно  высказанное Каньяр-Латуром,
было   блистательно  доказано   Гельмгольцем;  что   несостоятельность
предположения о самопроизвольном  зарождении была убедительно доказана
Шваном; что  мысль о связи  заразных болезней с  присутствием бактерий
задолго  до  Пастера  нашла  себе горячего  защитника  в  Генле,  что,
наконец,  Райе  и Давэн  очень  точно  доказали паразитарный  характер
сибирской язвы.

Упоминая об этих  фактах, уменьшаем ли мы  хоть сколько-нибудь заслугу
Пастера?

-  Нимало.  Все  эти   проблески  мысли,  вспыхивавшей  и  потухавшей,
не  оставляя  по себе  прочного  следа,  только выдвигают  вперед  все
значение Пастера. Дарвин  в одном месте своей  автобиографии, говоря о
некоторых своих открытиях, которые были потом приписаны другим ученым,
замечает: "Повидимому,  недостаточно высказать  новую идею,  нужно еще
высказать ее  так, чтоб она  произвела впечатление, и тому,  кто этого
достиг, принадлежит  по праву  и главная честь".  Генле был  убежден в
паразитарной теории зараз, но, видно, не  умел доказать ее ни себе, ни
другим,  так как  его мысль  чуть ли  не двадцать  лет оставалась  без
плода.  Все,  что высказывал  Пастер,  вынуждало  на согласие.  А  это
происходило от  того, что он  не только  высказывал идеи, но  и создал
новый метод  и при  помощи этого метода  превращал идею  в неотразимый
факт.

Эту, им  созданную, новую науку обыкновенно  называют бактериологией и
совершенно  неправильно,  потому что  она  обнимает  круг существ,  не
исчерпывающихся  одними бактериями.  Вернее  было бы  ее назвать  хоть
микробиологией *.

*  Если  гнаться  за  точностью,   то  и  это  название,  конечно,  не
верно.  Попытаемся  определить  точнее границы  деятельности  Пастера.
Он   изучал  не   биологию,  а   скорее  физиологию,   микрофизиологию
и  притом   микрофизиологию  исключительно   растительных  организмов,
следовательно,  микро-фито-физиологию. Наконец,  и в  этой области  он
ограничился исключительно организмами из класса грибов. Следовательно,
микро-мико-физиология  ---  вот  единственный  термин,  который  точно
обозначил  бы   область  новой  науки.   Едва  ли,  впрочем,   за  ней
когда-нибудь сохранится такое педантическое название.

Существование   этих   микроскопических  растений   коренным   образом
отличается  от существования  высших организмов.  При изучении  высших
существ,  растений и  животных,  мы  изучаем их  самих  и действия  на
них  той среды,  в которой  они  существуют. Их  воздействие на  среду
сравнительно не важно. Наоборот, в  жизни этих микроорганизмов чуть не
на первый план выступает именно их воздействие на обитаемую ими среду,
откуда становится попятным тот  с первого взгляда парадоксальный факт,
что мы знали  действие этих существ, когда еще не  знали их самих. Эти
действия  --- брожения  и заразные  болезни. Пастер  создал метод  для
изучения этих невидимых существ и  их воздействия на ту среду, которая
им служит почвой, будет ли то  бродящая жидкость или тело человека. Он
показал,  что над  этими  бесконечно малыми  и  над такими  бесконечно
сложными объектами, какими являются  зараженные ими животные, мы можем
экспериментировать с  такой же  точностью и  уверенностью относительно
получаемых результатов,  как в каком-нибудь простейшем  физическом или
химическом опыте. Вот в чем его главная  сила. И в этом смысле, к чему
бы ни  привела наука  будущего, как  бы ни  изменились ее  задачи, она
будет итти по открытому им пути.

Какому же выдающемуся  качеству этого могучего ума,  какой его faculte
maitresse *,  как выразился  бы Тэн,  следует приписать  главную тайну
его  успеха? Самой  выдающейся его  особенностью была  не какая-нибудь
исключительная  прозорливость,  какая-нибудь  творческая  сила  мысли,
угадывающей то,  что скрыто от  других, а, без  сомнения, изумительная
его способность, если  позволительно так выразиться, "материализовать"
свою  мысль, выливать  ее в  осязательную форму  опыта, ---  опыта, из
которого природа, словно стиснутая в  тисках, не могла бы ускользнуть,
не выдав своей тайны.

* Преобладающая способность.

Это  был  гений или  само  воплощение  экспериментального метода.  Вся
деятельность  его была  блестящим  опровержением  тех знаменитых,  так
часто  упоминаемых и  подвергавшихся  многочисленным толкованиям  слов
Гете:


Geheimnissvoll am  lichten Tag  Lasst sich  Natur des  Schleiers nicht
berauben, Und was sie deinern  Geist nicht offenbaren mag, Das zwingst
du ihr nicht ab mit Hebelii und mit Schrauben

Средь бела дня полна таинственными  снами. Не даст тебе природа покров
с  себя сорвать,  И то,  что разуму  сама не  может передать,  Тебе не
выпытать у нее ни рычагами, ни тисками.

В  этих словах  выражалось целое  миросозерцание, в  основе враждебное
экспериментальной науке;  в них  высказался, как известно,  не столько
Гёте Фауста, сколько  Гёте Farbenlehre *, ---  Гёте, воображавший, что
своим умственным оком, обращенным на  природу, как она есть, он проник
в  сущность  явлений  света  гораздо глубже,  чем  Ньютон,  пытавшийся
вымучить  у природы  ее тайну  в темной  комнате, при  помощи какой-то
призмы и узкой щели.

* Учение  о цветах.  --- Научное  сочинение Гёте,  в котором  он думал
опровергнуть учение  о цветах  Ньютона. Доказано, что  его собственное
учение было основано на грубой ошибке поспешно сделанного опыта.

Известно, что  философ Шопенгауэр похвалялся  тем, что один  из первых
оценил  это превосходство  Гёте  перед Ньютоном,  а другой  мыслитель,
Карлейль,  презрительно хохотал  при  мысли,  что какие-то  математики
могут  быть судьями  над Гёте.  А между  тем, почти  за двести  лет до
Гёте, был  ему дан  прямой ответ, была  высказана точка  зрения, прямо
противоположная той, которая выражена в его звучных стихах.

"Occulta  naturae  magis  se   produnt  per  vexationes  artium,  quam
cum  cursu  sua  meant",  ---  говорил  еще  Бэкон  в  Novum  organum.
"Тайны природы  успешнее выпытываются  искусством, чем  при наблюдении
естественного ее течения".

И  еще  ранее,  во  втором   своем  афоризме:  "Nec  manus  nuda,  nес
intellectus  sibi permissus  mulbum  valet;  instrumentis et  anxiliis
res  perficitur".   -  "Невооруженная   рука  и  разум,   себе  самому
предоставленный, не много  стоят. Все достигается при  помощи орудий и
иных пособий" *.

*  Бэкон  разумел  здесь  и орудие  логическое  ---  индукцию.  Пастер
неоднократно  заявлял,   что  в  cвоих  исследованиях   он  пользуется
исключительно этим оружием.

Пастер показал, чего можно достигнуть при помощи этих ненавистных Гёте
Hebein und Schrauben, и если кто желает поучиться этому величайшему из
искусств,  искусству допрашивать  природу и  выпытывать ее  тайны, над
которыми глумился Гёте,  тот найдет в трудах  Пастера редко досягаемые
образцы  экспериментальной логики-этой  логики в  действии. А  те, кто
все  еще полагают,  что  intellectus sibi  permissus  может с  пользой
громоздить системы над системами и в витиеватых или неуклюжих периодах
что угодно  опровергать, что угодно  доказывать, --- пусть  поучатся у
него, что значит, на языке точной  науки, это слово доказать. Вот один
из его заветов ученикам:

"Не  высказывайте ничего  такого,  чего не  можете  доказать просто  и
несомненно.  Преклоняйтесь перед  духом  критики. Сам  по  себе он  не
раскрывает  новых  идей  и  не  побуждает  к  великим  делам.  Но  без
него  ничто не  прочно. За  ним всегда  остается последнее  слово. Это
требование, которое я вам предъявляю,  а вы предъявите своим ученикам,
- самое тяжкое, какое только можно предъявить исследователю, делающему
открытия."

"Быть уверенным,  что открыл важный научный  факт, гореть лихорадочным
желанием оповестить о том весь свет и сдерживать себя днями, неделями,
порою годами;  вступать в  борьбу с самим  собой, напрягать  все силы,
чтобы  самому  разрушить  плоды   своих  трудов,  и  не  провозглашать
полученного  результата, пока  не испробовал  всех ему  противоречащих
гипотез, --- да, это --- тяжелый подвиг".

"Но зато, когда после стольких усилий достигаешь полной достоверности,
испытываешь   одну  из   высших   радостей,   какие  только   доступны
человеческой душе".

В этих словах кроется вторая  тайна успеха этой, почти беспримерной по
своим  плодам,  научной  деятельности. Этот  гений  экспериментального
метода отличался  трудолюбием, упорством  в труде.  почти превышающими
всякое  вероятие.  За  какими-нибудь несколькими  строками  в  Comptes
Rendus *, где он возвещает  о своих открытиях, скрываются сотни, порою
тысячи опытов.

* Журнал, в котором он помещал первые известии о своих открытиях.

Вот  еще  один  распространенный предрассудок,  уничтожаемый  примером
этого   необыкновенного  человека,   предрассудок,   будто  талант   и
трудолюбие  не идут  рука  об руку.  Ренан в  речи,  по случаю  приема
Пастера  во Французскую  Академию,  превосходно  освещает эту  сторону
научного характера  Пастера. Он начинает  с той карикатуры,  в которой
Жозеф  де-Местр  *,   этот  фантастический,  исступленный  ненавистник
прогресса и науки, изображает современного ученого.

* Клерикальный писатель начала XIX века.

"В кургузом платьишке... с  томами и инструментами подмышками, бледный
от трудов  и бессонных ночей, весь  забрызганный чернилами, задыхаясь,
плетется он по дороге к истине,  уткнув в землю свое лицо, испачканное
алгебраическими знаками...".

"Как  хорошо, ---  восклицает Ренан,  ---  что вас  не остановила  эта
дворянская брезгливость!  Природа сама --- не  аристократка (la nature
est  roturiere); она  требует, чтобы  трудились; она  любит мозолистые
руки и делает свои откровения только челу, изборожденному морщинами".

Один из его  учеников сообщает, что Пастер в  эпоху самых плодотворных
своих  исследований  имел  обыкновение вечером,  после  своих  дневных
трудов, еще  долго ходить взад и  вперед по коридору Ecole  Normale *,
взвешивая в  уме полученные результаты, обдумывая  завтрашние опыты. В
одну из  таких прогулок  ученики, следившие за  ним из-за  угла, могли
подслушать, как, внезапно  остановившись и как бы не  в силах сдержать
себя, он пробормотал вслух: "Que c'est  beau!.. Que c'est beau!" --- и
через  минуту: "Il  faut travailler"  **.  В этих  двух фразах,  почти
междометиях, сказался весь Пастер.

* Высшее училище,  где он работал  и учил * "Какая  прелесть!.. Какая.
*прелесть! Теперь надо работать!"                                    .

"Едва ли, ---  говорит тот же ученик, ---  существовал когда-нибудь ум
более страстный и в то же время более терпеливый". Овладевшая им мысль
приводила его в состояние какого-то экстаза; даже по ночам, во сне, он
нередко вскрикивал, и прислушивавшиеся  могли смутно разобрать, что он
бормотал какие-то  научные термины. Но чем  увлекательнее казалась ему
зародившаяся  идея,  тем строже  он  к  ней относился,  сознавая,  что
недостаточно бросить в мир счастливую мысль, --- необходимо прежде еще
облечь ее в форму неопровержимого факта.

Изложение  у  Пастера,  как   письменное,  так  и  устное,  отличалось
замечательной безыскусственностью  и простотой, --- как  бы умышленной
заботой об отсутствии всякого эффекта. Живо помню, как летом 1877 года
мне привелось слышать одно из  его замечательных сообщений в Парижской
Академии. Это  был один из  интереснейших и знаменательных  моментов в
его деятельности.  С различных  сторон, вдруг,  как будто  по сговору,
стали всплывать  возражения против верности  не только его  теории, но
и  самых  фактов,  на  которые  она  опиралась.  Какой-то  туман  стал
заволакивать только  что выяснившееся учение о  паразитарном характере
зараз.

Серьезные  фактические  возражения  были предъявлены  против  верности
наблюдений Давэна над сибирской  язвой, служивших точкой отправления и
для  Пастера. Поль  Бэр, посредством  нового приема,  заключавшегося в
действии сгущенным  кислородом, казалось,  несомненно доказал,  что яд
сибирской язвы  не живой, не  организованный; наконец, Бастиан  в ряде
любопытных  опытов через  пятнадцать  лет после  поражения Пуше  смело
вновь выступил защитником явления самозарождения.

Все  здание, составившее  прочную  славу  Пастера, казалось,  шаталось
в  своем  основании.  Пастер   выступил  перед  Академией  с  докладом
о  результатах  своих  новых  исследований  над  сибирской  язвой.  Он
разъяснил,  что все  показания,  противоречащие исследованиям  Давэна,
происходят от  того, что  явления заражения сибирской  язвой смешивают
с  септицемией,  гнилокровием,  зависящим от  другого  микроба,  через
несколько  часов  после  смерти животного,  уже  вытесняющего  бацилла
сибирской  язвы.   Он  показал,   что  этот  бацилл   образует  споры,
относящиеся  совершенно иначе  к  внешним  деятелям, чем  вегетативные
формы, и этим объяснил наблюдения Поля Бэра и так далее, и так далее.

По мере того, как он говорил,  туман, нависший над вопросом, все более
и  более  расходился,  противоречивые наблюдения  получали  совершенно
новое освещение,  из возражений  они превращались в  факты, находившие
место  в его  теории, в  качестве разъяснений  или дополнений.  Когда,
после  почти часовой  речи, он  опустился в  свое кресло,  для всякого
понимающего  дело  было  ясно,  что  его  учение  было  в  эту  минуту
более  сильно,  чем  когда-либо.  И  все  эти  исследования,  стоившие
усидчивых трудов, требовавшие  совершенно особенной проницательности в
деле,  для него  тогда почти  новом, были  рассказаны так  просто, так
непритязательно,  что если  бы не  напряженное почтительное  внимание,
с  которым слушали  его  товарищи-академики,  и какой-то  возбужденный
трепет  ожидания, пробежавший  в  публике при  словах: "Пастер  встал!
Пастер говорит!",  --- поверхностный наблюдатель мог  бы подумать, что
это  делает  сообщение какой-нибудь  заурядный  ученый  и по  вопросу,
интересному разве только для одних ветеринаров.

Замечательно было также отношение  Пастера к своим противникам. Близко
его знавшие рассказывают о каких-то  "les fureurs de monsieur Pasteur"
*,  -  о  приступах  неудержимого  гнева,  вызывавшихся  сомнениями  в
верности его исследований.  Но, вероятно, он давал  время улечься этим
вспышкам,  так  как  в  полемиках с  такими,  выходившими  за  пределы
приличий,  возражателями, как,  например,  Брефельд, он  воздерживался
от  всяких  резкостей и  только  презрительно  давил своего  соперника
неотразимой   убедительностью   своих   фактов.  И   нельзя   сказать,
чтобы  его   терпение  не  подвергалось  испытанию:   ему  приходилось
бороться,  отстаивая  почти  каждую  из своих  идей.  Стоит  вспомнить
презрительно-самонадеянные  отзывы  Коха   о  пастеровских  прививках,
сводившиеся к тому, что "трудно им  поверить, - слишком уж это было бы
хорошо", или  постоянно враждебное отношение нескольких  его коллег по
медицинской академии, по словам  некоторых его биографов, отозвавшееся
даже на его здоровье.

* "Вспышки ярости господина Пастера"

Пастер, как мы  сказали, не был охотником до фраз,  и потому тем более
необходимо  остановиться на  одной фразе,  лежащей, если  не ошибаюсь,
главным  образом,  на  ответственности  его  зятя-биографа,  несколько
раз  возвращающегося к  ней  в своем  прекрасном  рассказе; фраза  эта
повторялась потом и некоторыми его поклонниками, как одно из положений
научной profession de foi * великого ученого.

* Исповедание веры.

А между тем она может привести только в недоумение всякого знакомого с
духом истинной науки.  Это --- фраза о пользе будто  бы в деле научных
исследований "предвзятых идей"  --- "des idees preconсues".  Едва ли в
этом выражении мы можем видеть  что-либо иное, кроме не совсем удачной
игры слов,  тем более предосудительной,  что Пастер как  член Acadеmie
Franсaise *, да еще занявший кресло Литтре **, должен был заботиться о
чистоте и точности французского языка.

* Академия  французской словесности  * Известный  составитель лучшего.
*словаря французского языка                                          .

Предвзятая  идея,  в  общепринятом  смысле этого  выражения,  это  ---
не  просто  мысль,  предшествующая,  предпосылаемая  всякому  опытному
исследованию и без  которой оно из систематических  поисков за истиной
превратилось бы в  какое-то блуждание в темноте  и наудачу. Предвзятая
идея,  это  ---  мысль,  не вытекающая  прямо  из  условий  изучаемого
явления, а навязываемая извне, - мысль, под которую стараются пригнать
факты. А подобная мысль в науке, конечно, может быть только вредна.

Если  бы в  том  могло быть  какое-нибудь сомнение,  то  мы можем  его
рассеять свидетельством  самого Пастера. В своих  полемиках, например,
с  Бертло  или с  Клод-Бернаром,  он  не  упускал случая,  в  качестве
последнего  аргумента,  бросить  своим  соперникам упрек  в  том,  что
они  руководятся  предвзятой  идеей,  между тем  как  он,  Пастер,  не
покидает почвы строгой индукции. Наконец, в своем классическом труде о
самозарождении он  обращается к Пуше  со словами:  "II est si  rare de
deviner juste quand on etudie la nature.  Et puis est ce que les idees
preconcues ne  sont pas  toujours la  pour placer  un bandeau  sur nos
yeux?"  --- "Изучая  природу, как  трудно угадывать  истину! И  потом,
разве предвзятые идеи не всегда  тут как тут, готовые наложить повязку
нам на глаза?"

Итак, если некоторые поклонники  Пастера ссылаются на какие-то темные,
непонятные  его выражения  о пользе  предвзятых идей,  то мы  можем им
предъявить  категорическое его  заявление, что  эти идеи  могут только
ослеплять ученого.

Но  если бы  мы  даже не  могли указать  этих  слов великого  ученого,
то  сами  могли  бы  извлечь  урок о  вреде  предвзятых  идей  из  его
собственной  деятельности.  Ему,  при всем  его  научном  скептицизме,
случалось высказывать  предвзятые идеи, и  этим идеям не  суждено было
оправдаться. Одною из этих  предвзятых идей было стремление отстаивать
какое-то  коренное различие  между  химией живого  организма и  химией
лаборатории. Он  упорно отстаивал мысль,  что только в  организмах или
при  их  содействии  образуются так  называемые  оптически  деятельные
вещества,  т.е. вещества,  вращающие  плоскость поляризации  светового
луча,  что  химик не  в  состоянии  их получить  своими  лабораторными
путями.  Но  органическая  химия  перешагнула и  через  эту  последнюю
преграду,  и  предвзятая  идея  о  существовании  этой  границы  между
органической химией и химией организмов оказалась несостоятельной *.

* В действительности, прав был и остается как раз Пастер. Лабораторный
синтез оптических изомеров ---  так назывемый асимметрический синтез -
возможен лишь из  других оптически активных веществ  или под действием
оптически  активных  реагентов  или  катализаторов.  Что  же  касается
асимметрии  компонентов  живой  материи,   ее  происхождение  ---  это
фундаментальная проблема, до сих пор не нашедшая решения --- V.V.

Сходная  мысль, мысль  об  исключительной способности  микроорганизмов
вызывать явления брожения,  легла и в основу  представлений Пастера об
этих процессах.  Его точку  зрения на  явления брожения  можно назвать
биологической,  иногда  даже   пытались  назвать  ее  виталистической.
Причина  брожения  ---   жизнь  микроорганизма;  найти  микроорганизм,
определить условия его существования --- вот задача исследователя, как
определял ее Пастер.

При оценке теории брожения  Пастера обыкновенно ее сопоставляют только
с опровергнутой  им теорией Либиха, но  на первых же порах,  при самом
возникновении биологической  теории Пастера, против нее  выступил один
ученый,  указавший  на то,  что  она  представляет разрешение  вопроса
только,  так сказать,  в  первой степени  приближения, что  необходимо
заглянуть в  этот процесс глубже.  Бертло в самом  начале шестидесятых
годов прямо высказал мысль, что такая ограниченная биологическая точка
зрения  не  может, не  должна  удовлетворять  физиолога, а  тем  более
химика. Причина брожения лежит в микроскопической клеточке; прекрасно,
но эта  клеточка не есть  последняя единица, которая должна  входить в
расчеты  физиолога,  a  тем  более  химика.  Эта  клеточка  ---  целая
лаборатория,  и  вступает она  в  химическое  взаимодействие не  своей
совокупностью, а через посредство входящих в ее состав веществ. Найти,
выделить эти  вещества, воспроизвести  их действие без  участия живого
элемента,  ---  вот  в  чем  должна  быть  настоящая  цель  стремлений
физиолога,  а  тем  более  химика.  Это  воззрение  Бертло  на  первых
же  порах  подтвердил  открытием  растворимого  фермента,  выделяемого
дрожжевым грибком *. И новейшие успехи  науки, --- не оправдали ли они
верность этого химического взгляда,  пытающегося заглянуть вглубь того
явления,  к  которому  биологическая  теория отнеслась  только  с  его
внешней стороны?  Все эти токсины и  антитоксины, эти все чаще  и чаще
произносимые слова: диастаз, диастатический  фермент, не доказывают ли
то, что выросшее на почве учения о брожениях учение о заразах вступает
на ту  новую ступень, которую  Бертло предсказал слишком  тридцать лет
тому назад? **

* Долгое  время главным препятствием, мешавшим  обобщению идей Бертло,
являлся факт невозможности  воспроизвести процесс спиртового брожения,
т.е.  распадения  глюкозы  на  алкоголь  и  углекислоту,  без  участия
микроорганизма. **  Через год  после этой  моей почти  одинокой защиты
Бертло появилось  замечательное исследование немецкого  химика Бухнера
(убитого теперь на войне),  выделившего из дрожжей растворимый фермент
спиртового  брожения.   После  этого  воззрение   Бертло  окончательно
восторжествовало, хотя  это не помешало его  врагам (например, Дюгему)
продолжать болтать о какой-то победе над ним Пастера. (Примечание 1918
г.).

После  прививки  бешенства, конечно,  ни  одно  открытие не  произвело
такого впечатления  на умы,  как открытие  лечения противо-дифтеритной
сывороткой.  Микроб   дифтерита  не  разносится  с   кровью  по  всему
организму,  как  бацилл  сибирской язвы,  его  развитие  исключительно
местное,  и,  тем  не  менее,  он  отравляет  весь  больной  организм.
Это   нечто,   чем   он   отравляет,   оказалось   растворимым   ядом,
быстро  распространяющимся  в  организме. Когда  ослабленные  разводки
дифтеритного микроба  будут привиты  животному, например,  лошади, она
безопасно, как  и в  сибирской язве, выдерживает  последующую прививку
микроба самого ядовитого.

Что  же происходит  в  организме  этой лошади,  что  делает ее  теперь
выносливой к этому ядовитому микробу дифтерита? В ее крови оказывается
нечто  жидкое, противоядие,  антитоксин, который  противодействует яду
микроба, токсину.  Этот токсин, этот  антитоксин --- уже  не организм,
а  химические  тела; стоит  слить  их  вместе, и  получится  бевредная
смесь, которую  без опасности  можно ввести в  организм. Этот  яд, это
противоядие,  взаимно  нейтрализующиеся  в  пробирке,  in  vitro,  это
---  уже не  биологическое, не  виталистическое явление,  а химический
процесс.

За этим успехом теории последовал  и громадный успех практики. Послать
в обгонку  ядовитому микробу болезни  культуру того же  микроба, менее
ядовитого и быстрее развивающегося, было возможно при водобоязни, с ее
неделями, месяцами инкубации.  Микроб дифтерита не ждет,  --- он разит
иной раз  через несколько  часов. Но  если действие  прививки сводится
на  образование  в  больном  организме под  ее  влиянием  противоядия,
антитоксина,  то  возьмем  этот антитоксин,  заранее  заготовленный  в
крови  лошади, прямо  введем  его  в тело  больного  дифтеритом, и  мы
будем с  в состоянии  гораздо скорее  оказать отпор  действию грозного
токсина,  предупредить  отравление  больного организма.  Вот  основная
мысль  блестящего  открытия Беринга  и  Ру,  вызвавшего так  еще  живо
памятный  всем  взрыв всеобщего  восторга,  с  которым было  встречено
известие, что  дифтерит уже излечим\footnote{Сюда же  должно отнести и
новое блестящее открытие, связанное  с именем Пастеровского института.
Кальмету,  как известно,  удалось найти  токсин и  антитоксин змеиного
яда,  а следовательно,  и  средство борьбы  с  этим бичом  тропических
стран.}.


Каково бы  ни было ближайшее объяснение  этого воздействия антитоксина
на  токсин,  окажется   ли  возможным  и  его   включить  в  блестящую
теорию  Мечникова  о  фагоцитозе  или он  сведется  к  более  понятной
непосредственной  нейтрализаций  двух  веществ,  ---  одно  только  не
подлежит сомнению, что новое учение возникло на почве уже химического,
а  не биологического  или  виталистического  представления о  сущности
процесса  брожения и  аналогических  ему  явлений, заразных  болезней,
вызываемых микробами.

Желаю ли я этим снова умалить значение Пастера? Нимало; от меня далеко
эта мысль. В  прошлом столетии кто-то сострил: il у  a quelqu'un qui a
plus  d'esprit que  Voltaire  --- c'est  tout  ie monde  *.  To же,  с
большим  еще правом,  можно сказать  о  науке. Есть  кто-то, кто  выше
ученых,  даже гениальных,  это  --- сама  наука  в ее  поступательном,
эволюционном движении.  Бертло, полемизируя с Пастером,  указывал, что
воззрение на  брожение, как  на химический  процесс, лежащий  в основе
того физиологического  явления, которое  наблюдал Пастер, ---  что это
воззрение  вытекает из  неизбежного исторического  хода развития  всех
наук и, в частности, физиологии,  по которому сложные явления сводятся
к  простым  и,  следовательно,  физиологические  ---  к  физическим  и
химическим. И,  как мы  видим, история  уже оправдывает  верность этой
ссылки на нее Бертло.

* "Есть некто, умнее самого Вольтера, и это --- весь свет"

Быть может, как это также не раз повторялось в истории наук, ограничив
область  своего  исследования, не  углубляясь,  как  Бертло, в  анализ
изучаемого явления, Пастер тем успешнее сосредоточил свои силы на том,
что в  настоящий момент  было всего важнее  прочно установить,  --- на
связи явлений  с наличностью микроба;  но также не  подлежит сомнению,
что будущее принадлежит этому более глубокому анализу явления.

Итак, во всяком случае, не  в предвзятых идеях, как полагают некоторые
его  поклонники,  и даже  вообще  не  в  абсолютной новизне  идей,  за
исключением гениальной идеи  прививок искусственно ослабленной заразы,
заключалось  главное влияние  этого  могучего ума;  оно заключалось  в
тайне сообщать  этим идеям  неотразимую, обязательную  силу, благодаря
его  непогрешимому  экспериментальному   методу.  Грядущие  поколения,
конечно, дополнят  дело Пастера,  но исправлять  им сделанное  едва ли
придется, и,  как бы  далеко они  ни зашли вперед,  они будут  итти по
проложенному им  пути, а  более этого  в науке  не может  сделать даже
гений.

III

От  качеств  ученого  перейдем  к качествам  человека.  Здесь  впереди
всех достоинств  выступал тот благородный энтузиазм,  то бескорыстное,
самоотверженное отношение, которое превращало его научную деятельность
из простого занятия  в служение идее и человечеству. В  его жизни были
минуты,  когда он  возвышался  до геройства,  конечно, не  уступавшего
геройству солдата па поле битвы или врача среди зараженного населения.
В самый  разгар одной  из его  работ, как  всегда поглощавшей  все его
физические  силы, так  как усиленная  умственная работа  усложнялась у
него  обыкновенно  бессонницей,  лечивший  его  врач,  видя,  что  все
увещевания напрасны, нашелся вынужденным  пригрозить ему словами: "Вам
угрожает,  быть может,  смерть,  а уж  второй  удар наверное".  Пастер
задумался  на минуту  и спокойно  ответил: "Я  не могу  прервать своей
работы. Я уже предвижу ее конец":

"Advienne que pourra, j'aurai fait  mondevoir" --- "Будь, что будет, я
исполню свой долг".

И  не  досадно ли  после  этого  читать,  как  ставили ему  в  заслугу
возвращение  какого-то  диплома,  отказ   от  какого-то  ордена,  ему,
хладнокровно высказавшему готовность ради науки отказаться от жизни *.

* Французские  шовинисты еще  истекшим летом  пытались воспользоваться
именем  Пастера  для  какой-то  крупной  демонстрации  против  участия
Франции в  Кильских празднествах; потребовалось  формальное запрещение
старого ученого, чтобы остановить их затею. И не одни шовинисты готовы
были эксплоатировать это  славное имя в свою пользу. Пастер  мог бы по
праву применить к себе известное изречение: оборони бог от друзей, а с
врагами я справлюсь сам. Такими непрошенными друзьями были французские
клерикалы  в  эпоху  его  спора  с  Пуше.  Теперь  этот  спор  кажется
отголоском седой старины, но нашему поколению живо памятны подробности
этой страстной  борьбы. Для  людей науки  вопрос о  самозарождении был
просто делом факта, результатом  опытного исследования. Но французские
клерикалы  поспешили  сделать из  него  вопрос  религиозный и  пробный
камень религиозной благонамеренности. Пастер был провозглашен истинным
сыном  церкви, а  на Пуше  и  его сторонников  посыпались обвинения  л
подрывании  основ религии  и  нравственности. Но  вот  что странно:  в
средние века, века искренней, глубокой  веры, ни одна верующая душа не
возмущалась  общепринятым  фактом,  что  какие-нибудь  угри  или  мыши
зарождаются из грязи, что глисты заводятся  в кишках, а не заносятся в
них  с  пищей.  Даже  в  XVIII  веке  люди,  искренно  верующие,  были
сторонниками произвольного  зарождения. Только  в XIX  веке, увидавшем
людей,  дли которых  их  вера стала  предметом  газетной рекламы,  ---
только  в XIX  веке  вопрос, о  самозарождении  стал тревожить  чутную
совесть этих  верующих. Французские  клерикалы стали приходить  в ужас
при одной мысли, чтобы какая-то микроскопическая точка, в которой сами
они, конечно, никогда не признали  бы живого существа, чтобы эта точка
могла возникнуть без родителей. Для людей, хладнокровно относившихся к
происходившему, было  ясно, что дело  не в  самом предмете спора,  а в
попытке  клерикалов воспользоваться  им, чтобы  наложить свою  руку на
свободу  научного  исследования;  если астрономия  и  геология  успели
ускользнуть, то тем более желательно  было дать почувствовать эту руку
- биологии.  И все, кому  была дорога  эта свобода, это  право ученого
приходить в конце своего исследования  к тем выводам, которые вытекают
из  данных опыта,  а не  к тем,  которые заранее  предписаны из  Рима,
конечно,  были возмущены  преследованиями, которым  подвергся Пуше  со
стороны клерикальной печати. Этим объясняется отпор, данный ей прессой
либеральной,  в жару  полемики  не всегда,  впрочем, делавшей  должное
различие  между  Пастером  и его  непризванными  покровителями,  между
блистательно  доказанным научным  фактом и  его эксплоатацией  в видах
преследования той же науки.

Это  высокое  представление  о  служении  науке  Пастер  сумел  словом
и  примером  сообщить и  ближайшим  своим  сотрудникам и  ученикам,  в
последние годы сгруппировавшимся вокруг  него в его институте. Приложи
эти люди свои таланты к  непосредственным услугам жизни, на заводе или
в  медицинской практике,  и все  они  сделались бы  богачами; а  какие
состояния  могли  бы  составиться при  каждом  моментальном  увлечении
общества новыми  завоеваниями науки, пример  тому можно было  видеть в
другой  стране,  по  случаю  преждевременного  провозглашения  способа
лечить  чахотку,  а между  тем  эти  люди, -говорит  автор  интересной
брошюры  "Pasteur  et  les  pastoriens"   *,  ---  являют  нам  пример
самых строгих  добродетелей, обета  бедности и  бескорыстного служения
человечеству,  какие мир  мог  видеть разве  только  в лучшие  моменты
существования первоначальных монашеских орденов.

* Пастер и пастерианцы.

Этот институт,  --- говорит  нам автор,  очевидно, коротко  знакомый с
его  жизнью  и  обитателями,  ---  этот  институт-монастырь  будущего,
посвященный новому культу --- культу науки.

"Испытываешь  какое-то отрадное,  возвышающее  душу  чувство при  виде
исполненной нравственного достоинства жизни этих отшельников, особенно
когда  посравнишь  их   жизнь  с  той  адской  скачкой   в  борьбе  за
существование, которую представляет жизнь наших медиков из мирских".

Этот новый  монашеский орден,  члены которого  прежде всего  как будто
наложили на себя обет бедности, пример чего показал Ру, отклонивший от
себя  и  предоставивших в  распоряжение  института  даже то  небольшое
увеличение  содержания, которое  совет института  присудил ему  за его
открытие,  --- этот  монашеский  орден имеет  и  своих миссионеров  in
partibus infidelium  *, как, например, Иерзена,  которого можно видеть
везде, где  опасность: в Китае  на чуме, на Мадагаскаре  на лихорадке;
имеет он и своих мучеников, как Тюилье, погибший в Египте на холере.

* "В  странах неверующих",  так называли  иезуиты далекие  страны, где
были их миссионеры.

Эти сходства выражаются и в каком-то общинном духе, который витает над
этим учреждением. Лично бескорыстные,  члены института своими трудами,
благодаря все возрастающему спросу на вакцины, дифтеритную сыворотку и
т.д., приобретают для института значительные средства.

"Придет день, --- говорит наш автор, ---  и он уже не далек, когда при
добровольно наложенном  на себя обете  бедности ее членов  сама община
будет богата. В этот день она освободится от министерских субсидий и в
то же время от государственной опеки.

Может быть, я ошибаюсь, ---  продолжает он, --- но мне представляется,
что  институт Пастера,  независимый, сильный  единением своих  членов,
которые противопоставят свою  нравственную чистоту и самоотверженность
все   возрастающей  алчности   врачей-практиков,  что   этот  институт
сделается  силой,   могучей  социальной  силой,  с   которой  придется
считаться.  И  этой  силой  он  воспользуется, в  том  не  может  быть
сомнения, на благо страждущего человечества и ради торжества истины".

В этих словах ученика и  горячего поклонника слышится отголосок самого
учителя,  не  раз вспоминавшего,  что  вся  его деятельность  чуть  не
разбилась  о препятствия,  выражающиеся  этими  словами: "субсидия"  и
"опека".  С  горечью  рассказывает  он  в  одной  из  своих  речей  об
одном  академике, который  в течение  10  лет сам  принужден был  мыть
свою  лабораторную посуду,  потому что  по "штату"  ему не  полагалось
лабораторного  служителя. В  другой  раз,  разоблачая свое  инкогнито,
Пастер  рассказыпаот, как  ему, уже  в то  время знаменитому  ученому,
министр  просвещения отказал  в  каких-то 1500  франках на  устройство
лаборатории на том  основании, что "в бюджете  министерства не имеется
такой статьи".

По  счастью,   борьба  с  этими   "штатами"  и  "статьями"   не  убила
окончательно  энергии  Пастера.  На скудные  собственные  средства  он
устроил  себе лабораторию  где-то на  чердаке Нормальной  школы, а  за
отказ  в  каких-то  жалких  1500  франках  отомстил,  подарив  Франции
миллиарды.  Известны  остроумные  слова Гекспи:  "Пастер  один  своими
открытиями уплатил большую  часть немецкой контрибуции". И  это --- не
звонкая  фраза, а  простое заявление  факта. Одно  шелководство давало
Франции около  100 миллионов в  год, но без вмешательства  Пастера вся
эта обширная  отрасль народного  труда была обречена  на окончательную
гибель. За  двадцать с  лишком лет,  истекших со  времени исследований
Пастера, это  составит уже  более двух  миллиардов. Усовершенствования
техники виноделия и пивоварения выражаются также почтенной цифрой.

Ущерб от  одной сибирской язвы оценивался,  приблизительно, в двадцать
миллионов  в год.  А сколько  жизней  спасло применение  его учения  в
хирургии и других областях медицины!  Один из наших известных хирургов
говорил  мне, что  смертность в  лазаретах Севастополя  и в  последнюю
восточную войну  представляла почти обратные цифры:  сколько умирало в
Севастополе,  столько выздоравливало  на  полях Болгарии.  И все  это,
главным  образом,  благодаря  Листеру  *, т.е.  Пастеру.  А  давно  ли
весь  мир  дрогнул  от  восторга, услыхав,  что  один  из  страшнейших
бичей последнего  времени, дифтерит,  подчинился методу Беринга  и Ру,
этих  проницательных исследователей,  развивающих далее  идеи Пастера.
Статистики,  если не  ошибаюсь, оценивают  человеческую голову  в 4000
франков.  Сколько  миллиардов  составят  в  самом  отдаленном  будущем
эти  четыре  тысячи,  помноженные  на  миллионы  человеческих  жизней,
спасенных  рациональным  лечением   болезней  или  их  предупреждением
здравой гигиеной, в первый раз почувствовавшей под собой прочную почву
благодаря Пастеру.

* Знаменитый английский хирург,  применивший в хирургии приемы Пастера
для  обеззараживания  ран  и   тем  уменьшивший  в  огромных  размерах
смертность от неудачных операций.

Но  есть  еще  нечто,  чего  статистики не  выражают  цифрами,  это  -
человеческие страдания. И кто попытается, хотя приблизительно, оценить
ту бездну горя  и душевных мук, которые исчезли и  еще исчезнут с лица
земли, благодаря Пастеру?

Старые алхимики  были не совсем неправы,  усматривая какое-то сходство
между  брожением  и  философским  камнем. В  руках  Пастера  учение  о
брожении,  если  не  превратило  в золото  неблагородные  металлы,  то
сделалось источником  несметного народного богатства; если  не открыло
тайны вечной молодости, то открыло  тайну сохранения жизни и борьбы со
смертью.

Где же исходная точка всех  этих блестящих успехов, уже осуществленных
в  практической жизни,  и тех  еще более  светлых ожиданий  в будущем,
предела которым  мы еще не в  состоянии даже и предвидеть?  В одной из
своих речей Пастер сам нам дает  ответ. Это чуть ли не единственная из
.его речей, в которой звучит  нотка страстности, столь чуждая всем его
другим произведениям.

Самое  название  указывает  на  жгучесть затронутого  в  ней  вопроса:
"Почему во Франции не нашлось людей, когда ей грозила гибель?"

Ответ Пастера,  может быть,  и односторонен, но  зато, по  отношению к
этой стороне  вопроса, едва  ли кто  из явивших в  XIX веке  людей мог
сказать более авторитетное слово. Ответ Пастера прост и ясен.

Во  Франции  в  минуту  бедствия   не  нашлось  людей  потому,  что  в
предшествовавшие  ему  десятилетия пало  уважение  к  точной науке,  к
теоретической  науке, чистой  науке  ---  единственной науке,  которую
признавал  Пастер.   "Высшие  школы  в  то   время  давали  философов,
историков,  литераторов", или,  наоборот,  "только людей,  прилагавших
свои  труды  к  промышленным операциям,  эксплоатации  мин,  постройке
железных дорог и т.д.", или, наконец, к медицине.

"Но  медицина,  --- замечает  Пастер,  ---  к сожалению,  представляет
скорее  искусство,  чем  науку,  и потому  влияние  ее  факультета  на
распространение знаний не могло быть ощутительным...

А в  то же  время, ---  говорит Пастер, ---  наш соперник,  не жертвуя
ничего на  развитие своего земледелия и  своей промышленности, отдавая
все на  нужды науки, сумел  перевести большую часть своего  уважения и
своих жертв на работы ума в наиболее их возвышенной и свободной части,
на прогресс  наук во всем,  что они  имеют бескорыстного, так  что имя
Германии связано по какой-то ассоциации идей с именем университетов...

Он  понял, ---  этот народ,  ---  что не  существует прикладных  наук,
а  только  приложения  науки...  Он понял  также,  что  первоначальное
образование может принести счастливые плоды  при том лишь условии, что
оно будет вдохновляться высшим образованием".

Он понял, что "на той ступени  развития, которой мы достигли и которая
обозначается именем «новейшей цивилизации», развитие наук, быть может,
еще более необходимо для  нравственного благосостояния народа, чем для
его  материального процветания...  Общественные  же  власти Франции  с
давних  пор не  ведали  этого закона  соотношения между  теоретической
наукой и практической жизнью".

Вспомним, что эти слова были сказаны в 1871 году, вспомним горячую, до
болезненности страстную  любовь Пастера к  своей родине, и  мы поймем,
чего ему  стоило это восхваление  Германии, какой жгучей  болью должна
была  отзываться  в  нем  хотя  бы эта  мысль,  что,  произнося  слово
"университет", невольно хочется прибавить  "немецкий", --- и мы оценим
глубокую  искренность высказываемых  им истин  и то  высокое значение,
которое они, очевидно, имели в его глазах.

Но послушаем его далее:

"Мало   найдется  людей,   понимающих  истинное   происхождение  чудес
промышленности  и народных  богатств. Как  одно только  доказательство
этого, я теперь  приведу все чаще и чаще употребляемое  в разговоре, в
официальном  языке, в  разного  рода  статьях совершенно  неподходящее
выражение  «прикладные науки».  Кто-то  недавно  в присутствии  одного
очень  талантливого министра  выразил сожаление,  что научные  карьеры
бросаются  людьми, которые  с  успехом могли  бы  на них  подвизаться.
Возражая на это,  государственный муж старался доказать,  что этому не
следовало удивляться, так как в настоящее время значение теоретических
наук уступило свое место господству прикладные наук.

Нет  ничего  ошибочнее этого  мнения,  нет  ничего, осмелюсь  сказать,
опаснее  тех  последствий, которые  могут  возникнуть  на практике  из
подобных  слов.   Они  запечатлелись  в  моей   памяти  как  очевидное
доказательство  настоятельной  необходимости реформ,  требуемых  нашим
высшим  образованием. Нет,  тысячу  раз нет,  не  существует ни  одной
категории наук, которой  можно было бы дать  название прикладных наук.
Существуют науки и приложение наук,  связанные между собой, как плод и
породившее его дерево".

Этот человек, более чем какой  другой смертный сделавший для жизненной
практики,  человек,  совершивший  переворот  почти  во  всех  отраслях
прикладного знания  --- в технологии,  в земледелии, в  медицине, этот
человек  отрицает самостоятельное  значение  этого знания,  отказывает
ему  в  названии  науки.  Не  должны  ли мы  видеть  в  этом  ответ  и
урок  житейским  мудрецам  и  негодующим  моралистам,  всегда  готовым
превозносить материальное и нравственное превосходство так называемого
прикладного  знания  перед  знанием  теоретическим.  Неужели  и  после
этого  яркого   примера  будет  считаться   государственной  мудростью
признание полезности практической деятельности тех, кто порою вкривь и
вкось  будут только  повторять,  применять сделанное  Пастером, и  ---
бесполезности теоретической  деятельности новых  Пастеров, тех,  кто в
своих лабораториях  будут продолжать его  дело? Неужели и  после этого
яркого примера найдутся смелые  моралисты, которые будут проповедывать
о  праздной,   эгоистической  жизни   ученого,  не   отзывающегося  на
непосредственные запросы жизни?

В  воображении  невольно  возникает  такая  картина.  Лет  сорок  тому
назад на  чердачок Ecole  Normale проникает  один из  таких негодующих
моралистов  и, застав  там  бледного,  больного человека,  окруженного
бесчисленными колбочками, разражается красноречивыми обличениями.

"Стыдитесь, --- говорит он ученому, --- стыдитесь, кругом вас нищета и
голод, а вы возитесь с какой-то болтушкой из сахара и мела! Кругом вас
люди бедствуют от ужасных жизненных  условий и болезней, а вас заботит
мысль, откуда взялась эта серая грязь на дне вашей колбы! Смерть рыщет
кругом  вас, уносит  отца, опору  семьи, вырывает  ребенка из  объятий
матери,  а  вы ломаете  себе  голову  над  вопросом, живы  или  мертвы
какие-то точки под вашим  микроскопом. Стыдитесь, разбейте скорее ваши
колбы, бегите  из лаборатории,  разделите труд с  трудящимися, окажите
помощь болящему, принесите слово утешения там, где бессильно искусство
врача".

Красивая роль,  конечно, выпала  бы на  долю негодующего  моралиста, и
ученому пришлось  бы что-нибудь пробормотать в  защиту своей праздной,
эгоистической забавы.

Но как изменились бы зато эти роли, если бы наши воображаемые два лица
встретились снова  через сорок лет.  Тогда ученый сказал  бы моралисту
приблизительно  следующее:  "Вы были  правы,  я  не разделял  труда  с
трудящимися,  ---  но  вот  толпы  тружеников,  которым  я  вернул  их
миллионный заработок;  я не подавал  помощи больным, --- но  вот целые
населения, которые  я оградил  от болезней. Я  не приходил  со словами
утешения к  неутешным, --- но  вот тысячи  отцов и матерей,  которым я
вернул их детей, уже обреченных на неминуемую смерть".

А в заключение  наш ученый прибавил бы со  снисходительной улыбкой: "И
все это было там, в той колбе с сахаром и мелом, --- в той серой грязи
на дне этой колбы, в тех точках, что двигались под микроскопом".

Я   полагаю,  на   этот  раз   пристыженным  оказался   бы  благородно
негодовавший, но близорукий моралист.

Да, вопрос не в том, должны  ли ученые и наука служить своему обществу
и человечеству,  --- такого  вопроса и  быть не  может. Вопрос  в том,
какой  путь короче  и вернее  ведет к  этой цели.  Идти ли  ученому по
указке практических  житейских мудрецов  и близоруких  моралистов, или
идти,  не  возмущаясь их  указаниями  и  возгласами, по  единственному
возможному пути, определяемому  внутренней логикой фактов, управляющей
развитием  науки; ходить  ли  упорно, но  беспомощно  вокруг да  около
сложного, еще не поддающегося анализу науки, хотя практически важного,
явления, или сосредоточить  свои силы на явлении,  стоящем на очереди,
хотя  с виду  далеком от  запросов жизни,  но с  разъяснением которого
получается ключ к целым рядам практических загадок?

Никто не станет  спорить, что и наука имеет свои  бирюльки, свои порою
пустые забавы,  на которых  досужие люди упражняют  свою виртуозность;
мало  того,  как всякая  сила,  она  имеет  и увивающихся  вокруг  нее
льстецов и присосавшихся к ней паразитов. Конечно; но не разобраться в
этом ни  житейским мудрецам,  ни близоруким  моралистам, и,  во всяком
случае,  критериумом истинной  науки  является не  та внешность  узкой
ближайшей пользы,  которой именно  успешнее всего  прикрываются адепты
псевдонауки, без  труда добывающие  для своих  пародий *  признания их
практической важности и даже государственной полезности.

*  См.  мою брюшюру  "Пародия  науки",  в  которой я  изобличил  такую
пародию  профессора А.П.  Богданова, когда-то  пользовавшегося большим
авторитетом, особенно у властей.

Великий ученый,  смерть которого мы  теперь оплакиваем, еще  при жизни
своей  оказал  такое  влияние  на  практические  стороны  человеческой
деятельности, какого, конечно, не оказывал  еще ни один человек за всю
историю цивилизации. В трех самых древних из человеческих искусств его
деятельность вызвала переворот.

В технологии  он поставил на  прочную рациональную почву  все отрасли,
имеющие в  основе процессы брожения,  и дал рациональные  указания для
практики шелковода.

В земледелии его идеи, благодаря тому развитию, которое они получили в
работах Шлезинга, Гельригеля и Виноградского, бросают совершенно новый
свет на самые основные приемы н задачи агронома.

В медицине...  но, кажется,  уже достаточно  было нами  сказано, чтобы
выяснить  его значение  в этой  области;  один из  его учеников  очень
остроумно  замечает,  что  в   истории  цивилизации  после  того,  как
первобытный  человек перестал  бояться  лесного зверя,  не было  более
решительного шага, как  тот, который сделал Пастер,  научив бороться с
еще более опасными, вездесущими микробами.

И   этот-то   по   результатам   своих   трудов,   казалось   бы,   по
преимуществу  практический деятель  был убежденным  теоретиком, только
за  теоретическими  знаниями  признавал выдающееся  значение  и  право
называться наукой.

Но,  может  быть,  заметят:  тем  не менее,  он  целые  годы  посвящал
вопросам исключительно узко практическим: пивоварению, болезни червей,
сибирской  язве.  Да,  но  но   потому,  чтобы  его  ослепляла  только
практическая  польза, могущая  получиться  в результате  его работ,  а
потому,  что именно  эти  вопросы  всего удобнее  и  как раз  во-время
укладывались в рамку его теории, представляли самую удачную конкретную
форму для ее проверки и дальнейшего развития.

Почему   обратил   он   внимание   на  пивоварение,   а   не   занялся
сахароварением? Ведь  также со всех сторон  раздавались сожаления, что
эта  отрасль  промышленности,  гордость  французской  нации,  одно  из
завоеваний  французского  научного  гения,  падает,  уступая  немецкой
конкуренции.

Почему  не  занялся он  филлоксерой,  наносившей  Франции еще  большии
ущерб, чем пебрина?  А просто потому, что эти факты  не имели никакого
отношения к  его теории. На  эти явления  она не могла  пролить нового
света,  а Пастер,  конечно,  не находил  полезным  блуждать во  мраке,
руководясь  только похвальным  желанием  добра, но  не имея  оснований
ожидать его осуществления.

Итак,  что  же  сообщило  новый  толчок  целым  областям  практической
деятельности,  что  вызвало  в  особенности тот  небывалый  в  истории
человеческих  знаний  переворот,  который   дал  право  одному  медику
сказать, что отныне история медицины будет делиться на два периода: до
и после Пастера? Что, собственно, случилось?

Химик   остановил   свое    внимание   на   физиологическом   вопросе,
представлявшем  исключительно теоретический  интерес,  а в  результате
изменилась   судьба   самой   осязательно-практической   из   отраслей
человеческой  деятельности  *.  Практической, в  высшем  смысле  этого
слова, оказалась не вековая практика  медицины, а теория химика. Сорок
лет теории  дали человечеству то, чего  не могли ему дать  сорок веков
практики. Вот главный урок, который  мы должны извлечь из деятельности
этого великого ученого.

*  При  этом  невольно  приходит  на память  и  другой  пример:  физик
Гельмгольц  совершенно изменил  характер  целой  отрасли медицины  ---
офтальмологии.

Стоя  на  пороге   XX  века  и  вдумываясь   в  значение  деятельности
типического  представителя   XIX  века,  отметившего   себя  небывалым
развитием  науки о  природе  и в  небывалой  мере увеличившего  власть
человека над природой, невольно переносишься мыслью в другую, не менее
великую эпоху  на рубеже XVI  и XVII  веков, когда только  возникло то
движение,  результаты  которого мы  теперь  так  ясно видим.  Невольно
останавливаешься на  вдохновенных словах мыслителя,  почти опьяненного
первыми успехами точного  знания и на пороге XVII  века пророчившего о
том, что  XIX век  в значительной мере  уже успел  осуществить. Пастер
является как бы живым воплощением  того идеала знания, который витал в
восторженном  воображении  Бэкона\footnote{Франсис Бэкон  ---  великий
английский  мыслитель,  первый  понявший  истинное  значение  науки  и
пророчивший ее предстоящее развитие. Прошедшие после того три столетия
блестящим образом подтвердили это пророчество.}.

В  третьем  афоризме своей  бессмертной  книги  Бэкон раз  и  навсегда
устраняет эту  ходячую антитезу  между теорией  и практикой  --- между
знанием  и властью  человека над  природой. Что  в теории  причина, то
средство  для практики.  Только знание  причины явлений  дает человеку
в  руки  и   средство  управлять  ими.  А   находить  причину  явлений
нас  учит  только  опыт.  Но   опыт  может  быть  двоякий:  существуют
experimenta  fructifera, опыты  плодоносные,  когда  человек в  погоне
за  ближайшей осязательной  пользой  нередко даже  вовсе не  достигает
своей цели  и, во всяком  случае, осуществляет немногое;  существуют и
experimenta  lucifera, опыты  светозарные,  в  которых, не  руководясь
узкой утилитарной целью, он стремится  только к объяснению явлений и в
результате освещает целые обширные области фактов.

Луи  Пастер и  был этим  гением экспериментального  метода, обладавшим
тайною этих "светозарных опытов", которые, объясняя природу, тем самым
сообщают  человеку власть  над  нею. Он  был  тот человек,  пришествие
которого  восторженно  возвещал  Бэкон,  ---  "человек,  истолкователь
природы и ее властелин", Homo naturae minister et interpres *.

*  Может  быть,   заметят,  что  мой  перевод  этих   слов  не  вполне
соответствует  латинскому тексту  Бэкона,  но зато  он приближается  к
непосредственно предшествующим  словам, т.е. к самому  заглавию книги:
Novum Organum  --- de interpretatione  naturae sive de  regno hominis.
Новое  орудие, или  об истолковании  природы и  о пришествии  царствия
человека.

Через два  года (в  1920 г.)  исполнится триста  лет со  дня появления
этого великого  произведения, и, если человечество  успеет очнуться от
охватившего его припадка острого безумия, оно, конечно, помянет одного
из величайших своих учителей. [Эта приписка сделана в 1918 году].


\begin{center}
  \textbf{А.А.ЛЮБИЩЕВ}
\end{center}


\begin{center}
  \textbf{ЛИНИИ ДЕМОКРИТА И ПЛАТОНА В ИСТОРИИ КУЛЬТУРЫ}
\end{center}

\clearpage

\section{ПРЕДИСЛОВИЕ АВТОРА}

Предисловия  обычно  пишутся  по  окончании написания  книги.  Если  я
отступаю  от  этого  разумного  обычая,   то  потому,  что  мне  скоро
исполнится  72 года,  а в  таком  возрасте я  не вправе  рассчитывать,
что  сумею окончить  труд,  рассчитанный  лет на  7--8.  Но круг  моих
друзей и  корреспондентов, проявляющих  интерес к  моим произведениям,
чрезвычайно  обширен: кроме  биологов разных  специальностей я  мог бы
указать медиков, физиков,  химиков, математиков, историков, филологов,
философов, юристов,  литераторов; я  поэтому вправе  рассчитывать, что
это сочинение вызовет  интерес в достаточно широких  кругах. Всем этим
возможным читателям  необходимо разъяснить  цель, причину  и программу
настоящей книги,  которая задумана  как главное сочинение  моей жизни,
резюмирующее все те мысли, которые накопились за несколько десятилетий
достаточно напряженной работы.

Автор   ---   биолог,   и   главное  содержание   книги   ---   разбор
общебиологических  представлений. Если  эта центральная  часть обросла
философскими  и  методологическими   размышлениями  (начало  книги)  и
соображениями по  части гуманитарных дисциплин, то  это есть следствие
того,  что, начав  с узкой  специализации, автор  все больше  и больше
убеждался в  единстве человеческого познания. Становилось  все более и
более  ясным,  что,  во-первых,  биология вообще,  в  особенности  так
называемая «описательная биология»,  морфология и систематика, требует
совершенного пересмотра тех положений,  постулатов или аксиом, которые
сознательно или бессознательно  кладутся биологами при конструировании
своих  теорий. Во-вторых,  что  такой пересмотр  немыслим без  ревизии
многих   гносеологических  и   онтологических  постулатов,   т.е.  тех
положений, которые  лежат в основе методологии  науки и мировоззрения.
Знакомство с  наукой у  меня началось очень  рано ---  при определении
насекомых в 1903 году, и уже ошибки в определении поставили меня перед
лицом какой-то загадки. Начав работу как узкий специалист, дарвинист и
сознательный  нигилист  типа  Базарова,  я  постепенно  расширял  круг
своих  интересов и  начинал сознавать  необходимость пересмотра  самых
разнообразных и  часто противоречивых постулатов,  которые выдвигались
как  непреложные  истины  представителями  разнообразных  направлений,
господствующих  в  тех или  иных  областях  знания. Излагать  подробно
эволюцию моих взглядов значило бы написать свою идейную автобиографию,
что заняло бы слишком много  места. Я ограничусь перечислением главных
постулатов из  ряда областей человеческой мысли,  которые мне пришлось
пересмотреть  и в  значительной  части отвергнуть,  заменив их  иными,
более  обоснованными. Так  как  автор ---  биолог,  и размышления  над
биологическими проблемами  составляют главное  содержание книги,  то я
начну  с  биологии и  изложу  их,  но не  в  логическом  порядке, а  в
том, в  котором они  постепенно влекли  к размышлению  над постулатами
общефилософского значения.

\begin{center}
  \textbf{А. Постулаты биологии}
\end{center}

\begin{enumerate}

  \item Определительные таблицы (в отличие от ключей) стремятся хотя
        бы в первом приближении отобразить естественную систему
        организмов;

  \item естественная система иерархична, как всякая система;

  \item система организмов, имея историческое обоснование, не может
        быть номотетической дисциплиной;

  \item форма организмов есть эпифеномен многочисленных физических
        сил, в силу сложности их взаимодействия не допускающая
        математической трактовки;

  \item проблема приспособления есть ведущая проблема морфологии;

  \item морфология подчинена физиологии: морфологические проблемы
        являются еще не разрешенными физиологическими или
        историческими проблемами;

  \item естественный отбор есть ведущий фактор эволюции;

  \item человек есть единственное целеполагающее начало в природе;

  \item все поведение животных и человека --- сумма рефлексов;

  \item понятие красоты возникло в связи с половым отбором;
        самостоятельного объективного значения красота не имеет;

  \item биология целиком сводима к физике и химии в том смысле, что мы
        не имеем основания полагать в организмах какие-либо силы или
        сущности, отсутствующие в неорганическом мире;

  \item витализм в любых его формах бесплоден методологически и
        неприемлем с точки зрения мировоззрения.

\end{enumerate}

\begin{center}
  \textbf{Б. Постулаты методологии науки}
\end{center}

\begin{enumerate}

  \item Развитие науки --- постепенное накопление окончательно
        установленных истин, не подлежащих ревизии;

  \item история науки поэтому имеет второстепенное значение;

  \item существует резкая грань между номотетическими и
        идиографическими науками;

  \item научные объяснения отличаются от ненаучных тем, что они
        соответствуют «реальному», «позитивному», «монистическому»
        или «материалистическому» мировоззрению: гносеология
        подчинена онтологии;

  \item историческая роль философии в науке сыграна и не подлежит
        восстановлению;

  \item постулат научного оптимизма заставляет стремиться к истине
        независимо от тех последствий, к которым приведет это
        стремление;

  \item единственно допустимый метод --- индуктивный, исходящий из
        фактов, свободный от всякой философской предвзятости;

  \item при наличии объяснения, удовлетворяющего четвертому
        постулату, мы должны его придерживаться, если: а) не
        существует иного объяснения и б) если предлагаемые иные
        объяснения противоречат этому постулату;

  \item все формы идеализма методологически бесплодны.

\end{enumerate}

\begin{center}

  \textbf{В. Постулаты онтологии}

\end{center}

\begin{enumerate}

  \item Монистическая философия --- единственно допустимая в
        науке;

  \item все существующее локализовано во времени и пространстве;

  \item реальное значение имеют только материальная и действующая
        причины; формальная и конечная причины в биологии носят лишь
        фиктивный характер;

  \item единственно реальное в природе --- атомы, шире ---
        элементарные частицы. Дифференциальный закон определяет
        однозначно положение нового этапа относительно уже
        пройденного;

  \item только меристическое миропонимание научно, холистическое же
        ненаучно;

  \item видимость холистических начал создается в результате борьбы
        (столкновения) и гибели неудачных комбинаций; нет гармонии
        как руководящего принципа;

  \item научное мировоззрение всегда было в корне противоположно
        религиозному. Поэтому всякая попытка ввести понятия,
        способные поддерживать религиозные предрассудки, является
        регрессом в науке.

  \item недопустима двойственная истина; мировоззрение должно быть
        единым в онтологии, биологии, этике и социологии.

\end{enumerate}

\begin{center}

  \textbf{Г. Постулаты этики, социологии и политики}

\end{center}

\begin{enumerate}

  \item Единое мировоззрение, постулируемое в последней строке
        последнего раздела, есть диалектический и исторический
        материализм, находящийся в постоянной и непримиримой борьбе
        со всеми разновидностями идеализма и поповщины;

  \item ведущим началом истории культуры являются экономические
        факторы; надстройка не имеет самостоятельного значения;

  \item ведущим фактором развития общества является классовая
        борьба;

  \item этические понятия не имеют самостоятельного,
        общечеловеческого значения, они подчинены интересам
        классовой, а следовательно, и политической борьбы;

  \item не имеет также самостоятельного значения и учение об
        искусстве и красоте, эстетика;

  \item политические критерии позволяют устанавливать истинное и
        ложное в науке и философии даже лицам не компетентным в
        частных разделах науки;

  \item наука и философия --- служанки социологии и политики.

  \item Платон не имеет права считаться основоположником социализма
        и коммунизма. Истинный научный коммунизм не имеет с Платоном
        ничего общего.

\end{enumerate}

Изложенные  постулаты  в известной  мере  покажутся  непонятными и  не
относящимися к делу. Я постараюсь показать их взаимосвязь.

Первый набросок, зародыш настоящего  сочинения, был составлен мной для
себя в 1917 году. Кое-какие частные вопросы удалось довести до печати:
1) О форме естественной системы организмов (1923); 2) Понятие эволюции
и кризис  эволюционизма (1925);  3) О природе  наследственных факторов
(1925).  Два доклада  на  2-м  (1927) и  4-м  (1930) съездах  зоологов
(«Понятие номогенеза» и  «Логические основания современных направлений
биологии») напечатаны  только в форме  тезисов. Сейчас сданы  в печать
две статьи.

Естественно возражение: один человек не может написать труд такого
диапазона в наш век специализации. Литература необъятна, а
использовать ее всю считается необходимым. На это можно ответить. По
философским вопросам сейчас выступает ряд видных ученых самых
разнообразных специальностей; создалась особая дисциплина «Философия
науки», которой посвящен ряд журналов и много книг. Несомненно, что
сейчас наряду со специализацией идет процесс сближения наук, синтеза
различных точек зрения. Моя работа имеет некоторое сходство по замыслу
с известной книгой Бернала «Наука в истории общества» и в значительной
мере является антагонистом этой содержательной и интересной книги. Для
биологии, сейчас вступающей в новый период своего развития, такой
процесс осмысления имеет еще большее значение, чем для неорганических
наук, и вместе с тем биология гораздо теснее связана с политическими
проблемами, чем физика и другие точные науки: закрывать глаза на это
--- значит уподобиться страусу.

За свою жизнь я много читал и думал по общебиологическим и философским
вопросам  и в  этом отношении  я квалифицирован  больше, чем  огромное
большинство специалистов-биологов.  Мой интерес к  математике заставил
меня  познакомиться  с  рядом  разделов этой  замечательной  науки,  и
поэтому  я легче  разбираюсь  в философии  точных  наук, чем  биологи,
морфологи  и систематики,  несведущие, как  правило, в  математике. Не
так  давно игнорирование  математики провозглашалось  многими ведущими
биологами как  обязательный для  биологов постулат. С  другой стороны,
математики  и физики,  выступающие  с  общефилософскими работами,  как
правило, не  понимают всей огромной сложности  биологических проблем и
всей  противоречивости взглядов  умных биологов.  Все эти  соображения
давали мне всю жизнь уверенность в разумности предпринятого мной дела,
и  я имею  право утверждать,  что  если моя  книга будет  недостаточно
убедительна,  то   во  всяком   случае  обвинить  меня   в  недостатке
обдуманности невозможно.

Изложение проблем  мной в значительной степени  ведется в историческом
аспекте  и   этот  аспект  доминирует  в   первой  части,  посвященной
неорганическим  наукам.   Сейчас  связь  биологии   с  неорганическими
науками, физикой  и химией  завоевывает все  больше признания,  но эта
связь  мыслится  трояко:  1)  использование  физической  и  химической
аппаратуры:  против  этого никто  не  возражал  (за исключением  явных
обскурантов); 2)  использование математического аппарата  физических и
химических  теорий;  вполне  почтенное  мероприятие;  3)  утверждение,
что  биология целиком  (хотя  бы принципиально)  сводится  к физике  и
химии  (постулат  А.II).  Моя  попытка  стремится  выяснить  четвертую
возможность  взаимосвязи: установить,  на основе  каких философских  и
общеметодологических представлений достигнуты представителями физики в
самом  широком смысле  (т.е.  всей наукой  о  неорганическом мире)  их
поразительные успехи и  какие уроки может извлечь  биология из истории
философских направлений в физике.

Мне приятно  сознавать, что огромную  моральную поддержку я  получил в
недавней  статье  одного  из ведущих  ученых  современности,  Норберта
Винера  «Высокая  миссия»  (журнал  «Америка», №  51,  1960,  с.  11),
касающейся долга ученых.

«... Они должны  неутомимо искать истину там, где до  тех пор ее никто
не видел. И если они не посвятят этой задаче всех своих сил, мы вправе
сказать, что они не отвечают своему назначению...

Это  очень суровая  миссия. Ее  не взять  на себя  тому, кто  избирает
предметом исследования те  области, где легко пролагать  новые пути, и
кто  уклоняется от  проверки гипотез,  которые в  процессе дальнейшего
изучения могут  оказаться ошибочными. Человек, никогда  не пробовавший
выйти за  пределы своих  возможностей и гордящийся  тем, что  в списке
его  достижений нет  ни  одной  ошибки, так  и  не испытал,  вероятно,
своих  сил до  конца.  Такой человек  скорее  заслуживает не  похвалы,
а  порицания  за то,  что  предпочел  душевное спокойствие  выполнению
духовного  долга... Все  сказанное приводит  к непререкаемому  выводу:
какое бы то ни было искажение истины человеком, призванным служить ей,
есть нарушение  служебного долга, совершенно  аналогичное преступлению
офицера,  бросившего  свою  часть  на поле  боя.  Но  такое  нарушение
воинского устава сурово карается законом, тогда как ученый, изменивший
правде,  может разгуливать  по улицам,  не рискуя  жизнью и  свободой,
поэтому  его  измена ---  нечто  худшее,  чем преступление  дезертира.
И  такой  ученый  заслуживает  по меньшей  мере  позора  и  бесчестия,
составляющих удел разжалованного офицера».

У нас часто слово «измена»  считают почти синонимом слову «изменение»,
хотя этимологически  эти слова  безусловно родственны. Я,  конечно, за
свою жизнь изменил огромному количеству твердых убеждений моей юности.
Но  я  не  изменил  тому формальному  принципу,  который  был  положен
Тургеневым в  определение понятия «нигилист»: «Нигилист,  это человек,
который  не  склоняется  ни  перед  какими  авторитетами,  который  не
принимает  ни одного  принципа  на  веру, каким  бы  уважением ни  был
окружен  этот принцип»  («Отцы и  дети»).  Я поэтому  и сейчас  охотно
называю себя нигилистом в этом исконном, тургеневском смысле слова.

\begin{flushright} Ульяновск, 10 февраля 1962 г. \end{flushright}

\clearpage


\section{ВВЕДЕНИЕ}


\subsection{Обвинение «Линии Платона» в научном обскурантизме}

«Могла ли устареть  за 2000 лет развития философии  борьба идеализма и
материализма?  Тенденций или  линий Платона  и Демокрита  в философии?
Борьба религии и  науки? Отрицания объективной истины  и признания ее?
Борьба  сторонников  сверхчувственного  знания  с  противниками  его?»
«Ленин, Материализм и эмпириокритицизм», соч., т. XIII, с. 106).

Такое  резко  отрицательное  мнение   о  «линии  Платона»  разделяется
не  только  Лениным  и его  многочисленными  единомышленниками.  Почти
тождественную  мысль  высказывает  решительный  противник  коммунизма,
современный  выдающийся ученый  Бертран  Рассел. В  его  очень живо  и
интересно написанной  книге «История западной философии»  (1959) на с.
91 читаем: «Демокрит  --- таково по крайней мере  мое мнение последний
греческий  философ, который  был  свободен  от известного  недостатка,
нанесшего вред всей более поздней древней и всей средневековой мысли».
Кто нанес  вред, ясно из  с. 92:  «Несмотря на гениальность  Платона и
Аристотеля, их  мысль имела  пороки, оказавшиеся  бесконечно вредными.
После них начался упадок  энергии и постепенное возрождение вульгарных
предрассудков.  Новое  мировоззрение  возникло  отчасти  в  результате
победы  католической ортодоксии;  но вплоть  до Возрождения  философия
не  могла обрести  вновь  той энергии  и  независимости, которые  были
свойственны предшественникам Сократа».

Выслушаем  еще  третье  мнение,  на  этот  раз  одного  из  крупнейших
современных зоологов, кильского профессора Реманэ. Здесь речь идет уже
не  о  споре  материализма  и  идеализма,  а  о  споре  механицизма  и
витализма в  биологии, но хотя  понятия витализм и идеализм  далеко не
тождественны,  они  несомненно  сродны,  и потому  суждения  по  этому
вопросу  имеют то  же  значение,  как первые  приведенные.  На с.  343
своей  книги   (1956)  Реманэ  после  весьма   объективного  изложения
противоположных   мнений  пишет:   «...  Большинство   биологов  стоят
практически на той точке зрения, что где-то в области биологии имеется
граница механической разрешимости, будем ли мы ее видеть подобно Максу
Гартману в психическом, что  тоже имеет филогенетическое развитие, или
захватим более  широкие круги.  Поскольку эта граница  не установлена,
механистическое исследование  в широком смысле слова  должно выдвинуть
постулат  объяснимости биологических  процессов,  и успех  оправдывает
такое поведение.  Пока такая  граница не  установлена, механистические
и   виталистические  или   психические   способы  рассмотрения   будут
пересекаться,  несмотря на  совершенно  различные исходные  положения,
причем с  точки зрения  истории науки первый  (механистический) играет
роль  смелого завоевателя,  второй ---  отступающего критика,  который
должен уважать уже завоеванную область».

Ясно, по мнению Реманэ, что механицизм или материализм --- единственно
активное мировоззрение в науке, роль  же витализма в лучшем случае ---
роль  критика,  охлаждающего  чрезмерное  головокружение  от  успехов.
Подобных высказываний  со стороны  образованных, умных,  талантливых и
честных ученых можно привести десятки. %

\subsection{Обвинение в политической реакционности}

Но  этого  мало.  Против  идеализма  (или  витализма)  выдвигается  не
только   обвинение  в   научном  обскурантизме,   но  и   обвинение  в
политической   неблагонадежности  и   с  точки   зрения  человеческого
прогресса. Я не буду цитировать общеизвестные мнения наших ортодоксов.
Приведу  слова  крупного  английского  ученого-марксиста  Д.Бернала  в
его  очень интересной  книге  «Наука в  истории  общества» (1956).  На
с.  34  читаем:  «Сторонники идеалистического  мировоззрения  являются
сторонниками ``порядка'', аристократии и  принятой религии, а наиболее
последовательным  сторонником  идеализма   является  Платон.  Согласно
идеализму,  цель науки  состоит в  объяснении того,  что вещи  таковы,
каковы они есть, и того, что невозможно и дерзко надеяться изменить их
сущность».  Но  почему  же  идеализм продолжает  существовать  и  даже
пытается возрождаться? На  этот счет у Бернала дан ясный  ответ --- с.
35:  «Само  постоянство  этой  борьбы,  несмотря  на  последовательные
победы,  одержанные  материалистической  наукой, показывает,  что  эта
борьба, по существу, касается не  только философии и науки, а является
отражением  политической  борьбы  в   сфере  науки.  На  каждом  этапе
идеалистическую философию  призывали представлять дело  таким образом,
что  имеющееся  в данное  время  недовольство  является иллюзорным,  а
также  оправдывать  существующее  положение  вещей.  На  каждом  этапе
материалистическая   философия   полагалась  на   практический   опыт,
почерпнутый из действительности и на необходимость изменений».

Политическую  реакционность  платонизма  утверждает  и  Б.Рассел.  Его
мнение  может  показаться особенно  убедительным,  так  как он  отнюдь
не  является  марксистом,  считается  по  крайней  мере  по  нашей  же
классификации  идеалистом  (сам он  вообще  считает  спор идеализма  и
материализма устаревшим) и ранние, самые плодотворные годы своей жизни
(когда он  написал вместе с  Уайтхедом свой классический  труд «Основы
математики»,  1910--1913), был  весьма  склонен к  платонизму. Не  так
давно (см.  предисловие С.Яновской  к книге Д.Гильберта  и В.Аккермана
«Основы  теоретической  логики»,  1947,  с. 6)  его  даже  обвиняли  в
агрессивном наступлении на материализм  и защите схоластики; по мнению
С.  Яновской,  «вооруженный  такой  идеологией  Б.Рассел  не  случайно
пропагандирует сейчас  применение атомной  бомбы против  СССР». Сейчас
позиция Б.Рассела изменилась  и он хорошо известен  как активный борец
за мир.  В главе  XII (Влияние Спарты)  уже цитированной  книги Рассел
пишет, что миф о Спарте  оказал влияние на политическую теорию Платона
и  что «для  нас  спартанское государство  представляется в  миниатюре
образцом того государства, которое установили  бы нацисты, если бы они
одержали  победу.  Грекам  это представлялось  иначе».  Более  тяжкого
политического  обвинения,  чем  родство  с нацизмом,  с  точки  зрения
прогрессивного мышления придумать трудно.

\subsection{Защита Платона: философы, математики, физики, биологи}

Не представляет, таким образом, никакого труда представить длинный ряд
неопороченных, умных и независимых  свидетелей обвинения против «линии
Платона», и  если бы представление достаточного  количества свидетелей
одной стороны считалось достаточным для вынесения приговора, то он был
бы  совершенно ясен:  «платонизм виновен».  Но священнейшим  принципом
всякого  судопроизводства даже  при наличии,  казалось бы,  совершенно
ясного  дела,  является  положение:   «да  будет  выслушана  противная
сторона».  И   оказывается,  что  против  мощной   фаланги  свидетелей
обвинения можно выставить не менее мощную фалангу свидетелей защиты.

Прежде  всего  \textit{философы}.  Огромную роль  Платона  в  развитии
философии  не   отрицает  никто   и  сейчас   наблюдается  несомненное
повышение  интереса  к Платону  как  к  философу. Но  этих  свидетелей
не   трудно  отвести.   Правильно  пишет   Б.Рассел:  (с.   152)  «...
совершенные  платоники,  за  немногими исключениями,  невежественны  в
области  математики, несмотря  на  огромное  значение, которое  Платон
придавал  арифметике  и геометрии,  и  несмотря  на огромное  влияние,
которое  они имели  на его  философию. Это  является примером  вредных
последствий специализации: человек, если он потратил так много времени
в  пору своей  юности  на изучение  греческого языка,  что  у него  не
осталось времени на изучение вещей,  которые Платон считал важными, не
должен писать  о Платоне».  Кроме того,  кто такие  философы (конечно,
буржуазные) с  точки зрения марксизма, в  частности Ленина: приказчики
своего  класса,  которым  не  следует   доверять  ни  в  одном  слове.
Независимость философов  оспаривается одной из сторон  и их приходится
отвести.

Труднее  отвести \textit{математиков},  а там  имя Платона  достаточно
популярно.  Даже  в  БСЭ,  второе издание,  в  статье  «Многогранники»
фигурируют «Платоновы тела» --- правильные многогранники.

Под  влиянием Платона  был  сам  Рассел в  период  составления им  его
трехтомного  труда  по  основам  математики,  а  его  соавтор  Уайтхед
продолжал и после работать в чисто платоновском духе.

Значение Платона  (и близкого  ему по  духу Пифагора)  в математизации
науки   не  оспаривают   и   противники.  Б.Рассел   пишет,  с.   179:
«Платон  под  влиянием  пифагорейцев чрезмерно  уподоблял  все  прочие
знания  математике.  Он  разделял  ошибку  со  многими  из  величайших
философов,  но это  тем  не  менее было  ошибкой».  В настоящее  время
пифагорейско-платоновская линия по сплошной математизации всего нашего
знания идет от  триумфа к триумфу и  потому не так уж  ясно, чтобы эта
точка зрения  была ошибочной. Но  даже если  бы это было  ошибкой, ---
какая  это плодотворная  ошибка!  «Тьмы низких  истин  нам дороже  нас
возвышающий обман» --- полезно вспомнить это замечательное изречение.

Марксист Бернал  также свидетельствует (с. 104):  «Независимо от того,
был ли Пифагор  легендарной фигурой или нет, школа,  носившая его имя,
была достаточно реальной и оказывала  огромное влияние в более поздние
времена,  особенно  благодаря  ее наиболее  выдающемуся  представителю
---  Платону... Но  независимо  от того,  был  ли Пифагор  зачинателем
этого  учения или  только передатчиком,  установленная им  связь между
математикой, наукой и философией никогда уже не утрачивалась».

В  пользу  Платона и  Пифагора  свидетельствуют  и многие  современные
\textit{физики}. Один из крупнейших  современных физиков Гейнзенберг в
блестящей статье (1958) прямо пишет,  что в физике совершается поворот
от  Демокрита  к  Платону.  Гейзенберг  не  одинок.  Более  или  менее
платоновские идеи высказывали такие ученые, как Эддингтон, Джинс и ряд
других. Возникло даже целое движение неопифагореизма.

В   \textit{биологии}    тоже   идет   брожение.    Выражением   чисто
демокритовской   линии  является   учение  Дарвина   о  ведущей   роли
естественного  отбора  в  эволюции  организмов, и  это  направление  в
настоящее  время   несомненно  господствует.  Но   имеется  достаточно
мощная  оппозиция.  Наблюдается  возрождение многих  идей,  по  мнению
дарвинистов «окончательно  опровергнутых» и, самое главное,  в истории
биологии эти  весьма плодотворные  идеи имели несомненное  отношение к
платонизму.  Процесс математизации  биологии идет  рядом путей.  Имеет
место  проникновение  математической  статистики,  слабо  связанной  с
тем  или  иным  философским  направлением.  Есть  демокритовский  путь
преимущественно в физиологии. Но  математика проникает и в морфологию,
и автор  превосходной сводки по  этому вопросу, Дарси  Томпсон (1942),
принужден часто  вспоминать Платона  и Пифагора,  хотя с  точки зрения
мировоззрения он еще сохраняет демокритовское понимание природы.

Конечно,  в  настоящее время  в  точных  и естественных  науках  линия
Платона представлена  меньшинством, но это меньшинство  имеет заметную
тенденцию к  росту. Вопросы в  науке не решаются  большинством голосов
и,  кроме   того,  большинство  ученых   придерживается  традиционного
мировоззрения, не давая  себе труда ворошить основы  науки. Наконец, в
период Ренессанса под знаменем Платона работало очень много выдающихся
деятелей науки, о чем нам придется в свое время поговорить.

\subsection{Защита Платона политическими мыслителями}

Но  среди великих  мыслителей  прошлого, сознательно  придерживавшихся
линии  Платона, особенно  много  \textit{политических мыслителей}.  Об
этом прекрасно изложено  у Б.Рассела в той же главе  о влиянии Спарты.
Он  считает,  что  спартанская  действительность  была  раскритикована
Аристотелем,  но,  несмотря  на  это,  Спарта  вызывала  восхищение  у
остальных  греков  в значительной  степени  благодаря  мифу о  Спарте,
оказавшему  влияние  на политическую  философию  Платона  и на  теории
бесчисленных  последующих писателей.  Миф  этот  в разработанном  виде
содержится в  жизнеописании Ликурга,  принадлежащем Плутарху,  с. 119:
«Но в воображении людей  сохранилась не Спарта, описанная Аристотелем,
а  мифическая  Спарта  Плутарха  и философская  идеализация  Спарты  в
``Государстве'' Платона.  Из столетия  в столетие молодые  люди читали
это сочинение  и загорались  честолюбивым стремлением  стать Ликургами
или философами-царями». В истории большую роль играют воспоминания, с.
120:  «Из  этих  воспоминаний воспоминания  Платона  имели  наибольшее
значение в  эпоху раннего христианства, а  воспоминания Аристотеля ---
для  средневековой  церкви, но  после  эпохи  Возрождения, когда  люди
начали ценить  политическую свободу, они обращались  главным образом к
Плутарху. Плутарх оказал глубокое  влияние на английских и французских
либералов XVIII  века и на  основателей Соединенных Штатов.  Он оказал
влияние  на романтическое  движение  в Германии  и продолжал,  главным
образом,  косвенно,  влиять на  немецкую  мысль  вплоть до  настоящего
времени.  В некоторых  отношениях  его влияние  было положительным,  в
других --- дурным; что касается Ликурга и Спарты, то оно было дурным».

Совершенно  несомненна  роль  Платона в  развитии  социалистических  и
коммунистических идей. Это наши марксисты решительно отрицают, так как
коммунизм Платона был не всеобщим, а касался только правящей верхушки.
Но  тогда  почему  афинский  государственный  строй  те  же  марксисты
называют  демократией,   хотя  фактически  в   Афинах  демократические
принципы касались только ограниченного круга полноправных мужчин и вся
афинская  государственность была  связана с  беспощадной эксплуатацией
рабов  и  союзников?  Ни  один  строй  не  выкристаллизовывался  сразу
полностью.  Поэтому   мы  вправе  считать  Афины   прообразом  (весьма
несовершенным)  настоящей демократии,  и  тогда «Государство»  Платона
будет не  менее совершенным прообразом  справедливого государственного
строя. Так  и думали великие предшественники  современного социализма,
так называемые утопические социалисты и коммунисты. Целиком на Платоне
строили  свои учения  признанные  предшественники научного  социализма
Томас  Мор  и Кампанелла,  а  если  мы  познакомимся с  книгой  нашего
советского  ученого В.П.Волгина:  «Французский утопический  коммунизм»
(1960), то увидим, что и в более поздние времена платоновский идеализм
входил  в системы  многих  утопических  коммунистов. Волгин  указывает
(с.  261),  что  французские  коммунисты XVIII  века,  за  исключением
Мелье (добросовестно  исполнявшим в течение многих  лет противоречащие
его  убеждениям обязанности  священника),  отнюдь  не отказывались  от
«естественной» или  «рациональной» религии, находили в  своих системах
место  для идей  Верховного Существа  и бессмертия  души. Деистические
тенденции  не были  чужды даже  бабувизму. К  религиозным христианским
традициям   примыкали   в   своей  пропаганде   многие   представители
утопического  коммунизма сороковых  годов.  «Они вполне  гармонировали
с  основными  положениями  того  направления  коммунистической  мысли,
которое считало жертвенность краеугольным  камнем коммунизма, а мирное
просветление  умов  ---  средством  его осуществления;  но  иногда  мы
находим  также попытки  придать  христианской  традиции смысл  учения,
обосновывающего революционные действия».

Излагая взгляды  утописта Вильгарделя, Волгин говорит  его словами (с.
254): «Изучая различные планы реформ,  которые были созданы от Платона
до наших  дней, я пришел  к заключению, что наиболее  достойной успеха
является  традиция,  идущая  от  первых  христиан  и  отцов  церкви  и
продолженная Мором,  Фенелоном, Мабли,  Морелли и другими.  Зерно этой
традиции составляет формула: ``от  каждого по способностям, каждому по
потребностям''. Но  существование этого идеала,  заявляет Вильгардель,
требует  такой  возвышенной  морали,  такой  чистой  добродетели,  что
он  не   решается  предложить  его  в   качестве  практического  плана
преобразования».

Влияния Платона  не избежали  и представители  «демокритовской линии».
Один  из  основоположников   английского  материализма  Фрэнсис  Бэкон
свой  проект идеального  государства  назвал  «Новая Атлантида»,  явно
с  уважением  вспоминая Платона,  которому  и  принадлежит сам  термин
Атлантиды,  и использует  с  модификациями некоторые  идеи Т.Мора.  Мы
видим, что самые разнообразные  проекты государственных реформ так или
иначе  связаны с  Платоном. Как  это не  похоже на  цитированное в  п.
2  мнение  Бернала,  что необходимость  изменений  всегда  выдвигалась
материалистами.

\subsection{Роль Платона в развитии христианства}

Колоссальную роль Платона в развитии человеческой мысли подчеркивает и
Рассел, в  особенности, конечно, в развитии  \textit{христианства} (с.
124): «Платон и  Аристотель были самыми влиятельными  из всех древних,
средневековых  и  современных  философов; из  них  наибольшее  влияние
на  последующие  эпохи  оказал  Платон.  Я  говорю  это  потому,  что,
во-первых,  Аристотель  сам  имеет своим  отправным  пунктом  Платона,
во-вторых, христианская теология и философия, во всяком случае до XIII
века,  были  больше  платоновскими, чем  аристотелевскими.  Поэтому  в
истории философской  мысли необходимо прежде всего  изучать Платона, а
затем  уже Аристотеля,  и изучать  более  полно, чем  кого-либо из  их
предшественников или преемников».

Конечно,  для   многих  антирелигиозных  деятелей   современности,  не
отягощенных   чрезмерными  знаниями,   этого  одного   уже  достаточно
для   осуждения   платонизма:    христианство   ---   опора   правящих
классов,  следовательно,  Платон  является наиболее  тонким  идеологом
эксплуататорских классов. Что уже  давно христианство использовали для
оправдания эксплуатации, в этом не может быть сомнения. Но укажите мне
учение, которое  не использовали для  той же цели:  взять хотя бы  и с
биологией.  И дарвинизм  и  ламаркизм  были использованы  реакционными
кругами и для  оправдания истребления целых народов  и для обоснования
наследственной олигархии. Что  же, мы должны отвергнуть и  то и другое
на этом основании?

А  рассматривая христианство  во  всем его  историческом развитии,  мы
увидим  в нем  не только  «религию  рабов», как  думал Ницше,  который
по  данном  вопросу  находился  в трогательном  единомыслии  с  нашими
антирелигиозниками  (считающими   себя  противниками   Ницше).  Раннее
христианство  до того  времени, когда  оно окостенело  в догматическую
церковь,  было  проникнуто  подлинно революционным  духом  и  истинным
интернационализмом.

Об  этом свидетельствует,  например,  Энгельс  по поводу  «Откровения»
(Апокалипсис) (К истории раннего христианства,  соч. т. XVI, ч. II, с.
419,  цитирую по  Румянцеву, 1937).  На с.  29 Энгельс  указывает, что
раннее христианство еще не осознавало себя самостоятельной религией, а
считало  себя лишь  истинным,  обновленным иудейством:  «Вот как  мало
сознавал еще  наш автор  в 69 г.  христианского летосчисления,  что он
являлся  представителем совершенно  новой фазы  религиозного развития,
которой  предстояло стать  одним  из самых  революционных элементов  в
истории  человеческого духа.  Итак,  мы видим,  что христианство  того
времени, еще  не осознавшее себя, было,  как небо от земли,  далеко от
позднейшей, закрепленной в догматах религии Никейского собора; там оно
совсем  неузнаваемо.  Здесь  же  в  нем нет  ни  догматики,  ни  этики
позднейшего христианства;  зато здесь  есть сознание того,  что борьба
идет со всем миром и что эта борьба увенчается победой».

В  другой  статье  «Бруно   Бауэр  и  раннее  христианство»  Ф.Энгельс
писал:  «Христианство не  знало никаких  вносящих разделение  обрядов,
не  знало даже  жертвоприношений и  процессий классической  древности.
Отрицая,  таким образом,  все  национальные религии  и  общую им  всем
обрядность, обращаясь ко всем  народам без различия, христианство само
становится первой  возможной мировой религией» (см.  Румянцев, с. 25).
Апостол  Павел провозгласил  великий лозунг  интернационализма: «Несть
эллин, ни  иудей, обрезанные  или необрезанные, варвар  и скиф,  раб и
свободный», и если мы спросим, какая мощная общественная организация в
наибольшей степени сохранила этот великий лозунг, то ответ будет один:
христианская,  в первую  очередь,  католическая церковь.  Мужественную
борьбу против папизма в Германии католической и протестантской церквей
засвидетельствовал  безупречный свидетель  --- Эйнштейн  (см. Науманн,
1960).

\subsection{Интернационализм истинного христианства и расизм некоторых
выдающихся атеистов}

Но, говорят нам, христианство  позабыло свои великие заветы равенства,
милосердия  и  интернационализма.  Знамя  этих  великих  идей  перешло
в  другие  руки  совершенно противоположного  направления.  Совершенно
несомненно, что  в числе  защитников великих идей  христианства сейчас
много  представителей «линии  Демокрита». Также  несомненно и  то, что
большинство  официальных  церковников  предали те  идеи,  официальными
защитниками  которых  они  себя  выставляют,  и  это  предательство  и
вызвало идейный  пафос современных атеистов. На  вопросы (составляющие
заглавия  соответствующих брошюр)  Луначарского «Почему  нельзя верить
в  бога»  или  Б.Рассела  «Почему  я  не  христианин»  отвечу  словами
глубоко  религиозного Владимира  Соловьева: «Бесчеловечный  бог создал
безбожного человека».  Предательство христианства  полностью объясняет
то положение, что  сейчас многие гуманные и  просвещенные люди считают
атеизм не просто допустимым, но и этически обязательным. Такая позиция
имела сильную  поддержку, если  бы можно  было установить,  что атеизм
тесно  связан с  гуманностью, свободолюбием  и интернационализмом.  Но
такой связи нет. Шагнем мысленно на сто лет назад, когда в Соединенных
Штатах  шла борьба  за  освобождение  негров. Вооруженной  гражданской
войне  предшествовала и  сопутствовала длительная  идейная борьба,  не
закончившаяся,  как  всем  хорошо  известно,  и  по  настоящее  время.
Очень  крупную, если  не ведущую  роль  в этой  идейной борьбе  играла
христианская  секта  квакеров,  о которой  так  трогательно  отзывался
наш  великий  гуманист Лесков.  Была  ли  квакершей Бичер-Стоу,  автор
знаменитого  романа  «Хижина  дяди  Тома», не  знаю,  но  весь  роман,
сыгравший  большую  роль  в  этой идейной  борьбе,  проникнут  истинно
христианской тенденцией.

И  вот  в  этот  момент  у нас  в  России  имелось  направление  среди
радикальной интеллигенции  в лице  одного из  «классиков революционной
мысли домарксистского  периода» Зайцева,  которое по этому  же вопросу
высказывалось  так (Зайцев,  с.  228): «Но  из  европейских ученых  не
найдется ни одного,  который бы не считал цветные  племена стоящими по
самым  условиям своего  организма  ниже белых.  Несомненно и  признано
всеми,  что невольничество  есть  самый лучший  исход, которого  может
желать цветной человек, придя в  соприкосновение с белою расою, потому
что  он достается  в удел  только наиболее  развитым и  сильным расам;
большая же  часть их  не может вовсе  существовать рядом  с кавказским
племенем и вскоре совершенно вымирает.  Ошибочно было бы винить в этом
европейцев». Само собой разумеется,  он издевается над «чувствительной
барыней» Бичер-Стоу.

Взгляды   Зайцева  по   этому  вопросу   были  решительно   поддержаны
сотрудничавшим с  ним Д.Писаревым, одним из  кумиров нашей радикальной
интеллигенции. На  с. 444 своей  статьи «Посмотрим» (Писарев,  соч. т.
III) он пишет: «Г.Зайцев высказал  вовсе не эксцентрическую мысль, что
закон  Дарвина  прилагается также  и  к  человеческим расам.  Если  г.
Антонович думает, что к человечеству  этот закон не прилагается, то г.
Антонович должен  объяснить, на чем он  основывает свое предположение.
Спрашивается, какое  свойство или  какая сила  человеческого организма
обусловливает собою это изъятие  из общего закона, распространяющегося
на  весь органический  мир? Все  известные исторические  факты говорят
самым красноречивым  образом в  пользу мнения  г. Зайцева.  Белая раса
везде  и всегда  играла роль  желтого таракана  и пасюка:  португальцы
истребили  гуанхов,  жителей  Канарских  островов;  испанцы  истребили
краснокожих обитателей Вест-Индии;  англичане истребили или поработили
индусов, австралийцев,  новозеландцев и  северо-американских индейцев;
русские  истребили   алеутов  и  многое  множество   разных  сибирских
инородцев. Всякий  желающий может  проливать потоки слез  над могилами
этих истребленных разновидностей, но называть человека лжереалистом за
то,  что  он  спокойно  констатирует  существующий  факт,  ---  значит
превращать науку  в ребяческое  и притворное  прославление либеральных
симпатий  и   сентиментальных  иллюзий.  Если  г.   Антонович  гонится
не  за  истиной,  а  за  утешительностью,  то  ему  следует  сделаться
не  только  идеалистом,  и  даже  супернатуралистом».  Не  правда  ли,
весьма  «прогрессивные» мысли?  И Писарев  прав в  том отношении,  что
сохранение  «сентиментальных  иллюзий»  по   отношению  к  иным  расам
в  значительной  степени  было   связано  с  «супернатурализмом».  Как
известно, материалистический  социал-дарвинизм открыл прямую  дорогу к
расизму  и нацизму  (здесь связь  куда  более ясная,  чем в  отношении
Платона, см. п.  3), а сподвижник Дарвина  по обоснованию эволюционной
теории Уоллес,  резко выступавший против расизма,  признал необходимым
принять   «супернатуралистические   факторы»  в   деле   происхождения
человека.  Это  было  резко   осуждено  нашим  профессором  Мензбиром,
который «заредактировал» соответствующие места в переводе «Дарвинизма»
Уоллеса, как  «явно ненаучные». Но,  может быть, сам  Мензбир оказался
свободным от  всякого расизма?  Нет, на с.  73 его  интересной брошюры
«Тайна  Великого  океана» (1922)  мы  читаем  такие слова:  «Так  дело
шло,  пока египетская  культура не  подпала под  гибельное для  всякой
государственности  семитическое влияние,  после чего  стала падать  и,
наконец, разрушилась совершенно». Недурно?

\subsection{Революционный характер в христианстве XIX века}

Но  если материализм  не  спасает от  самых  реакционных взглядов,  то
зато,  может  быть,   революционный  характер  христианства  относится
только  к  раннему  его  периоду,   а  после  образования  церквей  он
окончательно  выдохся?   Это  может  считаться  справедливым   лишь  в
той  мере,  в  какой,  пожалуй,  все  христианские  церкви  противятся
революционным изменениям. Но наряду  с ортодоксами всегда существовали
еретики,  раскольники,  сектанты  и,   как  правило,  появление  новой
ереси  ставило  своей  задачей  «ревизию»  господствовавшей  церкви  и
возрождение  первоначального христианства  с его  революционным духом,
унаследованным им от  предшественника, иудаизма. Недаром протестантизм
наш  выдающийся ученый  Зелинский (в  книге «Соперники  христианства»)
назвал  реставрацией  иудаизма  на  христианской  почве.  Два  великих
начала  ---  «справедливость»  и  «милосердие»  до  известной  степени
антагонистичны. Если мы перегибаем  в сторону чрезмерного всепрощения,
утрачивается  справедливость,  стремление   к  справедливости  нередко
заставляет позабывать о милосердии:  милосердный Иисус отступает перед
справедливым, но беспощадным  Иеговой. Ветхий Завет восстанавливается,
как  руководство  к  действию.  Опираясь  на  Ветхий  Завет,  Кромвель
требовал казни  Карла I (прецедент  с царем Агатом); он  истребил всех
жителей Дрогелы, опираясь на прецедент с Иерихоном.

Как   идеология   революционного  движения,   христианство   сохранило
свою  силу   вплоть  до  середины   XIX  века.  В  1850--1864   гг.  в
Китае   имела   место   Великая  крестьянская   война,   приведшая   к
образованию Тайпинского  государства (см. Кара-Мурза,  1957). Движение
тайпинов  было  ярко  выраженным религиозным  движением,  в  идеологии
которого  кроме  христианства  видное место  занимают  древнекитайские
философско-религиозные движения, но лидеры движения были сознательными
христианами,  членами особой  своеобразной  секты. Организатором  этой
секты был сельский учитель Хун Сю-Цуань, который юношей познакомился с
христианским  учением  и  стал  проповедником  христианства.  Сам  Хун
Сю-Цуань стал религиозным главой движения, а сын угольщика Ян Сю-Цинь,
тоже член  новой секты, стал  политическим и военным  вождем движения.
Конечно, новая секта имела мало общего с европейским христианством, но
она восстановила  многие черты первичного  христианства: уравнительный
принцип   в  землепользовании,   коммунистическая  организация.   Было
ликвидировано рабство и продажа женщин и девушек в наложницы. Отменено
приданое  при вступлении  девушек  в брак,  уничтожен обычай  убийства
и  подкидывания  девочек,  калечение ноги  девочек  путем  бинтования.
Женщины были вовлечены в  общественную жизнь. Тайпины обращали большое
внимание  на образование,  были  созданы  народные школы,  подчиненные
христианскому  воспитанию.   Было  поднято   здравоохранение,  введена
прививка  оспы.  Суд был  реформирован,  отменены  пытки и  варварские
способы смертной казни. Смертная казнь  была сохранена, и не только за
тяжкие преступления,  но и за  преступления против аскетизма.  В армии
тайпинов было декретировано строгое безбрачие и недопустимость связи с
женщиной до тех пор, пока  «небесное государство не распространит свою
власть на  весь Китай».  Нарушение этого закона  каралось смертью  и в
целях  предупреждения женщины  были  выделены в  особый городок.  Было
запрещено курение опиума, табака, пьянство и азартные игры. Само собой
разумеется, солдаты  долго не  могли выдерживать  подобного аскетизма.
Дело дошло  до серьезных беспорядков  и в конце концов  безбрачие было
ликвидировано  с  возвращением  старых  китайских  обычаев.  У  высших
командиров  были разрешены  даже  гаремы,  впрочем ---  нормированные,
не  более  одиннадцати   жен  у  высших  князей.   Как  и  свойственно
всякому  новому  религиозному  движению, оно  перерождалось  в  слепой
фанатизм, приводивший к многим  жестокостям и нетерпимости: поголовное
истребление  мандаринов   и  маньчжуров  с  их   семьями,  истребление
жрецов буддизма  и даосизма,  уничтожение буддийских и  даосских книг.
Эти  ненужные  эксцессы  создавали   им  репутацию  жестоких  варваров
и  оправдывали  интервенцию  в  глазах  многих  людей.  Но  эти  черты
свойственны всем крупным революциям, независимо от того, происходят ли
они под религиозным знаменем или нет.

Тайпины  владели  значительной  частью Китая  довольно  долгое  время,
создали  новую династию,  но  в  конце концов  погибли  не столько  от
интервенции, сколько от внутреннего  разложения. Тайпинские вожди сами
превратились в феодалов. Повторилось в  ином варианте то, что уже было
раньше с двумя удачными  крестьянскими революциями в Китае, приведшими
к созданию  Ханьской и Минской империй.  Но как будто во  всей истории
человечества  не было  случая,  когда  крестьянская удачная  революция
привела бы  к новому устойчивому строю.  Для нас сейчас важно  то, что
если в чем можно упрекнуть христианство, так конечно не в том, что оно
обязательно  приводит к  смирению  и покорности.  Оно  много раз  было
источником  чрезвычайно  бурных  революционных  движений,  которые  не
приводили к  ожидаемому результату просто потому,  что намеченная цель
была неосуществима.  Но выставление как близко  достижимых по существу
неосуществимых  целей  свойственно  всем  крупным  революциям:  оно  и
является источником энтузиазма революционеров.

\subsection{Пристрастность Дж.Бернала и Б.Рассела}

Нам еще не раз придется  коснуться влияния платонизма во всех областях
культуры. Ограничусь пока еще  одним высказыванием Б.Рассела (1959, с.
140):  «Но  Платон  оказал  влияние не  только  на  философию.  Почему
пуритане  возражали  против  музыки   и  живописи  и  пышного  ритуала
католической  церкви?  Вы  найдете  ответ   на  это  в  книге  десятой
``Государства''  Платона.  Почему  заставляете  детей  учить  в  школе
арифметику? Обоснование этому сделано  в книге седьмой ``Государства''
Платона».  Какой  же  фанатической нетерпимостью  к  инакомыслию  надо
обладать,  чтобы  написать следующие  слова  (Бернал,  1956, с.  112):
«Маркс был очень  уж добр к философии или, возможно,  он думал о своем
прежнем  любимце  Эпикуре,  когда сказал:  ``Философы  лишь  различным
образом  объясняли мир,  но  дело заключается  в  том, чтобы  изменить
его''.  Задача  же,  которой совершенно  сознательно  занялся  Платон,
заключалась в том, чтобы помешать  миру измениться, по крайней мере, в
направлении  к демократии».  Бернал,  пожалуй, прав  лишь  в том,  что
оценка  Марксом бездеятельности  философов справедлива  в отношении  к
Эпикуру и,  как увидим дальше, в  отношении большинства материалистов.
Философия  же  Платона,  как  ясно  уже  из  вышеизложенного,  оказала
огромное влияние на изменение мира.

Отрицательная оценка  политического значения Платона  Б.Расселом имеет
совершенно иной смысл.  Конечно, и он обвиняет  Платона как противника
демократии, но по мнению Рассела (о чем нам позднее придется подробнее
поговорить),  демократия  в  современной Европе  уже  осуществлена  и,
видимо,  свою  родину,  Англию, он  считает  совершенной  демократией,
забывая то, что большинство земли в Англии принадлежит лендлордам, что
там имеется  наследственная палата  лордов, имеются и  другие солидные
пережитки  феодализма.  В   Спарте  его  отталкивают  коммунистические
тенденции, и  его взгляд  на общую  тенденцию в  эволюции правительств
отнюдь не является оптимистическим.  С. 214--215: «С эпохи Возрождения
греческая концепция правительств,  состоящих из культурных благородных
людей,  начинает  постепенно  превалировать  все  больше  и  достигает
расцвета в  XVIII веке. Этому  состоянию дел положили  конец различные
обстоятельства,  прежде всего  демократия, воплощенная  во Французской
революции,  с ее  последствиями.  Культурные  люди, подобные  Периклу,
должны были  защищать свои привилегии  от простонародья и в  ходе этой
борьбы  и  сами  перестали  быть культурными  и  благородными.  Второй
причиной  был  индустриальный  мир  с  его  научной  техникой,  весьма
отличный  от  обычной  традиционной культуры.  Третьей  причиной  было
общедоступное образование, которое давало возможность читать и писать,
но  не  давало культуры,  что  позволило  новому типу  демагога  вести
пропаганду по-новому, как это проявилось в диктатурах. Так, --- хорошо
это или плохо, --- дни власти культурных людей миновали». Рассел вовсе
не  противник того  идеала,  который ставил  перед  собою Платон.  Его
отталкивает  в Платоне  недостаток  свободы  и наличие  тоталитаризма.
Тогда  становится понятной  эволюция философских  взглядов Рассела.  В
прошлом, в начале XX века (он  родился в 1872 году) он был сторонником
платонизма.   Его   привлекали  общетеоретические   взгляды   Платона,
политические  же  его воззрения  он  считал  совершенно устаревшими  и
потому  безвредными. Но  когда многое  из того,  о чем  мечтал Платон,
стало  осуществляться  (хотя  и  не под  платоновским  знаменем),  это
оттолкнуло Рассела  от Платона  и, в  конце концов,  убеждения чувства
взяли в этом вопросе верх над убеждениями разума.

\subsection{Три  точки зрения  на Платона:  центр эллинской  культуры,
волшебник слова и простой приказчик класса-эксплуататора}

Этот  беглый  предварительный  обзор показывает,  как  много  спорного
имеется  в  суждениях о  значении  платоновского  наследства. Но  есть
один  пункт,  где  как  будто  споров  не  возникает:  это  красота  и
умственная  мощь  творений  великого   эллина.  Бернал  полагает,  что
неудача политических стремлений привела Платона к стремлению посвятить
свою  жизнь философии.  С. 113:  «Это  привело его  на путь  идеализма
в  философии,   и  действительно  он  навсегда   стал  величайшим  его
представителем. Ибо,  хотя Платон, конечно, не  был первым идеалистом,
он смог  изложить свои  взгляды в  форме диалогов  с такой  красотой и
убедительностью,  которая никогда  не была  превзойдена в  философских
сочинениях.  На  самом  деле красота  изложения  мешала  последовавшим
поколениям людей увидеть уродливость выраженных в них идей».

Так  же высоко  в смысле  одаренности оценивает  Платона и  Рассел. Он
считает возможным  усомниться в образе Сократа,  созданном Платоном, в
силу исключительной  одаренности Платона.  С. 103:  «Именно выдающееся
мастерство  Платона  как  писателя  вызывает  сомнение  в  нем  как  в
историке.  Его   Сократ  является  последовательным   и  исключительно
интересным характером, какого не смогло бы выдумать большинство людей,
но  я считаю,  что  Платон мог  бы выдумать  его».  Отчего не  сделать
следующий шаг? Сократа вовсе не  существовало, ведь о нем не упоминает
ничего Фукидид.

Наконец,  приведем мнение  одного  советского ученого,  с которым  нам
придется  дальше часто  встречаться ---  С. Я.  Лурье. В  конце своего
разбора значения Платона  он пишет (1947, с.  346--347): «Итак, Платон
по  существу  идеолог  своего  класса, а  никак  не  ``беспристрастный
философ,  стоявший выше  классовой  борьбы''. Научных  открытий он  не
сделал, а  ``наукообразное'' обоснование  его метафизики  построено на
логических ошибках.  Тем не менее  он сыграл громадную роль  в истории
всей  позднейшей  идеологической  и  религиозной  философии.  Решающее
значение при этом имели, конечно, его реакционные политические взгляды
и  построенная  им   стройная  идеалистическая  мистически-религиозная
система; но  не менее  важно было  и соединение  глубокого внутреннего
чувства  с высокохудожественной  формой изложения  и с  аргументацией,
производящей на читателя впечатление строгой логичности».

Что  платонизм  был связан  с  разнообразнейшими  явлениями в  области
культуры,  можно   считать  бесспорным:  во  всех   областях  науки  и
искусства платоники проделали огромную  работу. Но эта связь толкуется
по-разному, и можно выделить три главных мнения.

1) Идеалисты делают то заключение, что здесь мы имеем органичную связь
между  мировоззрением  и  методологией.  Платонисты  достигли  крупных
успехов именно потому, что они стояли на почве объективного идеализма.

2)  Большинство  современных  ученых   полагает,  что  здесь  не  было
органической  связи между  мировоззрением и  методологией. Выразителем
этого  мнения  является,  например,  Бернал.  Он  отмечает  (1956,  с.
104), что  в пифагорейском  учении сочетаются  две тенденции  идей ---
математическая  и  мистическая.  Первую  тенденцию  Бернал  безусловно
одобряет, вторую же считает  вредным привеском, который никакой пользы
в развитии  математической тенденции не играл.  Иллюзия плодотворности
этого  мировоззрения  возникла  благодаря  исключительной  одаренности
Платона.

3)  Но  есть третье  мнение,  упорно  защищаемое  в ряде  работ  нашим
современником,  С.Я.Лурье,   уже  отчасти  выраженное   в  приведенной
выше  цитате.  Платон не  принес  никакой  пользы науке,  все  ценное,
приписываемое Платону или его школе, заимствовано им от его противника
Демокрита, работы которого он уничтожил  и добился того, что они вовсе
не сохранились. Не Платон является центром великой эллинской культуры,
а Демокрит.

Сейчас,  через две  тысячи лет  после смерти  Платона, мы  стоим перед
вопросом, кто же он был: центр эллинской культуры, волшебник слова или
просто  приказчик  класса-эксплуататора.  С.Я.Лурье открыто  стоит  на
марксистской точке зрения и даже его можно назвать ультрамарксистом. В
начале  статьи «Механика  Демокрита» он  указывает на  большой интерес
истории  античной  механики  не  только  для  истины  науки,  но  и  с
социологической  точки зрения  (1935, с.  129): «Многим  кажется, что,
если,  вообще говоря,  можно  с полным  правом  говорить о  буржуазной
историографии,  буржуазной  политической   экономики  и  социологии  и
т.д.,  то  перенесение такой  оценки  на  математику и  механику  было
бы  непозволительной  натяжкой.  Причина  заключается  в  том,  что  и
математика,  и механика  в наше  время оперируют  огромным количеством
бесспорных фактов, одинаково обязательных для всех».

«Но с точки зрения исторического материализма естественно ожидать, что
классовая точка зрения должна сказываться и в математике и в механике.
Особенно ярко это должно быть заметно для тех эпох, когда математика и
механика  еще  не  располагали столь  большим  количеством  бесспорных
фактов,  когда, в  частности, механика  еще строилась  на ряде  весьма
произвольных гипотез».  Лурье и указывает,  что в античной  науке были
борющиеся  между  собой  математические  и  механические  системы,  из
которых  одни имели  базой идеологию  землевладельческой аристократии,
другие --- городской демократии.

Этот взгляд Лурье вряд ли будет сейчас защищаться даже ортодоксальными
марксистами, но нельзя отрицать  принципиальную допустимость того, что
он для прежних эпох был правилен.

\subsection{Широкое понимание «Линии Платона»  С.Я.Лурье в связи с его
обвинениями}

Взгляды С.Я.Лурье, несмотря на их парадоксальность, заслуживают самого
тщательного  разбора.  Во-первых,  трудно  найти  более  компетентного
защитника  линии  Демокрита.  Ведь  мы  знаем,  что  творения  Платона
сохранились  полностью, Аристотеля  ---  в  очень значительной  части,
творения же  Демокрита в  оригинале вовсе не  сохранились. И  мы знаем
также, что С.Я.Лурье потратил много лет упорного труда на разыскивание
еще не известных фрагментов Демокрита у самых разнообразных авторов.

Во-вторых,  С.Я.Лурье  не  страдает той  узостью,  которой  отличается
большинство  ученых.  Он  совмещает  в  себе  компетентного  филолога,
историка,  математика и  социолога.  Эта  многогранность его  научного
облика ярко  выступает, например, в его  превосходной книге «Архимед»,
где все  стороны деятельности великого ученого  освещены равномерно на
фоне  общей  трагедии  эллинистического  мира,  раздавленного  римским
солдатским сапогом.

В-третьих, привлекает  внимание оригинальность и  крайность занимаемой
им позиции. Так  как свои воззрения он связывает с  марксизмом, то это
усиливает их интерес, так как столь детального марксистского освещения
истории греческой культуры как будто не существует.

В-четвертых, не вызывает ни  малейшего сомнения независимость мышления
С.Я.Лурье и  его добросовестность  как ученого.  А это  в связи  с его
долголетней работой приводит к тому,  что в его сочинениях можно найти
много материала и для более полной оценки значения линии Платона.

Все  это приводит  к тому,  что тщательный  разбор взглядов  С.Я.Лурье
является  лучшим средством  для  правильной оценки  значения тех  двух
линий, о которых идет речь.

Замечу  тут  же,  что  крупным недостатком  воззрений  Лурье  является
то,  что он  «Линию Платона»  понимает слишком  широко. Если  касаться
античной  культуры и  Средних  веков,  то можно  в  том широком  русле
идей,  которые  Лурье  понимает   под  линией  Платона,  различить  по
крайней  мере  три  потока:  а)  собственно  линия  Платона:  Пифагор,
Сократ, Платон, Академия,  неоплатонизм; б) оппозиционное ответвление:
Аристотель и перипатетики; в) христианские богословы, которые наряду с
близостью платоновской  и перипатетической линиям имели  и собственную
линию  развития: много  различных  направлений  имели и  средневековые
схоластики. Все  три потока  охватывают только  настоящий, объективный
идеализм, субъективный идеализм  я оставляю без рассмотрения,  да он в
античной культуре и не играл существенной роли.

Что  касается  линии  Демокрита,   материализма,  то  она  в  основном
представлена следующими школами и именами: милетская школа, Анаксагор,
Левкипп, Демокрит, Эпикур, Лукреций.

Многие рассматривают в основном как материалистов таких философов, как
Эмпедокл, Гераклит, Протагор. Несомненно, что  у них можно найти много
точек соприкосновения  с материалистами, но назвать  их материалистами
без оговорок вряд ли возможно.

\textit{Задача  моей  работы}  и   заключается  в  следующих  основных
разделах:  1)  разобрать  то обвинение,  которое  выдвигает  С.Я.Лурье
(повторяя  некоторых старых  авторов,  но усиливая  это обвинение)  по
сознательному уничтожению  и замалчиванию «Линией Платона»  всего, что
связано с Демокритом; 2)  сравнить достижения различных областей науки
по  обеим линиям  и установить  связь этих  достижений с  философскими
взглядами  ученых;  3)  разобрать  обвинения  Платона  в  политической
реакционности  и  классовом  характере его  философии;  4)  определить
истинное значение идеализма и  материализма в возникновении и развитии
культуры.

\clearpage

\section{РАЗБОР ОБВИНЕНИЙ ЛИНИИ ПЛАТОНА}

\subsection{Обвинение Платона во вредительстве и плагиате}

Сформулируем  обвинение, выдвинутое  С.Я.Лурье,  в злостном  характере
деятельности  Платона  и  его   последователей  против  своих  идейных
противников. В предисловии  к своей книге «Очерки  по истории античной
науки» С.Я.Лурье  пишет (1947,  с. 6--7):  «Своей работой  я продолжаю
и  развиваю  тот подход  к  активной  науке  вообще  и к  Демокриту  в
частности,  который был  характерен для  ученых эпохи  Возрождения ---
Бруно,  Бэкона,  Галилея и  др.,  а  затем  был надолго  заброшен  для
того, чтобы  возродиться в чрезвычайно поучительной,  хотя и несколько
фантастической и гиперболической работе Левенгейма... Вслед за Бэконом
и  Галилеем, вслед  за Левенгеймом  я стремлюсь  показать, что  Платон
и  Аристотель  только  по  недоразумению  попали  в  число  творческих
деятелей  античного  естествознания.  Как  я показал  в  своей  статье
«Платон  и Аристотель  о точных  науках...», они  себе такой  задачи и
не  ставили; кропотливую  работу естествоиспытателя  Аристотель считал
делом, достойным раба,  а не свободного человека.  Платон и Аристотель
заимствовали из античного  естествознания различные положения, которые
им казались  полезными для  построения их  идеалистических философских
систем. При этом античная литературная этика вовсе не требовала, чтобы
при заимствовании чужой мысли указывался  ее автор; что же касается, в
частности, Демокрита, сочинения которого, как говорили, Платон скупал,
где  только  мог,  и  сжигал,  то его  сплошь  и  рядом  умышленно  не
упоминали, чтобы  в потомстве  не сохранилось  памяти об  этом вредном
материалисте.  Уже древние  обращали  внимание на  то,  что у  Платона
встречаются учения Демокрита, но ни разу не упоминается его имя».

Обвинение  в  скупке  и  сожжении сочинений  Демокрита  повторяется  и
в  книге  «Архимед» (Лурье,  1945,  с.  22),  где упоминается,  что  в
результате энергичной  деятельности Платона и  Аристотеля произведения
Демокрита  стали редкими  и малодоступными.  На  с. 107  той же  книги
говорится о прямом плагиате: «Демокрит был запрещен; в идеалистической
науке было правилом, следуя Платону,  не упоминать его имени даже там,
где это  было нужно, а между  тем почти для каждого  нового открытия в
области  естествознания и  математики приходилось  непосредственно или
через третьи руки обращаться к Демокриту. Особенно часто заимствуют ту
или иную часть учения Демокрита  платоники и пифагорейцы, и кто знает,
сколько еще великих открытий Демокрита стало известно нам в искаженном
виде и под чужим именем».

Идеалистические философы считали,  что неупоминание Платоном Демокрита
указывало на  более позднее  появление сочинений  Демокрита. С.Я.Лурье
это решительно оспаривает и в прим.  17, с. 362, (1947) пишет: «Теперь
уже доказано, что Платон хорошо  знал учение Демокрита, но не упоминал
Демокрита, чтобы не создать популярности заклятому врагу».

С.Я.Лурье,  конечно, знает,  что злостное  замалчивание Демокрита  как
лучшего из когда-либо живших  философов, по мнению Аристоксена (Лурье,
1947, с. 337), оспаривается: «Нынешние поклонники Платона видят в этом
сообщении  злостную клевету  аристотелевской школы,  но с  этим нельзя
согласиться». Лурье  указывает, что Платон в  «Законах» предлагает для
борьбы с  приверженцами материалистических  учений куда  более суровые
меры (казнь, бичевание, тюрьма, лишение гражданских прав, конфискация,
изгнание), чем  сжигание книг  и, следовательно, обвинение  в сжигании
книг и  умалчивании становится правдоподобным.  Аргументация С.Я.Лурье
неубедительна.  «Законы»   разберем  своевременно,  но   даже  крайняя
жестокость по  отношению к  противникам не  обязательно сопровождается
бесчестностью,  а ведь  Лурье  обвиняет Платона  в бесчестных  приемах
борьбы. Кроме  того, клевета на Платона  приписывается аристотелевской
школе, но ведь  Аристотель, по Лурье, относится к той  же линии и, как
ясно из предыдущей цитаты из «Архимеда», Аристотель также обвиняется в
замалчивании Демокрита.  Аристотель и  перипатетики ---  соучастники в
преступлении, они не могли быть авторами обвинительного ложного акта.

Наконец, чтобы  подчеркнуть «единство линии Платона»,  Лурье сообщает,
что  христианские  богословы,  исходя  из  известного  им  факта,  что
божественный  Платон  сжигал  сочинения Демокрита,  считали  Демокрита
безнравственным человеком (Лурье, 1947, с. 273).

Мы   можем,   таким  образом,   на   основании   этих  текстов   точно
сформулировать  обвинение  против Платона  и  всей  его линии:  1)  он
сознательно  уничтожал творения  Демокрита; 2)  не упоминал  его имени
тоже со злостной целью; 3)  заимствовал ценные открытия без упоминания
имени; 4) вся  эта политика проводилась всей линией  Платона в широком
ее  понимании  (см. 1.10);  5)  в  результате этого  исчезли  творения
Демокрита; 6)  получилось совершенно  искаженное представление  о роли
двух  противников, и  на  законное место  центральной фигуры  античной
науки,  Демокрита,  стали  узурпаторы   Платон  и  Аристотель;  7)  во
времена  Возрождения  Галилей,  Бруно и  Бэкон  пытались  восстановить
справедливость,  но,   несмотря  на   огромный  авторитет   этих  лиц,
идеалистической философии  удалось взять  реванш вплоть  до последнего
времени. Разберем эти обвинения.

\subsection{Необоснованность  обвинения Платона  в сожжении  сочинений
Демокрита}

Начнем с легенды  о скупке, сжигании книг Демокрита  и об умалчивании.
С.  Я.   Лурье  говорит  везде  во   множественном  числе:  «древние»,
«говорили», но  приводит только одно указание  из античной литературы,
именно Диогена Лаэрция IX, 40 (см. Лурье, 1947, с. 383, прим. 31а и с.
386, прим. 6).  В личной беседе, когда я  спросил Соломона Яковлевича,
имеются ли  какие другие источники для  этой легенды, я на  это ответа
не  получил.  И  в  других руководствах  по  философским  вопросам  мы
наталкиваемся только на эту цитату.  Видимо, никаких других указаний и
не существует.

Сообщение Диогена Лаэрция упоминается  в историях философии (например,
у  Ибервег-Гейнце,  Виндельбанда),  но   ему  не  придается  значения.
Виндельбанд  (1911,  с.  159)  просто называет  сообщение  о  сожжении
плоской  историей,  останавливаясь только  на  вопросе  о том,  почему
Платон не  упоминает Демокрита. Но  раз этому придается  такое большое
значение, то необходимо привести соответствующее место Диогена Лаэрция
полностью (в переводе с немецкого издания): «Аристоксен рассказывает в
исторических достопримечательностях,  что Платон имел  намерение сжечь
все сочинения  Демокрита, которые  бы он  мог собрать,  но пифагорейцы
Амикл  и  Клейниас  отговорили  его  от  этого,  как  от  бесполезного
предприятия, так как книги уже  широко распространены. Но одно ясно: в
то время как Платон упоминает почти всех прежних философов, он никогда
не делает этого  по отношению к Демокриту, даже там,  где он выдвигает
против  него какое-либо  возражение, потому  что он  сознавал, что  он
имеет дело  с лучшим из  философов, которого хвалит также  Тимон» (как
первого философа). Вот и все.

Выходит:  1) Платон  не  покупал  и не  сжигал,  а только  намеревался
сжигать;  2) отговорили  его его  же  друзья пифагорейцы;  3) так  как
Диоген  сообщает со  слов Аристоксена,  то, очевидно,  имеется не  два
свидетеля, Диоген  и Аристоксен,  а только один;  4) сам  Диоген явный
сторонник Демокрита,  поэтому его показания не  могут считаться строго
объективными. Не всякому сообщению о Демокрите можно верить. Сам же С.
Я. Лурье  указывает (1947, с.  135), что некоторые  биографы Демокрита
говорили, что  Демокрит не только понимал  птичий язык, но и  сам умел
говорить на  птичьем языке, он  умел безошибочно предсказать  урожай и
потому мог в любой момент сделаться самым богатым человеком в Абдерах,
но  он не  хотел  этого» (в  другом  месте  мы узнаем,  что  он и  был
одним  из  самых богатых  людей).  Лурье  правильно пишет  дальше:  «К
сожалению, даже  сообщения, в которых  на первый взгляд  не содержится
ничего  неправдоподобного, часто  оказываются  ученой комбинацией  или
произвольным  толкованием  неправильно понятого  случайного  замечания
Демокрита».   Возможно,  что   и   в   основе  истории,   рассказанной
Аристоксеном,  лежит какое-нибудь  изолированное высказывание  Платона
вроде того, которое произнес один  из видных деятелей Возрождения, что
все книги Аристотеля следовало бы сжечь.

\subsection{Популяризация  Демокрита неопифагорейцами  и христианскими
богословами}

Из  слов Аристоксена  у Диогена  Лаэрция вытекает,  что пифагорейцы  в
общем  были бы  не  прочь уничтожить  вредные  творения Демокрита,  но
сознавали,  что  это  невозможно.  Ожидать от  них,  что  они  активно
участвовали  в  сохранении  творений   Демокрита,  казалось  бы,  было
невозможно. Однако  мы знаем (и  это указано  и у С.Я.Лурье,  1947, с.
138),  что  сочинения  Демокрита были  собраны  придворным  астрологом
императора  Тиберия Фрасилом  (или Тразиллом),  пифагорейцем по  своим
философским взглядам. Лурье не упоминает,  что тот же самый Фрасил дал
и то  собрание сочинений  Платона, которое дошло  до нас  в позднейших
списках. Одно и то же лицо, принадлежащее к линии Платона, озаботилось
потратить  немалый труд  по  сохранению  творений обоих  антагонистов,
великолепный  пример  «мирного  сосуществования» двух  направлений  во
время эллинистической эпохи. Как увидим  дальше, это не единичный факт
и  недобросовестных приемов  борьбы в  эту великую  эпоху, по  крайней
мере, у настоящих ученых мы не видим.

Но тот же Лурье С.Я.  приводит факты, показывающие, что высокую оценку
Демокриту  давали не  только материалисты,  но и  многие представители
линии Платона  в широком  смысле этого слова.  В качестве  эпиграфа ко
второй части (о  Демокрите) своей книги (1947, с. 127)  Лурье дает две
цитаты: «И слыша, что  глаголет Демокрит, философ первый...» (Послание
Вассиана, митрополита Ростовского к  Ивану III, Софийская II летопись,
ПСРЛ, VI,  с. 237, 1480).  «Демокрита, дивного философа  и математика,
пропала физика  и математика  от зависти Аристотелеса,  который желал,
чтобы только его книги читали» (А.Фархварсон, Эвклидовы элементы, СПб,
1789, с. 3).

Оказывается,  восторженным  поклонником   материалиста  Демокрита  был
русский  православный  митрополит  в  XV  веке!  Что  касается  мнения
Фархварсона,   то   тут   виновником  исчезновения   работ   Демокрита
оказывается не Платон, а Аристотель, и это обвинение еще более нелепо,
чем в  отношении Платона,  так как: 1)  Аристотель много  раз цитирует
Демокрита и отзывается о нем во многих местах с большой похвалой; 2) в
математике,  насколько  мне  известно,  у  Аристотеля  нет  решительно
никаких  заслуг и  потому ему  нет  смысла завидовать  в этой  области
Демокриту; 3)  он также  жил на несколько  столетий раньше  Фрасила и,
следовательно, никак не мог  быть ответствен за пропажу математических
и физических теорий Демокрита.

Но  как  случилось,  что  православный  митрополит  так  высоко  ценил
безбожного  философа  и  притом   не  скрывал  своей  высокой  оценки?
Да,  высокая оценка  Демокрита была  чрезвычайно распространена  среди
христианских  богословов.  Это  свидетельствует сам  С.Я.Лурье  (1947,
с.  273): «Но,  с  другой стороны,  вся  античная традиция  изображает
Демокрита величайшим мудрецом  древности, исключительно бескорыстным и
справедливым человеком,  пожертвовавшим личной  жизнью ради  науки. Он
--- автор собрания нравственных  изречений, сплошь и рядом чрезвычайно
близких к христианской морали. И, наконец, его атом в ряде характерных
черт  чрезвычайно близок  к пифагорейской  монаде ---  душе, усвоенной
христианской  догмой.  Неудивительно,   что  в  христианской  традиции
(сначала  у гностиков-валентиниан,  положивших  в основу  христианской
теологии  учение  Демокрита и  Эпикура,  а  позже и  у  ортодоксальных
богословов  (Ириней,  Евсевий,   Кирилл  Александрийский,  Августин  и
др.) Демокрит  оказывается предтечей Христа,  предсказавшим пришествие
Христа,  и  его  атом  отождествляется  с  Христом,  ему  влагается  в
уста  целый  ряд  изречений  о  всемогущем  и  всеблагом  боге,  прямо
противоречащих  его основным  предпосылкам». С.  Я. Лурье  считает это
фальсификацией,  начавшейся еще  в  дохристианские времена.  Этических
и  религиозных  взглядов  Демокрита  нам придется  коснуться  в  своем
месте, здесь  же для  нас важно,  что важная  ветвь линии  Платона ---
христианское богословие не только  не пыталось замолчать Демокрита, но
популяризовало его  и брало его философию  на вооружение христианского
богословия. И сам С.Я.Лурье, как известно, проделал огромную работу по
изучению  святоотеческой  литературы,  откуда он  извлек  очень  много
неизвестных фрагментов Демокрита.

Этика   Демокрита   носила   несомненно   аскетический   характер,   и
неудивительно, что христианские монахи создавали в отношении Демокрита
легенды  в чисто  христианском духе:  например, что  он ослепил  себя,
чтобы избежать  соблазна от лицезрения  женщин (С. Я. Лурье,  1947, с.
138, по Тертуллиану).

\subsection{Объективность Платона к противникам (Протагор, Аристофан)}

Но  если линия  Платона никак  не может  быть обвинена  в замалчивании
Демокрита, то почему о нем не  упоминает Платон? На этот вопрос нельзя
дать такой же исчерпывающий ответ, как на совершенно нелепое обвинение
линии  Платона,  но  кое-какие  доводы  привести  можно.  Неупоминание
неприятных авторов  широко практиковалось у нас  в недобрую сталинскую
эпоху, и  нельзя сказать,  чтобы оно полностью  было изжито  и сейчас.
Тогда же  оно проводилось систематически: бывали  случаи, когда статья
печаталась  вовсе без  имени  автора только  с упоминанием  редактора.
Естественно,  что подобный  «порядок» отбрасывался  гипотетически и  в
прошлое.  Придерживался ли  Платон  такого казуса?  Конечно нет.  Ведь
одними  из главных  противников  он, следуя  своему учителю,  Сократу,
считал  софистов,  а именно  Платон  в  наибольшей степени  увековечил
их  идеи.  Гиппию  посвящены  два  диалога,  Протагору  ---  один,  по
одному  ---  представителям  другой   враждебной  школы  ---  элеатам,
Пармениду  и  Горгию.  Имена  других софистов  рассыпаны  в  творениях
Платона,  и  мы знаем,  что  учения  многих софистов  сохранились  для
нас  как  раз в  диалогах  Платона.  Но,  может быть,  Платон  излагал
учение  своих  противников  в  карикатурном  виде  или  искажал  их...
Этого сказать  никак нельзя.  В «Пармениде» приведена  довольно трудно
изложенная,  но весьма  последовательная критика  учения об  идеях ---
великолепный  образец  самокритики.  Многим специалистам  эта  критика
кажется  столь убедительной,  что они  отказываются верить,  чтобы она
могла  быть написана  самим Платоном.  Эти «специалисты»  не понимают,
что  подлинная диалектика,  т.е.  возможно  полное освещение  предмета
со  всех сторон  как  раз  характерна для  Платона.  Или возьмем  один
из  его шедевров:  «Протагор».  Первая часть  ---  речь Протагора,  по
художественной форме одно из лучших  мест Платона. Дальше спор Сократа
с Протагором, не идущий в сравнение  в смысле формы с речью Протагора,
весьма запутанный  и тягостный. А результат  диалога? После длительных
разборов  деталей оказывается,  что Сократ  защищает ту  точку зрения,
которую первоначально защищал Протагор (что искусству управления можно
научиться),  а  Протагор встал  на  точку  зрения Сократа.  Весь  спор
остается  нерешенным. Такой  конец, как  известно, скорее  типичен для
Платона.

Наконец,  коснемся одного  из самых  ожесточенных противников  Сократа
---  драматурга Аристофана.  Хорошо  известен  тот «портрет»  Сократа,
который был  дан Аристофаном в  его комедии «Облака».  С.Я.Лурье прямо
характеризует это как совершенно  искаженную комедийную пародию (1947,
с. 317). Многое, о чем пишет  Аристофан в «Облаках», было повторено на
процессе Сократа: что  он не верит в богов, развращает  юношество и т.
д.  Аристофана  можно  было  бы  обвинить  как  одного  из  виновников
гибели  Сократа.  И  в  «Апологии  Сократа»  обвинение  Аристофана  (с
упоминанием имени  драматурга) приведено (Апология, 19,  С). «Апология
Сократа»  --- один  из  неоспариваемых никем  шедевров  Платона, но  в
смысле художественной формы,  вероятно, большинство отдадут первенство
его  «Пиру» или  «Симпозиуму» (откуда  и пошло  слово «симпозиум»  для
обозначения  коллективных обсуждений  одного  вопроса  с разных  точек
зрения). И  вот, к нашему удивлению,  в «Пире» мы видим  Аристофана не
только  в кругу  друзей  Сократа, но  произносящим  одну из  блестящих
речей. Лично мне речь Аристофана в «Пире» нравится гораздо больше, чем
его  комедии. Вот  уже  тут  поистине загадка:  какой  смысл было  так
возвеличивать своего врага. Это вызывает недоумение и у комментаторов.
В примечании  33 к  французскому изданию  Платона Робен  пишет: «Можно
делать только предположения, какие основания заставили Платона сделать
Аристофана, обвиняемого в осуждении его  учителя (ссылки на Апологию и
Федона), соучастником Сократа на ужине у Агафона. Трудно даже сказать,
не скрывает  ли дружественная взаимная вежливость  какого-то глубокого
враждебного намерения,  поддерживаемого рядом  черт и  не исключающего
восхищения гением великого комика». Поистине страшная месть!

Конечно,  можно  выдвинуть  новую  гипотезу  или,  вернее,  даже  две:
(кажется, никто не выдвигал, я  делаю заявку), руководясь той «научной
методикой»,  которой  так  широко пользуются  многие  специалисты:  1)
если  устанавливается явное  противоречие  взглядов у  одного и  того,
же  автора, то  работы,  содержащие  противоречие, принадлежат  разным
авторам (почему раздраконили Гомера на несколько независимых авторов);
2)  если  об  одном  лице  имеются  противоречивые  указания  или  нет
упоминания в  важном источнике,  то это  лицо вообще  не существовало.
Руководствуясь  первым  положением,  следует  признать  «Пир»  Платона
подделкой; на  «Пир» как отдельное  произведение, кажется, до  сих пор
никто не посягал, но эта  гипотеза поддерживает взгляд Н.А.Морозова (в
его  сочинении «Христос»),  что вся  античная литература  --- подделка
времен Возрождения. Руководствуясь  вторым положением, можно выставить
гипотезу,  что  Сократа  вообще   не  существовало  (как  известно,  в
«Истории» его  современника Фукидида этот «мудрейший  из греков» вовсе
не  упоминается) и  что  он целиком  выдуман  Аристофаном и  Платоном.
Высказывая эти две гипотезы, я отнюдь не думаю побить рекорд нелепости
гипотез.\footnote{Чрезвычайно высокая  оценка художественного значения
Аристофана имеется  и в одном  двустишии Платона (см. Гердер,  1801, №
152), которое привожу в русском переводе.

\begin{verse} Грации искали никогда не разрушающегося храма\\* И нашли
его в духе Аристофана. \end{verse}

Поэзией  Платон  занимался в  юности  и  он, очевидно,  сохранил  свои
эстетические  вкусы даже  тогда, когда  высокоценимый им  как художник
Аристофан  вступил  в  ожесточенную борьбу  с  высокочтимым  учителем.
Большую независимость художественных суждений от личных мотивов трудно
придумать (Прим. автора.)}

\subsection{Непопулярность Демокрита в Афинах}

Так   почему  же   Платон   не  упоминал   Демокрита,  если   злостное
умолчание может  считаться совершенно исключенным?  Наиболее вероятным
предположением будет то, что  афинянин Платон адресовался прежде всего
афинянам, а в Афинах Демокрит вовсе не был известен в его время. Довод
в пользу  этой гипотезы можно  почерпнуть у С.Я.Лурье (1947,  с. 137):
«Труд Демокрита, получивший громкую известность в Абдерах и, вероятно,
во  всей Малой  Азии, привлек  к себе  внимание и  в Афинах.  Демокрит
попытался войти  в кружок  Анаксагора, жившего тогда  в Афинах,  но из
этого  ничего не  вышло: Анаксагор  отнесся к  нему холодно,  причиной
чего, может быть, был  слишком прямолинейный материалистический монизм
Демокрита (как  мы видели,  причиной популярности Анаксагора  в Афинах
мог  быть  его  Верховный  Разум,   в  известной  мере  примирявший  с
ним  идеалистов). Впоследствии  Демокрит, рассказывая  в своих  трудах
о  встречах  с  Анаксагором,  очень  неодобрительно  отзывался  о  его
теориях происхождения  мира и о Разуме,  называя их «стариковскими»...
«Анаксагор не принял  Демокрита к себе», ---  сообщает Диоген Лаэрций.
Еще меньше впечатления он, естественно,  произвел на Сократа: «Он знал
Сократа, но Сократу он был неизвестен». Некоторый успех он имел только
среди более  молодых, еще не вошедших  в силу, ученых. Целый  ряд черт
его учения  был заимствован  физиком Диогеном  из Аполлонии  на Крите,
жившим  тогда  в  Афинах;  афинский софист  Антифонт  заимствовал  все
основные положения его учения, переработав их в софистическом духе. Но
не эти люди задавали тогда тон в Афинах, и Демокрит мог лишь с горечью
констатировать: «Я прибыл  в Афины --- и оказалось, что  никто меня не
знает».

Как и всегда, имеются неувязки: от того  же Лурье (с. 363, прим. 28 со
ссылкой  на Диогена  Лаэрция) узнаем,  что Димитрий  Фалерский отрицал
приезд Демокрита в Афины. Существует, таким образом, три показания: 1)
Демокрит вовсе  не был в  Афинах; 2) он там  был, но его  Анаксагор не
принял; 3)  Анаксагор с ним беседовал,  но отнесся холодно, но  нет ни
одного,  свидетельствующего,  что  Демокрит был  популярен  в  Афинах.
Материалистическая  философия не  была популярна  в Афинах.  Анаксагор
после некоторого  периода популярности  был осужден  и едва  спасся от
гибели,  пала  и  популярность  его учения.  Сочинение  Анаксагора  «О
природе», очень популярное благодаря влиянию Перикла, переписывалось в
свое время  в большом количестве  экземпляров, но потом спрос  на него
так упал, что в начале IV века его можно было купить за драхму (Лурье,
1947,  с.  98)  или  шесть  оболов,  т.е.  двухдневный  паек  гелиаста
(присяжного заседателя).

Это   объяснение  принимает   и  Виндельбанд   (1911,  с.   158--160):
демократизм  был  с  самого  начала   оттеснен  назад,  а  Платон  был
определяющим  гением философии  будущего. В  общем развитии  античного
мышления,  которое приняло  платоновскую традицию,  материалистическая
метафизика   играла  роль   сравнительно  бесплодного   восстановления
старого.

\subsection{Спорность положения, что Платон знал сочинения Демокрита}

Но  Виндельбанд  считает  невероятным,  чтобы Платон  не  знал  трудов
Демокрита. Однако  нельзя вовсе  исключить и  гипотезу, что  Платон не
знал трудов Демокрита. Весьма возможно, что  он о них и слыхал, но уже
тогда литература  подобного рода была  так многочисленна, что  не было
возможности с ней  ознакомиться полностью. Хронологические соображения
совершенно несущественны.  И в  настоящее время  мы знаем,  что многие
опубликованные  работы  долго  остаются  неизвестными  (вспомним  хотя
бы  классическую  работу  Менделя)  широким  кругам  и  их  игнорируют
добросовестные  специалисты.  Но  приводится много  цитат  из  Платона
(впрочем, не  так уж много), показывающих  изложение учения Демокрита.
Это считается доказательством,  что он читал Демокрита и  не хотел его
цитировать:  цитаты,  которые  толкуют как  доказательства  знакомства
Платона  с учением  Демокрита,  касаются двух  сторон  его учения:  1)
атомной гипотезы и 2)  атеизма. Но как и в ряде  других случаев, С. Я.
Лурье дает нам  материал для того, чтобы усомниться  в надежности этих
доводов.  На с.  135 его  «Очерков»  (1947) мы  узнаем, что  учителями
Демокрита  были персидские  маги  и  халдеи: их  оставил  в доме  отца
Демокрита, одного из наиболее почтенных граждан Абдер, персидский царь
Ксеркс по  пути похода  на греков. С  отцом Демокрита  Ксеркс заключил
союз гостеприимства. Восточной  религиозной философии было свойственно
отрицательное  отношение к  религиозным  богам  народной веры,  учение
о  четырех  элементах и  учение  о  математически неделимых  частицах.
В  философии  персидских  магов существовало  учение  о  богах-образах
(=  греческим  «идолам»),  чрезвычайно близкое  к  позднейшему  учению
Демокрита.  Демокрит кроме  персидской изучал  и эллинскую  науку. Его
учитель  и друг  Левкипп мог  познакомить  его с  учениями милетцев  и
элеатов; он тщательно изучал  пифагорейскую науку. Все это показывает,
что упоминание атомизма у Платона могло иметь в виду и не Демокрита, а
других  философов. То  же касается  и  атеизма. Здесь  у С.Я.Лурье  мы
наталкиваемся на  прямо удивительное противоречие.  На с. 135,  как мы
видели, понятие «идолов» он связывает с религиозным учением персидских
магов,  следовательно,  и  придает  ему  религиозное  происхождение  и
толкование.  А  на с.  273  той  же  книги  читаем, что  «уже  Цицерон
справедливо  отмечает,  что взгляд  Демокрита,  по  которому боги  ---
это  образы  (...«идолы»),  по  существу  говоря,  безбожие...  Апулей
впоследствии  тщетно пытается  отвести  от  Демокрита это  обвинение».
Ясно,  таким образом,  что  защита  религиозного толкования  «образов»
вовсе  не может  считаться  тщетной. Безбожие  же  отнюдь не  является
изобретением  Демокрита.   Известен  рассказ  о   беседе  французского
энциклопедиста  Дидро  с  митрополитом   Платоном.  Дидро  заявил:  «я
утверждаю, что бога нет». Митрополит  Платон ответил: «это не новость,
еще в Псалтире сказано: ``рече безумен в сердце своем, несть Бог''».

Для того  чтобы решить на основании  текстов, что Платон имеет  в виду
именно Демокрита  при своей критике, необходимо  тщательно сопоставить
взгляды  Демокрита  и других  мыслителей  ---  эллинов и  философов  и
мыслителей доэллинской  цивилизации, с  которой были хорошо  знакомы и
Платон, и  Демокрит. Вероятно,  для этого сейчас  данных недостаточно,
так как, как правило,  сочинения, вытесненные в идеологической борьбе,
не  сохранились. Но  тот же  С.Я.  Лурье дает  прекрасный пример,  как
трудно  на  основании текста  приписать  его  тому или  иному  автору.
На  с. 153  (1947)  он  указывает, что  Аристофан  в комедии  «Облака»
пародирует  концепцию Демокрита  о  сквозной  причинности. Дий  (Зевс)
смещен  с  престола  и  на  его место  сел  Вихрь  атомистов  (Облака,
379,  1471). Но  ведь  Демокрит  у Аристофана  не  упоминается, а  эта
демокритовская  концепция вложена  в уста  Сократу, философский  образ
которого  совершенно  искажен  у Аристофана  (это  обстоятельство  уже
никаких  сомнений не  вызывает). Мы  знаем также  из предыдущего,  что
Демокрит  не оставил  глубокого  впечатления в  Афинах. У  Аристофана,
противника Сократа, не  было оснований скрывать имя  Демокрита. Имея в
виду  очень  невысокий  философский  уровень  Аристофана,  можно  быть
вполне  уверенным, что  до него  доходили обрывки  философских учений,
распространенных  в Афинах,  поэтому мы  вправе  и учение  о вихре  не
связывать только с Демокритом.

\subsection{Свобода науки в Александрии}

Мы должны поэтому полностью снять обвинение самого Платона и его линии
в  том,  что творения  Демокрита  не  дошли  до  нас. Но  прежде,  чем
попытаться ответить на вопрос, почему  же сочинения Демокрита не дошли
до нас,  постараемся разобрать другой вопрос,  который, кажется, никто
не задавал: почему  так долго сохранялись творения  Демокрита и почему
его так часто цитируют первые  христианские богословы и притом, как мы
видели, далеко не всегда с целью обличения.

Ответ  на  это  может  быть двоякий,  или  правильнее  сказать,  этому
благоприятствовали  две причины:  первая  --- исключительная  свобода,
которая  господствовала  в  блестящем  центре  эллинистической  эпохи,
Александрии. Вторая ---  что в момент зарождения и  первые века своего
существования  христианство  было  лишено  строгого  догматизма  и  не
отказывалось  от  родства с  двумя  идеологиями,  его породившими  ---
иудаизмом и --- эллинизмом. Разберем первую.

В своей превосходной книге «Архимед»  С. Я. Лурье рисует замечательную
картину  Александрийского  Музея,  организованного  Птолемеем  Сотером
I  (1949,  с.  43).  С   юридической  стороны  Музей  был  религиозным
сообществом  при  храме  Муз,  но на  структуру  его  оказала  большое
влияние платоновская Академия.  Центром философских занятий оставались
Афины,  в  Александрии  расцветали специальные  науки.  «Идея,  легшая
в  основание  организации  Музея,  была  весьма  гуманной:  собрать  в
Александрии крупных,  зарекомендовавших себя ученых, освободить  их от
всяких жизненных забот, предоставить им  максимальный досуг и дать им,
таким образом, возможность заниматься,  чем каждый желает, без всякого
давления с чьей бы то ни  было стороны. Знаменитые ученые, собранные с
различных концов  мира, жили при  храме Муз на полном  иждивении царя;
они обедали совместно, и эти обеды сопровождались научными беседами на
самые  различные темы».  На научные  исследования отпускались  большие
средства,  например,  на  работы знаменитого  Эратосфена,  измерившего
впервые  радиус   Земли.  Ценнейшей   частью  Музея   была  знаменитая
библиотека,  где были  собраны почти  все греческие  книги. Ученые  не
относились  со  слепым  почтением  к старым  книгам,  даже  к  Гомеру,
игравшему  у  греков роль  священного  писания.  Они критиковали  даже
Гомера,  исправляли ошибки,  допускали  сомнения  в авторстве  Гомера.
Работавший  в  Музее  врач  Герофил  выступил  против  обычного  в  то
время  мнения, что  душа человека  находится в  сердце, и  открыл, что
органом  мышления является  мозг, что  артерии наполнены  не воздухом,
как  думали раньше,  а кровью.  «К  этим выводам  он пришел,  вскрывая
человеческие  трупы;  до него  никто  не  решался на  такие  вскрытия,
---  это считалось  кощунством».  Таким образом,  даже  там, где  дело
соприкасалось с  религиозными представлениями (а Музей,  как мы видели
раньше,  был  по  мысли  религиозным  сообществом),  вольнодумство  не
преследовалось.  Такое свободомыслие,  видимо, было  связано только  с
Александрией  и,  по-видимому,  с  близкими  по  духу  с  Александрией
Сиракузами. Но, например, в  конкурирующей с Музеем Пергамской научной
школе, ориентировавшей  на Рим, этого  не было: царь Аттал  I приказал
казнить «грамматика»  Дафила за недостаточно почтительное  отношение к
Дельфийскому священному  оракулу (С.Я.Лурье, 1945, с.  44). Однако, по
мнению  С. Я.  Лурье,  впечатление высокого  расцвета  науки и  полной
свободы научной мысли является  поверхностным (там же). Расцвет науки,
по  Лурье,  носил односторонний  характер,  и  в области  гуманитарных
наук, например истории, философии, наблюдается отсутствие оригинальных
трудов,  усталость мысли  и упадок.  Не берусь  судить об  истории, но
что  касается  философии, то  тут  дело  вовсе  не было  в  недостатке
интереса к  ней или отсутствии  свободы мысли. Сам Лурье  считает, что
сочинения  Демокрита  были  в  Александрийской библиотеке  и  что  нет
основания  думать,  что изучение  Демокрита  запрещалось  или что  его
произведения хранились в каком-либо  особом секретном фонде библиотеки
(до этого в античном мире, по-видимому, не додумались). Просто, по его
мнению, люди уже  утратили навыки свободного мышления (1945,  с. 46) и
«приучались с  детства направлять мысль по  определенному, одобренному
начальством фарватеру».

\subsection{Платонизм александрийской школы}

В  качестве  примера того,  что  привыкшие  к одобренному  начальством
мышлению  люди  старались  забегать  вперед,  угадывая  мысль,  власть
имущих, Лурье приводит ряд примеров, но не из научной или философской,
а исключительно из  придворной области. На этой же с.  46 указано, что
руководитель библиотеки поэт Каллимах  счел долгом в стихах прославить
убийство  пятнадцатилетней царевной  Вереникой любовника  своей матери
Деметрия. Мать же  Вереники (жена киренского царя)  хотела выдать свою
дочь  за своего  любовника,  а  не за  египетского  царя Птолемея  III
Евергета (за которого Вереника была просватана в детстве и за которого
в конце концов и вышла замуж),  так как по каким-то причинам не желала
соединения Кирены  и Египта в  одних руках (Вереника  была наследницей
киренского  престола). Не  знаю, какие  мотивы руководили  Вереникой в
ее  убийстве, но  с  современной точки  зрения, находящей  возможность
известной реабилитации Стелло, Алеко, Тонио, Арбенина и других гнусных
убийц,  убийство молодой  девушкой  навязываемого  ей матерью  бывшего
любовника  может защищаться  с  гораздо  большим основанием.  Вереника
же,  судя  по изложению  Лурье,  любила  своего  мужа, так  как  перед
отправлением его в поход принесла свои волосы в дар богам. Так как они
потом исчезли,  то чтобы утихомирить мужа  Вереники Евергета (владение
волосами  дает огромную  власть над  бывшим их  обладателем) известный
астроном  и математик  Конон из  Самоса, работавший  в Музее,  заявил,
что  волосы Вереники  перенесены  богами на  небо,  где они  красуются
и  по  сие время  под  именем  созвездия  «волосы Вереники».  С  таким
«подхалимством» можно  еще мириться, но  в научные теории  Птолемеи не
лезли с «руководящими указаниями».

Другим доводом в пользу того,  что свобода мышления в Александрии была
мнимой, Лурье  считает то,  что никто  из ученых  Музея не  проводил в
своих сочинениях материалистические взгляды  (1945, с. 45). «Поскольку
нам  известны  взгляды  ученых  Музея,  они  стояли  на  платоновских,
академических  или стоических  позициях. В  ряде областей  эти позиции
делали невозможным дальнейший  прогресс науки. Как мы  увидим, как раз
наиболее  выдающиеся ученые  в  ряде вопросов,  не  связанных тесно  с
материалистическим мировоззрением, фактически  возвращаются к позициям
Демокрита,  но при  этом  следы заимствования  стираются».  Но в  ряде
областей  прогресс несомненен  ---  математика и  астрономия (с.  48).
Тут  Лурье  дает  обычное   объяснение,  что  прогресс  этих  областей
«объясняется  отчасти  чрезвычайным   развитием  военного  дела,  тем,
что  для  военных  усовершенствований  необходима  была  основательная
математическая, механическая  и техническая основа, а  для торгового и
военного мореходства --- основательное знание астрономии». Приходиться
поражаться,  как  такая  странная  аргументация  приводится  умными  и
образованными людьми.  Как раз книга об  Архимеде свидетельствует, что
там,  где  механические  знания были  использованы  гениальным  ученым
для  обороны  Сиракуз,  римская  военная  машина  оказалась  бессильна
овладеть  небольшим  куском  земли  и  взяла  город  только  благодаря
предательству.  Почему  уже  Египет,   имевший  в  своем  распоряжении
обширный конклав  ученых, не  мог сопротивляться  Риму? Почему  Рим не
использовал александрийскую  культуру для  своих целей, почему  она не
развивалась в Риме, главной целью которого было военное превосходство:
и этой  цели, как  известно, Рим достиг  без александрийской  науки. И
какое значение  для военного, морского  и торгового дела  того времени
имела  геометрия  Евклида  и  Аполлония,  измерение  радиуса  Земли  и
расстояний Земли  от Луны  и Солнца, создание  астрономической системы
Птолемея  и гелиоцентрической  системы Аристотеля  и многое,  и многое
другое.  Величие александрийской  эпохи  и заключалось  именно в  том,
что  она  последовательно  проводила  платоновские  руководящие  идеи:
чистую,  теоретическую науку,  не обязательно  связанную с  практикой,
признание  практикой Космоса  как  целого и  математизацию науки.  Эти
положения отсутствовали  в демокритовском  наследстве и  именно потому
демокритовская философия не могла выдержать конкуренции с платоновской
и, как правило, ученые Александрии  Демокрита не читали. Как известно,
не выдержал конкуренции с Платоном и его ближайший и величайший ученик
Аристотель, несмотря  на то что последний,  как воспитатель основателя
Александрии,  Александра  Македонского,   мог  бы,  казалось,  оказать
большое влияние.

\subsection{Широта и диалектичность учения Платона, «учителя ищущих»}

Непреходящее величие  Платона состоит  именно в  том, что  его система
необычайно  широка  и  диалектична   в  истинном  и  лучшем  понимании
этого  слова,  заключающемся   во  всесторонности  и  самокритичности.
Поэтому-то Платон жив и сейчас. Это великолепно выражено в предисловии
Лиона  Робена к  французскому изданию  Платона 1950  года: «Творчество
Платона отличается  изумительной жизненностью и даже  сейчас действует
притягательно  на  умы.  Я  думаю,  это  объясняется  тем,  что  сразу
после  ознакомления  с ним  западным  миром  оно оказалось  ферментом,
исключительно способным пробуждать духовную энергию. ``Думать, говорил
Платон, это значит для души  беседовать сама с собой''. Чтение Платона
постоянно побуждает к подобным беседам. Именно поэтому всякий понимает
его  по-своему; поэтому,  начиная с  античности, когда  Платон казался
догматиком у своих непосредственных преемников, он оказался чуть ли не
скептиком в Новой  Академии с Аркесилаем и  Карнеадом; в неоплатонизме
Плотина  и его  последователей  его рационализм  смешался с  восточным
мистицизмом; евреи и христиане,  претендовавшие на философию, видели в
нем инструмент  божественных откровений. Поэтому я  отказался от мысли
дать  при  этом переводе  содержание  его  доктрины,  как я  ее  лично
понимаю. Всего важнее думать вместе  с ним и производить то совместное
с ним  исследование, которое  он постоянно рекомендовал.  Этим человек
приучается к критическому духу, научается  быть искренним сам с собой,
идти смело  до конца  исследования, которому  подвергается собственная
совесть, ничего не уступать престижу непроверенных мнений. Для Платона
действительно  философия не  преподается: она  обозначает поведение  и
правило жизни... Работать без устали, чтобы превзойти самого себя, вот
сущность и  сила его  философии. Об этом  свидетельствуют великолепные
слова Федра: ``преподавать, это не значит вдалбливать в душу застывшие
и  безжизненные  идеи, но  осеменять  ее  так, чтобы  собранная  жатва
осеменяла в свою очередь другие души и так бесконечно''».

Та же мысль  выражена кратко в предисловии к  русскому изданию Платона
(к  сожалению,  приостановленном  на  шестом  выпуске)  С.Жебелевым  и
Э.Радловым (1923,  с. 6): «Даже  тот, кто оказался бы  наиболее чуждым
положительным доктринам  Платона, мог  бы всегда приветствовать  в нем
своего  руководителя,  потому что  в  области  знания Платон  навсегда
останется  тем, чем  он  хотел  быть для  основанной  им Академии  ---
\emph{учителем  ищущих}».  Александрийская школа  полностью  следовала
заветам Платона и нашла очень многое и не только в точных науках. Ведь
родоначальником неоплатонизма является  Аммоний Саккас из Александрии,
а отсюда пошла  римская школа во главе с  знаменитым Плотином; получил
образование  в Александрии  и важнейший  представитель афинской  школы
Прокл.  И в  философии  нельзя поэтому  говорить  об усталости  мысли;
равных же или близких по величине  к Платону философов мы вообще можем
пересчитать по  пальцам; некоторую  конкуренцию с ним  могут выдержать
только славнейшие философы немецкой идеалистической школы.

\subsection{Несовместимость платонизма и догматизма}

А  отсюда  мы  получаем  ответ, почему  Платона,  Демокрита  и  других
эллинских философов так охотно  цитируют первые христианские богословы
и почему потом почти все  они, кроме Аристотеля, были преданы забвению
вплоть  до Возрождения.  Всякая новая  крупная идеологическая  система
(безразлично,  политическая,  философская,  религиозная  или  научная)
проходит  закономерно  ряд  стадий.   В  момент  своего  зарождения  и
некоторое время  после она носит  живой, творческий характер  и будучи
уверенной  в  своей правоте  не  стремится  зажать рты  инакомыслящим.
Тогда  Платон,  учитель  ищущих,  представляет  огромный  интерес,  но
кроме того, тогда усиленно  разыскивают предшественников нового учения
среди  мудрецов  древности.  Для раннего  христианства  ни  эллинская,
ни  иудейская  культуры не  были  чем-то  враждебным. Среди  греческих
философов разыскивали  предшественников христианства.  Наиболее близок
был, конечно, Платон, но как было указано выше, и у Демокрита находили
элементы христианского мировоззрения.  Юстин-мученик (казненный в Риме
ок. 166 года)  доказал, что почти все  содержание христианского учения
уже имеется  в языческой философии и  мифологии. И это потому,  что «у
христианства  и философии  один  и тот  же  источник ---  божественный
логос, разлитый во всем мире. В  Христе этот логос только проявился во
всей полноте.  К христианам  Юстин относил всех  тех, кто  прожил свою
жизнь  с ``логосом''.  Таковы из  греков Гераклит  и Сократ»  (История
философии, т.  1, с. 386). Мы  знаем, что синтез иудаизма  и эллинизма
дал Филон  Александрийский, один из  непосредственных предшественников
христианства и слова Филона ---  «в начале было слово» (логос) перешли
в начало канонического четвертого Евангелия Иоанна.

В  христианские «святцы»  перешел  целый ряд  античных мыслителей  без
различия «линий»:  Гераклит, Демокрит, Сократ, Платон,  Аристотель. Но
как всегда  бывает: творческий период нового  мировоззрения приводит к
возникновению  многочисленных  школ, направлений  или,  по-церковному,
ересей.  Раздоры проникают  в новую  церковь и  становится необходимым
навести   какой-то  порядок.   Созываются   соборы  для   установления
«единомыслия».  Постепенно происходит  «чистка»: философов  не столько
хвалят,  когда находят  кое-что согласное  с христианским  вероучением
(как  было  с  Демокритом),   сколько  ругают,  когда  находят  что-то
неподходящее. Не говоря уже о Демокрите, и Платон становится неудобным
именно в  силу того, что он  «учитель ищущих»: для нашедших  истину он
не  нужен  и даже  вреден:  платоническое  направление в  христианстве
сменяется  или полным  отрицанием  всякой  языческой премудрости,  или
аристотелевским  направлением,  более  сродным  всякому  догматику  по
своей  систематичности  и   последовательности.  Завершение  почитания
Аристотеля  находим у  главного  богослова  католической церкви,  Фомы
Аквината,  учение которого  и сейчас  является философией  католицизма
(так  называемый  неотомизм).   Православная  церковь  поступила  куда
решительнее.  Император Юстиниан  закрыл  в Афинах  Академию и  выгнал
всех  философов  без  различия  направлений: не  этим  ли  объясняется
столь решительное  отставание Восточной  Европы от Западной  в области
философии, продолжающееся до сего времени.

\subsection{Вред догматизма во всех областях культуры}

Переход  от  разномыслия  к принудительному  единомыслию  ---  явление
закономерное  и не  связано  с религиозным  характером нового  учения.
Просто  фанатизм   основателей  нового  учения,  готовых   принести  в
жертву самих  себя для  торжества нового учения,  сменяется фанатизмом
церковников,  склонных  более  принести в  жертву  своих  противников.
Например,  т.н. «классик  революционной мысли  домарксовского периода»
В.А.Зайцев  так  рассуждает  о   разнообразии  мнений,  возникающем  в
результате   великого  и   благодетельного  события   ---  эмансипации
человеческого  ума,  которому  человечество обязано  многими  великими
последствиями (с.  343): «Но нельзя не  видеть, что оно имеет  и очень
вредные последствия, и что с  течением времени эти вредные последствия
перевешивают приносимую ими пользу. Вредные последствия обнаруживаются
всегда  немедленно; слишком  известен исторический  факт, что  всякий,
например,  религиозный переворот,  составляющий шаг  вперед в  истории
человечества, сопровождается непременно раздроблением новой религии на
бесчисленные секты. Это разумеется,  прямой результат события, которое
само по себе благодетельно; но этот результат вовсе не благоприятный».
Аналогию с религиозными учениями Зайцев распространяет и на науку. Ему
вовсе не  нравится разнообразие  мнений (с.  345--346): «В  наше время
нередко можно слышать мысль, которая  в старину никому не могла прийти
в голову: что всякое мнение должно  быть равно уважаемо и что можно не
соглашаться с  ним, но нельзя  оспаривать права иметь его,  потому что
абсолютно истинного и  честного нет, а, следовательно,  каждый прав со
своей  точки  зрения,  как  бы  ни  были  противоположны  их  взгляды.
\textit{Терпимость  в отношении  к этим  проповедникам терпимости  ---
самая худшая  из всех  терпимостей}. Невозможно ничего  выдумать более
развращающего,  чем подобная  терпимость.  Неужели же,  в самом  деле,
так-таки  и нельзя  решить, какой  взгляд на  данный предмет  истинен,
верен и честен?»

И не  следует думать,  что требование «честных  убеждений» применялось
только к политическим  понятиям, где оно во  многих случаях оправдано:
требование   восстановления  рабства   ---   есть  подлое,   нечестное
убеждение. Нет,  видные представители нашей  радикальной интеллигенции
считали,  что   и  в  далекой   от  жизни  науке  можно   говорить  об
обязательных,  честных  убеждениях.  Например, Н.  Г.  Чернышевский  в
письме  к своим  детям от  6 апреля  1878 г.  (1938, с.  526) пишет  о
космогонической гипотезе Лапласа как о таком положении, «сомневаться в
котором совершенно недопустимо. ``Я почти нисколько не сомневаюсь, что
лапласова  гипотеза верна'',  ---  это нечто  совершенно иное,  нежели
простое:  ``лапласова гипотеза  верна''.  Друзья мои,  кто сказал  бы:
``очень  правдоподобно,  что таблица  умножения  верна'',  тот был  бы
трус,  или лжец,  или невежда.  О научных  истинах выражаться  так ---
неприличная вещь, пошлая вещь, бессмысленная вещь».

Этот  принцип «утвержденных  истин»  во всех  науках  (то, что  Дюринг
называл «окончательные  истины в последней инстанции»),  как известно,
широко  проводился  в  недавнее  сталинское  время,  да  и  сейчас  он
вовсе  не изжит  во  многих областях  (искусство, гуманитарные  науки,
биология,  философия) и  мы знаем,  что везде,  где он  проводился, он
приводит обязательно к застою данной области культуры. Поэтому отличие
религиозного (например, католического) догматизма от антирелигиозного,
претендующего  быть   «научным»,  заключается  в   следующем.  Церковь
устанавливает   определенные  догматы,   сомневаться   в  которых   не
полагается, но  даже в  области богословия сохраняются  так называемые
«теологумены»,  где дискуссии  допускаются.  Вне  же сферы  богословия
остается  обширная  область   для  исследования,  почему  католические
монастыри  и  выдвинули  большое  число  предшественников  и  активных
деятелей  Возрождения. В  мнимо же  научной догматике,  действующей по
методу Зайцева-Чернышевского, решительно все  догматизируется, и уж на
такой базе  ожидать возрождения не приходится.  Возрождение происходит
лишь по мере освобождения отдельных областей науки от этой системы.

\subsection{Причины  исчезновения  творений Демокрита:  их  догматизм,
отсутствующие   школы.   Галилей,   Кеплер,  Ньютон   ---   противники
Аристотеля, а не Платона}

И  теперь мы  можем ответить  на  вопрос, почему  же исчезли  творения
Демокрита, так хорошо известные античности. Уже сейчас, до тщательного
сопоставления обеих линий по отдельным отраслям знания, можно сказать,
что  на  это  имелись  две  основные  причины.  Во-первых,  в  отличие
от  направления  Платона,  допускавшего  развитие  основных  положений
(что  и имело  место в  Академии и  ее ответвлениях),  у Демокрита  мы
имеем жесткую  догматическую систему.  Этого не отрицает  и С.Я.Лурье.
После разбора  учения о  познании Демокрита, указав  на ряд  натяжек и
противоречий, он  пишет (1947,  с. 164): «Это  приводит, как  в случае
закона  ``равнораспределенности'',  к  противоречиям  и  натяжкам,  но
Демокрит  перед этим  не  останавливается: он  твердо  убежден в  том,
что  его  система математически  доказана  и  непререкаема; удобная  и
плодотворная гипотеза  им воспринимается как закон  природы». Вместе с
тем  система Демокрита  гораздо менее  удовлетворительна, чем  система
Аристотеля.  Поэтому,  а также  в  силу  ее более  материалистического
характера,   догматики  воспользовались   более  подходящей   системой
Аристотеля, а не системой Демокрита.

Преемников же  у него в  смысле школы  не оказалось, которые  могли бы
пронести заветы учителя через время  гонений на эллинскую культуру. Об
этом  ясно  сказано  у  Лурье  (1947,  с.  287):  «Во  всяком  случае,
у  Демокрита  еще не  может  быть  речи  о  мудрецах, как  о  какой-то
особой  замкнутой  группе  или  «школе»,  наподобие  платоновской  или
эпикурейской; его  мудрецы ---  просто наиболее умные  и образованные,
чтимые государством люди...». Эпикур,  как известно, не был преемником
Демокрита  и во  многом с  ним расходился,  но он  основал школу  «сад
Эпикура»:  результат  ---  его  идеи хорошо  сохранились  в  сочинении
Лукреция «О природе вещей».

Нет оснований думать поэтому, что  обычное представление о Платоне как
центре эллинской культуры несправедливо. Вся картина развития культуры
показывает,  что  Демокрит  и  Платон не  были  равноценными  фигурами
и  значение Демокрита  как  основоположника мировоззрения  выдвинулось
только после Возрождения.  Но как же быть с  Галилеем, Бруно, Бэконом,
которых Лурье считает своими  единомышленниками в оценке значения двух
линий.  Подробнее я  разберу этот  вопрос,  когда мы  дойдем до  эпохи
Возрождения, сейчас мы заняты античной  культурой. Но уже сейчас можно
сказать вкратце; что все утверждение  С. Я. Лурье основано на смешении
двух школ: Аристотеля и Платона, как бы единой линии. Идейным стержнем
Возрождения было  не тождество  линии Демокрита  над линией  Платона в
широком смысле,  а ниспровержение авторитета Аристотеля  к вящей славе
Платона.  Это  подчеркнуто  достаточно независимым  от  пристрастия  к
Платону свидетелем,  Берналом (1956).  На с.  272--273, в  таблице, он
отмечает  в  числе  развенчанных  авторитетов  в  философии,  механике
и  гидравлике  Аристотеля,  в  качестве  восстановленных  в  философии
Платона,  механике  ---  перешедшего в  христианство  александрийского
ученого Филопона. Что  касается Демокрита, то внимание  к нему привлек
провансальский священник Гассенди,  впервые придавший атомной гипотезе
научное,  а не  мировоззренческое значение.  Благодаря его  набожности
атомы были освобождены от  их атеистических, разрушительных ассоциаций
(Бернал, с.  256). Если  принять во внимание,  что два  других великих
ученых, доведших атомную теорию  до окончательной формулировки, Ньютон
и Дальтон, были интенсивно религиозными  людьми, то мы придем к весьма
интересному  выводу, что  в деле  формирования научной  атомной теории
материалистическое мировоззрение роли не играло.

Коперник эпиграфом своей великой книги избрал изречение Платона: «и не
вступает  сюда никто,  не знакомый  с  геометрией». Как  пишет тот  же
Бернал (с. 116): «...но только в эпоху Возрождения работы Платона были
вновь изучены в оригинале и оказали  влияние, по крайней мере столь же
большое, как и  в то время, когда они были  написаны. Главным образом,
благодаря  Платону взгляды  представителей раннего  гуманизма не  были
научными (по  терминологии Бернала это  значит, что они  были заражены
мистикой. --- \emph{А.Л.}).  В XVI и XVII  веках, однако, свойственное
Платону  увлечение  математикой  сыграло важную  роль  в  формировании
мышления  Кеплера, Галилея  (с.  234 и  далее)  и, через  кембриджских
платоников, также и Ньютона (с. 265)».

Как   известно,  на   известном  процессе   Галилея  он   обвинялся  в
«пифагорейском учении» и вполне справедливо, так как учение Коперника,
которое он защищал, восходит к школе Пифагора, а не к Демокриту. И сам
Лурье, защищая  свой тезис  о приверженности Галилея  линии Демокрита,
все время противопоставляет Демокрита  Аристотелю, умалчивая о Платоне
или смешивая в  одно целое Аристотеля и Платона (Лурье,  1945, с. 129,
130, 155 и 156).

\subsection{Ф.Бэкон --- противник учения Коперника}

Коснемся теперь  Ф.Бэкона, защитника  Демокрита в  противовес Платону.
С.Я.Лурье в  качестве эпиграфа  к первой части  своей книги  (1947, с.
9,  Греческая  наука  до  Демокрита) приводит  цитату  из  Ф.  Бэкона:
«Название  софистов... подходит  ко всей  этой породе  --- к  Платону,
Аристотелю, Феофрасту  и к  преемникам... Но  более древние  из греков
---  Эмпедокл,  Анаксагор,   Левкипп,  Демокрит,  Парменид,  Гераклит,
Ксенофан... и остальные (Пифагора  мы не касаемся, как основоположника
суеверия)...  отдавались  отысканию  истины»  (Ф.  Бэкон  Веруламский:
Новый  Органон, 1,  71; с.  138  рус. пер.).  Формулировка ясная:  все
софисты в  смысле Бэкона очевидно  ничего, кроме суеверий  и софистики
(в  обычном  бранном  смысле),  не дали.  Значит,  все  истинное  дано
материалистами?  Ну  а как  быть  с  теорией  Коперника? Ф.  Бэкон  не
признавал теории  Коперника, хотя  жил (1561--1626) почти  на столетие
позже  Коперника  (1473--1543)  и  был  современником  великих  борцов
за  новое миропонимание,  Бруно (1548--1600),  Галилея (1564--1642)  и
Кеплера  (1571--1630). В  этой великой  борьбе  он был  не на  стороне
прогресса. Поэтому только люди, совершенно игнорирующие историю науки,
могут  полагать, что  Ф.  Бэкон может  быть  сопричислен к  величайшим
деятелям Возрождения и считаться  в числе основоположников современной
науки. Он,  конечно, имел  влияние на  развитие науки.  Его пропаганде
эксперимента  мы  обязаны  многим,   ему  приписывают  важную  роль  в
деле  организации Лондонского  Королевского  Общества  и, конечно,  он
является главным апостолом индуктивного метода. Но переоценка значения
эксперимента и индукции приводит  к грубым методологическим ошибкам, и
отрицание им новой  системы Коперника не было  случайным. Очень многие
крупные  мыслители  критически относились  к  роли  Ф. Бэкона,  назову
таких  разнородных, как  Либих и  Ф.  Энгельс. В  начале своей  статьи
«Естествознание  в  мире  духов»  (Диалектика природы,  1949,  с.  28)
Энгельс пишет:  «Существует старое положение диалектики,  перешедшее в
народное сознание:  крайности сходятся.  Мы поэтому вряд  ли ошибемся,
если станем  искать самые  крайние степени фантазерства,  легковерия и
суеверия не у того  естественно-научного направления, которое, подобно
немецкой  натурфилософии, пыталось  втиснуть объективный  мир в  рамки
своего  субъективного мышления,  а наоборот,  у того  противоположного
направления, которое,  чванясь тем, что оно  пользуется только опытом,
относится к мышлению с глубочайшим презрением и, действительно, дальше
всего ушло по части оскудения  мысли. Эта школа господствует в Англии.
Уже ее  родоначальник, прославленный  Фрэнсис Бэкон  жаждет применения
своего  нового эмпирического,  индуктивного  метода  прежде всего  для
достижения следующих  целей: продление  жизни, омоложение  в известной
степени, изменение телосложения  и черт лица, превращение  одних тел в
другие,  создание новых  видов, владычество  над воздухом  и вызывание
гроз; он жалуется на то,  что такого рода исследования были заброшены,
и дает в своей естественной истории форменные рецепты для изготовления
золота и совершения разных чудес».

В другом  месте, в статье  «Старое предисловие к  ``Анти-Дюрингу'' ---
о  диалектике»  (Диалектика  природы,  с.  24--25)  Энгельс  указывает
две  формы  диалектической  философии, которые  могут  стать  особенно
плодотворными  для   современного  естествознания:  «Первая   ---  это
греческая  философия. Здесь  диалектическое мышление  выступает еще  в
первобытной  простоте, ---  не нарушенной  теми милыми  препятствиями,
которые сами по себе создала метафизика XVII и XVIII веков --- Бэкон и
Локк  в  Англии,  Вольф  в  Германии  ---  и  которыми  она  заградила
себе путь  от понимания  единичного к  пониманию целого,  к постижению
всеобщей  связи  вещей.  У  греков  --- именно  потому,  что  они  еще
не  дошли  до  расчленения,  до   анализа  природы,  ---  природа  еще
рассматривается  в  общем,  как  одно целое.  Всеобщая  связь  явлений
природы  не  доказывается  в  подробностях: она  является  для  греков
результатом непосредственного созерцания.  В этом недостаток греческой
философии, из-за которого она  должна была впоследствии уступить место
другим воззрениям.  Но в  этом же заключается  и ее  превосходство над
всеми  ее позднейшими  метафизическими  противниками. Если  метафизика
права по отношению к грекам в  подробностях, то в целом греки правы по
отношению к метафизике.  Это одна из причин, заставляющих  нас снова и
снова  возвращаться в  философии,  как и  во  многих других  областях,
к  достижениям того  маленького  народа,  универсальная одаренность  и
деятельность которого  обеспечили ему в истории  развития человечества
место, на  какое не может  претендовать ни один другой  народ». Второй
формой диалектики по Энгельсу является классическая немецкая философия
от Канта до Гегеля.

Несомненно, Ф.Бэкон  имел заслуги, но  он был не свободен  от суеверий
(что  обычно считается  монополией идеалистов)  и не  понимал важности
изучения природы  как целого:  это последнее свойство  --- характерная
особенность  линии  Платона;  поэтому \emph{не  случайно,  что  долгое
развитие  гелиоцентрической   системы  целиком  протекало   на  «линии
Платона».}

\subsection{Необоснованность обвинений «Линии Платона»}

Из указанных в  п. 1 семи пунктов обвинения против  Платона я разобрал
шесть  и  постарался  показать,   что  они  совершенно  необоснованны.
Остался третий  пункт, что линия Платона  заимствовала ценные открытия
линии  Демокрита  и  их  присваивала  себе.  И  этот  пункт  из  всего
изложенного маловероятен, и плодотворность  науки в античные времена и
в  период Возрождения  так окрашена  платоновским миропониманием,  что
крупные  заимствования маловероятны.  Бернал, при  всей его  антипатии
к  платоновскому  мировоззрению,   принужден  признавать  плодотворное
значение  Платона  по  сравнению,  например,  с  Конфуцием  (с.  116):
«Предпочтение, которое Платон  отдавал математике, обеспечило наличие,
по крайней  мере, одной  действительно научной дисциплины  в обучении,
которое иначе могло бы быть  чисто литературным. Конфуций, чье влияние
на китайское образование  было почти столь же  длительным, как влияние
Платона  на  Западе,  совершенно   не  занимался  математикой.  Этому,
возможно, в значительной  мере способствовала сравнительная отсталость
китайской науки» (не  имея под руками подлинника, не  могу ручаться за
правильность перевода.  По смыслу  надо бы читать  «Это способствовало
отсталости китайской науки». --- \emph{А.Л.}).

Но  если тезис  Лурье о  том, что  Демокрит, а  не Платон  был центром
эллинской  науки и  неверен,  то может  быть  все-таки роль  Демокрита
преуменьшена,  и  что  очень  многое  из  его  сочинений  использовано
Платоном, Аристотелем  или другими учеными  линии Платона. В  этом нет
ничего удивительного: всякое  великое открытие имеет предшественников.
Евклид только  завершил длинную  работу геометров,  а вовсе  не создал
своей геометрии в целом.  «Конические сечения» Аполлония вытеснили все
существовавшие до него труды по коническим сечениям, которые поэтому и
не дошли до нас (Лурье, 1945, с. 200). В древности даже было высказано
предположение, что «конические сечения»  Аполлония есть просто плагиат
книги  с  тем же  названием  Архимеда  (Лурье,  1945, с.  166).  Лурье
разбирает и  опровергает это предположение,  но вместе с  тем считает,
что Архимеду было  известно сочинение Аполлония, которого  он нигде не
цитирует, хотя  иногда пользуется  его терминологией (Лурье,  1945, с.
202, 237; 1956, с. 39). Мало  того, и Демокрита обвиняли в плагиате по
отношению к Левкиппу,  а Эпикур полагал, что  никакого Левкиппа вообще
не было (Лурье, 1947, с. 134). Как  видим, и по линии Демокрита не все
ясно.

В  следующих главах  эта работа  будет  проведена по  отделам науки  и
культуры,  причем по  каждому  разделу  будут сопоставлены  достижения
линий Платона и Демокрита, установлено,  как эти достижения могут быть
связаны с  соответствующим мировоззрением,  и на этом  основании будет
сделан вывод  о возможном заимствовании.  Начнем с математики  и затем
через точные и естественные науки перейдем к гуманитарным отделам.

\clearpage

\section{ЛИНИИ В МАТЕМАТИКЕ}

\subsection{Античные математические школы}

Математические достижения  древних греков составляют,  бесспорно, одно
из  величайших,   если  не  величайшее  украшение   античной  культуры
и  вместе  с   тем  они  особенно  отличаются  по   характеру  как  от
достижений  предшественников Эллады,  так  и от  ее преемников.  Связь
с  философскими  направлениями  недостаточно подчеркивается  и  потому
этот  вопрос заслуживает  особо  пристального  рассмотрения. Вместе  с
тем  он будет  служить  образцом для  разбора  аналогичного вопроса  и
в  отношении  других  разделов  культуры.  Рассмотрение  целесообразно
вести, разбирая последовательно следующие  вопросы: 1) связь эллинской
культуры  с предшествовавшими;  2)  специфичность эллинской  культуры;
3)  положительный   вклад  в  науку  обеих   школ:  идеалистической  и
материалистической; 4) методический вклад  тех же школ; 5) возможность
заимствования или  плагиата идеалистами  достижений материалистической
школы; 6) связь с философией органического, а не личного характера; 7)
личный вклад глав  школ по сравнению с достижениями школы;  8) связь с
практикой; 9) связь с религией; 10) связь с политикой.

Что  касается   первого  вопроса,  о  \emph{преемственности},   то  он
совершенно ясен и как будто не вызывает в основном дискуссий. Главными
предшественниками эллинской математики были  Египет и Вавилон, а также
Финикия. Вавилонская и египетская  культура проникала в Элладу многими
путями. Во-первых, через Персию,  которой были подчинены малоазиатские
колонии  греков.  Математикой  занимались  виднейшие  натурфилософские
школы (см. Рыбников, 1969, с. 25):  1) ионийская (VII--VI вв. до н.э.);
2) пифагорейская (VI--V вв. до н.э.); 3) афинская (со второй половины V
в. до н.э.). Прямой преемницей  афинской школы была 4) александрийская
школа,  где античная  математика  достигла высшего  развития. Все  эти
школы были  связаны между  собой: Пифагор по  рождению (о.  Самос) был
ионийцем, близким  соседом центра ионийской школы,  Милета; переехав в
Южную Италию, он создал там  свою школу, теснейшим образом связанную с
афинской, а эта последняя дала  александрийскую школу. Мы знаем также,
что  лидеры  двух  линий,  Демокрит  и  Платон,  были  хорошо  знакомы
с  вавилонской  и  египетской   культурой  как  по  своему  воспитанию
(Демокрит, получивший воспитание у персидских ученых, магов и халдеев,
см.  Лурье,  1947, с.  135),  так  и  во  время своих  путешествий  (и
Демокрит, и Платон). Отметим сразу,  что из перечисленных четырех школ
только первая считается  представительницей примитивного материализма,
остальные три  школы ясно  выраженного идеалистического  характера. Об
особой  школе  или  направлении  Демокрита  в  математике  упоминается
лишь  вскользь  даже  в   цитированной  книге  по  истории  математики
Рыбникова (1960), хотя эта книга представляет собой содержание лекций,
читаемых  для  математиков-специалистов в  Московском  государственном
университете,  и  книга  допущена  Министерством  высшего  и  среднего
специального  образования  РСФСР  в   качестве  учебного  пособия  для
университетов. Автор историю излагает с марксистской позиции и считает
вслед за Ф.Энгельсом, что предметом математики являются количественные
отношения  и  пространственные  формы действительного  мира.  В  книге
Рыбникова говорится лишь (с. 48), что атомистические взгляды Демокрита
распространились и  на математику  и явились источником  некоторых его
высказываний  о математически  бесконечно малых  и о  применении их  к
определению некоторых геометрических величин.  Но тут же прибавляется,
что с  математической стороны  подобных высказываний  известно слишком
мало; гораздо больше  известно о возражениях их  научных противников и
что апории  Зенона убедительно доказали, что  для точных доказательств
и   логически   исчерпывающих   решений  задач   нельзя   пользоваться
бесконечностью, опираясь на наивные атомистические соображения.

Таким образом, даже рассматривая  историю античной математики с высоты
птичьего полета, мы получаем впечатление,  что за период ее наилучшего
развития не идеализм отступил перед материализмом (как многие думают),
а  наоборот, материализм  перед  идеализмом. Но,  конечно, это  первое
впечатление должно  быть тщательно  проверено, принимая  в соображение
все  доводы материалистов,  в  частности  С.Я.Лурье, мнение  которого,
очевидно, не использовано в современной советской истории математики.

\subsection{Характер эллинской математики}

Перейдем  теперь к  следующему вопросу:  \emph{специфичности эллинской
математики}.  В   общем  виде  и   этот  вопрос  не   вызывает  спора.
Предшественники Эллады накопили большое число эмпирических фактов. Без
знания  этих  фактов  немыслимы  были бы,  конечно,  те  поразительные
сооружения  Египта,  Вавилона  и  проч.,  которые  и  сейчас  вызывают
восхищение.  Но  «классическим   примером  образования  математических
теорий и становления математики  как науки является математика древней
Греции» (Рыбников,  1960, с.  25). Появление  теорий и  общих методов,
систематизация знаний --- вот что  характерно для Эллады. Вместе с тем
достигается  высокая  логическая  строгость  суждений,  рационализация
знания.  Этот   высокий  теоретический  уровень  эллинской   науки  не
имел  непосредственных  преемников.  «Мировая империя  римлян  в  ходе
завоевательных войн разрушила все научные  центры и не создала условий
для их восстановления и развития» (Рыбников, с. 71).

В  начале   нашей  эры  ученые  Александрийского   музея  были  лишены
государственной поддержки. Мы знаем, что значительно раньше величайший
математик Греции, Архимед, погиб,  защищая свой родной город Сиракузы,
а вместе с тем и эллинистический  мир и эллинскую культуру, от римских
варваров. Широко распространенная  легенда о мусульманском завоевателе
Омаре, как будто уничтожившем Александрийскую библиотеку (если там то,
чего нет в Коране, то эти книги вредны, если то же, что есть в Коране,
то они излишни, в обоих случаях их следует сжечь), как известно, давно
опровергнута,  так как  Омару  достались лишь  жалкие остатки  прежней
библиотеки. Также  неверно думать, что основной  удар Александрийскому
Музеуму нанесли  фанатические христиане, хотя  и они приложили  руку к
этому  делу. Основной  удар был  нанесен Римом,  «где науке  уделялось
мало  внимания, и  она совершенно  отсутствовала в  западноевропейских
королевствах  варваров» (Бернал,  1956, с.  9). Наследие  Греции вновь
вернулось на Восток,  откуда оно и пришло, а потом  снова вернулось на
Запад.

Эта  сторона эллинской  науки  всеми считается  положительной. Но  под
словом «теория» понимается часто не только противоположность ползучему
эмпиризму, но и противоположность «практике», технике, и здесь древняя
греческая наука характерна пренебрежением практикой. Это признают даже
представители  идеалистической  мысли,  например  профессор  Богомолов
(1928, с. 12): «В противоположность своим египетским учителям, Фалес и
Пифагор не  были заинтересованы  в непосредственных  приложениях своих
открытий. Принося,  по преданиям,  богам гекатомбу в  благодарность за
открытие  своей знаменитой  теоремы,  Пифагор был  полон энтузиазма  к
чистому знанию и  не спрашивал, чему это может  послужить на практике.
Такая  постановка вопроса,  которой  осталась  верна греческая  наука,
повела  к удивительным  последствиям:  в  течение нескольких  столетий
греки  неизмеримо   опередили  своих   учителей-египтян,  занимавшихся
геометрией  в продолжение  тысячелетий. Впоследствии  греческие ученые
применили свои теоретические достижения  к практическим нуждам и сразу
достигли замечательных результатов: достаточно упомянуть об их успехах
в геодезии  и астрономии, а  также об  открытиях Архимеда. Но  в общем
греческих ученых  можно упрекнуть в  другой крайности, а именно  --- в
полном  пренебрежении  к  тому, что  касалось  интересов  повседневной
жизни».

И  здесь  мы  опять  видим несоответствие  фактической  истории  науки
с  тем  положением,  которое  защищают  материалисты:  наука  родилась
под  влиянием потребностей  и  развивается по  мере возникновения  все
новых  потребностей.  Мы же  видим  скорее  отрицательную связь  между
развитием техники  и развитием  теоретической науки.  По грандиозности
сооружений  древняя Эллада  значительно уступала  и Египту,  и Римской
Империи. Это  в значительной  мере, конечно, было  связано с  тем, что
Древняя Греция не дала ни  одного крупного государства с деспотическим
централизованным управлением, которое смогло бы сосредоточить средства
на  выполнении  длительных  сооружений, требовавших  огромной  затраты
физического,  в  то время  рабского,  труда.  Даже такие  сравнительно
скромные  сооружения, как  ряд  зданий во  время  правления Перикла  в
Афинах,  шли  в значительной  степени  за  счет союзников  и  вызывали
обвинения  Перикла в  расточительстве.  Эллинские математики  работали
«впрок», и их работы, не  имевшие часто никакого прикладного значения,
были использованы значительно позже западноевропейской культурой.

\subsection{Достижения ионийской, пифагорейской и афинской школ}

\emph{Вклад  в  математику}.  Начнем   с  генеральной  линии  развития
эллинской  математики,  последовательности  четырех  школ:  ионийской,
пифагорейской, афинской и александрийской. Разумеется, коснемся только
главнейших достижений.

Достижения \emph{ионийской}  (милетской) школы  невелики и  в точности
трудноустановимы. Позднейший комментатор, Прокл, утверждает, что Фалес
доказал  несколько  геометрических  теорем: о  равенстве  вертикальных
углов, о равенстве углов при основании равнобедренного треугольника, о
том,  что диаметр  делит круг  пополам (Богомолов,  1928, с.  11). Эти
результаты  невелики, но  Богомолов  подчеркивает  важное различие  по
сравнению с  египетской математикой: они оторваны  от непосредственных
практических  предложений, не  имеют  прямой связи  с «землемерием»  и
высказаны в совершенно общем виде.

Подлинный прогресс в математике  связан с \emph{пифагорейской} школой.
Здесь вполне определился характер математики как чистой науки, которой
интересуются  независимо  от  ее  приложений,  поэтому  многие  ученые
считают Пифагора  родоначальником чистой математики (Бляшке,  1957, с.
112).  Пифагор  впервые  поднял  знамя  сплошной  математизации  наших
знаний. Пифагор из  Самоса (около 540 г.). Число  --- основное начало:
«число  есть  сущность  всех  вещей,  и  организация  Вселенной  в  ее
определениях представляет  собою вообще гармоническую систему  чисел и
их  отношений...».  Подобно  тому  как  число  подчинено  определенным
законам,  так подчинена  им  и Вселенная;  этим впервые  высказывается
мысль  о закономерности  Вселенной.  «Пифагору приписывается  сведение
музыкальной  гармонии   к  математическим  отношениям»   (Ф.  Энгельс:
Воззрение   древних  на   природу.   Диалектика   природы,  1949,   с.
148).  Пифагору  приписывают  теорему,  носящую его  имя,  и  открытие
иррациональных  чисел. Сейчас  многие  оспаривают принадлежность  всех
этих  открытий  Пифагору  и, сообразно  очень  распространенной  моде,
доказывают, что  Пифагор вообще никогда не  существовал. Этого вопроса
придется  немного  коснуться.  Для  нас   сейчас  это  не  так  важно.
Существенно  то, что  школа,  носившая имя  Пифагора, сделала  великие
открытия в области математики.

Кажется,  никто не  сомневается в  реальности существования  правителя
Тарента,  пифагорейца   Архита,  друга  Платона.   Архиту  приписывают
решение  задачи  удвоения  куба  методом  пространственных  (объемных)
геометрических мест (Лурье,  1945, с. 35), он же  развил дальше теорию
иррациональных чисел.  Благодаря связям Архита с  Евдоксом и Платоном,
математика  была  перенесена  в  Афины, где  обосновалась  в  Академии
Платона (Бляшке,  с. 114).  По-видимому, уже  пифагорейцам принадлежит
открытие правильных  многогранников, теория которых  была окончательно
развита  в  школе  Платона,  отчего   они  и  называются  до  сих  пор
Платоновыми телами.

В  платоновской Академии  была проделана  огромная работа  по развитию
геометрии, закончившаяся «Началами» Евклида.  Величие Евклида ясно уже
из того,  что он, как  и многие другие выдающиеся  мыслители прошлого,
некоторыми  галертерами  считается  даже мифической  фигурой  (Бляшке,
1957,  с. 115);  банальные деятели  такой чести  не удостаиваются.  По
свидетельству позднейшего комментатора,  Прокла, «Начала» основываются
на совместной работе круга  геометров из Академии Платона, проделанной
за  период  между  370  и  350  гг.  Прокл  говорит:  «Составляя  свои
``Элементы'', Евклид  собрал многие теоремы Евдокса,  завершил то, что
начал  Теэтет,  и  дал  строгие доказательства  тому,  что  нашли  его
предшественники»  (Бляшке, с.  116).  В  «Началах» впервые  появляется
«аксиоматический метод», окончательное  завершение получивший только в
XX веке. Основная деятельность Евклида  протекает уже в Александрии, и
мы знаем, что  составленная им геометрия была сделана  так хорошо, что
ее использовали в качестве учебника вплоть  до XIX века. Евдокс же дал
общую теорию пропорций,  он же предложил метод  исчерпывания на основе
аксиомы Евдокса  (иначе аксиомы Архимеда)  и применил ее  к вычислению
объемов пирамиды и других тел (БСЭ, 2-е изд., 1952, статья: Евдокс).

Теэтет  был видным  деятелем  платоновской  Академии. Ему,  смертельно
раненному в  сражении при  Коринфе, посвящен  один из  важных диалогов
Платона,  начинающийся с  описания обстоятельств  его смерти.  Теэтету
принадлежит  строгое  доказательство   существования  пяти  правильных
многогранников,  Платоновых   тел.  Сейчас   мы  привыкли   к  системе
правильных многогранников и она нам кажется совсем простой, но вот как
судит об этом  открытии один из крупных  современных математиков Вейль
(1952,  с.  74): «Открытие  последних  двух  (икосаэдра и  додекаэдра)
является, несомненно, одним из самых  красивых и особенных открытий во
всей истории математики».

В  платоновской  Академии родилось  и  учение  о конических  сечениях.
Начало  положил друг  Платона  Менехм;  знаменитый ученый  Александрии
Эратосфен называет три основных  конических сечения «триадой Менехма».
Потом  работал  Аристей  (пять  книг об  «Объемных  местах»),  Евклид,
написавший «Конические  сечения», затем  Архимед, и завершил  работу в
этом направлении ученый александрийской  школы Аполлоний (Лурье, 1945,
с. 36).

Наконец,  несомненно,  что  в  платоновской  Академии  разрабатывалась
теория  чисел,  получившая  затем  развитие  также  в  Александрии.  В
сочинениях  Платона  есть  упоминание о  так  называемом  «совершенном
числе»  (числе,  которое  равно  сумме своих  множителей).  У  Евклида
часть  его  «Основ»  посвящена  теории чисел,  где  он  дает  теорему,
позволяющую  на  основе  суммы  степеней  двух  определять,  будет  ли
число  совершенным  или  нет,  причем  вопрос,  исчерпывают  ли  числа
данного вида  все множество  совершенных чисел, остается  нерешенным и
в  наше  время  (Рыбников,  с.  45). У  Платона  в  его  «Государстве»
совершенное  число  упоминается в  связи  с  так называемым  «брачным»
числом. Радемахер  и Теплиц (с.  143, 149, 150) указывают,  что теория
совершенных чисел  сейчас рассматривается  как курьез,  но в  ней есть
искорка, которую  удалось воспламенить  впоследствии Эйлеру  и которая
сверкает  ныне пламенем  одного из  наиболее ярких  учений современной
математики,  именно, теории  распределения  простых  чисел. Из  одного
малоисследованного  места  в книге  V  «Законов»  Платона Радемахер  и
Теплиц заключают, что в этой области Платону и другим математикам было
известно больше того, что приведено у Евклида.

\subsection{Александрийская школа}

Мы  видим,  таким  образом,  что в  Афинах,  в  платоновской  Академии
были   заложены  основы   всех   тех   отраслей  математики,   которые
получили  затем пышное  развитие в  Александрии. Главнейшими  фигурами
\emph{александрийской}  школы  являются: Евклид,  Архимед,  Эратосфен,
Аполлоний и  Диофант. Некоторые  из них работали  большей частью  не в
Александрии  (Архимед  ---  в  Сиракузах, Аполлоний  ---  в  Пергаме),
но  все  они  получили  образование в  Александрии.  Об  Евклиде  было
сказано уже  выше, да имя  это достаточно хорошо  известно. Достаточно
упомянуть, что учение о целых числах  и их отношениях взято в основном
из  пифагорейской математики  (Рыбников, с.  44). Рыбников  указывает,
что  известные  недостатки  «Начал»   Евклида  возникли  под  влиянием
ограничительных тенденций идеалистической философии.  Однако на той же
странице 47 он пишет: «В  течение всей многовековой истории математики
``Начала''  являлись фундаментом  всех геометрических  изысканий. Даже
решающее  изменение  всей  системы геометрии,  вызванное  введением  в
начале  XIX века  в работах  Н.И.Лобачевского неевклидовой  геометрии,
в   значительной  степени   связано  с   попытками  усовершенствования
``Начал''».  Он же  прибавляет,  что знакомство  с «Началами»  Евклида
полезно всякому математику и в наши дни.

Об  Архимеде вряд  ли нужно  подробно говорить,  так как  величие этой
фигуры  не оспаривается,  насколько  мне  известно, решительно  никем.
Кроме  того,  как  хорошо  известно, он  был  не  только  математиком,
но  проложил  дорогу теоретической  механике  и  физике. Огромна  роль
его  и  в  области  методов,  о чем  речь  будет  дальше.  Вкратце,  в
области  чистой математики  ему принадлежат  следующие достижения:  1)
вслед  за  системой  правильных  многогранников  он  построил  систему
полуправильных  многогранников,  так  называемых Архимедовых  тел;  2)
определил площадь  и объемы многих тел  и показал, что в  ряде случаев
(сегмент параболы, некоторые тела)  они выражаются точно рациональными
числами;  3)  сделал очень  много  в  теории конических  сечений  (как
было указано  выше, некоторые  утверждали, что  содержание «Конических
сечений»  Аполлония принадлежит  в основном  Архимеду); 4)  сделал шаг
в  построении  десятичной  системы  чисел (Рыбников,  с.  68).  Однако
приходится удивляться, что этот мощный  ум не сделал решительного шага
по  установлению позиционной  системы счисления.  Философских взглядов
Архимед  в своих  сочинениях, видимо,  нигде не  высказывал, относясь,
очевидно,  к  числу  тех  математиков,  которых  близко  философия  не
интересует.

В  противоположность  Архимеду,  его  современник  и  друг,  Эратосфен
(одно  из  очень важных  сохранившихся  сочинений  Архимеда имеет  вид
письма к  Эратосфену) не  скрывает своего  уважения к  Платону (Лурье,
1945, с. 51--52); главное  программное сочинение Эратосфена называется
«Платоник», и он получил прозвище  «второй Платон» или «новый Платон».
Как  математик,  Эратосфен   знаменит  своим  «Эратосфеновым  решетом»
(способ  составления таблицы  простых  чисел),  работой по  коническим
сечениям и  нахождению одной,  двух и более  средних пропорциональных,
при  помощи которых  решались  знаменитые задачи  об  удвоении куба  и
трисекции угла.  Несмотря на  дружбу с Архимедом,  по ряду  вопросов у
них  были и  расхождения, что  характерно  для всех  тех случаев,  где
культивируется действительно свободная наука.  Эратосфен был не только
математиком. Он был чрезвычайно разносторонним ученым и сделал крупный
вклад в астрономию.

Аполлоний  знаменит  своими  «Коническими сечениями».  Это  сочинение,
завершившее работу  эллинских математиков  по этому  вопросу, усиленно
изучалось математиками после нового расцвета наук. Достаточно сказать,
что  из  восьми  книг  этого  сочинения  до  нас  дошли  первые  семь.
Предполагаемое   же  содержание   восьмой  книги   было  восстановлено
знаменитым  астрономом  Галлеем  (1656--1742),  исходя  из  содержания
первых  семи  книг  и  сведений,  сообщенных  комментатором  Аполлония
(Рыбников, с. 64). Работа Аполлония была столь законченной, что спустя
почти  2000 лет  Кеплер  и  Ньютон смогли  ее  использовать почти  без
изменений для выявления  свойств планетных орбит (Бернал,  с. 127); из
положений  Аполлония  исходили  при создании  аналитической  геометрии
Декарт и Ферма (БСЭ, 2-е изд., т. 2, с. 557).

Последним  крупным  математиком  александрийской  школы  был  Диофант,
работавший в III веке н.э., когда Александрия уже была под пятой Рима.
Ему принадлежит книга о многоугольных  числах --- понятие, возникшее в
пифагорейской математике  (Рыбников, с. 73). Работы  Диофанта в теории
чисел  были отправной  точкой  для работ  великих  ученых: Ферма  (так
называемое  «великое предложение  Ферма»  сформулировано  им на  полях
сочинения Диофанта), Эйлера, Гаусса и др. (статья «Диофант» в БСЭ, 2-е
изд., т. 14, с. 399). Наконец, Диофант сделал важный шаг в переходе от
так называемой риторической алгебры  к символической, вводя сокращения
выражений («синкопическая» алгебра) (Рыбников, с. 74.)

Упомянем  Никомеда  (II-й  век  до н.э.),  построившего  конхоиду  для
решения  задачи  трисекции угла,  и  Герона  (I-II вв.  н.э.)  давшего
практические приемы вычисления (Рыбников, с. 71, 73.)

Для  завершения пифагорейской  линии следует  упомянуть еще  Никомаха,
неопифагорейца.   Его  «Введением   в  арифметику»   пользовались  как
учебником арифметики во все Средние  века и даже некоторое время после
Возрождения.

Сомнительное   положение  в   смысле  отнесения   к  пифагорейской   и
платоновской  линии  занимает  математик  Феодор  (Теодор)  из  Кирены
(Северная  Африка,  нынешняя  Ливия).  Он  установил  иррациональность
квадратного корня из ряда чисел  (Рыбников, с. 28). Согласно преданию,
имеющемуся у  Диогена Лаэрция  (II, 8, 103),  Платон, во  время своего
путешествия  после  казни  Сократа, занимался  у  Феодора  математикой
(цитирую по Серебренникову, 1936, с. 172). Если принять в соображение,
что иррациональные  числа разрабатывались  только по  «линии Платона»,
что Феодор  играл роль  в математическом  образовании Платона,  то мы,
пожалуй, не  ошибемся, если  отнесем его тоже  к линии  Платона. Может
быть, не  случайно и то,  что несомненный платоник Эратосфен  был тоже
родом из Кирены.

\subsection{Достижения Демокрита и его линии}

Теперь  перейдем  к  рассмотрению,  что  же  дала  линия  Демокрита  в
математике.  Начнем  с  самого Демокрита.  Демокриту  принадлежит  ряд
математических  сочинений: «О  касании круга  и шара»,  «О геометрии»,
«Числа»,  «Об иррациональных  линиях  и телах»,  но  эти сочинения  не
сохранились (История  философии, т. I,  с. 117). Поэтому  очень трудно
судить, что именно сделано  Демокритом. Ему приписывают ряд достижений
в  стереометрии,  в частности  определение  объема  пирамиды и  конуса
и,  может  быть, объема  шара.  Роль  Демокрита в  этих  исследованиях
засвидетельствована  Архимедом  (Лурье,  1945,   с.  138).  Филопон  в
своих  комментариях  сообщает, что  Демокрит  доказывал,  что из  всех
многогранников  одинакового объема  наименьшую  поверхность имеет  шар
(Лурье,  1945, с.  132). Является  ли  открытие формул  объема шара  и
пирамиды  оригинальным достижением  Демокрита,  мы не  знаем, так  как
возможно, что эти формулы были известны уже древним египтянам, которые
получили  их  либо из  опыта,  либо  путем примитивной  математической
логики  (Лурье,  1947, с.  174).  По  мнению  С. Я.  Лурье,  Демокрит,
вероятно, ограничился тем,  что подвел под эти  формулы более солидную
научную базу, а  главное, он выковал прекрасное  орудие для нахождения
новых  математических истин.  Это  главное  --- метод  интегрирования,
которым  впоследствии  широко  пользовался  Архимед и  о  котором  нам
придется поговорить в разделе, посвященном методическим достижениям.

Непосредственных учеников  в области математики у  Демокрита как будто
не было,  а в дальнейшем «линия  Демокрита» в лице Эпикура  и Лукреция
совершенно  оторвалась   от  математики;  видимо,   на  атомистических
позициях,  близких   Демокриту,  стоял  крупный   математик  Гиппократ
Хиосский (середина V века до н.э.), не следует смешивать с основателем
медицины, Гиппократом  с о.  Кос. Он достиг  первого успеха  в решении
задачи  об удвоении  куба, эта  задача получила  окончательное решение
только  через  200 лет  (Рыбников,  с.  31,  32). Ему  же  принадлежит
первая сводка основных  математических знаний, о которой  до нас дошли
сведения, так называемые «Начала». Это сочинение, как и многие другие,
принадлежащие  другим авторам,  оказалось  забытым  и утерялось  после
того, как появились «Начала» Евклида (Рыбников, с. 40).

Как указывает С. Я. Лурье,  Гиппократ, стоя, видимо, на атомистических
позициях   (а   атомистическая   математика   отрицала   существование
несоизмеримых величин),  пытался доказать соизмеримость  любых величин
(Лурье,  1947, с.  329). На  этом  пути он  достиг известных  успехов,
открыв   известные  гиппократовы   луночки,  вполне   квадрируемые.  В
античности Архимедом  была дана  точная квадратура параболы.  Есть еще
некоторые  квадрируемые площади,  ограниченные кривыми  линиями. Но  в
целом, конечно, Гиппократ, как и все атомистические математики, ставил
перед  собой  неосуществимую задачу.  Они  не  могли принять  открытия
иррациональности,  а  тем самым  отрезали  себе  путь к  плодотворному
развитию математики.

Внес вклад в математику и знаменитый софист Гиппий из Элиды, о котором
Платон  рассказывает в  трех  своих  диалогах. Это  тоже  был один  из
энциклопедических умов Древней Греции. К  какой «линии» его отнести, к
платоновской или  демокритовской, сказать трудно. Гиппий  применил для
решения задачи  трисекции угла трансцендентную кривую  --- квадратрису
(Рыбников, с. 33).

Вот   обзор  в   самых  кратких   чертах  достижений   древнегреческой
математики. При этом  обзоре не касались методов решения  задач, о чем
будет речь впереди. Из обзора видно, что достижения линии Демокрита не
идут  ни  в  какое  сравнение  с  основным  направлением  в  эллинской
математике, стоявшим  целиком на линии Пифагора  --- Платона. Перейдем
теперь к вопросам методики.

\subsection{Метод исчерпывания}

В  книге  Рыбникова  пятая  лекция  посвящена  \emph{инфинитезимальным
методам} в античной Греции  и математическому творчеству Архимеда. Ряд
проблем требовал  для своего решения исследовать  предельные переходы,
бесконечные  процессы, непрерывность  и т.д.  Огромное значение  имело
обнаружение несоизмеримости величин.

Некоторые группы ученых искали выход  из этих затруднений в применении
к математике  атомистических философских  воззрений, в  первую очередь
школа  Демокрита. Рыбников  указывает  (с. 48),  что о  математической
стороне подобных высказываний известно слишком мало. Видимо, Рыбникову
неизвестны  работы С.  Я. Лурье.  Но известны  те возражения,  которые
были выдвинуты против  атомистического \emph{метода неделимых}, именно
знаменитые апории  Зенона Элейского.  Хорошо известны эти  апории (см.
Рыбников, с.  49): 1)  дихотомия: невозможно осуществить  движение; 2)
Ахиллес  не догонит  черепаху;  3) полет  стрелы  невозможен. Все  эти
выводы  показывают,  к  чему  приводят  попытки  получать  непрерывные
величины из  бесконечного множества  бесконечно малых  частиц. «Апории
Зенона убедительно  показали, что,  если искать  точные доказательства
и   логически  исчерпывающие   решения   задач,  нельзя   пользоваться
бесконечностью,  опираясь   на  наивные   атомистические  соображения.
Для  подобных  целей  необходимо разрабатывать  и  привлекать  методы,
содержащие  наряду  с  разновидностями  сведений  о  бесконечно  малых
элементы предельного перехода» (Рыбников, с. 49).

Одним  из   самых  ранних  методов  такого   рода  явился  \emph{метод
исчерпывания}. Изобретение его приписывается ученику Платона, Евдоксу,
наиболее широкое  развитие он получил  у Архимеда. Не  следует думать,
что метод исчерпывания был предпочтен по сравнению с методом неделимых
только из  уважения к  апориям Зенона. Метод  неделимых был  не только
недостаточно строг,  но при  неосторожном пользовании мог  приводить к
грубым  ошибкам.  Один из  таких  примеров  приводит С.Я.Лурье  (1945,
с.  21).  Треугольник  состоит  из  тесно  приложенных  друг  к  другу
прямых, параллельных одному из  катетов. Каждая такая прямая пересечет
гипотенузу  и другой  катет  в  одной точке.  А  так  как число  точек
на  этом  катете  и  гипотенузе одинаково,  то,  значит,  катет  равен
гипотенузе.  Метод исчерпывания  был  лишен этих  недостатков. Он  был
и  безупречно  строг,  и  гарантировал   от  ошибок,  и  применялся  в
некоторой стандартной  форме. Например, весьма  изящное доказательство
квадратуры  параболы   проходит  такие   этапы.  В   сегмент  параболы
вписывается треугольник, потом в сегменты между сторонами треугольника
и отрезками  парабол вписываются  новые треугольники, и  эта процедура
проделывается неограниченно  долго. Доказывается из  формулы параболы,
что  мы получаем  сумму ряда  1+1/4 +  1/42+1/43 и  т.д., что  в сумме
дает  4/3. Это  получается на  основе доказательства,  что приведенная
последовательность действительно «исчерпывает» параболический сегмент,
и   доказательством  от   противного   о  единственности   полученного
результата.

Логическая строгость метода  исчерпывания оставалась непревзойденной в
течение многих  веков, но  форма его оставалась  весьма несовершенной:
метод  развивался  только  в  связи  с  конкретными  задачами,  он  не
приобрел  вид  абстрактного  метода, единственность  доказывалась  для
всякой задачи  заново. Это приводило, конечно,  к большой громоздкости
всякого  доказательства.  Но  устранение  этих  недостатков  требовало
преодоления таких  трудностей, которые  древние не могли  преодолеть и
которые  были преодолены  только через  несколько веков  (Рыбников, с.
52).

\subsection{Механические методы, метод интегральных сумм}

Метод  исчерпывания был  чисто  геометрическим методом,  и им  Архимед
пользовался  для окончательного  строгого доказательства  найденных им
результатов. Но  только в двадцатом веке  окончательно выяснилось, что
для отыскания  решений Архимед  пользовался иными, менее  строгими, но
более  легкими  методами. Это  произошло  после  находки в  1906  году
сочинения  Архимеда «Послание  к Эратосфену».  Архимед, как  известно,
много  работал  по механике  и  у  него \emph{механические}  приемы  и
аналогии  проникли и  в математические  методы. Для  вычисления объема
шара он  пользуется механической интерпретацией, основанной  на законе
рычага (Рыбников,  с. 53), на  этом же  методе основан и  другой метод
получения  квадратуры параболы,  который был  потом переведен  на язык
метода  исчерпывания с  обязательным  завершением  в каждом  отдельном
случае, доказательством от противного.

Следующей  разновидностью  инфинитезимальных  методов  является  метод
\emph{интегральных сумм} (Рыбников, с. 54), применявшийся в сочинениях
Архимеда: «О шаре и цилиндре»,  «О спиралях», «О коноидах и сфероидах»
(Рыбников, с. 54--58). Исследуемое тело или поверхность разбивается на
части и  каждая часть аппроксимируется описанными  и вписанными телами
или  кривыми.  Аппроксимирующие  сверху  и снизу  тела  и  поверхности
выбираются  так, чтобы  разность объемов  или поверхностей  могла быть
сделана  сколько  угодно малой.  Вычисление  суммы  ряда дает  искомый
результат. Этот прием Архимед  применял, например, к вычислению объема
эллипсоида  вращения  или  площади  первого  витка  спирали  Архимеда.
Метод  чрезвычайно   схож  с  методом   определенного  интегрирования,
но  он  применялся  индивидуально   для  каждой  конкретной  задачи  и
общетеоретические основы не были оформлены.

Наконец, у того же Архимеда  мы находим методы, которые ретроспективно
могут быть охарактеризованы  как \emph{дифференциальные} (Рыбников, с.
58), например метод нахождения касательной к спирали.

В инфинитезимальных  методах получили первое выражение  элементы новых
математических средств, приведших к созданию анализа бесконечно малых.
Они послужили исходным пунктом  многих исследований ученых математиков
XVI  и  XVII  веков.   Особенно  часто  подвергались  изучению  методы
Архимеда.  Лейбниц  по этому  поводу  писал:  «Изучая труды  Архимеда,
перестаешь удивляться  успехам современных математиков»  (Рыбников, с.
61).

Можно  подумать, что  Архимед  сделал так  много  по пути  обоснования
анализа бесконечно  малых, что для завершения  этой отрасли математики
остались только доделки. Это совершенно неверно. Задача окончательного
(или,  скажем  осторожнее,   удовлетворительного)  построения  анализа
бесконечно малых настолько трудна, что для завершения ее потребовались
ряд  столетий и  напряженная работа  ряда выдающихся  математиков. Эта
история  также изложена  в книге  Рыбникова.  Выберем из  нее то,  что
интересно  даже для  нематематиков. Кеплер  тщательно изучал  творения
Архимеда,  но  вместе  с  тем  старался  разгадать  замысел  Архимеда,
приведший  его   к  столь   поразительным  результатам,   и  догадался
(теперь  мы  видим,  что   догадка  Кеплера  была  справедливой),  что
этот  метод  состоял  в  разложении   фигуры  или  тела  на  множество
бесконечно  малых частей.  Пренебрегая  абсолютной строгостью,  Кеплер
этим  путем  вычислил  объем  92   тел  вращения  (Рыбников,  с.  154,
155).  Ослабление  строгости  метода   вызвало  резкие  возражения,  и
ученик  основоположника  символической  алгебры,  Виета,  А.  Андерсон
выпустил даже  специальное сочинение «В защиту  Архимеда», где обвинял
Кеплера в  оскорблении памяти Архимеда.  Но эта критика  не остановила
ученых,  и  дальнейший  крупный   шаг  был  сделал  его  \emph{методом
неделимых}  учеником  Г.Галилея,   Бонавентурой  Кавальери.  Кавальери
с  1629   года  по  рекомендации  Галилея   занял  кафедру  математики
в  Болонье,  будучи   по  совместительству  настоятелем  католического
монастыря  ордена иеронимитов  (Рыбников, с.  157). Совокупность  всех
неделимых по существу вводит понятие определенного интеграла. У метода
появилось  много приверженцев,  в  частности  известный Торичелли.  Но
работа  все-таки была  не  завершена, и  только  после работ  Паскаля,
Роберваля, Ферма, Декарта (с его методом неопределенных, см. Карно, с.
221),  Валлиса наступило  время, когда  Ньютон и  Лейбниц дали  первый
синтез  анализа бесконечно  малых. И  до них  методы интегрирования  к
60-м  годам  XVII века  охватывали  обширные  классы алгебраических  и
тригонометрических функций  и было  решено огромное  количество задач,
но  методы интегрирования  развивались независимо  от дифференциальных
методов  и  необходимо  было   установить  связь  и  взаимообратимость
дифференциальных и интегральных исследований.

\subsection{Роль Ньютона и Лейбницa}

Но значит, Ньютон и Лейбниц завершили синтез анализа бесконечно малых?
Сами Ньютон и Лейбниц так не  думали. Как и все великие мыслители, они
понимали  ясно крупные  несовершенства своих  построений. Это  ясно из
ряда  фактов,  прекрасно  изложенных в  книге  Рыбникова.  Большинство
результатов  своей теории  флюксий Ньютон  получил в  течение 60--70-х
годов  XVII  века.  В  1676--1677 году  Лейбниц  завязал  переписку  с
Ньютоном, где оба сообщали о  своих результатах и хорошо понимали друг
друга.  Переписка  прекратилась,  так  как  Ньютон  перестал  отвечать
на  письма.  Как  будто  забота  о  приоритете  должна  была  побудить
обоих ученых  к скорейшей  публикации своих  результатов (а  мы знаем,
что  в   дальнейшем  этот  спор  разгорелся   и  составляет  печальную
страницу в  истории науки), однако  Лейбниц первый мемуар всего  на 10
страницах  опубликовал  только  в  1684  году,  а  Ньютон  еще  позже,
хотя,  видимо,  Ньютон  добился   основного  успеха  раньше  Лейбница.
Мало того,  знаменитые «Математические начала  натуральной философии»,
появившиеся в 1686--1687 гг. оказались написаны без применения методов
теории флюксий,  хотя многие из  приведенных в этой  книге результатов
первоначально были получены средствами  этой теории. Получилось полное
повторение поведения  Архимеда: эвристический метод,  как недостаточно
обоснованный, в конечном изложении  был заменен другим, более строгим.
Проблема обоснования  анализа бесконечно  малых оказалась не  под силу
Лейбницу, как и Ньютону. «В области обоснования новый анализ в течение
XVII в. и в значительной  части XVIII в. переживал ``мистический'', по
меткому выражению К. Маркса, период» (Рыбников, с. 184).

Совершенно ясно,  что сам Ньютон считал  употребление бесконечно малых
чисто  эвристическим  приемом,  лесами,  которые  должны  быть  убраны
по  окончании  постройки  (А.  П.  Юшкевич,  1936,  с.  28).  В  своих
«Математических  началах»  он   использует  и  терминологию  Кавальери
и  считает,  что  метод  неделимых  менее  геометричен  и  поэтому  он
предпочитает  свести  доказательство  к   методу  первых  и  последних
отношений. Он защищает применение предельных отношений в поучении к II
лемме кн. I, разд. I. «Можно возразить, что если существуют предельные
отношения исчезающих  количеств, то  существуют и  предельные величины
их  самих  и,  следовательно,  всякое количество  должно  состоять  из
неделимых,  что  опровергнуто  Евклидом  в  десятой  книге  элементов,
в  учении   о  несоизмеримых  величинах».  Дальше   Ньютон  разъясняет
правомерность  предельных отношений.  Ньютон,  таким образом,  целиком
стоит  на  строгой  позиции  древних   математиков  и  не  забывает  о
недостаточной строгости нового метода.

\subsection{Независимость Архимеда от Демокрита}

Строгое   обоснование  анализа   бесконечно   малых  потребовало   еще
длительной работы, вкратце хорошо изложенной, например, в статье А. П.
Юшкевича.  Как  известно, очень  многие  крупные  ученые не  принимали
нового  метода,   например  Гюйгенс.  С   теоретическими  возражениями
выступил знаменитый философ Д. Беркли, выпустив памфлет «Аналист». Как
часто бывает,  умные и  образованные противники  способствуют развитию
нового учения.  Повторилась история с  Зеноном Элейским. Ф.  Кеджори в
своей  истории  вопроса,  сравнивая  «Аналист» с  бомбой,  попавшей  в
математический  стан,  расценивает  его как  выдающееся  произведение,
«явившееся  поворотным  пунктом  в истории  британской  математической
мысли» (Юшкевич,  с. 41). Вместе с  тем одна из идей  Беркли послужила
одним  из принципов  обоснования исчисления  бесконечно малых  в XVIII
в.  Но  как  же  Беркли   объясняет  то,  что  новый  анализ  получает
правильные результаты?  Он вводит идею  компенсирующихся погрешностей.
Это объяснение  приняли и  многие ревностные защитники  нового метода,
например Лагранж  и Карно.  «Анализ есть не  что иное,  как исчисление
компенсирующихся погрешностей» (см. Юшкевич, с. 72). Другие математики
для  защиты  от  критики  Беркли выпускали  сочинения  с  целью  более
строгого  обоснования  метода.  Таким был,  например,  фундаментальный
«Трактат  о флюксиях»  Маклорена. Но  все эти  работы не  дали полного
обоснования анализа. Юшкевич заканчивает  свою статью словами (с. 76):
«XIX  в., как  известно,  пошел в  другом  направлении. Не  побоявшись
объединить идеи  обеих школ,  предел и алгоритм  исчисления бесконечно
малых,  Копти и  иные ученые  создали  то стройное  здание анализа,  в
котором  новые логические  трещины появились  лишь много  десятилетий,
чуть ли не  век, спустя --- почти на наших  глазах» (видимо, имеется в
виду  Вейерштрасс и  другие математики).  Длительный процесс  создания
исчисления  бесконечно малых  ведет от  Евдокса и  Архимеда к  Ньютону
и  Лейбницу.   Вся  эта  линия  связана   с  платоновско-пифагорейским
направлением, без всякого влияния линии Демокрита. Как указывает С. Я.
Лурье,  Архимед упоминает  Демокрита  и признает  его  заслугу в  деле
вычисления объемов  некоторых тел,  но несомненно, к  этому результату
и  притом со  строгим  доказательством  Архимед пришел  самостоятельно
(Лурье, 1945,  с. 138)  и ознакомился с  сочинениями Демокрита  уже по
возвращении в Сиракузы из Александрии (там  же). По мнению С. Я. Лурье
(там  же),  «обнаружив  в Сиракузах  математические  труды  Демокрита,
Архимед, несомненно, с  жадностью набросился на них. В  самом деле, он
оказался  здесь  у  истоков того  ``атомистического''  интегрирования,
которое  ему  с  трудом  и по  частям  приходилось  реставрировать  из
отдельных  намеков и  приемов  в трудах  по  механике, написанных  его
предшественниками».

Здесь  мы имеем  два утверждения:  1) Архимед  признает, что  Демокрит
впервые  определил объем  конуса и  пирамиды,  но дал  этот объем  без
строгого  доказательства;  Архимед,  однако,  дошел  до  этих  объемов
самостоятельно;  2)  обнаружив  это  обстоятельство,  Архимед  стал  с
жадностью  изучать труды  Демокрита  для ознакомления  с его  методом.
В  работах  самого Лурье  можно  найти  высказывания для  того,  чтобы
показать, что  ни то  ни другое положение  не верно.  Коснемся сначала
первого  вопроса, о  приоритете  Демокрита.  В п.  5  я уже  указывал,
что  сам Лурье  не уверен,  что  определение объемов  пирамиды и  шара
является  оригинальным  достижением  Демокрита, а  возможно,  что  оба
объема были известны  уже египтянам (Лурье, 1947,  с. 174). Совершенно
несомненно,  что  и Платон,  и  Демокрит  были знакомы  с  математикой
египтян,  Архимед же,  видимо, историей  математики не  интересовался.
Александрийская  школа  уже  так  далеко  ушла  от  египетской  науки,
что  большинство ученых,  вероятно  (как  это свойственно  большинству
ученых во  все времена), интересовалось только  наукой своих ближайших
предшественников. В основном  платоновской Академии и аристотелевского
Ликея.  Доводы   Лурье,  что  Архимед  заимствовал   свою  методику  у
Демокрита, крайне неубедительны. Для полноты картины разберу их. На с.
146 книги об Архимеде (Лурье, 1945)  Лурье упоминает о двух задачах по
определению объема  тел: 1)  образованного двумя цилиндрами  с взаимно
перпендикулярными осями  и 2)  части цилиндра,  отсеченной плоскостью,
проходящей  через ребро  верхнего основания  описанной призмы  и через
центр нижнего  основания. Обе  задачи интересны тем,  что представляют
стереометрическую параллель  (по рациональности  объема) гиппократовым
луночкам и квадратуре параболы. То  обстоятельство, что вторая из этих
задач решена  путем неделимых в  частном виде, без  всякого применения
механики (закона рычага), доказывает, что мы тут имеем дело с приемом,
прямо заимствованным  у Демокрита. Лурье считает  отнюдь не случайным,
что этот прием появился в  сочинении, которое открывается указанием на
заслуги Демокрита. Но ведь в чем  точно состоял метод Демокрита, мы не
знаем, и  сам Лурье,  как было  уже указано, не  уверен в  деталях его
методики. Архимед  же настолько мощный  ум и разнообразие  его методов
(из коих  большинство заведомо  и не  связано с  Демокритом) настолько
велико,  что мы  имеем  полное право  допустить здесь  самостоятельное
творчество Архимеда, пока не доказано противное.

\subsection{О терминологии Архимеда}

Другой факт в пользу того, что Архимед многое заимствовал у Демокрита,
чисто  терминологический   (Лурье,  1945,   с.  163).   Как  известно,
современная   терминология  конических   сечений  (эллипс,   парабола,
гипербола) ведет  начало от Аполлония, которого  Архимед не упоминает:
видимо, отношения  между ними  были не  из приятельских.  Могла играть
роль  здесь  и  различная  политическая  ориентация:  Аполлоний,  хотя
и  получил  образование  в  Александрии   и  как  ученый  относится  к
александрийской  школе, работал  в конкурирующей  с Музеем  пергамской
научной школе, ориентировавшейся на Рим  (Лурье, 1945, с. 44); Архимед
же, как прекрасно показано тем же Лурье, был ярым противником Рима.

В  замечательном  сочинении Архимеда  «О  коноидах  и сфероидах»,  где
он,  видимо,   был  пионером,  идет   речь  о  телах,   полученных  от
вращения  сегментов   конических  сечений  вокруг  оси.   То,  что  мы
называем  теперь параболоидом  вращения, Армед  называл «прямоугольным
коноидом», гиперболоид вращения --- «тупоугольным коноидом». Эллипсоид
же  вращения  он не  называл  (в  том  же  духе, как  остальные  тела)
«остроугольным коноидом», а «сфероидом»,  причем различал два их вида,
«удлиненный» и  «сплющенный» сфероиды (сейчас их  называют вытянутый и
сжатый эллипсоиды вращения).  Различие этих двух видов  связано с тем,
вращают ли эллипс вокруг большой или малой оси. В названиях «сфероиды»
Лурье  видит прямое  влияние  атомистов,  так как  с  их точки  зрения
эллипс рассматривался как  круг, в котором каждая  из составляющих его
ординат уменьшена в  одном и том же  отношении. Непоследовательность в
терминологии  Лурье  объяснял  тем, что  названия  «сфероиды»  Архимед
придумал (или  усвоил у  предшественников) еще  в раннюю  эпоху своего
творчества, а  затем уже не хотел  их менять в угоду  стройности своей
системы.  Но ведь  в ранний  период свой  деятельности, как  указывает
сам  Лурье,  Архимед не  знал  Демокрита,  следовательно, нет  никаких
оснований думать, что он эти термины  усвоил от него. Очевидно, в ходу
был  некоторый  запас  обезличенных математических  сведений,  имевших
и  более  раннее происхождение.  Подтверждение  этому  можно видеть  в
работе Лурье  «Три этюда  к Архимеду»  (1956). На  с. 16  указано, что
уже  древние египтяне  умели находить  с хорошим  приближением площадь
эллипса, рассматривая  ее как «тень» (параллельную  проекцию) круга, и
получали площадь эллипса как  площадь других теней. Достижение периода
расцвета  атомистической  математики,  что эллипс  ---  косое  сечение
цилиндра, есть  простой пересказ другими  словами того, что  знали уже
египтяне. По сравнению с теми  методами, которыми пользовались в школе
Платона  и в  александрийской,  они настолько  проще,  что нет  ничего
удивительного, что Архимед сам до  них додумался. Может быть, конечно,
«Архимед  восстановил  в  правах  старый, ненаучный,  но  наглядный  и
удобный  атомистический  метод  интегрирования, но  только  как  метод
нахождения решений, правильность которых для каждого отдельного случая
должна была затем доказываться строго геометрическим способом» (Лурье,
1956, с.  17). Аксиомы  атомистической математики  (имевшие, очевидно,
додемокритовское  происхождение) уже  во времена  Платона были  прочно
опровергнуты, хотя среди греков, малознакомых с математикой, они могли
еще  иметь хождение,  на  что указывает  приведенная  Лурье цитата  из
«Законов» Платона  (VII, 21,  р. 820  АВ) (Лурье,  1956, с.  21): «Что
касается отношения линий и площадей к  телам или площадей и линий друг
к другу, то разве мы, греки,  не думаем, что их возможно измерять одни
другими?... Но это никак невозможно...».

Чтобы покончить  с терминологией  тел вращения,  можно сказать,  что в
пользу термина  «сфероид» говорит  то, что  ведь эллипс  можно вращать
около  двух действительных  осей, отчего  и получается  их два  сорта.
Парабола же имеет одну ось,  и потому параболоид вращения только один.
У  гиперболы же  кроме  действительной имеется  и  мнимая ось,  отчего
имеется два  сорта гиперболоидов  (однополостный и  двуполостный), но,
по-видимому, так как Архимед вращал только каждую из ветвей гиперболы,
он  вращения  около  мнимой  оси  не  рассматривал,  отчего  получался
только один  вид гиперболоида. Это своеобразное  положение эллипсоидов
может служить  объяснением того, что  терминология у Архимеда  не была
выдержана.

\subsection{Слабое знакомство Архимеда с творениями Демокрита}

Теперь разберем второй пункт §9:  именно, что Архимед тщательно изучал
труды Демокрита. Но  на с. 56--57 того же труда  об Архимеде Лурье для
доказательства того, что  Аристарх Самосский во многом  стоял на точке
зрения Демокрита,  принужден признать, что Архимед  не читал сочинений
атомистов.  Вот  это место:  «Как  сообщает  Архимед в  своем  ``Числе
песчинок''  (``Псаммит''),  Аристарх  говорил,  что  ``окружность,  по
которой Земля  движется вокруг Солнца,  так относится к  расстоянию до
неподвижных  звезд,  как  центр  шара к  его  поверхности''.  Архимед,
который  не  читал  сочинений  атомистов  и  не  знал  их  математики,
недоумевает и видит  в этом выражении сплошную  нелепость: ``Ясно, что
этого быть не может; так как  центр шара никакой величины не имеет, то
следует полагать, что никакого отношения между ним и поверхностью шара
быть  не может''.  С  точки  зрения геометрии  Евдокса  и Евклида  это
действительно нелепо,  но не с  точки зрения математики  атомистов, по
которой центр имел не ``никакую'',  а предельно малую величину; он был
``амерой'', самой  маленькой из  математических величин.  Из Фемистия,
комментатора  Аристотеля, нам  известно, что  атомисты утверждали  это
именно  о  центре  круга:  ``Нельзя   разделить  круг  на  два  равных
друг  другу  полукруга,  ибо  центр  всегда  окажется  при  разрезании
присоединенным  либо  к  одной,  либо к  другой  половине,  и  сделает
эту  половину большей''.  Аристарх,  как свидетельствует  впоследствии
Витрувий,  был одним  из  образованнейших людей  и лучших  математиков
своего времени. Он не мог бы сказать такой нелепости, если бы он стоял
на позициях  Демокрита и Эпикура и  примыкал к ней, хотя  открыто и не
заявлял об  этом, что и ввело  в заблуждение Архимеда, не  знакомого с
математикой атомистов».

Из  этой  цитаты  совершенно  ясно, что  Архимед  в  момент  написания
«Псаммита» не был хорошо знаком с сочинениями Демокрита, но совершенно
неясно, чтобы  Аристарх придерживался математики  Демокрита. Аристарх,
Коперник древнего мира,  был по своим взглядам  пифагорейцем (см. БСЭ,
2-е изд., 1950, т. 3, с.  5) и достаточно стойким по своим убеждениям,
так как был обвинен (за то,  что поставил Солнце в центре Вселенной) в
безбожии и должен был покинуть  Афины. Высокая же квалификация его как
математика не  допускает мысли, чтобы он  считал невозможным разделить
круг  на два  равных  полукруга. Кроме  того,  совершенно нельзя  было
говорить  о размерах  «амер»,  так как  это  было чисто  умозрительное
понятие. Поэтому это выражение Аристарха  было или его личной опиской,
или ошибкой переписчика, и никаких выводов о его близости Демокриту не
позволяет сделать.

Но,  может   быть,  «Псаммит»  был   написан  до  того,   как  Архимед
познакомился  с   сочинениями  Демокрита.  И  на   этот  счет  данные,
приводимые  Лурье  в книге  об  Архимеде,  позволяют составить  вполне
определенное суждение.  Письмо к Эратосфену, где  Архимед ссылается на
Демокрита, относится к первому  периоду геометрических работ Архимеда.
Более  поздние  работы  Архимеда  посвящены: 1)  проблемам  счета,  2)
математическим  играм  и  3)  гидростатике, не  считая,  конечно,  его
трудов по  изобретению военных машин.  «Если в предыдущую  эпоху жизни
Архимед посвящал  свои труды своим коллегам  по Александрийскому Музею
---  Конону, Эратосфену,  Гераклиту, Досифею,  то теперь  он посвящает
свои  труды  сиракузским  монархам  Гиерону и  Гелону»  (Лурье,  1945,
с.  173).  Как известно,  и  Гиерон,  и  Гелон были  родственниками  и
друзьями  Архимеда;  Гиерон  не  получил  власть  по  наследству,  но,
будучи талантливым полководцем в  войсках Пирра, захватил власть после
возвращения Пирра в Грецию (Лурье,  1945, с. 11). «Псаммит» несомненно
относится к позднему периоду творчества  Архимеда (и Лурье его относит
к поздним работам) и по характеру работы, и по тому, что она посвящена
соправителю Гиерона, царю Гелону (Лурье,  1945, с. 198). Значит, ясно,
что его  он написал уже  после ознакомления с приоритетом  Демокрита в
определении  объема конуса;  ясно также,  что  это не  побудило его  к
внимательному  ознакомлению  с  атомистической  математикой,  так  как
последнюю он считал пройденным этапом и даже не понимал (при его уме!)
выражения, могущие  быть истолкованными только с  атомистической точки
зрения.

\subsection{Бесспорно   положительная   роль  идеализма   в   развитии
эллинской математики. Самокритичность Архимеда}

Нам  остается коснуться  еще одного  возражения, которое  делает Лурье
против  строгих  методов  Архимеда.   Лурье  соглашается  с  тем,  что
открытие  иррациональных,  несоизмеримых  величин, последнее  слово  в
математике  V века,  было неприемлемо  для атомистической  математики.
«Доводы, выставленные  математиками идеалистического  лагеря, казались
неопровержимыми,  и  математика  атомистов  быстро  вышла  из  моды  и
была  предана  забвению»  (Лурье,   1945,  с.  22).  Довольно  странно
звучат  слова  «доводы  казались неопровержимыми»:  доводы  идеалистов
и  оказались неопровержимыми.  И  сейчас существование  иррациональных
чисел, насколько  мне известно, не оспаривается  ни одним математиком,
прибавились  еще трансцендентные,  комплексные  числа  и т.п.  Правда,
сейчас  «атомистическая  математика»  существует в  форме  «исчисления
конечных  разностей»,   но  никому   не  приходит  в   голову  считать
ее  единственно  возможной.  Но  на   той  же  странице  Лурье  пишет:
«Новая  математика выросла  на  фоне яростной,  ожесточенной борьбы  с
материализмом;  поэтому способы  аргументации  в  ней были  совершенно
иными, чем  в математике V в.  Математик этого времени не  видит уже в
читателе своего друга и  ученика, безусловно доверяющего ему, которого
он хочет ввести в самые сокровенные методы нахождения и доказательства
математических решений. Нет, математик этой эпохи смотрит на читателя,
как на  настороженного противника, который готов  ухватиться за всякую
ошибку, за всякое произвольное  или плохо сформулированное утверждение
автора.  Меньше  всего  этот  автор расположен  делиться  с  читателем
секретами своего производства --- как он  дошел до той или иной мысли,
откуда он взял  то или иное решение; до этого  читателю не должно быть
дела. Важно путем цепи силлогизмов загнать читателя в угол и заставить
его --- хочет он этого или не хочет --- признать, что предлагаемое ему
решение,  откуда  бы  автор  его  ни  взял,  единственно  возможное  и
правильное».  Дальше Лурье  говорит,  что  авторы математических  книг
черпают  свою  аргументацию  из практики  уголовного  судопроизводства
и  как  уголовный  преступник  стремится  перед  судом  показать,  что
постулированная обвинителями  картина преступления  абсурдна, так  и у
античного  математика способ  аргументации ---  приведение к  абсурду.
Однако:  «\emph{Влияние адвокатской  практики  и красноречия  софистов
дало  важные  положительные   результаты}:  аргументация  стала  более
строгой,  основанной на  правильных и  точных, научно  безукоризненных
определениях.  Математика  перестала  быть  связанной  с  определенной
философской,  моральной или  политической  системой:  ее выводы  стали
\emph{общеобязательными  для всех  людей}»  (Лурье,  1945, с.  23--24,
курсив ---  автора. ---  \emph{А.Л.}). Давая такую  объективную оценку
строгости новой математики, Лурье дальше  указывает, что она имела два
существенных  недостатка:  1)  новый способ  доказательства,  делающий
излишними  какие бы  то ни  было «недостаточно  очевидные» предпосылки
вроде предпосылки о существовании неделимых частиц, хорош для проверки
и доказательства результата, уже заранее известного или угаданного, но
не годится  для нахождения  новых, еще не  известных решений;  2) этот
метод  скорее огорашивает  читателя,  чем развивает  его ум;  читатель
не  получает  сколько-нибудь   отчетливой  картины  взаимосвязи  между
отдельными  элементами. Аналогичные  суждения  Лурье  высказывает и  в
других местах книги (с. 114, 116, 134 и др.).

Читаешь  и   не  понимаешь:  что  это   такое?  Защита  атомистической
математики или,  напротив, самый беспощадный обвинительный  акт против
математики  Демокрита.  Неужели  так плохо,  что  математика  достигла
такого уровня, что она стала общеобязательной, независимой от политики
и философии.  Неужели ученик  должен безусловно доверять  учителю (это
уместно в  школах, преподающих догматы  какой-нибудь религии, но  не в
свободных школах),  а не проверять учителя  в каждом его слове?  И как
же  велика заслуга  идеалистической  философии, если  под ее  знаменем
достигнут такой прогресс, такая подлинная свобода в отношениях учителя
и  ученика.  Но  тогда  сейчас же:  учитель  скрывает  секреты  своего
производства. А откуда  это взято? В научных  сочинениях по математике
и  сейчас  никто  не  пишет  весь ход  рассуждений,  приведший  его  к
окончательному  результату,  и   сейчас  чрезвычайно  много  отводится
«догадке».  Известно,  например,  что  при  дифференцировании  функций
существуют  вполне определенные  правила, а  при интегрировании  много
зависит  от «искусства  интегрирования», умения  использовать тот  или
иной прием, заменить переменные и т.д. В современной высшей математике
чрезвычайно много такого, что  «огорашивает» даже опытного читателя, и
чтение  многих  авторов  есть  очень нелегкий  труд.  Мало  того,  для
многих  математиков известно,  что  когда встречаешь  там такие  слова
как  «нетрудно  видеть»,  то  это  и есть  самое  трудное  место,  для
разбора которого  часто приходится  потратить гораздо  больше времени,
чем  там,  где  такой  оговорки  нет.  Дело  объясняется  просто:  для
математика  крупного калибра  многое совершенно  интуитивно «ясно»,  и
он  не  нуждается в  доказательстве.  Недаром  один крупный  математик
(кажется, Адамар)  сказал: «гениальные математики  предлагают теорему,
талантливые ее доказывают». Вся подготовительная работа математика ---
это леса, которые, естественно,  убираются после возведения постройки.
А если бы обо всем этом писать,  то объем работ возрос бы во много раз
без  нужды для  дела, так  как  опытные математики  разбираются и  без
лесов.  Кроме того,  очевидно,  дело вкуса:  мне лично  доказательство
квадратуры параболы методом исчерпывания кажется и более ясным и легче
запоминается, чем механический метод.

Все  это  касается, конечно,  чисто  научных  сочинений. В  литературе
чисто  педагогической,  конечно,  должны быть  подробно  освещены  все
методы  работы. И  тот  же  С.Я.Лурье нас  информирует,  что и  такого
рода  работы остались  в творчестве  Архимеда. В  спокойной обстановке
Сиракуз,  где,  несмотря  на  близость  Архимеда  ко  двору,  не  было
специфической  придворной обстановки,  существовал обычай  предложения
для  доказательства  новых  математических   истин  (Лурье,  1945,  с.
98);  результаты  потом  обсуждались.   При  таком  обсуждении  иногда
обнаруживались и  ошибки. Не был  безошибочным и сам Архимед.  В одном
выводе он  сам потом обнаружил ошибку,  и хотя никто из  математиков в
нем  ошибки не  обнаружил,  он публично,  в  работах, рассчитанных  на
широкое  распространение,  заявил о  своих  ошибках  и прибавил  такое
самокритическое  замечание (Лурье,  1945,  с. 100):  «Пусть это  будет
устрашающим  примером того,  как люди,  утверждающие, будто  они умеют
доказывать все то, что они предлагают решать другим, но не прилагающие
собственных  решений этих  вопросов, в  конце концов  принуждены будут
убедиться в том, что они  брались за невозможное». И письмо Эратосфену
имело целью  популяризовать эвристический недостаточно  строгий метод,
который Архимед, конечно,  легко мог бы скрыть (Лурье,  1945, с. 141).
Но если бы Архимед пользовался методом  Демокрита, то он не мог бы его
скрыть и Эратосфен о нем бы знал; не пришлось бы ему писать и письмо с
разъяснением этого  метода. Думаю, таким  образом, что у нас  есть все
основания для  утверждения, что эвристический метод  Архимеда вовсе не
представляет  собой  заимствования  у  Демокрита,  а  является  чем-то
несравненно более совершенным, изобретением самого Архимеда.

\subsection{Демокритовский тупик в математике}

Предыдущее   изложение   уже    позволяет   прийти   к   определенному
решению  относительно  вопроса   о  возможности  заимствования  линией
Платона  математических  достижений  линии Демокрита.  Ссылка  на  то,
что  оригинальные  сочинения   Демокрита  не  сохранились,  совершенно
неубедительна.   Об  идеях   Демокрита  мы   имеем  достаточно   ясное
представление,  и  можно  категорически  утверждать,  что  все  здание
античной математики  настолько проникнуто антидемокритовым  духом, что
ни  о  каком  заимствовании  и  речи быть  не  может.  Самое  большее,
что  можно   утверждать,  это  то,  что   они  полностью  переработали
математические  основы Демокрита  в идеалистическом  духе. Это  мнение
защищает Лурье (Лурье, 1946, с.  333): «Работы Демокрита и его великих
друзей и последователей произвели  сильное впечатление и в реакционном
лагере... Эти «так называемые  ``пифагорейцы'' удачно повели нападение
на  наименее защищенное  место в  теории Демокрита  --- на  его учение
о  мельчайших  неделимых  математических элементах.  Приняв  делимость
до  бесконечности и  заменив  амеры  (математические атомы)  Демокрита
непротяженными  точками  ---  монадами, они,  с  философской  стороны,
правда, ослабили  аргументацию Демокрита,  но с точки  зрения развития
математики это было шагом вперед,  так как подготовило учение Евдокса,
\emph{неуязвимое и  в математическом},  и в  \emph{философском смысле}
(курсив мой.  --- А.Л.). С  точки же зрения  идеалистической философии
достигалось то,  что материя составлялась из  нематериальных элементов
и, таким образом,  оказывалась только видимостью. На  вопрос же, каким
образом  из этих  нематериальных точек  получаются материальные  тела,
современные  Платону  пифагорейцы  дали  следующий  остроумный  ответ,
основанный на своеобразной диалектике: линия не составляется из точек,
ибо,  сколько  бы  непротяженных  точек мы  ни  складывали,  линии  не
получится; но при \emph{движении} (курсив  автора. --- А.Л.) точки она
переходит  в качественно  новую  сущность ---  в  линию; при  движении
прямой линии  возникает плоскость; при  движении плоскости ---  тело и
т.д. Точно  так же точка не  является элементом линии, так  как она не
протяженна, а является границей, пределом линии». Ясно, таким образом,
что  пифагорейская ---  платоновская  школа не  могла заимствовать  от
Демокрита  самые  ценные  свои  достижения,  совершенно  несовместимые
с  идеологией  Демокрита.  В  дальнейшем я  постараюсь  показать,  что
работы  пифагорейцев не  были реакцией  на работы  Демокрита, так  как
Демокрит и Евдокс --- различная реакция на критическую работу элеатов.
Отметим  также, что  пифагорейское  понимание материи  очень близко  с
представлением современной физики; подробнее речь будет дальше.

Сущность    демокритовского   атомизма    заключалась   в    признании
математических атомов, амер (т.е. не  имеющих частей) (Лурье, 1947, с.
167): «Такое  математическое тело гораздо труднее  помыслить себе, чем
физический атом, а представить его  конкретно и вовсе невозможно: оно,
очевидно, не  должно иметь правой и  левой стороны, верха и  низа и т.
д., и тем  не менее оно должно быть материальным  и обладать известным
протяжением. Очевидно, такое тело неделимо и математически, так как из
него  нельзя и  мысленно  выделить какую-либо  часть;  таких частей  у
него  не существует».  С  обычной точки  зрения  это нечто  совершенно
сверхъестественное, похожее на  бред сумасшедшего. И однако  на той же
странице (168)  Лурье высказывает одобрение: «Можно  построить науку о
природе, внутренне логичную, чуждую каких  бы то ни было противоречий.
Можно  продолжать  строить   величественное  здание  восточногреческой
науки,  совершенно  игнорируя  выпады  италийских  мракобесов,  врагов
естествознания,  ибо  эти выпады  сходны  с  бредом сумасшедшего.  Эта
постановка  сразу  же  сводила  на нет  все  элейские  хитросплетения.
Математика   была   выведена   из   тупика».  Но   ведь   выводом   из
демокритовского математического атомизма было отрицание иррациональных
чисел, несоизмеримости двух отрезков.  «Значение этого шага в развитии
математики трудно переоценить. С ним в математику вошло такое понятие,
которое  представляет  собой  сложную  математическую  абстракцию,  не
имеющую достаточно  прочной опоры в донаучном  общечеловеческом опыте»
(Рыбников, 1960,  с. 27). А как  смотрели на это атомисты?..  «В ответ
на  утверждения, основанные  на существовании  иррациональных величин,
атомисты заявляли,  что таких величин  не может существовать,  так как
неделимое является  общей мерой всех  величин» (Лурье, 1947,  с. 177).
Точно  так  же атомисты  возражали  против  теорем, доказывающих,  что
всякую прямую можно разделить на две равные части. Они доказывали, что
в рассуждениях не  принята во внимание ширина прямой.  Поэтому с точки
зрения  атомистов  все геометрические  теоремы  дают,  в сущности,  не
точный  результат, а  приближенный, с  погрешностью в  одно неделимое.
Безупречным   доказательствам   существования   иррациональных   чисел
атомисты упорно сопротивлялись (Лурье, 1947, с. 328).

Сейчас  совершенно  бесспорно,  что  именно  признание  иррациональных
чисел  обеспечило прогресс  математики,  отрицание их  --- это  тупик.
Я  совершенно отказываюсь  понимать,  как  демокритовский тупик  можно
назвать выходом из тупика.

Я  думаю,  этого совершенно  достаточно,  чтобы  говорить об  основном
значении демокритовской линии в математике было совершенно невозможно.
Невозможно говорить  и о сколько-нибудь существенном  заимствовании из
атомистической математики. Но, может быть, Демокриту можно приписывать
роль  трамплина,  от  которого отталкивались  математики  платоновской
линии?  Краткий  обзор  истории  философии  математики  этого  периода
позволяет  с совершенной  категоричностью  ответить и  на этот  вопрос
отрицательно.

\subsection{Спорность онтологической позиции элеатов}

Перейдем  поэтому  к  \emph{философии  математики}  этого  периода.  Я
уже  указывал в  предыдущем  параграфе, что  неверно утверждение,  что
работа  пифагорейцев была  реакцией  на  атомизм Демокрита.  Ошибочное
мнение о  связи элеатов и  Демокрита во  времени можно получить  и при
недостаточно  внимательном  чтении  Рыбникова  (с.  48),  но  если  мы
приглядимся внимательнее,  то увидим,  что Зенон элейский  примерно на
сорок  лет  старше  Демокрита.  Как указывает  Лурье  (1947,  с.  62),
Таннери  доказал,  что уже  до  Зенона  был древнейший  математический
атомизм, где первоначалом были  материальные, но не протяженные точки.
Такие представления  были и  в древней Индии,  а по  одному сообщению,
Демокрит заимствовал  свой атомизм у финикиянина  Моха. Отличием этого
древнего атомизма  от демокритовского было  то, что элементы  его были
непротяженными, но  из них при сложении  получаются протяженные линии,
плоскости и т.д.

Против  этого  учения и  выступили  элеаты,  сыгравшие значительную  и
своеобразную  роль  в  истории  человеческой мысли.  Но  прежде  всего
спросим,  на  какой  они  «линии» стояли?  На  материалистической  или
идеалистической?

Как  известно,  главными   представителями  элейской  школы  являются:
Ксенофан (родом  с Колофона из  Фокеи, откуда был изгнан),  Парменид и
Зенон, жившие  в Элее, и Мелисс  с острова Самоса. По  мнению «Истории
философии» (1941, т. 1, с. 65,  68, 72, 77), все они были абстрактные,
метафизические  материалисты. Хотя  Ксенофан  говорит о  боге, но  его
«бог» есть абстрактно понимаемый материальный субстрат космоса.

Любопытно, что,  несмотря на  абстрактность их  философских воззрений,
все  они, за  исключением  Ксенофана, были  весьма  активными и  часто
успешными политическими деятелями: Парменид свое собственное отечество
привел  в  порядок  отличнейшими  законами, Зенон  был  менее  удачен,
восставал  против тирана,  был подвергнут  пыткам и  казнен, Мелисс  в
качестве  самосского стратега  в 442  г. до  н.э. руководил  борьбой с
афинским флотом и успешно  боролся с афинскими командирами, знаменитым
трагиком Софоклом и не менее знаменитым Периклом.

Однако  не   все  считают  элеатов  представителями   материализма.  В
отношении  Парменида говорится,  что он  был знаком  хорошо с  учением
Гераклита,  но прежде  чем  усвоить учения  Ксенофана  и Гераклита  он
пропитался  западногреческими учениями:  традиция единодушно  называет
его  последователем  пифагорейских   теорий.  Ионийскую  философию  он
поэтому применил, главным образом, для  того, чтобы углубить и поднять
на  уровень современной  ему науки  эти учения  (Лурье, 1947,  с. 65).
«Если, как  утверждал Ксенофан,  божество едино,  если оно  разлито по
всей  Вселенной, то  и  Вселенная едина.  Но если  она  едина, то,  по
мнению Парменида,  ее необходимо считать всюду  однородной и одинаково
плотной.  Стало быть,  отдельных предметов  не существует,  и как  раз
наука последнего  дня доказывает правильность  старинного религиозного
положения,  что  воспринимаемый нами  мир  есть  только скверный  сон,
только  ``домысел'',  что  настоящий  вечный и  совершенный  мир  чужд
и  принципиально  противоположен  этому  миру»  (там  же).  Лурье  (с.
67) соглашается  с мнением  английского историка  философии Корнфорда,
который, говоря об элейцах, считает,  что матерью скептицизма была не
наука,  а религия.  «В самом  деле, если  все учения  одинаково ложны,
то  не проще,  не  удобнее ли  всего  предпочесть обычные,  освященные
традицией взгляды? Так Зенон возвратился от двух элементов Парменида к
четырем элементам мистерий; Мелисс (с. 9) доказывал, что истинный мир,
существующий объективно,  должен быть  \emph{нематериален, бестелесен}
(курсив  Лурье). Идеалом  элейской  школы,  как и  орфикопифагорейцев,
была  аристократическая  безмятежная неподвижность...».  Конечно,  это
подкрепляется  «анкетными данными»,  вплоть  до того,  что Мелисс  был
уроженцем Самоса, родины  «основателя реакционной италийской философии
Пифагора»  (с.  66).  Правда  биографии Парменида,  Зенона  и  Мелисса
как  будто   опровергают,  что   они  стремились   к  «безмятежности».
Куда  же   отнести  элеатов:  к   «прогрессивной»,  материалистической
или   «реакционной»,  идеалистической   линии?  Сопоставление   мнений
показывает, что  очень много  философов можно при  желании «причесать»
под  материалиста  или  идеалиста,  но   в  данном  случае  речь  идет
о  противопоставлении  совсем  в   другой  плоскости,  рационализма  и
эмпиризма, о чем придется подробнее сказать позже.

\subsection{Влияние  Зенона Элейского:  четыре направления  в античной
математике}

С точки  зрения математики наиболее интересным  представителем элеатов
является,  конечно,  Зенон,  выставивший  свои  знаменитые  апории,  в
частности наиболее известную, что Ахиллес никогда не догонит черепаху.
Апории  Зенона  были  основаны  на  критике  общепринятых,  по  мнению
древних, основных законов математики:

1) Сумма бесконечно большого числа любых, хотя бы и чрезвычайно малых,
протяженных величин обязательно должна быть бесконечно большой.

2)  Сумма  любого, хотя  и  бесконечно  большого, числа  непротяженных
величин всегда равна  нулю и никогда не может  стать равной некоторой,
заранее данной, протяженной величине» (Лурье, 1947, с. 68).

Последовательно применяя  эти «самоочевидные»  истины, Зенон  и пришел
к  своим  апориям,   которые,  следовательно,  являются  опровержением
считавшихся  абсолютными истинами  аксиом, приведением  их к  абсурду.
Долгая история  развития анализа бесконечно малых  привела к отрицанию
второй  аксиомы:  в  наше  время суммы  с  бесконечно  большим  числом
«бесконечно малых»  членов оказываются  равными конечному  числу. Это,
конечно, знает Лурье (1947, с. 69).

Какова же  роль Зенона?  Современные математики высоко  оценивают роль
Зенона. Творец  теории множеств,  гениальный Георг  Кантор, превращает
Зенона в  мыслителя сверхвременного масштаба, поставившего  задачи, не
разрешенные  доныне. Это  решительно оспаривает  Лурье (1947,  с. 70),
который считает,  что задача Зенона была  метафизическая, антинаучная.
«Но  в своих  стараниях  подорвать авторитет  молодой ионийской  науки
ему  удалось  нащупать  действительно слабое  место,  удалось  указать
на  логическое  противоречие.  Поэтому  Зенон  оказал  большую  услугу
математике,  показав, что  она должна  лучше обосновать  свои исходные
положения.  Эту задачу  выполнили (каждый  по-своему) с  одной стороны
Демокрит, с другой --- Евдокс  или его неизвестный нам предшественник.
Но  эта  реформа  математики  имела  место  только  через  20--30  лет
после  выступления  Зенона.  В  ближайший момент,  пока  новые  орудия
математической мысли  не были выкованы,  возможно было только  одно из
двух: или  вовсе отказаться от отвлеченных  геометрических построений,
или  просто  игнорировать возражения  Зенона.  По  первому пути  пошел
Протагор, по второму --- Эмпедокл и Анаксагор».

Мы видим, таким  образом, что апории Зенона оказали  мощное влияние на
четыре направления философской мысли. Поэтому, какие бы ни были у него
субъективные  мотивы, но  роль  мощного фермента  мысли он  несомненно
сыграл.  Разберем  вкратце,  как  можно  охарактеризовать  эти  четыре
направления.

а)  Что касается  Эмпедокла  и  Анаксагора, то  у  них  как будто  нет
математических заслуг.  Игнорирование ими  апорий Зенона ---  это, так
сказать, политика страусов, прячущих голову в песок.

б) Немногим лучше (если вообще  лучше) позиция Протагора (Лурье, 1947,
с. 123): полное отрицание  теоретической геометрии. Вся геометрия была
низведена на уровень чисто прикладной геодезии; полный отказ от всяких
обобщений философского характера. Как известно, вопросы науки Протагор
предлагал решать  большинством голосов.  Позиция Протагора  --- полное
банкротство теоретической науки.

\subsection{Дуалистическое преодоление апорий Зенона}

Третьим  направлением является  направление Демокрита.  Математические
доводы  Зенона  при  античных   предпосылках  (т.е.  при  наличии  тех
двух  указанных выше  аксиом,  в верности  которых  в древности  никто
не  сомневался)   могут  быть  опровергнуты  только   путем  допущения
существования неделимых величин. «Это допущение, как мы увидим, и было
впоследствии  сделано Демокритом,  но огромное  большинство ученых  не
захотело идти по этому пути» (Лурье, 1947, с. 70).

Оно  пошло по  четвертому, идеалистическому  пути, начиная  с Евдокса.
Демокрит   цеплялся   за  устаревшие   аксиомы   и   не  мог   принять
прогрессивнейших открытий  своего времени. В линии  же Платона сделали
правильный,  дуалистический   вывод.  Множество  дискретных   чисел  и
множество непрерывных величин подчиняются разным законам, хотя имеются
и  сходства,  допускающие  практическое использование.  Поэтому  никак
нельзя положения,  доказанные для  чисел, переносить,  без критической
проверки в каждом отдельном случае, на непрерывные величины. Этот путь
и  есть генеральная  линия  развития математики,  а не  демокритовский
тупик.  В  математике,  но  не  в  физике.  Там  идеи  Демокрита  были
плодотворны,  но  он  пытался  подчинить  одним и  тем  же  законам  и
физические тела,  и математические  понятия. «Геометрия  Демокрита ---
это часть  физики; всякий геометрический  образ имеет длину,  ширину и
глубину (хотя бы чрезвычайно малую,  как у точки, линии, плоскости), и
геометрия  учит о  пространственных  взаимоотношениях физических  тел»
(Лурье, 1947, с. 151).

Идеалистический  уклон огромного  большинства математиков  не является
поэтому ни  следствием приверженности устарелым воззрениям,  ни обязан
вообще каким-либо  вненаучным влияниям.  Это есть  следствие специфики
математики как  науки. И занятие математикой  не опровергает идеализм,
а  способствует   развитию  идеализма  даже  у   тех  ученых,  которые
первоначально  были близки  к  материализму.  Один из  великолепнейших
примеров --- великий Лейбниц, математические работы которого находятся
в  тесной связи  с его  философскими воззрениями.  Рыбников пишет  (с.
179): «Мы не имеем  возможности подробно описывать философские позиции
Лейбница  и их  эволюцию от  сочувствия механическому  материализму до
своеобразной  разновидности  метафизического  объективного  идеализма.
Отметим  лишь,  что во  всех  различных  по содержанию  математических
занятиях  он исходил  из одной  цели. Цель  эта философская:  создание
универсального метода научного познания,  по терминологии Лейбница ---
всеобщей характеристики». Великая цель Лейбница только сейчас начинает
осуществляться созданием  кибернетики. Вот как  ответил основоположник
этой новой  науки, Н. Винер, на  вопрос о том, какие  философские идеи
влияли на период  создания кибернетики: «Мне очень  трудно ответить на
этот  вопрос. Но  я  могу  сказать, что  из  философов прошлого  один,
несомненно, занимался бы сегодня  проблемами кибернетики. Это Лейбниц.
Современная теория информации  является прямой наследницей логического
исчисления Лейбница и его ``Матезис универсалис''» (Вопросы философии,
1960,  №  9,  с.  167).  Мы  знаем,  как  встретила  кибернетику  наша
советская  казенная философия:  как «реакционную,  лженауку, возникшую
после Второй мировой войны в  США и получившую широкое распространение
и в  других капиталистических  странах... Кибернетика  является, таким
образом, не только  идеологическим оружием империалистической реакции,
но и  средством осуществления ее агрессивных  военных планов» (Краткий
философский  словарь, изд.  4-е,  1954, с.  236--237:  не смешивать  с
дополнительным тиражом того же четвертого  издания, 1955 г., где слово
«кибернетика» вообще отсутствует).

\subsection{Последователи    Платона    в   современной    математике:
теоретико-множественный формализм Г.Кантора,  влияние на Колмогорова и
Лузина}

Платоновская  линия   в  математике  не  исчерпала   себя  исчислением
бесконечно  малых.  Как  указывает  Вейль (1934,  с.  69),  в  истории
человечества были предприняты три  попытки представить непрерывное как
некое  бытие в  себе. «Согласно  первой  и самой  радикальной из  них,
континуум состоит  из определенного исчислимого  количества дискретных
элементов,  атомов.  Для случая  материи  этот  путь, на  который  еще
в  древности  вступил  Демокрит,  был пройден  до  конца  с  блестящим
успехом  современной   физикой.  Для  случая   пространства  концепция
последовательного атомизма  была развита, кажется, впервые  Платоном с
ясным сознанием поставленной им себе  цели --- ``спасения'' явлений от
идеи. Атомистическая теория пространства была возобновлена в философии
ислама  Мутакаллимуном...,  а  на  Западе  ---  в  учении  о  минимуме
Джордано Бруно», с. 70:  «Второй попыткой является введение бесконечно
малых»,  с.  72:  «Третью  попытку  ``спасти''  непрерывное  в  смысле
Платона  мы  встречаем  в лице  современного  теоретико-множественного
обоснования  анализа». Мы  видим, что  платоновский идеализм  жив и  в
современной  математике.  Это прекрасно  изложено  и  в статье  нашего
математика  А.  Д.  Александрова  «Об идеализме  в  математике».  Если
Вейля, как «буржуазного» математика,  можно заподозрить в пристрастии,
то  в отношении  А.Д.Александрова  с идеологической  точки зрения  как
будто все  обстоит благополучно.  В числе  современных идеалистических
направлений  в   математике  он   указывает  на   философию  создателя
теории множеств,  Георга Кантора,  которую несколько  условно называет
«теоретико-множественный  идеализм»  (с.  6).  «Если  отбросить  такую
крайность,  как обращение  Кантора к  господу-богу, то  сущность этого
направления сводится к  следующему. Абстрактные математические понятия
и  прежде  всего  именно  бесконечные множества  (как  множество  всех
чисел,  множество  всех  функций  и  т.п.)  понимаются  как  некоторые
самостоятельные  сущности,  подлежащие   идеальному  познанию.  Это  и
есть  платонизм  в  математике,  ибо   Платон  как  раз  и  приписывал
самостоятельное  существование идеям...  Кантор выдвинул  принцип, что
``сущность математики в ее свободе'',  выражая этим ту свою установку,
что всякое свободное математическое  творение разума имеет объективное
идеальное  существование. Принцип  этот  чрезвычайно  удобен, так  как
не  стесняет математического  творчества и  заранее оправдывает  любые
абстрактные построения. Поэтому теоретико-множественный идеализм с его
односторонним  развитием любой  математической  абстракции в  абсолют,
оторванный  от  материи,   оказывается  достаточно  распространенным».
Следя за  чистотой материалистического  мировоззрения, А.Д.Александров
находит совершенно  аналогичные воззрения  даже в  статье «Математика»
в  первом   издании  БСЭ,  написанной  нашим   выдающимся  математиком
А.Н.Колмогоровым. На  с. 7 той  же статьи Александров пишет:  «Но если
система абстрактных  объектов вполне определена аксиомами,  то она тем
самым превращается  в нечто  вполне самостоятельное.  Строго замкнутая
теория может развиваться как  бы одними логическими рассуждениями. Она
как  бы целиком  переносится в  область понятий  и получает  идеальное
существование.  Аксиомы, взятые  из опыта,  из ``живого  созерцания'',
переходят  в  абстракцию  и  остаются  в ней,  а  переход  к  практике
якобы  не нужен  для  теории;  он нужен  якобы  лишь  для целей  самой
практики. Практика,  таким образом, из критерия  истины превращается в
потребителя, пользующегося  милостыней теории». Замечу  мимоходом, что
история европейской  техники как  будто подтверждает именно  эту точку
зрения  Колмогорова (правильно  или неправильно  ему приписанную,  это
другой  вопрос). Современная  техника  потребляет  плоды, выросшие  на
роскошном дереве  теоретической науки, а там,  где теоретическая наука
не  отрывалась  от практики  (Вавилон,  Египет,  Эгейская и  Микенская
культура, Мексика, Перу, Рим), там и практика скоро достигла потолка.

Другой наш выдающийся математик, П.Н.Лузин, тоже, конечно, относится к
идеалистам.

А.Д.Александров вовсе  не склонен  считать теорию  множеств ошибочной;
на  с. 11  той  же  статьи он  пишет:  «...теория  множеств привела  к
грандиозным успехам  математики, и  без ее идей  немыслимы были  бы ни
современный  анализ,  ни  современные  геометрия  и  алгебра.  Но  эти
успехи  неразрывно  связаны  с  задачами,  идущими  в  конечном  счете
от  естествознания и  техники,  а не  сводятся  к ``свободному  полету
математической мысли''».  При всем уважении к  А.Д.Александрову, как к
математику,  этому  последнему  утверждению  невозможно  поверить.  Он
постоянно упрекает идеализм, что он  ведет к «поповщине». Что он часто
ведет к религии или связан с религией, это несомненно (самый блестящий
пример --- Г.  Кантор), но, ведя к «поповщине»,  он одновременно ведет
науку к поразительным успехам. Что же касается утверждения, что всякий
успех  математики  и  другой  теоретической науки  связан  в  конечном
счете с  естествознанием и техникой,  то это тоже  «поповщина», только
материалистическая, а не идеалистическая, и эта «поповщина» отличается
от идеалистической тем,  что с такими блестящими  успехами, как теория
множеств, она не связана.

\subsection{Плодотворность   идеализма   в  его   свободе,   отрицании
обязательной связи математики с реальностью}

Почему связь  особой продуктивности  математики с идеализмом  мы можем
считать  не  случайной,  а  закономерной?  Превосходный  материал  для
этого  дает  та  же  статья   А.  Д.  Александрова,  где  есть  особая
глава,   посвященная  кризису   буржуазной  математики.   «Большинство
математиков  продолжало   и  продолжает  придерживаться   по  существу
канторовской  ``свободы   математики'',  так   как  она   менее  всего
стесняет  математическое творчество.  Но в  противовес этому  возникли
течения, стремившиеся  как-то ограничить эту свободу,  чтобы устранить
порождаемые ею противоречия и трудности» (с. 4). Об этих течениях речь
будет  впереди, а  пока  заметим мимоходом,  что фанатический  католик
Г.  Кантор  в науке  проповедует  максимальную  свободу творчества.  У
нас  часто наоборот:  те, кто  претендует на  монополию свободомыслия,
стремятся  установить «единственно  возможное»  решение  тех или  иных
научных вопросов.

В чем  сущность неограниченной  свободы математического  творчества? В
допущении  вводить такие  понятия, которым  ничего не  соответствует в
реальной действительности. Это есть  прямое нарушение или своеобразное
толкование  (в  лучшем  случае)  определения  математики,  данного  Ф.
Энгельсом,  по которому  предметом математики  являются количественные
отношения и  пространственные формы действительного мира.  И сейчас мы
часто  слышим, что  на  каждом этапе  развития  теории ее  определения
и  формулы   должны  иметь   какое-то  реальное   содержание.  Правда,
формулу  Энгельса   можно  примирить  с  допущением   полной  свободы,
если  «действительность»  понимать  в  платоновском или,  что  то  же,
канторовском смысле, т.е.  как реально существующий мир  идей, но вряд
ли Ф. Энгельс согласился бы с таким толкованием его определения.

Установка  Демокрита и  соответствовала  вполне определению  Энгельса.
Единственно реальное  в природе: атомы, следовательно,  и в математике
им соответствуют реальные амеры.  Но, конечно, определение Ф. Энгельса
шире  демокритовского понимания,  так  как  включает и  количественные
отношения.   Поэтому  в   процессе   развития  математики   постепенно
«материализовались»  и  те  понятия,  которые  первоначально  казались
совершенно нереальными.

Иррациональные числа прочно вошли в науку (хотя по самому названию они
значат «неразумные»).

Как это  ни может показаться  странным, позже вошло  обозначение числа
нуль и  отрицательные числа. «Для  чего нуль, что он  обозначает? Ведь
ему ничто в природе не соответствует».  Но мы знаем, что введение нуля
было необходимо для осуществления огромного прогресса в арифметике ---
позиционной  системы  счисления.  И  в теории  множеств  есть  «пустое
множество»?  Так ведь  это  противоречие в  самом  понятии: как  можно
говорить о «множестве», если в нем нет ни одного элемента.

Отрицательные корни уравнения  Кардано называл «фиктивными» (Рыбников,
с. 119), и даже Декарт называл их «ложными» (Вейль, 1934, с. 60), хотя
отрицательные величины уже понимались иногда  в форме «долга», а затем
были истолкованы как величины, имеющие противоположное направление.

Квадратные корни из отрицательных  величин так и назывались «мнимыми».
Кардано их называл «софистическими». Однако  уже в 1572 г. итальянский
математик Бомбелли  показал, что в так  называемом неприводимом случае
вещественный  корень  получается  как  сумма  двух  комплексных  чисел
(Рыбников,  с.   120).  В  данном  случае   вещественные  коэффициенты
кубического  уравнения  «обрабатывались»   так,  что  «превращались  в
мнимые»,  и  сумма  мнимых   дала  вновь  вещественный  корень.  Явное
нарушение  правила, по  которому каждый  шаг не  должен отрываться  от
реальной действительности. Математики этим не смущались, а потом Гаусс
показал, что мнимые и комплексные числа теряют всякую «софистичность»,
если их рассматривать как направленные величины на плоскости.

Дальнейший  шаг ---  кватернионы  (величины,  содержащие три  «мнимых»
компонента:  при умножении  некоторых  чисел знак  зависит от  порядка
множителей)  ---  также  получил  аналогичное  разъяснение,  когда  (в
векторном анализе) направление рассматривалось уже в трех, а не в двух
измерениях.

\subsection{Различное   понимание   положения  «всё   истинное   имеет
объективное существование» --- у материалистов и идеалистов}

Отсюда  мы  получаем   противоположение  материализма  и  объективного
идеализма   в   понимании   соотношения  истины   и   реальности.   По
материализму, всякая истина есть  отражение реального мира. По Кантору
--- тоже «все истинное имеет  объективное существование». Но разница в
том,  что реальное  при  самом широком  понимании понятия  материализм
всегда  локализовано  в пространстве  и  во  времени, платоновские  же
вполне «объективные»  идеи могут  и не  иметь локализации.  Поэтому мы
знаем, что и идеалисты, и материалисты часто употребляют термин «общие
идеи»,  но  под  этим  термином  можно  понимать  весьма  существенные
вещи:  1)  являющиеся  действительно  отражением  материального  мира;
в  разработке   этих  идей   материалисты  сыграли  важную   роль;  2)
предвосхищение особенностей материального мира, как-то: атомистическая
гипотеза,  взгляды  Фарадея  на  электричество  и  проч.,  комплексные
числа  и  проч.;  часто   здесь  обнаруживалось  непонимание  и  здесь
часто  материалисты  занимали  консервативную  позицию;  3)  отражение
или предвосхищение  нематериальных особенностей вполне  реального мира
идей;  4) просто  удобные средства  упорядочения наших  восприятий, не
претендующие  ни на  какой  реальный смысл;  чисто махистский  подход.
Махизм допускает максимальную свободу в выборе теоретических орудий, и
он, так сказать, принципиально не  связан ни с какой метафизикой, хотя
некоторые видные махисты, например, Дюгем, придерживаются определенной
метафизики. Но  не все  махистские «конструкты» могут  претендовать на
объективное  существование,  даже  если  мы  придерживаемся  положения
Кантора, что  «все истинное имеет объективное  существование», так как
на  объективное  существование,  очевидно, могут  претендовать  только
точные истины, если же мы заведомо пользуемся приближенными понятиями,
то настаивать  на их  объективном существовании  мы не  имеем никакого
права.

Вторая  категория  общих  идей,   конечно,  могла  бы  вполне  успешно
разрабатываться и  материалистами. Но часто материалисты  чинили такой
разработке препятствия,  так как  подобные общие идеи,  не противореча
материализму вообще,  противоречили тому сложившемуся  представлению о
материальном  мире, которое  в  определенное  время господствовало.  В
самое недавнее время мы имели этому пример, когда предвиденный Дираком
на  основе математической  теории  «антиэлектрон» казался  совершенным
абсурдом. Сам Дирак считал, что  если теория приводит к такому выводу,
значит,  она  неверна,  но  через  год  этот  антиэлектрон,  по-новому
позитрон, был  открыт экспериментально, а затем  и другие античастицы,
так что был  открыт новый мир как бы отрицательной  материи. Но если в
настоящее  время  теоретическая  физика и  математика  завоевали  себе
полную свободу исследования, то не всегда так было. Любопытно указание
на отношение  к геометрии  Лобачевского. Сам Лобачевский  называл свою
геометрию  «воображаемой», т.е.  как  будто не  придавал ей  реального
существования.  Впоследствии  оказалось,  что  геометрия  Лобачевского
двух  измерений  вполне  приложима  ко  вполне  реальной  поверхности,
псевдосфере.   Возможно,  что   сам  Лобачевский   был  материалистом.
Приписываемое  ему  мнение, что  нет  сколь  угодно абстрактной  ветви
математики, которая не получила бы в свое время приложения к реальному
миру (см. Томпсон, 1942, с. 10), не однозначно в смысле мировоззрения.
Но как его  геометрия, так и аналогичные взгляды Гаусса,  в свое время
не получили признания. Гаусс даже  боялся «беатийцев» и не опубликовал
своего мнения по этому вопросу. Кто  же эти «беатийцы»? Из рецензии И.
М. Яглома (1956,  с. 257) можно получить ответ на  этот вопрос: «Можно
также отметить несколько наивное замечание на  с. 263 о том, что Гаусс
не  решился опубликовать  свои результаты  по неевклидовой  геометрии,
``боясь испортить  отношения с  церковью и  властями''; на  самом деле
следует считать,  что Гаусс  под ``беатийцами'', которых  он опасался,
подразумевал  материалистов, а  не церковников»  (см. по  этому поводу
обстоятельную статью А.П.Нордена «Гаусс  и Н.И.Лобачевский», в IX томе
«Историко-математических исследований»).

\subsection{Материалистическое требование, чтобы  каждое понятие имело
физический смысл --- тормоз для развития науки}

Вернемся  к  формулировке  Кантора: «Все  истинное  имеет  объективное
существование».   Истинность  же   понимается   в  смысле   отсутствия
внутренних   противоречий.  Истина   в  этом   смысле  есть   критерий
существования.   Материалистическое  понимание   утверждает  обратное:
существование  есть критерий  истинности. При  платоническом понимании
это  не  просто  прямая  и обратная  теоремы.  Истинное  существование
---   идеальное,  в   призрачном  мире   явлений  можно   наблюдать  и
противоречие; наши сны, бред сумасшедшего не подчиняются логике. Часто
неполное  знание считается  противоречащим логике.  В старом  учебнике
исторического материализма  Н. И.  Бухарина с  торжеством опровергался
догмат  Троицы,   на  том  основании,  что   он  противоречит  таблице
умножения. Как может  быть 3x1 = 1?! Вероятно, и  сейчас этот аргумент
фигурирует в арсенале наших антирелигиозников. Но теория множеств того
же Г. Кантора  показала, что таблица умножения  справедлива только для
конечных, хотя  бы и  очень больших чисел.  Для бесконечных  чисел она
неприложима. Возьмем три бесконечных множества:

1, 4, 7, 10, 13, 16, 19, ...

2, 5, 8, 11, 14, 17, 20, ...

3, 6, 9, 12, 15, 18, 21, ...

Числа не повторяются,  так что все три  множества полностью отличаются
одно от  другого. Все три ---  одинаковой мощности, так как  из одного
путем прибавки  или убавки единицы  или двойки можно  получить другое.
Имеется  биоднозначное соответствие.  Но,  с  другой стороны,  имеется
биоднозначное соответствие между каждым из этих множеств и натуральным
рядом чисел:  1, 2, 3,  4, 5, ... .  Например, из третьего  ряда можно
получить натуральный ряд делением всех  цифр на три, а из натурального
ряда третий  ряд путем  умножения на  три. Натуральный  ряд получается
также  сложением  всех трех  рядов.  Таким  образом от  сложения  трех
одинаковых по мощности множеств мы получили новое множество одинаковой
мощности с каждым из слагаемых: 3x1 = 1.

Еще  более  удивительный  случай приведен  в  статье  А.Д.Александрова
(с.  10):  «...далекое  чисто   логическое  развитие  представлений  о
непрерывности  как о  множестве отдельных  точек ведет  к результатам,
которым  не  удается  приписать  физического  смысла.  Так,  доказано,
например, что  существует разбив ``математического'' шара  на конечное
число  таких  частей, из  которых  можно  сложить  два таких  же  шара
(не  меньших, а  равно  таких  же размеров).  Эти  части, как  говорят
математики,   ``неизмеримы'',  т.е.   им  нельзя   приписать  никакого
определенного  объема,  и  это  неизбежно, так  как  иначе  получилось
бы   противоречие:  объем   шара  равнялся   бы  сумме   объемов  двух
таких  же  шаров,  т.е.  единица  равнялась  бы  двум.  Но  вследствие
``неизмеримости''  частей тут  никакого формального  противоречия нет.
Однако реальный  смысл остается  неясным... теорема заведомо  не имеет
прямого физического смысла. Стало быть,  она может иметь лишь какой-то
более  абстрактный смысл,  но какой  --- неизвестно».  Может быть,  на
идеалистическом древе  познания созревают такие плоды,  даже мечтать о
которых материалистам не дозволено под угрозой изгнания из рая.

Уже  приведенных  примеров  пожалуй достаточно,  чтобы  показать,  что
идеализм  очень часто,  а  может быть  даже  большей частью,  является
не  тормозом  науки,   а  знаменем  ее  развития,   и  что,  напротив,
требование материализма, чтобы каждое  понятие имело физический смысл,
часто  является тормозом.  Но, говорят,  чрезмерная свобода  мышления,
допускаемая платонизмом, вредна (см. начало п. 16), и что существуют в
математике  течения, ограничивающие  эту  свободу.  Положение о  вреде
от  платонической  свободы  мышления Александров  не  доказывает,  но,
вероятно,  эти иные  течения, ограничивающие  свободу, окажутся  более
плодотворными  и,  вероятно,   материалистическими  (по  теории  «двух
лагерей»). На этом пути материалистов ждет, однако, разочарование.

\subsection{Другое  идеалистическое   направление  в   математике  ---
формализм Д.Гильберта}

Как  указывает  А.Д.Александров,   философским  конкурентом  Г.Кантора
является  другой выдающийся  математик Давид  Гильберт, основоположник
формализма  в   математике  (Александров,  с.  4):   «Задача,  которую
поставил  перед собой  основатель  формализма  в математике  Гильберт,
состояла в  том, чтобы устранить противоречия,  порожденные ``свободой
математики'',  и  сохранить в  математике  все  ценное путем  сведения
математики к формальным исчислениям, а самой бесконечности --- к чисто
формальной  идее... Таким  полным отвлечением  от содержания  формул и
правил  вывода достигается  бесспорность вывода,  ибо о  формуле в  ее
точном  выводе  спорить  нечего:  они  не  могут  быть  истинными  или
неистинными;  они просто  \emph{есть},  ибо они  написаны на  бумаге».
Дальше на с. 7: «Существуют формулы, а вопрос о том, что они означают,
не принадлежит, по убеждению формалистов,  к математике, а относится к
``философии'' или, по Гильберту, к ``метаматематике''».

По   мнению  Александрова,   «полное   и  окончательное   опровержение
формализма в  самом его ``зачатке''  содержится в заключении  Ленина о
том, что уже самое простое содержательное заключение диалектично, т.е.
заключает  момент  перехода,  развития,  неисчерпаемости  содержания».
Другую   статью  «Ленинская   диалектика  и   математика»  Александров
заканчивает словами:  «Четыре страницы  Ленинской заметки  ``К вопросу
о  диалектике''  ---  это   неисчерпаемое  богатство  содержания,  это
верное  указание  к руководству,  это  мощный  стимул к  исследованию,
к  новому  движению мысли,  к  творчеству».  Александров считает,  что
формальное  обоснование  математики  невозможно   и  что  это  находит
также математическое подтверждение,  как доказал австрийский математик
Гедель,  что  даже учение  о  целых  числах  не может  быть  исчерпано
формальным исчислением (с. 5). Дальше:

«Формализм  есть разновидность  ``математического''  идеализма и,  как
всякий  идеализм,  есть  дорога  к  поповщине.  За  ним  потянулась  у
самого Гильберта  аксиоматизация физики, связанная с  надеждами свести
физику  к геометрии,  а  на этой  почве  развился, например,  идеализм
Эддингтона  и ряда  других  физиков и  астрономов,  которые не  только
повторяли кантианские  утверждения об  априорности законов  физики, но
фактически дошли до  боженьки, подводя свою науку  под ярмо поповщины.
Извращение науки связывается с явным мракобесием и поступает на службу
политической реакции.  Так, независимо от любых  добрых намерений кого
бы то ни  было из ``математических'' идеалистов, их  философия ведет в
то  грязное  болото,  где  среди  ядовитых  цветов  идеализма  ползают
философские  динозавры ---  эддингтоны,  смэтсы  и расселы  (очевидно,
имеется в  виду Б.Рассел,  фамилию которого пишут  самым разнообразным
образом. ---  \emph{А.Л.}), где  рядом с утонченным  извращением науки
гнездятся  ``атомная'' философия,  борьба против  мира и  демократии и
прочие мерзости империалистической идеологии».

Как  видим, хрен  редьки не  слаще. Помимо  «боженьки», выдвигаются  и
политические  обвинения и,  кроме  того, сообщается,  что формализм  в
самом зачатии получил полное и окончательное опровержение.

Статья  А.Д.Александрова  появилась  в  1951 году,  в  мрачный  период
полного  расцвета  культа личности  Сталина.  Но  если возьмем  только
советскую  литературу  и  будем судить,  руководствуясь  ленинским  же
принципом «судить  по делам,  а не по  словам», то  получим совершенно
иную картину.

В  1947 году  был издан  перевод книги  Гильберта и  Аккермана «Основы
теоретической  логики».   В  предисловии  к  книге   наш  марксистский
математик  С.  Яновская  указывает,   что  историю  теоретической  или
математической логики  надо начинать с  «универсальной характеристики»
Лейбница  (вспомним   слова  Винера,  приведенные  выше,   в  п.  16).
Развивалась она в  XIX веке, но в основном является  одной из новейших
научных дисциплин, характерных для науки  XX века, когда она стала, по
существу,  частью  математики.  В  числе ряда  авторов  по  разработке
математической логики  выделяются имена  Гильберта и Рассела.  В конце
предисловия Яновская  пишет, что развитие  логики, и притом  с помощью
построенного самим же Гильбертом аппарата, обнаружило неосуществимость
его  надежд для  оправдания своей  формалистической и  идеалистической
точки зрения на  математику (с. 12). Но реакционные  ученые, по мнению
Яновской,  не  хотят  признавать,  что  работы  Гильберта  подтвердили
правильность  философских установок  марксизма-ленинизма. Ссылаясь  на
недавнее  выступление  А.А.Жданова  на  дискуссии  по  философии,  она
заканчивает  предисловие словами  Жданова: «Кому  же, как  не нам  ---
стране победившего  социализма и  ее философии, ---  возглавить борьбу
против растленной  и гнусной буржуазной  идеологии, кому, как  не нам,
наносить ей сокрушительные удары!»

\subsection{Плодотворность формализма в математике}

Но   прошло  двенадцать   лет   и  в   1959   году  появилась   книжка
нашего  крупного математика,  ныне  академика П.С.Новикова:  «Элементы
математической логики». О философии как  таковой в книге нет ни слова,
нет  и  философского  предисловия.  Во  введении  автор  указывает  на
неудовлетворенность  математиками  основаниями  своей  науки  и  снова
возвращается  к старым  антиномиям Зенона  (Ахиллес и  проч., с.  16).
Несмотря  на  огромные  успехи  математического  анализа,  то  и  дело
всплывали  трудности.  С.  20:   «Идеи  Гильберта  явились  переломным
моментом  в  вопросах  оснований  математики и  началом  нового  этапа
в  развитии  аксиоматического   метода».  Гильберт  проделал  огромную
работу по  осуществлению своей программы, с.  35: «Однако впоследствии
выяснилось,   что  в   буквальной  своей   постановке  эта   программа
невыполнима.  Хотя,  действительно,  все  математические  высказывания
и  всякая  логическая  дедукция могут  быть  представлены  посредством
формальных систем Гильберта и  в этом смысле формализмы («исчисления»,
с. 29)  могут неограниченно  охватывать все математические  знания, но
даже для решения вопросов о непротиворечивости основных математических
дисциплин финитизма Гильберта недостаточно. Дело  в том, что понятия и
принципы  всей математики  не  могут быть  полностью выражены  никакой
формальной системой, как  бы мощна она ни была.  Это обстоятельство, в
частности,  проявляется  в том,  что,  как  показал Гедель,  вопрос  о
непротиворечивости формальной системы не  может быть решен средствами,
которые формализуются в той же  системе... Но выход за рамки финитизма
не  уничтожает  основной  идеи   метода,  предложенного  Гильбертом  и
состоящего в формализации тех  математических систем, которые подлежат
обоснованию средствами некоторого  круга понятий, в силу  тех или иных
соображений  принятого в  качестве  основы. На  самом  деле, если  для
решения указанных выше вопросов средств финитизма недостаточно, то для
постановки этих вопросов этих  средств вполне достаточно». Заканчивает
свое  введение Новиков  следующими  словами (с.  37): «Описанные  нами
новые  идеи,  возникшие  из  вопросов оснований  математики,  как  это
часто бывает,  в своем развитии  вышли из первоначального  круга своих
задач.  Они  внесли  принципиально  новые понятия  и  методы,  которые
стали  применяться  и  в  вопросах,  не  связанных  непосредственно  с
основаниями математики. Аппарат математической логики нашел применение
в вычислительной математике и в технике в связи с конструкцией сложных
автоматических устройств».

Имя  Гильберта  фигурирует  в  тексте и  в  списке  литературы  статьи
«Метатеория» (БСЭ, 2-е  издание, 51-й дополнит, том, 1958,  с. 196). В
списке литературы  фигурирует 11 имен, ни  одного советского. Никакого
указания о «растленной, гнусной буржуазной» идеологии.

Не  нужно быть  математиком  и понимать  математическую логику,  чтобы
прийти  к  совершенно  ясному выводу.  Заслуги  Гильберта  грандиозны.
Допустим,  что  он, исходя  из  ложных  философских представлений,  не
осуществил полностью  свою программу. Здесь, как  ясно из изложенного,
критическую  работу  произвел  Гедель.  Но хотя  он  и  не  осуществил
свою  программу полностью,  его  работа имела  неисчислимые не  только
теоретические, но  и практические последствия. На  «ложном, классовом»
дереве идеализма созрел снова великолепный плод.

А  почему  же  в  списке  литературы по  «метатеории»  нет  ни  одного
советского имени  (как и в  близких по содержанию  статьях «семантика»
и  «синтаксис»,   помещенных  в   том  же  томе)?   После  выступления
«философского юнги»  (как он сам себя  называет, с. 4) А.  А. Жданова,
все  эти  направления  были   запрещены  как  проявления  «растленной,
гнусной и  буржуазной» идеологии  и потому на  этом фронте  наша наука
на  несколько  лет  отстала.  Но  так  как  кадры  математиков  у  нас
превосходные,  то после  1953  года у  нас  довольно быстро  выправили
отставание.  Что  касается   «полного  и  окончательного  опровержения
формализма»  в  самом его  зародыше  Лениным,  то  об этом  сейчас  из
приличия умалчивают.

\subsection{Трудность   провести   границу   между   материализмом   и
идеализмом при помощи одного критерия}

Что   касается  «философского   динозавра»  Б.Рассела   (иногда  пишут
Рассель), то его роль в построении формальных систем тоже очень велика
(см. предисловие Яновской к книге Гильберта и Аккермана), но сейчас он
уже перестал быть «философским динозавром», так как стал выступать как
активный  борец за  мир, несмотря  на свой  весьма преклонный  возраст
(родился в  1872 г.).  Сравним характеристику  Рассела в  двух тиражах
4-го  издания  «Краткого  философского словаря»,  1954,  с.  505--506:
«Реакционный  английский   философ,  один  из   главарей  современного
философского  идеализма,  воинствующий   идеолог  империализма...  Это
---   один  из   самых  оголтелых   идеологов  империалистического   и
антидемократического лагеря».  Та же  статья в 1955  г.: «...известный
английский логик и философ,  один из лидеров современного идеализма...
В последнее  время Рассел  выступает за  запрещение атомного  оружия и
ослабление международной напряженности».

Подождем немного, может быть,  скоро Рассела причислят к материалистам
(как  Спинозу  у  нас   изображают  материалистом  и  атеистом).  Ведь
один  из  главных  критериев,   которым  оперируют  наши  философы  (и
А.Д.Александров), что  идеализм ведет  к «боженьке», значит,  атеизм =
материализму. А Рассел не скрывает своего воинствующего атеизма и даже
сотрудничает  в  нашем  журнальчике  «Наука и  религия».  Его  брошюру
«Почему  я не  христианин»  я  уже цитировал.  Кроме  того, он  резкий
противник  Платона (о  чем тоже  говорилось выше),  а если  существуют
только «два  лагеря в философии»,  то тот, кто резко  выступает против
линии Платона, автоматически зачисляется на линию Демокрита.

Но если пользоваться формулой:  бытие определяет сознание или сознание
определяет бытие,  то ясно, что  Демокрит с его  атомизацией геометрии
сводит геометрию  к физике, бытие определяет  математическое сознание,
попытки же  математизации физики  сводят физику к  математике, значит,
сознание  определяет бытие.  Мы видим,  таким образом,  что не  так-то
легко провести границу между материализмом и идеализмом.

\subsection{Идеализм интуиционистской школы}

Но теоретико-множественный  идеализм и формализм не  исчерпывают всего
разнообразия  философских   течений  современной  математики.   А.  Д.
Александров в  той же статье  (с. 4--6) касается  еще «интуиционизма»,
который представляет другую, в  известном смысле слова противоположную
формализму попытку ограничения свободы. Он допускает в математике лишь
«интуитивно ясное». По Брауеру,  существует столько математик, сколько
есть математиков. По мнению Александрова (с. 6), интуиционизм со своим
требованием  «интуитивной ясности»  поставил  такие преграды  развитию
математики, что  его не принял  почти никто из  математиков. Изложение
А.Д.Александрова  не дает  представления об  интуиционизме. Непонятно,
как  можно  совместить  «ограничение  свободы»  и  такое  анархическое
утверждение, что сколько математиков, столько и математик. Более ясное
представление можно получить из  книжки известного математика Г. Вейля
«О  философии  математики»;  Вейль тоже  относится  к  интуиционистам.
Книжке  предпослано   предисловие  С.Яновской   (с.  3--4)   и  второе
предисловие от переводчика А. Юшкевича  (с. 4--7). Из этих предисловий
мы  узнаем, что  «наиболее интересным  явлением в  области современной
философии математики безусловно следует признать интуиционизм» (с. 3).
Яновская  пытается, как  это  и полагается  марксисту, связать  кризис
математики с  эпохой империализма и  утверждает, что «между  наукой, в
муках  рождающей  диалектический  материализм,  и  философией  класса,
в  устах  представителей  которого  все  чаще  и  чаще  звучит  теперь
лозунг  ``назад  к  варварству!'',  интуиционисты  выбрали  философию.
Они  принесли  основные  органические части  живого  тела  современной
математики  в жертву  своей  реакционной установке,  в жертву  стоящим
вне  науки  метафизическим  догмам.   Это  не  исключает  правильности
отдельных положений  интуиционизма, особенно в критической  его части,
направленной против формально-логических методов  в математике». Но на
той же странице оказывается,  что, пожалуй, привлекать империализм для
объяснения  возникновения интуиционизма  нет  оснований.  «Если еще  в
начале  текущего  столетия  большинство  математиков, в  том  числе  и
столь  крупных,  как  Ф.Клейн,  были  убеждены  в  том,  что  работами
Кантора, Дедекинда и Вейерштрасса  проблема обоснования анализа решена
окончательно  и  бесповоротно,  что  проблемы  иррационального  числа,
например, больше не существует,  если такое убеждение распространяется
еще и в-  настоящее время среди подрастающего  поколения наших молодых
советских математиков  --- не  только студенчества, но  и аспирантуры,
--- то работы  Вейля во всяком случае показывают, что  вопрос этот еще
спорный,  что над  проблемами числа  и  континуума еще  много и  много
придется  поработать. Больше  того, если  такому крупному  математику,
каким  является Вейль,  приходится  констатировать  наличие тупика,  в
который это обоснование заходит, если он вынужден заговорить поэтому о
кризисе  основ  математики,  то  это  является  еще  одним  прекрасным
доказательством  невозможности вообще  обосновать математику  на путях
идеализма».

А.Юшкевич  также указывает,  что  кризис основ  математики был  вызван
в  значительной   мере  (не   лучше  ли   сказать,  только   ими.  ---
\emph{А.Л.}) ростом самих математических теорий, выдвинувших ряд новых
и  поставивших ряд  старых  методологических проблем.  Идеалистический
характер  интуиционизма  стоит  вне  сомнений, с.  7:  «Этот  идеализм
в  философии   математики  полностью  согласуется   с  гуссерлианством
Вейля   и  с   субъективным   идеализмом   и  волюнтаризмом   Брауера,
декларированным  последним в  его  докладе  в Вене,  в  котором он,  в
частности,  рассматривает мир  как  творение нашей  воли и  утверждает
индетерминированность  его». Там  же из  высказываний Брауера:  «Среди
математических  рассмотрений, навязанных  всем людям  совокупной волей
всего человечества,  --- пишет Брауер,  --- надо прежде  всего назвать
предпосылку гипотетического  ``объективного пространственно-временного
мира''.  Само собой  разумеется,  что  все существование  какой-нибудь
каузальной  последовательности заключается  в  том,  что она  является
коррелятом  некоторой,  вызывающей   математические  акции,  установки
человеческой  воли;   не  может  быть   и  речи  о   каузальной  связи
мира  независимо  от  человека». Приведя  еще  несколько  высказываний
Брауера,  столь  же парадоксальных,  Юшкевич  пишет:  «И эта  насквозь
идеалистическая  фантастика представляет  собой философскую  установку
одного из крупнейших математиков  современности». «Из настоящей работы
читатель увидит все же, что интуиционизм ставил ряд важнейших вопросов
в  своей  критике  формально-логического направления  в  математике  и
теории континуума. В этом  нет, пожалуй, ничего удивительного. ``Когда
один  идеалист  ругает  другого,   на  этом  выигрывает  материализм''
(Ленин). И значение работ Вейля именно в этой критической стороне».

\subsection{Разнообразие  и  продуктивность   идеалистических  школ  и
отсутствие  свежей  материалистической  мысли  у  крупных  математиков
нашего времени}

И   эта   беглая   характеристика  интуиционизма   позволяет   сделать
один  вывод:   во  главе  стоят  крупнейшие   математики,  которым  их
идеалистическая  философия  не  помешала  проделать  важную  работу  в
математике.  Вместе с  тем  эти три  философские  школы враждуют  друг
с  другом  или,  по  крайней  мере,  развиваются  независимо  одна  от
другой. Значит,  в одном  «идеалистическом» лагере имеется  по крайней
мере  три  так  сказать  «подлагеря».  При  такой  междуусобной  брани
идеалистов,  естественно,  надо  было  ожидать,  как  это  и  высказал
Ленин,  что  выиграют   материалисты.  Разовьется  мощная  философская
школа,  которая  вытеснит  идеалистов.  Но мы  этого  не  видим:  наши
материалисты, которым предоставлена  полная свобода критики идеалистов
и  построения  материалистической системы  математики,  ограничиваются
более  или менее  (чаще более,  чем  менее) грубой  бранью и  писанием
«обезвреживающих»  предисловий, а  за  последние годы  и это  исчезло;
как  правило,   исчезает  и  всякая  философская   окраска.  Создается
впечатление, что потерялась всякая связь  философии и математики. И до
известной степени  это верно.  Всякая наука не  развивается монотонно,
но  по  красивому сравнению  академика  Несмеянова,  бывает работа  «в
одном  этаже» и  «переход  из одного  этажа в  другой».  Для работы  в
пределах  одного  этажа  философского  обоснования  не  требуется,  но
чтобы проделать  переход в  новый этаж,  требуется отрыв  от привычных
представлений,  пересмотр  укоренившихся   понятий,  полный  отрыв  от
требования обязательного «отображения»  внешнего мира. Идеалистическая
философия для этого несравненно пригоднее, чем материалистическая, так
как  она принимает  призрачность  нашего мира  явлений, независимо  от
характера идеалистической философии.  Объективный идеализм постулирует
существование иного, непризрачного мира,  а субъективный утверждает (в
первом  приближении), что  никакого  мира,  кроме призрачного,  вообще
не  существует.  Понятно,  почему  творчество  идеалистов  несравненно
разнообразнее  и свободнее,  чем творчество  материалистов. Пробившись
в  новый  этаж,  пионер   науки  создает  новую  систему  плодотворных
аксиом   и   с  этой   системой   можно   уже  работать   без   всякой
философии. Поэтому,  несмотря на усиленную  пропаганду диалектического
материализма,  это  направление  среди математиков  (ограничимся  пока
ими)  не пользуется  распространением  даже у  нас. Имеется  известное
количество несомненных идеалистов, но  их уста сомкнуты по независящим
обстоятельствам.   Большинство  равнодушны   к   философии,  а   среди
действительно  квалифицированных  математиков  (а   их  у  нас  вполне
достаточно) нашелся  только один  А.Д.Александров, который  выступил с
критикой идеализма,  да и то,  как видно, вовсе неудачно.  При желании
его  бы даже  можно обвинить  в  скрытой пропаганде  идеализма и  даже
религии.  Показано  разнообразие идеалистических  школ,  возглавляемых
крупнейшими  математиками,  и  даже  не упомянуто  о  существовании  в
математике  материалистических школ,  возглавляемых не  менее крупными
математиками.  Нельзя   же  на  одних  цитатах   из  Ленина  построить
математику.  И  ссылка на  гнет  классового  общества, запрещающий  на
Западе  выработать материалистическую  математику, не  может считаться
убедительной.  Ведь  в физике,  например,  имеются  на Западе  крупные
ученые материалисты,  например Бернал,  а уж  про биологов  и говорить
нечего: большинство современных биологов --- неодарвинисты, безусловно
стоящие  на  линии Демокрита;  то,  что  у нас  называют  идеалистами,
имеет совершенно  особое объяснение, о  котором я писал  достаточно. А
кроме того,  почему наша  блестящая фаланга математиков,  удивляя весь
мир  своими  успехами, не  выработала  до  сих пор  материалистической
философии  математики?  Почему  так  затянулись  роды,  долженствующие
дать человечеству диалектический  материализм? Со времени высказывания
Ленина прошло  более полувека. Надеюсь значительно  позже добраться до
этого  вопроса,  а  пока  можно  со  всей  уверенностью  сказать,  что
единственное более  или менее  развернутое выступление  против всякого
математического идеализма  А. Д. Александрова, есть  просто порождение
сталинского безвременья. Оно  и появилось в 1951 году,  в пору полного
подавления всякой свободной мысли.

\subsection{Непонимание  материалистами диалектики,  частое отсутствие
свободы и строгости мышления}

В разобранных выше выступлениях  наших марксистских математиков против
математического  идеализма  обращает  на  себя  внимание  один  пункт:
игнорирование диалектики. Юшкевич в критике интуиционизма считает, что
голое  отрицание интуиционистами  закона  исключенного третьего  носит
совершенно недиалектический характер (с.  7). Это высказывание неясно.
Утверждает  ли  Юшкевич,  что  всякое  отрицание  закона  исключенного
третьего противоречит  диалектике или только «голое»  отрицание, и чем
голое  отрицание  отличается  от  неголого. Но  хорошо  известно,  что
диалектическая логика и характеризуется именно тем, что отрицает закон
исключенного  третьего.  Но  диалектику наши  марксисты  позабывают  и
в  другом  смысле.  Вейль  и многие  другие  математики  высказываются
иногда  о  кризисе основ  математики  (в  других быстро  развивающихся
науках  мы часто  слышим подобные  высказывания). Слово  «кризис» наши
марксисты всегда понимают как нечто отрицательное, свидетельствующее о
слабости позиции.  Для марксистов это понятно:  наш общественный строй
гордится тем, что он  не переживает кризисов перепроизводства (правда,
не  столь  редки кризисы  недопроизводства,  но  об этом  предпочитают
умалчивать),  которые  характерны  для  капиталистического  мира.  Да,
конечно, в  экономической жизни страны кризисы  --- вещь нежелательная
и  надо  стремиться построить  такой  строй,  в котором  экономических
кризисов  не наблюдалось  бы.  Но являются  ли идеологические  кризисы
показателем нездорового состояния данной отрасли знания? --- по-моему,
нет. Это  только иное  название тому,  что Гегель  называл накоплением
противоречий, антитезисом. Это ---  закономерный этап развития идейной
системы,  приступ  к  «переходу   в  новый  этаж»,  выражаясь  словами
Несмеянова. Создается какое-то крупное  идейное построение. Оно служит
плодотворным руководством к действию и  обогащает науку, но с течением
времени  выясняется,  что это  не  «окончательная  истина в  последней
инстанции»  (такие  истины,  как  известно,  диалектическим  мышлением
совершенно отрицаются),  а только  более или  менее удовлетворительное
приближение. Можно,  и это  будет, так  сказать, реформистский  путь в
науке, постараться исправить положение, введя дополнительные гипотезы,
новые  члены в  уже  известные формулы  и т.д.,  и  очень часто  такие
поправки вносят  существенное улучшение в дело.  Но, наконец, наиболее
прозорливые  умы догадываются,  что поправками  к существующей  теории
ограничиться  нельзя, надо  перестраивать  теорию  сверху донизу.  Это
революционный  диалектический путь  в  науке,  и, как  свидетельствует
история науки,  в новом  идейном построении часто  используются многие
идеи,  казалось  бы,  окончательно  отвергнутые  на  предыдущем  этапе
развития. Общеизвестно сравнение  диалектического развития с движением
не  по  кругу,  а  по  спирали:  но  при  обороте  по  спирали  мы  не
возвращаемся  к  пройденным этапам,  а  приближаемся  к ним.  С  точки
зрения  диалектики нельзя  говорить ни  об «окончательно  утвержденных
положениях», ни об «окончательно опровергнутых положениях», и примеров
этого  из  истории  науки   можно  привести  достаточно.  Поэтому  для
беглой  оценки того,  развивается  ли наука  или  не развивается,  как
раз  отсутствие   кризисов  является  подозрительным,  и   если  такое
«благополучное»  состояние  длится  слишком  долго,  можно  почти  без
ошибки сказать,  что наука  пришла в состояние  догматического застоя.
Так  случилось с  великой  перипатетической школой,  и именно  поэтому
перипатетики,   неумеренные  поклонники   великого  Аристотеля,   были
наиболее ожесточенными противниками новых веяний во время Возрождения.
Сейчас же, в  самых замечательных науках, математике и  физике, мы все
время  наблюдаем  «святое  недовольство».  Уж у  них-то  казалось  бы,
могла закружиться  голова от неслыханных успехов,  но, оказывается, не
кружится. Они  все время  говорят о кризисах,  а их  науки развиваются
поистине с  бешеной скоростью,  потому что там  нет ни  догматизма, ни
культа личности.  Материализм со  своим требованием,  чтобы математика
ограничивалась отображением реального  мира, непродуктивен уже потому,
что даже  сейчас наши знания о  реальном мире далеко не  исчерпаны (да
вряд ли когда могут  быть исчерпаны). Материализм ограничивает свободу
мышления  и  не  доверяет  строгости разума,  если  разум  приходит  в
противоречие  с  привычными нам  представлениями  о  реальном мире.  У
него  нет ни  свободы, ни  строгости. Подлинный  же идеализм  связан с
максимальной свободой и строгостью мышления.

\subsection{Рационализм  Зенона   Элейского.  Неоднозначность  понятия
«рационализм».  Противоположности:  а)   эмпиризм,  б)  догматизм,  в)
эмоционализм,  г)  иррационализм,  д) интуитивизм,  е)  мистицизм,  ж)
мизологизм}

А  теперь  вернемся  снова  к  элеатам, прежде  всего  к  Зенону.  Как
же  разрешить  вопрос о  том  (затронут  в  п. 14),  материалисты  они
или  идеалисты? Роль  Зенона  высоко оценивают  как раз  представители
современного математического  идеализма. Думаю, что элеаты  лишний раз
опровергают  ходячую  у нас  теорию  «двух  лагерей» и  «двух  линий».
Мне  думается,  что  главная  заслуга Зенона  в  его  рационализме,  а
противоположение  рационализм-эмпиризм лежит,  так  сказать, в  другой
плоскости, чем  противоположность материализм-идеализм. Так  толкует и
А.Д.Александров  (с.  4):  «Примером  одностороннего,  преувеличенного
развития тех  сторон познания,  которые особенно сильно  проявляются в
математике,  могут служить  рационализм  и кантианство.  Представление
рационализма о том, что только разум, в противоположность чувственному
опыту, является  источником знания,  несомненно имело  своим отправным
пунктом  внутреннюю  убедительность  математических  выводов,  которые
осуществляются   чисто   умозрительно  и   представляются   совершенно
бесспорными,  даже более  бесспорными, чем  заключения, основанные  на
опыте».

Здесь   рационализм   противопоставляется  эмпиризму,   но,   конечно,
нельзя  говорить, что  рационализм  обязательно ведет  к идеализму,  а
эмпиризм  к материализму.  Однако известная  корреляция есть.  Крайний
рационализм  почти  обязательно  идеалистичен,  крайний  эмпиризм  ---
материалистичен, но,  например, материализм  ученых XIX века  впитал в
себя  значительную дозу  рационализма. Не  надо забывать,  кроме того,
что  понятие рационализма  неоднозначно. 1)  Уже указано:  рационализм
против  эмпиризма   и  крайний,   так  называемый   ползучий  эмпиризм
гораздо  вреднее  для прогресса  науки,  чем  крайний рационализм.  2)
Антагонист  догматизма; в  этом смысле  слово рационализм  употребляет
французский журнал  «Мысль», имеющий подзаголовок  «орган современного
рационализма». Правда, обычно рационализм этой категории подразумевают
узко в смысле  борьбы с фидеизмом и  религиозными догматами. Настоящий
рационалист  не   признает  никаких   догматов:  ни   религиозных,  ни
антирелигиозных. Поэтому  наилучшим наименованием рационализма  в этом
смысле  был бы  старый тургеневский  термин «нигилизм».  «Нигилист ---
это  человек,  который не  склоняется  ни  перед какими  авторитетами,
который  не   принимает  ни   одного  принципа   на  веру,   каким  бы
уважением  ни был  окружен  этот  принцип» (Отцы  и  дети,  гл. V).  В
этом  смысле  можно  сказать,  что  современная  математика  и  физика
достигли  подлинно  нигилистического  высокого уровня.  3)  Антагонист
эмоционализма.  Рационалистом  часто  называют человека,  который  или
вообще недооценивает эмоциональную сферу, или считает, что все чувства
должны  быть под  контролем разума.  4) Рационализм  и. иррационализм.
Это  противоположение само  может быть  разбито на  ряд видов.  Первое
противоположение  рационализма  иррационализму  совершенно  аналогично
противоположению рациональных  и иррациональных  чисел. Иррациональные
числа  вовсе  не  «неразумные»,  что  они  буквально  обозначают,  они
противоречат  только  привычному  разуму   и  требуют  введения  новых
вполне разумных  понятий. Другой смысл иррационализма  --- апелляция к
наличию подсознательных мыслительных  способностей человека, интуиции.
Здесь мы  имеем противоположение  рационализма и  интуитивизма. Третий
вид  иррационализма  ---  апелляция к  сверхчеловеческим  сущностям  и
возможность постижения  истины от  общения с  этими сверхчеловеческими
сущностями. Это называется мистицизмом  в истинном смысле этого слова:
познание через сверхъестественное озарение.

И наконец,  четвертым видом  иррационализма является  полное отрицание
возможности   познания   самых  существенных   особенностей   природы,
признание банкротства разума.

Всего  мы получаем,  таким образом,  семь различных  пониманий термина
рационализма. Если  прибавить, как всегда, что  существует рационализм
критический или скептический и  рационализм творческий, то мы получаем
еще большее разнообразие понимания термина рационализм.

\subsection{Зенон  ---  представитель   критического  рационализма,  в
противовес  эмпиризму  (Демокрит,  Ф.  Бэкон, Дарвин),  в  отличие  от
творческого рационализма линии Платона}

Зенона, очевидно,  надо отнести  к критическим рационалистам  в первом
понимании, и  поэтому, естественно,  против него ополчились,  в первую
очередь, эмпирики.  Это прекрасно  изложено в  известном стихотворении
А.С.Пушкина «Движение»:

\begin{verse}

\emph{Движенья нет,  сказал мудрец брадатый.\\* Другой  смолчал и стал
пред ним  ходить.\\* Сильнее  бы не мог  он возразить;\\*  Хвалили все
ответ  замысловатый.\\* Но,  господа,  забавный  случай сей\\*  Другой
пример на  память мне приводит:\\*  Ведь каждый день пред  нами Солнце
ходит,\\* Однако ж прав упрямый Галилей.}

\end{verse}

Краткий  комментарий  к   этому  стихотворению:  Рационалист  (Зенон):
разумом  мы утверждаем  --- движенья  нет. Эмпирик:  но мы  видим, как
собеседник Зенона встал и ясно ходит, следовательно, движение реально.

Рационалист: всегда ли, когда мы наблюдаем движение, мы должны считать
его реальным?

Эмпирик: ну, конечно.

Рационалист:   значит,  гелиоцентрическая   система  мира,   считающая
неподвижным Солнце, которое совершенно «очевидно» движется, неверна.

Эмпирик: разумеется.

Понятно,   почему  эмпирик   Ф.Бэкон   не  признал   гелиоцентрической
системы  Коперника.  Понятно  и отрицательное  отношение  Демокрита  к
иррациональным  числам.   Демокрит,  конечно,  не   был  стопроцентным
эмпириком,  но  как материалист,  он  был  ближе  к эмпиризму,  чем  к
рационализму. Разум доказывает  существование иррациональных чисел. Не
будем слепо верить разуму. Если он мог привести к такой нелепости, что
Ахиллес  не  догонит  черепаху,  то  возможно,  что  и  доказательство
существования иррациональных  чисел основано на таком  же мыслительном
фокусе. Будем искать квадратуры.  Нашли уже ряд: гиппократовы луночки,
сегмент параболы и др., постепенно доберемся до всего.

Рационалисты-платоники приняли всерьез  рассуждения Зенона. Они верили
в силу  человеческого разума, они  не были мизологами  (не доверяющими
разуму)  и сделали  вывод  о существовании  двух принципиально  разных
величин;  и хотя  при измерении  величин мы  всегда можем  найти общую
меру,  они  поддавались  тому   решению,  что  несоизмеримые  величины
действительно существуют.

Критический рационализм Зенона породил творческий рационализм Евдокса,
Евклида,  Архимеда  и  проч.   Но  в  дальнейшем  развитии  чрезмерная
осторожность древних эллинов была  оставлена и в математику внедрилась
значительная  доза  эмпиризма:  раз   удается,  значит,  годится.  Эта
практика  получила одобрение  в известном  афоризме Даламбера:  «Идите
вперед и  уверенность придет»  (Вейль, 1934,  с. 12).  Но рациональная
сущность математики  продолжала свое развитие и  кризисы в современной
математике порождаются все время рационалистическим ревизионизмом.

Но в науках,  посвященных реальному миру, эмпиризм  прочно внедрился и
там на рационализм  поглядывают с опаской. Демокритовская  линия в XIX
веке  получила  завершение  в  дарвинизме,  в  учении  о  естественном
отборе  как ведущем  факторе  эволюции. Сам  Дарвин  не скрывал  своей
верности  принципу индукции  Ф.  Бэкона. Он  даже  старался не  делать
преждевременных выводов, старался собирать  побольше фактов. Он не был
догматиком  и  ясно  сознавал  многие  трудности  своей  гипотезы,  но
думал  (подобно  геометрам  демокритовской  линии),  что  когда-нибудь
все  образуется.  За  прошедшие   шестьдесят  лет  выдвинуто  огромное
количество фактов и  рационалистических соображений, не укладывающихся
в прокрустово ложе дарвинизма.  Дарвинисты возражают двояко: накопляют
новые факты  в пользу  существования естественного отбора  (путая этот
вопрос с вопросом  о его ведущей роли) и  аргументируют от «боженьки»,
совсем  так, как  математик  А.  Д. Александров.  Мы  должны верить  в
дарвинизм, так как иначе придется поверить в Бога.

\subsection{Попытка  отрицать личные  заслуги  Пифагора при  признании
заслуг пифагорейцев}

Мне    думается,   после    всего   вышеизложенного    можно   считать
достаточно   прочно   установленным,  что   объективно-идеалистическая
пифагорейско-платоновская  линия   была  ведущей  линией   в  развитии
математики  и этого  значения  не потеряла  и  сейчас, хотя,  конечно,
математика в  значительной степени  освободилась от  всякой философии;
позитивизм  разного  сорта,  играющий   огромную  роль  в  современной
науке,  прямо  утверждает,  что  время  для  онтологии  (метафизики  в
старом понимании этого слова)  миновало. Эвристическую ценность такого
утверждения  нельзя  отрицать,  но  лишь в  определенных  условиях,  в
некритические периоды развития науки.

Но тогда выдвигается другое утверждение. Плодотворные линии в развитии
науки только  по недоразумению связаны  с именами Платона  и Пифагора.
Возникли школы  в науке,  а потом честь  своих открытий  они приписали
основателям школ, которые,  как Пифагор, может быть  вообще никогда не
существовали. Разберем эти возражения и начнем с Пифагора.

Лурье (1947, с.  31) справедливо указывает на  родство пифагорейских и
орфических учений;  этого сходства и генетической  связи их, насколько
мне известно, никто  не отрицает. Только орфические  учения, по Лурье,
имели широкий круг адептов среди крестьянства, и главным божеством там
был крестьянский  бог Дионис, а пифагорейские  учения имели замкнутый,
аристократический  характер,  и  главным  божеством  этих  союзов  был
аристократический бог Аполлон  (оказывается, классовые расслоения были
и на Олимпе). Лурье присоединяется к  мнению Бернета, что орфизм с его
мистериями  не  мог быть  для  философов  источником  каких бы  то  ни
было  определенных  научных  теорий,  и считает,  что  немногим  лучше
обстоит дело  и с ранним  пифагореизмом. На той же  странице: «Правда,
нам  сообщают,  что основатели  пифагорейской  школы  много сделали  в
математике.  Но, не  говоря  уже  о том,  что  сообщения об  открытиях
пифагорейцев   в  области   математики  вообще   сильно  преувеличены,
действительно  крупные  открытия в  математике,  как  показали Фохт  и
Франк,  были  сделаны  поздними   пифагорейцами,  а  только  приписаны
главе  школы, который  вообще  никаких крупных  трудов  после себя  не
оставил:  возможно,  что  в  области  математики  он  удовольствовался
ознакомлением греков  с египетской  наукой». Так как  ученики Пифагора
давали строгий обет молчания, то сведения, просочившиеся в литературу,
очень  отрывочны и  не дают  оснований  для того,  чтобы утверждать  о
математических открытиях.

Но почему же Пифагор пользуется  такой громкой славой до сего времени?
И на  этот вопрос  Лурье в  той же книге  дает ответ,  с. 33:  «Чем же
объяснить,  что  историки  философии  с  упорством,  достойным  лучшей
участи,  пытаются  восстановить  сложное здание  математической  науки
Пифагора? Один из талантливых историков античной философии, Бернет (с.
111 его труда), невольно раскрывает  нам эти побудительные мотивы. ``К
концу V в. математические вопросы привлекают к себе всеобщий интерес в
Греции, их изучают не с прикладной  целью, а ради их самих. Этот новый
интерес, очевидно не мог быть создан в какой-либо школе; возникновение
его  могло быть  только  делом рук  какого-либо  великого человека,  и
Пифагор  --- единственный  человек,  которому мы  можем приписать  эту
заслугу''».  Нас  эта  аргументация  никак не  может  убедить.  Почему
интерес  к какому-либо  учению  не может  возбудить  в обществе  целая
научная  школа,  а  только  отдельный  человек?  Почему  этим  великим
человеком не  мог быть  кто-либо другой, например  Анаксагор, Демокрит
(речь идет о  конце V в.)? Поэтому  вся эта теория имеет  для нас лишь
тот интерес, что она с  предельной ясностью раскрывает нам те, скрытые
обычно,  внутренние  причины,  которые побудили  приписывать  Пифагору
незаслуженно  большую роль  в  истории математики.  Орфико-пифагорейцы
не   оказали  влияния   на  положительную   науку,  поэтому   они  нас
здесь  интересуют  прежде  всего  как  родоначальники  и  вдохновители
идеалистически-мистической  псевдонауки, впоследствии  возглавлявшейся
Платоном  и его  последователями.  Эта  псевдонаука вела  ожесточенную
борьбу с материализмом,  впоследствии с атомистическим материализмом».
Такова формулировка, разберем ее.

\subsection{Специфичность   эллинской   математики  требует   принятия
персонального основоположника}

Лурье  ссылается на  Бернета,  на которого  он  с одобрением  сослался
на  с.  31,   и,  соглашаясь  с  мнением   английского  историка,  что
орфизм не  мог стать  источником для  определенных научных  теорий, не
соглашается с ним, что преемник и реформатор орфизма Пифагор мог стать
таким источником.  Лурье кажется неубедительным,  почему возникновение
действительно оригинального направления чистой, а не прикладной науки,
характеризующей именно античную Элладу,  не могло быть приписано целой
школе, а не  какой-либо личности. А знаем ли мы  случай в истории, где
новое,  оригинальное идеологическое  построение  в религии,  политике,
науке, философии, искусстве, сразу  возникло в целом коллективе (вроде
пресловутого  Персимфанса),  а не  возникло  сначала  в одной  голове.
Оставим в  стороне пока  такие древние  учения, как  иудаизм, буддизм,
христианство,   магометанство,  так   как  требующие   «документальных
подтверждений»  гелертеры сомневаются  в существовании  Моисея, Будды,
Христа,   Магомета.  Но   возьмем  такие   учения,  как   лютеранство,
кальвинизм, кантианство, гегельянство, дарвинизм, марксизм, ленинизм и
т.д. Все крупное и  действительно оригинальное связывается обязательно
с одной  личностью, которая и  оказывается создателем более  или менее
обширной  школы.  «Опус  пробанди», обязательно  доказать  возможность
возникновения   крупного  оригинального   направления  сразу   в  виде
коллектива,  лежит,  по-моему,  на   Лурье.  А  пифагореизм  в  смысле
выдвижения  самостоятельной  теоретической  науки и  ее  математизации
является единственным  явлением в  истории культуры вообще.  Он возник
один  раз в  истории, так  как все  последующее развитие  чистой науки
преемственно  связано с  эллинским  источником.  Тут, конечно,  Бернет
прав: такую школу  мог создать только действительно  великий гений, и,
если бы о Пифагоре нам  не было никаких исторических свидетельств, его
надо было бы просто постулировать.

Совсем уж  никуда не  годится уступка  Лурье: но  уж если  должен быть
великий человек,  так почему не  Анаксагор или Демокрит, так  как речь
идет  о конце  V  в. Вот  уж  тут приходится  просто  рот разинуть  от
удивления,  как  мог  такую  вещь  сказать  образованный  Лурье?  Ведь
он  же  сам  говорит  несколькими строками  позже,  что  пифагорейская
псевдонаука вела  ожесточенную борьбу с материализмом,  впоследствии с
атомистическим  материализмом.  Так  как  же  мог  лидер  материализма
основать  школу,  которая  систематически  боролась  с  материализмом?
А  в  сфере  математики   как  мог  ученый,  принципиально  отрицавший
иррациональные  числа  и  несоизмеримые  величины,  возглавить  школу,
одной из  важнейших заслуг которой  было именно введение  в математику
иррациональных  чисел. А  что касается  хронологии, так  ведь в  конце
пятого  века  пифагорейская школа  уже  развилась,  потому даже  чисто
хронологически  ни  Демокрит  (предпол.  год рожд.  480  до  н.э.)  ни
Анаксагор (предпол. родившийся в 500 г.  до н.э.) не годились. Годы же
жизни Пифагора (предпол. 571--497 до н.э.) удовлетворительно подходят,
как родоначальника нового направления в науке.

\subsection{Творческое     развитие    линии     Пифагора-Платона    в
противоположность учению Демокрита}

Несомненно, и это, кажется, никто не оспаривает, что многие достижения
пифагорейской  школы  были  приписаны  его учениками  даже  после  его
смерти своему учителю.  Умаляет ли это роль  Пифагора или увеличивает?
По-моему,  увеличивает.  Какие  бывают  отношения  между  учителями  и
учениками  хотя   бы  на  практике  последних   веков:  1)  наихудшие:
учитель использует  достижения своих  учеников и приписывает  их себе;
естественно,  при  таких отношениях  ученики  относятся  к учителю  не
особенно почтительно,  в особенности  после его  смерти; 2)  учитель и
ученики  строго корректны,  и  достижения обеих  сторон публикуются  с
указанием  истинного  автора; 3)  учитель  вкладывает  много мыслей  и
труда  в  работу  учеников,  но, желая  помочь  ученику,  отступает  в
момент публикации  на задний план,  и работа, которая в  сущности была
коллективной,  выходит как  индивидуальная  работа автора.  И в  нашей
русской  действительности мы  знаем ряд  примеров такого  благородного
отношения учителя  к своим ученикам.  Взять хотя бы  И.П.Павлова: иные
годы он не печатал  ни одной работы со своим именем, но  в эти же годы
выходил  ряд диссертаций,  выполненных в  его лаборатории.  Имя И.  П.
Павлова фигурировало только  в списке лиц, которым  автор выражал свою
благодарность, но все понимали, что  немало мыслей Павлова вошло в эту
работу.  Можно назвать  и  ряд других  имен,  в том  числе  и из  ныне
живущих.

Есть  и другая  сторона:  степень догматичности  учителя. Одни  ученые
требуют, чтобы ученики  работали только в указанном  им направлении: в
случае расхождения  дело может доходить до  разрыва. Другие, напротив,
допускают  широкую свободу,  и  их  ученики отличаются  необыкновенным
разнообразием  и  сфер деятельности,  и  даже  мировоззрения. К  таким
относится,  например,  выдающийся  немецкий  биолог  Иоганнес  Мюллер.
Догматизм особенно свойствен, конечно,  официальным религиям, но также
и антирелигиозным  учениям: учение модифицируется,  канонизируется, но
никакого дальнейшего развития или ревизии не допускается.

Если мы посмотрим теперь на пифагорейско-платоновскую линию, то тут мы
найдем  нечто совершенно  исключительное. «Линия»  не догматизируется,
но, напротив,  развивается. Ученики  сохраняют глубочайшее  уважение к
учителю  после  его  смерти.  Это уважение  сохраняется  столетиями  и
доходит  до той  высшей степени,  что ученики  лучшие свои  достижения
приписывают учителю. В этом  отличие пифагорейского направления, как и
платоновского,  от иных  форм религии,  так как,  несомненно, что  эта
линия никогда связи  с религией не порывала, но  с религией свободной,
не  догматической.  Поэтому  вполне  возможно, и  даже  вероятно,  что
Пифагору принадлежит только часть из того, что ему приписывают, но все
то, что приписывают Пифагору, проникнуто пифагорейским духом, и потому
эти утверждения не «ложь», но только «неполная истина».

Под «внутренними  причинами», которые заставляют  приписывать Пифагору
незаслуженно  большую  роль,  Лурье,  очевидно,  подразумевает  всякие
классовые и  проч. вненаучные влияния.  Я не склонен находить  у Лурье
подобные причины  его странных  противоречий, полагаю, здесь,  как это
свойственно многим старым русским (и не только русским) интеллигентам,
играет роль фанатическая антирелигиозность.

\subsection{Личная роль Платона в математической работе Афин}

Теперь разберем аналогичный вопрос о личной роли Платона. Лурье (1947,
с.  337) пишет:  «Таким образом,  хотя Платон  и был  хорошо знаком  с
наукой своего времени, главным образом, с математикой пифагорейцев, он
никак  не  может считаться  самостоятельным  деятелем  в области  этих
наук.  Несомненно, Платон  знакомился с  ними, прежде  всего, с  целью
опровергнуть ненавистную ему материалистическую философию Демокрита».

Лурье       решительно       оспаривает,       что       вдохновителем
реакционно-идеалистического направления в эллинской философии является
учитель Платона, Сократ (1947, с. 335). Он считает, что Платон получил
от Сократа только интерес к  вопросам нравственности и к рациональному
обоснованию  морали,  а  также  использовал  форму  устной  пропаганды
Сократа  в своих  диалогах. «Но  во  всем остальном  между Платоном  и
Сократом чрезвычайно мало общего:  Платон --- один из замечательнейших
художников слова с таким даром фантазии, который редко можно встретить
у какого-либо  другого философа.  У Сократа  и эстетическое  чувство и
фантазия были,  по-видимому, атрофированы. И  эстетически обоснованная
стройная богословская система Платона, и  его учение о переселении душ
не только совершенно  чужды Сократу, но и несовместимы  с его учением.
Не  имея  необходимых  познаний  в области  точных  наук,  но  обладая
зато  неудержимой фантазией,  Платон  сделал  смелую попытку  подвести
под  эту науку  идеалистическую  базу; при  этом  ко всей  кропотливой
работе  натурфилософов он  относился высокомерно-презрительно;  Сократ
относился с большим уважением к  работе естествоиспытателей, но сам не
хотел  вмешиваться в  их  дело,  считая его  не  имеющим значения  для
нравственного  усовершенствования (Апология,  19 с)».  Лурье связывает
различное  отношение  к  точным  наукам  с общей  реакцией  в  IV  в.,
когда требовалось  спешно и неумелыми  руками переводить всю  науку на
идеалистические рельсы (с. 336). «В этом отношении предшественниками и
учителями  Платона были  элейцы и  ``так называемые''  пифагорейцы (с.
333), а  никак не Сократ.  Почти вся  философия точных наук  у Платона
заимствована отсюда. Этот факт важен потому, что до последнего времени
существовала  сильная  тенденция  видеть в  Платоне  крупного  деятеля
в  области  точных наук,  особенно  математики.  При более  тщательном
изучении вопроса оказалось, что  собственные произведения Платона были
чисто метафизического и даже мистического свойства».

Таким   образом,   Лурье   усиливает  роль   Платона   в   обосновании
идеалистической  философии и  старается  умалить его  роль в  развитии
наук.  Заметим  тут же,  что  на  двух  страницах мы  наблюдаем  явное
противоречие.  На с.  335 говорится,  что Платон  не имел  необходимых
познаний  в  областях  точных  наук,  а  на с.  337  ---  что  он  был
хорошо знаком с  наукой своего времени, главным  образом с математикой
пифагорейцев.

\subsection{Трудность выделения личного вклада  Платона из его трудов:
пренебрежение вопросами приоритета}

Хорошо известно,  что ученые  потратили немало труда,  чтобы разделить
учение Сократа от учения Платона, --- так как кроме немногих сочинений
(например,  Законы), где  Сократ  вовсе отсутствует,  все учение  свое
Платон  излагает от  имени Сократа.  Но  и Сократ  в диалогах  Платона
нередко  говорит не  от  своего имени.  Заключительное  свое мнение  в
«Пире»  Сократ высказывает  от имени  Диотимы, в  другом месте  --- от
имени Аспазии.  Поэтому странной  кажется попытка Лурье  выставить как
антагонистов Сократа и Платона в отношении работы естествоиспытателей:
Сократ --- с  уважением, Платон --- презрительно. Но  о мнении Сократа
мы знаем от  того же Платона по одному месту  в великолепной «Апологии
Сократа»,  которая,  конечно,  далека   от  того,  чтобы  походить  на
стенографическую запись  речи Сократа.  Предсмертная беседа  Сократа в
«Федоне», где  он развивает свое  учение о бессмертии души,  по Лурье,
есть собственное творчество  Платона; но этот же  Платон, написавший в
«Апологии» защитную речь Сократа, влагает в уста своего учителя мнение
о работе естествоиспытателей, противоречащее его собственным взглядам.
Трудно этому поверить!

Если бы следовало  считать принадлежащим Платону только то,  на что он
сам заявил претензию, что он  изложил от своего собственного имени, то
пришлось  бы  прийти к  заключению,  что  ни  в какой  области  Платон
решительно ничего  не сделал,  так как решительно  все свое  учение он
излагает  от чужого  имени, прежде  всего,  Сократа. И  даже там,  где
Сократ не  упоминается, на  сцену выступает анонимное  лицо (например,
афинянин в «Законах»), но не сам Платон. Это его стиль работы.

Отношение к наукам ясно из того, что основы преподавания Платон сводил
к математике, астрономии, музыке и  диалектике. Правда, не было физики
и биологии,  но ведь  ни та,  ни другая наука  во времена  Платона как
наука еще не сложилась.

Само собой разумеется, что Платон не был в первую очередь математиком,
он, конечно, в первую очередь  был философом несомненно с богословским
уклоном.  Но  как было  показано  выше,  идеалистический характер  его
философии благотворно влиял на развитие  прежде всего математики, и из
платоновской  Академии  вышел  ряд выдающихся  математиков,  в  первую
очередь такие фигуры, как Менехм, Евдокс  и Теэтет, о чем было сказано
достаточно выше.  Что касается  разработки других  наук, то  для этого
требовались,  прежде всего,  значительные средства.  Поэтому их  могли
разрабатывать  сильнее  очень  богатый  Демокрит,  учитель  Александра
Македонского  Аристотель, которому  его ученик  присылал разнообразных
животных,   и  в   особенности   богато  поддерживаемый   государством
Александрийский музей.

Несомненно,  в  платоновской   Академии  шла  энергичная  коллективная
математическая работа, в которой и сам Платон принимал видное участие,
как образованный  математик. Но он, как  и те ученые высшего  ранга, о
которых  я говорил  выше,  не заботился  о  закреплении приоритета,  а
охотно  предоставлял  славу открытий  своим  ученикам,  с которыми  он
совместно разрабатывал  те или  иные проблемы. Он  упомянул и  то, как
всегда,  не  от своего  имени  в  «Тимее»,  учение о  пяти  правильных
многогранниках, и  поэтому совершенно правильно  поступают математики,
что,  не вдаваясь  в мелочный  спор  о приоритете,  за которым  Платон
никогда не гонялся, и сейчас называют эти тела «Платоновыми телами».

\subsection{Сомнительность   категорического  пренебрежения   техникой
Платоном. Архимед}

А то,  что Платон  принимал близкое  участие в  математических работах
своей  Академии,  явствует  из  слов самого  Лурье,  который  обвиняет
Платона, используя свидетельство Плутарха, в том, что он препятствовал
своим  ученикам использовать  механические приемы  в математике.  Друг
Платона, правитель  Тирента, Архит и  ученики его (и друзья)  Евдокс и
Менехм использовали инструменты, месографы для вычерчивания конических
сечений и для решения таким образом задачи удвоения куба (Лурье, 1945,
с.  40  и 68).  Плутарх  указывает,  что основателями  механики  «были
Евдокс  и Архит,  которые дали  геометрии более  пестрое и  интересное
содержание,  игнорируя  ради  непосредственно осязаемых  и  технически
важных применений этой науки ее отвлеченные и недоступные графическому
изображению проблемы... Платон порицал их за это». «При таких решениях
пропадает и гибнет благо геометрии, возвращающейся назад к чувственным
вещам. При этом она не поднимает  нас ввысь, не приводит нас в общение
с вечными и  бестелесными идеями, пребывая с которыми  бог всегда есть
бог...»  (с.  40).  Но  тут  же Лурье  прибавляет,  что,  несмотря  на
запрещение Платона, с этим запретом  не считались его ближайшие друзья
и  ученики.  Вряд ли  бы  это  случилось,  если  бы запрет  был  такой
категорический. Тот же Плутарх сообщает, что и великий Архимед смотрел
на работу  инженера и на  все, что служит  удовлетворению потребностей
жизни, как  на неблагородное и  простонародное дело (Бернал,  1956, с.
129). И,  однако, мы знаем,  что никто не  дал в античном  мире больше
практических приложений от науки, чем именно Архимед.

Такое категорическое пренебрежение к механическим приемам в математике
и практике засвидетельствовано, главным образом (если не единственно),
Плутархом, кстати  сказать, приверженцем  Платона и  вполне одобряющим
такую установку. Но можем ли мы безусловно доверять Плутарху? Солидный
червь сомнения, чтобы не сказать  больше, у нас зарождается при чтении
книги  Лурье «Архимед»  (1945,  с. 170--172).  Лурье приводит  большую
восторженную выдержку  из Плутарха,  где тот  между прочим  пишет, что
задачи Архимеда  изложены в настолько  простой и наглядной  форме, что
читатель приходит к  убеждению, что мог бы решить их  сам; так ровна и
коротка дорога,  которою он ведет к  доказательствам. Лурье совершенно
справедливо указывает, что эти  слова свидетельствуют, что сам Плутарх
не читал  и не  мог понимать математических  работ Архимеда.  Ведь эти
работы  сохранились,  они  отличаются  исключительной  строгостью,  но
отнюдь  не  простотой,  краткостью  и наглядностью.  На  это  обратили
внимание даже такие выдающиеся  математики, как учитель Ньютона Барроу
и Лейбниц. Лурье добавляет, что Плутарх  часто говорит не о том, каким
был Архимед,  а о том, каким  должен был быть идеальный  ученый. Думаю
поэтому,  что  позиция  Платона  была  приблизительно  такова.  Он  не
запрещал  пользоваться механическими  приемами,  но  указывал, что  не
следует им придавать  слишком большое значение. Это ---  леса науки, а
не сама наука. Так именно и поступал Архимед. Уже раньше было указано,
что письмо  Эратосфену свидетельствует,  что Архимед  не скрывал  и не
стыдился  механических  методов как  эвристических,  и  в этом  смысле
придавал  им  надлежащее  значение:  «Я мог  бы  сохранить  в  секрете
этот  золотой ключ,  но не  хочу этого  делать, так  как убежден,  что
оказываю этим немаловажную услугу математике; я полагаю, что многие из
математиков нашего или будущего времени, ознакомившись с этим методом,
будут в состоянии находить все новые и новые теоремы» (Лурье, 1945, с.
141).  Архимед  также  ограничил  число  инструментов,  применяемых  в
математике: геометр может ссылаться только на манипуляции, выполняемые
при помощи циркуля и линейки (Лурье 1945, с. 135).

\subsection{Полезность  для  науки  ограничения Платоном  в  геометрии
числа инструментов}

Был  ли  запрет Платона  вредным  или  искусственным? Как  будто  нет.
Возьмите любой учебник аналитической геометрии и другие математические
учебники, вы там не найдете описаний приборов для черчения кривых. Все
время речь идет, согласно завету Архимеда, только о циркуле и линейке,
и при  этом линейке  без отметок. Хорошо  известно, что  Гаусс доказал
невозможность  разделить любой  угол  на три  равные части  (трисекция
угла), и часто  думают, что значит, эта задача  вообще невозможная. На
самом  деле, невозможна  она  в тех  случаях,  где можно  пользоваться
только циркулем и линейкой без  отметок. Достаточно нанести на линейку
две  отметки, и  задача трисекции  угла становится  осуществимой. Мало
того,  в  проективной  геометрии предлагается  методика  построить  по
нескольким  точкам  коническое  сечение, пользуясь  одной  линейкой  и
совершенно  не пользуясь  циркулем.  Конечно,  среди математиков  есть
ученые, которые не брезгуют наглядностью и графическим методом вообще.
Другие,  напротив, совершенно  избегают чертежей.  К таким  относится,
например, знаменитый  Вейерштрасс. Что это ---  каприз, аристократизм,
желание   быть  непонятным   широкой  массе?   Конечно,  нет,   просто
Вейерштрасс  учел  ошибку,  которая произошла  вследствие  чрезмерного
доверия  к  наглядности.  До Вейерштрасса  математики  были  убеждены,
что  всякая  непрерывная  функция  имеет  производную.  Это  убеждение
основывалось  на том  положении, считавшемся  безусловно справедливым,
что всякая  непрерывная функция может  быть изображена в  виде кривой,
которая  в  каждой  точке  имеет  касательную.  Вейерштрасс  обнаружил
такие функции (которые теперь называются вейерштрассовскими), которые,
будучи непрерывными,  производных не  имеют. Графически  их изобразить
невозможно.

Но   мало  того,   попытки   ввести  механику   в  самое   обоснование
математических понятий  делались позже.  Такую попытку  сделал великий
Декарт.  Декарт,   конечно,  не   был  материалистом.   Его  философия
ясно  дуалистическая, но  эвристически  он  придавал большое  значение
механицизму,  так как  с  точки зрения  мировоззрения чрезвычайно  рас
ширял  область  машинообразного.  Как  известно,  он  считал  машинами
всех  животных, делал  исключение для  одного человека,  ввиду наличия
у  него  мышления. Декарт  (см.  Рыбников,  1960,  с. 138)  делит  все
кривые  на два  класса:  1) те,  что  сейчас называют  алгебраическими
кривыми  (по Декарту,  допустимые), которые  могут быть  построены при
помощи  плоских  шарнирных  механизмов,   в  которых  каждое  движение
первых  звеньев  полностью  определяет   движение  остальных;  2)  те,
что  сейчас  называют  трансцендентными  (по  Декарту,  недопустимые),
и  свойства  которых  могут  быть  открыты  лишь  случайно,  благодаря
специфическим  приемам,  не   носящим  систематического  характера.  В
основу  классификации   кривых  он   клал  число   звеньев  шарнирного
механизма  и разделял  кривые  по родам  (жанрам).  Полезен был  такой
подход  Декарта  или вреден?  Рыбников  (с.  136) указывает,  что  эта
классификация  не удержалась  и  вообще был  устранен тот  недостаток,
что  «область  этой науки  (аналитической  геометрии.  --- А.Л.)  была
еще  чрезмерно сужена  априорными требованиями,  проистекающими скорее
из  философских источников,  чем  из  потребностей метода,  ограничена
только  алгебраическими  кривыми.  Неудачной  оказалась  классификация
алгебраических кривых по  жанрам (родам), а не  по степеням уравнений,
их  выражающих».  Отказ  от  механического  подхода  расширил  область
аналитической  геометрии  и  вообще математики,  и  эта  классификация
Декарта  сейчас  имеет  только  историческое  значение.  Шарнирные  же
механизмы изучаются в своем месте,  где они приносят, конечно, большую
пользу. Платон  пророчески предвидел, что злоупотребление  механикой в
геометрии пользу  принесет небольшую, а вред  может причинить немалый.
Этим  недостатком  не  страдала аналитическая  геометрия  современника
Декарта, тоже выдающегося  математика, Ферма, который последовательнее
Декарта  внедрял координатный  метод.  Будущая геометрия  использовала
работы обоих великих французов (Рыбников, 138).

\subsection{Разумное отношение Платона к прикладной науке}

Так же можно ответить и на  вопрос, как относился Платон к практике. В
дальнейшем  нам  придется  вернуться  к этому,  когда  речь  зайдет  о
вкладе  в технику,  сейчас  ограничимся немногим.  Платон не  гнушался
приложениями науки,  но он понял историческую  миссию Эллады, создание
чистой,  теоретической  науки, и  к  себе  он приглашал  только  таких
учеников, которые стремились строить  здание чистой науки и философии.
Этому  завету следовали  и его  последователи. Бляшке  (1957, с.  115)
приводит следующую  легенду о  Евклиде: «Некий юноша  спросил Евклида,
какую пользу приносит геометрия. Тогда Евклид велел рабу сунуть монету
в руку юноши, желающему извлечь  из геометрии практическую выгоду. Эта
легенда  говорит  о  существовавшей  будто бы  у  греческих  геометров
антипатии  к  прикладным  наукам;  это, однако,  не  помешало  Евклиду
написать сочинение по  оптике. Сократ, кажется, даже  защищал мнение о
том,  что в  математике надо  оказывать предпочтение  всему тому,  что
имеет практические приложения». Позицию Евклида понимают все настоящие
учение: они стремятся привлечь таких  учеников, которых влечет к науке
тяга к чистому  знанию, сопряженная с готовностью  перенести лишения и
страдания. Студентов же, которые на первом курсе спрашивают: а сколько
мы будем  получать жалованья? справедливо оценивают  невысоко. И вовсе
не нужно быть  идеалистом, чтобы так думать. Марксист Бернал  на с. 26
своей книги с сочувствием цитирует  слова Дж. Томсона: «Исследования в
прикладной  науке приводят  к  реформам, исследования  в чистой  науке
приводят к революции».

Поборником идеи превосходства теоретической науки перед прикладной был
и К. А. Тимирязев. В  начале подлинного шедевра его творчества, лекции
«Луи Пастер»  мы читаем  следующие слова:  «Теория и  практика, чистая
наука  и  прикладная  наука.  Как  часто,  чуть  не  на  каждом  шагу,
приходится  слышать  это  сопоставление, причем  если  указывающий  на
него  полагает, что  его устами  гласит житейская  или государственная
мудрость,   то  почти   непременно   высказывается  за   превосходство
практического знания  перед теоретическим, за  преимущество прикладной
науки перед чистой. А если это  будет моралист, то он еще почтет своим
долгом сделать  внушение теоретику, эгоистически  изучающему предметы,
не имеющие прямого, непосредственного отношения к общему благу... Если
когда-нибудь слова ``благодарное  человечество своему благодателю'' не
звучали риторической  фразой, то,  конечно, на  могиле Луи  Пастера. А
между  тем  вся деятельность  этого  человека,  словом и  делом,  была
одним  сплошным опровержением  этого  ходячего  мнения о  преимуществе
практического  знания  перед   теоретическим».  Прочтите  великолепные
страницы 272--275 и  вы увидите в них иллюстрацию  основной мысли этой
замечательной биографии:  «... не существует ни  одной категории наук,
которой можно было бы  дать название прикладных наук. \emph{Существуют
науки и применения наук}, связанные между собой, как плод и породившее
их  дерево» (с.  274, курсив  Тимирязева). В  конце биографии  Пастера
Тимирязев пишет: «Практической, в высшем смысле этого слова, оказалась
не вековая практика  медицины, а теория химии.  \emph{Сорок лет теории
дали человечеству  то, чего не  могли ему дать сорок  веков практики»}
(с. 278, курсив К.Т.).

Эту  статью   Тимирязева  сейчас  неохотно  цитируют   его  лицемерные
почитатели. Был  взят курс на  практицизм, но гонение на  чистую науку
привело в 1948 году к торжеству не прикладной, а грязной науки.

Но,  может быть,  Платон не  допускал никаких  практических применений
науки? И это неверно. В №  4 популярного журнала «Знание-сила» за 1961
г.  на с.  47  помещена анонимная  заметка  «Будильник» под  аншлагом:
«Когда это было сделано  впервые». Оказывается, перед зданием Академии
Платона в  Афинах была  установлена статуя с  флейтой, и  каждое утро,
благодаря использованию принципа  водяных часов, в одно и  то же время
из  флейты  лились  сильные  звуки, призывавшие  учеников  Платона  на
занятия. «В Академии Платона  начались занятия, --- говорили прохожие,
заслышав эти звуки». Вряд ли этот будильник был бы установлен, если бы
Платон  преследовал  или  даже  не  сочувствовал  занятиям  прикладной
механикой.

\subsection{Связь с  религией характерна не  только для Платона,  но и
для многих крупных математиков}

Теперь о  связи с \emph{религией}.  Связь пифагореизма и  платонизма с
религией совершенно  бесспорна, и  это является основной  причиной той
ненависти, которую питают  к платонизму фанатические антирелигиозники.
Опять  и   этот  вопрос  во   всей  полноте  нам   придется  разобрать
значительно позже; пока коснусь, главным образом, того страшного слова
«мистический», которое действует на наших безбожников как в свое время
слово «жупел» на замоскворецких купчих.

И  налет  мистического  сопутствует  математике  очень  долгое  время;
пожалуй, он  не исчез  и сейчас.  С. А.  Богомолов в  очень интересной
книге  «Эволюция  геометрической  мысли»  (1928)  на  с.  216  обронил
такую   фразу:  «Прежде   всего   заметим,  что   все  мистическое   в
вопросе  о четырехмерном  пространстве  --- и,  надо сознаться,  самое
интересное  в  нем  ---  уже выходит  за  пределы  математики...»  Эта
фраза в  свое время  вызвала негодование  одного из  наших блюстителей
идеологического порядка.  И я  благодарен означенному  блюстителю, так
как  его  заметка  побудила   меня  проштудировать  эту  популярную  и
очень  интересную  книгу. Богомолов  на  с.  52 говорит,  что  Паскаль
свой  шестиугольник (фигурирующий  в его  известной теореме),  называл
«мистическим  шестиугольником». Клейн  (1935, с.  108) указывает,  что
мнимые  числа  долго  сохраняли несколько  мистический  характер.  Это
пугало материалистов и восхищало идеалистов, например Лейбница (1702):
«Мнимые  числа ---  это  прекрасное и  чудесное убежище  божественного
духа, почти что сочетание (амфибиум)  бытия с небытием». Известно, что
Яков Бернулли, найдя, что эволютой логарифмической спирали оказывается
тоже  логарифмическая  спираль,  усмотрел в  этом  символ  воскресения
из  мертвых  и завещал  изобразить  логарифмическую  спираль на  своей
могиле (подобно  тому как  Архимед завещал  поставить на  своей могиле
изображение  шара, вписанного  в цилиндр).  Вполне понятно,  что можно
поверить известной легенде,  что Пифагор в восторге  от открытия своей
теоремы принес в жертву богам сто быков (гекатомбу).

Интуиционизм, конечно,  всегда подвергался нападкам  как «мистическое»
направление,  и  не без  основания.  Вейль  указывает (1934,  с.  46),
что  об аналитическом  методе Галилей  высказывает распространенное  в
то  время  убеждение,  усматривающее  в  постепенном,  идущем  шаг  за
шагом доказательстве отличие  человеческого познания от божественного.
«Тогда   как  Он   познает   посредством  одного   лишь  узрения,   мы
переходим  от   одного  умозаключения  к  другому   путем  постепенных
рассуждений...  Напротив,  божественный  разум в  одном  лишь  узрении
сущности окружности  познает мгновенно  и безо всяких  рассуждений все
бесконечное множество  ее свойств  (однако интенсивно, с  точки зрения
объективной  достоверности  каждой   обретенной  истины,  человеческий
разум  не уступает  божественному)».  Вот подлинно  прометеев дух:  не
отрицание божественного и мистического  в периоде, а решимость достичь
божественного совершенства  в мышлении. Вспомним и  нашего Ломоносова:
«даже перед самим Господом Богом в холуях ходить не намерен».

Вспомним   и  обращение   гениального  Г.   Кантора  к   господу  богу
(Александров, 1951,  с. 6)  и закончим  тем, что  сам К.  Маркс совсем
не  относился  с суеверным  страхом  к  «мистическому». Как  известно,
К.  Маркс  назвал  таинственным,   «мистическим»,  тот  этап  развития
анализа бесконечно малых, когда  он «существует, а затем разъясняется»
(Рыбников, с. 178).

Страшно   подумать,  что   случилось  бы   с  наукой   и  всей   нашей
цивилизацией, если бы над  ней тяготела власть современных блюстителей
идеологического порядка.

\subsection{Обвинение Платона  в склонности к  геометрии исключительно
из-за его политических взглядов не выдерживает критики}

Что  касается  связи  с  политикой,  то  здесь  коснемся  лишь  одного
пункта,  одной весьма  оригинальной  «догадки» С.Я.Лурье.  Большинство
исследователей  Платона  считает,  что  Платон ценил  математику  и  в
частности  геометрию ради  ее  самой и  что не  было  связи между  его
стремлением к геометрии и  его политическими взглядами, которые многие
считали реакционными,  антидемократическими. В своей статье  «Платон и
Аристотель о  точных науках»  (1936, с.  307--309) Лурье  считает, что
Платон потому  предпочитал геометрию  арифметике, что  он рассматривал
геометрическую пропорцию соответствующей аристократическому «равенству
по  достоинству»,  тогда  как арифметическая  пропорция  соответствует
«равенству  по   числу»,  что  защищали  демократы.   При  этом  опять
ссылается   на   Плутарха,   согласно  которому   Ликург   изгнал   из
Лакедемона арифметическую пропорцию как демократическую и свойственную
черни,  ввел  же  геометрическую  как подобающую  мудрой  олигархии  и
законной  царской  власти.  Лурье  приводит и  слова  самого  Платона:
«Геометрическое равенство  имеет большую силу  и среди богов,  и среди
людей,  а  ты проповедуешь,  чтобы  люди  захватывали  то, что  им  не
принадлежит.  \emph{Ты  пренебрегаешь  геометрией}»;  с.  308:  «Итак,
знаменитое  выражение  Платона:  ``ты пренебрегаешь  геометрией'',  на
которое  так  часто  ссылаются,  вовсе  не  приглашает  всех  мыслящих
людей  заниматься  геометрией,  а   означает  скорее  всего:  ты  чужд
``геометрической'',  т.е. умеренно-аристократической  точки зрения  на
государственное устройство»; с. 309: «Другими словами: выражение ``бог
всегда геометризует''  означает только: бог ---  враг демократического
поравнения!» Наконец, и знаменитую надпись на входе в Академию: «Пусть
никто, чуждый геометрии,  не войдет под мою крышу»  Лурье толкует так:
«Пусть  ни  один противник  геометрического  равенства,  т.е. ни  один
демократ, не войдет в мой дом!».

Очевидно,  С.Я.Лурье  догадался  о  том,   о  чем  до  него  никто  не
догадывался. Ведь изречение «Пусть  никто, чуждый геометрии, не войдет
под мою крышу» избрано Коперником в качестве эпиграфа его бессмертного
труда.  Кеплер  избрал  пять   Платоновых  тел  для  построения  своей
«мистической космологии». Все это были противники демократии?

Предпочитая  геометрию,  Платон не  гнушался  и  арифметики, и  в  его
сочинениях  много  рассуждений  о  числах, как  и  подобает  философу,
близкому Пифагору. Вполне возможно, что Платон проводил аналогии между
математикой  и  своими политическими  взглядами,  но  вряд ли  он  мог
думать,  что  рассуждением  о  геометрической  пропорции  он  достигал
\emph{математического доказательства}  превосходства аристократической
формы   правления.  В   своих  политических   произведениях  он   этой
аргументацией  не  пользуется. И  здесь  С.Я.Лурье  пал жертвой  своей
фанатической преданности Демокриту.

\subsection{Заключение:   Платон   ---    центр   развития   эллинской
математики}

Постараюсь   резюмировать  результат   этой  главы.   Можно  выставить
следующие тезисы:

1) линия Пифагора-Платона и есть генеральная линия развития математики
не  только в  античные  времена, но  за всю  историю  науки вплоть  до
настоящего времени (Кантор, Гильберт и др.);

2)  этим  своим значением  эта  «линия»  обязана  тому,  что в  ней  в
наибольшей полноте выразился дух эллинской культуры;

3) эллинская математика совершенно оригинальна по следующим признакам:
а) свободное  теоретическое творчество, б) синтетический  характер, в)
отсутствие догматизма, г) рационализм;

4) придание высокого значения теории не означало пренебрежения опытом,
а лишь придание опыту вспомогательного значения;

5)  синтетический   характер  связан   с  холистическим   (от  целого)
пониманием  античной  математики  в   отличие  от  меристического  (от
частей).  Различие  античной  математики  от  аналитической  ---  см.,
например, книгу Извольского (1941);

6) отсутствие  догматизма имело следствием длинное  развитие эллинской
математики, сочетавшей исключительное почтение к родоначальнику чистой
математики Пифагору,  с полным отсутствием культа  личности, мешающего
развитию науки;

7)  рационализм афинской  и александрийской  школ является  правильной
реакцией на чисто скептический рационализм элейской школы;

8) что касается линии Демокрита, то в математической области она почти
исчерпывается одним Демокритом. Это ---  тупик, а не генеральная линия
математики, так  как здесь мы имеем  догматизацию некоторых положений,
чрезмерное уважение к практическому  опыту, что выразилось в отрицании
иррациональных чисел, игнорировании критической работы элейской школы;

9)  Платон поэтому,  несмотря на  неясность его  личных математических
достижений,  может   с  полным  правом  считаться   центром  эллинской
математики, вершиной ее, конечно, является Архимед;

10)  о  каких-либо  серьезных  заимствованиях  платониками  достижений
Демокрита  в  математической  области  не   может  быть  и  речи,  так
как  основные достижения  эллинской  математики (аксиоматика  Евклида,
иррациональные  числа,  метод  исчерпывания  и  проч.)  глубоко  чужды
догматической математике Демокрита;

11)  религиозный  дух  пифагорейско-платоновской  линии  не  мешал,  а
благоприятствовал  развитию  математики,   так  как  благоприятствовал
холистическому   мировоззрению,   побуждал  искать   гармоничность   и
закономерность мира,  внушал веру  в силу разума,  способность постичь
тайны  мироздания.   Понятие  «мистический»,  что   заставляло  многих
материалистически  настроенных ученых  отвергать  или опасаться  таких
понятий как отрицательные, иррациональные,  мнимые числа, нисколько не
пугало идеалистов;

12) попытка  связать интерес к  геометрии Платона с  его политическими
взглядами не выдерживает ни малейшей критики;

13) блестящее развитие математики  могло осуществиться только на линии
Платона, но никак не на линии Демокрита.

\begin{flushright} 15 августа 1961 г. \end{flushright}

\clearpage

\section{ЛИНИИ В АСТРОНОМИИ. 1. До Коперника}

\subsection{Значение     истории     астрономии,     в     особенности
гелиоцентрической системы,  в истории культуры.  Формулировка основных
обвинений против идеализма в широком смысле слова}

Известно,  какое первенствующее  значение имеет  астрономия в  истории
человеческой   культуры.  Здесь   мы   имеем   и  первое   грандиозное
проникновение математики в  истолкование внешнего мира, исключительной
широты  синтез  в теории  всемирного  тяготения  и, наконец,  огромное
влияние  на  формирование  мировоззрения.  Не зря  часто  всю  историю
человеческой науки делят на два периода:  до и после Коперника. «Чем в
религиозной  области  было  сожжение  Лютером  папской  буллы,  тем  в
естествознании  было великое  творение Коперника,  в котором  он, хотя
и  робко  после  35-летних  колебаний  и,  так  сказать,  на  смертном
одре, бросил  вызов церковному суеверию. С  этого времени исследование
природы, по  существу, освободилось от религии...»  (Энгельс, 1949, с.
153).

Мнение Энгельса  иными словами  изложил современный  крупный астроном,
академик  Амбарцумян (1961):  «Астрофизика... не  является ``чистой'',
``отвлеченной'' наукой,  ``познанием мира ради самого  познания''. Она
--- по  сути, по целям,  глубоко ``земная'', как и  ее предшественница
и  родительница  --- астрономия,  которая  с  первых же  своих  шагов,
устанавливая  самые примитивные,  с  нашей  сегодняшней точки  зрения,
закономерности, вооружала  людей знаниями, необходимыми для  их земных
дел: умением вести счет времени, ориентироваться на суше и в океанском
безбрежье...  А  потом,  окрепнув, она  восстала  против  религиозного
мракобесия, и  это был  один из знаменательнейших  шагов человеческого
разума на пути к свободе и счастью».

Общеизвестна   постоянно    утверждаемая   связь    между   классовой,
политической  и  идеологической  борьбой. В  популярнейшей  форме  это
изложено, например, в известной драме Бертольда Брехта «Жизнь Галилея»
(1957, с. 111):

\begin{verse}

\emph{Вскочил ученый  Галилей,\\* Отбросил святое  писание,\\* Схватил
трубу,  закусил губу,\\*  Осмотрел сразу  все мироздание.\\*  И Солнцу
сказал: шагу сделать не смей?\\* Пусть вся Вселенная, дрожа,\\* Найдет
иные круги;\\* Отныне станет госпожа\\* Летать вокруг прислуги!}

\end{verse}

Мнение, что  астрономия как  наука родилась с  Коперником, чрезвычайно
широко распространено.  Это мнение высказал и  президент Академии наук
СССР Несмеянов  по случаю 410-летнего  юбилея со дня  смерти Коперника
(Коперник, 1955, с.  6). Он с одобрением цитировал  слова Сталина, что
без Коперника у нас «не было бы вообще науки, скажем, астрономии, и мы
все  еще пробавлялись  бы  обветшалой системой  Птолемея». Это  широко
распространенное  мнение  о  связи  науки  и  идеологии  можно  кратко
сформулировать  в  виде  следующих положений:  1)  история  астрономии
является прекрасной иллюстрацией борьбы двух лагерей: прогрессивного и
реакционного; 2)  первое противоположение: наука на  службе практики и
«чистая  наука»;  3) второе:  наука  развивалась  в постоянной  борьбе
с  религией  и церковью;  4)  третье:  эта  борьба  была связана  и  с
политической  борьбой  угнетенных  народов и  классов;  5)  четвертое:
прогрессивная сторона связана  с линией Демокрита, материалистической,
консервативная --- с идеалистической линией Платона; 6) сообразно двум
линиям  и в  астрономии  существовали только  две системы:  донаучная,
Птолемея, и  научная, начиная с  Коперника; 7) в  силу принципиального
различия враждующих  систем система  Коперника знаменует  собой резкий
разрыв  с   системой  Птолемея;  8)  преимущества   системы  Коперника
настолько  очевидны,  что  доказать   ее  можно  самыми  элементарными
средствами: это  хорошо показано  в той же  драме Брехта;  9) наконец,
упорство консерваторов  в защите обветшалой системы  Птолемея не имеет
никаких научных  оснований, а  в лучшем случае  объясняется косностью,
обычно же вмешательством реакционных, антинаучных побуждений.

В защиту  всех этих положений  обычно приводится целый  ряд бесспорных
исторических  фактов:  осторожность  Коперника в  опубликовании  своей
бессмертной книги, сожжение Джордано  Бруно, включение книги Коперника
в индекс  запрещенных книг,  процесс Галилея  в самый  разгар жестокой
Тридцатилетней войны, последней попытки  реакционных кругов во главе с
католической  церковью подавить  прогрессивные стремления  Реформации,
преследования  гелиоцентрической  системы  в России  даже  во  времена
Елизаветы  и  т.д.  Все  это  широко  используется  в  антирелигиозной
пропаганде, и  многие пропагандисты  совершенно искренне  уверены, что
все «честные и прогрессивные» люди не могут не разделять перечисленных
тезисов.

Я  вовсе не  склонен отрицать  все перечисленные  факты, служащие  для
подтверждения  мнения цитированных  авторов. Я  только утверждаю,  что
если мы несколько углубимся в историю  науки и примем в соображение не
только те факты, которые подтверждают эти  мнения, но и те, которые им
противоречат,  то  мы получим  совершенно  иное  представление о  роли
разных  идеологий  в развитии  астрономии.  Ни  один из  перечисленных
тезисов  не   выдержит  испытания,  и  их   придется  заменить  иными,
большей частью  прямо противоположными. Ясно,  что к такому  выводу мы
можем  прийти только  после  тщательного  разбора истории  астрономии.
Целесообразно, по-моему, не начинать разбора с вопроса о возникновении
астрономии, а сначала разобрать  историю гелиоцентрической системы ---
центральной теории в развитии научной астрономии, а уж потом коснуться
зачатков астрономии. Это я делаю потому, что история гелиоцентрической
системы гораздо  более хорошо  известна, и  потому все  выводы гораздо
более убедительны.  Принимая в соображение  разнородность выставленных
тезисов, придется и вопроса коснуться  со всех сторон, а это, конечно,
увеличивает объем работы.

Возникает, естественно,  вопрос, каким образом человек,  не специалист
в  области  астрономии,  решается  ревизовать  мнения  многих  крупных
астрономов. Во-первых, потому, что  защищаемое мной мнение, к которому
я пришел на  основании тщательного ознакомления с  предметом, вовсе не
ново, а только довольно  основательно позабыто. Во-вторых, потому, что
при  современной специализации  астрономы, историки  астрономии, часто
вовсе  не  касались  философской  стороны дела,  считая  ее,  как  это
свойственно очень многим ученым, совершенно несущественной для дела, а
философы, напротив, совершенно  игнорировали техническую сторону дела,
прекрасно изложенную компетентными  астрономами. И наконец, в-третьих,
потому, что очень многие  современные изложения истории астрономии как
раз проникнуты теми самыми  вненаучными влияниями, о которых говорится
в девятом тезисе.

Я  не сделал  никаких  документальных открытий,  я использовал  только
печатную  литературу, в  подавляющей  части  напечатанную в  советские
времена, но мне думается, что  я, следуя Гоголю, сумел «вызвать наружу
все,  что ежеминутно  перед  очами  и чего  не  зрят равнодушные  очи»
(Мертвые души. Глава 7).

\subsection{Этапы развития космологии}

Когда  на  популярных  лекциях  излагается  история  гелиоцентрической
системы,  то упоминаются,  конечно,  только  факты, благоприятные  для
ходячего мнения,  и тогда получается  все гладко. Но если  мы немножко
углубимся  в детали,  то натолкнемся  на совершенно  неожиданный факт.
На  инквизиционном  процессе   Галилея  учение  Коперника  именовалось
«пифагорейским  учением»,  и  сам   Галилей  с  этим  соглашается.  Но
Пифагор вовсе не был  представителем «линии Демокрита», следовательно,
подвергается  сомнению справедливость  пятого  тезиса.  Но как  только
мы  начинаем  в  этом   разбираться,  сразу  убеждаемся  в  неверности
шестого  и  восьмого  положений,  а  дальше  возникают  сомнения  и  в
остальных.  Действительно,  можно  ли считать,  что  система  Птолемея
есть  примитивная,  донаучная,  связанная  с  религией,  в  частности,
католической,  а   система  Коперника  ---   совершенная,  единственно
научная.  Достаточно самого  скромного  знакомства  с историей,  чтобы
убедиться,  что геоцентрическая  система Птолемея  отнюдь не  является
примитивной  системой.  Она  есть  результат  очень  долгого  развития
астрономии, вставшей  на подлинный научный путь.  Она, конечно, сейчас
устарела,  но и  система Коперника,  в том  виде, как  ее дал  великий
торинец,  тоже устарела.  Что же  можно считать  подлинно примитивной,
донаучной  системой  мироздания?  Конечно, то  представление,  которое
сохранилось и до  наших времен у невежественных людей.  Что земля есть
нечто  плоское  и  совершенно  неподвижное,  что  солнце  каждый  день
действительно всходит и заходит. Что имеется небесный свод, на котором
большинство звезд прикреплено совершенно  неподвижно, и лишь небольшое
число  совершает  движения. Что  с  небесного  свода иногда  срываются
звездочки  и падают  на  землю. Бывают  и сверхъестественные  события,
выходящие  за  обычные  рамки  и внушающие,  естественно,  страх,  как
землетрясения внушают страх, так  как нарушают привычное представление
о  совершенной неподвижности  Земли.  Эти сверхъестественные  события:
затмения  солнечные и  лунные, появление  комет, падение  метеоритов и
«каменных дождей»  и т.д. Само  собой разумеется, давно  заметили, что
все  тела  падают  вниз,  и, следовательно,  понятия  «верх»  и  «низ»
являются абсолютными понятиями.  Представления об «антиподах» казались
нелепостью: как же  люди могут стоять на земле  головой вниз? Конечно,
эти представления о плоской и неподвижной Земле дополнялись фантазиями
о трех  китах или гигантской  черепахе, на которой  расположена Земля.
Представление о  трех китах  долго держалось,  в частности,  в русском
народе,  хотя  оно,  конечно,  не  имело  ни  малейшего  научного  или
религиозного основания.

От  представления  о неподвижной,  плоской  Земле  до системы  Ньютона
необходимо было  проделать огромную  работу, которую в  самых основных
этапах можно охарактеризовать  так: 1) Земля ---  замкнутое, висящее в
пространстве тело;  2) полюса ее равноправны,  следовательно, возможны
антиподы; 3) Земля вращается вокруг  оси, а не звездное небо вращается
около Земли; 4) Солнце является  центром для вращения части планет; 5)
Солнце является центром всей Солнечной системы, включая Землю. Все эти
этапы  характеризуют, так  сказать,  качественную сторону  космологии.
Но  наиболее важная  ее  сторона, математическая,  также шла  этапами;
6)  независимо  от  того,  принимали  ли  центром  системы  Землю  или
какое-то другое небесное тело, это движение предполагалось по кругам с
равномерной  скоростью;  7)  вторым  этапом было  движение  по  точным
эллипсам;  8) наконец,  при принятии  взаимодействия небесных  тел это
движение  происходит по  более сложным  кривым, первым  приближением к
которым являются эллипсы.

\subsection{Учение о шарообразности Земли связано с Пифагором}

Наука античной  Эллады развивалась,  конечно, в преемственной  связи с
предшествовавшими ей культурами Халдеи, Египта и Израиля. Влияние двух
других древнейших культур, Индии и Китая  --- спорно, и в нашем очерке
мы это можем игнорировать.

В  хронологии главнейших  астрономических дат  (Огородников, 1950,  с.
323) отмечены как первые четыре важнейших  даты: «ок. 3000 лет до н.э.
--- первые  зачатки астрономич.  наблюд. в  Китае, Египте  и Вавилоне;
1100 до н.э.  --- определение наклонения эклиптики  к экватору (Китай,
Чу  Конг);  VII-VI вв.  до  н.э.  ---  установление сароса  ---  цикла
солнечных  затмений (в  Вавилоне);  VI в.  до  н.э. ---  возникновение
учения о шарообразности Земли (греческий ученый Пифагор)».

Таким образом,  два важнейших астрономических открытия  --- наклонение
эклиптики к  экватору и  правильность в повторении  солнечных затмений
--- были сделаны  до греков, но учение о шарообразной  форме Земли ---
целиком  эллинское  открытие. Вполне  понятно,  что  этот шаг  сделать
труднее,  чем  первые два,  так  как,  как правильно  отмечает  Уэвель
(1867,  с.  190),  «...это  ---   первое  из  тех  убеждений,  которые
неопровержимо  доказывает астрономия,  хотя  они прямо  противоположны
видимому свидетельству  чувств. Объяснить людям,  что верх и  низ суть
только  различные  направления  в  разных местах;  что  море,  которое
кажется  таким плоским,  на  самом деле  выпукло;  что Земля,  которая
утверждена, по-видимому, на таком прочном  основании, на деле не имеет
никакого основания, --- все это  были великие победы, и силы открытия,
и силы убеждения. Мы легко согласимся  с этим, ежели вспомним, как еще
недавно считалось чудовищным  и еретическим учение об  антиподах или о
существовании  жителей  Земли,  которые находятся  на  противоположной
стороне ее и стоят ногами к нашим ногам».

Тот философ,  с которого  обычно начинают изложение  истории греческой
философии, Фалес  Милетский (624--547 гг.  до н.э.), еще  считал Землю
плоским  диском, плавающим  на воде.  Может быть,  это было  некоторым
шагом  вперед по  сравнению  с  египетскими представлениями,  согласно
которым  «Земля  представляет  собой окруженную  горами  продолговатую
впадину, по середине которой протекает  Нил; по окружающим эту впадину
горам  течет  небесный  Нил,   по  которому  ежедневно  плавает  барка
солнечного бога;  по четырем углам  мира водружены столбы,  на которых
покоится  плоское железное  небо. Такие  представления о  Земле как  о
впадине отчетливо заметны у  многих греческих мыслителей: у Демокрита,
Архелая  (ученик  Анаксагора  и  учитель Сократа),  вплоть  до  времен
Платона» (Веселовский, 1961, с. 15--16). Не следует понимать дело так,
что и  Демокрит и  Платон принимали такое  представление о  Земле. Как
увидим дальше, у них имеются только следы таких представлений.

Как  указывает  там же  Веселовский,  первенствующую  роль в  развитии
греческой культуры тогда играли италийские колонии («Великая Греция»),
а собственно в Греции еще господствовали египетские представления.

Но на Востоке, как указывает  там же Веселовский, возникло религиозное
представление об  яйцеобразном строении  мира. Оно существовало  уже у
древних персов и отчетливо  проявляется в орфических гимнах, восточное
происхождение которых является несомненным.  Так как Пифагор считается
реформатором  орфической религии,  то  становится  понятным, что  этой
религиозной идее он придал научную форму.

Разные  источники   по-разному  решают   вопрос  о  том,   кто  первый
высказался в  пользу сферичности  Земли. Это  некоторыми приписывается
тому  же Фалесу  (Веселовский, с.  17). Ученику  Фалеса, Анаксимандру,
приписывается  мнение, что  Земля  имеет вид  столба  или цилиндра,  а
также,  что он  говорил о  ее  шарообразности (Уэвель,  с. 190-  191).
Воззрения  древних  на  форму  Земли  резюмированы  Коперником  в  его
бессмертном труде (Коперник,  1947, с. 197): «Итак,  Земля не плоская,
как полагали Эмпедокл  и Анаксимен; не тимпановидная,  как у Левкиппа,
не чашевидная,  как у Гераклита,  не какая-либо иначе вогнутая,  как у
Демокрита,  а  также  не  цилиндрическая, как  у  Анаксимандра,  и  не
опирается нижнею  частью на  бесконечно глубокое и  толстое основание,
как у  Ксенофана, но совершенно  круглая, какой ее  считали философы».
Под  именем  «философов»  Коперник, очевидно,  считает  представителей
пифагорейского направления, о чем подробнее в главе о Копернике.

Какие  доводы  были  в  пользу  шаровидности  Земли?  Они  изложены  у
Аристотеля,   который  здесь,   как  и   во  многих   других  случаях,
систематизировал и  резюмировал развитие эллинской мысли.  Два главных
аргумента:  1)  изменение  положения  звезд над  горизонтом  в  разных
странах; 2) выпуклость разделительной  линии при лунных затмениях, так
как тогда уже  отчетливо была понята природа  лунных затмений (Уэвель,
с. 192).

\subsection{Характерные    черты    пифагореизма:    мистика    чисел,
математизация науки, первичность Космоса (порядок, красота), холизм}

Мы приходим  к той  легендарной фигуре,  имя которой  известно каждому
школьнику  и которая  и сейчас  вызывает большие  споры, ---  Пифагору
(даты  жизни  предположительно  571--497  гг. до  н.э.).  Споры  идут,
главным образом, о  том, можно ли приписать Пифагору  все те открытия,
которые  ему приписываются.  В данном  случае  вопрос для  нас не  так
актуален, так  как в  большинстве случаев считается,  что то  или иное
открытие сделано не самим Пифагором,  а одним из пифагорейцев, которые
в порядке уважения к учителю (уже умершему в их времена) приписали ему
свое открытие, или считается,  что историки науки переоценили значение
Пифагора. Что такая переоценка имела место, это невозможно оспаривать.
Например,  Плутарх  считал  Пифагора  автором открытия,  что  круг,  в
котором Солнце движется между звездами, лежит наклонно к тем кругам, в
которых звезды движутся около полюса  (Уэвель, с. 136). Уэвель резонно
говорит (и это сейчас, видимо,  общепринято), что это открытие сделано
египтянами или халдеями. Еще удивительнее  указание того же Уэвеля (с.
290), что, по мнению Шаля  (1839), наша современная десятичная система
не является ни арабской и ни  индийской, а происходит от Пифагора или,
по крайней мере,  от пифагорейской школы. Шаль  основывает свое мнение
на рукописи  Боэция (480 524  гг. после н.э.) «Абак  или пифагорейская
таблица», а также  на работах Герберта (папа Сильвестр  II, около 1000
г.).

В  современной советской  литературе  не  отрицаются огромные  заслуги
Пифагора и его  школы. Не говоря уже о  математике, громадное значение
они  имели в  развитии музыки  и в  небесной механике,  называемой ими
сферикой (История  философии, 1941,  с. 46).  Я уже  цитировал высокое
мнение  Энгельса  (гл.  III,  3).  О  роли  Пифагора  и  его  школы  в
математизации  науки и  философии  говорит и  Бернал  (1956, с.  104):
«Независимо от  того, был ли  Пифагор целиком легендарной  фигурой или
нет, школа,  носившая его  имя, была  достаточно реальной  и оказывала
огромное  влияние  в  более  поздние времена,  особенно  благодаря  ее
наиболее  выдающемуся  представителю  ---  Платону  (427--347  гг.  до
н.  э.). В  пифагорейском  учении сочетаются  две  тенденции идей  ---
математическая  и мистическая.  Неизвестно, какая  часть пифагорейской
математики  была создана  самим  Пифагором... Но  независимо от  того,
был  ли  Пифагор зачинателем  этого  учения  или только  передатчиком,
установленная его школой связь  между математикой, наукой и философией
никогда уже не утрачивалась».

Положительный   вклад   пифагорейцев,    таким   образом,   никем   не
оспаривается,   но    они   упрекаются   в    «мистике»,   религиозном
обосновании  своих воззрений.  Вот  если бы  они были  материалистами,
антирелигиозниками,  то  совсем было  бы  хорошо.  Так ли  это?  Можно
ли  отделить  в  пифагореизме   их  математическую  и  их  мистическую
устремленность?  Посмотрим,   в  чем  состояла   основная  философская
установка Пифагора,  его «мистика чисел». Совершенно  несомненно одно:
«Число есть  сущность всех вещей»,  но, кроме того,  ему приписывается
введение  в  философию  двух  понятий:  философ  и  Космос.  «Философ»
(буквально  любитель мудрости,  любомудр  в русском  переводе) как  бы
означало  скромность  притязаний   на  мудрость,  в  противоположность
другому  понятию  «софист»,  или  мудрец.  Мудрец  считается  нашедшим
истину, любитель мудрости ее ищет.

Космос вовсе не синоним Вселенной. Первоначальный смысл слова «космос»
---  украшение, красота.  Отсюда  ---  косметика, искусство  украшения
(подобно  тому  как  кибернетика  ---  искусство  управления).  Отсюда
--- родственные  понятия порядка,  гармонии, симметрии  (Энцикл. слов.
Брокгауза  и  Ефрона,  1895,  т.   XVI,  с.  379).  Называя  Вселенную
«Космосом»,   Пифагор  тем   самым   выдвигал  постулат   первичности,
объективности  красоты,  гармонии и  порядка.  Вселенная  не хаос,  из
которого путем борьбы частей  возникает нечто упорядоченное. Напротив,
порядок  и   есть  нечто  первичное.   Точно  так  же  и   красота  не
есть  нечто  субъективное,  а  она  есть  вполне  объективный  атрибут
природы,  а следовательно,  она  подчиняется закономерностям,  могущим
быть открытыми  человеком. Поэтому  утверждение об  открытии Пифагором
числовых закономерностей  в длинах струн, вызывающих  гармонию, вполне
гармонирует с этой философией.  Становится вполне понятным, что, когда
пифагорейская  философия  считалась  совершенно  устаревшей,  никакого
прогресса объективная эстетика не сделала. Понятие Космоса как красоты
естественно ведет к целостному,  холистическому представлению о мире и
к  догадке,  что лежащие  в  основе  движения планет  законы  доступны
простой математической  формулировке. Но доступные  прямому наблюдению
небесные тела  (Солнце, Луна) круглы,  круглой оказалась и  Земля, как
указано  выше. Поэтому,  естественно, обобщение,  что круг  и шар  ---
наиболее совершенные тела, что  вполне соответствует их математической
природе,   ---   исходный   пункт  всех   дальнейших   космологических
исследований.

Но все  дальнейшее развитие космологических  представлений показывает,
что  эта  «мистическая  установка»  вовсе не  была  жестким  догматом,
стеснявшим дальнейшее  исследование. Вся  она сводилась  к следующему:
мир есть нечто целое, законы его доступны математической формулировке;
но математические формулы должны  проверяться сопоставлением с данными
наблюдений.  Конечно, те  или  иные основные  положения нельзя  менять
постоянно,  но когда  встречается  необходимость, конкретные  основные
предположения должны  подвергаться ревизии. Должна  сохраняться только
твердая  уверенность, что  мир построен  разумно и  что человек  может
проникнуть в тайны  природы и сформулировать их  в виде математических
законов. Религиозная форма такого  убеждения не ослабляет, а усиливает
интеллектуальную мощь талантливого  человека: об этом свидетельствует,
как я постараюсь показать, вся дальнейшая история.

\subsection{Связь   Пифагора   с   космической   религией,   возможный
предшественник --- Соломон}

Были ли  предшественники у пифагорейцев в  их «мистико-математическом»
понимании мира?  По-видимому, да.  Такое понимание  мира соответствует
тому, что  Эйнштейн называет  третьей ступенью  религиозного сознания,
космической  религией (первые  две ступени:  религия страха  и религия
социально-моральная). Замечательная работа Эйнштейна «Мир, каким я его
вижу» мне, к сожалению, известна лишь в небольших отрывках. Источником
космической  религии,  по  Эйнштейну, является  сознание  совершенного
устройства и  удивительного порядка,  обнаруживаемых в  природе, равно
как и в  мире мысли. С самых первых стадий  развития религий, например
во многих псалмах

Давида, как и  у некоторых пророков, можно  найти элементы космической
религии.  Более  ясные  черты   этого  рода  религиозности  имеются  в
буддизме,  о  чем, в  частности,  нам  говорят сочинения  Шопенгауэра.
Величайший ученый XX в.,  вопреки господствующему мнению, считает, что
космическая религия --- наиболее могущественный и наиболее благородный
стимул  для научных  исследований. Эйнштейн  считает, что  космическая
религия не  знает ни  догматов, ни Бога,  имеющего образ  человека, не
может быть также основанием церкви. Представителей космической религии
Эйнштейн  видит  в еретиках  всех  времен,  которых одни  современники
считают атеистами, других ---  святыми. Поэтому Эйнштейн полагает, что
такие люди,  как Демокрит,  Франциск Ассизский  и Спиноза,  могут быть
поставлены рядом друг с другом.

К   этим   замечательным    словам   величайшего   и   благороднейшего
представителя  науки  XX  в.  нельзя, однако,  не  добавить  несколько
критических замечаний. Во-первых, нельзя категорически утверждать, что
космическая  религия не  может  служить основанием  для церкви.  Такой
первой «церковью» был  замечательный пифагорейский союз, обнаруживший,
несмотря  на преследования,  удивительную  живучесть  и идеи  которого
были затем  унаследованы рядом выдающихся  представителей христианской
церкви, отнюдь не считавшихся еретиками. Это мы увидим в дальнейшем. С
другой  стороны,  догматизм приносит  не  меньший,  а, вероятно,  даже
больший  вред, когда  он связан  с якобы  чисто научными  постулатами.
Во-вторых,   космическая  религия   в  самом   широком  понимании   не
обязательно связана с математическим  и холистическим пониманием мира.
Если и можно отнести Демокрита и Франциска Ассизского к представителям
космической религии,  то их  никак нельзя назвать  (вернее, Демокрита,
взгляды  Франциска   мне  неизвестны)   представителями  пифагореизма.
Взгляды буддистов  мне также неизвестны,  но мне хорошо  известно, что
Шопенгауэр, большой пропагандист буддийской философии, был решительным
противником  математизации науки  и  считал  вообще математику  наукой
низшего  ранга. И  наконец, в-третьих,  удивительно то,  что Эйнштейн,
хорошо  знакомый по  своему происхождению  с  Библией, нашел  в ней  в
качестве предшественников космической религии только Давида и пророков
и  упустил самого  важного, которого  действительно можно,  больше чем
кого-либо другого, считать предшественником Пифагора --- царя Соломона
Давидовича. О времени царствования его имеются разногласия: по Энцикл.
слов. Брокгауза и  Ефрона, статья «Соломон», т. XXX, 1900,  с. 811, он
царствовал  40 лет,  1020--980  гг. до  н.э.,  а по  БСЭ,  2 изд.,  т.
40,  1957, с.  47,  всего 25  лет, 960--935  гг.,  но, несомненно,  он
жил  много  раньше Пифагора.  А  в  его «Книге  премудрости»  (забудем
пока, что  принадлежность этой книги Соломону  оспаривается некоторыми
гелертерами)  есть такое  замечательное изречение  о Боге:  «вся мерою
и  числом  и  весом  расположил еси»  (цитирую  славянский  текст),  а
далее Соломон  излагает краткий  обзор известного в  то время  о мире:
составление мира и  действие стихий, начало, конец  и середину времен,
лет  круги и  звезд  расположение, естество  животных  и гнев  зверей,
ветров усилие  и помышление  человека и  т.д. Соломону,  таким образом
(или тому  человеку, который был истинным  автором «Книги премудрости»
Соломона),  были совершенно  ясны  и важность  тех  знаний о  природе,
которые, как он считал, он  получил от Бога, и математический характер
законов природы. То, что Эйнштейн  упустил Соломона, делает понятным и
то, что  он не упомянул  Пифагора в качестве  важнейшего представителя
древнейшей космической религии.

\subsection{Неоспоримость  роли религиозных  представлений в  развитии
космологии, оригинальность пифагореизма}

То,  что  богословские соображения  играли  огромную  роль в  развитии
космологических представлений, признают  и противники «линии Платона».
Так, С.Я.Лурье пишет  (1947, с. 333--334): «К  учению о шарообразности
Земли  пифагорейцы пришли  уже до  Демокрита из  чисто эстетических  и
богословских соображений  (шар --- самое  совершенное из тел).  Но эта
теория случайно  оказалась правильной  и дала  позднейшим пифагорейцам
возможность внести ряд улучшений  в астрономическую картину Демокрита.
Точно так  же представление  о том,  что Земля,  как и  другие планеты
(Солнце  и  звезды),  вращается вокруг  какого-то  центрального  огня,
основано   на  примитивно   метафизических  соображениях;   достаточно
сказать, что каждое  из этих светил мыслится как бы  прибитым к сфере;
вся эта сфера  вращается, причем издает гармонические звуки  и т.п. Но
смелая  попытка  вывести  Землю  из ее  неподвижности  имела  огромное
значение,  и  впоследствии  Аристарх  Самосский,  заменив  центральный
огонь Солнцем,  пришел к  системе, явившейся  предшественницей системы
Коперника (учение  о вращении Земли вокруг  оси, возможно, содержалось
уже в учении Демокрита, выше, с. 208)».

Мы  видим, таким  образом,  что даже  ожесточеннейший противник  линии
Платона и,  вероятно, лучший знаток  Демокрита в наше  время принужден
признать,  что  основные  достижения  гелиоцентрической  теории  имели
место  на линии  Пифагора, но  пытается апеллировать  к «случаю».  Эта
апелляция к случаю характерна для всей линии Демокрита и в особенности
для  современных  демокритовцев,   неодарвинистов.  Но  у  дарвинистов
все-таки естественный отбор отбирает  нечто упорядоченное из огромного
количества  случайных, неупорядоченных  изменений, а  по мнению  С. Я.
Лурье,  такая поразительная  идейная  конструкция как  космологическая
система,  имевшая   ряд  последовательных  этапов,   могла  возникнуть
«случайно»  при  наличии  очень   ограниченного  числа  мыслителей.  И
почему-то,  как   увидим  дальше,  такие  замечательные   «случаи»  не
возникали  на линии  Демокрита.  Видимо,  современные материалисты  не
менее склонны верить в абсурд, чем столь презираемый ими Тертуллиан.

Не отрицая значения пифагорейской школы, С. Я. Лурье пытается вслед за
Эрихом Франком  доказать, что сам  Пифагор сделал гораздо  меньше, чем
ему приписывают  пифагорейцы, и  не одни пифагорейцы.  На с.  32 книги
Лурье (1947)  мы с  удивлением читаем, что  среди ученых,  стоявших на
высоте науки начала  IV в., возникла «мода»  приписывать свои открытия
Пифагору и что всего удивительнее, этой моде подчинились и Эмпедокл, и
даже  Демокрит. Понятно,  что  пифагорейцы  приписывали свои  открытия
главе  школы, Пифагору.  Еще можно  примириться с  тем, что  это делал
Эмпедокл.  Но какая  нужда была  Демокриту некоторые  из своих  учений
приписывать  своему  противнику? Политическое  значение  пифагорейский
союз того времени  имел только в Великой Греции, да  и там пифагорейцы
неоднократно  подвергались погромам  со стороны  невежественной черни.
Мне неизвестно,  какие свои  учения Демокрит приписывал  Пифагору, но,
если это  было так,  то, мне думается,  наиболее вероятными  будут два
следующих  предположения: 1)  или  эти  учения Демокрит  действительно
заимствовал  от Пифагора  и,  как  честный ученый,  не  желал их  себе
приписать;  2) или  же это  были настолько  оригинальные взгляды,  что
надо  было  прикрыться  великим  авторитетом Пифагора  для  их  лучшей
популяризации.  Лурье  указывает,  что  никто  из  ученых,  живших  до
Демокрита, ничего  не сообщает о математических  открытиях Пифагора, а
говорят только о религиозных его воззрениях, о переселении душ. Но кто
эти ученые?  Лурье приводит:  Геродот, Ксенофан, Гераклит,  Эмпедокл и
Ион Хиосский,  с. 32: «Поэтому к  додемокритовскому пифагореизму можно
относить  только учение  о  шарообразности Земли,  с которым  Демокрит
уже  полемизировал,  установление  математических  зависимостей  между
музыкальными звуками  и, может  быть, учение  о вращении  Земли вокруг
центрального  огня. Однако  и эти  теории не  могли возникнуть  раньше
второй  половины V  в. К  эпохе  же Пифагора  могут относиться  только
мистическая спекуляция числами и  учение о противоположностях» (курсив
Лурье.  ---  А.Л.).  Из  этого  отрывка мы  узнаем,  что  Демокрит  не
принимал  шарообразности Земли  (о  его  собственных воззрениях  будет
дальше).  Но  любопытнее  тот  метод,  которым  пользуется  Лурье  для
доказательства своего положения. Он приводит без разбора всех античных
авторов  независимо  от  их  взглядов и  специальности.  Геродот  был,
бесспорно, великий историк, но не  имевший никаких заслуг в астрономии
и,  вероятно, не  следивший  за успехами  этой  науки. Еще  любопытнее
ссылка  на Гераклита.  Собственные астрономические  взгляды Гераклита,
как  увидим дальше,  были до  крайности примитивными.  Отрывок же  40,
на  который ссылается  Лурье, полностью  гласит (Материалисты  Древней
Греции, 1955, с. 45): «Многознание не научает быть умным, иначе бы оно
научило Гезиода  и Пифагора, а  также Ксенофана и  Гекатея». Очевидно,
всех своих идейных противников Гераклит не считает умными, несмотря на
их многознание:  но все  четыре имени  и сейчас  пользуются уважением.
Но  зачем   же  разбираться  в  мнениях   своих  неумных  противников?
Гераклит,  как известно,  не  был многоречив.  Гекатей был  историком,
предшественником  Геродота.  Поэтому   их  свидетельства  в  отношении
Пифагора не имеют решительно никакой цены.  Но если бы даже были правы
Франк и примыкающий к нему Лурье, что все так называемые пифагорейские
учения были  реакцией на учение  Демокрита, то бесспорной  является та
заслуга  Пифагора,  над  которой  так  издеваются  наши  материалисты:
мистическая  спекуляция   числами,  гармонические   звуки,  издаваемые
вращающимися небесными сферами, и проч. Ведь именно эти идеи и двигали
мысль  последующих космологов  вплоть до  Кеплера. Может  быть, это  и
неверная  идея, но  зато  какая плодотворная.  Мы  увидим дальше,  что
материалисты,  напротив, несмотря  на то  что им  сейчас выдан  патент
единственно правильного учения,  оказались совершенно бесплодны вплоть
до эпохи Ньютона.

\subsection{Первая негеоцентрическая  система ---  система пифагорейца
Филолая --- пироцентрическая (вокруг центрального огня)}

Шарообразность  Земли  была  установлена   если  не  самим  Пифагором,
то  скоро  после  него  и   вошла  прочно  в  систему  астрономических
воззрений  древних  греков.  Следующие этапы:  отказ  от  геоцентризма
и  принятие  вращения  Земли  вокруг оси.  По  свидетельству  Плутарха
(вернее  псевдо-Плутарха),  которого  цитируют  Плиний  и  Цицерон,  и
Диогена Лаэрция (см. Ибервег-Гейнце,  т. 1, с. 63), Филолай-пифагореец
полагал, что Земля вращается кругом огня по наклонной круговой орбите,
Никет-сиракузянин (у Цицерона Гикет), Гераклит Понтийский и пифагореец
Экфант  принимали  вращение Земли  вокруг  оси  и неподвижность  звезд
(Никет  принимал  неподвижность не  только  Солнца,  но и  Луны)  (см.
Веселовский,  1961, с.  13, 14  и 18).  Всех этих  авторов в  качестве
своих  предшественников  приводил  и  Коперник  (1947,  с.  191--192).
Сейчас  пытаются  отрицать реальность  Филолая,  Гикета  и Экфанта  (в
существовании Гераклита  Понтийского, кажется, никто  не сомневается).
Н. Таннери  пытается отрицать  существование Экфанта и  Гикета, считая
их  просто персонажами  диалогов  Гераклита Понтийского,  с чем  готов
согласиться Веселовский.  Франк отрицает  и существование  Филолая, от
сочинения  которого остались  только  отрывки;  этот вывод  оспаривает
Веселовский (с.  14). Но если  и не было  Экфанта и Никета,  то кто-то
высказал эти  замечательные мысли, и для  нас важно, что все  они были
пифагорейцами.  О Гераклите  Понтийском, ученике  Платона, речь  будет
дальше. Он дал дальнейшее развитие  теории. Пока же, игнорируя Экфанта
и Никета,  примем, что Филолаю  принадлежат и мнение о  вращении Земли
вокруг оси, и  мнение о ее движении вокруг центрального  огня. Этим мы
просто повторим мнение Коперника  (1947, с. 199--200): «Действительно,
о том, что Земля вращается и даже различным образом блуждает, и о том,
что  она принадлежит  к  числу светил,  утверждал пифагореец  Филолай,
столь недюжинный математик,  что именно ради свидания с  ним Платон не
замедлил отправиться в Италию, как передают жизнеописатели Платона».

Филолай, ученик  Пифагора, был  примерно современником Сократа  и, как
видно из  последнего указания, Платона.  Его ученики, Симмий  и Кебес,
согласно Платону  (Федон), были друзьями Сократа  и присутствовали при
смерти этого мученика  за свободу мысли. Филолаю  приписывают и первое
письменное  изложение  системы  Пифагора  (Ибервег-Гейнце,  т.  1,  с.
57).  Детали  системы  Филолая  спорны  (Веселовский,  с.  18).  Самое
существенное  в  ней  ---  вращение Земли  вокруг  центрального  огня.
Солнце  сияет светом,  отраженным  от центрального  огня. Кроме  Земли
было другое  тело, контр-Земля,  находящееся на  том же  расстоянии от
центрального  огня.  Как пишет  Б.Рассел  (1959,  с. 234),  для  такой
странной, на  наш взгляд,  теории у Филолая  было два  основания: одно
научное, а другое ---  проистекавшее из их арифметического мистицизма.
Научным основанием служило правильное  наблюдение, что лунное затмение
временами происходит и  тогда, когда и Солнце и  Луна вместе находятся
над  горизонтом. Преломление  лучей (рефракция),  составляющее причину
этого явления, было  им неизвестно, и они думали, что  в таких случаях
затмение должно вызываться  тенью какого-то другого тела,  а не Земли.
Вторым основанием служило  то, что Солнце и Луна,  пять планет, Земля,
контр-Земля и  «центральный огонь»  составляли десять небесных  тел, а
десять было мистическим числом  у пифагорейцев. Рассел прибавляет, что
хотя эта  теория «в определенной  степени совершенно ненаучна,  но она
очень  важна,  поскольку включает  в  себя  большую часть  тех  усилий
воображения, которые понадобились, чтоб зародилась гипотеза Коперника.
Начать думать о  Земле не как о  центре Вселенной, но как  об одной из
планет,  не  как о  навек  прикрепленной  к  одному  месту, но  как  о
блуждающей в пространстве, --- свидетельство необычайного освобождения
от  антропоцентрического мышления.  Когда  был  нанесен удар  стихийно
сложившимся  представлениям человека  о Вселенной,  было не  столь уже
трудно при  помощи научных аргументов  прийти к более  точной теории».
Гипотеза Филолая  о центральном  огне не  сделалась догматом  и вскоре
после Платона была отброшена самими пифагорейцами.

\subsection{«Академический»     этап     космологии     ---     теория
гомоцентрических сфер Эвдокса, первый  шаг к гелиоцентрической системе
Гераклита Понтийского}

Система Филолая  перестала быть  геоцентрической, но не  сделалась еще
гелиоцентрической. Но почему же люди  не видят центрального огня, если
Земля вокруг него вращается? По мнению Филолая, это происходит потому,
что  Земля  вращается  вокруг  центрального  огня  так  же,  как  Луна
вращается вокруг Земли, т.е., будучи постоянно обращена к центральному
телу одной  своей стороной.  Этот довод  Филолай, таким  образом, тоже
черпал из опыта, так как  единственный хорошо известный спутник, Луна,
соответствует установке Филолая. Сейчас вошло в моду видеть в малейшем
сходстве  с  современными  теориями уже  предшественников  современных
теорий или  доказательство того, что  древняя наука была  гораздо выше
того уровня, который обычно принимается.  Следуя этой моде, можно было
бы признать Филолая предшественником Джорджа Дарвина (сына Ч.Дарвина),
выдающегося математика и космолога XIX  в., который доказывал, что под
действием приливов и отливов каждый спутник в конце концов оказывается
обращенным одной  стороной к тому  телу, вокруг которого  он вращается
(это имеет  место, насколько мне  известно, не только  у Луны, но  и у
Меркурия).

Вращение  Земли вокруг  центрального  огня  предполагало ее  обращение
вокруг  своей  оси  за  тот  же промежуток  времени,  за  который  она
обращается  вокруг  центрального  огня.   Видимо,  давно  уже  древние
установили  громадное   расстояние  светил   от  Земли,  и   в  пользу
неподвижности  большинства звезд  говорило  то  обстоятельство, что  в
противном  случае  пришлось   бы  признать  совершенно  исключительную
скорость  вращения небесного  свода. Этот  «аргумент Филолая»  в своей
книге  повторяет Коперник.  Его  приводит и  Ревзин  (1949) в  истории
гелиоцентрического учения.

Отвергнув теорию центрального огня,  более поздние пифагорейцы вернули
Земле ее центральное положение. Для  того чтобы возникла уже настоящая
гелиоцентрическая  система  Аристарха,  потребовалось  много  времени.
Но  мысль  греческих   астрономов  не  стояла  на   месте.  Этот  этап
греческой  космологии  целиком  связан с  платоновской  Академией.  Он
является полным  выражением того  завета, который,  согласно Порфирию,
Платон поставил  перед своими  учениками: установить,  «какие гипотезы
равномерных и упорядоченных движений надо сформулировать так, чтобы их
следствия не противоречили  бы явлениям» (История философии,  т. 1, с.
252).  В этой  установке сохранились  только пифагорейские  требования
равномерности  движений и  их  упорядоченности  (понимаемой тогда  как
круговые  движения),  все  же  остальное должно  было  быть  подобрано
так,  чтобы получилось  соответствие с  явлениями. Этому  соответствию
предъявлялись все большие требования в смысле точности и потому теории
постепенно  совершенствовались. Можно  различать по  крайней мере  три
главных направления в развитии космологии: 1) теория «гомоцентрических
сфер» Евдокса, т.е. сфер, имеющих общий  центр с Землей, она привела к
системе Аристотеля;  2) теория  Гераклита Понтийского:  вращение Земли
вокруг оси и первый шаг по направлению к гелиоцентрической системе ---
Меркурий  и  Венера  вращаются  вокруг  Солнца;  эта  теория  является
прообразом  системы Тихо  Браге и  вместе  с тем  переходом к  системе
Аристарха  Самосского, Коперника  древнего мира;  3) отказ  от строгой
гомоцентричности ---  теория эпициклов  и эксцентрических  кругов, что
привело  к  построению  величайшего достижения  астрономической  мысли
Древней Греции --- системе  Птолемея. Все три направления, несомненно,
родились в  школе Платона,  но дальнейшее  развитие протекало  в более
поздние времена.

Следует разобрать их более подробно.

\subsection{Развитие теории гомоцентрических сфер школой Аристотеля}

Евдокс  (или  Эвдокс,   пишут  по-разному)  жил  в   409--356  гг.  до
н.э.  Молодость провел  в  школе Платона,  а  затем после  путешествий
обосновался  в   родном  городе   Книде  на  юго-западе   Малой  Азии.
Его  математические  заслуги  упомянуты  в  главе  III,  он  является,
бесспорно, одним  из величайших математиков античного  мира, по мнению
Бернала  (1956,  с. 109),  величайшим  греческим  математиком и  столь
же  великим астрономом.  Развивая картину  мира, созданную  Пифагором,
Евдокс пытался  объяснить движение  Солнца, Луны  и планет  при помощи
ряда  концентрических сфер,  причем каждое  из этих  тел вращалось  на
оси,  закрепленной в  сфере,  находящейся  вне ее.  От  этой грубой  и
механической модели  происходят все астрономические приборы  вплоть до
приборов настоящего времени. Видимо, на основе теории Евдокса построил
и  свою  знаменитую  «сферу»  Архимед. Эта  сфера  считалась  шедевром
Архимеда и понятно, что победитель Сиракуз, Марцелл, воспользовался ею
как единственным  трофеем после  взятия Сиракуз. На  сфере, приводимой
в  движение, по-видимому,  водяным  двигателем,  можно было  наблюдать
движение Луны  и Солнца, солнечные  затмения и проч. (Лурье,  1945, с.
65--66).

Система Евдокса  оказывается уже  очень сложной.  Одна сфера  была для
неподвижных звезд,  но для каждой планеты  приходилось строить систему
сфер, как бы  вложенных один в другой детских  деревянных шариков. Все
три  постулата сохранялись:  строгая  гомоцентричность сфер,  круговые
движения строго равномерные. Для суточного и других движений строилась
особая сфера и  всего получилось по три  сферы для Луны и  Солнца и по
четыре ---  для пяти планет,  а всего  27 сфер (Веселовский,  1961, с.
20--21).

Для своего времени система  Евдокса была значительным достижением, так
как позволяла  рассчитывать многие  движения планет, но,  конечно, для
дальнейшего  развития  математической  космологии  строгое  соблюдение
всех  трех постулатов  послужило непреодолимым  препятствием, поэтому,
например, в  своей содержательной  статье по истории  планетных теорий
Н.И.Идельсон (1947,  с. 85)  считает эту систему  фатально непригодной
для  астронома-теоретика... Идельсон  считает,  что настоящая  древняя
астрономическая наука начинается с Гиппарха и Птолемея.

Дальнейшая разработка теории Евдокса уже связана со школой Аристотеля,
т.е.  той   школой,  которая  во  многих   своих  основных  положениях
решительно порвала  с пифагорейско-платоновским  направлением... Число
сфер Аристотелем  и его учеником  Калиппом было доведено до  55. Школа
после  смерти  основателя  утратила  в  астрономии  свое  значение.  В
продолжение  александрийского периода  влияние  школы Аристотеля  было
слабым  и  возродилось  в  эпоху  Римской  империи,  когда  ---  после
Августа  ---  восторжествовали   консервативные  течения,  а  традиции
александрийской науки были в значительной степени забыты (Веселовский,
1961, с.  21). В  Средние века с  торжеством учения  Аристотеля теория
гомоцентрических  сфер  приобретает   вновь  значение  (Аверроэс);  ее
выдвигали, хотя  и безуспешно, против теории  Птолемея. Часто полагают
(и  в этой  путанице виновен,  между прочим,  и Галилей),  что системы
Аристотеля и Птолемея идентичны. На самом деле сходство у них только в
том,  что  обе  они  ---  геоцентричны, но  в  то  время  как  система
Аристотеля  не способна  дать  точное описание,  а  взамен этого  дает
видимость  физического  «объяснения»,  система  Птолемея,  как  увидим
дальше, отказывается  от физической «понятности», но  дает несравненно
лучшее математическое описание.

\subsection{Гераклит Понтийский --- предшественник Тихо Браге}

Второе  направление  связано с  именем  одного  из виднейших  учеников
Платона,  Гераклита  Понтийского  (род.  ок. 390,  умер  ок.  315--310
гг.  до н.  э).  Странными кажутся  попытки  наших философов  оспорить
причисление Гераклита, как и Евдокса, к платоникам (История философии,
1941, т.  1, с.  252). Нашим философам  вообще привычно,  что малейшее
изменение  общепринятых   взглядов  является  уже   ревизионизмом,  но
вся  пифагорейско-платоновская линия  и  характерна  тем, что  ученые,
сохраняя общее  понимание мира, непрерывно ревизируют,  в соответствии
с  накоплением   данных  конкретных   теорий,  и   усовершенствуют  их
математический аппарат.  Идейная же близость Гераклита  и Платона явна
уже из того,  что именно Гераклиту (иностранцу,  не афинянину!) Платон
поручил  руководство  Академией во  время  своей  последней поездки  в
Сицилию (Ибервег-Гейнце, 1894, с.  192). Гераклит продолжал работать в
Академии  и  после  смерти  Платона  и в  339  г.  намечался  на  пост
руководителя  Академии  вместо умершего  Спевсиппа,  непосредственного
преемника Платона,  но потерпел неудачу  на выборах, уехал из  Афин на
родину,  в Гераклею  на  Мраморном  море. В  своих  диалогах он  много
говорит о судьбе  души после смерти, т.е. и в  этом отношении является
прямым продолжателем Платона. В  отличие от Евдокса, Гераклит развивал
филолаевское представление о вращении Земли вокруг своей оси, принимал
бесконечность  мира (Ибервег-Гейнце,  с. 192)  и, что  самое, пожалуй,
важное, сделал  первый шаг (если  и его  не сделал спорный  Экфант) по
пути  к созданию  гелиоцентрической  системы мира.  В системе  Евдокса
был  тот существенный  недостаток,  что как  бы  ни усложнять  систему
гомоцентрических сфер, все планеты,  постоянно двигаясь по поверхности
одной и  той же сферы,  должны находиться на постоянном  расстоянии от
Земли. А наблюдения Венеры и Марса,  т.е. как раз тех планет, движение
которых,  по  Евдоксу,  не  получает  удовлетворительного  объяснения,
показывает,  что блеск  этих  планет меняется  (Веселовский, 1961,  с.
22).  Вероятно,  конечно,  сторонники системы  Аристотеля  придумывали
какие-нибудь объяснения  этому явлению, сохраняя  гипотезу одинакового
расстояния, но  проще всего  было предположить,  что расстояние  их от
Земли меняется. Это соображение и  некоторые другие и привели к мысли,
что центром движения внутренних планет  --- Меркурия и Венеры является
Солнце.  Гераклит,  видимо,  ограничился  Венерой  и  Меркурием...  Но
естественно распространение этой гипотезы на «менее послушную» планету
--- Марс,  а затем и на  Юпитер и Сатурн: получается  теория планетных
движений, предложенная гораздо  позже Тихо Браге. В  античном мире был
сделан, наконец, и последний шаг:  признать Солнце центром движения не
только от всех пяти планет, но и Земли --- Аристарх Самосский.

Гипотеза  Гераклита  о  вращении   Меркурия  и  Венеры  вокруг  Солнца
держалась  довольно долго:  Плиний Старший,.  Витрувий, друг  Цицерона
Варрон  и др.,  и только  в XIII  в. была  вытеснена теорией  Птолемея
(Веселовский, 1961, с. 24).

\subsection{Аристарх --- Коперник античного  мира --- был пифагорейцем
и не получил признания по причинам, не связанным с идеологией}

Мы подошли к Аристарху, Копернику  древнего мира. Ему посвящена работа
И.Н. Веселовского  (1961). К сожалению, эта  работа Аристарха, которая
дает ему  право считаться основоположником  гелиоцентрической системы,
не дошла до  нас. О том, что он был  ее автором, имеются свидетельства
Архимеда, Плутарха,  Аэция. Их высказывания, а  также некоторые данные
у  Птолемея  свидетельствуют,  что теория  Аристарха  была  достаточно
разработанной (Веселовский,  с. 61--64).  Единственное же  дошедшее до
нас полностью сочинение Аристарха «О  величинах и расстояниях Солнца и
Луны» показывает,  что он  был, несомненно,  крупным астрономом,  но в
момент  написания этой  работы  он  придерживался еще  геоцентрической
теории.

К какой  же школе относится  Аристарх? Коснемся кратко  его биографии.
Он  родился  на  родине  Пифагора,  острове Самосе  около  310  г.  до
н.э.,  получил образование  у преемника  Аристотеля Стратона,  который
одно  время  стоял  даже  во   главе  перипатетической  школы.  Кое  в
чем  Стратон отошел  от  взглядов Аристотеля,  например  в согласии  с
учением  Демокрита допускал  существование пустоты.  Стратон считается
основателем пневматики,  которая, вполне  в аристотелевском  духе дает
лишь  качественное объяснение,  без  математики.  Аристарх не  остался
подражателем  Стратона,  что  засвидетельствовано  римским  теоретиком
архитектуры  Витрувием:  «Те  же,   кто  обладают  от  природы  такими
способностями, сообразительностью и памятью,  что могут в совершенстве
постичь  геометрию, астрономию,  музыку  и прочие  науки, идут  дальше
того, что требуется архитекторам,  и становятся математиками; им легко
выступать  в  спорах по  этим  наукам,  потому  что они  во  всеоружии
многих знаний. Однако  подобные люди встречаются редко;  такими в свое
время были  Аристарх Самосский, Филолай и  Архит Тарентский, Аполлоний
Пергский, Эратосфен Киренейский и Архимед и Скопин из Сиракуз, которые
на основании многих вычислений и законов природы изобрели и разъяснили
для потомства множество  вещей в области механики  и устройства часов»
(Веселовский, с. 27--28). Из этого ясно, что Аристарх (как, впрочем, и
вся александрийская школа) оказался  под сильным влиянием пифагореизма
и в БСЭ, 2-е изд., т. 3,  1950, с. 5, в статье «Аристарх» пишут: «Хотя
Аристарх придерживался чисто умозрительных  и не вполне ясных взглядов
пифагорейцев, он поставил  Солнце, а не Землю в  центре Вселенной». Мы
знаем хорошо,  что пифагорейцы  отнюдь не были  чистыми «умозрителями»
и  именно  у  пифагорейцев  развилась  критика  геоцентризма,  поэтому
это  «хотя»  звучит  довольно  забавно. Сам  Аристарх  тоже  занимался
астрономическими наблюдениями в  Александрии. Но, конечно, «умозрение»
играло у  Аристарха, как у  всех пифагорейцев, большую роль.  В основе
его теории  было два  чисто пифагорейских положения  об обязательности
кругового и  равномерного движения, но,  кроме того, он  принимал, что
все планеты  должны вращаться вокруг центрального  материального тела,
что  было  свойственно  и  системе Филолая.  Таким  образом,  основные
постулаты у Аристарха заимствованы целиком от платоновской школы.

Как  отнеслись  к  Аристарху  современники? За  свое  учение  Аристарх
был  обвинен в  безбожии и  был вынужден  покинуть Афины.  Обвинителем
Аристарха был глава стоической  школы Клеанф, который возглавлял школу
после смерти  основателя Зенона (ок. 264  г. до н.э.) до  своей смерти
(ок. 232  г. до н.э.).  Аристарх был обвинен в  том, что он  сдвинул с
места Очаг Вселенной.

Какими мотивами  руководился Клеанф? Веселовский (с.  64--65) приводит
из сочинений Сенеки молитву стоиков  и на этом основании полагает, что
Клеанфом руководило  религиозное чувство  и стоическая  теория фатума,
подкрепляемая  астрологическими теориями  вавилонских астрономов.  Это
было начало завоевания западного мира восточной астрологией. Евдокс за
сто  лет  до  Аристарха  был противником  астрологии.  Во  времена  же
Аристарха приходилось считаться с требованиями астрологических теорий,
для  которых  Земля и  человечество  должны  были необходимо  занимать
центральное  положение во  Вселенной.  Мне не  кажется это  объяснение
достаточно убедительным,  но сам  Веселовский признает,  что нападения
астрологов  в то  время сами  по себе  были не  так опасны.  Во всяком
случае, Аристарх серьезных неприятностей не испытал.

Стоики, насколько  мне известно, никакими заслугами  в развитии точных
наук не прославились. Гораздо  интереснее мнение великого современника
Аристарха,  Архимеда (287--212  г. до  н.э.),  который был  на 27  лет
моложе Аристарха. Как уже было указано, Архимеду была известна система
Аристарха,  но  своего отношения  к  ней  он определенно  не  проявил.
Поэтому С. Я. Лурье считает (1945,  с. 61 и 62): «Позволительно думать
поэтому, что гениальный Архимед в душе сочувствовал теории Аристарха».
Лурье считает  возможным набросить тень на  научную честность великого
Архимеда, считая, что он свое истинное мнение скрывал, так как систему
Аристарха прокляла официальная философия. «Официальной философии» в то
время  вообще не  существовало, здесь  просто современность  отброшена
в  прошлое. В  то  время  конкурировали платоники-пифагорейцы  (считая
даже их  за нечто  единое, хотя  там были  разногласия), перипатетики,
стоики,  эпикурейцы. Академия  платоников,  Ликей перипатетиков,  Стоя
стоиков  и Сад  Эпикура  действовали  в одном  городе,  а уж  Сиракузы
совершенно  не  были подчинены  Восточной  Греции.  Прибавим еще,  что
Архимед был близким  родственником правителя Сиракуз. Даже  если бы он
был шкурником и  трусом, подобно многим современным ученым,  у него не
было  решительно никаких  оснований для  беспокойства за  свою судьбу,
если  бы  он  поддержал  Аристарха.  Позабыв  то,  что  он  написал  в
1945  г.,  тот  же  Лурье  в статье  «Архимед»  (1950)  отмечает,  что
Архимед  был  чужд угодничества  и  открыто  ссылался на  материалиста
Демокрита и отзывался  с сочувствием о системе  Аристарха, несмотря на
то  что господствовавшие  тогда идеалистические  воззрения осуждали  и
бойкотировали эти учения. Но ведь Аристарх-то был пифагореец, идеалист
и к  воображаемой идеалистической «официальной философии»  был гораздо
ближе,  чем  к  Демокриту  и Эпикуру  с  их  довольно  многочисленными
последователями, о преследованиях которых что-то неизвестно.

Не  был последователем  Аристарха  и выдающийся  астроном и  математик
Эратосфен (Лурье, 1945, с. 59), хотя идеологически они были близки.

У современников,  притом даже  самых выдающихся, система  Аристарха не
получила серьезного признания и  «ее считали еретической, абсурдной, с
точки зрения  философии и  противоречащей повседневному  опыту. Однако
она  осталась  устойчивой  ересью,  переданной  арабами,  возрожденной
Коперником  и активно  подтвержденной Галилеем,  Кеплером и  Ньютоном»
(Бернал, 1956, с. 128).

\subsection{Непризнание       Аристарха      целиком       объясняется
несвоевременностью с научной точки зрения его теории}

Чем же  тогда объяснить, что,  «несмотря на исключительную  простоту и
убедительность его  теории, Аристарх не нашел  ни одного последователя
не  только в  Александрии, но  и во  всем мире,  исключая одного  лишь
Селевка из Селевкии на Тигре»  (Лурье, 1945, с. 58). Система Аристарха
не  была  вовсе задавлена,  но  не  развивалась, несмотря  на  большое
уважение к  Аристарху. Ведь то  сочинение Аристарха, которое  дошло до
нас и где он придерживается еще геоцентрических взглядов, впоследствии
«попало в  ряд обязательных произведений, которые  начинающий астроном
должен был изучать  после окончания чтения Евклида и  до начала чтения
Птолемея» (Веселовский, с. 70).

В своей  книге о  Копернике Ревзин (с.  110) указывает  четыре причины
того,  что  идеи   Аристарха  так  долго  не   получали  развития:  1)
противно здравому смыслу; 2)  религиозный протест; 3) все возрастающий
авторитет  Аристотеля  и   4)  замечательное  математическое  развитие
геоцентрической  системы древних  (подразумевается, очевидно,  система
Гиппарха-Птолемея). Веселовский  (с. 66) отмечает влияние  стоицизма и
вавилонской вычислительной астрологии.  Если прибавить еще возможность
личных  мотивов,  то  мы   получим  следующие  возможные  причины:  1)
осуждение  личности; 2)  религиозные возражения;  3) политические;  4)
философия  стоиков;  5)  философия перипатетиков;  6)  астрология;  7)
здравый смысл; 8) система Птолемея. Разберем их по очереди.

Личность  Аристарха  не  претерпела  преследований,  кроме  истории  с
Клеанфом,  умер  он  в  230  г.  до  н.э.,  80  лет  от  роду,  и  его
геоцентрическое сочинение, как указано, очень одобрялось.

Религии   того   времени   были   многочисленны   и   разнообразны   и
определенных  догматов,  осуждающих  гелиоцентризм,  как  правило,  не
было.  Основоположник же  критики геоцентризма  Филолай был  в большом
почете  у  тогдашних  мыслителей.  Не забудем,  что  многие  восточные
религии обоготворяли  Солнце, и их  религия должна бы  поддерживать ту
теорию,  которая заставляет  Землю вращаться  вокруг Бога.  Мы слишком
привыкли  к господству  иудео-христианской  религии,  где Солнце,  так
сказать, идеологически подчинено Земле, и полагаем, что во все времена
религиозные люди  думали сходно с религиозными  людьми Средневековья и
более позднего периода.

Политика  того  времени была,  конечно,  бурной.  Шла жестокая  борьба
эллинистического  мира   с  жестокой  и  полуварварской   силой  Рима,
закончившаяся,  к  сожалению,  победой  Рима. Но  в  область  культуры
тогда,  насколько мне  известно, Рим  не вмешивался  и даже  позже для
реформы календаря Юлий Цезарь  вызвал из Александрии Созигена. Римские
правители вряд  ли могли  преследовать Аристарха прежде  всего потому,
что они его не понимали.

Если  стоики  и  были противниками  гелиоцентрического  мировоззрения,
то  они никогда  не  были  монополистами в  философии.  Также не  были
монополистами  перипатетики. Если  бы они  были монополистами,  то они
сумели  бы остановить  развитие  и  распространение системы  Птолемея,
враждебной системе  гомоцентрических сфер  Аристотеля... Даже  когда в
разгаре Средних  веков перипатетики  завоевали монополию  в философии,
они не сумели изгнать систему Птолемея, хотя и делали попытки. Об этом
несколько слов скажу потом.

Также невероятна эффективность  сопротивления астрологов. Во-первых, в
этом  времени  астрологи  только  начинали  завоевывать  господство  в
Западном мире, а во-вторых, господство астрологов в период Возрождения
не  помешало  ни  Копернику,   ни  Кеплеру,  которые  сами  занимались
астрологией, произвести революцию в астрономии.

Система  Птолемея  не могла  служить  препятствием,  так как  Птолемей
жид  после  Рождества  Христова  (90--168 гг.  н.э.)  и  даже  Гиппарх
(190--120 гг.  до н.э.)  жил после Аристарха.  Разумеется, медленность
развития астрономии от Гиппарха до Птолемея (который родился через 280
лет  после рождения  Гиппарха)  можно  объяснить политическим  упадком
Александрии после римского завоевания.

Никаких  удовлетворительных  причин   религиозного,  философского  или
политического характера для объяснения  неуспеха Аристарха мы найти не
можем.  Его  неуспех  целиком  объясняется  научными  причинами,  т.е.
тем,  что он  не мог  преодолеть ряда  возражений как  со стороны  так
называемого здравого смысла, так и со стороны компетентных астрономов.
Здравый смысл выдвигал  такие возражения: если Земля  вертится с такой
страшной скоростью, почему мы этого  не чувствуем? Уже Эратосфен очень
хорошо определил размеры  Земли, и мы знаем, что  на экваторе скорость
обращения  около 450  метров в  секунду, т.е.  во много  раз превышает
скорость  самого страшного  урагана. Почему  мы не  замечаем, в  каком
направлении летят птицы: ведь должно  было бы быть различие, смотря по
тому, куда они  летят --- на запад  или на восток. Мы  знаем, что сила
этого возражения была уничтожена только  Галилеем почти через 2000 лет
после Аристарха.

Если Земля вращается вокруг Солнца, а звезды неподвижны и находятся на
разных расстояниях (о  чем можно догадываться по  различию их блеска),
почему  мы  не замечаем  годичного  смещения  звезд друг  относительно
друга,  так  называемого  звездного  параллакса?  Не  параллактическое
смещение было точно обнаружено только в 1839 г.

Но  были  еще  более  серьезные  возражения,  приведенные  Веселовским
(с.  65--69). Как  уже было  указано, Аристарх  принимал два  (вернее,
три)  основных  положения:  1)  все планеты  должны  вращаться  вокруг
центрального материального  тела, 2) вращение  должно идти по  кругу и
3)  быть  равномерным.  Вавилонские  астрономы  кроме  астрологических
представлений  доставили греческим  ученым  массу наблюдений,  которые
показали,   что   эти   положения   несовместимы.   Еще   до   Евдокса
греческие астрономы  Метон и  Евктемон указали,  что продолжительность
астрономических  времен  года не  является  одинаковой,  а ко  времени
Аристарха это  было доказано.  Через сто  лет после  Аристарха Гиппарх
отбросил первое положение, заставив  все планеты обращаться равномерно
вокруг нематериальной геометрической точки --- даже не центра Земли, а
центра некоторого  эксцентрического по отношению к  Земле круга. Через
восемнадцать веков после Гиппарха Кеплер поступил как раз наоборот: он
заставил планеты вращаться вокруг  центрального материального тела ---
Солнца,  но отказался  от  принципа  равномерности круговых  движений,
который  не мог  нарушить и  Коперник. Правильнее  будет сказать,  что
Кеплер  заменил движение  по  кругам движением  по  эллипсам и  придал
понятию равномерности движения другой смысл,  как это будет показано в
свое время (второй закон Кеплера).

Всех  этих  чисто  научных  возражений  совершенно  достаточно,  чтобы
понять, что система Аристарха для своего времени была несвоевременной.
Потомки, знакомящиеся с той или иной теорией, часто удивляются, как не
могли  предки  понять  такой  простой  вещи,  склонны  обвинять  их  в
косности,  влиянии политических,  классовых  и иных  мотивов. Дело  же
объясняется  тем, что  всякая новая  крупная теория  должна преодолеть
огромное  количество  серьезных  возражений  и  на  переходных  этапах
своего развития  принуждена игнорировать многие факты,  что и вызывает
оппозицию серьезных ученых.

\subsection{Победа  приблизительно  геоцентрической  системы  Птолемея
объясняется   разработанностью   теории    эпициклов   и   эксцентров,
произведенной Гиппархом, Аполлонием и Птолемеем}

Теория Аристарха  была на  длительный период  побеждена кинематической
теорией  эксцентров  и  эпициклов,  которая давала  столь  же  хорошее
математическое  описание  явлений  и   была  свободна  от  возражений,
вытекающих из нашего повседневного опыта.

Знаменитый математик Аполлоний Пергский,  известный своим сочинением о
канонических сечениях,  показал, что при помощи  модели эпицикла можно
объяснить характерные движения планет  --- прямые и попятные движения.
Эта теория  вошла в «Альмагест»  Птолемея, и почти  буквальный перевод
этой главы Птолемея был помещен  Коперником в конце пятой книги своего
классического сочинения  (Веселовский, с.  68)... Другой  механизм ---
эксцентра --- давал для определения  условий столь же хорошее описание
движений любой планеты, и  комбинация эксцентров и эпициклов Гиппархом
Никейским (II в. до н.э.) и Птолемеем (II в. н.э., т.е. примерно через
300 лет)  и привела  к созданию того,  что обычно  называется системой
Птолемея. Наиболее  важное ее  отличие от  системы Аристотеля  --- это
то,  что круги,  по  которым вращаются  планеты,  вращаются не  вокруг
материальных точек. Поэтому теория  Птолемея может быть названа только
приблизительно геоцентрической системой, так  как в этой системе центр
движения не совпадает точно с центром Земли. Как указывает Веселовский
(с. 70), Коперник сделал все  для примирения Аристарха с Птолемеем, но
полного торжества  гелиоцентрическая теория Аристарха  добилась только
тогда,  когда  схемы  эпициклов  и эксцентров  были  заменены  теорией
эллиптического движения  Кеплера и  небесной механикой,  основанной на
законах Ньютона.

Постараемся  изложить   вкратце  сущность   теории  Гиппарха-Птолемея,
главным  образом  по  прекрасной  статье  Идельсона  (1947).  Гиппарх,
который считается  величайшим астрономом древности, наряду  с прежними
астрономами, а после него  Птолемей установили целый ряд «неравенств»,
прежде  всего  установление  неравномерности  движения  Солнца  внутри
года,  так  называемую  эвекцию  Луны  и др.  Все  эти  отклонения  от
равномерного движения требовали составления солнечных и лунных таблиц,
в связи  с чем  возникла и  новая математическая  наука тригонометрия,
автором  которой  тоже считается  Гиппарх.  Выше  уже указано,  что  и
Аполлоний  Пергский  принимал  участие  в  разработке  теории.  Следуя
поставленной задаче  свести все  неравномерные движения  к равномерным
круговым, и была построена  теория эпициклов и эксцентров. Разумеется,
этим  же законам  должны  были быть  подчинены  и планетные  движения,
неравномерность движения которых была известна давно.

Теория эпицикла  Аполлония заключается в следующем:  точка 1 вращается
вокруг центра С по кругу  против часовой стрелки, этот круг называется
деферентом. А  планета Р  вращается вокруг точки  1 по  другому кругу,
эпициклу, в обратном направлении. Тогда в точке О (при геоцентрической
системе --- с Земли) будем наблюдать  и прямые и попятные движения. Но
Аполлоний  дал теорему,  где,  используя  схему шарнирного  механизма,
показал, что движение планеты Р  можно описать, используя один круг, а
именно, если  планета будет двигаться  в прямом направлении  по кругу,
эксцентру, радиус которого равен радиусу  деферента, а центр Т отдален
от точки  О на отрезок,  равный и параллельный радиусу  эпицикла. Это,
конечно,  имеет место  лишь  при определенных  условиях, если  угловая
скорость  движения  по  эпициклу  равна, но  противоположна  по  знаку
угловой скорости по деференту. Тогда возможна замена гипотезы эпицикла
гипотезой  простого эксцентра.  Вся теория  движения Солнца  построена
Гиппархом  и Птолемеем  на гипотезе  простого эксцентра  как на  более
простой по сравнению с гипотезой эпициклов. Но кроме гипотез эксцентра
и  эпицикла  Птолемей  использует еще  гипотезу  «биссекции»,  которую
Идельсон  (1947, с.  110)  считает шедевром  древней науки,  бесспорно
принадлежащим Птолемею. Важность этой  гипотезы заключается в том, что
она  является  предвосхищением  идей Кеплера.  По  гипотезе  биссекции
эксцентриситета планета  Р движется по-прежнему по  кругу, называемому
эксцентром, но так, что равномерно будет двигаться не радиус эксцентра
PC,  а  радиус  другого  круга, который  был  потом  назван  эквантом.
Важность этой гипотезы заключается в том, что движение планеты в круге
эксцентра  не  только  «представляется»  неравномерным  наблюдателю  в
центре мира, но  оно идеально неравномерно: отказ от  того постулата в
равномерности круговых  движений, от  которого, как увидим  дальше, не
мог отказаться сам Коперник.

Как  пишет  Идельсон  (с.  115),   «отказываясь  здесь  от  той  догмы
равномерных  круговых движений,  которой была  насыщена вся  греческая
философия,  на которой  настаивали в  течение столетий  мыслители школ
Пифагора, Платона, Аристотеля,  Птолемей дал, на наш  взгляд, такой же
мощный  толчок  мыслям Кеплера,  как  Аристарх  Самосский и  некоторые
ранние пифагорейцы  своими высказываниями  о движении Земли  влияли на
зарождение коперниканской доктрины». И Кеплер начал свои исследования,
прилагая к движению Земли теорию биссекции.

\subsection{О  прогнозах затмений  в доэллинских  цивилизациях (Китай,
халдеи)}

Солнечные таблицы, или Канон  Гиппарха-Птолемея, не только учитывали с
достаточной точностью движения Солнца и  Луны, но и давали возможность
рассчитывать   солнечные   и   лунные  затмения.   Конечно,   точность
предсказаний  далеко уступала  точности  современных предсказаний,  но
совпадение было достаточно  удовлетворительным и успешное предсказание
затмения  Колумбом  (на  основе  так  называемых  Альфонсовых  таблиц,
основанных на теории Птолемея), сыграло существенную роль в успехе его
деятельности. Поэтому,  если для  оценки научной теории  ставить такой
разумный  критерий, как  возможность  прогноза  событий, то,  согласно
этому критерию, теория Птолемея есть, бесспорно, научная теория (этого
вопроса нам придется  еще касаться, когда речь зайдет  об общей оценке
гелиоцентрической теории).

Но   тогда   естественно    возникает   возражение.   Затмения   могли
предсказывать и до Птолемея, следовательно, заслуга Птолемея в этом не
так велика. Это надо разобрать.

В своей  книге «Современные представления  о Вселенной» (1949,  с. 27)
академик В. Г.  Фесенков пишет: «У китайцев уже  с незапамятных времен
появлялись официальные  извещения относительно  предстоящего затмения,
что вменялось в обязанность астрономам.  В китайской книге ``Шу кинг''
рассказывается  о  солнечном затмении  в  2137  г. до  н.э.,  которое,
вопреки  требованиям, не  было предсказано  астрономами Хи  и Хо,  что
стоило им  головы». Как  указывает Фесенков,  возможность предсказания
затмений увязывается  с суевериями  Сиампы, по Тэйлору,  полагают, что
астрономы настолько  проницательны, что знают, когда  дракон обедает и
какое затмение потребуется для его насыщения.

Но  ведь  сообщение  в  книге ``Шу  кинг''  указывает  на  неудавшееся
предсказание: более интересно было бы  знать, имеются ли данные той же
эпохи  о случаях  удавшихся  предсказаний. И  в  явном противоречии  с
тем,  что  он  пишет  на  стр.  27 своей  книги,  Фесенков  на  с.  12
сообщает,  что  от иезуитов  XVII  столетия  западный мир  узнал,  что
в  Китае  существует  полная  астрономическая  система,  которая,  как
показали современные  критические исследования, основана на  данных, в
большинстве относящихся  к эпохе ранее 400  г. до н.э., а  в отдельных
случаях ---  даже до 1500 г.  до н.э. Но Хи  и Хо жили еще  на 600 лет
раньше.

Поэтому,  например, Уэвель  (1867,  с. 196)  пишет:  «Но этого  нельзя
считать  за действительное  событие:  потому что  в течение  следующих
десяти столетий мы не находим в китайской истории ни одного наблюдения
или  факта, связанного  с  астрономией». Можно  добавить, что,  весьма
вероятно,  и  в  столь   отдаленные  времена  деспоты  отличались  тем
свойством,  что  требовали  от науки  таких  практических  применений,
которые наука по своему состоянию того времени вообще не могла дать, и
казнили ослушников, или «вредителей».  Поэтому, я думаю, нет оснований
сомневаться в казни  Хи и Хо, можно только сомневаться  в том, что они
были казнены за действительную вину.

Настоящие  прогнозы   затмений  начали  делать   халдеи,  установившие
правильное чередование  лунных и солнечных затмений,  то, что египтяне
называли  сарос.  Это  уже  давало  возможность  предсказывать  лунные
затмения, которые  были видимы  на всей Земле.  Предсказания солнечных
затмений  несравненно труднее,  так  как известно,  что  тень от  Луны
пробегает  по  Земле  узкой  полосой и  необходимо  высокое  состояние
математической теории, чтобы точно  рассчитать, как эта полоса пройдет
по  Земле.  Но,  конечно,  само  установление  периодичности  является
высоким достижением,  так как затмения перестали  быть непредвиденным,
зловещим явлением, а  подчинялись строгой закономерности. По-видимому,
и сейчас  остается неясным, каким  образом так давно  халдейские жрецы
могли открыть периодичность затмений.  То, что их тщательно записывали
в  летописях,  ничего  не  говорит. В  наших  русских  летописях  тоже
записывали затмения как чудесные явления,  но примерно более чем через
две тысячи  лет после халдеев  нашим предкам  не приходило и  в голову
искать  закономерность в  этих явлениях.  Очевидно, халдеи  и египтяне
исходили  из  того  же  космического понимания  Вселенной,  которое  в
неясной  форме  существовало  и  до  Пифагора  (см.  §5  этой  главы).
Несомненно, что халдейская астрономия была теснейшим образом связана с
астрологией (сами  слова «халдеи»  и «астрологи»  иногда употреблялись
как синонимы): это ясно показывает,  что астрология отнюдь не являлась
сплошным  набором   фантастических  представлений,  а   была  способна
порождать и  вполне научные  представления. Эта  роль не  исчезла, как
увидим, и дальше.

\subsection{Сомнительность  научного  значения  предсказания  затмения
Фалесом}

Среди   греков  первое   удачное   предсказание  солнечного   затмения
приписывается  Фалесу, основателю  милетской  школы. По  свидетельству
Геродота,\footnote{У  Геродота (гл.  XXIV) говорится,  что была  война
между Лидийцами и  Мидянами; после разных оборотов  счастья, «в шестой
год  произошло  сражение,  и   когда  битва  началась,  то  случилось,
что  день  внезапно превратился  в  ночь.  И эту  перемену  предсказал
им  Фалес   Милетский,  определительно  назвав  год,   в  который  это
событие действительно  произошло. Лидийцы  и Мидяне, увидев,  что день
превратился в ночь, перестали сражаться;  и обе стороны пожелали мира»
(Уэвель,  с. 207).}  Фалес предсказал  солнечное затмение,  положившее
конец войне мидян и лидийцев. На этом основании историки устанавливают
дату последнего сражения  585 г. до н.э. Но как  мог Фалес предвидеть,
что полоса  солнечного затмения  пройдет именно через  место сражения?
Если  даже  Фалесу  и  удалось   предсказать  затмение,  то  это  было
делом  чистой  удачи.  Поэтому  давно  уже  высказывались  сомнения  в
справедливости утверждения об удачном предсказании Фалеса. Анри Мартен
в книге о Тимее (цитирую по: Уэвель,  1867, с. 209) пишет, что ни один
писатель  не  сообщает, чтобы  Фалес  или  его преемники,  Анаксимандр
или  Анаксагор, когда-нибудь  еще пытали  счастья таким  путем. Но,  с
другой стороны, сообщают, что  Анаксимандр предсказал землетрясение, а
Анаксагор падение аэролитов ---  истории, очевидно, баснословные, хотя
о них  говорится с  такой же  уверенностью, как  и о  затмении Фалеса.
Наконец, о том же Фалеев Аристотель сообщает в своей «Политике»: Когда
Фалеса попрекали его бедностью, так как-де занятия философией никакого
барыша  не приносят,  то, рассказывают,  Фалес, предвидя  на основании
астрономических данных богатый урожай  оливок, арендовал маслобойни на
о.  Хиосе и  в Милете  и потом  сдавал их  по высокой  цене (Б.Рассел,
1959, с.  44). Поэтому  Рассел считает,  что предсказание  Фалеса было
чистой удачей.  Таких удачных предсказаний на  мнимо научном основании
можно  привести  достаточно и  в  новейшее  время.  Но может  быть,  у
Фалеса были  такие знания,  которые отсутствуют у  современных ученых?
Сейчас  есть  тенденция думать,  что  в  прошлом были  культуры  более
высокие, чем  современная... И  в «Истории философии»  (1941, с.  28 и
29)  без  всякой  иронии  сообщается,  что  Фалес  «использовал,  свои
метеорологические знания  для того, чтобы предсказать  урожай оливок».
Но на тех же страницах  мы получаем такие сведения о метеорологических
и геологических  познаниях Фалеса:  «Фалес установил, что  разлив Нила
происходит  потому,  что течение  его  задерживают  пассатные ветры  и
вода  в устье  не имеет  выхода» (Геродот,  II, 20).  «Астрономические
предсказания  тесно  увязывались  у   Фалеса  с  геологией.  Например,
землетрясения Фалес  объяснял тем,  что землю  качает, как  корабль во
время бури,  так как, по  его представлению, Земля ``лежит  (как нечто
плоское) на  воде''. Земля имеет  множество пор, пещер, каналов  и рек
внутри себя, она является как бы полой, и вода, на поверхности которой
плавает Земля, проникает в Землю, создает извержения и столкновения, а
вода моря  сбивает землю то  в одну, то  в другую сторону».  При таком
уровне  астрономических и  геологических представлений  трудно думать,
чтобы  Фалес мог  иметь  точную  математическую теорию  предвычисления
затмений.

Есть указания,  что в  древности удачные предсказания  затмений делали
Геликон Кизикский и  Залем (статья «Астрономия» в  словаре Брокгауза и
Ефрона). Мне  неизвестно, когда  было впервые точно  предсказано (т.е.
указано  место полного  затмения)  и  хотя бы  день,  когда оно  будет
(солнечное  затмение), во  всяком  случае это  было значительно  позже
Фалеса.

\subsection{Крупный   успех   теории   Гиппарха-Птолемея   объясняется
приблизительной  справедливостью  пифагорейской догадки  о  господстве
круговых  движений и  тем,  что  не было  серьезных  оснований для  ее
пересмотра}

Теория Гиппарха-Птолемея,  отправляясь от эстетики  круговых движений,
достигла  значительного  успеха.  Причина этого  успеха  прежде  всего
в   том,  что   орбиты   планет  действительно   близки  к   круговым.
Эксцентриситеты (отношение  расстояния между фокусами к  длине большой
оси эллипса)  у всех  планет, кроме  Меркурия, очень  малы: наименьший
у  Венеры  ---  0,00681,  наибольший  у  Марса  ---  0,0933,  у  Марса
отношение  малой  оси к  большой  равно  0,994  и  даже у  Меркурия  с
его  эксцентриситетом в  0,4  это отношение  0,915,  т.е. этот  эллипс
очень  похож  на  окружность.  «Если бы  эксцентриситеты  планет  были
существенно больше, чем это  имеет место в действительности (например,
имели  порядок эксцентриситетов  периодических  комет),  то простые  и
изящные  модели древних  неминуемо разбились  бы о  неприступные скалы
природы»  (Идельсон,  с.  129).  Пифагорейская  догадка  о  господстве
круговых  движений  была  приблизительно  справедлива,  и  для  людей,
довольствующихся  приблизительным  соответствием  теории и  опыта,  не
было  никаких  оснований  для  дальнейших  исканий.  Так  и  поступали
перипатетики.  Но  для   людей,  не  довольствующихся  приблизительным
соответствием, система Птолемея  заключала основания для недовольства,
что хорошо изложено у того  же Идельсона. Птолемей устанавливает такой
порядок планет: Луна, Меркурий,  Венера, Солнце, Марс, Юпитер, Сатурн.
Птолемей опровергает  некоторых астрономов, считавших, что  Меркурия и
Венеру надо тоже полагать за Солнцем.

Идельсон указывает,  что третья глава девятой  книги Птолемея содержит
много  загадочного   материала.  Птолемей  приводит   ряд  постулатов,
относящихся к движению планет, и некоторые численные соотношения между
их движениями.  Все эти соотношения  верны, но  откуда он их  взял, на
основании какой  доктрины, остается  совершенно неясным и,  как думает
Идельсон, останется навсегда необъясненным в истории науки.

Особенно интересно  указание, что сумма зодиакальных  и сонолитических
скоростей  Марса, Юпитера  и  Сатурна  равна одной  и  той же  угловой
скорости,  и эта  скорость  есть  не что  иное,  как среднее  суточное
движение Солнца  по долготе.  Ее значение, совершенно  совпадающее для
всех  трех  планет,  дано  по  вавилонской  шестидесятеричной  системе
(сохранившейся  до сих  пор в  делении градусов  на минуты,  секунды и
терции)  с точностью  до одной  сексты  градуса. Но  одна секста  дуги
равна  0,0000000772  секунды или  на  поверхности  Земли на  меридиане
равна примерно  2,4 микрона. Совершенно  ясно, что такой  точностью не
обладают и современные наблюдения, не  говоря уже о времени Гиппарха и
Птолемея. Очевидно,  все это  является результатом  какой-то доктрины,
которую  Птолемей  не счел  нужным  сообщить  читателям. Найденные  им
условия движения  планет вызывают вопрос: как  мог такой замечательный
астроном-теоретик не учесть, что эти найденные им условия обнаруживают
такие соотношения  и гармонии,  которые были бы  решительно немыслимы,
если  бы  движения всех  планет  не  были  сопряжены и  связаны  между
собой  единым движением  Солнца.  Как  мог он  не  прийти к  элементам
гелиоцентрической системы? Но даже не придя к полной гелиоцентрической
системе, он  мог сделать первый  шаг, который  уже был сделан  до него
Гераклитом Понтийским,  а впоследствии уже после  Коперника был сделан
Тихо  Браге; эта  система  --- вращение  планет  (кроме Земли)  вокруг
Солнца --- давала значительное  кинематическое упрощение, сохраняя тот
принцип, что всякое абсолютное движение совершается вокруг Земли.

Эти   неразгаданные   тайны    истории   науки   заставили   некоторых
исследователей  полагать,  что,  может быть,  геоцентрическая  система
Птолемея есть только переделка  и отзвук кем-то детально разработанной
гелиоцентрической  системы,   быть  может,  заброшенной   потом  из-за
разнообразных  опасений и  предрассудков.  Этого мнения  придерживался
такой  выдающийся  историк  физики,  как  Дюгем  (согласно  Идельсону,
с.  145). Разумеется,  в  истории наук  есть  еще много  неразгаданных
тайн, но  история других  наук полна случаев,  где мимо,  казалось бы,
кричащих фактов, требующих  пересмотра традиционных мнений, равнодушно
проходили  выдающиеся   умы,  требовавшие  не  косвенных   доводов,  а
прямых, экспериментальных  подтверждений. Приведу только  два примера.
Как  много  косвенных  данных  было в  пользу  превращения  химических
элементов!  Однако так  как  не  было экспериментальных  доказательств
этого,  то  практически  все  выдающиеся   химики  XIX  в.  вплоть  до
Менделеева  решительно отказывались  считаться с  косвенными доводами,
хотя  никаких  вненаучных  объяснений такого  консерватизма  не  было.
Только  с  открытием  радиоактивности и  получением  экспериментальных
доказательств превращения  элементов консерватизм был  сломлен. Сейчас
сам  факт  закономерной  связи   периодической  системы  трактуют  как
выражение родства структуры элементов, чего творец системы, Менделеев,
вовсе  не признавал.  Примерно  то же  можно  сказать об  эволюционной
теории. Сейчас принято  говорить на лекциях, что  вся систематика, вся
сравнительная  анатомия и  эмбриология  и  биогеография есть  сплошное
доказательство трансформизма, но ведь факты, которые сейчас приводят в
пользу  трансформизма, были  известны и  до Дарвина.  А мы  знаем, что
не  только  прежние биологи,  но  даже  Т. Гексли,  впоследствии  один
из  пламеннейших  апостолов  дарвинизма, был  знаком  с  эволюционными
гипотезами и не придавал им значения, и опубликование первых сообщений
Дарвина  и  Уоллеса не  произвело  ни  малейшего впечатления.  Поэтому
неудивительно,  что  и  гелиоцентрическая  система  ждала  века,  пока
дождалась  признания,  и  гипотеза  о  какой-то  ранее  существовавшей
разработанной системе, пожалуй, является излишней. Но почему астрономы
не перешли к более простой  гелиоцентрической теории, хотя бы из чисто
практических соображений? Для многих и прежде всего практических целей
в этом не было надобности. Ведь мы-то сами живем на Земле и неизбежно,
независимо  от теории,  ведем  наблюдения с  геоцентрической,  а не  с
гелиоцентрической точки  зрения. Поэтому до сих  пор в астрономическом
языке удержалась терминология Гиппарха; мы говорим о таблицах движения
Солнца,  о моментах  вступления Солнца  в знаки  зодиака, о  перигее и
апогее  солнечной орбиты,  вместо того  чтобы говорить  о перигелии  и
афелии орбиты Земли (Идельсон, с. 107).

\subsection{Огромная   прогрессивная  роль   платонических  постулатов
равномерного и кругового движения}

Астрономия Птолемея  завершает длинный путь развития  греческой науки,
исходя  из  тех принципов,  которые  проводили  оба гиганта  греческой
философии,  Платон и  Аристотель:  небесные тела  совершенны по  своей
природе,  и  потому  им  приличествует  только  совершенное  движение,
равномерное  и   круговое  (Идельсон,  1947,   2,  с.  18).   Уже  эти
постулаты  равномерного  и  кругового  движения не  были  жесткими  и,
как  было указано,  давали  три  конкретных решения:  аристотелевское,
гомоцентрических  сфер  (о  взглядах  Аристотеля  поговорим  несколько
дальше),  гелиоцентрическое   Аристарха  и   геоцентрическое  Птолемея
---   Гиппарха.   Постулаты   равномерного   и   кругового   движения,
несомненно, обладают значительной долей произвольности, поэтому многие
представители  так   называемого  индуктивного  направления   в  науке
протестуют  против таких  произвольных,  предвзятых предположений.  Но
основоположники индуктивного метода думали иначе. Вот что пишет Уэвель
(1867, с.  234): «Предположение, что введенные  таким образом круговые
движения  все  совершенно  равномерны,  есть  основной  принцип  всего
процесса. Это предположение можно назвать  ошибочным, и мы видели, как
фантастичны были  некоторые из аргументов, которые  первоначально были
приводимы в  его пользу. Но какое-нибудь  предположение необходимо для
того,  чтобы  можно  было  связать  как-нибудь  движения  в  различных
пунктах обращения  известного светила, т.е.  для того, чтобы  мы могли
иметь  какую-нибудь теорию  этих движений,  и невозможно  было выбрать
предположение проще того, какое мы  упоминали. Заслуга этой теории та,
что, получив  количество эксцентриситета, место апогея  и, быть может,
другие элементы из немногих наблюдений, она выводит из них результаты,
согласные со  всеми наблюдениями, как  бы ни были они  многочисленны и
разновременны».  И  дальше,  с.  235:  «Мы  можем  объяснить  это  еще
больше,  заметив,  что  такое разрешение  неравных  движений  небесных
тел  на равномерные  круговые движения,  в сущности,  равнозначительно
тем  новейшим и  усовершенствованным  процессам,  какие применяются  к
подобным  движениям у  новейших  астрономов. Их  общая метода  состоит
в  том,  что   они  представляют  все  неравенства   движений  в  виде
рядов, которых  отдельные члены  изображают отдельные части,  из каких
составляется каждое неравенство.  Эти члены заключают в  себе синусы и
косинусы  известных углов,  т.е. они  заключают известные  технические
средства,  с   помощью  которых  измеряются  круг   и  также  круговые
движения,  предполагая,  что все  круговые  движения  бывают вместе  и
равномерные и потому находятся в постоянном отношении со временем, ---
предположение,  которое  древние  также поставили  в  основание  своей
теории  эпициклов... И,  таким образом,  проблема разрешения  небесных
движений на равномерные круговые движения, поставленная две тысячи лет
тому  назад  в школе  Платона,  все  еще остается  предметом  изучения
новейших астрономов,  наблюдателей и математиков». Как  увидим дальше,
Коперник  принимал  теорию  эпициклов   и  необходимость  кругового  и
равномерного  движения по  орбите.  Правильно пишет  Уэвель (с.  239):
«Итак, в  этом смысле Гиппархова  теория была реальной  и неразрушимой
истиной, которая не была брошена  и заменена другого рода истинами, но
была  принята  и вошла  в  состав  всякой последующей  астрономической
теории и которая никогда не может  перестать быть одной из важнейших и
основных частей нашего астрономического знания».

\subsection{Сущность  истинного пифагоро-платоновского  мировоззрения:
сочетание  признания  гармоничности,  а не  хаотичности  Вселенной,  и
наличия математических законов, доступных человеку}

Но   постулаты  равномерного   и   кругового   движения  не   являются
философскими постулатами в точном  смысле этого слова. Здесь философия
превращается в рабочие гипотезы,  достаточно разнородные, чтобы из них
можно  было сделать  выбор на  основе анализа  наблюдений. Философские
постулаты пифагореизма  и шире и вместе  с тем уже. Они  заключаются в
признании  гармоничности,  космичности,  а не  хаотичности  Вселенной,
примата  холистического подхода  перед  меристическим и  существования
сравнительно простых,  доступных математической  формулировке законов.
Только  такое  сочетание  гармонического  понимания  и  математической
трактовки    может   назваться    подлинно   пифагорейско-платоновским
направлением.  Но мы  увидим  дальше, что  кроме истинно  платоновской
линии  имеются две  другие:  одна,  сохраняя холистичность  понимания,
отказывается  или  пренебрегает точной  математической  формулировкой.
Это,  в  первую  очередь,  Аристотель  и  перипатетики,  в  дальнейшем
ряд  мыслителей,  близких  к  платонизму,  не  только  не  применяющих
математику,  но  даже  враждебно относящихся  к  математизации;  сюда,
например,  относится  Гете.  С  другой  стороны,  когда  математизация
науки сделала  уже большие  успехи, стали считать  первую, философскую
сторону  излишним,  а  может   быть,  даже  ненужным  привеском.  Сюда
относятся развившиеся гораздо позже  разные течения позитивизма. Конт,
как  известно, признавал  три  стадии  развития науки:  теологическую,
метафизическую и  научную. Наука  ни в  какой философии  не нуждается,
она  сама себе  философия. Но  позитивисты не  отрицали огромной  роли
философии в прошлом и сейчас  полезно выяснить истинную роль Платона в
развитии космологических представлений.

Жизнь Платона протекала в период становления гелиоцентрической теории.
Он, видимо,  был знаком с  Филолаем, первым критиком  геоцентризма, но
не  пришедшим  еще к  гелиоцентризму,  и  имел в  качестве  ближайшего
сотрудника  Гераклита Понтийского,  сделавшего  первый  крупный шаг  в
сторону  подлинного гелиоцентризма.  Коперник  древнего мира  Аристарх
Самосский жил уже значительно позже.  Все три имени, Филолай, Гераклит
и  Аристарх,  связаны с  пифагорейским  направлением.  Что вложил  сам
Платон  в  дело  разработки   гелиоцентризма?  Как  всегда,  вопрос  о
приоритете особенно труден в отношении Платона, который, как известно,
не  высказывал ни  одного  из своих  учений от  своего  имени. В  этом
сказывалось  то,  что   можно  назвать  интеллектуальным  коммунизмом,
свойственным, как и социальный  коммунизм, пифагорейскому движению. «В
организованное им общество на равных  условиях принимались и мужчины и
женщины;  все  члены общества  владели  собственностью  сообща и  вели
одинаковый образ жизни, точно так же научные и математические открытия
считались коллективными  и мистическим образом  приписывались Пифагору
даже после его смерти» (Рассел, 1959, с. 51).

Чтобы понять Платона и его  высказывания, надо понимать, кто такой был
Пифагор, преемником которого был Платон, и это можно прекрасно сделать
по изложению  того же Б.Рассела.  «Пифагор является одной  из наиболее
интересных  и противоречивых  личностей  в  истории... Пифагора  можно
коротко  охарактеризовать,  сказав,  что  он соединяет  в  себе  черты
Эйнштейна и  миссис Элли  (Элли --- основательница  американской секты
``Христианская наука''.  --- Прим.  перев.). Пифагор  основал религию,
главные  положения  которой состояли  в  учении  о переселении  душ  и
греховности употребления в пищу бобов» (с. 50).

В  интеллектуальном  мистицизме  Пифагора   большую  роль  играли  два
понятия, которые теперь (как  и понятие «космос») приобрели совершенно
другое значение. Одно  из понятий --- оргия.  «Это слово употреблялось
орфиками в смысле ``причастия''. Цель  причастия состояла в том, чтобы
очистить  душу верующих  и помочь  им избежать  кругового рождения.  В
отличие от жрецов  олимпийских культов, орфики основали  то, что может
быть названо  ``церквами'', т.е.  религиозные общества, в  которые мог
быть принят всякий без различия  расы или пола. Благодаря влиянию этих
сообществ, возникла концепция философского способа жизни» (с. 41).

\subsection{Если  даже пифагорейская  философия  ложна, она  оказалась
чрезвычайно полезной и практичной}

Любопытно также первоначальное значение слова «теория». (Рассел, 1959,
с.  52).  «Это слово  первоначально  было  орфическим словом,  которое
Корнфорд истолковывает  как ``страстное и  сочувственное созерцание''.
В  этом  состоянии,  говорит Корнфорд,  ``зритель  отождествляет  себя
со  страдающим  богом,  умирает  с  его  смертью  и  рождается  вместе
с  его возрождением''.  Пифагор  понимал  ``страстное и  сочувственное
созерцание'' как интеллектуальное созерцание,  к которому мы прибегаем
также в математическом познании. Таким образом, благодаря пифагореизму
слово  ``теория'' постепенно  приобрело свое  теперешнее значение,  но
для  всех тех,  кто был  вдохновлен  Пифагором, оно  сохранило в  себе
элемент экстатического  откровения. Это может показаться  странным для
тех,  кто  немного  и  весьма  неохотно  изучал  математику  в  школе,
но  тем,  кто  испытал   опьяняющую  радость  неожиданного  понимания,
которую  время  от времени  приносит  математика  тем, кто  любит  ее,
пифагорейский взгляд  покажется совершенно естественным, даже  если он
не  соответствует истине.  Легко  может  показаться, что  эмпирический
философ  ---  раб исследуемого  материала,  но  чистый математик,  как
и  музыкант,  ---  свободный творец  собственного  мира  упорядоченной
красоты».   Бертран  Рассел   ---  один   из  величайших   современных
мыслителей,  страстный противник  платонизма,  как  и всякой  религии,
однако  и у  него  вырывается, вопреки  его собственным  «установкам»,
восторженное преклонение  перед пифагорейским духом, так  как, вопреки
его исповедованию,  сам Б.Рассел  --- мыслитель пифагорейского  духа и
пишет в том же блестящем  стиле, в каком писал блестящий последователь
Пифагора --- Платон. Но, может быть, «теория» в смысле Пифагора вредна
для  практической  деятельности,  она  тормозит  развитие  прикладного
знания, необходимого для человечества? Ответ на это дает тот же Рассел
(с. 53):  «Современные определения  истины, которые  даются, например,
прагматизмом  или  инструментализмом,  --- скорее  практическими,  чем
созерцательными  учениями, ---  являются  продуктом индустриализма,  в
его  противоположности аристократизму...  Идеал созерцательной  жизни,
поскольку  он вел  к созданию  чистой математики,  оказался источником
полезной  деятельности. Это  обстоятельство  увеличило престиж  самого
этого  идеала, оно  принесло ему  успех  в области  теологии, этики  и
философии, успех,  которого в противном  случае могло бы и  не быть...
Так  обстоит дело  с  объяснением двух  сторон деятельности  Пифагора:
Пифагора как религиозного пророка и Пифагора как чистого математика. В
обоих отношениях  его влияние  неизмеримо, и эти  две стороны  не были
столь самостоятельны, как это может показаться современному сознанию».

«При своем  возникновении большинство  наук были связаны  с некоторыми
формами ложных верований, которые придавали наукам фиктивную ценность.
Астрономия  была   связана  с  астрологией,  химия   ---  с  алхимией.
Математика  же  была связана  с  более  утонченным типом  заблуждений.
Математическое  знание  казалось  определенным   и  точным  ---  таким
знанием,  которое  можно  применять  к  реальному  миру;  более  того,
казалось,  что это  знание получали,  исходя из  чистого мышления,  не
прибегая  к  наблюдениям.  Поэтому  стали думать,  что  оно  дает  нам
идеал знания,  по сравнению  с которым будничное,  эмпирическое знание
несостоятельно. На  основе математики было сделано  предположение, что
мысль выше чувства, интуиция выше  наблюдения. Если же чувственный мир
не  укладывается  в  математические  рамки,  то  тем  хуже  для  этого
чувственного  мира. И  вот всевозможными  способами начали  отыскивать
методы исследования, наиболее близкие к математическому идеалу».

Истинный пифагореизм остался верен  математической трактовке мира. Его
основное  положение, что  «все вещи  суть числа»,  привело к  учению о
гармонии  в  музыке,  привело  к  таким  понятиям  в  математике,  как
«гармоническая средняя»  и «гармоническая  пропорция», легло  в основу
атомной теории.  И в  полной дисгармонии  с тем  прекрасным пониманием
пифагорейского духа, которое  мы видели в приведенных  цитатах, тот же
Б.Рассел на  с. 54 пишет: «К  несчастью для Пифагора, эта  его теорема
сразу же привела к открытию несоизмеримости, а это явление опровергало
всю его философию». Как было  показано в третьей главе, иррациональные
числа были  несчастьем для демокритовской философии,  для пифагорейцев
же,  чуждых  узкого догматизма  в  науке,  это открытие  было  началом
блестящего пути  развития, получившего  завершение у Евклида.  Если бы
открытие иррациональных  чисел опровергло  философию Пифагора,  то как
могло бы  случиться, что  «опровергнутая» философия в  виде блестящего
продолжателя  Пифагора, Платона,  просуществовала даже  организационно
в  течение  многих  столетий,  а  идейно  продолжает  существовать  до
настоящего  времени и  показывает  сейчас признаки  возрождения, а  не
дегенерации.  Пифагорейцы не  отказались  от математической  трактовки
мира, но  они поняли, что есть  две разные области ---  рациональных и
иррациональных  чисел,  и этим  областям,  как  и двум  математическим
наукам, арифметике и геометрии,  они придали независимую, даже слишком
независимую трактовку.

\subsection{Платон  стремился  изучением   видимых  предметов  постичь
особенности  их   прообразов,  и  его  понимание   мифа  соответствует
современному понятию гипотезы}

Поняв сущность пифагореизма, легче понять и Платона. Платон сохранил и
интеллектуальный мистицизм Пифагора, как и уверенность в необходимости
математической  трактовки бытия.  Он сохранил  и учение  о переселении
душ, но Платон понял, что кроме гармонии и совершенства в природе есть
много и несовершенного. Он как бы предвидел тот спор между архангелами
и  Мефистофелем,  который изобразил  Гете  в  прологе на  небе  своего
«Фауста». Возникло  новое, специфически  платоновское учение  об идеях
---  совершенном  идеальном  мире, слабым  и  искаженным  изображением
которого является  наш реальный мир.  Но так как он  все-таки является
отражением, вернее,  тенью идеального  мира, то  тщательное наблюдение
реального мира  может нам открыть  и законы мира  идеального. Небесный
мир  ближе к  идеалу, чем  земной, поэтому  наблюдение небесного  мира
является  лучшим   путем  для  познания   совершенного,  математически
оправданного мира идей. Поэтому  и обучение (теоретическое) у Платона,
согласно  его проекту  в  «Государстве», сводилось  к четырем  наукам,
математическим  или  доступным математизации:  арифметике,  геометрии,
астрономии и музыке.  Под именем квадривиума эта  программа прошла все
Средневековье и  подготовила Возрождение.  Так что, когда  цитируют то
изречение, что  философы только объясняли мир,  то совершенно очевидно
игнорируют Платона. Эта мысль о том, что наблюдаемый нами мир является
отражением  идеального и  может  быть  источником познания  идеального
мира, проникла во многие религиозные учения, родственные платонизму. В
Талмуде,  мне  говорили,  есть изречение:  «Чтобы  познать  невидимое,
смотри внимательнее на видимое». В Новом Завете: «Видим убо ныне якоже
зеркалом в гадании, тогда же лицом к лицу».

В наиболее  ясной форме  мысли Платона  о важности  математизации всех
наук,  в  первую  очередь  астрономии, выражены  в  его  замечательном
«Эпиномисе»  (Послесловие  к  «Законам»),  которое  филологи  неохотно
цитируют, так как принадлежность этого сочинения Платону оспаривается.
Допустимо, что это произведение написал не сам Платон, но, несомненно,
один  из  его  верных  учеников, записавших,  может  быть,  по  памяти
содержание бесед Платона. Там развивается мысль, что истинная мудрость
заключается в  познании чисел и  что «признавая за  всеми дисциплинами
право на существование,  можно утверждать, что ни  одна не сохранится,
но, наоборот, все исчезнут, если  из этих дисциплин исключить познание
числа»  (Эпиномис,  977,  е).  Указывая  на  важность  математики  для
астрономии, он  считает нелепым ее название  геометрии, т.е. измерение
земли (Эпиномис, 990, Уэвель,  1867, с. 214). Математической трактовке
астрономии Платон придает религиозную  основу (Эпиномис, 988). Историк
индуктивных наук, Уэвель даже взял  это место эпиграфом для книги III:
История греческой астрономии (Уэвель, с.  146): «И никому из греков не
приходило в  голову опасение,  чтобы смертным  не следовало  вникать в
действия Высших  Сил, каковы те, какими  совершаются движения небесных
тел; но люди, напротив, должны подумать, что Божественные Силы никогда
не действуют без цели и что они знают природу человека: они знают, что
с их  руководством и помощью  человек может понять те  учения, которые
сообщаются ему об этих предметах».

Эти высказывания в «Эпиномисе» вполне гармонируют с тем, что говорится
в  Платоновом   «Государстве»,  принадлежность  которого   Платону  не
оспаривается,  кажется,  решительно  никем  (если  не  считать  Н.  А.
Морозова  в  его «Христе»,  который  всю  античную литературу  относит
примерно  к  раннему Возрождению).  В  7-й  книге «Государства»  (529)
Платон говорит, что астрономия заставляет нашу душу смотреть наверх, и
там же (530):  «...как в астрономии приковывается наш  взгляд, так уши
наши приковываются  гармоническим движением: и эти  две науки являются
сестрами,  как  утверждают  пифагорейцы,  и мы  с  ними  вместе».  Как
указывает  Робен  в  комментариях  к  этому месту  (т.  1,  с.  1362),
это  является вероятным  намеком на  знаменитую «гармонию  сфер», где,
согласно  пифагорейскому  учению,  каждая сфера  в  небесном  концерте
издает ноту,  зависящую от струны, соответствующей  радиусу ее орбиты.
Как  увидим  дальше,  эта  идея   в  измененном  виде  была  одной  из
руководящих идей Кеплера. Робен указывает, что упоминание пифагорейцев
и  Пифагора приводится  Платоном только  еще в  одном месте  (X глава,
600), где речь идет о пифагорейском  образе жизни. Как видим, даже тех
философов, которые были ему особенно близки и близость к которым он не
скрывал, Платон цитировал не часто.

Совершенно правильно  пишет Уэвель (с. 140):  «Платон считает явления,
представляемые  природой  нашим  чувствам,  за  простые  намеки  и  за
грубые очерки тех предметов,  которые должен созерцать философский ум.
Небесные  тела и  весь  блеск неба,  хотя они  и  прекраснее всего  из
видимых  предметов,  но,  будучи только  видимыми  предметами,  далеко
ниже  тех истинных  предметов,  которым  они служат  представителями».
За  несовершенными  образами  наших  чувств  Платон  всегда  видит  их
совершенный  прообраз, и  это  составляет  ту двойственность,  которая
сквозит во  многих выражениях  Платона. А так  как он,  будучи убежден
в  своей  основной  философской  установке,  отнюдь  не  догматизирует
конкретные гипотезы,  то свои  космологические представления  он часто
формулирует  в виде  мифов или  легенд. Поэтому  очень часто  там, где
говорит Платон о мифе, мы можем подставить более понятное современнику
слово «гипотеза».

\subsection{В соответствии со своими  взглядами о реальных и идеальных
вещах Платон различал землю нашу и  поднебесную, обе близки по форме к
шару}

Теперь мы  можем обратиться к рассмотрению  конкретных космологических
представлений Платона, прежде всего, о форме Земли. Как уже говорилось
раньше, представление о шарообразности  Земли родилось раньше Платона,
и он его полностью поддерживает,  включая представление об антиподах и
об  условности понятий  «верх» и  «низ». Это  ясно выражено  в «Тимее»
(62--63) и хорошо изложено у  Уэвеля (с. 201--202): «В Природе имеются
две  области, прямо  противоположные и  которые разделяют  между собой
Вселенную: низ,  куда стремится  все то, что  имеет телесную  массу, и
верх ---  то, куда ничто  не идет  по собственному побуждению;  но это
мнение совершенно ошибочно. Все  Небо имеет сферическую форму; поэтому
все крайние точки, находящиеся на равном расстоянии от центра, могут с
одинаковым правом  называться крайними.... Такова природа,  и какое же
из этих  мест мы  можем считать верхом  или низом???...  поэтому, если
принять  твердое тело,  находящееся в  равновесии в  центре Вселенной,
оно  не  будет  стремиться  ни  к одной  из  крайних  точек,  так  как
все  направления  равноценны, и  если  по  этому твердому  телу  будет
перемещаться человек,  делая оборот вокруг  него, то он  несколько раз
пройдет мимо антиподов и той же точке твердого тела он последовательно
будет давать название верха и низа».

Но в §3 я цитировал мнение Веселовского, что представление о Земле как
о  впадине  заметно вплоть  до  времени  Платона. Как  совместить  это
утверждение с ясными словами «Тимея»? Веселовский цитирует «Федона» (в
примечании неправильно указано 190 А, надо 109). Ввиду исключительного
интереса  этого  места  я  позволю себе  привести  достаточно  длинные
выдержки (Федон, 109--111). В  предсмертной беседе Сократ говорит, что
он слышал от одного человека, что Земля не такова и не таких размеров,
как ее  считают те, кто привык  о ней говорить. «Если  земля находится
посредине  неба,  будучи круглою,  для  нее  ничего  не нужно  ---  ни
воздуха,  чтобы  ей не  упасть  (здесь  он полемизирует  с  некоторыми
философами, о  чем будет речь  дальше, когда дойдем до  Демокрита. ---
\emph{А.Л.}), ни какой-либо иной необходимой  (точки опоры) в таком же
роде; для того, чтобы земля  держалась, достаточно того, что само небо
равномерно окружает  землю и  имеет, как  и сама  земля, равновесие...
Затем, продолжал Сократ: я думаю,  что земля есть нечто очень великое,
и что мы, обитающие от Фасиса  (Рион на Кавказе) до Геракловых столпов
(Гибралтарский  пролив),  занимаем  только  незначительную  часть  ее,
около моря,  все равно  что муравьи или  лягушки, которые  живут около
какого-нибудь болота. Много  и других людей живет там и  сям во многих
подобного  рода местах.  Дело  в  том, что  повсюду  на земле  имеется
множество  разнообразных, по  форме  и по  величине, углублений,  куда
собирается вода,  туман и воздух.  Сама же земля, чистая,  покоится на
чистом небе,  там же,  где и  звезды. Это  небо большинством  тех, кто
обыкновенно говорит  о такого  рода предметах, называется  эфиром. Его
осадками служит все то, что непрерывно стекает в углубления земли. Мы,
живущие в  углублениях земли,  забыли и об  этом и  представляем себе,
будто живем наверху, на земле.

Это все равно, как если бы кто жил в середине морского дна и думал при
этом,  будто он  живет на  море,  и, взирая  сквозь воду  на Солнце  и
на  остальные  созвездия,  считал  бы  море  небом.  Вследствие  своей
неповоротливости  и  слабости такой  человек  никогда  не достигал  бы
поверхности моря и ни сам не  увидел бы, выныряя и высовывая голову из
моря в  здешние места, насколько  они чище и прекраснее  обитаемого им
места, ни  от другого,  видевшего их,  не слышал бы  об этом.  В таком
же  положении находимся  и  мы.  В самом  деле:  мы  живем в  каком-то
углублении земли, а думаем, будто живем на ее поверхности; мы называем
воздух небом  и думаем,  что по  воздуху, так как  он ---  небо, ходят
звезды.  И то  и  другое объясняется  тем, что  мы  по нашей  слабости
и  неповоротливости  не в  состоянии  проникнуть  до крайних  пределов
воздуха.  Ведь если  бы  кто  достиг вершин  его  или, ставши  птицей,
взлетел  бы к  ним, тот  уподобился бы  рыбам, вынырнувшим  из моря  и
видящим то, что находится на земле,  --- и он, вынырнув, увидел бы то,
что  находится там.  И если  бы,  по природе,  такой человек  оказался
достаточно силен  выдержать то, что  ему предстояло увидеть,  он узнал
бы, что  это и  есть истинное небо,  истинный свет,  истинная земля...
Я  могу  рассказать прекрасный  миф  о  том,  что находится  на  земле
поднебесной... прежде всего  говорят, что земля сама  имеет такой вид,
что, если смотреть на нее  сверху, она кажется сделанным из двенадцати
кусков кожи  мячом, пестрым,  расписанным красками,  подобными здешним
краскам, употребляемым  живописцами... Много  животных обитает  на той
земле,  много  людей;  одни  живут внутри  земли,  другие  ---  вокруг
воздуха, подобно  нам, живущим  вокруг моря,  третьи ---  на островах,
расположенных у материка и окруженных  воздухом. Одним словом, чем для
наших потребностей  служит вода и  море, тем на тамошней  земле служит
воздух, а  чем для нас является  воздух, тем для тамошних  жителей ---
эфир. Времена года у них так  уравновешены, что те люди не страдают от
болезней и  живут гораздо дольше, чем  мы, а остротою зрения  и слуха;
рассудительностью и всем т.п.  превосходят нас настолько же, насколько
воздух  своею чистотою  превосходит воду,  а эфир  --- воздух.  Есть у
тамошних людей рощи богов и святыни, где боги действительно и обитают.
Есть у  них и пророчества,  и прорицания,  и явления божеств,  и иного
рода общение с ними. Солнце, луну и звезды тамошние люди видят такими,
какими они  суть в действительности,  и этому соответствует и  во всем
прочем их блаженное состояние». Дальше идет уже о подземном царстве.

\subsection{Принимавшийся   Платоном    порядок   планет    связан   с
несовершенством астрономических наблюдений (фазы Венеры)}

Полезно   прокомментировать  это   описание   нашей   земли  и   земли
поднебесной: 1) ясное представление о  шарообразности земли как и всей
Вселенной.  2) Наша  земля очень  велика и  известная для  греков того
времени  земля ---  есть  небольшая часть  нашей  земли. 3)  Атмосфера
ограничена, и за пределами ее открывается совсем иной вид по сравнению
с тем,  что мы наблюдаем, находясь  на поверхности нашей земли.  4) Та
поднебесная земля, которая окружена эфиром, а не воздухом, несравненно
совершеннее  нашей.  5)  Для   формы  этой  поднебесной  земли  Платон
использует фигуру додекаэдра, которой и  в другом месте (Тимей, 55) он
придает важное значение как образцу, по которому Бог творил Вселенную.
Это,  конечно,  связано с  тем  значением,  которое Платон  приписывал
описанным им  впервые правильным многогранникам, называемым  и до сего
времени Платоновыми  телами. 6) Кроме людей  принимается существование
иных, более  совершенных существ,  обитающих в  других условиях.  Я не
знаю, были ли  подобные высказывания и до  Платона, здесь утверждение,
что кроме человека  и богов есть и иные разумные  и еще более разумные
существа, высказаны вполне определенно, хотя и в виде гипотезы (мифа).
В  дальнейшем эта  мысль высказывается  многими мыслителями  вплоть до
Канта в его естественной истории  неба. 7) Если теперь сравнить учение
Платона с  приведенным в §3 этой  главы учением египтян о  земле как о
впадине,  то углубления  Платона  имеют только  формальное сходство  с
египетским учением.  Платон знал, конечно, о  существовании впадин или
углублений, но для него это  не главная характеристика строения Земли,
а  деталь строения  шарообразной  земли (или  в форме  пентагонального
додекаэдра, близкого по форме к шару).

У Платона были  ясные представления о причинах затмений  и фаз планет,
и  эти  представления  он  применял для  решения  вопроса  о  взаимном
расположении планет.  Этот вопрос  лучше всего  изложен у  Коперника в
главе X «О порядке небесных  орбит». Коперник указывает на разногласия
в мнениях древних относительно Венеры  и Меркурия, с. 309: «некоторые,
подобно Платонову Тимею (38), полагали,  что эти две планеты находятся
выше  Солнца; другие  же, по  примеру Птолемея  и многих  иных ученых,
полагали  их  ниже Солнца;  Альпетрагий  же  помещает Венеру  выше,  а
Меркурий ниже  Солнца». Для  пояснения скажу,  что по  Платону порядок
известных  древним  семи  планет  был  такой:  Луна,  Солнце,  Венера,
Меркурий,  Марс,  Юпитер,  Сатурн,  по   Птолемею  же  (см.  §16  этой
главы):  Луна, Меркурий,  Венера,  Солнце, Марс,  Юпитер, Сатурн.  Чем
объясняется  то, что  Платон придерживался  менее правильного  учения?
Это  объясняет  Коперник на  той  же  странице: «Те,  которые  следуют
мнению Платона,  считают, что  все темные  тела, получающие  свет свой
от  Солнца, если  они  находятся  под Солнцем  и  не  слишком от  него
удалены,  должны представляться  то вполовину  освещенными, то  вообще
освещенными  лишь отчасти  (фазами);  наконец,  должны лишиться  всего
своего  света, подобно  тому  как  мы это  видим  в новолуние.  Далее,
если  бы планеты  эти находились  между  Землей и  Солнцем, то  должны
были  бы  (смотря по  величине  своей)  задерживать свет  солнечный  и
вследствие этого произвести затмения, чего, однако же, мы не замечаем;
а  из  этого  следует,  что  они  находятся  поверх  Солнца.  Напротив
те,  которые  полагают Венеру  и  Меркурий  ниже Солнца,  основываются
на  расстоянии,  существующем  между  Солнцем и  Луной...  Дабы  такое
значительное  расстояние  не  оставалось вовсе  пустым,  они  помещают
внутри  его орбиты  Венеру и  Меркурий таким  образом, что  после Луны
следует тотчас Меркурий, а далее  Венера... Кроме того, они допускают,
что  обе эти  планеты  одарены собственным  светом  или же  проникнуты
светом  солнечным  и поэтому  светят  при  всяком положении;  впрочем,
могущие  произойти  от них  солнечные  затмения  бывают весьма  редки.
Венера, а в особенности  Меркурий, имеют столь незначительный диаметр,
что никогда  не могут закрывать  собой более одной сотой  доли солнца,
как полагает Альбатегний».  Мне неизвестно то место у  Платона, где он
говорит о  фазах, но  все последователи Платона  использовали аргумент
отсутствия фаз  у Венеры  как довод  в пользу  того места,  которое ей
отводил Платон.  Этот аргумент об  отсутствии фаз  у Венеры играл  и в
дальнейшем большую роль. Он был опровергнут только Галилеем, открывшим
при  помощи  телескопа  фазы  Венеры,  а  «явление  Венеры  на  Солнце
наблюденное»  (слова  Ломоносова)  подтвердило  справедливость  мнения
Альбатегния. Мы видим, что во  всех этих спорах аргументы, почерпнутые
из наблюдения, играли большую  роль, и неправильные выводы объяснялись
часто просто  отсутствием оптических  инструментов. С  другой стороны,
правильное  мнение  о  нахождении  Венеры  и  Меркурия  между  Солнцем
и  Землей  защищалось  при  помощи  таких  аргументов  (недопустимость
большого расстояния), которые многим современным ученым кажутся чистым
предрассудком.

\subsection{Платон  дал  толчок  к созданию  системы  гомоцентрических
сфер, но вряд ли может считаться основоположником теории эпициклов}

Теперь  перейдем   к  рассмотрению  собственных  взглядов   Платона  в
отношении  места Земли  в Солнечной  системе. Учение  о шарообразности
Земли  возникло до  Платона  и  прочно вошло  в  учения всех  учеников
и  последователей  Платона  в  самом  широком  смысле  слова,  включая
перипатетиков: тут и  тени сомнений не было. Но уже  было указано, что
на  вопросе об  относительном положении  земного шара  по отношению  к
другим небесным шарам  у пифагорейцев, в школе Платона  и возникшей на
ее  базе александрийской  школе были  разные мнения  примерно в  таком
хронологическом  порядке: 1)  пироцентрическая система  Филолая (отказ
от  геоцентризма,  но  вращение  Земли  не  вокруг  Солнца,  а  вокруг
центрального огня); 2) строго геоцентрическая система гомоцентрических
сфер Евдокса-Аристотеля;  3) первый шаг к  гелиоцентризму --- Гераклит
Понтийский;   4)  гелиоцентрическая   система  Аристарха   Самосского;
5)  наиболее  математически  разработанная  система  Гиппарха-Птолемея
эксцентров и эпициклов: отказ от строгого геоцентризма. Эти взгляды не
всегда  различались  строго  и  потому  получалась  большая  путаница.
Например,  господствовавшим  долгое  время было  мнение,  что  Филолай
утверждал,  что Земля  обращается вокруг  Солнца. Это  мнение до  того
господствовало,  что  аббат  Буйльо,   или  Буллиальо,  как  его  чаще
называют, дал  своим сочинениям  в защиту системы  Коперника названия:
Филолай  (1639) и  Филолаическая астрономия  (1645), а  его противник,
последователь  аристотелевской философии  Кьярамонти свой  ответ издал
под  названием: Антифилолай  (Уэвель,  1867, с.  203--204).  Но и  сам
Уэвель  не различает  систем  Евдокса-Аристотеля, с  одной стороны,  и
Гиппарха-Птоломея, с другой. Поэтому  возникновение теории эпициклов и
эксцентрических кругов, которую он считает составивший эпоху в истории
астрономии  (с. 210),  он склонен  приписать самому  Платону (с.  149,
213,  214).  Но  на с.  219  мы  читаем  у  Уэвеля такие  слова:  «Эти
догадки  и  предположения,  естественно, вели  к  установлению  разных
частей теории  эпициклов. Относительно планет эта  теория принималась,
вероятно,  во времена  Платона или  еще раньше.  Аристотель разъясняет
ее  следующим образом.  ``Евдокс'',  говорит  он, ``приписывал  каждой
планете четыре сферы...''» Но  система Евдокса-Аристотеля есть система
гомоцентрических сфер, а не эпициклов и эксцентрических кругов.

Уэвель  обосновывает  свое  мнение  о Платоне  как  зачинателе  теории
эпициклов  на  известном месте  из  десятой  книги «Государства»  (как
известно, греческое слово «Политейа» переводят как «Политика»; термин,
употребляемый  Уэвелем, сейчас  избегают,  прежде  всего во  избежание
смешения  с  другим диалогом  Платона  «Политик»),  где излагается  то
видение, которое  видел памфилиец Эр  (не Алкин, как пишет  Уэвель) во
время  своей  мнимой  смерти  (Уэвель,  с.  213--214).  Эр  видел  тот
механизм, посредством  которого движутся все небесные  тела. Как пишет
Уэвель,  к прялке,  которую Судьба  (или Необходимость,  как у  других
переводчиков)  держит  между   своими  коленями,  прикреплены  плоские
кольца, с помощью которых движутся планеты.

Толкуемое  место в  «Государстве»  (X, 616--618)  принадлежит к  числу
трудных мест  у Платона  (у него, как  известно, немало  трудных мест,
как,  впрочем,  у  всякого  крупного философа)  и  разные  переводчики
многие  термины   переводили  по-разному.  В  моем   распоряжении  два
перевода  «Государства»: немецкий  Шлейермахера  (1901) и  французский
Робена  (1950); греческий  оригинал  мне, к  сожалению, недоступен  за
незнанием языка.  То, что Шлейермахер обозначает  как вульст (опухоль,
утолщение,  желвак),   то  Робен  обозначает,  как   пезон  (пружинный
безмен, по-итальянски сталера, т.е.  рычаг, качалка, десятичные весы).
Очевидно,  оба  автора  совсем   по-разному  поняли  Платона.  Уэвель,
очевидно, то же  обозначает как плоское кольцо. Как  тут разобраться в
таких  противоречиях?  Пожалуй,  можно  догадаться, что  имел  в  виду
Платон, если  обратиться к названию классического  труда Коперника: де
революционибус  орбиум целестиум.  Это название  обычно переводят:  об
обращениях небесных сфер, но некоторые переводили иначе: об обращениях
небесных кругов. Если мы  возьмем современный латинско-русский словарь
И.X.Дворецкого  и  Д.Н.Королькова  (Москва,   1949),  то  увидим,  что
латинское слово «орис» имеет такие значения: окружность, круг, колесо,
кольцо,  боевой  порядок, диск,  щит,  чашка  весов, зеркало,  кимвал,
кругооборот, небо, смена, человеческий  род, область, царство, система
наук. Вот и выбери подходящее слово.  Но в духе Коперника мы переведем
это слово как «сфера». Это ясно из дальнейшего описания, где эти сферы
(соответствующие орбитам планет) оказываются вложенными одна в другую,
подобно тому как  существуют разновесы, где одна гиря  в форме чашечки
вложена в другую.  Такие разновесы существовали и в России  в дни моей
молодости.  Каждая большая  чашечка  плотно  обнимает меньшую.  Каждая
такая сфера (чашечка)  имела особый цвет, всего их  было восемь: сфера
неподвижных звезд и семь  сфер, соответствующих планетам. Каждая сфера
издавала  особый звук,  все  вместе давали  созвучный  аккорд (гл.  X,
617). И  Кирхман, комментатор  перевода Шлейермахера (примеч.  277), и
Робен совершенно правильно усматривают  в этом изложение пифагорейской
гармонии сфер. Но Кирхман, как и Уэвель в примечании 276, впадает, мне
кажется, в ошибку, видя в  этом предварение теории эпициклов Птолемея.
Кирхман  правильно  отмечает в  том  же  примечании, что  его  вульсте
обозначают  небесные  сферы неподвижных  звезд  и  планет, но,  как  и
многие  другие,  не  делает разницы  между  системой  гомоцентрических
сфер,  описанной   Платоном,  и   системой  эпициклов   и  эксцентров.
Поэтому  можно  сказать, что  Платон  дал  толчок к  созданию  системы
гомоцентрических сфер,  развитой потом Евдоксом и  Аристотелем, но это
место  «Государства» никак  нельзя толковать  как хотя  бы даже  самый
примитивный набросок теории эпициклов.

\subsection{Платон вызвал  к жизни  стремление искать  закон планетных
расстояний,  он   рассматривал  планеты  как  инструменты   времени  и
настаивал  на использовании  круговых движений  для описания  движения
планет}

Но если в «Государстве» вряд  ли можно найти начатки теории эпициклов,
то  в  более поздних  сочинениях  Платона  можно найти  идеи,  которые
внимательные  читатели  Платона  могли  использовать  для  дальнейшего
развития космологических представлений. А мы знаем, что Платона вообще
и его диалог «Тимей» в частности усердно читали в оригинале величайшие
астрономы  по крайней  мере до  Кеплера включительно  и многие  неясно
выраженные идеи  они толковали по-своему.  Очень важное место  в Тимее
(36--38), где  он из общефилософских  соображений о душе  Мира выводит
математические соотношения в  движении планет, развивая диалектическое
соотношение  между  неделимой  реальностью (Тождественное)  и  делимым
телами  (Иное),  порождающими  нечто  среднее и  делимое  и  неделимое
одновременно.  Используя   математические  соотношения,   он  пытается
построить отношения расстояний между  орбитами. В этом уже заключается
мысль,  что расстояния  между  планетами не  произвольны, а  подчинены
какому-то математическому закону. Но  именно руководствуясь тем, чтобы
не  было  слишком  большого   расстояния  между  планетами,  некоторые
астрономы (вопреки  взглядам Платона,  изложенным в  «Государстве») и,
вероятно,  один из  первых Гераклит  Понтийский, и  помещали Венеру  и
Меркурий  ниже Солнца  (§22). Убеждение,  что имеется  закон планетных
расстояний, было  одним из  самых твердых  убеждений Кеплера.  В конце
концов, как мы знаем, в эмпирической формуле Тициуса-Бодэ такой закон,
хотя  и несовершенный,  был найден,  и, как  увидим в  свое время,  он
оказался плодотворным.

Понятие  «Иное» нам  кажется странным,  но, как  увидим в  свое время,
одно  из важных  и  чрезвычайно трудных  сочинений Николая  Кузанского
называется  «О неином»  и  стиль Кузанского  довольно  схож со  стилем
«Тимея». Уважая Кузанского как диалектика, мы должны вспомнить, что не
в меньшей степени диалектиком является и Платон.

Чрезвычайно  любопытны  рассуждения  Платона о  времени  (Тимей,  38):
«Время, таким образом, порождено вместе  с Небом, так что, порожденные
вместе, они должны и разложиться  вместе, если только такое разложение
когда-либо  будет иметь  место». В  этом, пожалуй,  можно видеть  одно
из  первых, если  не  первое представление  об отсутствии  абсолютного
времени.  Сходные мысли  развивал  бл. Августин,  но  вот как  ответил
Эйнштейн (конечно, в  шутливой форме, но не  противоречащей истине), в
чем заключается существо теории  относительности: «Прежде считали, что
если все материальные тела исчезнут из Вселенной, время и пространство
сохранятся. Согласно  же теории относительности, время  и пространство
исчезнут вместе с телами» (Б.Г.Кузнецов, 1962, М., с. 253).

Планеты, по  Платону, и  были инструментами  времени, и  вот, описывая
движения  планет,  он  говорит  о  двух  движениях,  что  и  в  теории
эпициклов.  Возможно,   что  это   был  первый  толчок   к  разработке
этой  теории,   но  намек   настолько  слабый,  что   считать  Платона
основателем  этой теории  невозможно. Но  достаточно помнить  Платона,
формулировавшего  или   пропагандировавшего  учение   о  необходимости
использования круговых  движений для  описания движения  планет, чтобы
считать,  что настойчивое  стремление к  выполнению этого  постулата и
привело  в конце  концов Гиппарха,  а  затем и  Птолемея к  разработке
теории эпициклов. И  Уэвель (с. 222) совершенно  справедлив в конечном
счете и считает изобретателем этой теории Гиппарха.

\subsection{Платон  стимулировал   брожение  умов  в  Академии   и  по
некоторым намекам  сам склонялся  в пользу  гелиоцентрической системы,
что  было  источником  последующего  прогресса  астрономии  вплоть  до
Коперника}

Самым  спорным является  вопрос, принимал  ли  Платон в  той или  иной
степени участие в разработке гелиоцентрической теории. В «Государстве»
он,  конечно,  стоит  на  точке зрения  геоцентризма,  но  Платон  жил
долго  и  мог к  концу  жизни  изменить  свое  мнение. Мы  знаем,  что
и  дошедшее  до  нас  сочинение  Коперника  древнего  мира,  Аристарха
Самосского, построено  на геоцентрической системе, а  позже он изменил
свое мнение.  Могло ли знакомство  с Филолаем и Гераклитом  не оказать
никакого влияния на Платона? И  такие следы можно найти. В «Эпиномисе»
(983)  мы читаем  такие  слова:  «Для доказательства  того,  что мы  с
полным  правом можем  считать  звезды  одушевленными телами,  подумаем
прежде  всего об  их  размерах.  Это вовсе  не  такие маленькие  тела,
как  нам  кажется  на  взгляд;  но,  напротив,  каждая  из  них  имеет
массу невообразимого  размера: этому  утверждению можно  поверить, так
как  его можно  убедительно доказать.  В  самом деле,  Солнце в  целом
следует  рассматривать как  значительно превышающее  Землю в  целом, и
размеры всех  звезд, совершающих  свое обращение на  небе, несомненно,
колоссальны». Из этого  Платон делает вывод о том,  что такие огромные
тела  или обязаны  своим  движением Богу,  или являются  одушевленными
телами.  Здесь  признается  еще  движение  звезд,  но  Земля  потеряла
свое  значение  самого крупного  тела  во  Вселенной, а  из  огромного
действительного  размера звезд  и их  кажущейся малости  можно сделать
вывод об огромных расстояниях между небесными телами.

У Платона  (Тимей, 40) есть  место, которое вызывало споры,  начиная с
древности,  и которое  некоторые толковали  как убеждение  в том,  что
Земля вращается  вокруг своей оси. Цицерон  излагает мнение Феофраста,
что  Платон  придерживался этого  мнения,  хотя  изложил его  в  Тимее
несколько темно (Уэвель, с. 203). Уэвель переводит это место так: «Что
касается  Земли,  которая  есть  наша кормилица  и  которая  привязана
к  оси,  тянущейся  через  Вселенную,   Бог  сделал  ее  виновницей  и
хранительницей дня и ночи».  Греческое слово, которое Уэвель переводит
как «привязана», некоторые переводят  как «обращается», и тогда Платон
делается единомышленником  Филолая, по крайней мере,  в том отношении,
что Земля считается вращающейся вокруг оси, и что вполне гармонирует с
указанным  выше  мнением  Платона  о сравнительной  малости  Земли  по
отношению к небесным телам --- Солнцу, звездам и т.д.

Этот  спор приобретает  особый  интерес  в связи  с  тем, что  учитель
Коперника  в Болонье,  Кодрус,  внушал Копернику  любовь  к Платону  и
толковал  это  место именно  в  смысле  обращения Земли.  На  молодого
Коперника Платоново поэтическое  видение мира подействовало сильнейшим
образом. Глубоко  взволновало его  одно место  в диалоге  «Тимей», где
речь идет об устройстве Вселенной: «Земле он определил быть кормилицей
нашей,  и так  как она  вращается  вокруг оси,  проходящей сквозь  всю
Вселенную,  блюстительницей и  устроительницей  дня  и ночи»  (Ревзин,
1949,  с. 150).  На с.  151 Ревзин  указывает, что  в пояснение  этого
темного  места  Кодрус приводит  мнение  Теофраста,  как его  приводит
Плутарх (Платоновские вопросы  VIII, I, см. Уэвель,  с. 203): «Платон,
когда стал стар, раскаялся в том,  что прежде отвел Земле срединное во
Вселенной  место,  которое  ей  вовсе  не  принадлежало».  Комментарий
Ревзина  (с. 151):  «В этом  месте ``Тимея''  Платон явно  колеблется,
он  пуще всего  боится  оскорбить ``бессмертных  богов''. Он  нарочито
придает  словам  своим  туманную,  двусмысленную форму.  Но  даже  еле
внятный  намек  в  устах  Платона  должен  был  произвести  сильнейшее
впечатление на  Коперника». Ревзин,  как и многие  другие, отбрасывает
современность  в прошлое.  Весь  «Тимей»  проникнут таким  религиозным
духом, что, несомненно, является  глубоко искренней попыткой построить
«естественное  богословие».   Несмотря  на  смелую   критику  афинских
порядков,  Платон,  как  известно,  никаким  преследованиям  в  Афинах
не  подвергался.  Самое  важное   в  этом  деле:  несомненное  влияние
Платона  на  Коперника (подробнее  об  этом  скажем в  соответствующем
месте). Роль же Платона  была действительно колеблющаяся, как крупного
мыслителя,  видящего  доводы  в  пользу  того  и  другого  толкования.
Никто  не считает  Платона основоположником  гелиоцентрической системы
и  не  оспаривает роль  Аристарха,  но,  несомненно, что  то  брожение
умов, которое  существовало в  платоновской Академии  и поддерживалось
и  стимулировалось  Платоном,  и было  источником  всего  последующего
прогресса астрономии.

\subsection{Резкая    разница     между    последователями    Платона,
стремившимися    к   математизации    всех   наук,    и   Аристотелем,
довольствовавшимся приблизительными объяснениями}

В  предыдущих параграфах  18--25  были  достаточно подробно  разобраны
космологические воззрения Платона. Полезно  было бы подвести итоги, но
эти  итоги  уже  были  подведены  более  ста  лет  назад  (1837--1846)
Уэвелем  в  его  истории   индуктивных  наук  в  отделах,  посвященных
«Тимею»  и   «Республике»  («Государству»)  Платона.  Это   тем  более
замечательно, что Уэвель, как  известно, в основном был последователем
родоначальника  английского материализма  Фр. Бэкона.  На с.  132--136
читаем: «Впрочем,  мы оставили бы  философию древних греков,  не отдав
должной справедливости  тем услугам, которыми физическая  наука во все
последующие века  была обязана  остроумному и проницательному  духу, в
котором велись их исследования в  этой области человеческого знания, и
широким  и  возвышенным  стремлениям,  которые  были  ими  обнаружены,
даже  в самой  их неудаче,  ---  если бы  не вспомнили  разнообразного
и  многообъемлющего  характера их  попыток  и  не вспомнили  некоторых
причин, ограничивших  их успехи в положительной  науке. Они занимались
умозрением  и теорией  под живым  убеждением, что  наука возможна  для
всех областей  природы и  что она  составляет достаточный  предмет для
упражнения  лучших  способностей  человека;  и  они  быстро  пришли  к
убеждению,  что  такая наука  должна  облечь  свои заключения  в  язык
математики.  Это убеждение  чрезвычайно ясно  в сочинениях  Платона. В
``Республике'', в  ``Эпиномисе'' и особенно в  ``Тимее'' это убеждение
заставляет  его  несколько  раз  возвращаться  к  обсуждению  законов,
которые  были установлены  или предполагаемы  в то  время относительно
гармонии  и оптики,  в том  виде,  как мы  видели выше,  и еще  больше
относительно астрономии, как мы увидим в следующей книге. Вероятно, ни
одна из дальнейших ступеней в  открытии законов природы не имела такой
важности,  как полное  усвоение  того  господствующего убеждения,  что
должны существовать  математические законы  природы и  что обязанность
философии  --- открыть  эти законы.  Во все  последующие века  истории
науки  это  убеждение  продолжает быть  одушевляющим  и  подкрепляющим
принципом  научных   исследований  и  открытий.  И   в  особенности  в
астрономии многие из ошибочных догадок, сделанных греками, заключают в
себе, если не  зародыш, то, по крайней мере,  оживляющую кровь великих
истин, которые были предоставлены будущим векам».

«И  кроме того,  греки искали  не только  таких теорий  для объяснения
специальных частей  природы, но и  общей теории Вселенной.  Опыт такой
теории есть ``Тимей'' Платона ---  попытка, слишком обширная и слишком
гордая, чтобы  иметь успех в  то время;  или, пожалуй, в  том размере,
в  каком  он  развивает  ее,  даже   и  в  наше  время  (середина  XIX
в.  ---  \emph{А.Л.}), но  сильный  и  поучительный пример  притязаний
человеческого ума --- объяснить всемирный порядок вещей и отдать отчет
во всем, что представляется ему внешними чувствами».

«Далее,  мы видим  в  Платоне,  что виной  неудачи  этой попытки  было
между  прочим предположение,  что причина,  почему все  вещи суть  то,
что  оно  есть,  и  как  оно  есть,  должна  быть  та,  что  эти  вещи
суть  лучшие,  по  тем  мнениям  о лучшем  и  худшем,  какие  доступны
человеку.  Сократ,  в  своей  предсмертной беседе,  как  она  передана
в  ``Федоне'',  объявляет,  что  именно этого  он  искал  в  философии
своего  времени, и  говорит своим  друзьям, что  он покинул  умозрения
Анаксагора, потому что  они не давали ему таких  причин для построения
мира.  ``Тимей''  Платона  есть  в сущности  попытка  восполнить  этот
недостаток и  представить теорию  Вселенной, где все  вещи объясняются
подобными  причинами.  Хотя  это  и была  неудача,  это  была  неудача
благородная и поучительная». Уэвель ссылается на Томпсона, что Платону
принадлежит заслуга  открытия, что  законы физической  Вселенной могут
быть  изображаемы  математическими  формулами  и что  об  этой  истине
Аристотель не имел  ни малейшего сознания. С.  135: «``Тимей'' Платона
заключает в себе схему математических и физических учений о Вселенной,
и  по  этой  схеме  ``Тимей''  гораздо  больше,  чем  какое-нибудь  из
произведений  Аристотеля,  представляет  аналогию с  теми  трактатами,
которые  появлялись  в  новейшие  времена  под  названиями:  Принципы,
Система  мира и  т.п. И,  к  счастью, это  произведение изучаемо  было
хорошо,  и при  том обращено  было внимание  не только  на язык,  но и
на  самые учения  и  их отношение  к  нашему действительному  знанию»;
указания, в  частности, на Штальбаума, Ботлера,  Томпсона и двухтомное
сочинение о «Тимее» проф. Мартена (1841).

\subsection{Плодотворное влияние Платона на развитие астрономии вплоть
до Ньютона неоспоримо}

Извлекли ли  из Платона ученые  все ценное,  что можно извлечь,  и то,
что  осталось, можно  целиком  оставить историкам  науки и  философии,
филологам  и  проч.?   И  на  этот  счет  у   Уэвеля  есть  интересные
высказывания, с. 149--151: «В десятой книге ``Республики'', на которую
мы указываем в тексте, прялка, которую Судьба или Необходимость держит
между  своими коленями,  с прицепленными  к ней  кольцами, посредством
которых  планеты  обращаются вокруг  нее,  как  вокруг оси,  ---  есть
уже  шаг  к  представлению   вопроса  в  смысле  построений  известной
машины». «Неудивительно поэтому, что Платон ожидал, что астрономия при
дальнейших успехах  способна будет объяснить многие  вещи, которые она
еще не  объяснила даже и  в наше время.  Таким образом, в  одном месте
седьмой книги ``Республики'' он  говорит, что для объяснения пропорции
между  днем и  месяцем  и между  месяцем и  годом  философ ищет  более
глубокого  и  более  существенного  основания,  чем  может  дать  одно
наблюдение.  И  астрономия  еще  не показала  нам  и  теперь  никакого
основания, почему  бы пропорция времен  обращения Земли на  своей оси,
обращения Луны  около Земли  и обращения Земли  около Солнца  не могла
быть сделана Создателем совершенно иной,  чем она есть. Но, спрашивая,
таким образом,  от математической  астрономии резонов, которых  она не
может дать,  Платон делал  только то, что  в позднейшем  периоде делал
автор  высоких астрономических  открытий,  Кеплер.  Один из  вопросов,
решения которых Кеплер в особенности  желал, был тот, почему есть пять
планет  и  почему  именно  на  таких  расстояниях  от  Солнца?  И  еще
любопытнее  то, что,  по его  мнению, он  нашел причину  этих вещей  в
отношениях тех  пяти правильных твердых  тел, которые, как  мы видели,
Платону  хотелось ввести  в  философию Вселенной.  Мы находим  рассказ
Кеплера  об  этом  воображаемом открытии  в  Мистериум  Космографикум,
изданном в 1596, как показано в нашей истории, кн. V, гл. IV, §2».

«По  мнению  Кеплера,  закон,  определяющий, таким  образом,  число  и
величину планетных орбит посредством пяти правильных твердых тел, есть
открытие не менее замечательное и несомненное, чем три закона, которые
дают его имени вечную славу в истории астрономии».

«Но  из  этого  мы  не  должны думать,  что  нет  твердого  критериума
для  различения между  воображаемыми  и  действительными открытиями  в
науке.  Как  открытие  делается  возможным  при  полном  просторе  для
догадок, так  оно делается  действительным при полном  просторе, какой
дается  наблюдению, ---  постоянно  и  решительно определять  ценность
догадок. Со  смелым воображением Платона Кеплер  соединил терпеливую и
добросовестную привычку проверять свои фантазии строгим и трудолюбивым
сравнением  их с  явлениями; и  таким  образом его  открытия повели  к
открытиям Ньютона».

Разберем  критически  эти  высказывания  Уэвеля  позже,  после  оценки
значения Аристотеля в  космологии, а пока сопоставим  этот хвалебный в
общем отзыв представителя механического материализма середины XIX в. с
отзывом крупного ученого  и представителя диалектического материализма
середины XX  в. Дж.  Бернала (Бернал, 1956,  с. 116):  «После закрытия
Академии  произведения Платона  в оригиналах  были целиком  забыты, за
исключением наиболее абсурдного из  них --- ``Тимея'', содержащего его
мистический  взгляд на  образование  мира. Учение  его  было (с.  135)
передано  через  неоплатонизм  еще  большего  мистика  Плотина.  Арабы
обнаружили некоторые из других работ  Платона и перевели их, но только
в эпоху  Возрождения работы Платона  были вновь изучены в  оригинале и
оказали влияние, по крайней мере, столь  же большое, как и в то время,
когда они  были написаны. Главным образом,  благодаря Платону, взгляды
представителей раннего гуманизма  не были научными. В XVI  и XVII вв.,
однако, свойственное Платону увлечение математикой сыграло важную роль
в формировании  мышления Кеплера,  Галилея (с. 234  и далее)  и, через
кембриджских платоников, также и Ньютона  (с. 265)». По мнению Бернала
и  других современных  материалистов,  примесь  религиозных мотивов  в
творениях плодотворнейших философов (а плодотворность взглядов Платона
вовсе не  отрицается) достаточна для  того, чтобы объявить  их взгляды
ненаучными,  а трудно  написанные сочинения  --- абсурдными.  Для нас,
русских,  такая «установка»  хорошо знакома:  мы знаем,  что некоторые
наши  новаторы и  поддерживавшие  их философы  и  Пастера зачисляли  в
реакционеры и обскуранты. С фанатиками спорить невозможно.

\subsection{Популярность Аристотеля объясняется  как его достоинствами
(широта,   систематичность,  глубина   мысли),   так  и   недостатками
(отсутствие математизации науки)}

Перейдем теперь  к Аристотелю.  Как известно, Аристотель  был учеником
Платона,  долгое   время  работал  в  Академии,   сохранил  многое  от
Платона,  но затем  настолько  уклонился от  своего учителя  (вспомним
известное: «друг  Платон, но  истина мне больший  друг»), что  эти две
линии  Платона и  Аристотеля никак  не следует  смешивать. Оторвавшись
от  платоновского  идеализма,  он  не  сделался  материалистом  и  для
сторонников  учения о  «двух  лагерях»  он представляет  непреодолимый
камень  преткновения,  так  как  он  объединяет  в  себе  черты  обоих
лагерей.  Очень  многие  считают Аристотеля  вершиной  древнегреческой
философии.  На этом  сходятся  такие противоположности,  как, с  одной
стороны,  схоластическая философия  средних веков,  сейчас развиваемая
католическим  неотомизмом  (возрождение  учения  Фомы  Аквинского),  с
другой  стороны, не  кто иной,  как Карл  Маркс, назвавший  Аристотеля
Александром Македонским греческой  философии (История философии, 1941,
т.  1,  с.  180).   Виднейший  представитель  неовитализма,  Г.  Дриш,
использовал для своего биологического мировоззрения понятие Аристотеля
«энтелехия»,  труднейшее  понятие,  которое можно  истолковать  и  как
существительное и  как обстоятельство  образа действия.  Конечно, если
рассматривать  великое  эллинское   наследство  как  собрание  мертвых
материалов, интересных  только для истории науки,  или могущих служить
фундаментом  для  окостенелых  учений,  то  этот  взгляд  на  значение
Аристотеля  оспаривать   не  приходится.   Вне  области   точных  наук
эвристическое  значение  творений   Аристотеля  и  сейчас  неоспоримо.
По  диапазону  научных  интересов, колоссальной  эрудиции,  совершенно
изумительной   наблюдательности   и   работоспособности,   способности
синтезировать  разнороднейшие и  противоречивые  воззрения и  излагать
их   связно  и   систематически,  Аристотель   не  уступит,   конечно,
никому  во  всей истории  человеческой  мысли.  Его формальная  логика
считалась Кантом  «законченной наукой»  и сохранилась  со сравнительно
незначительными  изменениями  до настоящего  времени,  пока  в нее  не
вторглась математика. Он сумел взять многое от антиподов --- Платона и
Демокрита, отчего его нельзя отнести ни к одной из «линий». Он положил
начало  систематической  зоологии,  гистологии и  эмбриологии,  а  его
ученик  Теофраст ---  ботанике и  минералогии. В  гуманитарной области
его  сочинения по  этике и  политике  и сейчас  не потеряли  свежести.
Как  будто  совершенно  универсальный гений?  Нет,  не  универсальный!
Слабость Аристотеля в  том, что составляет главную  силу школ Пифагора
и  Платона:  недооценка  математики,  и  эта  слабость  объясняет  его
популярность  во   многих  кругах   и  то  вредное   влияние,  которое
Аристотель,  а в  особенности  его усердные  ученики и  последователи,
перипатетики, оказали на дальнейшее развитие науки. На дверях Академии
Платона стояли  строгие слова:  «Да не вступает  никто, не  знакомый с
геометрией». Аристотель  таких требований  не предъявлял и  к Пифагору
относился довольно  пренебрежительно: по  его мнению,  «Пифагор сперва
занимался  математикой и  изучил  свойства чисел;  впоследствии же  он
недалеко  ушел  от сказок  Фередика»  (учитель  Пифагора, см.  Бляшке,
1957,  с.  113). По  мнению  Энгельса,  Аристотель правильно  упрекает
пифагорейцев  в  том, что  своими  числами  «они не  объясняют,  каким
образом возникает движение, и как без движения и изменения имеют место
возникновение и уничтожение, или  же состояния и деятельности небесных
вещей» (Метафизика, I, 8) (Энгельс, 1949, с. 148).

Это  противоположение  является  основным в  сопоставлении  платонизма
и  перипатетизма,  вернее,  чистой пифагорейско-платоновской  линии  и
перипатетизма,  так  как  в   платонизме  и  далее  были  ответвления,
порывавшие  связь  с  математикой.  Для   истории  же  наук  точных  и
стремящихся быть точными это противоположение --- основное.

Великолепно  это  выражено  в   знаменитом  диалоге  Галилея  «О  двух
главнейших  системах»  с.   284--285:  «Симпличио  (перипатетик):  Эти
умозрения  (если  я смею  откровенно  высказать  свое мнение)  кажутся
мне теми  геометрическими тонкостями,  за которые  Аристотель упрекает
Платона, обвиняя  его в том,  что слишком усердные  занятия геометрией
удалили  его   от  настоящего   философствования;  я  знал   и  слушал
величайших философов-перипатетиков, которые  советовали своим ученикам
не  заниматься   математическими  науками,  так  как   они  делают  ум
придирчивым и неспособным к правильному философствованию, --- правило,
диаметрально противоположное  правилу Платона,  который не  допускал к
философии того, кто не овладел предварительно геометрией».

«Сальвиати  (выражает  мнение  Галилея. ---  \emph{А.Л.}):  Я  одобряю
совет  этих перипатетиков,  удерживающих своих  школьников от  занятий
геометрией,  так  как  нет   ни  одной  науки,  более  приспособленной
для  раскрытия их  ошибок;  но  вы видите,  насколько  они отличны  от
философов-математиков:  последние гораздо  более  охотно рассуждают  с
теми, кто хорошо осведомлен  в обычной перипатетической философии, чем
с теми, у  кого нет таких познаний  и кто в силу  такого недостатка не
может провести параллели между двумя учениями». Великолепная ирония по
отношению  к  современным  биологам,  чуждающимся  математики,  прежде
всего, к огромному большинству дарвинистов.

\subsection{Различие  подходов Платона  и Аристотеля  к науке  связано
как  с   общефилософскими  представлениями,  так  и   с  требованиями,
предъявляемыми   к   науке:   точное  описание   или   приблизительное
объяснение}

С  точки  зрения  методологии  науки это  отличие  является  основным:
если  вы придаете  математике  лишь вспомогательную  роль в  понимании
природы,  то вы  --- последователь  Аристотеля. Если  же ей  придается
фундаментальное  значение   в  естествознании,  то  вы   ---  платоник
или  пифагореец  (Госкин,  1960,  с.  17).  Можно  подумать,  что  это
отличие  независимо от  философии и  что  оно связано  прежде всего  с
большей  или  меньшей любовью  ученого  к  математике и  большими  или
меньшими  способностями. До  известной  степени  это, конечно,  верно,
и  многие  невежды  в  математике  пытаются  свой  дефект  возвести  в
добродетель,  например,  современным   утверждением,  что  Мендель  не
имеет  никакого значения  в науке.  Но  даже в  настоящее время  можно
привести высказывания выдающихся математиков, считающих, например, что
в  биологии трактовка  органических  форм  не доступна  математизации.
Это  возражение  легко  формулировать  словами  моего  усопшего  друга
В.  Н.  Беклемишева,   сказанными  им  еще  в   дни  нашей  молодости:
«Органическая форма  есть эпифеномен сложнейших  процессов, невозможно
ожидать,  чтобы такие  сложные процессы  могли привести  к результату,
охватываемому сравнительно  простой формулой». То, что  сейчас говорят
относительно биологии на  заре науки, говорили и  о космологии, исходя
из общефилософских позиций. Если принимать  в природе только атомы или
другие изолированные  тела, двигающиеся в беспорядке  и сталкивающиеся
друг  с другом,  то  как  можно ожидать,  что  из  такого хаоса  могут
возникнуть  математически  правильные  круговые  движения  или  вообще
сравнительно просто  формулируемые математические  законы? Уверенность
в   возможности  или   невозможности  математизации   явлений  природы
теснейшим  образом  связана  с теми  требованиями,  которые  различные
ученые  предъявляют  к научной  теории.  Одни  ученые стремятся  найти
количественные  соотношения  в  явлениях природы,  дающие  возможность
прогноза  и  управления  явлениями, придавая  второстепенное  значение
«объяснению»  с   точки  зрения  обычного  здравого   смысла.  Другие,
напротив,  заинтересованы прежде  всего тем,  чтобы свести  различия к
«качествам», не особенно интересуются количественной стороной предмета
и охотно  удовлетворяются часто  призрачными «объяснениями».  К первой
категории относится Платон, ко второй --- Аристотель. Бернал, которого
нельзя  упрекнуть в  симпатиях к  Платону, совершенно  правильно пишет
(1956,  с.  118),  что  предмет научного  исследования  по  Аристотелю
заключался в  отыскании природы всех  вещей. Он должен  был охватывать
все,  начиная от  объяснения того,  почему  все камни  падают вниз,  и
кончая  тем, почему  некоторые люди  являются рабами.  В любом  случае
ответ был  одинаковым: «Такова уж  природа их». Это  был действительно
всеобъемлющий ответ, равноценный фразе: «Они таковы потому, что такова
воля бога», но он звучал более научно... В работах Аристотеля «Физика»
и  «На небесах»  он  применяет  свой метод  к  тому,  что мы  называем
физической  Вселенной,  там,  где  это  меньше  всего  применимо.  Его
объяснение было едва ли  более правдоподобным, чем объяснение Платона,
и было  лишено как эмоциональной приподнятости,  так и математического
интереса. Но так как оно  являлось частью великого учения Аристотеля о
логическом строении  Вселенной, оно  стало основной формой,  в которой
теория греков о строении Вселенной была передана потомству. Ей суждено
было доказать,  в частности,  свою бесплодность для  прогресса физики.
Джордано Бруно  должен был быть  сожжен и Галилей осужден  прежде, чем
доктрины, которые  были взяты скорее  из концепции Аристотеля,  чем из
Библии, были  разбиты (с.  236 и далее).  Последующая история  науки в
большей части  в действительности является  историей последовательного
развенчивания Аристотеля. Поистине Рамус  был недалек от истины, когда
он утверждал в своем знаменитом тезисе  в 1536 г., что «все, чему учил
Аристотель, является ложным».

\subsection{Дух   Аристотеля,  приносящий   огромный  вред   благодаря
догматическим эпигонам,  полезен во всех областях  науки, не созревших
для математического толкования}

В другом  месте Бернал  показывает, что огромный  авторитет Аристотеля
объясняется  именно его  большей доступностью  и приспособленностью  к
консервативному образу мышления (там же,  с. 121). «Как будет показано
в  следующем разделе  (4.7),  то, что  вытекало из  исследовательского
метода Аристотеля, очень скоро  должно было подорвать или опровергнуть
большинство  его  умозаключений,  включая основной  вывод  о  конечных
причинах. В  действительности его взгляды на  многие проблемы устарели
еще до того,  как он их выдвинул. Однако он  оказывал огромное влияние
на арабскую и средневековую мысль,  несмотря на такие ограничения или,
возможно, благодаря им. Последние достижения греческой науки были либо
полностью утрачены, либо, подобно  трудам Архимеда, не признавались до
эпохи  Возрождения.  Их  никто  не  мог  понять,  кроме  очень  хорошо
подготовленных и  искушенных читателей,  которых нелегко было  найти в
эпоху  раннего  средневековья. Однако  труды  Аристотеля  при всей  их
громоздкости  не требовали  (или казалось,  что не  требовали) для  их
понимания ничего, кроме здравого  смысла. Аристотель, подобно Гитлеру,
никогда  не говорил  кому-либо что-то  такое,  во что  те не  поверили
бы.  Не было  необходимости в  опытах  или приборах  для проверки  его
наблюдений,  не  нужны  были  трудные  математические  вычисления  для
извлечения результатов  из них или мистическая  интуиция для понимания
какого бы то  ни было внутреннего смысла.  Платон действительно больше
обращался к  воображению и  обладал большей моральной  страстностью, а
Аристотель объяснял, что мир такой, каким все его знают, именно такой,
каким  они  его знают...  До  тех  пор,  пока  мир оставался  тем  же,
Аристотель был приемлем, но, как мы увидим, мир не оставался тем же».

Вот и  разберись! Сам Карл  Маркс называл Аристотеля (в  виде похвалы)
Александром  Македонским  греческой  философии, а  человек,  считающий
себя последователем  Карла Маркса,  и сам  --- крупный  физик, считает
его  учение почти  сплошь  ложным и  сравнивает  с Гитлером.  Конечно,
между  Александром  Македонским  и  Гитлером то  общее,  что  оба  ---
деспоты,  но деспотизм  Александра  заключал  много прогрессивного,  а
Гитлер  стремился  повернуть  колесо истории  вспять.  Разумеется,  ни
Рамус  с  его решительным  отвержением  всего  Аристотеля, ни  Р.Бэкон
с  его  рекомендацией сожжения  всех  сочинений  Аристотеля не  правы,
да,  как увидим  дальше,  смысл их  изречений  был неправильно  понят.
Врагами перипатетизма,  как ясно из  истории идей Возрождения,  был не
столько сам  Аристотель, сколько  его не  по разуму  усердные ученики,
но  такова судьба  большинства  великих  мыслителей. Эпигоны  забывают
о  лучших  чертах  своего  учителя  и  настаивают  как  на  абсолютных
истинах  на  его ошибках.  Такое  свойство  эпигонов сохранилось  и  в
нашем, двадцатом  веке, и, вероятно,  оно сохранится до тех  пор, пока
точное  математическое исследование  не  проникнет  решительно во  все
области всех  наук. Поэтому аристотелевский дух  безусловно полезен во
всех  областях  наук, не  созревших  еще  до математической  обработки
и  нуждающихся  в чисто  логической  систематизации.  Мало того,  и  в
доступных для математизации науках главное аристотелевское направление
может принести пользу и принесло в истории науки большую пользу. Здесь
мы  подходим к  другому отличию  Аристотеля от  Платона, связанному  с
учением о причинах.

\subsection{Многообразие и неравноценность причин всех явлений}

Уже  у Платона  есть ясные  указания,  что то,  что обычно  называется
причиной, может  иметь совершенно разный характер  и что одно и  то же
явление  имеет  несколько  причин.  В  основном  логическом  сочинении
Аристотеля «Аналитики» --- это изложено так (Вторая аналитика, гл. II,
1952, с. 266).  «Причин же существует четыре вида.  Первая --- которая
объясняет суть бытия  вещи, вторая --- что это  необходимо есть, когда
есть  что-то  (другое),  третья  --- то,  что  есть  первое  движущее,
четвертая  ---  то,  ради  чего  (что-нибудь  есть)».  Как  указано  в
примечаниях на с.  423--424, по созданной в  Средние века терминологии
эти причины названы: 1)  материальная --- материя, материальная основа
всех  вещей  (то, что  лежит  в  основании, субстрат);  2)  формальная
причина --- форма, активная сила,  суть предмета; 3) производящая, или
движущая, причина  --- источник перемены явлений,  начало движения; 4)
конечная причина --- цель движения. Там же приведен пример: архитектор
и его искусство --- производящая причина; план --- формальная причина;
строительный материал  --- материальная  причина, построенный  дом ---
«конечная причина» (цель).

Но из того, что каждое явление  имеет несколько причин, не значит, что
все причины  равноценны. Мы имеем  право ту  или иную из  них признать
ведущей. Великолепное  рассуждение на эту тему  имеется в предсмертной
беседе Сократа  (Платон, Федон, 98--99). Сократ  критикует Анаксагора,
«который вовсе не пользуется разумом и не указывает никаких причин для
объяснения  устройства  мироздания,  но,  вместо  того,  ссылается,  в
качестве причин, на воздух, эфир, воду и многие подобные несуразности.
Я вынес  впечатление, что  Анаксагор попал в  такое же  положение, как
если  бы кто  сказал: все,  что делает  Сократ, он  делает посредством
своего  разума,  а затем,  пытаясь  указать  причины каждого  из  моих
поступков, стал  бы говорить так:  я сижу  здесь потому, что  мое тело
состоит из костей и мускулов, потому что кости тверды и разделены одна
от другой суставами, мускулы  же способны растягиваться и сокращаться,
что  кости окружены  плотью и  охватывающей  ее кожей.  Так как  кости
подвижны в своих  сочленениях, то мускулы, вследствие  их сокращения и
растяжения,  дают мне  возможность  сгибать мои  члены,  что и  служит
причиною, почему я сижу здесь,  согнувшись. Пожалуй, и по поводу нашей
беседы Анаксагор стал бы приводить  такого же рода причины и ссылаться
на звук,  воздух, слух  и т.п. до  бесконечности, а  истинные причины,
поведшие к беседе, пренебрег бы  назвать: именно, что афиняне сочли за
лучшее  осудить  меня и  что  поэтому  и  я,  со своей  стороны,  счел
за  лучшее  сидеть здесь  и,  в  ожидании более  справедливой  участи,
подвергнуться тому наказанию, к  какому афиняне меня приговорили. Ведь
клянусь собакою,  мне думается, эти  мускулы и  кости давно были  бы в
Мегарах, либо в Беотии, если бы я счел это за наилучшее и если бы я не
считал  более  справедливым  подвергнуться тому  наказанию,  к  какому
присудило  меня государство.  Однако приводить  такие вещи  в качестве
причин совсем  нелепо. А если бы  кто-либо сказал, что, не  имея всего
этого,  т.е. мускулов  и  костей и  всего прочего,  чем  я обладаю,  я
не  в состоянии  привести  в  исполнение свое  решение,  он сказал  бы
правду. Сказать  же, что все  эти поступки я  совершаю, руководствуясь
разумом, и  при этом заявлять,  что причиною моего  поведения являются
названные вещи,  а не предпочтение  наилучшего, было бы очень  и очень
легкомысленно. Ведь это значило бы  не быть в состоянии различить, что
одно есть  действительно причина,  а другое ---  то, без  чего причина
никогда не была бы причиною».

\subsection{Платонизм  связан с  особенно  высокой оценкой  формальных
причин,   у  Аристотеля   ---  конечных,   но  Аристотель,   оставаясь
идеалистом, делает ряд шагов в направлении материализма}

Совершенно ясно,  что Платон  склонен называть  причиной лишь  то, что
в  смысле  Аристотеля  можно  назвать главной  или  ведущей  причиной,
а  вспомогательные причины  называть  условиями осуществления  главной
причины. Мы знаем,  и к этому нам  придется возвращаться неоднократно,
что с  господством механического материализма  «причинным объяснением»
стали  называть  сведение  всех  явлений к  материальным  (обычно  это
называлось условиями)  и к  действующим причинам, конечные  же причины
перестали считать причинами вовсе. А  как же в математике --- основной
области проявления  формальных причин? Так как  в математике следствие
вовсе  не следует  за причиной,  то  тут тоже  стали избегать  термина
«причина» и в общей форме  говорить о «законе достаточного основания».
Этот термин был максимально широким, и Шопенгауэр так и назвал одно из
своих сочинений: «О четверояком  корне закона достаточного основания».
Но эти четыре  корня повторяют четыре формы  причинности Аристотеля. У
Аристотеля повсюду на первый план  выдвигается конечная причина, но он
отнюдь не  отрицает важности действующих  причин. Энтелехия и  есть то
реальное, что осуществляет цель  (конечные причины). Разумеется, он не
был чужд признания формальных причин,  но поскольку форма у Аристотеля
мыслилась в  неразрывном единстве  с материей (хюле  и морфе),  она не
имела онтологического обоснования. У  Платона же, несомненно, конечные
причины  играют  тоже  существенную  роль,  но  особенное  значение  в
его  философии имеют  причины  формальные, онтологически  обоснованные
его  теорией идей  ---  учением специфическим  для Платона.  Признание
реального  существования  внепространственных  и вневременных  идей  и
есть  «объяснение»   возможности  математической  трактовки   мира,  к
которой все  время стремился Платон  и истинные его  последователи. Но
онтология Платона дуалистична:  мир идей и мир явлений  не находятся в
состоянии взаимно однозначного  (короче, биоднозначного) соответствия.
Мир  явлений  ---  искаженное  отображение   мира  идей  и  при  таком
искажении  ослабляется  возможность  математической  трактовки.  Всего
менее  искажен мир  в области  небесных тел:  вот почему  к космологии
и  были  обращены  стремления  самого Платона  и  его  последователей.
Платон  стремился внедрить  математические понятия  даже в  свои чисто
политические трактаты:  совершенные числа,  брачное число,  число 5040
---  произведение первых  семи  чисел для  определения  числа семей  в
поселке в «Законах» и т.д., но это все были намеки, терявшиеся в массе
материала, чуждого математике. Дальнейшая  история науки показала, что
кризис  платонизма произошел  не от  поражений, а  от слишком  больших
побед в области математизации  науки. Математизация оказалась возможна
и там, где не было никакой надобности в теории идей как онтологическом
обосновании. Так для чего же тогда теория идей?

Аристотеля  справедливо считают  философом, сделавшим  большой шаг  по
направлению к материализму,  хотя он в общем  остается еще идеалистом.
Этот шаг  к материализму  выражается рядом  свойств. Во-первых,  и сам
Аристотель и его ученики были склонны считать небесные сферы реальными
твердыми  телами  (Уэвель,  с.  237), в  то  время  как  последователь
платоновской  линии  Птолемей  считал  их  воображаемыми.  Да  систему
Птолемея и невозможно себе  вообразить материально. Во-вторых, отрицая
теорию  идей,  Аристотель отказался  и  от  искания идеалов.  В  своих
примечаниях к  Уэвелю Литтров  пишет (Уэвель,  с. 522):  «Платон искал
науки, возвышающейся  над ограниченностью земных отношений,  которые и
он  признает  и должен  признавать,  и,  оставляя в  стороне  нынешние
ограниченные отношения  человека, рассматривает  его в его  будущем, в
более чистом и высоком состоянии. Но Аристотель рассматривает человека
так,  как  находит  его,  и  к  этому  настоящему  человеку  старается
приноровить  и свою  науку».  Там же,  с. 573:  «Платон  жил больше  в
будущем, чем в настоящем, он жил  надеждами и питался идеями. С другой
стороны, более мужественный ум  Аристотеля твердо и уверенно опускался
в глубину настоящего»...

В  астрономии  роль  Аристотеля  заключается  в  том,  что  он  прочно
обосновал учение о шарообразной  форме Земли и пропагандировал систему
гомоцентрических сфер. Эта система давала приблизительное «объяснение»
небесных  явлений,  но  «не  заключала   в  себе  даже  и  возможности
правильного  подхода  к  проблеме  определения  планетных  расстояний:
поэтому  она фатально  непригодна для  астронома-теоретика» (Идельсон,
1947, с. 85),  и подлинная астрономическая наука  древних начинается с
Гиппарха и Птолемея.

\subsection{Противоположение взглядов  Платона и Аристотеля.  Сократ и
Анаксагор}

От  Академии  Платона  берут   начало  два  крупнейших  направления  в
философии  и  науке:   собственно  платоническое  и  перипатетическое.
Как   ясно   из   изложенного,  они   хорошо   характеризуются   рядом
противоположений: 1) романтическое воображение, талантливая интуиция и
«трезвое»  отношение  к  действительности; 2)  теория  вневременных  и
внепространственных  идей  и отрицание  таких  идей;  3) стремление  к
математизации науки и избегание математики; 4) господство в построении
объяснения формальной и финальной (конечной) причинности; 5) юношеское
устремление в  будущее и старческий консерватизм.  Выше было показано,
что все эти  противоположения отнюдь не являются  независимыми друг от
друга,  а показывают  органическую,  хотя и  не  вполне тесную  связь.
Аристотель был  мастер систематизации и  в этом отношении  его заслуги
бесспорны, но он был беспомощен в прогнозе, «...мы не можем указать ни
одного  из  принимаемых  теперь  физических учений,  которое  было  бы
предугадано  у Аристотеля,  в  том роде,  как  система Коперника  была
предугадана Аристархом,  как размещение  небесных движений  в круговые
было указано Платоном, как  объяснение численных отношений музыкальных
интервалов  приписывается Пифагору»  (Уэвель, 1867,  с. 78).  И вполне
закономерно, что когда человечество  переживало свою новую молодость в
период  Возрождения, оно  вновь  обратилось к  Платону.  Вместе с  тем
протест против господства Аристотеля принял чрезвычайно резкие формы и
некоторые  из таких  протестантов  заплатили за  свой протест  жизнью,
например, Рамус (Уэвель, с. 574).

Все      предыдущее      изложение       касалось      в      основном
пифагорейско-платоновского  направления  эллинской культуры,  и  вывод
был  тот,  что это  направление  и  является стержнем  этой  культуры.
Ответвление  платонизма,  перипатетизм,  дало   очень  много,  но  это
направление не привело к вершинам науки. Но обе рассмотренные линии не
являются материалистическими. Для полноты картины необходимо разобрать
взгляды «линии Демокрита». Задача несколько облегчена тем, что крупный
вклад  в  космологию  (шарообразность  Земли,  отказ  от  геоцентризма
и  проч.), сделанный  пифагорейцами,  не  оспаривается и  решительными
противниками  пифагореизма, как  Лурье (см.  §6), но  приписывается им
«случаю». Но любопытно посмотреть  космологические взгляды Демокрита и
других материалистов.

Из  предшественников  Демокрита,  вероятно, наиболее  ценный  вклад  в
астрономию дал Анаксагор (500--428  до н.э.), утверждавший, что Солнце
есть просто раскаленное  тело. За это он едва избежал  казни в Афинах.
Кроме  того, он  первый  указал, что  Луна  светит отраженным  светом,
дал  правильное  объяснение затмений.  Он  считал,  что Солнце  больше
Пелопоннеса, принимал,  что на Луне  есть горы,  и он думал  даже, что
Луна населена (Б.Рассел,  1959, с. 81). Платон,  который родился через
год после  смерти Анаксагора (427  до н.э.), знал учение  Анаксагора и
критикует его,  как всегда,  от имени Сократа  в своем  «Федоне» (97):
«Кто-то, как я однажды слышал,  прочитал в одном сочинении Анаксагора,
что  разум ---  устроитель и  причина всех  вещей. Я  обрадовался этой
причине и  решил, что дело,  до известной степени,  налаживается, коль
скоро разум  есть причина  всего; если так,  думал я,  если устрояющий
разум все устраивает, то и каждую  вещь он помещает туда, где ей лучше
всего  находиться... Соображая  обо  всем этом,  я  пришел к  отрадной
мысли, что в лице Анаксагора я открыл учителя о причинах всего сущего,
который мне пришелся по сердцу, и  что этот учитель скажет мне, прежде
всего, какова Земля --- плоская или круглая, а когда скажет это, сверх
того объяснит мне причину и необходимость, почему это так должно быть,
причем  укажет, почему  Земле  лучше быть  таковой  (т.е. плоской  или
круглой). Если  Анаксагор скажет,  что Земля  находится в  центре, то,
думал я, он мне объяснит также, почему ей лучше находиться в центре. А
если бы он объяснил мне все это, я готов был отказаться от отыскивания
другого рода  причин. Я приготовился  таким же точно  образом получить
сведения  и  о  Солнце,  и  о  Луне, и  о  прочих  созвездиях,  об  их
относительной  скорости,  об  их  движении и  всех  прочих  свойствах,
(именно  узнать) почему  лучше, чтобы  каждое (из  этих небесных  тел)
совершало  и испытывало  то,  что оно  испытывает.  Так как  Анаксагор
говорил, что все  они приведены в порядок посредством разума,  то я не
думал, что он приведет для этого какую-либо иную причину, а только то,
что самое лучшее для них быть тем,  чем они есть. Я полагаю, что, если
Анаксагор укажет для каждого из них  и для всех вместе причину, то он,
сверх  того, и  объяснит, что  является наилучшим  для каждой  вещи, и
общее благо для всех их. И я не продал бы дешево своей надежды! Нет, я
с большим  рвением взялся за книги  и читал их так  быстро, как только
мог, чтобы как можно скорее узнать, что есть наилучшее и что худшее».

\subsection{Демократ не дошел до шарообразности Земли и не догадывался
о ее размерах}

Продолжение  всего  этого  приведено  в   §31,  и  мы  поймем,  почему
Платон  считает  воздух, эфир,  воду  и  т.д.  в качестве  причин  для
устройства мироздания  «несуразностями». Ведь  большинство современных
ученых  считает  несуразностью  как  раз  подход  Платона,  который  в
качестве  причины  берет  указание  на  то,  что  лучшее  для  Солнца,
Луны  и  планет.  А  так  как Платон,  как  явствует  из  изложенного,
хорошо знал  учение Анаксагора и  от него отталкивается, то  ясно, что
такой  ход мысли  Платона, с  точки зрения  современных материалистов,
является «реакционным».  Между прочим, на  этом примере мы  видим, что
Платон  вовсе  не  склонен  замалчивать  своих  противников,  как  это
инкриминируется ему в отношении Демокрита.

Анаксагор,  очевидно, является  непоследовательным материалистом.  Его
установка «разум  --- устроитель и  причина всей вещей»  --- выражение
идеалистического рационализма (см.  §27 главы III этой  работы) и даже
наиболее  крайнего  его выражения:  разумом  можно  не только  постичь
Вселенную, но Разум  и устрояет Вселенную. Но дальше  он не пользуется
этим  положением, и  за такую  непоследовательность Платон  его вправе
обвинить  в  «несуразности».   Но  многие  современные  «рационалисты»
понимают рационализм  в шестом смысле  §26, именно в  смысле отрицания
объективного  существования   Разума  вне  человека.   Их  рационализм
заключается  в  следующем:  хотя  Вселенная  устроена  не  Разумом,  а
Случаем и Борьбой, но законы Вселенной постижимы человеческим разумом.
Анаксагор в  своем основном постулате является  представителем первого
вида  рационализма,   в  конкретных  объяснениях,   вызвавших  критику
Платона,  --- другого.  По первой  линии пошел  Платон, по  другой ---
Демокрит, ставший  сознательно и  последовательно на  позиции шестого,
чисто  материалистического  вида  рационализма.  Предыдущее  изложение
показало развитие в космологии линии Платона, посмотрим, к чему пришел
Демокрит.

Начнем  с  представления о  форме  Земли.  В сохранившихся  фрагментах
Демокрита   имеются   достаточно  противоречивые   высказывания,   но,
по-видимому,   несомненно,  что   Демокрит  не   дошел  до   признания
Земли  шарообразной,  хотя  в  его  время  пифагорейцы  уже  принимали
шарообразность  Земли,  что  стало прочным  достоянием  школы  Платона
и  всех его  последователей,  включая  перипатетиков. С.Я.Лурье  пишет
(1947,  с.  206): «...по  Демокриту,  Земля  не шарообразна,  а  имеет
форму барабана  с вогнутыми  к центру основаниями.  В центре  одной из
этих  вогнутостей  находится Средиземное  море,  а  вокруг него  живут
люди  (см. вертикальный  разрез Земли  через центр  ее на  прилагаемом
рисунке 14). Это  тем более удивительно, что теория,  по которой Земля
шарообразна, была  уже в  ходу в  эпоху Демокрита,  и она,  как нельзя
лучше,  подходила к  его учению  о  вихре; в  результате вращения,  по
его  учению, получался  сферический  небосвод;  руководствуясь тем  же
принципом  симметрии, он  должен был  придавать и  центральным тяжелым
массам шарообразную  форму. Однако засвидетельствовано,  что некоторые
естествоиспытатели (по-видимому,  Демокрит или его  предшественники) в
своих  трудах  полемизировали  с теориями  сторонников  шарообразности
Земли  (Аристотель. О  небе).  Они  указывали на  то,  что будь  Земля
шарообразной,  нижний   край,  по  которому  диск   заходящего  Солнца
пересекается с горизонтом, имел бы форму  дуги, а не прямой линии, как
в действительности».

Из этой цитаты ревностного сторонника Демокрита ясно:

1) Демокрит не  признавал шарообразности Земли, хотя это  учение в его
время было распространено и ему было известно;

2)  Лурье удивляется,  что  Демокрит не  распространил принимаемое  им
положение  о сферичности  небосвода на  центральную тяжелую  массу, но
принцип симметрии --- пифагорейский,  а не материалистический принцип,
и таких требований к материалисту Демокриту предъявлять нет оснований;

3)  то  обстоятельство,  что,  по  мнению  Демокрита,  прямолинейность
пересечения   заходящего   Солнца    с   горизонтом   говорит   против
шарообразности  Земли,   показывает,  что  по  Демокриту   Земля  была
сравнительно невелика. Аргумент Демокрита не имеет силы именно потому,
что  Земля велика  и  потому небольшой  отрезок  окружности ---  линия
пересечения  Солнца и  горизонта не  отличается практически  от прямой
линии.  Для Платона  же  была  ясна огромность  Земли  (см. §21),  где
приведены  именно  и те  рассуждения  о  симметрии Вселенной,  которые
заставляют отрицать абсолютное понимание «верха» и «низа».

\subsection{Демокрит  принимал форму  Земли  как дисковидную  (а не  в
форме барабана) и не ушел от примитивных представлений}

Но  вряд  ли  можно  согласиться  с  той  реставрацией  разреза  Земли
(рисунок  14 ---  Лурье),  которую Лурье  приписывает Демокриту.  Если
верить этому  рисунку, то все отличие  демокритовского понимания формы
Земли  от  пифагорейского в  том,  что  на обоих  полюсах  сферической
Земли  сделаны выемки  в форме  часовых стекол.  Но если  бы это  было
так,  то  отпал бы  аргумент  Демокрита  против шарообразности  Земли,
основанный  на «прямолинейности»  линии пересечения  заходящего Солнца
и  горизонта.  У такого  «барабана»  линия  пересечения тоже  была  бы
криволинейна.  И  по-видимому, никто  из  прежних  авторов не  понимал
демокритовского толкования  формы Земли так,  как это делает  Лурье. В
цитированном в §3 резюме Коперника «тимпановидная» форма приписывается
Левкиппу, а  Демокриту «какая-либо иначе вогнутая».  Слово «тимпан» С.
Я.  Лурье  переводит как  «барабан»  по  аналогии, очевидно,  с  нашей
«барабанной перепонкой» (тимпанум). Лурье в цитате, приведенной в §34,
ссылается на Аэция и на  Ватиканское собрание изречений 381, но другие
переводчики  иначе  переводят  эти  места.  В  сборнике  «Материалисты
Древней Греции» (1955, с. 116) читаем: «157 Аэций III, 10, 5 (Д 377; о
форме Земли).  Демокрит: (Земля, рассматриваемая в  отношении) ширины,
имеет  форму диска,  посредине  же  она полая.  Евстафий  к  Н 446  р.
690.  Обитаемую (часть)  Земли  стоики Посидоний  и Дионисий  называют
пращевидной. Демокрит же ---  продолговатой. Ватикане, собрание изреч.
381: Земля ни  полая, как (думает) Демокрит, ни  плоская, как (думает)
Анаксагор».  На с.  123  того  же сборника  мы  читаем, из  Агафемера:
«Первый же Демокрит, многосведущий муж, познал, что Земля продолговата
и что длина ее в полтора раза больше ширины. С ним согласен (в этом) и
перипатетик Дикеарх».

Таким образом,  то слово  (тимпанум), которое С.Я.Лурье  переводит как
«барабан»,  А.О.Маковельский в  сборнике переводит  как «диск».  Какой
перевод  правилен?  Слово  «тимпан»  существует и  в  латинском  языке
(взято, конечно,  из греческого),  оно сохранилось  и в  славянском, и
вряд  ли тогда  означало «барабан».  Возьмем латинско-русский  словарь
И.X.Дворецкого и  Д.Н.Королькова 1949 г., там  «тимпанум» переводится:
тимпан, тамбурин, бубен  (употреблялся преимущественно на празднествах
в честь Кибелы  и Вакха), подъемное колесо, дисковое  колесо без спиц.
Существовал ли настоящий  барабан в античном мире,  мне неизвестно, но
полагаю,  что  на  празднествах  в  честь Кибелы  и  Вакха  плясали  с
дискообразными  бубнами,  а не  с  шарообразными  (если только  вообще
когда-либо существовали почти шарообразные барабаны) барабанами.

Различие в  толковании слово  «тимпанум» имеет огромное  значение, так
как С.Я.Лурье  во многих  местах использует свое  понимание Демокрита,
именно что, по мнению Демокрита, Земля  хотя и не была шарообразна, но
почти шарообразна. При  обычном же понимании выходит,  что Демокрит не
сделал  ни  шагу в  принятии  шарообразной  формы  Земли и  отошел  от
Анаксагора только в том смысле,  что вместо плоской Земли принял почти
плоскую Землю с впадиной,  заселенной людьми. Что заставляло Демокрита
принять впадину?  По мнению С.Я.Лурье,  это стоит  в связи с  тем, что
Демокрит принимал вращение Земли вокруг  оси (1947, с. 208): этого нам
придется  коснуться дальше.  Вторым доводом  является то,  что уже  во
времена  Демокрита восход  и заход  звезд происходил  в разных  местах
Земли  в разное  время (Лурье,  1947,  с. 209).  По мнению  С.Я.Лурье,
при  предположении, что  населенная часть  земли вогнута,  эти явления
становятся  вполне понятными  и  этот  аргумент использован  Архелаем,
находившимся,  по-видимому, в  ряду  вопросов  под влиянием  атомизма.
Но  мы  знаем, что  разница  во  времени  восхода  и захода  Солнце  и
звезд правильно использована сторонниками  выпуклости, а не вогнутости
Земли,  и в  примечании на  с. 370  сам Лурье,  ссылаясь на  Ф.Франка,
указывает, что более обширные  наблюдения, конечно, опровергли бы идею
о  вогнутости  Земли. Думается,  поэтому,  что  более прав,  например,
Веселовский  (1961,  с. 15--16),  который  представление  о Земле  как
впадине у Демокрита,  Архелая (ученик Анаксагора и  учитель Сократа) и
даже Платона  (об этом  уже была  речь в  §21) считает  воспринятым по
традиции от  египтян, которые считали  Землю немного больше  их Египта
(см. §3):  впадина ---  это Египет с  окружающими горами;  впадина ---
продолговата,  как это  принимал и  Демокрит. Оригинального  тут очень
немного. Кстати, Веселовский ссылается на то же уже цитированное место
Аэция.

\subsection{Учение об антиподах проповедовалось еще пифагорейцами, сам
термин введен, видимо, Платоном}

В пользу такого толкования говорит  и уже приведенный довод, а именно,
что Демокрит  не представлял себе  огромности размеров Земли,  а также
его своеобразная теория землетрясений (Матер. Древней Греции, с. 117):
«162.  Аристотель Метеор.  Д. 7  365а1. Демокрит  говорит, что  Земля,
будучи полна воды и, сверх того,  принимая в себя много дождевой воды,
по этой  причине приходит  в движение. А  именно, когда  дождевой воды
становится  слишком  много  вследствие  того, что  полые  места  не  в
состоянии  (уже)  вмещать  ее  в себе,  она  вынуждаемая  (создавшимся
положением),  производит землетрясение,  а  также  когда Земля  начнет
высыхать и образуется влечение (масс материи) к пустым местам из более
полных,  то (эта)  происходящая перемена  приводит Землю  в движение».
Аналогичные разнообразные объяснения землетрясений приводятся и дальше
со ссылкой на Сенеку.

Мне  кажется, отсюда  ясно, что  Демокрит принимал  Землю сравнительно
небольшой  (немногим  больше  известной  части,  ойкумены)  и  имеющей
вид  диска со  впадиной.  Но,  по Лурье,  Демокрит  принимал и  вторую
противоположную  впадину, как  и  нарисовано на  рис.  14. Мало  того,
С.  Я.   Лурье  считает  возможным   утверждать,  что  и   это  второе
вогнутое  днище  Демокрит  считал  населенным  (1947,  с.  209),  т.е.
признавал существование антиподов. Что же касается сферической боковой
поверхности  изобретенного  Лурье  «барабана», то  она  «была,  скорее
всего, необитаема за недостатком воздуха» (Лурье, там же, с. 371). Так
как это место чрезвычайно важно для связи учения Демокрита и Платона и
так  как  оно  написано  далеко  не вполне  ясно,  то  я  приведу  его
полностью, как и примечания, только обозначая греческие слова русскими
буквами. Вот это обоснование того, что Демокрит принимал существование
антиподов и что этот термин у него заимствовал Платон (Лурье, 1947, с.
209 и 371):

«Комментатор  Аристотеля  Александр  Полигистор  (у  Диогена  Лаэрция)
говорил (Диоген  VIII, 26),  что учение об  обитателях противоположной
стороны земли проповедовалось уже  древними пифагорейцами. Но заметим,
что с  пифагорейской точки зрения земля  шарообразна; поэтому говорить
о  ``противоногих'', об  антиподах  (антиподес),  живущих на  стороне,
противоположной населенной земле (ойкумене) было бы непоследовательно;
как замечает Александр Полигистор, они называли землю ``периойкомене''
--- ``населенной  вокруг''; жители других  частей земли, с  этой точки
зрения,  должны называться  ``живущими вокруг''  (периойкой...), а  не
``антиподами''.  Точно  так  же  и  Платон  в  ``Тимее''  (63  А,  ср.
Арист.  О небе  IV,  1, 308а  17  с комм.  Симпл.  679, 1)  протестует
против  употребления ``противоположных  названий'' (ономата  энанциа),
каковы ``верх и  низ'', ``антиподы'' и т.д.,  ибо ввиду шарообразности
земли  всякий человек  в любой  точке будет  по сравнению  с самим  же
собой, находящимся в противоположной точке, одновременно и находящимся
вверху  и находящимся  внизу и  все  время будет  сам себе  антиподом.
Отсюда совершенно ясно,  что термин ``антипод'' был  очень популярен в
эпоху  Платона и  заимствован  им  из философии,  в  которой Земля  не
имела  (курсив Лурье.  --- \emph{А.Л.})  шарообразной формы,  и людям,
живущим на одной из поверхностей, ``противостояли ногами'' люди другой
поверхности  ---  ``антиподы''. Можно  ли  сомневаться,  что одним  из
сторонников  этого  учения  был  и Демокрит,  для  которого  оно  было
логически необходимо, хотя бы ввиду  отрицания верха и низа и принципа
``исономии''? прим. 19.  Указание Диог. 111, 14  ``протос ен философия
антиполас  ономасе'', в  лучшем случае  означает, что  Платон придумал
название  для  антиподов:  оно  содержится в  чрезвычайно  раздутом  и
неправдоподобном перечне заслуг Платона».

Разберем  подробно  всю  эту   выдержку:  она  заслуживает  этого.  Из
последнего примечания  ясно: по мнению Диогена  Лаэрция, Платон первый
ввел в философию  название антиподов, и этого не  оспаривает Лурье, но
он считает, что это название  прилично не для сторонников шарообразной
формы  Земли, а  для  противников этого  учения,  для сторонников  той
формы  Земли, которую  принимал  Демокрит, по  мнению  Лурье. Но  ведь
форма  Земли, нарисованная  (как принимаемая  Демокритом) Лурье,  тоже
шарообразна,  но  лишь  с  чашеобразными  выемками  на  двух  полюсах,
следовательно,  здесь  термин  «антиподы»,   по  мнению  Лурье,  более
приличен  только  потому,  что  за  ненаселенностью  из-за  отсутствия
воздуха  боковой поверхности  «барабана»  человек из  ойкумены не  мог
попасть  на  противоположную часть  Земли,  а  во вполне  шарообразной
Земле это  было сделать  возможно: теоретически  во времена  Платона и
практически после кругосветных путешествий.

\subsection{Разбор мнения  С. Я.  Лурье о  том, что  Демокрит принимал
существование антиподов}

Но   позволительно  задать   себе  вопрос.   Если  термин   «антиподы»
несовместим с учением  о шарообразной форме Земли,  почему этот термин
и  сейчас  в  употреблении? Малоупотребителен,  но  существует  термин
«периэки»,  «окольножители» (см.  Энц,  сл. Брокгауза  и  Ефрона т.  1
А  XXIII,  статья «Антиподы  и  периойки  или  периэки»). В  учении  о
Земле  периэками называют  обитателей  одного и  того  же полушария  и
одной и  той же  географической широты,  но разъединенных  между собой
на  180  градусов.  Но  в  то  время  как  термин  «антиподы»  введен,
по-видимому, Платоном,  термин «периэки» имел в  те времена совершенно
другой  смысл: это  были представители  первичного населения  Лаконии,
оттесненные дорянами  к окраинам, окружавшие как  бы кольцом дорийские
поселения. Принимая термины «антиподы» (или «антэки», противожители) и
«периэки» широко, мы можем сказать, что, положим, для жителей северных
умеренных широт  периэками являются  жители тропиков,  а «антиподами»,
или  антэками, жители  южных  умеренных широт.  Тогда  выходит, что  с
представлениями  Платона были  совместимы и  антиподы и  периэки, а  с
представлениями  Демокрита  (как  их  понимает  С.  Я.  Лурье)  только
антиподы, так как боковая поверхность  Земли по его представлению была
необитаема.

По-видимому,  несомненно,  что   первым  предложил  термин  «антиподы»
Платон.  Это ясно  и  из цитированного  высказывания Диогена  Лаэрция,
а  также  из  того, что  ни  у  Демокрита,  ни  у кого-либо  из  более
древних авторов  этого термина как  будто не обнаружено.  Такой знаток
Демокрита,  как Лурье,  очевидно,  не обнаружил  этого  слова во  всех
фрагментах  Демокрита.  Если  бы  термин «антиподы»  был  популярен  в
эпоху Платона,  его бы,  наверное, чаще употребляли.  Однако в  той же
статье об  антиподах (в Энц.  слов. Брокг.  и Ефрона) мы  читаем такие
слова: «Шарообразный вид  Земли очень скоро привел  к представлению об
антиподах, и их учение признавалось уже философами до Цицерона, именно
стоиками. Но отцы церкви нашли в этом учении противоречие с Библией, и
в VIII  ст. дело дошло  до того,  что подвергался отлучению  от церкви
всякий, кто  открыто разделял учение  об антиподах. Только  после того
как кругосветные плавания доказали несомненность учения о шарообразной
форме  Земли,  существование  антиподов получило  право  гражданства».
Автор  этой  статьи,  появившейся  в  1890 г.,  видимо  не  знал,  что
шарообразность  Земли  и  существование антиподов  признавалось  и  до
стоиков, так как  основоположник стоицизма Зенон родился  около 336 г.
до н.э., т.е. через 11 лет после смерти Платона.

Совершенно непонятно,  где Лурье  нашел, что Платон  протестует против
употребления термина «антиподы» (когда он сам его ввел). Он протестует
лишь  против   абсолютизации  понятий  «верх»  и   «низ»  (само  собой
разумеется, что в повседневном разговоре он эти термины употреблял), а
из  выдержки, приведенной  в §21,  ясно,  что он  считал антиподов  не
какими-то  принципиально отличными  существами, а  что проходя  кругом
по  земле,  он  будет  становиться  сам себе  антиподом  и  той  точке
противоположной поверхности  земли, которую  мы сейчас  считаем низом,
он  будет  давать название  верха.  А  отсюда пропадает  и  какая-либо
убедительность  предположения,   что  понятие  антиподов   Платон  мог
заимствовать от Демокрита.

В пользу того, что Демокрит считал Землю плоской и заселенной только с
одной  стороны, говорит  и  прямое свидетельство  Аристотеля (О  небе,
11,  13 294,  в. 19),  приводимое  Лурье (1935,  с. 170):  «Анаксимен,
Анаксагор и Демокрит говорят, что  причина того, что Земля остается на
месте,  --- ее  плоская  форма.  В самом  деле,  она  не рассекает,  а
накрывает, как крышкой, находящийся  под ней воздух, что, по-видимому,
бывает с  плоскими телами».  Лурье соглашается с  принадлежностью этой
теории  Анаксимену  и  Анаксагору,  но  сомневается  в  справедливости
приписывания ее  Демокриту. Основанием к такому  сомнению являются: 1)
слова  Симпликия, который  прибавляет  слово «кажется»,  но это  слово
относится ко всем трем философам, а не только к Демокриту; 2) указание
Аэция  (Лурье,  с.  171),  противопоставлявшего  Демокрита  Анаксимену
(опускаю в скобках греческие  термины): «Демокрит (говорит, что земля)
недвижна вследствие того, что отстоя со всех сторон одинаково от неба,
находится  в  состоянии  равновесия (...)  ввиду  отсутствия  причины,
почему бы она  устремилась скорее в одну, чем в  другую сторону (...).
Анакеимен говорит, что она  держится на поверхности воздуха вследствие
плоской  формы».  Из  этой  цитаты  Лурье  (там  же,  с.  172)  делает
заключение, что Платон  в «Федоне» (в тексте  Лурье неправильно указан
«Федр»)  99  в.,  полемизируя  со взглядами  Эмпедокла  и  Анаксимена,
использует (не  указывая имени Демокрита) мнение  своего противника. В
пользу этого  Лурье приводит  филологический аргумент, именно  то, что
Платон  в  «Федоне»  употребляет  термин  «исороппия»  (равновесие)  и
«маллон», которые в связи  со взглядами Демокрита употребляются Аэцием
и Диогеном Лаэрцием.  Мне кажется, все эти доводы Лурье  не имеют силы
по  следующим  основаниям:  1)  греческие  термины  могли  применяться
независимо двумя авторами;  2) ссылка Аэция касается  только вопроса о
подвижности или  неподвижности Земли, а  не формы Земли  и возможности
антиподов;  3)  наконец,  что,  пожалуй, самое  важное,  Лурье  слегка
сокращает выдержку  из Аэция  (111, 15, 7),  приведя же  ее полностью,
мы  увидим,  что  аргументация  Лурье теряет  свою  силу.  В  переводе
Маковельского это место выглядит так  (Матер. Древней Греции, 1955, с.
116):  «Парменид, Демокрит:  Земля пребывает  в равновесии  вследствие
равного расстояния отовсюду, ибо нет  причины, которая заставила бы ее
скорее наклониться в одну сторону,  чем в другую. Вследствие этого она
может лишь сотрясаться,  но не двигаться». Это  мнение, таким образом,
оказывается не  только мнением Демокрита,  но и Парменида,  а Парменид
жил раньше  Демокрита (его «акмэ» относится  к 500 г. до  н.э., раньше
на  40  лет  до  рождения Демокрита).  Парменид  был  хорошо  известен
Платону (один  из знаменитых диалогов Платона  посвящен этому славному
мыслителю),  влияние пифагоризма  на Парменида  несомненно, и  поэтому
если  Платон  и  заимствовал  свои воззрения  от  кого-либо,  то  тут,
очевидно, мы должны принимать Парменида, а не Демокрита.

Наконец, когда мы дойдем  до разбора взглядов последнего представителя
в  античном мире  «линии  Демокрита», Лукреция,  мы увидим,  насколько
примитивны были космологические взгляды античных материалистов.

\subsection{Неверно  мнение,  что  Демокрит  принимал  вращение  Земли
вокруг оси}

Но Лурье не только старается  доказать, что форма Земли, по Демокриту,
была «почти шарообразной», но стремится внушить читателю, что Демокрит
был сторонником вращения Земли вокруг  оси. Это утверждение Лурье тоже
заслуживает внимательного рассмотрения. Вот оно (Лурье, 1947, с. 208):
«Теперь  примем во  внимание,  что, по  свидетельству Диогена  Лаэрция
(правда, обычно перетолковываемому, так как оно противоречит некоторым
другим свидетельствам),  Левкипп (Демокрит)  учил, что земля,  придя в
центр,  не стоит  неподвижно,  а ``висит  в  воздухе, кружаясь  вокруг
центра''» (Диоген,  IX, 30). «Засвидетельствовано, что,  по Демокриту,
массы,   образующие  землю,   первоначально  вращались   на  небольшом
расстоянии от центра (как это  наблюдается в водоворотах) и лишь затем
пришли в центр (Аэций, 111, 13,  4). Полагали, что и у Диогена Лаэрция
имеется в виду то  время, когда земля еще не пришла  в центр; но общая
связь  мыслей у  Диогена  показывает,  что речь  может  идти только  о
настоящем времени. Очевидно, Левкипп-Демокрит говорит здесь о вращении
земли вокруг  оси. Это подтверждается учением  пифагорейца Экфанта (из
Дильса). Этот никому не известный  Экфант из Сиракуз в общем развивает
специфические основные  положения теории Демокрита: он  учит об атомах
в  пустоте,  причем  атомы   считает  телесными,  материальными».  (Он
полемизирует с Демокритом только в вопросах, касающихся пифагорейского
символа веры: «Тела, по его  учению, приводятся в движение не тяжестью
и не ``толчком'' (полемика: ... --- демокритовский термин1 --- С. Л.),
а божественной  силой, которую он  называет разумом и душой»  (разум и
душа  ---  синонимы, как  у  Демокрита!);  «мир возник  из  соединения
атомов, но управляется провидением»).

«В  этом причесанном  под  идеалистическую  гребенку учении  Демокрита
содержится, между прочим, и учение  о вращении Земли вокруг оси. Ввиду
свидетельства Диогена Лаэрция я считаю более чем вероятным, что теория
вращения  Земли  вокруг  оси  была  впервые  высказана  не  никому  не
известным Экфантом, а  Демокритом, у которого ее  заимствовал вместе с
другими атомистическими положениями Экфант».

«Этот длинный экскурс дает читателю  понять, что если наши предпосылки
верны,  то   та  форма,   которую  придал  земле   Демокрит,  является
единственно  правильной  с точки  зрения  его  учения о  ``вращающей''
силе:  точки земной  поверхности, близкие  к земной  оси, ``вследствие
неподвижности''  испытали   притяжение  к   центру  (земля   была  еще
полужидкой)  и  приблизились  к  центру,  вследствие  чего  получилась
вогнутость; в  точках, далеких от  оси, ``вращающая сила''  победила и
уничтожила  силу  притяжения  к  центру, и  здесь  шарообразная  земля
сохранила свою первоначальную форму (рис. 14)». Так пишет С.Я.Лурье.

Если  верить Лурье,  то Демокрит  не только  высказал идею  о вращении
Земли,  но был  провозвестником  будущего учения  Ньютона  о том,  что
вращение Земли  определяет ее  форму. Правда, по  современным взглядам
должен  получиться сплюснутый  эллипсоид вращения,  а не  шар с  двумя
чашеобразными вогнутостями  у полюсов,  но нельзя  предъявлять слишком
больших  требований к  античности. Но  способ доказательства  Лурье не
может не  вызвать удивления.  Есть указание,  что гипотезу  о вращении
Земли высказал  пифагореец Экфант из  Сиракуз. Но так как  этот Экфант
придерживается  также атомистических  воззрений, то,  значит, он  свое
учение (помимо атомизма)  заимствовал от Демокрита. Далеко  мы уедем с
такими методами  доказательства! Но  отметим: 1)  атомизм не  был чужд
пифагореизму  вообще,  и дискуссионным  является  вопрос  не только  о
монополии Демокрита в области атомизма,  но даже о приоритете; 2) само
собой разумеется,  что сходство  двух мыслителей в  одном не  дает нам
право  принимать сходство  и в  остальном, тем  более что  пифагореизм
Экфанта никем не оспаривается, а  вращение и даже шарообразность Земли
с атомизмом  совсем не связаны;  3) античные авторы указывают  в числе
основоположников теории  вращения Земли кроме Экфанта  также Филолая и
Гикета (см. эту работу §7),  но, быстро разделавшись с Экфантом, Лурье
об остальных не считал даже нужным  упомянуть; но мы знаем, что учение
о Солнечной системе  получило на линии Пифагора  --- Платона блестящее
развитие, о  чем достаточно написано  выше, но линия же  Демокрита ---
никакого, как увидим дальше; 4) и, наконец, видимо, филологи не пришли
к соглашению  о переводе фрагмента  из Аэция 111,  13, 4. Вот  как дан
этот  перевод в  «Материалистах Древней  Греции» (с.  116): «Демокрит:
вначале Земля блуждала вследствие своей малости и легкости; с течением
же времени,  сделавшись плотнее  и тяжелее,  она пришла  в неподвижное
состояние». «Блуждала»,  а не  «кружилась», пришла не  в «центр»,  а в
«неподвижное состояние».

\subsection{Утверждение  о критике  Платоном учения  о множественности
миров основано на неправильном понимании}

Можно   прийти  к   выводу,  что   все  попытки   показать  какое-либо
заимствование пифагорейской  школой чего-либо от  Демокрита совершенно
необоснованны.   Все  развитие   гелиоцентризма   не  касается   линии
Демокрита.  В  «Истории философии»  (1941,  с.  102) делается  попытка
показать,  что  учение  Демокрита   о  бесконечности  миров  пробивало
брешь  в разделяемой  самим  Демокритом  геоцентрической точке  зрения
и  подготовляло  почву  для возникновения  гелиоцентрической  системы.
Но,  несомненно, мыслимо  многообразие  миров  при отсутствии  всякого
порядка  в  каждом  мире  (на  с.  101  «Истории  философии»  указано,
что бесконечность  миров Демокрит обосновывает  бесконечностью причин,
т.е.  атомов) и  принятие конечности  Вселенной не  помешало Копернику
построить гелиоцентрическую систему.

Считается,  что учение  Демокрита  о  бесконечности миров  подверглось
резкой  критике Платона  (Матер.  Древней Греции,  стр. 113):  «Платон
``Тимей'' 55  Д (против Демокрита). Принятие  бесконечного числа миров
есть  мнение  подлинно  безграничного  невежества»  ---  одно  из  тех
мест,  которое приводится  сторонниками Демокрита  в качестве  примера
догматизма Платона и  его грубости по отношению  к своему закоренелому
врагу,  которого  он, следуя  своей  привычке,  не называет.  Незнание
греческого языка не позволяет мне сверить этот текст непосредственно с
оригиналом, но  применим тот  же метод  сравнения разных  переводов. В
моем распоряжении  современный французский перевод Робена  (1950). Это
место  следует непосредственно  за  изложением знаменитых  «Платоновых
тел», вдохновивших  Кеплера к его исканию  расстояний между планетами.
Попробую  перевести: «Если,  подсчитав  все это,  мы поставим  вопрос,
следует  ли  принимать  миры  в бесконечном  или  конечном  числе,  то
можно  заключить,  что  совсем  не проницательно  принимать  их  число
бесконечным; это учение тех, кто не понимает, что следует понимать под
проницательностью (истинным знанием). Но сколько миров, один или пять,
произвела природа?  Как надо утверждать?  Касаясь числа, у  нас больше
оснований  для неуверенности.  Что  касается нас,  мир один,  согласно
нашему вероятному  рассуждению. Так следует согласно  природе, так нам
указывает Бог;  но другой  исследователь после иных  соображений может
прийти к иному мнению».

Когда  мы  берем  весь  абзац, то  оказывается,  Платон  не  настаивал
догматически на единственном мире.  Исчезает и грубость, как известно,
вовсе не свойственная Платону. Французский переводчик Робен указывает,
что  в  этом  тексте  заключена   игра  слов  (прием,  которым  широко
пользуется  Платон)  именно  на  двойном смысле  слова  «апейрос»  ---
бесконечный  и  несведущий.  Французский  перевод  не  «фрагментирует»
Платона, как  это делает  Маковельский, и сохраняет  ту художественную
форму, которая  не отрицается  у Платона даже  его врагами.  Но почему
Платон  утверждает,  что  мир  един?   Об  этом  говорится  в  том  же
«Тимее»  (30,  31),  и  здесь  речь идет  не  о  единственности,  а  о
единстве Вселенной.  Для Демокрита Вселенная есть  собрание совершенно
независимых друг от друга миров, для Платона вся Вселенная живая и как
живой  организм  она  едина  и  имеет  закономерное  строение.  Всякий
биолог,  поймет, в  чем  разница. Сложный  живой  организм состоит  из
клеток, и каждая клетка есть  живой организм, способный в определенных
условиях  (культура  тканей)   к  самостоятельному  существованию,  но
это  не  уничтожает  единства   целого  организма.  Поэтому  концепция
Платона  не  закрывает дороги  к  суждению  о числе  миров  (по-новому
---  систем,  Галактик,  сверхгалактик), она  только  утверждает,  что
все  эти подчиненные  миры составляют  единое целое.  Когда дойдем  до
Коперника и Кеплера (которые оба прекрасно знали Платона в оригинале),
я  думаю,  мне   удастся  показать,  что  мое   понимание,  невежды  в
греческом языке, ближе к пониманию творцов современной астрономии, чем
понимание знатоков греческого  языка, ослепленных материалистическим и
антирелигиозным фанатизмом.

\subsection{Значение Лукреция Кара для реконструкции взглядов античных
материалистов}

Защитники огромной  роли Демокрита  и материалистов вообще  в античной
культуре  широко используют  «убежище невежества»  для своей  цели. От
Демокрита  и  Эпикура остались  только  фрагменты  и, конечно,  трудно
доказать, что в исчезнувшем  наследстве Демокрита не содержалось идей,
использованных его противниками. Но,  к счастью, из материалистической
литературы  Древнего мира  сохранилось полностью  сочинение эпикурейца
римской эпохи  Лукреция Кара  «О природе вещей»,  которое неоднократно
переводилось  на русский  язык  с  многочисленными комментариями.  Кар
считается продолжателем атомизма Демокрита и Эпикура и потому сейчас в
большом почете.

Лукреций  жил  гораздо позже  Платона,  Аристотеля,  Аристарха, жил  в
первом столетии до н.э. (примерно  99--55 гг.), т.е. был современником
Цицерона и Юлия Цезаря. Следуя обычаю древних, он мало упоминает имена
философов,  даже  тех,  кому  он  близок  по  мировоззрению.  Эпикура,
которому он посвящает восторженное вступление к третьей книге (ст. 14:
«Ибо лишь только  твое, из божественной мысли  возникнув, стало учение
нам о  природе вещей проповедать»)  и к  пятой книге, он  упоминает по
имени только раз, кн. III, 1042--1044:

\begin{verse}

\emph{Сам   Эпикур  отошел   по   свершении   поприща  жизни,\\*   Он,
превзошедший людей дарованьем своим и затмивший\\* Всех, как и звезды,
всходя, затмевает эфирное солнце.}

\end{verse}

Имя Демокрита упоминается  три раза (кн. III, 370, 1039  и кн. V, 623)
два раза с эпитетом «Демокрита  священное мненье», которое он, однако,
в одном случае оспаривает. Кроме этих  из ранних для него философов он
отзывается довольно одобрительно об  Эмпедокле (кн. I, 716), упоминает
Анаксагора  (кн.  I, 830  и  876,  критикует учение  о  «гомеомериях»)
и,  наконец,  Гераклита  Эфесского, о  котором  отзывается  совершенно
неодобрительно (кн. I, 635--641):

\begin{verse}

\emph{Вследствие этого те,  кто считал, что все  вещи возникли\\* Лишь
из  огня,  и  огонь  полагали основою  мира.\\*  Кажется  мне,  далеко
уклонились  от  здравого  смысла.\\* Их  предводителем  был  Гераклит,
завязавший  сраженье,\\*  По темноте  языка  знаменитый  у греков,  но
больше\\*  Слава его  у  пустых, чем  у  строгих искателей  правды,\\*
Ибо  дивятся  глупцы и  встречают  со  строгим почтеньем\\*  Все,  что
находят они в изреченьях  запутанных скрытым;\\* Истинным то признают,
что  приятно ласкает  им ухо,\\*  То,  что красивых  речей и  созвучий
прикрашено блеском.}

\end{verse}

Прочих  философов Лукреций  вообще  не называет,  хотя, как  указывает
Петровский  (с.  11), в  его  поэме  есть  намеки на  учения  Кратила,
Анаксимена,  Диогена  Аполлонийского,  Фалеса, может  быть,  Ферекида,
Ксенофана  и Энопида  Хиосского.  Постоянно  полемизируя со  стоиками,
Лукреций не  только не называет  их по имени,  но даже не  называет их
школы. В  книге третьей,  100 и  далее, он говорит  о том,  «что греки
зовут  гармонией», и  говорит  (там же  132):  «Слово ``гармония''  ты
отвергни»,  этим самым  показывая  знакомство  с учением,  развиваемым
в  диалоге  Платона  «Федон»   Симмием,  учеником  Филолая  (с.  252).
Ясно,  таким образом,  что тот  упрек,  который был  брошен Платону  в
замалчивании имени своего противника  (знакомство Платона с Демокритом
отнюдь не  доказано), в гораздо  большей степени может быть  отнесен к
Лукрецию.  А  если  это  был  просто  такой  обычай  в  древности,  то
никто упрека  не заслуживает.  Но ясно также,  что Лукреций  не только
мог  быть  знаком  с творениями  великих  философов-идеалистов,  своих
предшественников и  противников, но и  был знаком с  этими творениями.
Как же отнесся он  к астрономическим воззрениям своих предшественников
и современников?

\subsection{Лукреций  имел примитивнейшие  сведения  о  форме Земли  и
отрицал существование антиподов}

Во всей поэме рассеяны замечания по космологии и космогонии, но только
в редких случаях  можно вывести из этих  замечаний ясные представления
Лукреция.

Трудно понять, какое представление имел  Лукреций о форме Земли. Можно
только сказать,  как это подчеркивает  и Петровский (с. 259),  что оно
было  «несовершенно».  Но  что касается  антиподов,  то  отрицательное
мнение  о   возможности  их   существования  выражено   с  совершенной
отчетливостью --- кн. I, 1052--1071:

Тут одного  берегись и не  верь утверждению, Меммий,  Что устремляется
все  к  какому-то  центру  Вселенной, Будто  поэтому  мир  и  способен
держаться без  всяких Внешних толчков;  и никак никуда  разложиться не
может  Верх  и низ  у  него,  ибо  все  устремляется к  центру  (Если,
по-твоему, вещь на себя опираться способна), Что, находясь под землей,
стремятся к ней  тяжести снизу И пребывают на  ней, обернувшись кверху
ногами, Как отраженья,  что мы на поверхности вод  наблюдаем; Будто бы
вниз головой и животные также под  нами Бродят, и будто с земли упасть
им никак невозможно В нижние своды  небес, как и наши тела не способны
Сами собой улететь  к высоким обителям неба; Будто бы  солнце у них, в
то время как ночи светила Мы созерцаем; что мы взаимно меняемся с ними
Сменой времен,  а их дни  ночам соответствуют нашим Но  лишь надменным
глупцам допустимо  доказывать это,  Ум у  которых всегда  к извращению
истины склонен.  Центра ведь  нет нигде у  Вселенной, раз  ей никакого
Нету конца.

Этой цитаты, я  думаю, совершенно было бы  достаточно для опровержения
мнения Лурье о том, что Демокрит является автором учения об антиподах,
что  я  разбирал в  предыдущих  параграфах.  Не мог  Лукреций  назвать
«надменным глупцом» человека,  которого он глубоко уважал и  не мог не
знать.

Но эта цитата интересна и  в другом отношении. Существование антиподов
Лукреций отрицает на основании  принимаемого им учения о бесконечности
Вселенной, которое  полнее дано в  книге первой, 958--1051.  По мнению
Петровского (с.  16), эта  идея, заимствованная Лукрецием  у Демокрита
и  Эпикура,  представляет собой  крупный  шаг  вперед сравнительно  со
взглядами  Пифагора, Платона  и Аристотеля,  которые допускали  только
нашу космическую систему: Землю,  Солнце, планеты, неподвижные звезды.
В связи с представлением  о бесконечной громадности Вселенной Лукреций
говорит о ничтожности человека (VI, 649, сл.).

Как  было  показано  в  §39,   надо  различать  понятия  «единства»  и
«единственности»,   а   также  «бесконечность»   и   «безграничность».
Современные астрономы склонны думать,  что Вселенная «безгранична», но
не  «бесконечна»  (об  этом  будет  речь  в  разделе  после  Ньютона),
и  все  аргументы  Лукреция  против конечности  теряют  силу.  Что  же
касается «единства» Вселенной, то  оно тоже возрождается в современной
астрономии. Странными кажутся слова  Петровского, что Пифагор и Платон
допускали  только нашу  космическую систему:  Землю, Солнце,  планеты,
неподвижные  звезды.  А что  еще  можно  кроме этого  придумать?  Спор
идет  не о  том,  существует  ли что-то  сверх  этого,  а является  ли
количество образований  во Вселенной  разных рангов  (звезды, системы,
подобные Солнечной,  галактики и сверхгалактики и  проч.) конечным или
бесконечным,  представляют ли  они  хаос  независимых образований  или
образуют какую-то закономерную структуру.

Но  «прогрессивная»,  по  мнению  Петровского,  идея  о  бесконечности
Вселенной сыграла отрицательную роль в вопросе об антиподах, отрицание
же  центра Вселенной  не  помешало полной  путанице  идей в  отношении
«верха» и «низа».

\subsection{Осознание относительности движения  совмещается у Лукреция
с твердой верой в неподвижность Земли}

Приведу два места. Кн. I, 984--997:

Кроме того, если все необъятной вселенной пространство Замкнуто было б
кругом и, имея предельные грани, Выло б конечным, давно уж материя вся
под давлением  Плотных начал основных отовсюду  осела бы в кучу,  И не
могло бы ничто  под покровом небес созидаться; Не было  б самых небес,
да и солнца лучи не светили б,  Так как материя вся, оседая все ниже и
ниже От  бесконечных времен,  лежала бы  сбившейся в  кучу В  самом же
деле, телам  начал основных совершенно  Нету покоя нигде,  ибо низа-то
нет  никакого. Где  бы, стеченье  свое прекратив,  они оседали.  Все в
постоянном, движении всегда созидаются  вещи, Всюду, со всяких сторон,
и  нижние  с  верхними  вместе  Из  бесконечных  глубин  несутся  тела
основные.

Здесь отсутствие «низа» понимается  в смысле отсутствия нижней границы
Вселенной, а  не в  смысле отрицания абсолютного  значения направлений
---  «вверх»  и  «вниз»  (иначе пришлось  бы  допустить  существование
антиподов, что делают только «надменные глупцы»).

Но в кн. V, 449--454 читаем:

Стали сначала земли тела все отдельные купно, ---

Так как они тяжелы и сцеплены крепко, --- сходиться,

Все в середине и в самом низу  места занимая. Чем они больше и больше,
сцепляясь друг с  другом, сходились, Тем и сильней  выжимали тела, что
моря образуют, Звезды, солнце, луну и стены великого мира...»

Там же, 534--542:

Далее,  чтобы земля  в середине  покоилась мира,  Мало-помалу легчать,
уменьшался в собственном  весе, Следует ей и иметь  естество под собою
другое,  Испокон века  в одно  сплоченное целое  тесно С  мира частями
воздушными, где она в жизнь воплотилась

Вот почему  и не в  тягость земля  и не давит  на воздух: Так  же, как
члены его человеку любому не в тягость, Как голова не обуза для шеи, и
нам не заметно, Что опирается, всей своей тяжестью тело на ноги.

По мнению Петровского (с. 255),  это нисколько не противоречит ст. 993
и ст. 1052 и сл. кн. I  (которые были приведены выше), где говорится о
Вселенной в целом,  которая безгранична, и следовательно,  не имеет ни
центра,  ни дна;  здесь  же  Лукреций говорит  о  нашем мире,  который
ограничен Сводом эфира, заключающим этот  мир «в объятиях жадных» (ст.
470).  Но  если бы  Лукреций  всерьез  принимал ограниченность  нашего
мира, то  отпало бы  его отрицание  антиподов (см.  §41), так  как оно
целиком  построено на  отсутствии  ограниченности  нашего мира.  Можно
только сказать,  что бесконечная  Вселенная по  Лукрецию поляризована,
т.е.  имеются направления  --- верх  и  низ, но  и  в ту,  и в  другую
сторону  Вселенная простирается  бесконечно.  Но так  как  верх и  низ
принципиально  различны, то  из  строения нашей  стороны Земли  нельзя
делать заключения  о строении противоположной части,  что делают лица,
принимающие существование антиподов.

В  связи  с  принятием  бесконечности Вселенной  у  Лукреция  стоит  и
признание множественности обитаемых миров, кн. 2, ст. 1072--1076:

Если вещей  семена неизменно способна природа  Вместе повсюду сбивать,
собирая  их тем  же порядком,  Как  они сплочены  здесь, ---  остается
признать неизбежно, Что во Вселенной еще  и другие имеются земли, Да и
людей племена и также различные звери.

Никаких  намеков на  критику геоцентрической  системы у  Лукреция нет,
хотя,  казалось бы,  что он  уже подошел  к пониманию  относительности
движений, кн. IV, ст. 387--397:

Кажется  нам, что  корабль, на  котором плывем  мы, недвижен,  Тот же,
который стоит причаленный,  --- мимо проходит; Кажется,  будто к корме
убегают  холмы  и  долины,  Мимо  которых  идет  наш  корабль,  паруса
распустивши. Звезды кажутся нам укрепленными  в сводах эфирных. Но тем
не менее  все они движутся  без перерыва. Так  как восходят и  вновь к
отдаленному  мчатся  закату,  Путь  совершив в  небесах  и  пройдя  их
сверкающим телом Кажется нам, что и  солнце с луной остаются на месте,
Стоя спокойно, хотя и несутся они в самом деле.

Осознание относительности  движений совмещается  у Лукреция  с твердой
убежденностью в неподвижности Земли.

\subsection{Лукреций совершенно равнодушен к подлинно научным спорам о
размерах небесных тел, о фазах Луны, о затмениях}

Есть у Лукреция и космогонические  представления. Они все построены на
представлении о первичном хаосе, кн. V, ст. 422--436:

\begin{verse} Если  ж начала  вещей во множестве,  многоразлично\\* От
бесконечных времен  постоянным толчкам подвергаясь,\\*  Тяжестью также
своей гнетомые, носятся вечно,\\* Всячески между собой сочетаясь и все
испытуя,\\*  Что  только могут  они  породить  из своих  столкновений,
---\\* То и случается тут, что они в этом странствии вечном,\\* Всякие
виды  пройдя сочетаний  и разных  движений.\\* Сходятся  так, наконец,
что  взаимная их  совокупность\\* Часто  великих вещей  собой образует
зачатки:\\*  Моря, земли  и  небес, и  племени  тварей живущих.\\*  Ни
светоносного круга  высоколетящего солнца\\* Не различалось  тогда, ни
созвездий великого мира\\* И ни морей,  ни небес, ни земли, ни воздуха
так же,\\* Как ничего из вещей, схожих с нашими, не было видно.\\* Был
только  хаос  один и  какая-то  дикая  буря\\* Всякого  рода  начал...
\end{verse}

О начале  неба и земли Лукреций  говорит и в других  местах (например,
кн. V, 243--246).

Любопытны суждения Лукреция о величии Солнца, кн. V, ст. 569- 576:

Значит, коль свет и тепло,  изливаясь обильно из солнца Наших касаются
чувств и  пространства земли озаряют, Солнце  нам видно с земли  в его
настоящих  размерах, И  полагать, что  оно или  больше иль  меньше, не
должно.  Да и  луна,  --- все  равно, несется  ль  она, озаряя  Светом
присвоенным  все, или  собственный свет  излучает, ---  Что бы  там ни
было, --- все ж и она по  размерам не больше, Чем представляется нам и
является нашему взору.

Что  значит «настоящие  размеры»  непонятно, но  совершенно ясно,  что
Лукреций  даже понятия  не  имел  о тех  подлинно  научных работах  по
измерению  величин  Солнца  и  Луны,  которые  велись  преимущественно
александрийской  школой. Вместе  с тем  в этой  цитате у  него впервые
проскальзывает  то равнодушие  к  решению  подлинно научных  вопросов,
которые  так характерны  и  для его  единомышленника,  Эпикура: он  не
пытается  разобраться в  споре ---  заимствует  ли свой  свет Луна  от
Солнца (Фалес,  Пифагор, Эмпедокл,  Анаксагор) или  светит собственным
светом (Анаксимандр, Ксенофан,  см. с. 256), хотя  ко времени Лукреция
этот спор был уже практически решен в научных кругах.

Полное  равнодушие (или  как теперь  любят говорить  --- «всеядность»)
к  решению  конкретных  астрономических вопросов  Лукреций  проявляет,
например, в вопросе о фазах Луны, затмениях (кн. V, ст. 715--761):

Но допустимо и то, что, собственный  свет излучая, Катится в небе луна
и  дает изменения  блеска,  Ибо  возможно, что  с  ней вращается  тело
другое, Что  заступает ей путь, постоянно  от нас заслоняя; Нам  же не
видно оно, ибо  мчится лишенное света. Может вращаться луна  и как шар
или, если угодно,  мяч, в половине одной облитый  сияющим блеском, При
обращеньи  своем  являя  различные  фазы,  Вплоть  до  того,  как  она
откроется нашему взору Той стороною,  где вся сверкает пламенем ярким,
Мало-помалу  затем обращался  вспять и  скрывая Всю  светоносную часть
своего шаровидного тела. Так вавилонские нам указуют халдеи, сужденьем
Этим  низвергнуть стремясь  ученье  других  звездочетов. Будто  нельзя
допустить, что возможно и то, и другое, Иль что учение то нисколько не
хуже, чем это. Иль почему,  наконец, невозможно луне нарождаться Новой
всегда  и менять  свои  фазы  в известном  порядке  И ежедневно  опять
исчезать народившейся каждой,  Чтобы на место нее  и взамен появлялась
другая, --- Было  бы трудно найти основанье и  довод бесспорный: Может
ведь  многое вновь  нарождаться в  известном  порядке Вот  и Весна,  и
Венера идет, и Венеры крылатый Вестник грядет впереди...

Вновь цепенеет Зима, и стучат ее зубы от стужи

Что же мудреного в том, что в положенный срок зарождаться

Может луна, и опять в положенный срок исчезает,

Если в положенный срок появляются многие вещи?

И омраченья луны и солнца затмения также

Могут, как надо считать, совершаться по многим причинам,

Ибо коль может луна от земли загораживать солнцу

Свет и на небе главу возвышать между ним и землею,

Темный свой выставив диск навстречу лучам его жарким,

Разве нельзя допустить, что на то же способно иное

Тело, что может скользить, навеки лишенное света?

Иль почему же нельзя, чтоб теряло огни, угасая,

Солнце в положенный срок и сызнова свет возрождало

В тех областях проходя, где пламени воздух враждебен,

Где потухают огни и где они временно гибнут?

\subsection{Задача   Лукреция  не   научная,   а  опровержение   всего
нематериалистического,  почему  он  и  пользуется  таким  уважением  у
материалистов}

Полное равнодушие к исканию  истинной теории небесных явлений Лукреций
проявляет даже  там, где  ему известно определенное  мнение Демокрита,
именно кн. V, 614--649:

Также нельзя  привести простой и  единой причины Для  объяснения того,
как солнце  из летних  пределов До  поворота зимой  в Козероге  идет и
обратно Вновь  к остановке своей  подходит на тропике Рака,  Или того,
почему луна в один месяц проходит  Тем же путем круговым, что солнце в
год пробегает.  Этим явлениям дать простого  нельзя объясненья. Можно,
во-первых, считать,  что все  это так происходит,  Как полагает  о том
Демокрита  священное  мненье. То  есть,  чем  ближе к  земле  проходят
светила, тем меньше

Могут  они увлекаться  вращеньем небесного  вихря, Ибо  стремленье его
и  напор,  постепенно  слабея.   Книзу,  становится,  меньше;  поэтому
мало-помалу Солнце  всегда отстает  от движения звездного  круга. Ниже
гораздо идя. пылающих в небе созвездий, А еще больше луна,...

Но  допустимо и  то,  что  от полюсов  мира  противных Воздух  потоком
двойным в положенный  срок вытекает Может он  солнце теснить, прогоняя
от летних созвездий До самого поворота зимой и до стужи холодной, И от
холодных теней ледяных отгонять его  снова В жаркие лета края, обратно
к  созвездиям знойным.  Можно считать,  что  подобным путем  и луна  и
планеты, Долгие годы  свои обращаясь по долгим  орбитам, Воздуха током
двойным подвигаются  в неба пространствах.  Разве не видишь,  что так,
противным  гонимые  ветром Тучи  вверху  и  внизу идут  в  направленьи
противном?  Так почему  ж  по  кругам необъятным  эфира  не могут  Эти
светила нестись, противным гонимые вихрем?

Во всех  этих высказываниях  --- ни намека  на ту  астрономию, которая
развивалась в Афинах  и Александрии и уже привела  к выдающимся трудам
Аристарха и Гиппарха. Если бы  судить по научному уровню высказываний,
то  можно  было бы  утверждать,  что  Лукреций  жил примерно  в  эпоху
Гераклита.  Но, видимо,  хронология  установлена  достаточно точно,  и
мы  должны  признать Лукреция  даже  не  дилетантом, а  обскурантом  в
астрономии как науке,  так как он, видимо,  не то, что не  знает, а не
может и не хочет знать достижений астрономии своего времени. В истории
астрономии ему место  только как свидетелю защиты  от обвинений «линии
Платона»  в  «заимствовании»  от  Демокрита  и  других  материалистов.
Вероятно, Лукреций взял  далеко не все от Демокрита  и сильно упростил
его учение, но он не мог так исказить его, чтобы не осталось ничего от
тех идей, которые приписываются  Демокриту С.Я.Лурье. Совершенно ясно,
что космология Демокрита была на самом примитивном уровне.

Конечно, Лукреций, как и его предшественники, Демокрит и Эпикур, более
всего знамениты  своей атомной теорией  и об  этом речь будет  в своем
месте,  и  я  там  постараюсь  показать,  что  и  в  этой  области  их
научный  уровень  лишь немного  выше.  Но  почему  же его  так  высоко
ценят многие  ученые? Почему «следуя за  Фейербахом, Чернышевский ясно
понимал, что  он является ``твердым приверженцем  того строго научного
направления, первыми представителями которого были Левкипп, Демокрит и
т.д. до  Лукреция Кара'' (Чернышевский в  Сибири, 11, с. 26),  т.е. он
считал себя приверженцем  материалистического направления в философии»
(Комментарии Григорьяна, см. Чернышевский, 1938, с. 542).

Ответ  на  этот  вопрос  дает   Петровский  (1958,  с.  13).  Даваемые
Лукрецием два или  несколько объяснений для одного и  того же феномена
проявляют лишь «характерное равнодушие  к точному ответу на конкретные
научные  вопросы,  лишь  бы   только  избежать  признания  чего-нибудь
нематериалистического». Тут Лукреций следует  по стопам своего учителя
Эпикура: нет надобности определить  единственную и безусловную причину
того или другого явления:  важно лишь установить естественность данной
причины. Мне  думается, что  здесь имеет  место важное  различие между
Левкиппом  и Демокритом,  с  одной стороны,  Эпикуром  и Лукрецием,  с
другой. Первые философы стремились  найти истинное понимание явлений и
пришли  к  выводу,  что таковым  будет  материалистическое  понимание.
Лукреций же  (возможно, уже  и Эпикур) исходили  из материалистических
догматов  и,  не  особенно интересуясь  конкретной  наукой,  старались
придумать кое-какие «объяснения» для явлений, совершенно не заботясь о
том, пригодны ли  эти объяснения для точного  описания явлений природы
--- наиболее  важной задачи  науки. Тут  явно проявляется  та эволюция
представления  об антагонизме  науки  и религии,  о котором  постоянно
твердят материалисты (включая многих позитивистов). Обычное понимание:
независимо развивающаяся  наука приходит к  положениям, противоречащим
религиозным  догматам,   и  религия  в  самом   широком  смысле  слова
стремится  задержать развитие  науки. Несомненно,  можно привести  ряд
исторических фактов в пользу  такого положения. Но материалисты делают
из  этого общий  вывод:  религия принципиально  против науки,  поэтому
отрицание религии уже является  научным достижением. Отказ от религии,
принципиальный  атеизм не  только  необходим для  того, чтобы  назвать
определенное учение научным, но и вполне достаточен для этого.

\subsection{Связь  античных материалистов  с догматической  философией
сделала для них невозможным участие в развитии научной космологии}

Такое  резко отрицательное  отношение  отнюдь не  характерно для  всех
противников объективного идеализма.  Например, лидер позитивизма Огюст
Конт признавал, как известно, три последовательных состояния мысли или
периода развития человека: теологический, метафизический (философский)
и положительный. Отдав  должное роли теологии и  метафизики в прошлом,
мы теперь пришли к такому  состоянию, где «наука сама себе философия».
Эту  формулу   о  трех  состояниях  Чернышевский,   например,  считает
совершенно вздорной: «правда  тут лишь в том, что,  прежде чем удастся
построить гипотезу, сообразную с истиной, очень часто люди придумывают
гипотезы неудачные. Ошибки очень  часто предшествуют истине --- только
и всего. А теологического периода  науки никогда не бывало; метафизика
в  том смысле,  как  ее понимает  Огюст Конт,  тоже  вещь, никогда  не
существовавшая» (Чернышевский, 1938, 1931).

Но  если  это  так,  если   атеизм  =  материализму  =  научности,  то
всякое  атеистическое  произведение  тем самым  делается  научным  при
полном  отсутствии  каких-либо  подлинно  научных  заслуг,  и  никакие
научные  в истинном  смысле слова  заслуги  не спасут  от обвинения  в
«реакционности»,  «мракобесии», «обскурантизме»  и проч.,  если данный
выдающийся  ученый   оставляет  хоть  малейшую   лазейку  «поповщине».
Считается, что  вредное для развития науки  утверждение «философия ---
служанка  теологии»  было  четко  формулировано  в  Средние  века.  Не
берусь судить, когда  впервые было высказано это  положение, но вполне
симметричное ему утверждение: «наука ---  служанка атеизма», хотя и не
было формулировано, но вполне ясно сквозит во всей поэме Лукреция. Это
прямо ставится в  заслугу Лукрецию Петровским (с. 20):  «У природы нет
цели: она выполняет свою задачу, не заботясь о тех, кого она стирает с
лица земли». Лукреций отрицает чудо во имя законов природы, провидение
--- во имя несовместимости блаженства богов с заботами о человечестве,
акт  творения   мира  ---  во  имя   атомистической  теории,  согласно
которой атомы  извечно несутся  по бесконечному  пустому пространству,
сталкиваясь между собою и вступая  во всевозможные комбинации, одна из
которых дала существование тому миру, в котором мы живем. Что касается
верующих или деистов, вообще всех  тех, кто видит во Вселенной частицу
божества,  то они,  по мнению  Лукреция, «мечтатели  или сумасшедшие».
Понятно поэтому,  почему Лукреций  так нетерпим  в тех  случаях, когда
затрагивают его общефилософские материалистические представления, хотя
бы они были весьма далеки от  науки, и так равнодушен к разногласиям в
научной  области,  многие  из  которых  уже  в  его  время  совершенно
устарели. Понятно,  почему не  упоминает ни  Пифагора, ни  Платона, ни
других идеалистов:  чего же упоминать  сумасшедших? Как это  похоже на
современность и  не только  по отношению к  той стране,  где считалось
возможным называть «реакционерами» и  «мракобесами» Пастера и Менделя,
но  и  по  отношению  ко  многим  современным  дарвинистам.  Дарвинизм
---  единственное  материалистическое,   антирелигиозное  учение;  кто
отвергает  дарвинизм,  тот  тем  самым  ---  противник  науки,  с  ним
можно  не считаться.  Л.С.Берг  мне когда-то  говорил, что  английский
перевод его  книги «Номогенез»  упоминался в  одном обзоре  в качестве
книги, относящейся к естественной  теологии, и подавляющее большинство
современных эволюционистов не считает нужным считаться с этой книгой.

Но   связав  себя   с  догматическим   атеизмом,  материализм   сделал
невозможным  свое   участие  в  разработке   космологических  проблем.
Известно,  что учение  о  первичном  хаосе восходит  к  Гезиоду, и  из
этого  первичного  хаоса  волей  богов  возник  Космос.  Идеалисты  во
главе с  Пифагором и  Платоном восприняли дерзкую  мысль: человеческим
разумом  постичь  математические  законы  мироздания,  идеи  Бога.  По
мнению  материалистов,   из  хаоса  сама  собой,   путем  бесчисленных
столкновений  атомов  возникла  Вселенная:  откуда  же  могут  взяться
простые   математические   законы,   управляющие   Вселенной?   Законы
природы  материалистами понимаются  часто  не  в смысле  математически
формулированных  положений, а  в смысле  общих постулатов:  «из ничего
даже волей богов ничего не  творится» или «закона естественного отбора
и борьбы  за существование»:  «все люди  братья, но  должны относиться
друг к другу как Каин к Авелю».

Лукреций   был  очень   популярен  в   Римскую  эпоху,   но  было   бы
несправедливым  считать,   что  все  римляне  думали,   как  Лукреций.
Например, Сенека писал: (Уэвель, с. 265): «Движения планет, сложные и,
по-видимому,  перепутанные, были  подведены под  правило; и  после нас
явятся люди, которые откроют нам  путь комет». Очевидно, он был знаком
с  подлинно научной  астрономией того  времени. Но  Сенека был  стоик,
стоицизм же резко противостоял эпикуреизму.

\subsection{Гераклит  может считаться  материалистом, в  частности, по
полному непризнанию прогрессивного пифагорейского направления}

Мы  разобрали  взгляды  материалистической  линии  античности,  и  мне
думается,  можно  прийти  к   твердому  выводу,  что  линия  Демокрита
в  отношении   разработки  космологии  безнадежно  отстала   от  линии
Пифагора-Платона.  Но  к  материалистам   у  нас  принято  относить  и
Гераклита  Эфесского, главным  образом  потому, что  его высоко  ценил
Энгельс и что  Ленин считал, что учение Гераклита  есть «очень хорошее
изложение  начал  диалектического  материализма»  (История  философии,
1941,   с.  49--51).   Рассмотрим   сначала  астрономические   взгляды
Гераклита, памятуя, конечно, что он  начал жить еще при жизни Пифагора
и  умер  лет  за  50  до   рождения  Платона.  Поэтому  мы  не  вправе
предъявлять ему высоких  требований. Фрагментов Гераклита, относящихся
к  астрономии, очень  немного и  потому можно  привести все,  данные в
книге «Материалисты Древней Греции» (1956, с. 41--50).

3. Аэций. (О величине солнца). «Шириной в ступню человеческую».

6. Аристотель. «Солнце  не только (как говорит  Гераклит) новое каждый
день, но вечно и непрерывно новое».

94. Плутарх.  «Солнце не  перейдет своей меры,  иначе его  бы настигли
Эринии, помощницы Правды».

99.  Плутарх.  «Если  бы  солнце  не  существовало,  ---  несмотря  на
остальные светила, была бы ночь».

100.  Плутарх.   Периоды.  «Солнце,   их  управитель   и  наблюдатель,
устанавливает, руководит, назначает и  обнаруживает переходы и времена
года, которые все приносят».

124.  Феофраст.  «Прекраснейший космос  (был  бы)  как бы  куча  сору,
насыпанная как попало».

Странный взгляд «солнце каждый день  новое» совместим с общей позицией
Гераклита «все  течет», но вряд  ли может быть назван  научным. Рассел
(Человеческое  сознание,  1957,  с.  514) полагает,  что  Гераклит  не
принимал шарообразности Земли. В  «Истории философии» (с. 53) делается
попытка  объяснить крайне  наивные  астрономические взгляды  Гераклита
общим  уровнем науки  того  времени, но  уже Анаксимандр,  значительно
более старший, чем Гераклит, учил,  что Земля свободно парит, никем не
поддерживаемая, Гераклиту известно было  учение Пифагора и, однако, он
отозвался о нем весьма  презрительно: «Пифагор, сын Мнезарха, предался
исследованию больше всех  людей, и, разыскав эти  сочинения, извлек из
них  свою  собственную мудрость:  многознание  и  обманы» (Бляшке,  с.
113). Что  Гераклит и многих  других своих выдающихся  современников и
предшественников ценил очень низко, было показано в §6.

По этому  признаку ---  полному неприятию  прогрессивной пифагорейской
теории, Гераклит может действительно считаться материалистом.

Материалистическим  является и  такое высказывание  Гераклита, носящее
почти атеистический характер (Матер. Древней  Греции, 1955, с. 44, 30.
Климент):  «Этот космос,  один и  тот же  для всего  существующего, не
создал никакой бог  и никакой человек, но всегда он  был, есть и будет
вечно живым огнем, мерами загорающимся и мерами потухающим».

И  наконец, весьма  материалистически  звучит  его основное  положение
(Мат. Древ.  Греции, с. 46,  Ипполит): «Война  --- отец всего  и всего
царь; одним она  определила быть богами, другим ---  людьми; одних она
сделала рабами, других --- свободными». Здесь боги --- просто наиболее
удачливые люди. Знали или не знали римские правители, но они поступали
по  Гераклиту: наиболее  удачливые люди  --- императоры,  после смерти
автоматически делались богами.

Принцип  борьбы (резкое  противоречие  холизму)  проходил у  Гераклита
через  всю его  философию  снизу  доверху (Мат.  Др.  Греции, с.  48).
Ориген: «Следует знать,  что война всеобща и правда ---  борьба, и что
все происходит через борьбу и по необходимости».

Рассел (1959, с. 60) приводит такое изречение Гераклита: «Гомер был не
прав,  говоря:  ``Да исчезнет  война  среди  людей  и богов''.  Он  не
понимал, что молится  за погибель Вселенной; ибо, если  бы его молитва
была услышана, все вещи исчезли бы».

\subsection{Неприятие Лукрецием Гераклита объясняется идеалистическими
тенденциями Гераклита: энергетикой и учением о Логосе}

Но  тогда  почему  же  материалист Лукреций  так  резко  отзывается  о
Гераклите? (см.  §40). Своя  своих не  познаша? Но,  пожалуй, Лукреций
не  ошибся.   Полным  материалистом  Гераклита  назвать   нельзя.  То,
что  осуждает   Лукреций  у  Гераклита,  отвергается   и  большинством
современных  материалистов:   первичное  ---  огонь.   Здесь  Гераклит
является  провозвестником  энергетики  Оствальда. Так  именно  толкует
Гераклита один из крупнейших современных физиков, Гейзенберг (1958, с.
170), открыто  поддерживающий «линию  Платона». Лозунг  Гераклита «все
течет» ---  отрицание субстанции  в каком бы  то ни  было материальном
смысле. Учение о  необходимости, детерминизме, соответствует, конечно,
взглядам  Демокрита,  но  Эпикур  и его  последователь  Лукреций  были
индетерминистами, и  это только  означает, что  признание детерминизма
или индетерминизма  не дает возможности  установить материалистический
характер мировоззрения.

Но самое главное  отличие Гераклита от материалистов ---  его учение о
Логосе,  которое,  исторически,  если не  логически,  послужило  одним
из  источников  христианской  философии.  Так  кто  же  был  Гераклит:
материалист  или  идеалист?  Ни  тот,  ни  другой.  Это,  видимо,  был
мыслитель большой широты, и обе «линии»  могут у него находить свое, а
при наличии  фанатизма могут ругать  как противника. Но если  судить о
принадлежности  к  той  или  иной  линии,  то  можно  сказать,  что  в
целом  Гераклит  был  идеалистом, но  со  многими  материалистическими
тенденциями. Гераклит,  как известно, был учителем  Кратила, одного на
ранних  учителей  Платона, который  посвятил  своему  учителю один  из
прекрасных диалогов. И  как раз некоторые из  наименее устаревших (или
вовсе  не устаревших)  изречений  Гераклита  сохранились в  сочинениях
Платона,  именно  в  Гиппий  Большом  (Платон, т.  IX,  с.  19):  289:
«Сократ... Друг мой, тебе неизвестно хорошее изречение Гераклита: ``Из
обезьян  прекраснейшая безобразна,  если  сравнить  ее с  человеческим
родом''; и  прекраснейший из горшков  безобразен, если сравнить  его с
девичьим родом, как говорит Гиппий  Мудрый... Не утверждает ли того же
самого и Гераклит,  на которого ты ссылаешься, когда  он говорит: ``Из
людей  мудрейший, по  сравнению  с богом,  покажется  обезьяной, и  по
мудрости, и  по красоте,  и по всему  остальному''. Ведь  мы признаем,
Гиппий, что самая  прекрасная девушка безобразна по  сравнению с родом
богов».

Когда некоторые  слишком усердные  дарвинисты утверждали,  что разница
между  «низшими»  расами  человека   и  обезьяной  меньше,  чем  между
«низшими» и «высшими» расами, то они сделали шаг назад от Гераклита, а
не вперед.

В  своих   лучших  высказываниях  Гераклит  приближался,   конечно,  к
идеализму,  ну   а  ошибки  его   были  связаны  с   меристическими  и
материалистическими элементами его мировоззрения.

\subsection{Прогресс космологии  в античности целиком связан  с линией
Платона, линия Аристотеля --- консервативна, линия Демокрита привела к
полной утрате научной космологии}

Теперь постараемся резюмировать значение философии в развитии античной
космологии.  Длительный  тщательный  разбор этого  вопроса  как  будто
показал, что  вся прогрессивная  космология античности  развивалась на
линии Пифагора  и Платона  (линия же  Демокрита практически  ничего не
дала).  Но эта  связь могла  быть случайной.  Аристарх был  язычником,
Коперник  --- католиком,  Кеплер ---  протестантом, однако  это трудно
связать  с  их  астрономическими   заслугами.  Но  в  случае  развития
космологии  в  античные времена  (как  увидим  дальше и  позже)  связь
гораздо  более  ясна. Не  может  быть,  конечно,  и  речи о  том,  что
прогресс космологии был «случайным»;  он был следствием упорной борьбы
ряда  выдающихся  мыслителей,   проникнутых  твердым  убеждением,  что
существуют простые  математически формулируемые законы.  Иначе говоря,
необходимо было придерживаться холистических взглядов, чтобы надеяться
найти математические законы,  руководящие движениями планет. Напротив,
то,  что чрезвычайно  характерно  для  материализма ---  меристическое
миропонимание  ---  представление,  что  вселенная  есть  совокупность
не  связанных между  собой  частей,  столкновением которых  получается
реальный  мир,  ---  отнюдь  не  располагает  к  исканию  относительно
простых математических законов.  Это мы видим даже  в современности по
той оппозиции  к математическому  (и вообще  холистическому) пониманию
морфологии, которую проявляют современные  материалисты. Но в античной
культуре надо говорить не о двух  линиях --- Платона и Демокрита, а по
крайней мере  о трех.  Третья линия, возникшая  в лоне  платонизма, но
потом  выступившая в  качестве  главного оппонента  линии Платона  ---
линия Аристотеля, которую, строго говоря,  нельзя отнести ни к чистому
идеализму,  ни  к  чистому  материализму.  Линия  Аристотеля  утратила
веру  в возможность  точного математического  описания Вселенной,  она
довольствовалась приблизительным  описанием, но, потеряв  стремление к
точности, она усугубила требовательность  к доступности в «объяснении»
явлений. В  этом и  было основание ее  успехов в  естественных науках,
недоступных в  то время  математическому описанию.  Идеалистический же
характер философии  Аристотеля ясен  в первенствующем значении  в этой
философии  телеологического подхода,  не чуждого  и чистому  идеализму
(платонизму), но играющему там второстепенную, а не ведущую роль.

Формализм  Платона и  телеология  Аристотеля чужды  третьей линии  ---
Демокрита, где ведущим является признание случайности и необходимости.
Первые  две линии:  и  формализм  с учением  об  идеях, и  телеология,
естественно, очень  близки религиозному миропониманию  и, естественно,
играли большую  роль в философии  христианства, причем то или  иное из
этих  направлений  преобладало в  разные  эпохи  вплоть до  сего  дня.
Материализм же,  хотя далеко не  однозначен с атеизмом, но  весьма ему
конгениален.

Линия  Платона  дала блестящее  развитие  космологии  да и  не  только
космологии.  Линия  Аристотеля  склонна к  консерватизму  и  временами
приводит к полному застою (как в космологии с теорией гомоцентрических
сфер), но,  вообще говоря,  она отнюдь не  бесплодна, в  особенности в
биологии и многих  других науках. Что же касается  линии Демокрита, то
во времена  Лукреция она  вообще перестала быть  наукой, превратившись
просто в популярную и весьма неглубокую философию.

Однако ряд современных ученых (а может быть, даже большинство) склонны
считать  наличие  телеологического  момента в  философии  достаточным,
чтобы объявить такую философию ненаучной. Например, читаем у Б.Рассела
(1959,  с.  76)   про  Эмпедокла:  «Он  рассматривал   ход  вещей  как
регулируемый  скорее  случайностью  и  необходимостью,  чем  целью.  В
этом  отношении  его  философия  была  более  научной,  чем  философия
Парменида, Платона  и Аристотеля». Но «абсурдная»  с современной точки
зрения философия оказалась  плодотворной (Б.Рассел, с. 151)  в связи с
эмпирической астрономией.  Эта «абсурдная»  точка зрения,  по Расселу,
заключается  в мысли  Платона, что  астроном не  должен слишком  много
беспокоиться  о  существующих  небесных  телах,  а  скорее  заниматься
математикой  движения  идеальных  небесных тел.  Удивительно,  к  чему
приводит  фанатическая  нетерпимость  к  определенному  мировоззрению.
Расселу кажется  абсурдным, как  размышление о  математических законах
движения идеальных  небесных тел могло  привести к успеху  в понимании
движения реальных небесных  тел. Но мы знаем,  что кинетическая теория
газов  в ее  первоначальной, классической  фазе руководилась  понятием
«идеального» газа  и привела  к огромным  успехам в  понимании законов
реальных  газов.  Почему  же  считать  подход  Платона  «абсурдным»  с
современной точки зрения?

\subsection{Философия  Платона проникнута  эстетическими и  этическими
лейтмотивами.   В   материализме  доминируют   богоборческие   мотивы,
имеющиеся и у Платона}

Но у Рассела есть и иначе формулированное обвинение против идеализма в
науке (1959,  с. 152): «Эта  часть научной истории  иллюстрирует общее
правило:  любая гипотеза,  как бы  абсурдна  она ни  была, может  быть
полезной для науки, если она дает возможность открывателю мыслить вещи
по-новому;  но когда  она служит  этой цели  случайно, она,  вероятно,
станет препятствием для дальнейшего развития вперед. Вера в благо, как
в  ключ для  научного понимания  мира, была  полезной на  определенной
стадии  развития астрономии,  но на  каждой более  поздней стадии  она
приносила вред. Пристрастие к этике  и эстетике, наблюдаемое у Платона
и еще более у Аристотеля, значительно способствовало тому, чтобы убить
греческую науку».

Трудно  понять,  как  такой  выдающийся  мыслитель,  как  Рассел,  мог
сформулировать столь несправедливое обвинение.

Эстетический подход к Вселенной характерен только для Пифагора (Космос
= Красота) и Платона, но  не Аристотеля и Демокрита. Этот эстетический
подход и был плодотворной идеей,  двигавшей и Коперника и Кеплера, как
увидим дальше: так  что «убить» греческую науку он  не мог. Аристотель
написал много (и многое очень хорошо)  по эстетике, но не в применении
к  Вселенной. В  отношении этики  у  Аристотеля можно  сказать, что  и
здесь  он написал  выдающиеся  сочинения,  которые сохранили  свежесть
до  настоящего  дня, но  ко  Вселенной  у  него  был не  этический,  а
телеологический подход, что  далеко не одно и то же.  Одно из основных
положений Аристотеля, принимавшееся и  Ньютоном, гласит (Ньютон, 1916,
с.  449 ---  упоминаются  «философы» без  имени Аристотеля):  «Природа
ничего не  делает напрасно, а  было бы напрасным совершать  многим то,
что  может быть  сделано меньшим.  Природа проста  и не  роскошествует
излишними причинами вещей». Этот телеологический подход принимает, что
природа совершенна, а  если она совершенна, то как  можно стремиться к
ее усовершенствованию? А там, где нет стремления к усовершенствованию,
нет надобности и  говорить об этике. Этика только там,  где может быть
выбор  между двумя  решениями: лучшим  и худшим.  И философия  Платона
проникнута не  только эстетическим,  но и этическим  лейтмотивом. Есть
совершенный мир идей, а наш реальный мир несовершенен: надо стремиться
к приближению к совершенству.

Этический подход  чрезвычайно важен в  эволюции религий. В  мире много
зла  и несправедливости:  кто за  это отвечает?  Если боги  всемогущи,
почему они терпят зло? Вот источник тех богоборческих течений, которые
пронизывают многие религии.

Здесь и книга Иова, и легенда  о борьбе Иакова с Иеговой, и знаменитый
миф о  богоборце Прометее. Элементы  богоборства имеются и  у Платона,
например,  в  его  диалоге  «Евтифрон», где  Сократ  ставит  вопрос  о
независимости  понятия  справедливости от  божественных  установлений.
Сюда же относится критика тех легенд о богах, которыми полны сочинения
Гомера. По платоновской линии шла серьезная критика эллинских богов, и
эта критика в конце концов привела к христианству и к гибели эллинских
богов. Обвинение  афинянами Сократа, что он  отрицает греческих богов,
было, таким образом, справедливо,  но в справедливости этого обвинения
заключается одно из главных оснований для права Сократа на бессмертие.
Этический  же импульс  у  Платона привел  его  к исканию  справедливой
формы  государства,  что  явилось началом  всего  социалистического  и
коммунистического движения. Таким образом, и здесь этический подход не
оказался бесплодным, но так как это направление в социологии враждебно
Расселу, то он здесь не может видеть никакой заслуги.

\subsection{Богоборческие  аргументы материалистов  имеют силу  только
при  понимании Бога  как  всемогущего деспота,  не связанного  никаким
законом}

Но   не   следует   думать,   что   этические   мотивы   чужды   линии
Демокрита.  Материалисты  тоже  недовольны  действительностью,  но  из
этого  недовольства   делают  совершенно   другие  выводы.   Не  менее
решительно,  чем   Платон,  Лукреций  критикует  древние   сказания  и
использует  их  для доказательства  того,  что  религия не  только  не
предохраняла от преступлений, а, напротив, принуждала к преступлениям:
жертвоприношение  Ифигении (кн.  I, с.  80--102). Лукреций  указывает,
что  злодеи остаются  безнаказанными, а  невинные гибнут  (кн. VI,  с.
390--398). Нелепо,  что молния,  орудие Юпитера, поражает  храмы богов
(кн.  I,  ст.  418) и  т.д.  Мы  видим,  что  Лукреций не  говорит:  я
даю  научное  объяснение  мира,  которое несовместимо  с  религией,  а
дает  такую  схему  рассуждений:  если  бы  существовали  справедливые
и  всемогущие  боги,  то  в  мире  не было  бы  зла.  А  так  как  зло
существует и часто  творится по указанию жрецов, то,  значит, богов не
существует,  жрецы-обманщики, и  тот, кто  защищает религию,  является
сообщником  обманщиков.  Эта  схема рассуждений  сохранилась  до  сего
времени как главный и самый сильный аргумент атеистической пропаганды.
Но  это рассуждение  в основном  этическое, а  не независимое  научное
рассуждение.

Но рассуждения  воинствующих атеистов, подобных Лукрецию,  основаны на
понимании  бога как  всемогущего деспота.  Он может  сделать буквально
все, даже изменить таблицу умножения  и законы логического мышления, и
он абсолютно  не связан  никакими законами.  Не таково  было понимание
пифагорейцев и  платоников, а также иудейской  и христианской религии.
Бог --- Законодатель,  судья, но данные им законы он  уже не изменяет.
Священные книги  христиан называются Ветхим (старым)  и Новым Заветом,
т.е. договором,  и мы  знаем, что  по Библии  Бог клялся  в соблюдении
договора. Примитивно мыслящие попики  дошли до утверждения абсолютного
всемогущества  Бога, но  вообще  человек, склонный  к  догматизму и  к
законченности  учения,  любит  абсолютировать  любезные  ему  понятия.
Так,  Вейсман говорил  о  всемогуществе  естественного отбора,  многие
современные коммунисты говорят о всемогуществе партии и т.д.

Не  следует   думать,  что  учение  об   ограниченности  «всемогущего»
бога  есть учение  еретическое. Б.Рассел  приводит мнение  крупнейшего
богослова католической церкви, Фомы Аквината. Фома утверждает, что бог
не  может  лишить человека  души,  но  что  это не  является  реальным
ограничением  его  всемогущества, а  есть  всего  лишь результат  того
факта, что человеческие души бессмертны и, следовательно, неуничтожимы
(Б.Рассел, 1959, с. 476, цитирую по Тьюрингу, 1960, с. 33). Выдающийся
философ  Лейбниц в  своей  «Теодицее» (что  значит «Оправдание  бога»)
доказывал, что бог создал наилучший из возможных миров, следовательно,
он не мог выбрать любой мир, а только один из возможных.

Наконец,  многие религии  и философские  системы, и  далеко не  только
примитивные, склонны  к признанию наличия злых  внематериальных начал,
пансатанизму;  из  философских  систем  такой  уклон  имеет  философия
Шопенгауэра.  Поэтому  аргументация   Лукреция  и  его  последователей
опровергает   только  одну   из  возможных   форм  религий,   по  всей
вероятности, никогда не существовавшую.

\subsection{С  постепенным  исчезновением  Прометеева,  богоборческого
духа  на  линии  Демокрит-Лукреций постепенно  исчезало  стремление  к
теоретической  науке, торжествовал  практицизм  палача, личная  мораль
эпикурейцев}

Роль  этического  и  религиозного  момента в  развитии  науки  гораздо
сложнее, чем  представляется многим.  Религия страха  перед всемогущим
деспотом, конечно, только тормозит развитие культуры. Представление же
о  благом, справедливом  законодателе стимулирует  развитие науки.  На
этот  путь  и  встали  Пифагор, Сократ,  Платон  и  их  многочисленные
последователи, и  эта вера  руководила ими  в победоносном  шествии по
пути  математизации  законов  природы  и  в  разыскании  справедливого
общественного строя.

Линия же Демокрита  в лице его основоположника  началась со свободного
научного исследования, вовсе не противного, по-видимому, религии, но и
не стремившегося к  радикальной реформе религии. Но  отсутствие веры в
имманентную красоту Космоса,  многочисленные факты дисгармонии природы
и  обилие  зла в  общественной  жизни  привели  уже Эпикура,  с  одной
стороны,  к  научному  индифферентизму,  а  с  другой,  ко  все  более
усиливающимся сомнениям  в существовании реального  порядка Вселенной.
Основоположники материализма, Демокрит и  Эпикур, судя по всем данным,
были лично  представителями исключительно высокой морали.  В отношении
Эпикура  все показания  сходятся  о его  высоком нравственном  облике.
Гегель, например (1932, с. 363 и  364), указывает, что ни один учитель
не пользовался  у учеников такой  любовью и уважением, как  Эпикур, но
это уважение  после смерти привело  к тому,  что в его  учение считали
невозможным внести какое-либо изменение, в противоположность тому, что
мы  знаем в  отношении  учеников Платона  и  стоиков. Единственный  из
учеников  Эпикура, Метродор,  развивший дальше  некоторые стороны  его
философии, был  и единственным,  перешедшим к Карнеаду,  платонику. На
эту верность  Эпикуру, на то,  что многие последователи  других систем
перешли  к Эпикуру,  но никто  (кроме  Метродора) не  изменил ей  ради
другой, руководитель  Новой Академии Аркезилай, как  указывает Гегель,
остроумно ответил: «Мужчины могут сделаться кастратами, но кастраты не
могут сделаться снова  мужчинами». Гегель и добавляет, что  вряд ли мы
должны очень жалеть о том, что большинство сочинений Эпикура до нас не
дошло.

Как говорят,  Эпикур много  занимался наукой, но  в согласии  со своей
философией, философией безмятежности (атараксия) он и в науке, как и в
общественной  жизни, не  был напряженным  искателем великих  тайн; для
него  наука была  приятное развлечение,  заполнение досуга,  а у  тех,
кто  получил  позднее название  эпикурейцев,  наука  и вовсе  исчезла.
Моральный  богоборческий стимул,  прометеев  дух, исчез.  С кем  можно
бороться,  когда  богов  нет?  Исчезла  и  высокая  наука,  исчезла  и
мораль.  Все очень  подходило к  торжествующему в  римскую эпоху  духу
государственности и практицизма. Отсутствие глубоких исканий привело к
тому,  что  и  в  практической области  римляне  были  преимущественно
заимствователями, и  даже такие достижения  в архитектуре, как  арка и
свод  и  купол, строительство  клоак  (канализация),  римские цифры  и
проч., которые обычно приписываются Римскому государству, по-видимому,
были  заимствованы  от  этрусков (де  Веер,  1962,177--178).  «Римский
солдат, убивший Архимеда, был  символом гибели оригинального мышления,
которую  принесло  римское  господство  всему  эллинистическому  миру»
(Б.Рассел,  1959, с.  237). «В  бедственном римском  мире было  стерто
грубой  рукой все  благородное и  прекрасное, представляемое  духовной
индивидуальностью»  (Гегель,  1932,  с.  325).  Исчезла  и  конкретная
нравственность и конкретная наука.  За внешним блеском Римской империи
скрывалось истинное царство мрака.

Но этот  термин «века мрака»  широко применялся к Средневековью.  Мы и
переходим к этому периоду, предшественнику Возрождения.

\subsection{Против  широко   распространенной  схемы   Средних  веков,
рассматривающей  эту   эпоху  как  «века  мрака»   в  силу  господства
христианской   идеологии,   можно    выдвинуть   возражения:   1)   не
христианство, а Рим ---  виновник крушения великой эллинской культуры;
2)  христианские императоры  преследовали  эллинскую  культуру не  как
христиане, а как императоры --- наследники Рима}

Между  античным  миром и  Ренессансом  лежит  огромный интервал,  где,
в  частности  в  области  космологии, как  будто  был  полный  застой.
Значит ли  это, что  целесообразно полностью игнорировать  этот период
Средних  веков  как пустой  для  развития  всякой науки?  Примерно  по
такой  схеме толковали  этот  период многие  историки науки  недавнего
прошлого.  Начало Средних  веков,  как известно,  совмещали с  гибелью
Римской империи. Христианство подорвало Рим, овладело всей культурой и
зажало  возможности ее  развития. Все  было подчинено  христианской, в
первую  очередь католической,  догматике. Возрождение  было связано  с
возвращением  от арабов  в  Европу  старой языческой  антихристианской
идеологии.  Века  мрака  были  побеждены  Реформацией  и  Ренессансом,
порвавшими в  той или  иной степени  с католической  церковью. Средние
века ---  это безвременье культуры,  если бы их не  было, человечество
только бы  выиграло. Такое представление господствовало  среди широких
кругов лиц,  считавших себя  прогрессивными, на рубеже  последних двух
столетий.

Сейчас против этой схемы выдвигается совершенно иное понимание.

1)  Не христианство,  а  Рим является  виновником крушения  величайшей
эллинской культуры.  Политически Эллада  была сокрушена  Македонией, а
потом  Римом.  Но  между  Македонией  и  Римом  ---  большая  разница.
Преемники  Македонии  ---   египетские  Птолемеи  были  воспреемниками
подлинной культуры и именно  в Александрии эллинская культура достигла
своего  апогея.  Рим   же  не  принял  великой   эллинской  науки,  не
принял  ни Платона,  ни Аристотеля.  В Риме  конкурировали стоицизм  и
эпикуреизм, оба  философских направления, далекие  от науки. Но  Рим в
значительной степени осуществил дело, начатое Александром --- создание
всемирной империи,  выражавшей в грубой, несовершенной  форме подлинно
католическую, т.е. вселенскую,  космополитическую идею. В определенных
отношениях --- политическом и этическом  --- Александр и римляне стали
причиной появления лучшей  философии, чем та, какую  развивали греки в
дни  своей свободы.  Стоики,  как мы  видели, верили  в  то, что  люди
---  братья, и  не  ограничивали своих  симпатий греками...  Концепция
человечества  как единой  семьи, единой  католической религии,  единой
универсальной культуры  и единого охватывающего всю  землю государства
преследовала  человеческую   мысль  со  времени   ее  приблизительного
осуществления Римом...  Роль Рима  в распространении  цивилизации была
чрезвычайно важна...  Можно сказать, что качество  цивилизации никогда
уже не было таким, как в Афинах Перикла, но в мире, потрясаемом войной
и разрушением, в конечном счете количество почти столь же важно, как и
качество, а количеством мир обязан Риму (Б.Рассел, 1959, с. 298, 299).

2) Христианские императоры приложили свою руку к делу борьбы с великой
эллинской  культурой,  но они  это  делали  не  как христиане,  а  как
императоры,  продолжавшие дело  Рима. Свирепый  гонитель христианства,
император  Домициан,   изгнал  из   Рима  «философов,   отравителей  и
математиков» (в  94 г. н.э.), причем  Рим должен был покинуть  и стоик
Эпиктет  (Гегель,  1932, с.  332).  Примерно  через 400  лет  политику
Домициана  проводил христианский  император  Юстиниан, обозначивший  в
своем знаменитом кодексе «злоумышленников, математиков и им подобных»,
а  в   529  г.  закрывший  платоновскую   Академию,  просуществовавшую
непрерывно около девяти столетий. Юстиниан, знаменитый своим кодексом,
главным  источником всемирно  известного  римского  права, не  слишком
разбирался в  философии и вместе  с платониками выгнал  всех языческих
философов,  в   том  числе  и  знаменитого   комментатора  Аристотеля,
Симпликия, о котором придется вспомнить, когда мы доберемся до Галилея
(Гегель, 1935, с. 74).

Такая неприязнь в  Риме и Византии к  «математикам» объясняется прежде
всего тем, что там имя математика было синонимом халдея или астролога.
Люди этого  разряда неоднократно изгонялись из  Италии постановлениями
Сената, как  во времена  республик, так и  при империи.  Как указывает
Тацит,  повторение этого  указа показывает,  что он  не имел  действия
(Уэвель, 1867, с. 366), в Греции же как будто государство не проявляло
вражды  к  астрологам, но  и  астрологи  там  далеко не  имели  такого
распространения, как в Риме.

Мы, живущие в XX в., не  будем слишком строги к Домициану и Юстиниану,
так как гонение на математиков за их действительную или мнимую связь с
ложным (по мнению правителей) учением  имеет место и в середине нашего
просвещенного века.

\subsection{3)  Отношение христианской  церкви  к языческой  философии
никогда не было единым и в  начале христианства было мощное течение по
синтезу платонизма и христианства}

3)  Отношение   христианства,  в  частности  католической   церкви,  к
античной,  языческой  философии  никогда  не было  единым,  и  наличие
разнообразных   мнений   сохранилось   до  настоящего   времени.   Это
обстоятельство очень важно для  понимания многих сторон истории науки.
Первые  отцы церкви  (богословы первых  веков христианства)  отнюдь не
отрицали связи  христианства не  только с  иудаизмом, но  и греческими
философами  (гл.  11,  §10).  Юстин-мученик не  был  одиноким.  Синтез
иудаизма  с эллинистической,  прежде  всего платонической  философией,
начатый  Филоном Александрийским  (ок.  20--54  гг. н.э.),  сохранялся
долгое время  в александрийской общине. Влияние  эллинизма, и особенно
неоплатонизма,  особенно усилилось  при Клименте  Александрийском (ок.
150--215 гг.  н.э. История философии,  1941, с. 389).  «Климент развил
теорию объединения  веры и  знания, которая была  принята христианской
церковью.  По  Клименту,  нет  знания  без веры  и  веры  без  знания.
Полная гармония  их требует изучения всего  круга человеческих знаний:
``семи свободных  искусств''. Никакой несовместимости  между языческой
философией и христианским учением, согласно  Клименту, нет: это как бы
две  ветви одного  и того  же ствола.  Истины христианства  согласны с
учениями  лучших из  язычников.  Философия представляет  собой как  бы
пропедевтику, преддверие  христианства. В философии  истина содержится
не целиком в одной какой-либо школе, а по частям во всех». У Климента,
как и у Филона, главным  приемом для введения философии в христианство
было «аллегорическое объяснение Священного Писания».

Преемником  Климента  по  наставничеству  в  Александрийской  школе  и
продолжателем  его  философской  линии   был  знаменитый  Ориген  (ок.
185--254  гг.).  Через  Оригена  древняя  философия  обильным  потоком
проникла в  христианство. «Целый  ряд учений Оригена  был впоследствии
отвергнут  церковью.  Так,   например,  неправомерными  были  признаны
учения  Оригена  о   бесконечном  количестве  миров,  предшествовавших
нашему, и,  следовательно, о  вечности мироздания. Отвергла  церковь и
(платоновское)  учение  о  предсуществовании  душ и  о  знании  как  о
припоминании. Наконец,  церковь осудила,  после долгой  и ожесточенной
борьбы, учение Оригена о том, что ``сын'' (вторая ``ипостась троицы'')
во  всем  ниже  ``отца''.  И   тем  не  менее,  даже  после  признания
многих учений  Оригена еретическими, авторитет его  среди христианских
писателей стоял  очень высоко»  (История философии,  1941, с.  390). В
полном  согласии с  его платонизмом  Ориген придавал  большое значение
естествознанию,  натурфилософии, геометрии  и астрономии,  и геометрию
считал образцом и  идеалом остальных наук (там  же). Теория подчинения
«сына»  «отцу»  привела  к осуждению  Оригена  двумя  александрийскими
синодами  231  г., приговорившими  его  к  изгнанию из  Александрии  и
лишению  звания  пресвитера.  Насколько  мне  известно,  другим  (если
не  главным)  поводом  к  лишению Оригена  сана  священства  было  его
самооскопление, которое он, руководствуясь  одним местом из Евангелия,
произвел во избежание соблазна:  по удивительному закону, действующему
и в наше время, выдающиеся  проповедники пользуются огромным успехом у
многих  чрезмерно  религиозных женщин.  Мы  видим,  таким образом:  1)
осуждение Оригена  было вызвано  не его астрономическими  взглядами, а
его  богословскими  суждениями; 2)  это  осуждение  не привело  его  к
отлучению  от  церкви; 3)  и  после  осуждения  и даже  до  настоящего
времени среди  богословов авторитет Оригена стоит  чрезвычайно высоко;
4)  наконец,  что  самое  важное,  платоновская  линия  Оригена  нашла
продолжателей  среди высших  представителей христианского  духовенства
даже   после  торжества   «антифилософской»   линии  в   христианстве,
сделавшейся  государственной  религией.  Продолжали линию  Климента  и
Оригена  Григорий,  епископ  Нисский  (ок. 335--394  гг.),  его  брат,
Василий  Великий (умер  в  379 г.),  и  третий знаменитый  каппадокиец
(Малая  Азия)  Григорий  Назианзский  (умер в  390  г.)  (см.  История
философии, 1941,  с. 391,  Ибервег-Гейнце, 1898,  с. 103).  Даже после
гибели  от  руки христианских  фанатиков  Гипатии  в Александрии  (415
г.)  ученик  Гипатии,  Синезий (365--430),  продолжал  быть  епископом
в  Птолемаиде  (посвященный  александрийским  патриархом  Теофилом)  и
развивал  совершенно  платонические воззрения  (Ибервег-Гейнце,  1898,
с.  133--136). Сделавшись  христианином и  епископом, Синезий  открыто
признавал,  что  не  во  всем  согласен с  церковным  учением.  Он  не
верил  в  гибель мира,  склонялся  к  идее предсуществования  душ.  Он
принимал бессмертие души, но учение о воскресении душ рассматривал как
священную аллегорию. Изложение господствующих догматов он рассматривал
как  популярные  мифы,  делающие более  доступными  трудно  постижимые
абстрактные идеи.  Он рассматривает Бога как  единство единств, монаду
монад, единство противоположностей  и т.д. Платонизм явно  выражен и у
вершины  патристики  Августина,  который,  как увидим  в  свое  время,
предвосхитил многие современные научные идеи.

К платонизму  были склонны  наиболее выдающиеся  представители раннего
христианства, занимавшие высшие должности в христианской церкви и, как
правило,  и  доселе  считающиеся образцом  ортодоксии.  Неудивительно,
что  неоплатонические  взгляды  ряда   авторов  были  прикрыты  именем
Дионисия  Ареопагита, первого  афинского христианского  епископа, хотя
труды, ему  приписываемые, имеют, как сейчас  принимают, более позднее
происхождение.

\subsection{Антифилософское направление  в христианстве, закончившееся
гибелью  Гипатии,  отчасти  связано с  совершенно  неудачной  попыткой
реставрации язычества Юлианом}

Но   если   господствующей   среди   высших   представителей   раннего
христианства  была склонность  к  сближению  христианства с  античной,
прежде всего платонической, философией, то  наряду с этим уже довольно
рано возникло другое  направление, противополагавшее христианство всей
античной  культуре.  Одним из  первых  был,  видимо, ассириец  Татиан,
ученик  Юстина-мученика. В  противоположность своему  учителю, еще  до
казни  последнего  (ок.  166  г.),   он  резко  выступал  против  всей
античной философии  в целом, причем приводил  много совершенно нелепых
аргументов. Греческих богов  он отождествляет с демонами,  не в старом
сократовском смысле, а в смысле  злых существ (История философии 1941,
с. 387, Ибервег-Гейнце 1898, с. 58).

Большее влияние в  том же направлении имел  знаменитый Тертуллиан (ок.
150--222), резко противопоставивший  нравственность --- чувственности,
божественное   откровение  ---   человеческому  разуму,   религию  ---
философии, христианство --- язычеству. В философии, однако, Тертуллиан
был под сильным влиянием стоического материализма и стоицизма (История
философии, 1941, с. 387--388).  Тертуллиан признавал телесность души и
бога,  считая, что  чувства нас  не обманывают  (Ибервег-Гейнце, 1898,
с.  74). Эта  приверженность  Тертуллиана материализму,  наряду с  его
пламенной христианской  верой, и привела его  к знаменитому выражению:
«Верю,  потому что  это  абсурдно». Это  изречение,  видимо, точно  не
встречается  в  его сочинениях,  но  вполне  соответствует их  смыслу.
Некоторые  невозможные  события  с  нашей  человеческой  точки  зрения
(например, воскресение из мертвых) мы  должны принять, так как они нам
оповещены божественным  откровением: понять это разумом  невозможно, в
это  надо верить.  Как мы  видим,  резкий антагонизм  между научным  и
религиозным пониманием  был свойствен  именно материализму, и  в таком
понимании он оказался вредным и для религии, и для науки.

Антифилософское  направление   в  христианстве   окрепло  в   связи  с
превращением   христианства  в   государственную  религию,   император
Константин умер  в 337 г.,  но уже в  325 г. под  его покровительством
состоялся  первый Вселенский,  Никейский собор  для борьбы  с ересями,
прежде всего ересью Ария. В 319 г. руководителем Александрийской школы
становится  «отец православия»  Афанасий Великий,  изгнавший из  школы
философское  направление  Климента  и  Оригена.  Внутренняя  борьба  в
христианской  церкви  приобретает  все  более  ожесточенный  характер.
Но  не следует  думать,  что Афанасий  полностью  отказался от  учения
Александрийской школы. Он известен как  главный противник Ария, а Арий
довел до конца мысль Оригена о том, что «бог-сын» ниже «бога-отца». По
учению  Ария, бог-сын  был сотворен,  т.е. не  существовал «от  века»,
против чего  в символе веры  прямо говорится: «иже от  отца рожденного
прежде всех  век». Вряд ли  Афанасий был враждебен  всей платонической
философии в целом,  иначе трудно было бы допустить, чтобы  один из его
учеников, Григорий  Назианзский, или Григорий Богослов,  один из «трех
святителей» мог быть  ясно выраженным платоником. Он  и свое прозвание
«Богослов» получил за то, что  развивал дальше учение о божественности
«Слова»  (логоса), намеченное  вкратце в  начале Четвертого  Евангелия
(Ибервег-Гейнце, 1898, с. 106).

Дивергенция  христианства  и  платонизма  продолжалась  все  дальше  и
кончилась  в  415 г.  трагической  гибелью  в Александрии  благородной
женщины Гипатии. В это время  патриархом александрийским был Кирилл (с
412 по 444  г.), которому и приписывают (например,  Б.Рассел, 1959, с.
383)  главную вину  в  этом  убийстве, как  и  в провокации  еврейских
погромов. Как могло  случиться, что в том же  городе, где проповедовал
просвещенный  Климент,  через  200  лет  после  его  смерти  произошли
такие  страшные злодеяния?  Может  быть, дело  станет  яснее, если  мы
сообразим,  что в  промежутке  между Климентом  и Кириллом  действовал
император Юлиан-отступник  (царствовал в  361 -363 г.),  который вновь
преследовал христиан и вместе с тем был открытым и, как увидим дальше,
довольно  неудачным  приверженцем  платонизма. А  Кирилл,  не  имеющий
философских  заслуг,  оставил  сочинение,  прямо  направленное  против
антихристианского сочинения  Юлиана (Ибервег-Гейнце, 1898, с.  136). А
по  широко  распространенной  примитивной схеме  двух  лагерей:  «друг
нашего врага ---  наш враг» эта пропаганда  платонизма Юлианом сыграла
плохую услугу, и через фанатиков христианства отразилась на взглядах и
действиях невежественной толпы.

\subsection{Платоновская линия  существовала и после гибели  Гипатии и
выражена талантливейшим Августином}

Гибель Гипатии  вовсе не означала разрыва  христианства с платонизмом.
Как  уже  было  указано, величайший  мыслитель  раннего  христианства,
Августин  (354--430)  был  ясно   выраженным  платоником.  Он,  пройдя
скептический  период, был  увлечен  «Эннеадами» Плотина,  а потом  уже
сделался  христианином,  но и  в  основном  своем сочинении  «О  граде
божием»  он  считал, что  в  богословии  «следует рассуждать  согласно
платоникам,  в  сравнении  с   мнениями  которых  учения  всех  других
философов  должны цениться  ниже» (История  философии, 1941,  с. 392).
То  обстоятельство, что  Августину  же  принадлежит сочинение  «Против
академиков» (имеется в виду Академия  Платона) обозначает лишь то, что
он выступал против того  скептического направления в Академии, которое
было  ей  свойственно  при руководстве  Академии  скептиком  Карнеадом
(примерно II в.  до н. з.). Во времена  Августина были единомышленники
Карнеада, а  то направление, которое поддерживал  Августин, называлось
неоплатонизмом.

Сейчас  принято выискивать  у  Августина  такие высказывания,  которые
вызывают   у   нас   справедливое  возмущение:   о   вечных   мучениях
мертворожденных  младенцев, о  необходимости  преследовать еретиков  и
др.  Но тому  же Августину  принадлежит утверждение  о несовместимости
смертной  казни  и христианства,  протест  против  судебных пыток  (Б.
Рассел, 1959,  с. 377) и  целый ряд  блестящих идей о  природе времени
(Рассел,  1959, с.  370),  а в  биологии идея  о  потенциальном, а  не
актуальном творении.  Иначе говоря,  Августин принимал  эволюцию живых
существ,  о  чем  подробнее  придется  говорить,  когда  доберемся  до
биологии.  По мнению  того  же атеиста  Б.Рассела, «широкая  концепция
противоположности  между градом  мира  сего и  градом божиим  осталась
для  многих  вдохновляющей  идеей  и   даже  ныне  может  быть  заново
изложена  небогословским языком»  (с. 371).  Если не  ошибаюсь, именно
Августин изложил идеал государства как «совершенная свобода частей при
совершенном единстве целого».

Августин    высоко   ценил    математику    и    подводил   под    это
богословско-пифагорейское  основание.  Николай  Кузанский,  о  котором
вскоре  придется говорить  подробнее,  писал: «Так,  платоники и  даже
первые из наших мыслителей  следовали математике настолько строго, что
св. Августин  и вслед за  ним Боэций утверждали, что  число неоспоримо
было в  мысли творца его  основным образцом для создания  вещей» (Ник.
Кузанский, 1937, с. 23--24).  Августин часто прибегал к математическим
обоснованиям своих мыслей о бессмертии души и проч.

Августин,  как и  другие  наиболее выдающиеся  отцы церкви,  продолжал
линию благожелательного отношения к науке и, прежде всего, математике,
которая была  свойственна Платону и  Пифагору, и  не их вина,  что эта
линия прервалась.  Ведь Августин  работал в  то время,  когда северные
варвары уже  надвинулись на  цивилизованный мир, и  его труд  «О граде
божием»  был  ответом (он  его  писал  с 412  по  427  гг.) на  мнение
язычников, что  разграбление Рима  готами в 410  г. является  карой за
забвение древних  богов (Б.Рассел,  1959, с. 370).  Экономическая база
развития науки  была подорвана и, естественно,  в обстановке нашествий
варваров возобладали мрачные, антиэллинистические настроения.

\subsection{4)    Среди    неоплатоников    было    ярко    выраженное
антихристианское направление}

4)  Но  если  идеологическая  близость платонизма  и  христианства  не
подлежит  никакому сомнению,  то  это не  значит,  что неоплатоники  в
целом сочувствовали христианству. Напротив,  наряду с изложенными выше
примерами  перехода  платоников  в  христианство,  мы  имеем  обратные
случаи. Так,  основатель неоплатонизма, Аммоний Саккас  из Александрии
(ок. 175--242  гг. н.э.) воспитывался  в христианской семье,  но затем
выступил  как  обоснователь  неоплатонизма  и  противник  христианства
(История  философии, 1941,  с. 372).  Ученик его,  наиболее выдающийся
представитель  неоплатонизма,  Плотин  (205--270 гг.  н.э.)  родом  из
Египта, но основавший  школу в Риме, продолжал  во многом платоновскую
линию, но оставался чуждым христианству, несмотря на то что он защищал
некоторые новые  идеи, сходные и  с идеями многих ранних  христиан, но
совершенно  чуждые Платону.  Так, например,  он заявлял,  что стыдится
своего  тела  (История философии,  1941,  с.  373), т.е.  защищал  тот
аскетизм, который шел вразрез с платоновским мнением о том, что только
варвары  стыдятся  обнажать  тело. Весьма  возможно,  что  утверждение
авторов  «Истории  философии»  неточно,  так как  Б.Рассел  (1959,  с.
308)  утверждает,  что  в   мистицизме  Плотина  нет  ничего  мрачного
или  враждебного  красоте.  Но  Рассел прибавляет,  что  он  последний
религиозный  учитель  на  много  десятилетий,  кому  может  быть  дана
такая  характеристика. «Красота  и все  наслаждения, связанные  с нею,
начинают рассматриваться как идущие от  дьявола, язычники, равно как и
христиане, начинают воспевать уродство и грязь. Юлиан-отступник, как и
современные ему  ортодоксальные святые, хвастался  населенностью своей
бороды. Обо  всем этом нет  ни слова  у Плотина». Рассел  отзывается о
Плотине с исключительной симпатией, несмотря на свое антихристианство.
Он  же  указывает, ссылаясь  на  настоятеля  Инге, как  много  обязано
христианство платонизму  вообще и в  частности Плотину. По  Инге, Фома
Аквинский  ближе к  Плотину,  чем к  подлинному Аристотелю  (Б.Рассел,
1959,  с. 301).  Плотин не  отрицал полностью  астрологии, но  пытался
поставить  границы  для нее  так,  чтобы  оставшуюся сферу  приложения
астрологии  совместить   со  свободной  волей.  Трудно   выше  оценить
общекультурное значение Плотина, чем словами того же Рассела (с. 312):
«В III  веке и в  те века,  которые следовали за  нашествием варваров,
западная цивилизация стояла на  грани окончательной гибели. К счастью,
в то время как теология была почти единственной сохранившейся областью
умственной деятельности,  система, принятая  тогда, не  была полностью
суеверна,  а сохраняла,  хотя временами  и глубоко  скрытые, доктрины,
воплощавшие многое из достижений греческой мысли и многое из моральной
набожности,  общей  стоикам  и неоплатоникам.  Это  сделало  возможным
возникновение  схоластической  философии,   а  позже,  с  наступлением
Возрождения, усилило влияние Платона в эту эпоху, а тем самым и других
древних  философов...»  «Философия  Плотина одновременно  и  конец,  и
начало: конец того,  что касается греков, и начало  того, что касается
христианства».

Продолжателями  Плотина  были  Порфирий  (родился в  232  г.),  Ямвлих
(умер  в  330 г.)  и  крупнейший  представитель афинской  школы  Прокл
(410--485). Здесь антагонизм с христианством достиг чрезвычайной силы.
Достаточно  сказать,  что  один  из  учеников  Ямвлиха,  Сопатер,  был
казнен  императором  Константином  за  то, что  он  при  помощи  магии
вызвал ветер,  отогнавший в сторону корабли  с зерном (Ибервег-Гейнце,
1894, с.  352). Другим учеником Ямвлиха  был император Юлиан-отступник
(361--363).  Прокл  отличался  чрезвычайной  религиозной  терпимостью.
«Философ, говорил  он, не  есть жрец одной  религии, но  всех религий,
существующих на  свете. Поэтому  он писал гимны  в честь  всех божеств
Греции, Рима,  Египта, Аравии;  одно христианство не  пользовалось его
благосклонностью» (Уэвель с. 361). Несмотря на это, как увидим дальше,
сочинения Прокла пользовались уважением у средневековых философов.

\subsection{Вражда  неоплатонизма и  христианства в  известной степени
объясняется антагонизмом  между рационалистическим  и апокалиптическим
направлениями в религии}

В  чем  же  причина,  что  такие  идеологически  близкие  учения,  как
неоплатонизм  и христианство,  несмотря  на то  что  их близость  ясно
сознавалась многими  выдающимися отцами  церкви, находились  скорее во
враждебных,  чем  в  дружественных отношениях?  Конечно,  играли  роль
случайные обстоятельства,  например особенности  биографии трагической
фигуры  Юлиана.  Ужасная   практика  первых  христианских  императоров
(убийство  всех возможных  соперников нового  императора), от  которой
чуть не погиб Юлиан, естественно могла вызвать в нем антипатию к новой
религии.  И  в  неоплатонизме он  видел  преимущественно  религиозную,
а  не  философскую систему.  Поэтому  он  пытался возродить  языческие
обряды,  старался,  чтобы  язычники  по своему  моральному  уровню  не
уступали христианам,  но в  области философии и  науки он,  видимо, не
сделал ничего. Но существовало основное расхождение, которое пролагало
довольно   резкую  грань.   Неоплатонизм  остался   верен  политеизму,
многобожию, тогда как христианство  строго проводило иудейский принцип
единобожия. Демоны  у Плотина и других  --- благожелательные существа,
как у Сократа, христианские же мыслители скоро превратили их, а заодно
и всех греческих  богов, в злые существа. А так  как неоплатоники, как
и  христиане,  не отрицали  возможности  магического  искусства, то  и
получилась та  возможность обвинения  Сопатера, которая привела  к его
казни. В  самом обвинении не было  ничего нелепого, ни с  тючки зрения
язычников,  ни с  точки  зрения христиан,  а  поклонники демонов  были
противники христианского бога. Понятно поэтому, что те из христианских
мыслителей,  которые в  той  или иной  степени  освобождались от  веры
в  реальность  злых  могущественных  демонов,  хорошо  видели  стороны
неоплатонизма, родственные христианству.

Все это  отображение той борьбы между  различными формами религиозного
мышления, которая  хорошо известна всем  добросовестным исследователям
даже лично чуждым симпатий к  религии. Примитивная религия --- религия
страха  перед  грозными,   капризными  и  могущественными  существами.
Высокие  религии  уже  проникнуты   убеждением  в  благости  верховных
существ, для  них возникает проблема существования  зла. Это верховное
существо таинственным  образом может сообщить свою  волю, направленную
на  благо людей.  И наконец,  это  верховное существо  не капризно,  а
подчинено закону, им же данному. Великолепно об этом сказано у Рассела
(1959,  с. 55--56):  «Личная  религия ведет  свое  начало от  экстаза,
теология --- из математики; и то, и другое можно найти у Пифагора... Я
полагаю,  что математика  является  главным источником  веры в  вечную
и  точную  истину,  а   также  во  сверхчувственный,  интеллигибельный
мир...  Со  времени  Пифагора и  особенно  Платона  рационалистическая
религия,   являющаяся  противоположностью   апокалипсической  религии,
находилась под полным влиянием  математики и математического метода...
Начавшееся с  Пифагора сочетание математики и  теологии характерно для
религиозной философии  Греции, Средневековья  и Нового  времени вплоть
до  Канта. До  Пифагора  орфизм был  аналогичен азиатским  мистическим
религиям.  Но для  Платона, св.  Августина, Фомы  Аквинского, Декарта,
Спинозы  и Канта  характерно тесное  сочетание религии  и рассуждения,
морального  вдохновения и  логического  восхищения  тем, что  является
вневременным, ---  сочетание, которое начинается с  Пифагора и которое
отличает интеллектуализированную теологию Европы от более откровенного
мистицизма Азии. Только  в самое последнее время  стало возможным ясно
сказать, в чем состояла ошибка Пифагора. И я не знаю другого человека,
который был  бы столь влиятельным  в области мышления, как  Пифагор. Я
говорю так потому, что кажущееся платонизмом оказывается при ближайшем
анализе  в своей  сущности  пифагореизмом. С  Пифагора начинается  вся
концепция вечного мира, доступного интеллекту и недоступного чувствам.
Если бы не он, то христиане не учили бы о Христе, как о Слове; если бы
не  он, теологи  не искали  бы логических  доказательств бытия  бога и
бессмертия души. У Пифагора все это дано еще в скрытой форме».

Конечно,   противоположение   рационалистической  и   апокалипсической
(откровенной, т.е. основанной на  откровении) религии недостаточно для
отображения многообразия  религиозной мысли. Мне думается,  что Рассел
несколько  самоуверенно  утверждает,  что  весь  пифагореизм  является
ошибкой. Для нас важно признание огромной творческой роли пифагореизма
и той  связи религии,  философии и  науки, которую  принужден признать
Рассел.

\subsection{5) Название  «века мрака»  справедливо только  примерно до
1000 года и то в отношении европейской цивилизации. Недооценка Средних
веков Гегелем}

5)  Название   Средних  веков   «веками  мрака»  не   может  считаться
справедливым.   Конечно,  в   Западной   Европе   имел  место   упадок
экономического  уровня  страны,  внешнего  благоустройства  и  внешней
культуры.  Но даже  этот период  крайнего упадка  внешней цивилизации,
вызванный  нашествием  варваров,  касался  только  Западной  Европы  и
продолжался, как считает  Рассел, примерно 400 лет --- от  600 до 1000
г. В  это время  блестяще развивалась цивилизация  Китая и  ислама, но
потом западноевропейская католическая  цивилизация перегнала китайскую
и арабскую  цивилизацию, которые  впали в  состояние упадка.  Никто не
отрицает  огромной заслуги  арабов, сохранивших  многое из  эллинского
наследства и  развивших ряд  наук, но  почему арабская  культура впала
в  упадок,  а   западноевропейская  непрерывно  развивалась?  Наиболее
вероятным  ответом будет  тот, что  в Западной  Европе имелась  мощная
организация, почти независимая от  правительств, в которой происходила
весьма интенсивная, хотя внешне почти незаметная умственная работа ---
католические  монастыри. Роль  монастырей  в эту  самую мрачную  эпоху
подчеркивает и Б.Рассел  (1959, с. 414): «1000 год  удобно принять как
веху,  знаменующую  завершение  процесса упадка  цивилизации  Западной
Европы, которая к  этому времени достигла самой низшей  точки. С этого
момента началось  движение по  восходящей линии,  которое продолжалось
вплоть  до 1914  года. На  первых  порах прогресс  был обязан  главным
образом реформаторскому  движению, исходившему  из монастырей.  Что же
касается  духовенства,  стоявшего вне  монашеских  орденов,  то оно  в
большинстве своем  одичало, обмирщилось  и вело  безнравственный образ
жизни:  оно  было  развращено  богатством  и  властью,  которыми  было
обязано  пожертвованиям верной  паствы. Тому  же регрессу  непрестанно
подвергались даже и монашеские ордена,  но реформаторы с новым рвением
возрождали их моральную силу, как только она приходила в упадок».

Резкое изменение  понимания значения католической  философии выступает
всего  нагляднее, если  мы сравним  трехтомный курс  истории философии
яркого  идеалиста Гегеля  и  «Историю  Западной философии»  Б.Рассела,
воинствующего атеиста (что по  схеме «двух лагерей» является синонимом
материалиста). Гегель начинает свою  средневековую философию (кн. III,
с. 79)  следующими словами:  «Первый период охватывает  тысячелетие от
Фалеса,  585 лет  до P.  X., до  Прокла, умершего  в 485  г. по  P. X.
и  до  закрытия  учреждений  языческой  философии в  529  г.  по  P.X.
Второй  период доходит  до XVI  в. и,  таким образом,  в свою  очередь
охватывает тысячелетие, которое мы пролетим, надевши сапоги-скороходы.
Между  тем как  до сих  пор философия  развивалась в  недрах языческой
религии,  она  отныне  получает  место в  рамках  христианского  мира,
ибо,  что  касается  арабов  и  евреев,  то  мы  должны  их  коснуться
вкратце только  внешним образом, исторически». Всего  на средневековую
философию Гегелем  отведено 86 страниц,  из них на  схоластическую 58.
Рассел же  отводит на  католическую философию  190 страниц  (около 20%
книги),  хотя  его книга  по  объему  примерно  в  два раза  меньше  и
доведена до современности, тогда как Гегель заканчивает свое изложение
Шеллингом.  Изложение   Гегелем  средневековой   философии  неизмеримо
ниже  его  изложения  античной  философии,  так  как  при  пользовании
«сапогами-скороходами»  он  чрезвычайно  невнимательно  коснулся  ряда
важнейших  фигур.  Так, о  Николае  Кузанском  и Гроссетесте,  учителе
Р.Бэкона, даже не упоминается. О  Роджере Бэконе --- всего три строки,
причем он отнесен  к «мистикам», о Боэции ---  две строки. Подверглись
почти только  упоминанию и выдающиеся арабские  философы: Авиценне ---
три  строки,  Аверроэсу  ---  около шести  строк.  Ясно  поэтому,  что
тот, кто  хочет судить  о средневековой  философии по  Гегелю, получит
довольно  искаженное  о  ней  представление, и  дело  здесь,  конечно,
не  в «партийности»  философии,  а просто  в  том обстоятельстве,  что
Гегель по данному пункту  не был оригинален, разделял господствовавшее
тогда  мнение  о  малом  значении  средневековой  философии  и  вообще
культуры.  Конечно,   и  тогда   были  известны   факты  существования
поразительных фигур во время  Средневековья, но они казались какими-то
изолированными  точками, чудесными  явлениями гениев-одиночек.  Сейчас
совершенно ясно, что  эти поразительные фигуры ---  Боэций, Иоанн Скот
Эриугена,  Роджер  Бэкон  и  другие  были  только  вершинами  могучего
умственного движения,  развивавшегося в тиши монастырей.  Это движение
было вовсе  не однородно.  Любопытно, что многие  параллельные течения
развивались  в рамках  различных  религий:  христианской, иудейской  и
магометанской, но наибольшую силу оно получило в пределах католической
церкви.  В  частности,  аллегорическое толкование  Священного  Писания
(которое  многим  воинствующим  безбожникам кажется  уступкой  религии
торжествующему  материализму)  возникло  еще  у  древних  христианских
богословов и не прерывалось и в Средние века.

\subsection{6) В  сохранении античной  культуры огромную  роль сыграла
католическая церковь}

6) В  сохранении античной культуры огромную  роль сыграла католическая
церковь. Любопытно  признание Б.Рассела (1959, с.  391): «То немногое,
что уцелело  от культуры древнего  Рима в обстановке  всеобщего упадка
цивилизации, наступившего во время  нескончаемых войн VI и последующих
столетий,  было  сохранено в  первую  очередь  церковью. Но  роль  эту
церковь выполняла весьма несовершенно, ибо далее крупнейшие церковники
того времени  находились во  власти фанатизма  и суеверия,  и светское
знание пользовалось  дурной славой. Тем не  менее церковные учреждения
образовали прочный  остов, в  рамках которого  в более  поздний период
стало возможно  возрождение знания и цивилизованных  искусств». Рассел
указывает на особое значение бенедиктинского ордена, основанного около
520 г.  в Монте-Кассино,  прославившегося своей библиотекой  и научной
деятельностью,  хотя  эта научная  деятельность  и  не была,  конечно,
главной целью для основателя.

В XIX в. прогрессивно мыслящие люди привыкли считать, что церковь была
по  существу  реакционной  организацией  и что  в  борьбе  светской  и
духовной властей прогрессу способствовала  победа светской власти. Для
определенных  периодов и  в определенных  странах это,  может быть,  и
верно, но для Западной Европы  независимость церкви от светской власти
сыграла положительную роль.  «С большим трудом, начиная  с XI столетия
церкви удается освободиться от контроля феодальной аристократии, и это
освобождение является  одной из причин  выхода Европы из  веков мрака»
(Б.Рассел, 1959,  с. 319).  «Католическая философия по  своей сущности
является  философией   учреждения,  а  именно,   католической  церкви;
философия же нового  времени, даже в тех  своих разветвлениях, которые
далеки от ортодоксальности, имеет дело  в основном (особенно в этике и
политической  теории) с  проблемами,  ведущими  свое происхождение  от
христианских взглядов на нравственные законы и от католических доктрин
по вопросу о взаимоотношениях церкви и государства (Б.Рассел, 1959, с.
321).  «В  итоге  Возрождение  и  Реформация  разрушили  средневековый
синтез, который, однако, не был  заменен чем-либо столь же целостным и
производящим впечатление» (Б.Рассел, с.  320). По общей диалектической
схеме развития человеческой мысли  всякий синтез должен быть разрушен,
так как всякая  система, несмотря на свое  несовершенство, стремится к
догматизму и  застою. Было бы  смешно говорить сейчас о  возвращении к
средневековому  синтезу,  но  по  той же  диалектической  схеме  можно
ожидать,  что  новый синтез,  которого  сейчас  ждет человечество,  во
многом восстановит  средневековые ценности  и исправит ошибки  в общем
прогрессивных течений --- Возрождения и Реформации.

\subsection{7)  В средневековой  церкви  было  гораздо больше  свободы
мысли,   чем,  в   особенности,  после   возникновения  реформаторских
движений}

7)  В средневековой  католической церкви  было гораздо  больше свободы
мысли,  чем  в более  поздние  времена,  особенно после  возникновения
реформаторских   движений.  На   пороге  Средневековья   мы  встречаем
изумительную по  привлекательности фигуру Боэция, министра  и сенатора
короля Италии, остгота Теодориха,  казненного по подозрению в заговоре
в 524  г. Его книга «Об  утешении философией» написана им  в тюрьме, в
ожидании  казни. Как  пишет Рассел  (1959, с.  387), «Книгу  проникает
полнейшее философское  спокойствие --- такое безмятежное,  что если бы
она могла быть написана на вершине  благополучия, то мы могли бы почти
упрекнуть Боэция  в самодовольстве.  Но написанная  в тех  условиях, в
каких  она  действительно  была  создана,  ---  в  тюрьме,  человеком,
осужденным на смерть, --- книга Боэция так же прекрасна, как последние
минуты  Сократа». На  протяжении  всего Средневековья  Боэция чтили  и
считали  даже  мучеником  (пострадавшим  от  арианина  Теодориха),  но
вся  его  книга проникнута  платоновским  духом  и показывает  большее
влияние  языческой  философии,  чем христианской.  Книга  «открывается
утверждением,  что  Сократ,  Платон  и  Аристотель  ---  это  истинные
философы;  стоики же  и эпикурейцы  и прочие  --- узурпаторы,  которых
невежественная  толпа  ложно  принимает за  друзей  философии.  Боэций
заявляет, что он повиновался пифагорейской заповеди ``следовать Богу''
(заметьте, что  он не  говорит ---  христианской заповеди)...  За этим
следует пространное  изложение чисто платоновской  метафизики... Затем
он переходит  к пантеизму, который  должен был бы привести  христиан в
ужас,  но почему-то  никого не  ужасает... Те,  кто достигает  свойств
божеских,  становятся  богами.  Посему  каждый  счастливый  есть  Бог;
правда, по природе Бог только один существует, через сопричастие ничто
не  препятствует  быть  богами  очень многим  существам»  (Рассел,  с.
386). Рассел  прибавляет, что на  протяжении двух предыдущих  и десяти
последующих столетий он не знает  ни одного европейского ученого мужа,
который был бы в такой же мере свободен от суеверий, как Боэций.

В VIII в. Вергилий, епископ Зальцбургский, утверждал, что существуют и
иные миры, помимо  нашего. Это утверждение не  осталось незамеченным и
вызвало возражения святого Бонифация,  бывшего архиепископом с 732 г.,
но  не помешало  тому,  что и  Вергилий был  причислен  к лику  святых
(Рассел, 1959, с. 410).

Другой  изумительной фигурой  Средневековья был  Иоанн Скот  Эриугена,
или  Эригена (предпол.  годы  жизни 800--877).  Он был  неоплатоником,
пелагианцем,  пантеистом.  «Разум он  ставил  выше  веры, а  авторитет
церковников  ни  во что  не  ставил;  тем  не  менее они  сами,  чтобы
разрешить свои споры, обращались  к его авторитетному мнению» (Рассел,
1959, с. 421).

Иоанн настаивал, что разум и откровение --- это два источника истины и
потому не  могут противоречить  друг другу,  но если  иной раз  они по
видимости противоречат друг другу,  то предпочтение должно быть отдано
разуму. Иоанн  развивал и  представления о  творении как  процессе, не
имеющем  начала во  времени, о  необходимости понимать  иносказательно
историю  творения, рая  и  грехопадения. Следуя  Оригену, Иоанн  Скотт
отвергал  вечные мучения  и  принимал, что  даже  дьявол будет  спасен
(Рассел, с. 422--423). Вызывали  ли протест подобные суждения? Конечно
вызывали, и два раза при его жизни,  в 855 и 859 гг. они были осуждены
соборами без  вредных для  него последствий,  благодаря заступничеству
французского  короля,   Карла  Лысого.   Несмотря  на   осуждение  его
главного  труда «О  разделении природы»,  его перевод  псевдо-Дионисия
пользовался большим влиянием  и способствовал примирению неоплатонизма
с христианством.  Главный его труд продолжал  существовать и, несмотря
на приказ папы Гонория  III в 1225 г. (с опозданием  почти на 400 лет)
сжечь все экземпляры, сохранился до настоящего времени.

Борьба мнений  шла, иногда она  принимала весьма суровые формы,  но не
было того застоя мысли, который так часто приписывается Средневековью.
Как правильно указывает Рассел, ищущие мыслители нередко были осуждены
на бродяжничество,  переход из той  страны, где воцарился  фанатизм, в
другую, где  было больше  свободы. Иногда бежали  с востока  на запад,
иногда --- в противоположном направлении. Рассел справедливо отмечает,
что и  в нашем веке многие  мыслители принуждены были бежать  из своей
собственной страны.

Христианство Западной Европы не  осуждало всякую культуру иноверцев, и
в X  в. знаменитая магометанская  академия в Кордове имела  даже славу
дать западному христианскому миру  одного папу (Сильвестр II, Герберт,
умер  в 1003  г.), который  собственным примером,  своими сочинениями,
своим воспитанием  императоров и королей, больше  чем кто-либо, оказал
благотворное  влияние  на   культуру  тогдашней  христианской  Европы,
столько  нуждавшейся  в образовании  всякого  рода  (Уэвель, с.  298).
Герберт оставил след в развитии математики.

\subsection{8)  Для  развития  средневековой  мысли  характерна  смена
Платона Аристотелем, а затем протест против господства Аристотеля}

8)   Развитие  философской   мысли  в   Средневековье  можно   вкратце
охарактеризовать так, что  первоначальное господство Платона сменяется
гегемонией аристотелевских воззрений, а затем возникает протест против
господства  Аристотеля.  Оба  философа пользовались  уважениям  как  в
античном  мире,  так  и  в  Средневековье,  но  в  более  ранние  годы
первое  место  занимал  Платон  и   лишь  в  XIII  в.  Фома  Аквинский
(1225--1274)   создал  богословско-философскую   систему  ---   синтез
Аристотеля и  христианства с  несомненной примесью  неоплатонизма. Эта
система и  сейчас лежит  в основе католической  философии в  форме так
называемого  неотомизма, хотя  имеет  и менее  выраженное течение  ---
неоавгустианство, более склонное к Платону.

Фома Аквинат был, несомненно, выдающимся  мыслителем, и в его работах,
как  и в  трудах Аристотеля,  к которому  он философски  очень близок,
заключается  очень много  ценного. Но  его система,  как и  все учение
Аристотеля, принесла  и большой  вред, послужив тормозом  для развития
свободной  науки.  Несомненно,  всякий  крупный  синтез  производит  и
тормозящее действие, вопреки даже желаниям его создателей. В случае же
Аристотеля и Фомы Аквината к этому примешивался еще ряд обстоятельств:
1) религиозная акция  нового синтеза; 2) то  обстоятельство, что труды
Платона  в  раннее  Средневековье  были  известны  лишь  из  вторых  и
третьих  рук и  оригиналы пришли  позже из  Византии и  от арабов;  3)
влияние арабской философии, в  очень сильной степени проникнутой духом
Аристотеля;  4) то  обстоятельство,  что как  было  уже указано  выше,
философия Аристотеля  была чужда стремлению математизировать  науку, а
без математизации  даже крупные  ученые склонны  удовлетворяться более
или  менее сносными  «объяснениями», убаюкивающими  критическую мысль.
Постепенно сторонники  Аристотеля заняли руководящие посты  в науке, и
это  было главной  причиной  той  ожесточенной идеологической  борьбы,
которая завязалась во время Возрождения.

Но  современником  Фомы Аквината  был  один  из ревностных  защитников
математизации  науки,  сторонник  опытного  исследования  францисканец
Роджер Бэкон  (1214--1294). Как  уже было  указано, Гегель  в «Истории
философии» (кн.  III, с.  152) уделяет ему  всего три  строки: «Роджер
Бэкон  работал,  главным образом,  в  области  физики, но  его  работы
остались без  влияния; он  изобрел порох,  зеркало, телескопы,  умер в
1294 г.». Отнес его Гегель и к мистикам. Несомненен большой интерес Р.
Бэкона  к  опытному естествознанию  и  технике.  Он предвосхитил  идею
построения  летательных машин,  кораблей без  парусов, утверждал,  что
свет  распространяется не  мгновенно. Он  отдал дань  и астрологии,  и
алхимии (История  философии, 1941, с. 472--473).  Его взглядам склонны
приписывать  ярко   выраженную  материалистическую   тенденцию  (Крат,
философ, словарь, 1955, с. 52), а заключение его в тюрьму --- карой за
его передовые воззрения. Но Р.Бэкон  был не только научным мыслителем,
но и политическим. Бэкон смело обличал феодальные притеснения, грабежи
и пороки  духовенства и  ставил программу: «Нужно,  чтобы справедливый
папа  со справедливым  государем, меч  материальный с  мечом духовным,
очистили церковь»  (История философии, 1941,  с. 471). И  ему пришлось
действительно иметь  дело со  справедливыми и  несправедливыми палами.
Когда  он  сделал папе  предложение  о  преобразовании клира,  то  был
посажен  в  тюрьму.  Следующий  папа, Климент  IV,  который  знал  его
лично  прежде, когда  он  был  еще кардиналом,  освободил  его, и  под
его покровительством  он написал  свое знаменитое  произведение: «Опус
майус». Но  при следующем папе,  Николае III,  он был снова  посажен в
тюрьму  и,  просидев десять  лет,  был  освобожден лишь  незадолго  до
смерти, прожив  далеко не короткую жизнь  --- 80 лет (больше,  чем его
современник Фома) (Уэвель, с. 330--331).

Р.Бэкон  был учеником  выдающегося математика  епископа линкольнского,
Р.Гроссетеста,  и сам  придавал  огромное значение  математике. В  его
главном  сочинении четвертая  часть  говорит о  пользе математики:  1.
необходимость  математики  в   человеческих  вещах;  2.  необходимость
математики в  божественных вещах: 1) этим  изучением занимались святые
люди, 2) география, 3) хронология, 4) циклы: золотое число и проч., 5)
естественные  явления,  как  радуга,  6)  арифметика,  7)  музыка;  3.
необходимость математики в церковных вещах: 1) удовлетворение веры, 2)
исправление календаря; 4. необходимость математики в государстве: 1) о
климатах, 2) гидрография, 3) география,  4) астрология (Уэвель, с. 431
-432). Мы видим, что не забыта и астрология.

\subsection{Две   черты   Р.Бэкона:  сознание   бесконечности   науки,
необходимость математизации. Он  главным образом против перипатетиков,
чем против самого Аристотеля}

У Р.Бэкона совмещаются две тенденции, характеризующие каждого крупного
ученого. Во-первых, смелая борьба  с авторитетами, во-вторых, сознание
бесконечного объема  науки, истинно сократовское смирение:  я знаю то,
что я ничего не знаю. «Человек  в этой жизни не способен к совершенной
мудрости; ему  трудно возвышаться  к совершенству и  легко скатываться
вниз  к заблуждениям  и суетности:  пусть  же он  не хвастается  своей
мудростью и  не превозносит своего  знания. То,  что он знает,  мало и
ничтожно в сравнении с тем, во что он верит без знания; и еще меньше в
сравнении с  тем, чего он не  знает. Тот безумен, кто  высоко думает о
своей мудрости; еще более безумен тот, кто выставляет эту мудрость как
нечто удивительное» (Уэвель, с. 436--437).

Неудивительно, что при таком умонастроении Р.Бэкон оказывается ближе к
Платону, чем к Аристотелю, хотя в то время подлинные сочинения Платона
на Западе,  видимо, были неизвестны.  Платонизм до него  доходил через
Августина, псевдо-Дионисия  и других  писателей. Как пишут  в «Истории
философии» (с. 471), Р.Бэкон  в метафизических представлениях «идет на
поводу августиновского  платонизма». Как  один из аргументов  в пользу
математики он использует чисто платоновское мнение, что математическое
знание врождено нам, причем он  ссылается на известный диалог Платона,
приводимый Цицероном  (Уэвель, с. 439). Большой  защитник индуктивного
опыта  и  Фр. Бэкона,  Уэвель  принужден  признать  (с. 437):  «Он  не
настаивал  исключительно  на  опыте,  с  сравнительным  пренебрежением
к  существующим  наукам и  представлениям,  ---  ошибка, которую  есть
некоторое  основание приписать  его  великому  соименнику и  преемнику
Фрэнсису Бэкону: с  другой стороны, он вовсе  не довольствовался одним
протестом  против школьного  авторитета  и неопределенным  требованием
перемены, --- как этим почти и ограничивается все то, что делали люди,
являвшиеся  реформаторами в  это промежуточное  время». У  нас нередко
сближают двух Бэконов. Сходство их, пожалуй, только в одном: в критике
системы  Аристотеля,  но  критиковали  они эту  систему  совершенно  с
противоположных точек  зрения, и в истории  человеческой мысли Р.Бэкон
неизмеримо выше Фрэнсиса, жившего на слишком 300 лет позже него.

Но  каково было  отношение Р.Бэкона  к Аристотелю?  Правда ли,  что он
готов был  сжечь все книги  Аристотеля и  что он считал,  что изучение
Аристотеля  увеличивает невежество  (об  этом в  современной книге  по
основам таксономии Г.Г.Симпсона,  1961, с. 96). Как часто  бывает, и в
данном случае в этих  высказываниях имеет место сильное преувеличение.
Р.Бэкон  протестует  против  ореола непогрешимости,  которым  окружили
Аристотеля, но  в своей истории  философии отзывается об  Аристотеле с
большой  похвалой  и  говорит,  что  «Аристотель  средствами,  которым
научает мудрость, мог дать  Александру владычество над миром» (Уэвель,
с.  436). Но  он  очень  резко отзывается  о  качестве  ходивших в  то
время  переводов Аристотеля  и  об этих  переводах  он говорит:  «Если
бы  у меня  была  власть  над сочинениями  Аристотеля,  я  бы сжег  их
все;  потому  что  изучать  их  есть  только  потеря  времени,  и  ряд
заблуждений,  и умножение  невежества,  какого  нельзя и  представить»
(Уэвель, с.  434). Ходячие цитаты  из Р.Бэкона точны, но  неверны, так
как являются  образцом «препаровки».  Как увидим дальше,  именно такое
критическое, но отнюдь не нигилистическое, отношение к Аристотелю ясно
выражено  и у  величайших ученых  Возрождения, которые  больше спорили
с  перипатетиками, не  по  разуму  усердными сторонниками  Аристотеля,
возведшими его в непререкаемый авторитет.

Р.Бэкон  в  процессе  своих  астрологических занятий,  за  которые  он
заслужил  репутацию   мага  и  волшебника,  сделал   некоторые  ценные
астрономические и  химические наблюдения (История философии,  1941, с.
473),  но  его  общие  воззрения не  имели  непосредственного  успеха.
Победили перипатетики, и это нельзя  связывать с католицизмом, так как
в  арабской  и еврейской  философии  того  времени Аристотель  занимал
даже более  доминирующее положение,  чем в  католической. Аристотелева
философия более приспособлена для  всякого догматического учения, пуще
всего боящегося «ревизионизма» и нашедшего окончательные, «абсолютные»
истины.  В  трогательном  единении   с  католической  и  мусульманской
церквами  находятся  и  современные догматические  марксисты,  которые
усиленно переводят Аристотеля и всячески замалчивают Платона.

\subsection{9)  Споры  защитников  Платона и  Аристотеля  продолжались
и  в  арабской  и  в   иудейской  науке.  Господство  перипатетизма  у
арабов сопровождалось развитием птолемеевской, платоновской астрономии
(Улугбек)}

9)  Мы  видим,  что   в  пределах  католической  идеологии  фактически
не  прекращались  споры  между  представителями  философий  Платона  и
Аристотеля,  не   говоря  уже   о  других   философских  разногласиях.
Совершенно параллельные течения имели место и в средневековых арабской
и  еврейской философиях.  Астрономы всех  трех вероисповеданий  иногда
работали  совместно,  например  во времена  короля  Испании  Альфонса,
собравшего  многочисленных астрономов,  составивших (на  основе теории
Птолемея) астрономические таблицы, названные Альфонсовыми. В философии
и  в  астрономии  арабы  сохранили  многое  из  античного  наследства,
прежде  всего  «магисте  синтаксис» Птолемея,  известного  обычно  под
арабским  искаженным  названием  «Альмагеста».  Но  арабские  наиболее
известные  философы  склонялись  к  Аристотелю еще  больше,  чем  Фома
Аквинат. Крупнейший арабский  философ Ибн-Рушд (1126--1198), известный
больше под именем Аверроэса  из Кордовы, как последователь Аристотеля,
отрицал какое  бы то ни  было значение за построениями  Птолемея, звал
назад  к Аристотелю,  к  системе концентрических  сфер. Аверроизм  как
философское течение  с материалистическими  тенденциями преследовалось
как мусульманской  религией, так и христианской,  но его последователи
были  и  в Италии:  школа  в  Падуе, где  как  раз  одно время  учился
Коперник (Идельсон  1947, с.  14). Аверроэс имел  большое влияние  и в
Парижском  университете в  XIII в.  Одним из  виднейших представителей
аверроизма в  Европе был  Сигер Брабантский (умер  прибл. в  1282 г.),
кончивший  плохо:  или  убит,  или  умер  в  тюрьме  (БСЭ,  2-е  изд.,
т.  38,  с.  664).  Сам  Ибн-Рушд после  занятия  ряда  должностей,  в
частности придворного  врача эмиров,  Юсуфа и др.,  умер в  изгнании в
Марокко (БСЭ,  2-е изд.,  т. 17,  с. 256).  Несмотря на  свое огромное
уважение  к  Аристотелю,  Аверроэс,  как известно,  вместе  со  своими
последователями  развивает преимущественно  материалистические стороны
учения Аристотеля:  отрицание божественного творения  мира, бессмертия
души,  движение столь  же вечно  и неуничтожимо,  как и  первоматерия.
Аверроэс  отрицал притязания  господствовавших  в  те времена  алхимии
и  астрологии  (Бернал, 1956,  с.  162).  Наконец, Ибн-Рушд  был  если
не  основоположником,  то  одним  из  виднейших  защитников  учения  о
«двойственной истине», которого придерживались также Дуне Скот, Оккам,
Ф.  Бэкон и  др. (БСЭ,  2-е  изд., т.  13,  с. 490).  По этому  учению
наука  и религия  в  конечном счете  должны приводить  к  одним и  тем
же  высшим  истинам,  но  могут  расходиться  между  собой  в  решении
конкретных  проблем. Поэтому  философия и  религия, имея  ограниченные
сферы применения, не должны вмешиваться в компетенцию одна другой. Это
учение сыграло большую роль в  прогрессе науки, но самого Ибн-Рушда не
уберегло от преследований.

Реализм Аверроэса, однако, сыграл  вредную роль в развитии космологии,
так как заставлял его отвергать систему Птолемея как нереалистическую:
«Астрономия  Птолемея  ничтожна  в  отношении  существующего,  но  она
удобна, чтобы вычислять то, что  не существует». Искание новой системы
имело место,  что выражено, например, в  словах выдающегося еврейского
философа Моисея Маймонида из Кордовы: «Посмотри как все это темно, ---
писал он в своем ``Путеводителе  заблудших''. --- Если истинно все то,
что утверждает  Аристотель в  науке физической,  то ни  эксцентров, ни
эпициклов существовать  не может,  и все  обращается вокруг  Земли; но
откуда же появляются эти сложные движения планет?» (Идельсон, 1947, с.
20).  Система  Аристотеля  была  тупиком для  развития  космологии.  В
еще большей  степени таким  тупиком являлись  чисто материалистические
представления Лукреция, о чем было выше, но они в Средние века роли не
играли.

Арабская (правильнее  сказать, мусульманская) космология  и астрономия
развивались не на главной аристотелевской линии, а на линии Птолемея и
Платона. Наиболее  выдающимся представителем  мусульманской астрономии
был  знаменитый внук  Тамерлана Улугбек  (1394- 1449),  слава которого
гремела далеко  за пределами Средней Азии.  Идеологическая обстановка,
окружавшая Улугбека, ясна из  того, что величайшей похвалой выдающимся
ученым  того времени  были:  «Платон своей  эпохи»  и «Птолемей  своей
эпохи» (Голубев, 1960, с. 76 и 87).

Кроме  аристотелевской   системы,  господствовавшей  в   философии,  и
платонизма,   господствовавшего  в   астрономии,   в  арабской   науке
были  зачатки  и  других   систем,  например,  Альгазеля  из  Багдада,
предвосхищавшего философию  Юма (Уэвель,  с. 322), но  целью Альгазеля
было  «разрушить все  системы  рациональной теологии  для того,  чтобы
открыть свободную дорогу не  только для веры, руководимой откровением,
но  также для  полного господства  мистического энтузиазма»  (там же).
Аверроэс  резко  выступал   против  него,  но  для   науки  этот  спор
представляет малый интерес.

\subsection{Остановка   магометанской  цивилизации   после  блестящего
расцвета на огромной территории}

Магометане  создали   в  первую  очередь  репутацию   Аристотеля.  Они
занимались  многими науками  и  ввели в  интернациональный язык  много
новых  терминов:  алгебра,  алкалоид,  алхимия,  куб,  азимут,  зенит,
алгоритм  и  т.д.  В  астрономии они  подвинули  технику  измерений  и
создали наилучший  для своего времени звездный  каталог. Они сохранили
и   передали  впоследствии   европейцам  многие   сокровища  эллинской
цивилизации,  но в  космологической  теории они  или остались  верными
Птолемею, или  предлагали вернуться к Аристотелю.  Указывают на Бируни
(973--1048),  как на  одного из  предшественников Коперника,  но кроме
слабых  намеков в  этом направлении  дело  не двинулось.  И это  очень
удивительно, так  как ареал  и разнообразие  мусульманской цивилизации
были поистине грандиозны. Уэвель на с. 295--299 дает краткую, но яркую
характеристику пышного расцвета арабской  культуры. Первые халифы были
заняты  военными делами  и ничего  не дали  для науки  и культуры:  им
приписывали только (в большей своей части несправедливо) окончательную
гибель  знаменитой Александрийской  библиотеки. Но  уже при  Аббасидах
(с  750  г.) начинается  процветание  той  отрасли культуры,  с  какой
обычно  культура  начинает  свое  развитие  ---  литературы.  А  затем
развились  многие  центры  блестящей цивилизации:  Багдад  (Альмансор,
Гарун аль-Рашид и др.), Бухара, Самарканд. Расцвет наук и искусств был
и в Персии (с VIII столетия),  Египте, Северной Африке и в особенности
в  Испании  при  Омейядах  (755--1038) и  в  особенности  при  халифах
Абдаррахмане III и Гакеме II. Как  указывает Уэвель: «В это время, ---
а  не после,  как  думают  многие, при  Фердинанде  и Изабелле,  когда
была  открыта  Америка,  ---  в  это первое  время  Испания  именно  и
достигла своего  действительно золотого  века и высшей  степени своего
процветания.  Тогда,  согреваемая  арабским  огнем,  Испания  богатыми
струями  проливала свой  духовный свет  на всю  остальную Европу,  где
господствовала  темная  ночь варварства,  и  даже  на далекий  восток,
откуда этот  свет явился в первый  раз... Быть может, никогда  наука и
всякое развитие человеческого  ума не ценились и  не уважались больше,
чем  при дворе  Гакема II,  и слава  его Академии  в Кордове  оставила
далеко за  собой славу  давно упавшей школы  Александрии и  даже славу
академий, незадолго перед тем основанных Гаруном и Маммуном в Багдаде,
Куфе,  Бассоре и  т.д. И  ни в  какое другое  время Испания  не видела
большего умственного развития, не была  богаче и счастливее, и никогда
не были  в лучшем состоянии даже  финансы, управление, промышленность,
внутренняя и внешняя торговля, земледелие и даже пути сообщения, как в
это  блестящее  время».  Образование  в  Кордове  получали,  как  было
указано, и христиане,  и воспитанником Кордовы был  и Герберт, будущий
папа Сильвестр II. Школы, ученые учреждения и богатые библиотеки были,
кроме Кордовы, в Гренаде, Толедо, Севилье, Валенсии, Мурсии, Альмерии,
Малаге  и др.  Гакем собрал  библиотеку  в 600  тысяч рукописей,  один
каталог  которых занимал  44 тома.  От ученых  он не  требовал ничего,
кроме окончания начатых ими произведений,  и старался доставить им для
этого все средства и необходимый досуг.

Но  мусульманская   государственность  и  культура   захватила,  кроме
перечисленных  стран, и  такие страны,  как Индия.  Покорители России,
татары,   тоже  приняли   ислам.  Было   время,  когда   магометанство
торжествовало  на  огромной   территории  и  если  не   везде,  то  во
многих местах  оставило следы  великолепной культуры, в  лучшем случае
остановившейся, а  как правило,  угасшей и с  трудом восстанавливаемой
теперь напряженной работой историков и археологов.

\subsection{Разбор  объяснений  остановки   или  упадка  мусульманской
цивилизации}

В  чем  же  причина  этого  подлинного  декаданса  арабской  культуры,
дошедшего до  того, что ряд  открытий, сделанных арабами,  потом снова
переоткрывались? Разумеется, в каждом  отдельном случае можно найти те
или  иные причины:  заговор  реакционных кругов,  внешнее нашествие  и
т.  д.,  но  если  бы  такие причины  были  решающими,  то  надо  было
бы  ожидать, что  неустойчивой  окажется западноевропейская  культура,
где  пространство  было  более  ограниченным,  шли  непрерывные  войны
между  государствами, а  с  востока шли  непрерывные орды  кочевников.
Поэтому это объяснение до  известной степени объясняет упадок культуры
Древней Руси,  но и то  не вполне:  добавили потом московские  ханы. В
мусульманском  же  мире  господствовала долгое  время  полная  внешняя
безопасность,  благодаря воинственности  ислама  и достаточно  высокой
государственной организации.

Как уже было показано выше, цивилизация христианского и арабского мира
была в значительной  степени общей и был  несомненный обмен культурой.
«С позиции одной только науки  было бы логичным рассматривать период с
IX  по XIV  век  как объединенное  арабско-романское усилие  примирить
религию  и философию  и завершить  классическую картину  мира. Но  это
означало бы пренебрежение географическими и экономическими различиями,
которые должны были обусловить  решающее различие в последствиях этого
предприятия.  В то  время как  в мусульманских  странах был  достигнут
компромисс,  делавший  прогресс  науки  бесплодным,  у  христиан  спор
продолжался  до тех  пор,  пока под  влиянием экономических  изменений
греческая  картина  мира  не   была  полностью  разрушена  и  заменена
другой»  (Бернал,  1956,  с.   177).  Бернал  признает  разную  судьбу
магометанской  и  христианской  культуры,  но пытается  это  свести  к
географическим  и  экономическим  различиям. Я  не  представляю  себе,
какие общие  географические свойства  характеризуют, с  одной стороны,
христианскую, с другой --- мусульманскую цивилизацию. Почему в Испании
при  христианах  не пришло  столь  же  высокой испанской  культуры,  а
в  Индии,  где  домусульманская  культура  достигла  высокого  уровня,
мусульманская культура  тоже остановилась? Что  касается экономических
причин,  то   это  объяснение   уж  совсем  странно.   Ведь  экономика
создается   деятельностью   людей,  создающих   ценности,   улучшающих
производительность труда  и ломающих  тот общественный  строй, который
мешает  развитию  производительных  сил.  Почему  этого  не  случалось
(несмотря на исключительное обилие всевозможных дворцовых переворотов)
в  мусульманских  странах?  Люди  принципиально  другие?  Может  быть,
справедлива  расовая  теория  развития  культуры  в  ее  разнообразных
формах? Для  полноты картины  коснемся и этого  объяснения. Совершенно
правильно  говорят,  что  нельзя  говорить  об  арабской  культуре,  а
следует говорить о  культуре ислама. Так как  магометане принадлежат к
самым  разнообразным расам  и племенам,  магометанская церковь  широко
открывает  двери   для  вступления  в  нее   иноземцам  с  соблюдением
совершенно ничтожных  формальностей. Лица  христианского происхождения
(ренегаты) или  от христианских родителей заполняли  гвардейские части
(янычары,  мамелюки),   достигали  нередко   высших  правительственных
должностей.  В  смысле  племенном  мусульманские  страны  представляют
большее разнообразие,  чем западноевропейские, у них  больше материала
для отбора «лучших».  Нисколько не убедительнее и  воззрения Р. Фишера
на гибель культур в результате, так сказать, обеднения генофонда из-за
малой плодовитости  элиты. Почти все культуры  дохристианского периода
были связаны с многоженством и  в народах ислама это практиковалось до
наших дней. «Элита» размножалась несравненно интенсивнее массы, но это
не остановило гибели культуры. Напротив, тот процесс, которым Р. Фишер
объясняет падение культуры, меньшая  плодовитость элиты, свойствен как
раз  христианским  странам,  где  культура  прогрессирует  уже  второе
тысячелетие и признаков общего декаданса мы не наблюдаем.

Может быть,  сказывается то обстоятельство, что  народы мусульманского
мира  «никогда не  пользовались,  как греки,  тем сознанием  личности,
независимостью  воли,  умственной   свободой,  которые  происходят  из
свободы  политических  учреждений.  Они не  чувствовали  заразительной
умственной  деятельности небольшого  города;  того одушевления,  какое
является  из   общего  сочувствия  понимающей  и   умной  аудитории  к
философскому  умозрению;  словом, у  них  не  было того  национального
воспитания, какое  делало бы  их способными  быть учениками  Платона и
Гиппарха. Поэтому  их новое литературное богатство  скорее подавляло и
порабощало  их, чем  обогащало  и усиливало,  при  недостатке любви  к
умственной  свободе  они  довольствовались  тем,  что  отдавались  под
руководство Аристотеля и других  догматиков. Их военные нравы приучали
их искать предводителя;  их уважение к своей  книге закона приготовило
их  и  к принятию  философского  Корана»  (Уэвель,  1867, с.  349).  В
размышлениях  Уэвеля много  верного и,  конечно, обстановка  небольших
государств  неоднократно  способствовала  пышному  развитию  подлинной
культуры, но признать этот фактор  главным невозможно. И в Александрии
и  в Сиракузах  не было  политической свободы,  однако главные  высоты
эллинской  цивилизации  были  достигнуты  именно  в  этих  городах.  И
конечно,  нельзя сказать,  чтобы  арабы были  просто комментаторами  и
передатчиками. В  ряде областей они были  инициаторами и оригинальными
творцами: алгебра, алхимия, оптика,  приведшая впервые к созданию линз
(Бернал, 1956,  с. 165) ---  прототипа телескопа, микроскопа  и прочих
оптических инструментов.  Почему все  это остановилось? И  многое даже
оказалось позабытым.

\subsection{Главная  причина:  у  магометан ---  единство  светской  и
духовной  власти, у  христиан (католиков)  --- независимость  духовной
власти от светской}

Но   при  всем   сходстве  в   философских  источниках,   при  наличии
взаимосвязи между  христианской и магометанской наукой  имеется весьма
существенное  отличие,   хорошо  подмеченное  Берналом  (с.   179):  в
христианском  мире «даже  самые  небольшие  научные исследования  того
времени предпринимались  почти исключительно  для религиозных  целей и
представителями  духовенства ---  священниками,  монахами или  членами
какого-либо  ордена.  В этом  отношении  условия  ее развития  заметно
отличаются от. условий  развития мусульманской науки, где  мало кто из
ученых  имел религиозное  призвание,  а большинство  руководствовалось
откровенно  утилитарными  целями». Как  он  же  указывает на  с.  162,
кроме  многочисленных  халифов,   оказывавших  науке  покровительство,
непревзойденное  со времени  основания Александрийского  музея, многие
богатые купцы и сановники оказывали покровительство науке, а некоторые
и  сами  проявляли  интерес   к  науке.  Это  покровительство  спасало
ученых  от  гнева  религиозных  фанатиков.  Но  затем,  очевидно,  это
покровительство  оказалось  недостаточно  эффективным  или  исчезло...
«Полная неудача  попыток примирить  науку с  устойчивыми особенностями
мусульманской религии была, очевидно,  главной причиной увядания науки
в  последние  века  существования   ислама,  который  в  культурном  и
интеллектуальном отношении переживал застой» (Бернал, с. 163).

В католическом мире церковь, по крайней  мере до начала XIII в., имела
монополию  на  ученость  и  даже  на  грамотность  (Бернал,  с.  174).
Университеты,  развившиеся  из  соборных школ,  были  главным  образом
учреждениями по  подготовке духовенства и скопированы  с мусульманских
медресе. Вот список первых университетов, большинство которых и сейчас
пользуются славой как центры учености  (Бернал, с. 176): Париж (осн. в
1160  г.),  Оксфорд (1167),  Кембридж  (1209),  Падуя (1222),  Неаполь
(1224), Саламанка  (1227), Прага  (1347), Краков (1364),  Вена (1367),
Сент-Эндрьюс  (1410).  Как  известно,  квадривиум  этих  университетов
(арифметика, геометрия, астрономия и музыка) был намечен еще Платоном.
Желание  дать духовенству  общее  образование и  привело  к тому,  что
образование  в  основном было  не  только  светским,  но и  научным  и
притом построенным по мусульманскому  образцу (Бернал, с. 176). Бернал
отмечает,  что  ни  история,  ни  литература не  нашли  себе  места  в
университетах и  что именно это  упущение должно было вызвать  в эпоху
Возрождения реакцию гуманистов против  всей схоластической системы. Он
же указывает, что «с миром природы или практическими ремеслами контакт
был весьма невелик,  так же как и  интерес к нему, но  по крайней мере
любовь к знанию, к спору поощрялась».

История  поставила  с  культурой любопытный  эксперимент:  два  мощных
комплекса  ---   мусульманский  и  христианский  мир,   имевшие  общих
культурных  предшественников  и  достигавшие  высокого  экономического
и  культурного  развития,  но   с  весьма  существенным  различием:  в
мусульманском  мире   ---  независимость   науки  от   духовенства,  в
христианском ---  монополия науки  духовенством; в  мусульманском мире
--- дух  практицизма, в христианском ---  оторванность от практических
целей (по крайней мере той  науки, которая изучалась в университетах);
в  мусульманском   мире  ---  огромное   покровительство  просвещенных
монархов, в  христианстве ---  просвещенных монархов  и поддерживаемых
ими  высококультурных учреждений  мы почти  не знаем;  в мусульманском
мире  ---   единство  светской  и  духовной   власти,  в  христианском
---  борьба  светской  и  духовной  властей.  А  результат:  остановка
и  регресс  блестящей  культуры   ислама,  подготовка  в  христианском
мире  замечательной культуры  Возрождения. Бернал  соглашается с  тем,
что  вклад  средневекового  христианства  в  науку  был,  быть  может,
несправедливо забыт в прошлом, но что сегодня опасность состоит скорее
в  преувеличении его  значения до  такой степени,  что это  мешает нам
разобраться в  истории науки вообще  (с. 181). Он склонен  думать, что
средневековая наука  в целом должна рассматриваться  скорее как конец,
чем начало интеллектуального движения.

Я не знаю, кого имеет в виду Бернал (с. 179), когда он говорит о моде,
принятой рядом современных ученых,  превозносить науку Средневековья в
ущерб науке  Возрождения. Правильный  вывод, по-моему, будет  тот, что
средневековые мыслители, в основном  монахи и духовенство, подготовили
ту почву, на которой  при достижении западноевропейскими государствами
экономической  зрелости  могло  пышно расцвести  Возрождение,  но  при
таком стремительном  взлете, как  всегда, наряду с  приобретением ряда
культурных ценностей, другие ценности утрачивались.

Прав  ли  Бернал, утверждая,  что  новая  «мода» «искажая  факты,  ...
особенно  несправедлива  в   отношении  средневекового  духовенства  и
схоластов,  приписывая им  то, чего  они не  делали, и  затушевывая их
действительный вклад в науку» (с. 179). Бернал отдает должное (с. 178)
таланту и оригинальности мышления Фомы  Аквината, но там же указывает,
что  Фома  вовсе  не  был  единственным  учителем  схоластики.  Бернал
считает, что не схоласты создали  современную науку, а такие люди, как
Леонардо,  Бэкон (имеется  в  виду, очевидно,  Фр.  Бэкон) и  Галилей,
непримиримо  отвергавшие  их цели  и  методы,  что устранение  вековых
наслоений  нелепостей  было самой  трудной  и  утомительной задачей  в
создании  науки. «Когда  мы  подумаем, что  потребовалась  чуть ли  не
тысяча лет для  того, чтобы создать то богатство идей,  на которое, не
будь  этих препятствий,  ушло бы,  быть  может, всего  двести лет,  мы
меньше  будем  благоговеть перед  теми,  кто  своими доктринами  столь
действенно задержал прогресс науки»  (с. 178). Эти соображения полезно
подвергнуть критическому  пересмотру, так  как они весьма  типичны для
представителей  «демокритовской  линии»  и принадлежат  перу  крупного
ученого-марксиста.

\subsection{Связь науки с католической церковью в Средние века была не
препятствием развития науки, а сильно содействовала прогрессу науки}

Предположение   Бернала,  что,   не   будь   препятствий  со   стороны
схоластиков,  богатство  идей  современной   науки  было  бы  создано,
может  быть,  за двести  лет,  а  не  за тысячу,  является  совершенно
произвольным. Бернал,  видимо, забывает,  что очень многие  явления, в
том числе  высокая культура, развиваются по  экспоненциальному закону,
т.е.  сначала  развитие  идет   медленно,  почти  незаметно,  а  затем
темп развития  ускоряется, если,  конечно, не  произойдет политическая
или  экономическая  катастрофа  или  не  возникнут  какие-либо  другие
непреодолимые  препятствия. С  другой  стороны,  пусть Бернал  укажет,
какая  из многочисленных  древних  или более  новых цивилизаций  имела
столь долгий,  в общем  непрерывный период  развития и  достигла столь
высокого уровня, как именно христианская цивилизация?

Нельзя   отрицать    тормозящего   влияния   всякой    законченной   и
последовательной  системы,  но  Бернал   сам  указывает,  что  система
Фомы  никогда   не  была  единственной   системой  и  вовсе   не  была
чисто   христианской.  В   рамках  христианства   продолжали  бороться
аристотелевская  и  платоновская  традиции, и  сам  Бернал  указывает,
что  в  период Возрождения  пошла  вверх  платоновская линия  в  ущерб
аристотелевской. Поэтому нельзя говорить, и  это мы увидим дальше, что
Возрождение целиком отвергло схоластические методы.

Очень  странно  утверждение  Бернала, что  современную  науку  создали
Леонардо, Бэкон (Фрэнсис)  и Галилей. Леонардо да  Винчи был, конечно,
гениальный человек,  но в основном  он был художник, а  затем инженер.
Многие же его  блестящие научные мысли, в которых  он предвосхитил ряд
современных  теорий,  как  известно,  были открыты  много  лет  спустя
после его  смерти, после  того как они  были открыты  другими учеными.
Непосредственное же  влияние на науку  у Леонардо было  весьма слабым.
Роль Фрэнсиса  Бэкона несомненно преувеличена  (как это ясно  из самих
слов  Бернала  в  других  местах),  так  как  он  не  был  сторонником
математизации  науки  (наиболее  прогрессивное  направление),  отрицал
гелиоцентрическую систему Коперника и в дальнейшем его влияние привело
к чрезмерному эмпиризму в науке. Любопытно, что в этом месте Бернал не
упоминает ни Коперника, ни Кеплера.  Что касается до Галилея, то этого
придется  коснуться подробнее  в  своем месте.  Роль  его огромна,  но
совсем не такова, как ее обычно изображают популяризаторы.

Что же касается того, что современная «мода» приписывает средневековым
ученым  в   заслугу  то,  что   они  не  делали,  и   затушевывает  их
действительный  вклад  в  науку,  то истинный  смысл  этого  выражения
чрезвычайно своеобразен. В сущности, Бернал не отрицает ни таланта, ни
смелости мысли ряда средневековых ученых, но  ставит им в вину то, что
они  свою  самоотверженную  работу  вели,  как  правило,  в  свободное
время  и в  теологических  целях. А.  так  как по  материалистическому
догмату  Бернала наука  принципиально  противна  религии, то,  значит,
если  они даже  достигли крупных  научных результатов,  но руководимые
религиозной  идеологией, их  научные  заслуги  являются мнимыми.  Этот
ход  рассуждений  чрезвычайно  напоминает мысли  крайних  христианских
фанатиков, которые  говорили, что, например, язычники  не могут вообще
иметь добродетелей, так  как даже если они выполняют  добрые дела, это
служит им  в осуждение, а  не в оправдание.  Чтобы показать, что  я не
искажаю  мысли Бернала,  приведу чрезвычайно  любопытную выдержку  (с.
179--180): «Даже Роджер Бэкон (ок.  1235--1315) в своих раздраженных и
извращающих правду  обличениях современников  --- он  называет великих
св. Альберта  и св.  Фому ``невежественными мальчишками''  --- никогда
не  подверг  бы  сомнению,  что главной  целью  науки  была  поддержка
откровения. Его  единственным отличием  от них было  то, что  он искал
подтверждения  своих положений  в  опыте, а  не разуме.  Средневековые
ученые  были вполне  компетентны  в научных  рассуждениях, замыслах  и
выполнении  опытов.  Эти  эксперименты  были,  однако,  изолированными
и,  подобно арабским  и  греческим, продолжали  оставаться в  основном
демонстрациями, не ведущими к  каким-либо научным революциям. Какой бы
похвалы  ни заслуживала  за  свои  достижения горсточка  средневековых
экспериментаторов, они фактически мало  прибегали к использованию этих
методов  для исследования  природы  и еще  меньше  --- для  управления
ею.  У  них не  было  стимула  это  делать,  но много  доводов,  чтобы
не  делать.  Будучи  духовными  лицами,  они  имели  множество  других
занятий: Герберт (ок. 940--1003) первый из западных ученых стал папой;
Роберт  Гроссетест  (ок.  1168--1253), наиболее  талантливый  из  них,
был  епископом и  президентом Оксфордского  университета, св.  Альберт
Великий  был архиепископом  Доминиканского ордена,  причем власть  его
распространялась на всю Германию; такую же должность занимал Дитрих из
Фрейбурга (1300),  лучший из  экспериментаторов. Даже  наиболее смелый
мыслитель позднего Средневековья Николай Кузанский (1401--1464) подпал
под влияние папской  пропаганды и стал епископом в  Бриксене. Все, что
они делали  для науки,  они делали в  свободное время.  Исключения ---
Роджер Бэкон  и таинственный  Петр Пилигрим ---  подтверждают правило.
Роджер  Бэкон  затратил  крупное  состояние на  научные  изыскания  и,
несмотря  на  папское благословение,  был  за  свои труды  заключен  в
тюрьму».  Добавим еще  со  с. 182:  «Роберт Гроссетест,  средневековый
схоласт,  обладавший,   вероятно,  самым  блестящим  умом   и  имевший
наибольшее  влияние  на  развитие  средневековой  науки,  рассматривал
эту  науку, по  существу, как  средство для  иллюстрации теологических
истин. Изучение  света и  проверка рефракции  линз опытным  путем были
предприняты  им  потому,  что  он представлял  себе  свет  как  аналог
божественного освещения». Даже для  Роджера Бэкона, призывавшего науку
служить человечеству  и предсказывавшего  завоевание природы  путем ее
познания, научное  знание ---  есть лишь  часть, наряду  с откровением
совокупной мудрости, которую следует созерцать, ощущать и использовать
на службу Богу.

\subsection{Антигосударственный  характер  христианства в  момент  его
возникновения,  интернационализм и  платонизм стимулировали  к исканию
трудных задач и их решению}

Кажется, ясно.  Средневековые ученые, как правило,  не были побуждаемы
в   своей  работе   ни   необходимостью   заработка,  ни   стремлением
непосредственно использовать науку в  практических целях. Как уже было
указано выше, преследования Роджера Бэкона (§61) были вызваны в первую
очередь его  общественной деятельностью,  и сам Бернал  указывает, что
он  был  несправедлив  по  отношению к  своим  великим  и  единоверным
современникам.  Что же  заставило лучших  представителей католического
духовенства   заниматься   «сверхплановой»  работой   и   подготовлять
возрождение наук и  почему у мусульманского мы этого  как будто совсем
не  видим?  Мне  думается,  что  это различие  связано  с  разницей  в
идеологии обеих религий.

Христианство  возникло,  как известно,  в  тот  период, когда  Римское
государство,  с  одной  стороны,  достигло  величайшего  могущества  и
вместе  с  тем  раскрыло  всю  мерзость  той  политической  идеологии,
выразителем которой является Рим:  этатизма. Этот термин применяется и
сейчас;  ему можно  дать самое  широкое определение:  этатизм ---  это
идеология, в  которой государство считается самоцелью  и все остальное
рассматривается  просто  как средство  для  служения  этой цели.  Один
из  основных  постулатов  христианства  «царство  мое  несть  от  мира
сего»  не признает  святости светского  государства, и  неудивительно,
что  оно  возникло  среди  евреев  ---  самого  революционного  народа
Римского  государства.  Неудивительно,  что  при  всей  веротерпимости
римлян  (постройка  Пантеона для  всех  богов  империи), они  не  были
терпимы ни  для иудеев,  ни для  христиан, так  как ясно  сознавали их
римскую антигосударственность, несмотря на то что политический дуализм
христианства,  выраженный  знаменитыми   словами  «Воздадите  кесарево
кесареви,  а  божье ---  богови»,  казалось  бы, допускал  возможность
мирного сосуществования. С превращением христианства в государственную
религию исчез на  время антагонизм между светской  и духовной властью,
но возникла  самая страшная опасность  для всякой культуры  --- полное
поглощение  духовной власти,  шире  идеологии,  светской властью,  то,
что  называется цезаропализмом  (соединение  в одном  лице цезаря  ---
светского владыки и палы --- владыки духовного).

Антагонизм духовной и светской власти и вызывал многочисленные попытки
выдающихся христианских  мыслителей --- искание «града  божьего», т.е.
идеальной  формы  общественного  устройства. Эти  искания  естественно
примыкали   к  старым   теориям  Платона,   к  которому   христианство
чрезвычайно  близко   идеологически,  что  ясно   сознавалось  многими
христианскими  мыслителями,  о  чем  было  говорено  раньше.  Но  если
откинуть Платона,  о котором  наши марксисты  не любят  вспоминать, то
утопический  социализм,  как  известно, связан  с  именем  доминиканца
Томазио Кампанеллы  и ревностного  католика Томаса  Мора, заплатившего
жизнью за свои католические убеждения.

В  тесной  связи с  идеальным  представлением  о государстве  стоит  и
лозунг интернационализма  и общего равенства: «несть  эллин, ни иудей,
обрезание и необрезание, варвар и  скиф, раб и свободь». Конечно, надо
отдать  справедливость Риму,  что и  там были  сделаны шаги  в сторону
интернационализма, и в последний период всем свободным жителям Римской
империи были дарованы права римского гражданина, но ликвидация рабства
потребовала длительной борьбы в пределах христианских государств.

Наконец, платоническое  понимание мира  и его  математических законов,
подкрепленное  авторитетом   ряда  отцов  церкви   (Августин,  Василий
Великий, Ориген, Климент и  др.) оправдывало свободное искание законов
природы  духовными лицами,  и  такие искания  считались  не только  не
богопротивными, но скорее богоугодными. Вместе с тем все политические,
общественные и  научные задачи  были настолько  трудны и  обширны, что
составляли неиссякаемый источник вдохновения для творческих умов.

\subsection{Ислам ставил конкретные достижимые  задачи, много достиг в
смысле более высокой бытовой морали,  но не стимулировал решения задач
далекого прицела}

Совсем  иначе обстоит  дело  с исламом.  Магометанство  родилось не  в
рамках обширного  государства, а  среди разрозненных  враждующих между
собой  арабских племен.  На очереди  было создание  государственности,
а  не   борьба  с  чрезмерной  государственностью.   Задачи  ставились
практические,  сугубо   земные,  и  власть  по   самому  происхождению
была  монистична.  Не было  противоположности  града  земного и  града
божьего. Поставленная  Магометом задача была блестяще  разрешена: было
достигнуто  не  только объединение  арабов,  но  и создание  огромного
цветущего государства, но с  достижением этой цели исчерпался источник
вдохновения для решения новых задач.

В  «Натане мудром»  Лессинга Натан  на вопрос  Саладдина о  том, какая
религия:  магометанская, иудейская  или  христианская лучше,  отвечает
легендой  об  одном истинном  и  двух  поддельных кольцах:  судить  об
этом  можно  только  по  практическим  последствиям,  по  тому,  какая
религия  приводит  к более  высокой  морали.  Этот критерий  вовсе  не
так  объективен,  как кажется.  Если  судить  по бытовой  морали,  то,
пожалуй,  из трех  религий пальму  первенства придется  отдать исламу,
так  как  там достигнут  в  населении,  может быть,  наиболее  высокий
уровень:  статистика показывает  в  странах  с разноверным  населением
наименьшее число  преступлений среди  магометан, магометане  в среднем
наиболее  трезвые.   В  христианстве   же,  может   быть,  совершается
наибольшее   количество  преступлений,   христианская  же,   в  первую
очередь  католическая  церковь,  прославилась  наибольшим  количеством
отступлений  от духа  истинного  христианства и  в наибольшей  степени
подвергалась «обмирщению», т.е. забвению  духовных идеалов, и вместе с
тем  в  недрах  этой  церкви  сохранилось  наибольшее  число  истинных
праведников  и людей  высокой культуры,  которые помогали  ей выходить
из  жесточайших  кризисов.  В  самой католической  церкви  никогда  не
исчезала  «фракционность»,  всегда  была  борьба  мнений,  и  так  как
все  фракции опирались  на  мнения прославленных  авторов, никогда  не
возникало  той  «монолитности   идеологии»,  которая  является  верным
средством затормозить всякий духовный прогресс. Поэтому можно с полной
уверенностью  утверждать,  что  западноевропейская культура  вышла  на
широкую дорогу невиданного до нее прогресса не в борьбе с католической
церковью,  а именно  потому, что  западноевропейская цивилизация  была
христианской цивилизацией.

На общем  фоне прогресса  христианской культуры имеются  и исключения,
но  эти  исключения,  как  всегда бывает,  подтверждают  правило.  Эти
исключения  ---  на  диаметрально противоположных  концах  Европы  ---
Испания  и Россия.  Ужасная испанская  инквизиция справедливо  вызвала
возмущение  всех гуманных  людей и  дала великолепнейший  материал для
наиболее  действенной  антирелигиозной  пропаганды. Вместе  с  тем  мы
знаем,  что блестяще  развивавшаяся культура  Испании скоро  заглохла,
и  Испания  сделалась  второстепенной  страной. В  России  то  же  ---
дотатарская Русь  по культуре  не уступала  Западной Европе,  но потом
отстала на  несколько столетий. Что  общего между Испанией  и Россией:
1)  общий  враг  ---  мусульманский  мир; 2)  большая  роль  церкви  в
идеологической организации  этой борьбы с исламом;  3) полное единство
светской  и  духовной  власти; 4)  крайний  деспотизм,  использовавший
успехи  в борьбе  с  национальным врагом  для создания  неограниченной
монархии.  Ницше  неплохо  сказал:  «Кто борется  с  чудовищем,  пусть
остережется,  чтобы самому  не сделаться  чудовищем» (кстати  сказать,
сам  Ницше не  «остерегся» и  создал чудовищную  идеологию). Борясь  с
цезаропалистическим  исламом,  Испания  и  Россия сами  впали  в  грех
цезаропализма. Характерными для России  были теория «Москва --- третий
Рим»  и  страшный  духовный  регламент  Петра  Великого,  по  которому
священники должны  были нарушать тайну исповеди  в случае политических
проступков. Все  это вполне  объясняет, почему  в России  с наибольшей
силой  свирепствовал  и  еще  свирепствует  «этический  атеизм»,  т.е.
убеждение, что  ни один честный человек  не имеет права верить  в бога
(основная идея Луначарского, 1959; и многих других авторов).

\subsection{Переход  между  гуманизмом  и средневековьем  ---  Николай
Кузанский, долгое время забытый}

Прежде  чем  перейти  к  Возрождению, полезно  остановиться  на  одной
замечательной  фигуре позднего  Средневековья, как  бы символизирующей
переход  между двумя  мирами, кажущимися  совершенными антиподами  ---
Средние  века,  схоластика  и  гуманизм.  Это  ---  Николай  Кузанский
(1401--1464). Этот  мыслитель интересен с  разных сторон: 1)  он может
одновременно считаться последним (или  одним из последних) схоластиков
и  одним  из  первых  гуманистов;  2)  он  считается  предшественником
Коперника;   3)   в   еще   большей   степени   он   может   считаться
предшественником  Джордано Бруно;  4)  наконец, чрезвычайно  интересна
судьба его идейного наследства: он  был почти позабыт примерно в XVIII
и первой половине XIX в., и интерес к нему снова возник в конце XIX в.
и в наше время; его основные труды изданы в Советской России в 1937 г.

Коснемся вкратце  его биографии, руководясь, главным  образом, очерком
жизни,  составленным   С.А.Лопашевым  (Николай  Кузанский,   1937,  с.
336--348).  Родился Николай  Кузанский в  деревне Кузтрирской  епархии
в  семье  работорговцев.  Он  убежал из  дому  и  при  покровительстве
графа  Мандершейда  поступил  в  школу,  потом  сделался  августинским
монахом,  потом  кардиналом,  епископом,   в  конце  жизни  работал  в
Риме, как  говорили многие,  в роли  «вице-папы». При  больших научных
знаниях,  таланте  и  стремлении  к   науке  он  мог  заниматься  этим
только  урывками, так  как вся  жизнь  его прошла  в очень  энергичной
церковно-административной  работе.  На  Базельском соборе  он  сначала
выступал  против  верховенства  Папы,  но потом  изменил  своей  точке
зрения и  стал защитником  Папы. Он принимал  видное участие  в работе
по  объединению католической  и  православной церквей  и на  основании
изучения  сочинений  Василия Великого  и  Иоанна  Дамаскина доказал  с
другими  римскими  богословами,  что  учение об  исхождении  Св.  Духа
«и  от Сына»  (основное догматическое  расхождение между  католической
и  православной   церквами)  встречалось   уже  у   древних  греческих
богословов. Вместе с тем он раньше Лоренцо Валла (гуманиста) доказывал
апокрифичность  так называемого  Константинова дара,  на котором  папы
долгое время  обосновывали свое право  на светскую власть  (Энц. слов.
Брокгауза  и Ефрона,  т.  XXI,  с. 117).  Писал  также против  гуситов
и  пытался  убедить  султана в  справедливости  христианского  учения.
Умер  Кузанский в  1464  г.,  сопровождая папу  Пия  II  в Анкону  для
организации морских сил против турок.  Весьма вероятно, что если бы он
пережил Пия  II, то  мог бы сделаться  сам папой.  Подготовлял реформу
календаря, и  введенный после  григорианский календарь  в значительной
мере использовал его труды.  Он неизменно пользовался покровительством
пап (Евгения  IV, Николая  V и  Пия II),  хотя подвергался  критике со
стороны  некоторых богословов  за  склонность  к пантеизму.  Энергично
боролся с  распущенностью духовенства и  на этой почве вступал  даже в
резкие конфликты, но не с духовной, а со светской властью, подвергался
даже тюремному  заключению со стороны Сигизмунда  герцога Тирольского;
папе  Пию II  стоило большого  труда отстоять  своего друга.  Конечно,
недовольство вызвало и  то, что Николай боролся  с денежным обложением
церковных имуществ  со стороны  князей и  городов, но  он одновременно
боролся  против  суеверий,  веры в  чудеса,  конкубината  духовенства,
распущенности  монахинь, с  массовым  паломничеством  к святым  местам
(Ник.  Кузанский, с.  341).  Он много  занимался,  хотя без  особенной
удачи, и дипломатической деятельностью.

Николай  Кузанский, кроме  латинского  и греческого,  знал арабский  и
еврейский языки и с широкой  научной образованностью (о чем будет речь
дальше) соединял оригинальность мышления  и считается наиболее ярким и
сильным  мыслителем XV  в.  (там же,  с. 343).  Он  может быть  назван
последним  из  средневековых  мыслителей.  Однако в  то  время,  когда
большинство  схоластиков  его времени  придерживались  аристотелевской
доктрины,  переработанной Фомой  Аквинатом (ее  в общем  придерживался
и  величайший  поэт  Италии  Данте),  философский  базис  Николая  был
значительно шире. На основе мистического учения Дионисия Ареопагита он
пришел к Платону,  использовал и неоплатонизм, был  знаком с еврейской
религиозной философией --- Каббалой, подготовку получил у мистика Фомы
Кемпийского. Его философия --- явный отход от Аристотеля.

Как  это  ни может  показаться  противоречивым,  но Николай  Кузанский
одновременно может считаться последним из крупнейших средневековых, т.
е.  схоластических, мыслителей  и  величайшим  из немецких  гуманистов
первого  поколения (Энц.  слов. Брок.  и Ефр.,  т. XXI,  с. 117).  Это
ясно  не  только из  того,  что  он  в некоторых  отношениях  является
предшественником  Коперника  и Бруно,  но  и  по  стилю его  работ.  В
числе его  сочинений имеется «Простец  об уме» (Идиота  до сапиенциа).
Первая книга  этого сочинения «О мудрости»  долгое время приписывалась
бесспорному гуманисту Петрарке под  названием «Об истинной мудрости» и
входила в собрание сочинений  прославленного поэта (Ник. Кузанский, с.
344). О связи  его с учениями Коперника и Бруно  будем говорить позже,
сейчас же коснемся удивительной судьбы его идейного наследства.

\subsection{Огромное  влияние  Кузанского   на  ближайших  мыслителей.
Кузанский как предшественник Коперника}

Николай   Кузанский  пользовался   большой  известностью   при  жизни,
сочинения  его распространялись,  а  с  появлением книгопечатания  ---
печатались и  отнюдь не оказались  без влияния. О Джордано  Бруно речь
будет  в своем  месте, но,  кроме него,  его влияние  можно проследить
вплоть до Лейбница и его «Монадологии». Леонардо да Винчи и Кампанелла
изучали и  развивали дальше  его наследство,  применяя к  областям, не
затронутым самим философом. Наиболее полное издание его сочинений было
в Базеле (1565,  Ник. Кузанский, с. 347). А затем  его известность все
уменьшалась и,  как уже было  указано, в трехтомной  истории философии
Гегеля  для  него  вовсе  места  не  оказалось.  А  в  конце  ХIХ  в.,
примерно через четыреста лет после  его смерти, на него снова обратили
внимание, с  1932 г. в  Лейпциге предпринято многотомное издание,  и в
советское время, при  господстве материалистической философии, впервые
на русском языке появляется первое издание его избранных произведений;
до  этого  переводили  только  некоторые отрывки.  Чем  же  объяснить,
что  христианский  мыслитель,  забытый на  долгое  время  идеалистами,
пропагандируется под эгидой материалистической  философии? Из статьи в
БСЭ,  2-е изд.,  «Николай Кузанский»,  и заметки  «От издательства»  в
издании  его  сочинений  ясно,  что, кроме  его  роли  предшественника
Коперника и  Бруно, ему ставится  в заслугу роль одного  из крупнейших
философов XV в.: пантеист,  ряд высказываний диалектического характера
(совпадение противоположностей), защита  опытного исследования (в этом
усматриваются  элементы материализма).  Отмечаются  заслуги в  области
астрономии, математики и механики, где во многом Кузанский значительно
опередил  свое время.  Важно и  то,  что Николай  Кузанский уже  тогда
понимал,  что  процесс  познания   бесконечен:  он  никогда  не  может
завершиться,  подобно  тому  как  вписанный  в  круг  многоугольник  с
увеличением числа  сторон приближается  к кругу,  но никогда  не может
совпасть  с  ним.  Наконец,  отмечается, что  Николай  составил  карту
Восточной и Средней Европы и выступал с проектом реформы календаря.

Выдающееся  значение Кузанского  как  разностороннего и  оригинального
мыслителя   не  оспаривается   сейчас   никем.  Но   для  нашей   цели
необходимо  разобраться,  связаны  ли  его бесспорные  заслуги  с  его
основным мировоззрением,  которое вне всякого сомнения  было искренним
христианским, или с теми  элементами пантеизма и материализма, которые
стараются отыскать в его сочинениях наши марксисты.

Разберем  теперь  вопрос  о значении  Кузанского  как  предшественника
Коперника.

О  подвижности Земли  Кузанский говорит,  по-видимому, только  в одном
своем сочинении «Об ученом невежестве» (Николай Кузанский, с. 97, 98 и
100; Уэвель, с. 462).

С.  97: «Как  невозможно, чтобы  мир был  заключен между  материальным
центром  и  окружностью,  то  мир  непостижим, ибо  центр  его  и  его
окружность суть  бог, и так как  наш мир не бесконечен,  все же нельзя
его считать  конечным потому, что  он не имеет границ,  между которыми
заключен. Так, Земля, не могущая быть центром, не может быть абсолютно
лишена  движения, даже  необходимо,  чтобы она  имела такое  движение,
чтобы могла иметь еще бесконечно менее сильное движение».

С.  98:  «Тот, кто  является  центром  мира,  иными словами,  бог,  да
святится имя его, является и центром Земли, и всех сфер, и всего того,
что есть  в мире,  и он  же вместе с  тем есть  бесконечная окружность
всяких вещей».

С. 100 и 462:  «Древние не достигли до этого знания,  потому что у них
не было ученого  невежества. Но для нас ясно,  что Земля действительно
находится  в движении,  хотя нам  этого и  не кажется;  потому что  мы
замечаем движение только по сравнению с чем-нибудь неподвижным. Потому
что, если кто-нибудь сидел в лодке  посередине реки, не зная, что вода
течет и  не видя  берега, то как  бы он узнал,  что лодка  движется. И
таким образом, так как всякий, будет ли он находиться на Земле, или на
Солнце,  или на  другой какой  звезде,  полагает, что  он находится  в
неподвижном центре, а что все другое  движется; то он назначал бы себе
различные полюсы,  одни, если бы был  на Солнце, другие ---  на Земле,
третьи --- на Луне и так далее. Потому что машина мира как будто имеет
свой центр повсюду и свою окружность нигде». Уэвель справедливо пишет,
что этот  ряд мыслей мог  быть приготовлением к  принятию Коперниковой
системы, но он вовсе  не похож на учение о том,  что Солнце есть центр
планетных движений.

На той же с. 100 читаем такие  слова: «Даже Земля не сферична, как это
говорили,  хотя стремится  к сферичности,  ибо фигура  мира ограничена
в  своих  частях,  как  и движение...  Более  совершенно,  чем  другие
фигуры,  движение  кругообразное  и наиболее  совершенная  фигура  ---
сферичная... Вот почему движение  всего старается насколько может быть
кругообразным, и  всякая фигура быть  сферичной. То же мы  наблюдаем в
членах  животных,  в  деревьях,  в небе...  фигура  Земли  подвижна  и
сферична, ее движение кругообразно, не будучи всецело совершенным».

\subsection{Центризм  Кузанского   ---  предшественник   Эйнштейна.  О
населенности других планет}

Из    этих   достаточно    неопределенных    высказываний   в    чисто
платоновско-пифагорейском  духе трудно  заключить, какие  именно формы
движения Земли имел в виду Кузанский. По «Истории философии» (1941, т.
II, с. 44), Кузанский считает, что Земля вращается вокруг оси с полным
оборотом  в течение  суток, и  у него  же имеются  туманные намеки  на
второй  вид  движения:  мыслитель  утверждает, что  любая  часть  неба
находится в движении.

Кузанский  был астрономом,  принял участие  в создании  проекта нового
календаря,  но, видимо,  беспокойная  жизнь не  давала ему  достаточно
досуга,  чтобы  проводить  систематические наблюдения  и  размышления.
Никаких намеков на математическую  теорию, подобную теориям Птолемея и
Коперника, Кузанский,  видимо, не оставил. Но  высказывания Кузанского
существенно отличаются  от некоторых основных положений  Коперника и в
этом отношении  предвосхищают Бруно  и некоторые современные  идеи. На
это  указывает  в  своей  интересной  статье  о  философских  аспектах
космологии  Гарре  (1962, с.  112--113).  Сходство  между Кузанским  и
Коперником  в  том,  что  оба  отрицают  геоцентрическую  систему,  но
в  то  время как  система  Коперника  была гелиоцентрической,  система
Кузанского может быть названа  ацентрической (не имеющей центра вовсе)
или  теоцентрической (центр  ---  бог). Принцип  Коперника: «Земля  не
является привилегированным началом  космографической системы». Принцип
Кузанского:  «Не  существует   вообще  привилегированного  начала  или
ориентировки космографической системы». Гарре указывает, что аргументы
Кузанского частично весьма утонченны (изоморфность всех геометрических
систем),  частично мистичны  (и центр,  и окружность  являются богом).
Система Коперника в  дальнейшем развитии Ньютона приводит  к тому, что
для  всякого конечного  множества  тел можно  найти  центр тяжести  и,
расширяя все время область изучения,  мы можем найти некоторый предел,
и этот предел и будет центром  Вселенной. У Кузанского же, выражая его
мысли ньютоновской  терминологией, бесконечное множество тел  не имеет
центра тяжести;  или, иначе говоря,  любой избранный пункт  может быть
сделан  центром  тяготения  при расширении  изучаемой  системы.  Гарре
указывает, что в  этом отношении мысли Кузанского  сходны со взглядами
Эйнштейна.  Можно  найти и  другое  сходство  с Эйнштейном,  именно  в
утверждении Кузанского  (предыдущий параграф), что мир  не бесконечен,
но   безграничен:  это   как  раз   характеризует  геометрию   Римана,
принимаемую многими современными выдающимися учеными.

У Кузанского гораздо  больше близости к Бруно, чем к  Копернику, и это
особенно  ясно  в  его  высказываниях по  поводу  множества  обитаемых
миров.  Для нашей  задачи  это  особенно важно,  так  как идея  обычно
приписывается  (по  крайней  мере, для  христианского  мира)  Джордано
Бруно, поэтому полезно привести точные высказывания Николая Кузанского
по этому  вопросу, с. 105:  «Нельзя говорить, раз Земля  меньше Солнца
и  находится  под  его  влиянием,  что  она  презренней  его  на  этом
основании... И хотя Земля меньше Солнца, как это очевидно по ее тени и
затмениям,  однако неизвестно,  насколько  область  Солнца больше  или
меньше области  Земли... Земля  не является  самой малой  звездой, ибо
она, как  показывают затмения,  больше Луны и  даже Меркурия,  а может
быть,  и еще  других звезд.  По соображениям  о размерах  Земли нельзя
заключить, чтобы она была презренной звездой».

С.  103: «Земля  есть  как  бы возможность,  Солнце  как  бы душа  или
формальная разумность  относительно нее, а  Луна --- связь  между ними
таким  образом, что  звезды  эти, находящиеся  внутри единой  области,
взаимно связываются своими  влияниями, а также связывают  их с другими
звездами:  Меркурием,  Венерой  и  всеми  звездами,  существующими  за
пределами,  как  говорили  древние  и даже  некоторые  из  современных
мыслителей. Таким  образом ясно, что имеется  соотношение влияний, при
котором одно не может существовать без другого.

Это  влияние будет  единым и  троичным, в  чем бы  оно ни  проявлялось
одинаковым образом, но в различных степенях. Поэтому ясно, что человек
не  может  знать,  какая  область Земли  более  совершенна  или  менее
благородна в сравнении с областями других звезд, Солнца и Луны.

«Хотя  бы бог  и был  центром и  окружностью всех  областей и  от него
проистекали  бы  различные  благородные  породы,  обитающие  в  каждой
области,  чтобы  не  дать   пустовать  стольким  небесным  и  звездным
пространствам,  а  не  только  Земле, населенной,  быть  может,  менее
благородными  существами,  однако  не представляется  возможным  найти
более  благородную и  более совершенную  породу, чем  разумная порода,
населяющая Землю  как собственную  область. И это  даже в  том случае,
если на других звездах имеются жители  иного рода. Человек не желает в
действительности  другой  породы,  другой натуры,  но  старается  быть
совершенным в своей, ему присущей».

С.  104:  «Мы   еще  менее  сможем  узнать   жителей  другой  области,
предполагая их в области  Солнца более солнечными, яркими, озаренными,
разумными, более одухотворенными, чем на Луне, обитатели которой более
материальны и  грубы, таким  образом, что  эти разумные  натуры Солнца
менее  в действительности  и  более в  возможности,  тогда как  земные
жители суши  менее в возможности  и более  в действии и  что обитатели
Луны являются посредствующими».

«Об этом  мы догадываемся по  огненному влиянию Солнца  одновременно с
водным и воздушным влиянием Луны  и материальной тяжестью Земли. Таким
же  образом  мы  рассуждаем  относительно  других  звездных  областей,
предполагая, что  ни одна из них  не лишена жителей, подобно  тому как
если  бы  существовало  столько  отдельных  частей  единой  Вселенной,
сколько имеется звезд  (а они бесчисленны), таким  образом, что единый
универсальный  мир ограничен  тройственно посредством  своей четвертой
прогрессии, могущей  перейти в столь многочисленные  частности, что им
не будет числа, если только не к тому, что сотворил все в числе».

С.  113: «О  возможности  существования других  разумных, созданий  мы
будем  говорить в  ``О  предположениях'', и  если  считается, что  они
принадлежат к человеческому виду по причине своей чувственной природы,
то  их  скорее можно  было  бы  назвать  духами, чем  животными,  хотя
платоники и считают их разумными животными».

\subsection{В   мировоззрении  Кузанского   совмещалось  убеждение   о
населенности  других  планет  и  звезд  с  его  основной  богословской
деятельностью,  лишенной всякого  фанатизма  и исполненной  стремления
дать синтез всех религиозных представлений}

Мы видим, что, защищая с совершенной ясностью учение о множественности
обитаемых  миров, Кузанский  вовсе  не претендует  на приоритет  этого
учения. Как  указывает в своей интересной  книжке И.С.Шкловский (1962,
с. 3), древние  идеи о множественности обитаемых  миров содержатся еще
в  древних  индийских  Ведах,  где они  связаны  с  религиозной  идеей
о  переселении  душ.  Дальше  эти идеи  высказывались  рядом  античных
философов,  как  материалистами,  так   и  идеалистами,  и  вплоть  до
второй  половины XIX  в.  были широко  распространены представления  о
повсеместном  распространении  разумной  жизни.  Этому  мнению  отдали
дань  и  Кант,  и  Лаплас,  и  Гершель,  и  Ньютон.  Шкловский  только
повторяет широко  распространенное заблуждение, что за  полторы тысячи
лет господства  христианской религии,  опирающейся на  учение Птолемея
и  считающей   Землю  средоточием  Вселенной,  ни   о  каком  развитии
представлений о множественности  обитаемых миров не может  быть и речи
(с.  4). Как  это ни  может показаться  странным, но  развитие строгой
науки привело  не к  расширению наших представлений  о множественности
обитаемых  миров,  и  сам  Шкловский в  своей  книге  доказывает,  что
Вселенная населена  гораздо реже,  чем предполагал Кузанский  и другие
религиозные мыслители.  Об этом нам придется  поговорить, когда дойдет
речь до Дж.Бруно.

Вся  аргументация Кузанского  показывает,  что он  не только  пытается
«примирить» учение  о множественности  обитаемых миров  с христианским
учением,  но  приводит  религиозные  аргументы  в  пользу  гипотезы  о
множественности   миров,   оперируя   вместе  с   тем   пифагорейскими
мистическими   числовыми  рассуждениями,   связанными   с  учением   о
троичности. А так как Кузанский был несомненно один из образованнейших
богословов  всех времен  и так  как  его взгляды  никогда не  вызывали
осуждения  со  стороны  католической  церкви, то  ясно,  что  гипотеза
об  исконном   антагонизме  учения  о  множестве   обитаемых  миром  с
христианством по крайней мере спорна.

Крупные  заслуги  Кузанского в  разных  областях  науки могут  создать
впечатление,  что   Кузанский  в  основном  был   ученым  и  занимался
богословием  только  в  силу  необходимости.  Просмотр  его  избранных
сочинений  совершенно   опровергает  такое  мнение.   Богословие  было
стержнем  всей  его  мыслительной  работы.  Это  ясно  уже  и  из  его
труднейшей  работы  «О  неином»  («Неиное»  ---  бог)  и  из  названий
некоторых  глав   его  основной  работы  «Об   ученом  невежестве»:  О
троичности  и  единой  вечности; Каким  образом  понимание  троичности
в  единстве  превосходит все;  О  могущественной  помощи математики  в
усвоении различных божественных истин; Еще вокруг троичности, и почему
четверичность  и еще  больше  невозможны в  божественных вещах;  Каким
образом божье  провидение соединяет противоречия; Наименование  бога и
утвердительная теология; Народы различно  именовали бога во внимание к
творениям; Отрицательная теология; О троичности Вселенной; Возможность
как материя Вселенной; Душа как  форма Вселенной; Дух Вселенной; Каким
образом этот максимум есть благословенный  Иисус, бог и человек; Каким
образом  Иисус, зачатый  через духа  святого, родился  от девы  Марии;
Тайна смерти Иисуса  Христа; Тайна воскресения. ---  Вообще вся третья
книга посвящена Иисусу Христу.

При своей обширной богословской,  и притом ортодоксальной католической
деятельности  (напомним  его  труды   не  только  против  мусульман  и
гуситов,  но  и  стремление  к объединению  с  православной  церковью)
Кузанский  поражает  полным  отсутствием  фанатизма  и  исключительной
широтой  мысли. Я  уже не  говорю о  его близости  к Сократу,  Платону
и  Пифагору,  о  чем  речь   будет  дальше,  но  использует  широко  и
поздних античных философов, заведомо враждебных христианству, например
Прокла  и  Апулея.  Широко   использует  Гермеса  Трисмегиста  (Трижды
величайший),  под  именем которого  дошла  до  нас литература  поздней
античности  теологически-эсхатологического содержания  (Ник. Кузанский
с. 357). Мало того,  вопреки тому представлению некоторых христианских
богословов, считавших,  что языческие боги ---  злые демоны, Кузанский
склоняется к  мнению, что  так называемый политеизм,  многобожие, есть
неправильно  понятое  единобожие  и готов  приписать  языческим  богам
многие свойства христианского бога.

С.   53:  «Язычники   называли   бога  в   зависимости  от   различных
представлений о  творениях: Юпитером  --- за его  удивительную доброту
(Юлиус Фабрициус говорит, что  Юпитер --- столь благоприятное небесное
светило,  что,  если  бы  Юпитер  царил один  в  небе,  люди  были  бы
бессмертны); Сатурном --- по  глубине мыслей и изобретению необходимых
в  жизни вещей;  называли бога  Марсом  на основании  побед в  войнах;
Меркурием  ---  вследствие осторожности  в  советах;  Венерой ---  как
носительницей  любви,  сохраняющей  природу; Солнцем  ---  по  причине
жизненной силы  для всего  рожденного природой;  Луной ---  по причине
сохранения соков,  от которых зависит жизнь;  Купидоном --- вследствие
дружбы  обоих полов;  и называли  его  даже природой,  потому что  она
сохраняет виды вещей посредством двойственности полов».

С.  54:   «Древние  язычники  смеялись  над   евреями,  поклонявшимися
единственному и бесконечному богу, которого  они не знали, но которого
они чтили в  его проявлениях и преклонялись перед ним  там, где видели
его божественные  деяния. Между  народами мира существует  та разница,
что если  бы все люди верили  в единого и максимального  бога, такого,
больше  которого  не могло  бы  быть  в мире,  то  одни,  как евреи  и
сиссениты,  поклонялись  бы ему  в  его  бесконечно простом  единстве,
заключающем в  себе все вещи, а  другие поклонялись бы ему  в вещах, в
которых находили  объяснение его  проявления, опираясь  на чувственные
знания как на путь к причине и основе».

«Этим путем  были привлечены простые  люди, народ, но они  отнеслись к
этому не как к изображению идола, а как к истине. Идолопоклонство было
введено в массу в то время, когда мудрецы обладали весьма определенной
верой  в  единство бога:  весь  мир  может  дать  себе в  этом  отчет,
достаточно  лишь внимательно  прочитать труды  древних философов,  как
например Туллия и ``О природе богов'' Цицерона».

Кузанский  явно стремится  найти  общее у  всех  религий и  обосновать
христианские  догматы  ссылкой  на древних  языческих  мудрецов.  Вот,
например, как он философски обосновывает догмат троичности бога.

С. 20: «Несомненно, наши пресвятые наставники называли отца единством,
сына равенством,  и духа святого  связью, но  они так называли  это по
подобию с превосходящими вещами. В отце и сыне есть общая природа; сын
равен отцу, ибо  в сыне нет ни больше, ни  меньше человеческого, чем в
отце,  и  между  ними  существует  известная  связь.  Согласно  такому
подобию,  как  бы  оно  ни было  отдаленно,  единство  названо  отцом,
равенство  --- сыном  и связь  в действительности  любовью, или  духом
святым,  ---  только  имея  в  виду творения,  как  мы  покажем  ниже,
когда дойдем  до этого. Вот,  по моему мнению,  согласно свидетельству
пифагорейцев, весьма очевидное  показание относительности троичности в
единстве и единства в троичности, достойное вовеки поклонения».

\subsection{Несомненный пифагоризм и  платонизм Кузанского совмещается
с  большим  уважением  к   Аристотелю,  но  при  расхождениях  Платона
и  Аристотеля  он,  как  правило,   примыкает  к  Платону,  отнюдь  не
догматизируя учение Платона}

Современные безбожники часто с торжеством доказывают, что христианство
не  оригинально,   так  как  догмат  троичности   ими  заимствован  от
предшествующих религий. Как видим из приведенной цитаты, они ломятся в
широко открытую  дверь, так  как один  из самых  выдающихся богословов
Средневековья утверждал то же самое...  Он предвидел и другой аргумент
атеистов  против догмата  троичности, так  как этот  догмат по  мнению
современных атеистов противоречит таблице умножения 3 х 1=1. Вот слова
Кузанского.

С. 61: «На  самом деле всякая часть  бесконечности есть бесконечность.
Было бы противоречием,  если бы обнаруживали большее  или меньшее там,
где  можно  достигнуть  бесконечности;  большее  и  меньшее  не  могут
соответствовать  бесконечности  и  не  имеют  никакого  соотношения  с
бесконечностью, ибо  было бы  необходимым, чтобы  даже и  они являлись
бесконечностью.  В  бесконечном  количестве  не  было  бы  правильным,
чтобы  ``два'' было  бы меньше  ``ста'', ибо,  поднимаясь, можно  было
в  действии  достигнуть и  этой  цифры,  как  не было  бы  правильным,
что  бесконечная  линия,  составленная  из  бесконечного  числа  линий
по  два фута,  была  бы меньше,  чем  бесконечная линия,  составленная
из  бесконечных линий  по  четыре фута».  Это  рассуждение, сходное  с
тем,  что мы  увидим впоследствии  у Галилея,  является зародышем  той
поразительной  теории множеств,  которая была  развита лишь  во второй
половине  XIX  в. гениальным  Георгом  Кантором,  кстати, бывшем  тоже
глубоко верующим  католиком, пожалуй,  только много  более фанатичным,
чем Николай Кузанский.

Близость  к пифагореизму  и платонизму  у Кузанского  совершенно ясна.
Несомненна  и  его  близость  к  Сократу. И  в  сочинении  «Об  ученом
невежестве» он непосредственно ссылается на Сократа (с. 7 и др.) и все
понятия «Ученого незнания» --- чисто сократические.

С. 11: «Сущность вещей, которая есть истина бытия, недостижима в своей
чистоте». В сочинении «Об уме» главный персонаж --- простец (по-латыни
--- «идиот»)  чрезвычайно сходен по  облику с Сократом:  все поведение
простеца,  что он  сначала  говорит, что  знает  гораздо меньше  своих
собеседников  ---  перипатетиков, а  в  конце  концов его  собеседник,
философ, принужден  признать в  простеце «в полном  смысле теоретика».
Собеседники застают простеца за работой  по выделке ложек, и простец в
разговоре оперирует аналогиями, взятыми из своей работы, и утверждает,
что эта работа ничуть не мешает размышлениям:

С. 160: «И  верю, что если тот, которого ты  приводишь, философ, то он
не станет меня презирать за то, что я отдаюсь работе ложечника».

«Философ. Ты  говоришь прекрасно.  Ведь и  Платон, как  читаем, иногда
занимался живописью  (в примеч. ссылки).  А этого, следует  думать, он
никогда бы  не делал, если  бы живопись противоречила  умозрению». Это
замечание  очень  интересно.  Николай  Кузанский  в  противоположность
господствующему  сейчас  среди  антиплатоников  мнению,  очевидно,  не
считал, что  Платон был  против практического  применения знаний,  и к
этому мнению Кузанского сейчас присоединяются многие исследователи. Об
этом будет  позже, когда дойдем  до роли идеологии в  развитии техники
--- не скоро.

В   разговоре   Философ   называет  простеца   пифагорейцем,   и   это
рассматривается как  похвала. О Пифагоре и  о пифагорейцах упоминается
много раз и всегда с положительной оценкой.

С. 7: «Так и Пифагор, авторитет  которого в его время был непоколебим,
считал, что это единство троично».

С.  18: «Пифагор,  первый  из философов,  гордость  Италии и  Греции».
«Разве  Пифагор, первый  из философов  по  достоинству и  на деле,  не
направил искание истины на числа?»

Пифагореизм   защищается  Кузанским   и   в  той   форме,  которую   в
педагогических целях придал ему Платон (квадривиум).

С.  106--107:  «Бог  пользовался   при  сотворении  мира  арифметикой,
геометрией,  музыкой  и  астрономией, всеми  искусствами,  которые  мы
также  применяем,  когда  исследуем  соотношение  вещей,  элементов  и
движений. При  помощи арифметики  бог создал из  мира одно  целое. При
помощи геометрии  он образовал  вещи так, что  они стали  иметь форму,
устойчивость и подвижность в зависимости  от своих условий. При помощи
музыки он  придал вещам  такие пропорции, чтобы  в земле  было столько
земли, сколько воды в воде, сколько воздуха в воздухе и огня в огне...
Потому-то и можно говорить, что  элементы созданы богом в изумительном
порядке, ибо все бог сотворил в  числе, весе и мере; число принадлежит
арифметике,  вес ---  музыке,  мера ---  геометрии».  Здесь мы  видим,
Кузанский по  своему обычаю  находит синтез  Соломона (на  которого он
тоже в одном месте ссылается), Пифагора и Платона.

Имя  Платона  повторяется у  Кузанского  много  раз, он  пользуется  и
терминологией  Платона; именно  он называет  астрономов в  современном
понимании  астрологами,  следуя  Платону,  что и  отмечено  в  примеч.
11  на с.  362.  Сам  Платон иногда  именуется  устами  Прокла как  бы
«вочеловечившийся бог»  (с. 273), «божественный Платон»  (с. 34), «без
квадривия нельзя философствовать» (с. 201).

Несомненно,  Кузанский  отошел  от   Аристотеля,  но  это  не  значит,
что  он  полностью  отрицает   или  игнорирует  Аристотеля.  Напротив,
имя  Аристотеля   тоже  упоминается   с  почетом,  но   без  атрибутов
божественности: «гениальное учение Аристотеля» (с. 222), «величайший и
проницательнейший перипатетик» (с. 268): «Хотя этот философ погрешил в
первой  философии,  или  философии  ума,  он,  однако,  написал  много
достойного всяческой похвалы в  рассудочной и моральной философии» (с.
272).

Во   многих   местах   Кузанский  сравнивает   учения   платоников   и
перипатетиков  и  склоняется,  как  правило,  на  сторону  платоников.
Особенно это ясно в таких вопросах:

1) о врожденности или неврожденности понятий; 2) об одном интеллекте у
всех людей или  разном; 3) элементарно --- рассудок  или интеллект; 4)
бытие  форм ---  только в  материи или  вне материи.  Во всех  случаях
Кузанский не  просто отвергает Аристотеля (как  сейчас любят говорить,
«с порога»),  а обычно  находит некоторый  синтез, который  ревизует и
учение  Платона, но  вместе  с тем  лишает  силы аргументы  Аристотеля
против  Платона. При  всем уважении  к Платону  у него  нет восприятия
этого учения  как безусловного  догмата, он  не только  исправляет или
видоизменяет  положения  Платона,  но  местами  просто  отвергает  их,
например в  том, что платоники  не видели  соединения духа и  воли (с.
213):  правда,  по  этому  пункту Кузанский  отвергает  и  Аристотеля.
Критикует платоников он и на с. 92.

С.А.Лопашев  в очерке  на с.  343  пишет, что  Кузанский «больше,  чем
гуманисты,  последователи Петрарки,  даже больше,  чем Лоренцо  Валла,
он  отвергает авторитет  Аристотеля». Общее  впечатление от  сочинений
Кузанского то,  что он  не просто  «отвергает авторитет»,  а старается
критически разобраться во  взглядах всех философов и ни  одного из них
не считает за непререкаемый  авторитет. Даже свои христианские взгляды
он никогда не защищает простой ссылкой на тексты Священного Писания, а
пытается дать им рациональное истолкование. Общее сравнение платоников
с  Аристотелем выражено  Кузанским  на с.  90:  «Платоники говорили  с
большой проницательностью  и уверенностью, и упреки,  которые им делал
Аристотель,  лишены всякого  основания: он  старался опровергнуть  их,
скорее придираясь к их словам, чем проникая в ядро их учения». Лично я
думаю,  что  эта формулировка  Кузанского  справедлива  и в  настоящее
время.

\subsection{Никаких тенденций  к материализму Кузанский  не проявляет;
резко отрицательное отношение к  Эпикуру, пантеистические же черты его
мировоззрения вовсе не противоречат идеализму}

Мы видим,  что основание  философии Кузанского  --- Пифагор  и Платон;
второстепенное, но  существенное место  занимает Аристотель.  А каково
отношение к  материалистам? О Демокрите вообще  ничего не упоминается.
Об Эпикуре и эпикурейцах я нашел два места.

С. 24: «Короче  говоря, разве теория эпикурейцев об  атомах и пустоте,
теория,  отрицающая  бога и  уничтожающая  всякую  истину, не  погибла
от  математического доказательства  пифагорейцев и  перипатетиков? Они
установили с  очевидностью, что  нельзя прийти  к неделимым  и простым
атомам, в  чем заключается  основной принцип,  выставленный Эпикуром»;
на  той же  странице:  «Кроме  того, если  бы  возможность вещей  была
ограничена, она  не могла  бы считаться  за основу  вещей, и  все вещи
происходили бы случайно, по ошибочному мнению Эпикура». Однако отрицая
атомы  в смысле  Эпикура, Кузанский  не отрицает  существования атомов
в  ином  смысле. О  разном  понимании  понятия «атом»  эпикурейцами  и
платониками  разберем  в  главе  об  атомной  теории.  Важно  то,  что
Кузанский отрицает атомы как философскую  основу бытия, но не отрицает
их в реальной природе.

С.  196--197:  «Философ.  А   что  такое,  по-твоему,  атом?  Простец.
Согласно  усмотрению  ума,  непрерывное  делится  на  то,  что  всегда
делимо,  и множественность  растет до  бесконечности. Но  в результате
действительного  деления  мы  доходим  до  части,  в  действительности
неделимой. Ее  я и  называю атомом, потому  что атом  есть количество,
которое  вследствие   своей  малости  неделимо   в  действительности».
Совершенно ясно, что никакой тенденции  к материализму, при всей своей
широте, Кузанский не обнаруживает. Это явно и из того, что из античных
философов, относимых к материалистам в  самом широком смысле, он берет
как раз не материалистические мысли. Так, упоминая об Анаксагоре, он с
одобрением отзывается о его высказывании,  что «любое --- в любом» (с.
73).  У Эмпедокла  он  находит достойной  упоминания  лишь мысль,  что
«единое  и  сущее»  является  дружбой. Кузанский  был  мистиком  и  по
образованию (Фома  Кемпийский, Дионисий Ареопагит,  Бернар Клервоский,
Бонавентура, Теодор  Шартрский, Раймунд  Луллий) и по  всему характеру
мышления, и  по стилю  изложения. У  нас часто  называют «мистическим»
очень  туманное, трудно  понятное изложение.  Работы Кузанского  очень
разнообразны по характеру изложения,  наиболее живо и понятно написана
работа  «Об уме»,  но  основная работа  «Об  ученом невежестве»  очень
трудна, а  работа «О  неином» приводила даже  переводчика (А.Ф.Лосева)
временами  в  отчаяние.  Этот  трактат не  входил  в  полное  собрание
сочинений, был впервые  найден во второй половине XIX  в., напечатан в
1888 г., написан  он был за два года до  смерти. Переводчик сознается,
что  местами неясность  текста так  и осталась  непреодоленной, а  про
шестую  главу пишет,  что вся  глава настолько  трудна для  перевода и
ясного понимания,  что почти  невозможно ручаться  за точность  того и
другого.  В этой  трудности  и «мистицизме»  и заключается,  вероятно,
главная причина того, что труды Кузанского оказались забытыми примерно
три столетия.

Но как совместить со всем  изложенным о философских воззрениях Николая
Кузанского то,  что «Философия Николая  Кузанского --- прикрытая  и не
вполне последовательная форма пантеизма»  (История философии, 1941, т.
II, с. 43--44). «История, философии»  не скрывает наличия у Кузанского
пифагорейско-неоплатоновской мистики,  но, очевидно,  считает наиболее
ценным у  него пантеистические взгляды,  как переход к  материализму и
атеизму.  Указание на  пантеизм  Кузанского  не ново.  Пантеистические
элементы его  мышления приписываются влиянию Экхарда  (Ник. Кузанский,
с. 343),  проявляются в его  главном сочинении «Об ученом  незнании» и
вызвали при жизни Кузанского (еще  в 1449 г.) критику гейдельбергского
богослова Иоганна  Векка, обвинившего  его в  «пантеистической ереси».
Вероятно,  сам термин  не был  использован Векком,  так как  по статье
«Пантеизм»  в БСЭ  термин  «пантеизм» введен  в литературу  английским
материалистом  Дж.  Толандом  (1670--1722). В  ответ  Векку  Кузанский
выпускает  «Апологию   ученого  незнания»  (1449),  и   тем  дискуссия
заканчивается.  Пантеистические элементы,  конечно, имеются  у Николая
Кузанского,  но, надо  сказать, что  они  имеются и  в христианстве  в
общеизвестном догмате о вездесущии Бога.

Но предположение о том, что  пантеизм обязательно ведет к материализму
и  атеизму,   целиком  основано   на  недоразумении.  Даже   в  статье
«Пантеизм» в БСЭ  указано, что в форме пантеизма  может выражаться как
идеалистическое, так и  материалистическое мировоззрение, и «пантеизм»
Кузанского  несомненно идеалистичен.  Подробнее об  этом поговорим  по
поводу Джордано Бруно.

\subsection{Николай Кузанский,  совмещавший в себе черты  схоластики и
гуманизма, иллюстрирует  то положение,  что гуманизм и  Ренессанс были
подготовлены развитием средневековой философии}

Николай Кузанский является наилучшим  образцом крупнейших мыслителей и
деятелей  Средних веков.  Он прекрасно  сознает критическое  положение
католической церкви, верным  и преданным сыном которой  он является до
самой смерти. Реформаторская деятельность  как епископа и кардинала не
оставляла Кузанскому достаточно времени  для научной работы, к которой
он искреннее стремился,  и не позволила в полной  мере развиваться его
выдающемуся дарованию.

Восторженный  почитатель Кузанского,  Джордано Бруно  выразился о  нем
так: «Если  бы монашеский клобук  не затмевал  и не стеснял  порой его
гения, то Кузанского  надо было бы считать выше  Пифагора» (Дж. Бруно,
1934, с.  229). Вместо «монашеский  клобук» было бы  правильно сказать
«кардинальская  шапка»,  так  как  если  бы  он  был  просто  монахом,
подобно  Р.Бэкону, то  имел  бы больше  времени  для научных  занятий.
Его реформаторская  деятельность не сопровождалась  крупными успехами.
Серьезной реформе католичество подверглось  только тогда, когда грянул
гром Реформации.

Очень любят  говорить, что идеологический подъем  Возрождения связан с
крупнейшими событиями, отмечающими конец  Средних веков и пробудившими
человечество  от  средневекового  сна: открытие  Америки,  изобретение
книгопечатания и микроскопов. Они открыли миры, неизвестные ни Библии,
ни  Аристотелю,  и  тем  заставили усомниться  в  непогрешимости  этих
авторитетов.  Но  Кузанский  работал   раньше  всех  этих  событий,  и
он  продолжал  работу  критиков перипатетизма  и  полную  реабилитацию
платонизма,   интенсивное  знакомство   с  которым   началось  задолго
до   Возрождения.  Возрождение   и   гуманизм   были  подготовлены   в
недрах  схоластики,  конечно  в  лице  крупнейших  ее  представителей,
пользовавшихся,  однако, огромным  авторитетом в  просвещенных кругах.
Представление о  том, что схоластика была  совершенно чужда прогрессу,
основано  просто на  невежестве. На  таком же  невежестве, принимающем
резкую ломку идей в конце Средних веков, основана и известная легенда,
что проект Колумба  об открытии Америки (по его мнению,  пути в Индию)
был отвергнут на том основании, что шарообразность Земли --- нелепость
и  противоречит Священному  Писанию. Я  знаю, что  есть даже  картина,
изображающая  торжественное заседание  Саламанкского университета.  Но
вот  что  мы читаем  в  статье  «Колумб  и его  открытие»  (Магидович,
1956,  с. 14):  «Необходимо  заметить, что  вымышленной  от начала  до
конца является  версия о торжественном заседании  совета Саламанкского
университета, на  котором якобы  был отвергнут  проект Колумба  на том
основании,  что ученые  мужи  были возмущены  соображениями Колумба  о
шарообразности  Земли. Никакого  торжественного заседания  не было,  и
комиссия  Талаверы  разрешала  вопрос в  келейной  обстановке;  притом
заключение ее последовало лишь спустя  четыре года после того, как она
начала свою работу. Следует отметить,  кстати, что к концу XV столетия
доказательства  шарообразности Земли  были настолько  убедительны, что
оспаривать их вряд ли решился бы какой-либо церковник, претендующий на
ученость.  Напротив, церковь  старалась в  то время  примирить данные,
подтверждающие  шарообразную форму  Земли, с  библейскими концепциями,
ибо прямое  отрицание истины,  которая стала уже  общеизвестной, могло
повредить ее  авторитету, и  без того  уже пошатнувшемуся».  Ясно, что
прогрессивные идеи постепенно проникали в сознание культурных людей, и
католическая церковь не была чужда доводам разума. А откуда же взялась
саламанкская  легенда? Конечно,  и  тогда были  невежды и  обскуранты,
многие из них занимали очень  видное общественное положение. Вот по их
высказываниям и судили о мировоззрении людей Средневековья. Но если мы
применим  этот метод,  то и  о культуре  нашего времени  можно создать
весьма  невысокое  мнение,  так  как лица,  достигшие  высшей  власти,
порют  совершенную  дичь, с  апломбом  выступая  по вопросам  науки  и
искусства. Вот для убеждения  таких людей --- высокопоставленной черни
---  убедительными  оказываются  такие  факты,  как  открытие  Америки
и  изобретение  микроскопа.  Культурные  люди в  таких  аргументах  не
нуждаются.

Экономический  расцвет итальянских  городов  и  городов других  стран,
внешние  ослепительные события,  а главное,  идеологическая подготовка
средневековых мыслителей и создали  предпосылки для необычайно бурного
расцвета времен Ренессанса. Обычная схема, противополагающая Ренессанс
Средневековью, изложенная в первом параграфе этой главы, рассматривает
мрачное чудовище Средневековья,  пораженное насмерть блестящим рыцарем
Ренессанса. Если  уж пользоваться аналогиями, то  лучше будет сказать:
из куколки  Средневековья, не  проявлявшей как будто  признаков жизни,
вылупилась внезапно блестящая бабочка Ренессанса, но если бы в куколке
не шли сложные процессы гистолиза, никакой бабочки не получилось бы. А
внешние условия?  Конечно, они нужны.  Если куколку держать  на морозе
или  подвергнуть  слишком  высокой  температуре,  никакой  бабочки  не
получится.

\subsection{Материалисты  средневековья не  оставили следа  в развитии
точных  наук. Ренессанс  --- порождение  идеалистической, христианской
идеологии}

Все развитие идей, кратко прослеженное  на предыдущих страницах, шло в
рамках  разных форм  идеализма ---  платоновского и  аристотелевского.
Конечно, у ряда авторов  можно усмотреть материалистические тенденции,
но основа всегда  идеалистическая. А были ли  настоящие материалисты в
Средние  века и  какова их  роль? Конечно,  могут сказать,  где же  им
быть,  когда все  задавлено  церковью, однако  мы  знаем, что  крайние
атеисты были  и в Средние  века. Им  удавалось даже писать  и издавать
(потом  даже печатно)  произведения,  сохранившиеся доныне,  например,
знаменитое сочинение  «Де трибус импосторибус» (о  трех обманщиках ---
подразумеваются  Моисей,  Христос,  Магомет),  изданное  анонимно,  но
приписывалось некоторыми  (по-видимому, ошибочно)  императору Фридриху
II Гогенштауфену,  который был  эпикурейцем и в  теории и  на практике
и  полагал,  что никакой  иной  жизни  не  последует за  этой,  земной
(Тарле, 1901,  с. 165).  Открытая борьба  папы и  императора, гвельфов
и  гибеллинов,   проходила  красной  нитью  через   историю  Италии  в
течение нескольких  столетий. Фридрих II, «эманципированный  от старых
верований,  не  заменивший  их  никакими  новыми,  лишенный  вместе  с
тем  каких  бы то  ни  было  принципов, кроме  принципа  преследования
собственной выгоды» (Тарле, 1901, с. 151), страстный охотник, не лишен
был научных интересов. Он оставил сочинение  по охоте, где (в XIII в.)
уже выставил положение об увеличении размеров теплокровных животных по
направлению  к северу,  в XIX  в. вновь  открытое Бергманом.  Никакого
следа  в  точных  науках  материалисты  Средних  веков  как  будто  не
оставили,  и  это  неудивительно,  так как  для  законов,  управляющих
Вселенной,  необходимо было  твердое убеждение  в существовании  таких
законов и  в возможности  их открытия  человеком. Это  убеждение тесно
связано с разными формами  пифагореизма, законным наследником которого
в  Средние века  была  католическая церковь.  Поэтому  можно с  полной
уверенностью сказать,  что Ренессанс стал возможен  только потому, что
почва для его развития была подготовлена христианской верой.

\begin{flushright}  Конец   четвертой  главы.   26  февраля   1963  г.
\end{flushright}

\clearpage

\section{V. ЛИНИИ В АСТРОНОМИИ. 2. Коперник и Бруно}

1. Имя Коперника настолько громко,  что в комментариях не нуждается, и
в мою задачу, конечно, не входит стремление сколько-нибудь умалить его
роль.  Но  в  соответствии  с  общим  планом  данного  сочинения  надо
разобрать вопрос  о той идеологии, которой  руководствовался Коперник.
Верно ли, что  он --- «хотя и робко, после  36-летних колебаний и, так
сказать,  на  смертном  одре  --- бросил  вызов  церковному  суеверию»
(Энгельс, 1949, с. 153); верно ли,  что его система --- начало научной
астрономии вместо «обветшалой  системы Птолемея»? Ввиду исключительной
важности роли  Коперника в истории  не только астрономии, но  и вообще
точной  науки  разберем  сначала  общий  характер  этой  замечательной
личности,  его предшественников,  научную  и  философскую стороны  его
труда,  а  затем разберем  его  отношения  с католической  церковью  и
политическими и общественными движениями его времени.

Коперника часто называют «баловнем  судьбы», никогда не знавшим заботы
о хлебе насущном  (Идельсон, 1947, с. 18), и  формально это правильно.
Копернику очень  повезло, так как  его дядя был  епископом Вармийским,
сделавшим  его  «каноником»  с   хорошей  «пребендой»,  т.е.  солидным
содержанием.  Очень  часто  Коперника  называют  священником.  Это  не
совсем верно.  Он был  каноником, духовным  лицом, но  не священником.
Материальное положение его было прекрасное, но он носил знак духовного
звания  --- тонзуру  (выбритое  на  макушке место)  и  был обречен  на
обязательное  безбрачие. Из-за  этого в  конце жизни  (когда умер  его
постоянный  покровитель, епископ  Вармийский),  он  даже вынужден  был
расстаться со  своей преданной внучатой племянницей,  так как правящие
круги  нашли  неудобным, что  с  духовным  лицом проживает  молодая  и
красивая женщина, не находящаяся с ним в близком родстве.

Слово  «каноник» в  случае  Коперника  обозначает: член  правительства
теократического государства. Родина его, Вармия, после 13-летней войны
западнопрусских  городов формально  перешла как  вассальная область  к
Польше, но  фактически осталась  самостоятельной церковной  областью и
правитель  ---  епископ  Вармийский назначался  непосредственно  папой
иногда даже  вопреки желанию польского  короля, как это и  случилось с
дядей  Коперника,  епископом  Лукой  Ваценролом.  Каноники  ---  члены
капитула совещательного органа при епископе (Ревзин, 1949).

Поскольку большинство  людей поддается развращающему  действию власти,
постольку многие каноники не пользовались  высокой репутацией и были в
ходу поговорки: «Толст как каноник», «Блудлив как каноник» (Ревзин, с.
183), и в частности брат  великого Коперника, Андрей, тоже проведенный
в каноники, ничем  не отметил своей деятельности.  Николай же Коперник
был не  только ученый  астроном, но проявил  изумительно разнообразную
деятельность. По своему образованию был  не только, вернее, не столько
астроном, как церковный юрист и  медик и притом пользовавшийся большой
известностью в  своей стране. Учился  он долго:  родился в 1473  г., а
диплом  доктора  канонического права  получил  в  1503 г.  в  Ферраре.
Возможность так долго учиться сначала в Кракове, где он познакомился с
современной  ему астрономией,  потом  в  Болонье каноническому  праву,
потом  в Падуе,  где он  учился  медицине, была  связана опять-таки  с
покровительством его  дяди, который, наверное, и  не подозревал, какие
услуги человечеству он оказывает, покровительствуя своему племяннику.

В   связи  с   изучением   канонического  права   Коперник  в   Италии
хорошо  усвоил  греческий  язык  и мог  в  оригинале  познакомиться  с
творением великих  эллинов: Платона,  Аристотеля и  других. Знакомство
с  древнегреческой   литературой  произвело  на  него   столь  большое
впечатление,  что  первым  его   печатным  произведением  был  перевод
Феофилакта  Симокатского,   напечатанный  в   1509  г.   Этот  перевод
характерен:  он показывает,  что Коперник  не  был в  стороне от  того
широкого гуманистического движения,  которое характеризует Ренессанс с
его  интересом  к классической  литературе,  прежде  всего к  Платону.
Сборник (И.И.Толстой, 1947)  состоит из 89 не связанных  друг с другом
небольших писем, будто бы принадлежащим разнообразным деятелям Греции.
Любопытен список  авторов, включающих философов,  ученых, политических
деятелей: Фалес, Сократ, Платон, Диоген, Антисфен, Эратосфен, Архимед,
Пириандр,  Фемистокл,   Перикл,  Кимон,  Клеон,   Алкивиад,  Дионисий,
Эсхин, Исократ,  Аспазия. Наблюдается  полное презрение  к хронологии:
Антисфен пишет Периклу, говорит об Александре Македонском. Преобладают
морально-философские  темы, но  есть и  любовные письма.  Имя Платона,
конечно, окружено глубоким уважением, и ему принадлежит заключительное
письмо,  адресованное Дионисию:  «Если ты  стремишься преодолеть  свою
печаль,  то ходи  прогуливаться среди  могил,  и это  исцелит тебя  от
страданий: ты увидишь, что владевшие даже величайшими богатствами люди
горстью  пыли  владеют  по  ту сторону  гроба».  И.И.Толстой  (с.  82)
полагает, что  выпуском глубоко языческого сочинения  Коперник решился
на смелый  шаг и  «косвенно заявлял  себя сторонником  новых взглядов,
чуждых средневековому  миросозерцанию, отходил  от реакционных  сил и,
как  бы  отмежевываясь  от  схоластиков, позволял  причислить  себя  к
лагерю гуманистов».  И.И.Толстой в стремлении отмежевать  Коперника от
«реакционеров»,  по-видимому,  забыл,  когда  писал  эти  строки,  что
знаменитая  Ватиканская  библиотека была  основана  в  середине XV  в.
папой Николаем  V, скупавшим  в Византии,  не щадя  средств, греческие
манускрипты, что Коперник познакомился, конечно, с этой библиотекой за
время своего визита  в Рим в 1500  г. и что папы  того времени, отнюдь
не  отличаясь высокой  моралью,  были в  большей  или меньшей  степени
(например, Лев X) покровителями  классической литературы и вообще были
гуманистами  в  смысле  поддержки  наук и  искусств.  Поэтому  перевод
Феофилакта свидетельствует о том, что Коперник интересовался не только
наукой, но и художественной литературой,  но никак не указывает на его
особенную смелость:  тут он  решительно ничем не  рисковал. Любопытно,
что  тому  же Копернику  инкриминируется  «робость»  там, где  никакой
робости не было, но об этом будет дальше.

2. Интерес  к искусству не  ограничивался у Коперника  литературой. Он
был также художником, и в биографиях его приводят его автопортрет.

Практическая  его  деятельность  не  исчерпывалась  тем,  что  он  был
для  своего времени  выдающимся  врачом. Он  был выдающимся  инженером
(Нестерук,  1955).   Он  был  администратором   коммунальных  владений
Вармийского церковного  управления, и в круг  его деятельности входила
организация городского  водоснабжения. В 12 городах  им были построены
гидротехнические сооружения.  Во Фромберке при  сооружении водопровода
впервые  вода  была  поднята  на  высоту  до  25  метров.  Сохранилось
предание,   что  образец   водопроводного   механизма  Коперника   был
затребован для Людовика XIV при устройстве водоснабжения в Марли.

Но  великий  Коперник  был  и  выдающимся  экономистом.  Он  занимался
вопросами денежного обращения  и написал в 1519 г.  трактат «О чеканке
монет», где впервые был  сформулирован открытый им экономический закон
о  том, что  «худшие деньги  вытеснят лучшие»,  закон, который  обычно
связывается  с именем  более молодого  английского экономиста  Грешема
(1512--1579)  (Щеглов, 1954,  11). Коперник  не сумел  только провести
свой  закон в  жизнь,  благодаря сословному  эгоизму  купцов и  дворян
(Ревзин,  с. 289).  Любопытно,  что провести  в  жизнь идеи  Коперника
и  добиться  того,  чтобы  монета была  доброкачественной,  удалось  в
другой стране,  Англии, великому последователю Коперника  Ньютону. Два
гения,  успешно  размышлявшие о  небесных  делах,  оказались и  весьма
«практическими» людьми, в то  время как многие экономисты, гордившиеся
своим «практицизмом», не могли как следует разобраться в этом деле.

Крупным  политическим деятелем  Коперник проявил  себя во  время войны
Польши  с  Орденом.  Ввиду  отсутствия  епископа  и  других  каноников
в  это  время  Коперник  оказался  фактически  единовластным  хозяином
Вармийской области.  Так как он  и раньше был  наместником Вармийского
капитула  в Ольштыне,  то он  заранее подготовился  к осаде  оружием и
продовольствием,  и Ольшанский  замок  --- единственный  в Вармии,  не
захваченный Орденом  в 1521  г. За  воинскую доблесть  польский король
Зигмунд наградил Коперника титулом комиссара Вармии (Ревзин, 270).

Мы видим,  таким образом, что  несправедливо думать так, как  поется в
старой студенческой песне:

Коперник весь свой век трудился, Чтоб доказать Земли вращенье.

Верно,  что, начав  свое  образование в  Краковском университете,  где
преподавание астрономии  стояло на наивысшем для  того времени уровне,
Коперник  всю  жизнь интересовался  астрономией  и  занимался ею  и  в
Италии, так сказать,  «вне плана», но у него  было огромное количество
посторонних дел, и только малую  часть своего времени он мог посвятить
любимой науке. В этом уже заключается  одна из важных причин того, что
к опубликованию своего труда в завершенном виде он приступил только на
склоне  дней. Для  него  служба  каноника не  была  синекурой, и  этот
«баловень  судьбы» с  лихвой  возместил своему  народу и  человечеству
получаемую  пребенду. После  первой  своей  трехлетней командировки  в
Италию  (где он  в Болонье  изучал  церковное право)  он вернулся  без
диплома, «не  выполнил плана», так  как занимался и астрономией,  но в
общем  можно  сказать, что  он  хорошо  усвоил все  предметы,  которые
изучал  в  Италии,  и  ознакомился  с  современной  ему  культурой  во
всем  ее  разнообразии. Как  подчеркивает  его  биограф Гассенди  (сам
выдающийся ученый, один из основоположников научного атомизма), работа
Коперника  шла в  трех направлениях:  1) он  чрезвычайно добросовестно
исполнял свои  пастырские обязанности,  2) не отказывал  в медицинской
помощи  бедным людям  и 3)  остальное время  посвящал науке  (Сказкин,
1947). Астроном, медик, инженер,  экономист, государственный и военный
деятель,  художник, знаток  классической  литературы ---  вот кто  был
Коперник. Его  можно назвать представителем гуманизма  в самом широком
смысле  слова. Роднит  его с  гуманизмом и  та черта,  что он,  будучи
богословом  по образованию,  не  был творческим  мыслителем в  области
теологии, в отличие, например, от Николая Кузанского и Исаака Ньютона.
Однако он оставался верным  сыном католической церкви: никаких намеков
на  отход  от  католического  учения  и  на  какую-либо  склонность  к
протестантству и другим религиозным нововведениям мы не обнаруживаем.

Коперник  полностью использовал  свое пребывание  в Италии  --- центре
культуры  того   времени,  переживавшей  эпоху  бурного   расцвета.  В
Болонье  было не  менее  15 тысяч  студентов:  даже видные  профессора
конкурировали, чтобы  привлечь студентов,  так как их  труд оплачивали
сами студенты.  А в Падуе за  каждую лекцию, на которую  пришло меньше
шести студентов,  профессор платил штраф в  шесть лир; ясно, что  в то
время много заботились о внешней красоте изложения (Ревзин, с. 149). В
философском  отношении господствовал  Аристотель  и его  последователи
(Аверроэс  и  др.),  но  с чрезвычайной  силой  развивался  интерес  к
Платону, чем, как известно, характеризуется все Возрождение.

3. Кто же был предшественником  Коперника, от какого учения он исходил
в своей  работе? Астрономическое  образование он  получил в  Кракове у
Брудзиевского.  Незадолго до  этого Региомонтаном  были изданы  в 1478
г.  составленные  его учителем  Пурбахом  «Новые  теории планет»,  где
излагается  теория Птолемея.  «Появление этой  книги знаменует  начало
возрождения  точной астрономической  науки  в Европе...  По нему-то  и
ведет в Кракове преподавание Брудзиевский... Таким образом, не может и
подлежать сомнению, что Коперник изучает  астрономию там, где она едва
ли  не лучше  всего преподавалась  в Европе  к концу  XV в.  Отсюда он
выносит --- и это на всю жизнь --- глубокое уважение к тому важнейшему
запасу наблюдений, который нам оставили древние, к строгости и глубине
методов, которые  они применяли» (Идельсон,  1947, с. 10). Но  как уже
было указано в четвертой главе, Птолемея критиковали и перипатетики, в
особенности Аверроэс, который был особенно популярен в Падуе. Коперник
в этом споре решительно отстаивал значение Птолемея, и когда в 1537 г.
появился перевод с  арабского трактата, автор которого  не скупился на
нападки по адресу  Птолемея, Коперник на титульном  листе под фамилией
автора характеризовал  его как  «крупного клеветника на  Птолемея». Из
другого источника известно, что  Коперник говорил: «Я направляю стрелы
в ту  же цель и  тем же  методом, что и  Птолемей, хотя лук  и стрелы,
которыми  он  пользовался,  сделаны  из  совершенно  иного  материала»
(Идельсон,  1947,  с. 11).  Как  увидим  дальше, знаменитое  сочинение
Коперника очень многое использует из трудов Птолемея.

Обучение  в  Кракове  страдало  одним  дефектом:  там  не  преподавали
греческого языка.  Этот дефект  был устранен  в Болонье,  где Коперник
сумел познакомиться  в оригинале  с сочинениями Платона,  Аристотеля и
древних поэтов.  Знакомство с греческим  языком привело, как  уже было
указано, к появлению перевода на  латинский язык сборника Феофилакта и
дало ту  эрудицию в классической литературе,  которой Коперник блистал
в  своих  сочинениях.  Философия  Платона и  его  научная  методология
наложили  глубокий отпечаток  на всю  работу Коперника;  подробнее это
будет показано  дальше. Ознакомление  с древней  литературой позволило
Копернику найти своих предшественников в античности.

Коперник  совершенно не  упоминает  в  числе предшественников  Николая
Кузанского, хотя, казалось бы, он должен был быть знаком с его именем,
если  не  с  его  сочинениями.  Ведь  и  Кузанский,  и  Коперник  были
привлекаемы к работе  по реформе календаря. Думаю,  что причиной того,
что  Кузанский не  упоминается  Коперником, было  то,  что Коперник  в
первую очередь  заботился о математической  разработке космологической
теории, а на такую разработку, как  было показано в четвертой главе, у
Николая  Кузанского просто,  видимо, не  хватало времени.  Существенно
различна и основная точка зрения на  Вселенную. Как было указано в §71
и 72  главы четвертой, Кузанский признавал  бесконечность Вселенной, а
тогда  нельзя  говорить  ни  о каком  центре  Вселенной.  Коперник  же
считал Вселенную ограниченной и  вместо геоцентрической системы строил
гелиоцентрическую.

Коперник совершенно  ясно перечисляет своих  античных предшественников
(Коперник,  1947,  с. 199--200):  «А  так  как  небо есть  общее,  все
содержащее и таящее  в себе вместилище, то отнюдь не  видно, почему не
приписать  движение скорее  содержимому, чем  содержащему, вмещенному,
чем  вмещающему.  Такого  мнения действительно  держались  пифагорейцы
Гераклит и Экфант, и,  согласно Цицерону, Никет Сиракузский, придавший
Земле вращение в центре мира».  Что касается движения по орбите Земли,
то Коперник пишет:  «Действительно, о том, что Земля  вращается и даже
различным  образом блуждает,  и о  том,  что она  принадлежит к  числу
светил, утверждал пифагореец Филолай,  столь недюжинный математик, что
именно ради  свидания с ним  Платон не замедлил отправиться  в Италию,
как передают жизнеописатели Платона». Для  Коперника ясно, что все его
предшественники  были  пифагорейцами,  не  совсем ясно,  почему  он  в
своем  сочинении  не  упоминает  Аристарха  Самосского.  Копернику  не
только  было известно  имя Аристарха,  но  он даже  упоминает его  как
предшественника в первоначальном тексте своей книги: эта рукопись была
обнаружена в  одной из библиотек в  Праге (см. Идельсон, 1947,  с. 29;
Веселовский,  1961,  с. 70).  Ясно,  что  Коперник не  имел  намерения
скрывать Аристарха (да  такие намерения и не  подходили его моральному
облику),  так как  упоминание о  нем вошло  в рукопись,  переданную им
другому  лицу.  Видимо, он  считал  возможным  не упоминать  Аристарха
потому, что математическая  теория гелиоцентрической системы Аристарха
остается неизвестной,  дошедшие до  нас сочинения  Аристарха построены
еще на  основе геоцентрической системы, а  само отрицание геоцентризма
проводилось и до Аристарха многими пифагорейцами.

Коперник не упоминает своего современника Челио Кальканьини (Идельсон,
1947, с. 16), да по всей вероятности  он о нем и не слыхал. Около 1525
г. Кальканьини  составил небольшую работу  над названием: «О  том, что
небо  неподвижно, а  Земля вращается,  или О  вечном движении  Земли».
Кальканьини начинает с утверждения, что  вовсе не все небо со звездами
и планетами с  невероятной скоростью вращается вокруг  Земли в течение
суток,  а  вращается  Земля,  но после  этого  утверждения  начинается
довольно  слабая аргументация  в его  защиту. Работа  Кальканьини была
опубликована в 1544 г. после смерти ее автора, как и Коперника.

4. Перейдем  теперь к краткому  изложению того вклада,  который сделал
Коперник.  Тут  мы  имеем  дело, во-первых,  с  фактическими  данными,
которые были  в распоряжении Коперника, математическим  характером его
теории и философской основой. Математические и философские постулаты у
Коперника  в  сильной  степени  переплетены, и  полное  разделение  их
невозможно, но все же в  изложении будем придерживаться по возможности
этого порядка.

Широко распространено мнение,  что всякий крупный шаг  вперед связан с
увеличением  числа фактов  и с  их уточнением.  Во многих  случаях это
действительно  так,  но  в случае  Коперника  широко  распространенное
мнение, что  он обосновал свою систему  на многочисленных наблюдениях,
более  точных,  чем  у   его  предшественников,  совершенно  ошибочно.
В  его  книге  упоминается  лишь  о  27  выполненных  им  наблюдениях,
и,  быть  может, около  20  наблюдений  им  было заимствовано  от  его
предшественников.  Он не  гнался  и за  большой точностью  наблюдений.
Использованные  им положения  звезд содержат  ошибки до  40 минут,  за
которые его  с полным основанием  мог бы упрекнуть  знаменитый древний
астроном Гиппарх, наблюдавший  много раз точнее за  полторы тысячи лет
до  Коперника.  (Щеглов,  1954,  с.  11). И  не  следует  думать,  что
после  Гиппарха  точность  инструментов  упала.  А  просто  деревянные
инструменты Коперника, которые  он делал сам, были  очень примитивны и
во многом  уступали инструментам обсерватории Улутбека  (1394--1449) и
Наиср-эл-Дина Туси  (1201--1274), а  также тем, которые  в то  время с
большим  искусством  изготовлялись  нюрнбергскими  мастерами  (Щеглов,
1954,  с. 10).  Ученик  Коперника, Ретик,  убеждал Коперника  повысить
точность  своих  наблюдений. «На  это  Коперник  отвечал, что  высокая
точность пока ему  не нужна. На первое время  ему достаточно убедиться
лишь  в  приближенном  совпадении  теории  с  наблюдениями».  Это  ---
замечательный  ответ. Масса  точнейших наблюдений  понадобилась тогда,
когда  началось  тщательное  изучение отдельных  особенностей  системы
Коперника, приведшее к открытию законов Кеплера, Ньютона, установлению
аберрации  света,  обнаружению  звездных  параллаксов,  поразительному
отысканию  Нептуна и  другим открытиям.  Для установления  же основных
положений Солнечной  системы обилие  фактов и  погоня за  их точностью
могли лишь затруднить задачу и даже сделать ее неразрешимой. Копернику
уже  тогда  было ясно  то,  что  ясно ныне  представителям  настоящих,
точных,  наук и  что большей  частью неясно  ретивым «фактистам»,  что
прогресс теоретической науки идет с максимальной эффективностью тогда,
когда  количество и  точность фактов  соответствуют очередной  задаче,
подлежащей разрешению, и что чрезмерное количество и чрезмерно высокая
точность могут быть не пособием, а препятствием.

Почему же  система Птолемея,  господствовавшая в разных  странах более
тысячи лет, требовала  ревизии? Практически она была  полезна, и новая
система Коперника лишь немного улучшала, а в некоторых отношениях даже
ухудшала  расчеты. Нельзя  было  просто  отвергнуть систему  Птолемея,
что  она и  слишком  сложна из-за  множества  деферентов и  эпициклов,
так  как,  как  увидим  ниже, система  Коперника  несколько  упростила
сложность системы,  но вовсе не  отказалась от эпициклов.  Кроме того,
постулат «природа проста» во многих случаях оправдывается, но не может
считаться  чем-то абсолютным.  Но сам  Птолемей пришел  к формулировке
таких  закономерностей, которые  из его  системы вовсе  не вытекают  и
которые заставляют искать много решений.  Об этом вкратце было сказано
в  §16 главы  IV. Перечислим  эти затруднения  по Идельсону  (1947, с.
254): 1) почему Марс, Юпитер и  Сатурн оказываются ближе всего к Земле
(в перигеях  эпициклов) только  при противостояниях  их с  Солнцем; 2)
почему  радиусы-векторы,  проведенные  из  центров  эпициклов  к  этим
трем  планетам,  всегда параллельны  между  собой  и параллельны,  как
подробно доказывает Птолемей, направлению от Земли к Солнцу; 3) почему
эпицикл Марса огромен,  эпицикл Юпитера меньше, а  эпицикл Сатурна еще
меньше; 4) почему  центры эпициклов Меркурия и Венеры  лежат всегда на
одной  прямой,  соединяющей  глаз  наблюдателя с  Солнцем,  а  времена
обращения  этих эпициклов  как раз  равны  одному году;  5) почему  ни
Солнце,  ни Луна  в системе  Птолемея не  получают попятных  движений?
Ясно, что  ревизия давно  господствующих положений показана  не только
в  тех  случаях,  где  имеются противоречия  и  несоответствия  теории
с  наблюдениями,  но  и  в  тех случаях,  где  наблюдения  приводят  к
нахождению  закономерностей,  новых,  не предусмотренных  теорией.  Но
как  далеко должна  идти ревизия?  Среди культурных  людей, осознавших
необходимость  пересмотра, можно  выделить два  типа, которых  уместно
назвать реформаторами и революционерами.

5. Реформаторы  считают необходимым  преобразование только  в пределах
совершенно  назревшей необходимости  и не  забегают вперед.  Не строят
такую  систему,  которая  не  обоснована  хорошо  известными  фактами.
Революционеры  же   несколько  забегают  вперед  и   производят  такую
радикальную перестройку, которая во  многих частях только впоследствии
получит  обоснование.  В  нашем случае  революционером  был  Коперник,
на  позиции  же реформаторов  стоял  не  признавший полностью  системы
Коперника   выдающийся  астроном   Тихо  Браге,   своими  наблюдениями
подготовивший следующий крупный этап  в развитии космологии, сделанный
Кеплером,  также   человеком  революционного  духа.   Это  подчеркнуто
Идельсоном  (1947, с.  172):  «Но читатель,  несомненно, заметил,  что
система,  к которой  мы  пришли  после геометрического  преобразования
системы Птолемея, есть  не что иное, как  та ``промежуточная'' система
Тихо  Браге», о  которой  шла речь  в §8.  Отсюда  вытекает, что  если
бы  в  нашем  распоряжении   были  только  и  единственно  определения
направлений на  Солнце и  планеты, то  решить вопрос  о том,  какая из
обеих  систем  имеет  место в  действительности,  «промежуточная»  или
коперниковская, было бы невозможно. По  счастью, мы обладаем и другими
наблюдениями,  из которых  видно, как  отображаются движения  Земли на
положениях планет  (звездные параллаксы),  имеются и  чисто физические
доказательства движения Земли (аберрация света); наконец, динамическая
теория прецессии,  развиваемая на основе закона  всемирного тяготения,
совершенно  исключает   возможность  «промежуточной»   системы.  Таким
образом, скромная  роль последней не идет  дальше того геометрического
преобразования, которое  есть лишь первый  этап к переходу  от древней
системы  к  новой. Копернику  принадлежит  открытие  истины; а  именно
этот  характер  своей  теории,  в  корне  отличающей  ее  от  древней,
Коперник  с  глубокой  прозорливостью подчеркивает  уже  в  посвящении
своей книги  Павлу III.  Можно ли так  решительно в  смысле истинности
противопоставлять  системы Тихо  Браге  (большинство планет  вращается
вокруг  Солнца,  которое,  в  свою очередь,  вращается  вокруг  Земли)
системе  Коперника?  Никто  сейчас  не защищает  систему  Тихо  Браге,
но  никто  сейчас не  защищает  и  систему  Коперника, если  под  этим
подразумевать гелиоцентрическую систему в том  виде, в каком ее создал
Коперник, да в  сущности никто сейчас не  защищает и гелиоцентрическую
систему  вообще,  если  под  таковой  подразумевать  признание  Солнца
центром  Вселенной (как  думал  Коперник). Даже  в пределах  Солнечной
системы, как известно, Солнце не  является точным центром, так как оно
находится в фокусе, а не в центре эллипсов, и планеты вращаются вокруг
общего центра  тяжести, который  почти (но  не абсолютно)  совпадает с
центром  тяжести  Солнца.  Но  Земля  после  Коперника  потеряла  свое
преимущественное  положение между  планетами и  в этом  широком смысле
все  последующие  системы  могут  быть  названы  коперниковскими,  как
апоцентрические или антигеоцентрические.  Поэтому при сравнении систем
Тихо Браге и Коперника ни одна  из них не может быть названа истинной,
если  под  «истинной»  подразумевать  абсолютно  правильную  и  точную
теорию.  Но система  Браге являлась,  так сказать,  оппортунистическим
усовершенствованием устаревшей системы,  а система Коперника открывала
широкий  путь  к  дальнейшему  развитию  и  предвидела  многое  такое,
что  в  то время  было  неизвестно  и  не  могло быть  известно  ввиду
недостаточной  точности наблюдений:  ведь  оптических инструментов  не
было  ни  у  Коперника,  ни  у  Тихо  Браге,  да  и  первые  телескопы
были  весьма несовершенны.  Вот в  этом пророческом  характере системы
Коперника и  заключается ее основное преимущество  перед системой Тихо
Браге.  Система Браге  появилась после  Коперника, но  почти не  нашла
почитателей,  хотя с  точки зрения  отсутствия противоречий  с многими
фактами (например,  знаменитый аргумент,  который был  разрешен только
Галилеем, что если бы система Коперника была справедлива, то мы должны
были  бы наблюдать  разницу в  скорости полета  птиц в  зависимости от
направления  полета) система  Браге  была менее  уязвима, чем  система
Коперника.  Но  консерваторы  науки   не  считали  необходимым  вообще
реформировать системы  Аристотеля и  Птолемея, а  те, кто  уже вкусили
коперниковского духа,  те, конечно, не считали  возможным остановиться
на промежуточной системе и,  невзирая на существовавшие трудности, шли
дальше по пути, указанном Коперником.

Широко распространено в кругах, далеких от науки, мнение, что Коперник
отбросил  систему Птолемея,  а вместе  с тем  все предпосылки  древней
науки.  Это  совершенно  неверно.  Он  твердо  придерживался  древнего
положения  о  необходимости  признавать  только  равномерные  круговые
движения.  Как пишет  Идельсон (1948,  с.  115): «В  вопросе об  общих
свойствах  планетных движений  великий реформатор  астрономии держался
существенно  ближе  к  догматике   древней  философии,  чем  Птолемей.
Несомненно,  в этом  отношении  сыграла роль  та ожесточенная  критика
птолемеевской  системы,  которую   Проповедовали  арабские  мыслители,
в  особенности  Аверроэс,  с   учениями  которых  Коперник  несомненно
детально ознакомился  в годы своего падуанского  студенчества; так или
иначе допущение  неравномерных круговых  движений (каковыми  в системе
биссекции было  движение точки  Р в птолемеевском  эксцентре) казалось
ему  совершенно  несовместным  с  разумной  системой  астрономического
знания».  Он поставил  себе  задачу изыскать  иное  решение и,  будучи
превосходным  кинематиком,  дал  вместо  биссекции  эксцентриситета  и
экванта систему двойного параллелограмма.

Идельсон дает чертежи  трех схем, которые придумал  Коперник и которые
он назвал: 1) эпи-эпицикл, 2) эксцентр-эпицикл и 3) эксцентр-эксцентр.
«Сам Коперник колебался в выборе  схем. Сначала он принимал для Марса,
Сатурна  и Юпитера  первую схему,  а потом  принял вторую.  Выявленное
Коперником деление  полного эксцентриситета на четыре  части позволило
ему достичь  только несколько худшего,  чем у Птолемея,  приближения к
истинным  кеплеровским законам  движения  планет (максимальная  ошибка
превосходит  птолемеевскую  в  два  раза,  поэтому  эта  теория  имела
бы  только  исторический  интерес,  если  бы,  развивая  ее,  Коперник
не  обнаружил,  что  орбита  планеты Р  не  может  представлять  собой
окружности, являясь эпициклоидой). Поэтому в итоге, спасая аксиоматику
древней  астрономии, поскольку  она  касается принципа  равномерности,
Коперник  должен  был  отойти   от  нее,  показав,  что  действительно
движение планеты вокруг  центра всех обращений не  может совершаться в
точности по  круговой орбите.  Но какова  истинная форма  этой орбиты,
было  раскрыто  только через  65  лет  после появления  трактата  ``Об
обращениях небесных сфер'' в книге, по праву носившей название ``Новая
астрономия( Кеплера)''» (Идельсон, с. 123).

В этом  превосходно видно огромное преимущество  точных математических
теорий  перед приблизительными.  И  Птолемей, и  Коперник исходили  из
одних  и  тех  же  постулатов,  но  Птолемею  пришлось  отказаться  от
равномерного,  Копернику  ---  от  точно  кругового  движения.  Третий
участник этой великой  эстафеты науки, Кеплер, сумел  найти исход, как
будто отказавшись от  обоих положений древних. На самом деле  он их не
отверг, а  преобразовал. Движение планет  проходит не по кругам,  а по
эллипсам,  иначе  говоря,  по  одному из  канонических  сечений:  тоже
правильная геометрическая фигура. Равномерность понимается не так, что
в равное  время проходится равное  расстояние или равный угол,  а так,
что  в равные  времена проходятся  равные секторы  площади эллипса.  И
наличие  правильной геометрической  орбиты  и  равномерность (в  новом
смысле) сохраняются.

7. Всем  известно, что огромная  заслуга Коперника заключается  в том,
что  он вывел  Землю  из  неподвижного состояния  и  установил те  две
формы  движения (вокруг  оси и  по орбите),  которые сейчас  считаются
общепринятыми.  Но, вероятно,  не многим  известно, что  Коперник ввел
и  третье  движение: годовое  вращение  земной  оси на  360  градусов.
Ненужность третьего движения была показана почти одновременно Кеплером
и Галилеем.

Это  связано с  тем,  что  Коперник не  довел  до  конца свой  принцип
относительности движения  и он у  него упирается в  схему неподвижного
абсолюта. Он принимал неподвижность восьмой  сферы, которая все в себе
заключает. Все это ---  непосредственный отклик философских убеждений,
что  вся  небесная  сфера  была  создана  богом,  для  нас,  чтобы  по
сравнению с ними, несомненно неподвижными на своих местах, мы могли бы
определить положения  и движения  других заключенных  в ней  сфер. Эта
философская концепция  Коперника приводит  к одной из  странных частей
его системы  звезд. Птолемей  дает первый  каталог 1022  звезд, относя
положения звезд  по долготе к  равноденствию его эпохи (140  в. н.э.).
Коперник  дает  долготы тех  же  звезд,  но стремится  создать  вечный
каталог  и  потому  отсчитывает  долготу от  определенной  звезды.  Ни
Коперник, ни  Ретик не  подозревали, что  звезды могут  быть подвижны.
Поэтому сейчас после Коперника в этом отношении вернулись к Птолемею и
все  звездные  каталоги  относят  к  равноденствиям  эпохи  наблюдений
(Идельсон, 1947, с. 35--37).

Мы  видим,  что  в  систему   Коперника  вошло  чрезвычайно  много  из
системы  Птолемея, и  можно прямо  сказать, что  не будь  эта огромная
подготовительная  работа  сделана  Птолемеем, вряд  ли  Коперник  смог
написать  свое   сочинение.  В  систему  Коперника   вошли  не  только
многочисленные  данные  наблюдений  его предшественника,  но  и  целый
ряд  элементов планетных  движений. Например,  из четырех  Птолемеевых
элементов  движения  планеты   три  оказываются  гелиоцентрическими  и
только один  --- радиус эпицикла (дельта)  имеет чисто геоцентрическое
значение: раскрытие его смысла  и значения в гелиоцентрической системе
и является для теоретической  астрономии самым существенным моментом в
той  революции знания,  которую  принесла  книга Коперника  (Идельсон,
1947, §9).  Сам Коперник в  V главе указывает, что  «система планетных
теорий  ``Альмагеста'' отнюдь  не  терпит изъяна  и  ущерба от  нового
учения  о  движении  Земли;   весь  численный  скелет  древней  теории
сохраняется  неизменным,  а  методика   ее  упрощается  и  улучшается,
поскольку  Птолемеевы  экванты  уничтожаются и  заменяются  движениями
более  совершенными  ---  круговыми  и равномерными;  к  тому  же  ряд
движений теперь  вовсе отпадает, поскольку все  они учитываются единым
движением Земли» (Идельсон, 1947, с. 164).

Это, конечно,  не умаляет  значения Коперника;  все эти  данные только
восстанавливают истинное значение «Альмагеста»:  это не бредни варвара
и  не   грезы  пифагорейца,   а  подлинная   теоретическая  астрономия
(Идельсон, с.  178). Но раскрытием реального  значения величины дельта
(радиус эпицикла) Коперник  указал на общую меру  расстояний от Солнца
как  для планет,  так  и  для Земли.  Это  использовал впервые  только
Кеплер.  Как  это  ни  может показаться  странным,  особого  прогресса
по  сравнению  с Птолемеем  Коперник  достиг  в IV  книге,  касающейся
движения  Луны.  В этой  книге  нет  ничего парадоксального,  так  как
Луна  всегда  считалась  вращающейся  вокруг  Земли  и  здесь  вопросы
гелиоцентрической  теории   роли  не  играют  и   имеет  место  только
техническая  переработка  птолемеевской   теории.  Устранив  эквант  и
заменив  его биэпициклом,  Коперник достигает  не только  одинаковой с
Птолемеем  точности, но  и существенно  улучшает лунные  расстояния. И
в  то  время как  здесь  у  Птолемея  получалось нечто  совершенно  не
соответствующее действительности, Коперник  дает результаты, близкие к
действительности.  Высокое  мастерство  IV  книги  по  праву  вызывало
восхищение современников.

8. Копернику была совершенно ясна огромность Вселенной, хотя, как было
указано  выше,  он  не  дошел до  признания  Вселенной  бесконечной  и
принимал восьмую  сферу неподвижных  звезд. Из  огромности неподвижной
сферы  звезд он  и делал  заключение  о том,  что вертится,  очевидно,
Земля,  а  не  вся  сфера:  «Гораздо  более  удивительным  должен  нам
представляться круг  обращения в  течение 24 часов  такой необъятности
мира, а не  столь малой части его, т.е. Земли»  (Коперник, с. 215). Но
Коперник  не отказывался  от  понятия центра  мира  и того  положения,
что  Земля  не  так  уж  далека  от этого  центра:  «Пусть  она  и  не
находится в  центре мира, однако  ее расстояние  от него будет  все же
неизмеримо  мало, в  особенности  по сравнению  со сферой  неподвижных
звезд»  (Коперник, с.  202). Как  увидим дальше  в главе,  посвященной
космологии XIX  и XX вв., эта  идея об относительной близости  Земли к
центру Вселенной не является столь нелепой, как кажется многим.

Введение  гелиоцентрической  системы, конечно,  значительно  упростило
структуру теории, но она осталась все-таки очень сложной. Сам Коперник
в конце  «Малого комментария», указывая на  достоинство своей системы,
говорит, что ему оказалось достаточным  34 кругов, чтобы объяснить все
строение Вселенной  и всю  пляску планет (Ревзин,  1949, с.  219). Эта
сложность и незначительное повышение точности вычислений и было важным
препятствием для принятия системы  Коперника, которую оценили сразу по
достоинству  лишь немногие  ученые.  Как было  уже  указано, в  учении
Коперника  были и  прямые ошибки:  1) он  сохранил верность  круговому
и  равномерному  движению, что  и  связано  с большой  сложностью  его
системы; 2) признавал абсолютную неподвижность звезд; 3) вводил третье
движение  Земли;  4)  мыслил  создать  вечный  каталог  звезд,  исходя
из  положения  неподвижности  звезд;  5)  не  принимал  прямолинейного
движения  как «естественного»  и  др.  Я уже  не  говорю  о его  общем
антропоцентрическом  взгляде  на  Вселенную,  так как  это  уже  чисто
философский,  а не  физический  или  математический постулат.  Поэтому
в  отношении  Коперника, как  и  в  отношении решительно  всех,  самых
величайших  гениев человечества,  можно  сказать:  велик Коперник,  но
какое было бы  несчастье для человечества, если бы какая  не по разуму
«прогрессивная» власть объявила бы  его учение окончательной истиной в
последней  инстанции, не  подлежащей никакой  ревизии. Движение  науки
остановилось  бы.  К  счастью,  этого  не  случилось  и  последователи
Коперника творчески  развивали его  учение, а не  догматизировали его.
Как было  уже указано, в  некоторых отношениях Птолемей  оказался даже
более правым (в звездном каталоге). Это --- один из примеров того, что
прогрессивное  учение  никогда не  бывает  на  100\% прогрессивным,  и
поэтому, следуя  диалектическому закону  развития идей,  следующий шаг
в  известном отношении  оказывается  возвращением  к пройденному  уже,
казалось бы, этапу. Это часто забывают иногда штатные диалектики, хотя
по этому  вопросу (для частного случая  биологической эволюции) хорошо
сказано  у Ф.  Энгельса  (Диалект, природы.  Борьба за  существование,
с.  249):  «Главное  тут  то,   что  каждый  прогресс  в  органическом
развитии  является  вместе  с  тем  и  регрессом,  ибо  он  закрепляет
одностороннее  развитие,  исключает  возможность  развития  во  многих
других направлениях.  Но это  основной закон». Само  собой разумеется,
что  это  не  означает,  что  положительные  и  отрицательные  стороны
какого-либо нового явления друг  друга уравновешивают. Это означало бы
полное отсутствие реального прогресса. Но правильно то, что нет такого
в общем прогрессивного  явления, к которому не  примешивались бы черты
регресса  и, наоборот,  в  самых реакционных  явлениях можно  отыскать
положительные черты. Игнорирование этого общего положения и приводит к
тому, что  догматизация нового прогрессивного явления  влечет за собой
то, что с особой силой защищает именно то, что защищать не следует.

9. Любопытно сопоставить в этом отношении один из основных аргументов,
который  древние,  в  частности Птолемей,  приводили  против  вращения
Земли и  который был окончательно  опровергнут только Галилеем.  И уже
давно, продолжает  Птолемей, «Земля, распавшись, пробила  бы небо (что
совершенно  смехотворно)  и тем  более  живые  существа и  все  прочие
свободные тяжести никоим образом не  остались бы не сброшенными с нее,
да  и  отвесно  падающие  тела  не попадали  бы  прямо  по  отвесу  на
назначенное им место, уже отнесенное  прочь с этой огромной скоростью.
К  тому  же и  облака  и  все парящие  в  воздухе  предметы мы  видели
бы  несущимися  всегда  к  западу»  (Коперник, с.  204).  Как  на  это
возражение отвечает Коперник (там  же): «Но если предполагать вращение
Земли, надо  непременно признать, что  это движение естественно,  а не
насильственно. Ибо  все, к чему  применена сила или  напор, непременно
должно распасться, и оно не в  состоянии долго пребывать в целости; но
то, что происходит естественным путем, пребывает в совершенной целости
и сохранности.  Поэтому Птолемей напрасно опасается  рассеяния Земли и
всего  земного превращения,  происходящего  силой природы,  совершенно
отличной от силы искусственной или могущей быть созданной человеческим
умом.  Но  почему же  не  предполагать  этого  в еще  большей  степени
относительно  Вселенной, движение  которой  должно  быть настолько  же
быстрее, насколько небо больше Земли?  Или небо стало необъятным из-за
того, что  оно с  несказанной силой движения  отделяется от  центра, а
иначе, будь оно недвижимо, оно бы рухнуло?».

Аргументы обеих  сторон не  могут считаться вполне  убедительными. Как
нередко  бывает  (не  только  в  отношении  высших  положений  разума,
как  думал  Кант), оба  противоположных  взгляда  антиномичны, т.е.  и
тот,  и другой  наталкиваются  на противоречия.  Птолемей,  как и  все
перипатетики, аргументирует от повседневного опыта: быстро вращающиеся
тела  склонны распадаться;  и при  такой скорости,  с какой  вращается
Земля,  она должна  распасться.  Коперник  приводит мало  убедительный
аргумент о  различии естественного и искусственного  движения. Возьмем
утверждение, что  «то, что происходит естественным  путем, пребывает в
совершенной  целости и  сохранности».  Но,  например, живые  организмы
возникают  вполне  естественным  путем,   однако  все  они  подвержены
медленному разрушению. Сторонники геоцентризма утверждают, что если бы
небо не вращалось, оно бы  рухнуло (позабывая свои собственные доводы,
что чрезмерно быстрое вращение должно вызвать распадение тел), но ведь
утверждение  Коперника о  неподвижных звездах  сейчас опровергнуто,  а
то,  что  ему  казалось  так  нелепым  ---  «что  небо  с  несказанной
силой движения  отделяется от центра»,  --- как будто  получило полное
подтверждение в современной теории разбегающейся Вселенной.

Действительное   подтверждение  движения   Земли   по  орбите   вокруг
Солнца  пришло  значительно  позже.   Уже  убедившись  в  правильности
гелиоцентрической    системы,   астрономы    искали   параллактическое
(периодическое  годичное)   смещение  звезд  на  небесной   сфере,  но
долгое  время  их  поиски были  безрезультатными  из-за  недостаточной
точности  инструментов. Но  английский астроном  Брадлей, не  сумевший
найти  параллактическое  смещение,  нашел  другое  явление,  столь  же
доказательное,  именно  годичную  аберрацию.  Звезда  в  течение  года
как  бы описывает  круг  диаметром примерно  в 40  секунд.  В 1728  г.
Брадлей  дал  правильное  объяснение  этому  явлению,  истолковав  его
как  результат  сочетания скорости  движения  Земли  по ее  орбите  со
скоростью света  (Фесенков, 1949,  с. 30,  31). Таким  образом, первое
чисто  астрономическое доказательство  движения Земли  по орбите  было
получено через 185 лет после  смерти Коперника. Что касается годичного
параллакса, то его удалось установить еще на столетие позже. В 1837 г.
профессор  Дерптского (Юрьев,  ныне Тарту)  университета В.  Н. Струве
нашел периодическое  смещение Беги от  одной из соседних звезд  в 0,25
секунды,  т.е.  годичный параллакс  равен  0,125  секунды. Почти  одно
временно  годичный параллакс  был  найден Бесселем  в Кенигсберге  для
звезды  созвездия Лебедя  и  Генларсеном на  Капской обсерватории  для
Альфы  Центавра. Последняя  звезда оказалась  самой близкой  к нам,  и
потому  это  открытие  было  сделано с  менее  точными  инструментами.
Инструменты  же  для  Струве  и Бесселя  были  изготовлены  знаменитым
Фраунгофером (Фесенков,  стр. 50, 51).  Разумеется, в XIX в.  никто из
астрономов не сомневался в  справедливости гелиоцентрической системы и
искание  годичных параллаксов  преследовало не  цель «доказательства»,
а  совсем  другую  цель:  установление истинных  расстояний  звезд  от
нашей планеты.  Как и  предвидел Коперник,  оно оказалось  огромным и,
несомненно, гораздо более огромным, чем предполагали самые смелые умы.

Поэтому совершенным  курьезом звучат слова Энгельса  (История ВКП( б).
Краткий  курс,  с. 108):  «Солнечная  система  Коперника и  в  течение
трехсот  лет  оставалась гипотезой,  в  высшей  степени вероятной,  но
все-таки  гипотезой.  Когда  же  Леверье,  на  основании  данных  этой
системы,  не  только  доказал,   что  должна  существовать  еще  одна,
неизвестная до тех пор, планета, но и определил посредством вычисления
место,  занимаемое ею  в небесном  пространстве, и  когда после  этого
Галле  действительно   нашел  эту  планету,  система   Коперника  была
доказана». Открытие Нептуна произошло в 1846 г., примерно через десять
лет после открытия параллаксов  звезд. Подробнее нам придется говорить
об этом в  главе о Ньютоне, когда придется разобрать  вообще вопрос об
оценке  предсказаний. Здесь  же  ограничимся тем,  что  видеть в  этом
году  окончательное доказательство,  теории Коперника  невозможно, тем
более, что  на основе собственных  теорий Коперника (не  принимавшей в
расчет  взаимного влияния  планет) предсказать  открытие Нептуна  было
невозможно.  Наиболее  же курьезно  то,  что  римская курия  пришла  к
признанию справедливости гелиоцентрической теории  раньше 1846 г., так
как в списке запрещенных книг 1821 г. еще фигурируют книги Коперника и
Галилея,  а в  списке 1835  г.  они уже  отсутствуют (Фесенков,  1949,
с.  36). Материалисты  оказались  более скептическими  по отношению  к
теории  Коперника,  пример  чему  можно  видеть  и  в  основоположнике
материализма Фр.  Бэконе, и это  неудивительно, так как  в обосновании
гелиоцентрической  теории играли  огромную  роль убеждения,  противные
материалистической  философии,  данных  же  опыта  и  наблюдения  было
слишком мало,  чтобы решительно склонить большинство  ученых в сторону
новой теории.

10.  Коперник,  как  и  все революционеры  науки,  отличался  истинной
свободой   мышления.  Ученик   Коперника,  Ретик   справедливо  избрал
эпиграфом для  «первого повествования» греческий стих:  «Надлежит быть
свободным мыслию  тому, кто желает достичь  мудрости» (Идельсон, 1947,
с. 33). Но  свободу мысли невозможно понимать так, что  ученый в своей
работе не  принимает никаких исходных  постулатов и подходит  к работе
совершенно  без всяких  «предвзятых  мыслей».  Такую свободу  пытаются
защищать некоторые чрезмерные  последователи индуктивного мышления, но
они просто не  осознают тех предвзятых мыслей, которые  есть у каждого
человека. Истинная  свобода мысли:  осознание тех  положений, которыми
руководствовались предшественники, продумывание тех положений, которые
могут   прийти  им   на  замену   и  критическое   пользование  новыми
постулатами. Огромное  значение имеет также вера  в справедливость тех
или иных  положений, которыми  ученый руководствуется в  своей работе.
Широкая образованность  Коперника, долгое пребывание его  в самой гуще
интеллектуальной жизни  того времени, спор различных  философских школ
заставили  его,  как  и большинство  передовых  деятелей  Возрождения,
решительно  встать на  сторону  платоновской  философии. Поэтому  «его
обоснование революционного изменения имело, по существу, философский и
эстетический характер» (Бернал, 1956, с. 223). Оно было в значительной
степени «умозрительным», но в  соответствии с общим духом платоновской
Академии умозрение контролировалось данными  опыта. Напомним эту чисто
платоновскую установку: «гипотезы равномерных и упорядоченных движений
надо  сформулировать  так,  чтобы  их следствия  не  противоречили  бы
явлениям» (История философии, т. I, с. 252).

Эпиграфом своей  бессмертной работы  Коперник поставил  девиз Платона:
«Да не входит никто, не знающий математики», и в строго математическом
духе   написано  все   его  сочинение.   Выпишем  полностью   короткую
первую  главу  сочинения  Коперника  (Коперник,  с.  194):  «Глава  1.
О  шарообразности  Вселенной.  Прежде  всего нам  следует  принять  во
внимание то, что Вселенная шарообразна,  как потому, что шар --- самое
совершенное по форме  и не нуждающееся ни в  каких скрепах безупречное
целое,  так и  потому,  что  из всех  фигур  это самая  вместительная,
наиболее подходящая  для включения и сохранения  всего мироздания; или
еще потому, что все самостоятельные части  Вселенной --- я имею в виду
Солнце, Луну  и звезды --- мы  наблюдаем в такой форме,  как это видно
по  капле  воды  и  остальным  жидким телам,  когда  они  стремятся  к
самозавершению. Поэтому никто не  усомнится, что таковая форма присуща
небесным телам».

Глава  2, о  том,  что  сферическую форму  имеет  и Земля,  начинается
словами: «Земля тоже шарообразна, потому что со всех сторон тяготеет к
своему центру. Тем не менее ее совершенная округлость заметна не сразу
из-за большой высоты  ее гор и глубины долин,  что, однако, совершенно
не искажает ее  округлости в целом»; и уже  далее приводятся известные
опытные  доказательства: на  разных  широтах мы  видим разные  звезды,
неодинаковость  времени затмений,  видимость  земли  с верхушки  мачты
корабля и пр.

Мы  видим уже  здесь  ясно  пифагорейско-платоновское мировое  зрение:
Космос  совершенен,  поэтому  он  должен быть  сферичен.  Все  ---  от
Вселенной до капли воды --- стремится к совершенной форме.

Совершенно ясны  философские корни  и для того  своеобразного различия
кругового и  прямолинейного движения,  которое приводил  Коперник (там
же, с.  206): «Итак,  утверждение, что  простому телу  присуще простое
движение, подтверждается прежде всего движением круговым, пока простое
тело пребывает на  своем месте и в своем единстве.  Потому что на этом
месте  его нет  другого  движения, кроме  кругового, которое  остается
целиком  в самом  себе, подобно  состоянию покоя.  В прямолинейное  же
движение приходит то,  что уходит со своего  естественного места, либо
сталкивается, либо так  или иначе оказывается вне его. А  ничто так не
противоречит порядку  и форме мироздания,  как быть вне  своего места.
Поэтому в  прямолинейное движение не приходит  ничто, кроме предметов,
находящихся  в  ненадлежащем  положении и  несовершенных  по  природе,
отделяющихся от своего  целого и теряющих с ним  единство». Идельсон в
примечании  17 к  этому  месту пишет:  «Это  своеобразное отношение  к
прямолинейным движениям, унаследованное Коперником  от эстетики сфер и
круговых движений Платона и  Аристотеля, укоренилось настолько прочно,
что даже  Галилей, создатель динамики, повторяет  эти слова Коперника»
(в «Диалогах»).

Очень  любопытно,  почему   Коперник  считает  прямолинейное  движение
неестественным:  «Кроме   того,  движение  чего-либо  вверх   и  вниз,
в  отличие   от  кругового,  не  является   простым,  единообразным  и
равномерным,  ибо  оно  не  способно быть  размеренным  легкостью  или
напором веса. И  все, что опускается, двигаясь  вначале медленнее, при
дальнейшем падении  увеличивает скорость. С другой  стороны, мы видим,
как  наш земной  огонь (а  другого мы  не знаем),  взметнувшись вверх,
сразу поникает, как бы указывая  на причину насилия в земном веществе.
Круговое  же движение  протекает  всегда равномерно,  ибо его  причина
неослабеваема. Причины  же тех  ускоряющихся движений  прекращают свое
действие на тела  по достижении ими своего места,  когда они перестают
быть легкими или  тяжелыми, и движение их  останавливается. Итак, если
движение Вселенной круговое, а движение ее частей еще и прямолинейное,
мы можем сказать, что круговое движение сочетается с прямым, как живое
существо  с  болезнью».  Ясно, почему  Коперник  считал  прямолинейное
движение неестественным: ему уже было известно ускорение при свободном
падении,  а ускоряющееся  движение не  может ускоряться  беспредельно.
Вечное  движение может  быть  только равномерным,  а таковым  является
только круговое  как совершенное  по природе.  Учение «о  своем месте»
предметов --  явно аристотелевское. Еще яснее  аристотелевское влияние
видим в  главе десятой  о порядке небесных  орбит (Коперник,  с. 211):
«Солнце  мы примем  неподвижным,  и на  этом  основании все  кажущиеся
движения могут быть  объяснены движением Земли. Радиус  ее орбиты, как
он ни велик,  все же ничтожен сравнительно  с расстояниями неподвижных
звезд, с  этим можно согласиться  тем легче, что это  пространство все
наполнено множеством  орбит, что допускают даже  те, которые принимают
Землю  за центр.  Нужно  взять  пример с  природы,  которая ничего  не
производит лишнего, ничего бесполезного, а, напротив, из одной причины
умеет выводить множество следствий.  Все это покажется неудобопонятным
и  даже  невероятным, но,  с  Божьей  помощью,  мы докажем  это  яснее
солнца, по крайней  мере, для знакомых с  математикой». Это знаменитое
изречение: «Природа  ничего не  делает напрасно» приведено  в качестве
основных положений и Ньютоном.

11.  Великолепное  эстетическое   обоснование  центральному  положению
Солнца со ссылкой на языческие  авторитеты (что кажется непонятным для
современников: как мог так  говорить каноник католической церкви) дано
на с. 212: «В середине всех  этих орбит находится Солнце, ибо может ли
прекрасный этот  светоч быть  помещен в  столь великолепной  храмине в
другом, лучшем месте, откуда он мог  бы все освещать собой? Поэтому не
напрасно называли Солнце душой Вселенной, а иные --- Управителем мира:
Трисмегист  называет  его ``видимым  Богом'',  а  Электра Софокла  ---
``Всевидящим''. И  таким образом  Солнце, как  бы восседая  на царском
престоле, управляет  вращающимся около  него семейством  светил. Земля
пользуется услугами Луны и, как выражается Аристотель в трактате своем
``Де анималибус'', Земля имеет наибольшее сходство  с Луной. А в то же
время  Земля оплодотворяется  Солнцем и  носит в  себе плод  в течение
целого года. Этот порядок  обусловливает собой удивительную симметрику
мироздания  и  такое  гармоническое  соотношение  между  движениями  и
величинами орбит, какого мы другим образом найти не можем».

Все это, не говоря уже о том,  что имя Платона упоминается в ряде мест
с  неизменным  уважением,  свидетельствует  о  том,  что  платоновская
философия  и  была  источником  идей Коперника,  и  прежде  всего  его
знаменитый  диалог «Тимей».  Об  этом  уже было  сказано  в §25  главы
четвертой  со  ссылкой  на   учителя  Коперника  в  Болонье,  Кодруса.
Копернику не было  надобности много раз ссылаться на  Платона, так как
после хорошего ознакомления  с Платоном в XV в.,  увлечение Платоном в
Италии  и  вообще  среди  образованных  людей  Европы  было  всеобщим.
Козимо  Медичи  учредил  знаменитую Платоновскую  Академию,  и  Платон
в  значительной  мере  стал  предметом  моды.  В  церквах  произносили
проповеди, в  которых вместо  Евангелия священники  цитировали диалоги
Платона.  Во  дворцах считалось  признаком  хорошего  тона ввернуть  в
светский разговор несколько фраз из Платона (Ревзин, 1949, с. 150). Не
всегда мода оказывается глупостью,  иногда модное оказывается наиболее
прогрессивным.

Известно,  что в  период  Возрождения  авторитет Аристотеля  подвергся
сильному ущемлению  в пользу Платона,  но Коперник вовсе  не разделяет
того  полного отрицания  Аристотеля,  которое было  выражено в  тезисе
Рамуса (1536): «Все, чему  учил Аристотель, является ложным». Критикуя
Аристотеля и Птолемея, Коперник сохраняет  очень многое от обоих своих
предшественников.  О Птолемее  говорилось  раньше. Коперник  разделяет
чисто аристотелевские положения о тяжести и легкости, о «своем месте»,
«природа не  делает лишнего»  и опровергает  Птолемея и  Аристотеля их
же  постулатами.  Ученик Коперника,  Ретик,  был  убежден, что  доводы
Коперника убедили  бы Аристотеля и  Птолемея (Ревзин, с.  345). Борьба
Коперника  шла  не  с  самим  Аристотелем, а  главным  образом  с  его
эпигонами, которые, как это свойственно большинству эпигонов, забывали
многое  ценное своих  учителей  и  пытались догматизировать  ошибочные
их  мнения. Поведение  эпигонов  не  изменилось и  в  XX  в. Столь  же
критическое объективное отношение (а  не полное отрицание) сохранил по
отношению  к Аристотелю  и Галилей.  Но  вот влияния  Демокрита и  его
последователей на  Коперника установить  совершенно невозможно,  и всё
пифагорейско-платоновское  мировоззрение  великого астронома  стоит  в
решительном  антагонизме с  демокритовским. Имя  Демокрита упоминается
как  будто только  однажды  в перечне  мнений о  форме  Земли (см.  §8
главы  четвёртой),  где  мнение  ряда  лиц,  в  том  числе  Демокрита,
противополагается мнению «философов».

Но  мировоззрение Коперника  заключало в  себе и  черты, казалось  бы,
совершенно ретроградные. Широко  распространено мнение, что астрономия
и  астрология глубоко  антагонистичны  и что  астрология могла  только
мешать  развитию   научной  астрономии.  Однако   Коперник  совершенно
искренне верил в астрологию. Это убеждение он заимствовал, конечно, не
от  Платона,  а  от  своего болонского  учителя,  Деменико  ди  Новара
(1451--1504),  который  в Болонье  читал  основной  курс астрологии  и
обязан был по уставу  университета давать астрологические предсказания
студентам  и  общие  предсказания   на  год  вперед.  Новара  приковал
Коперника к колеснице астрологии на  всю жизнь. Ревзин пишет (с. 155):
«Не удивительно ли, что великий  преобразователь науки о небе не сумел
освободиться от  наваждения звездочетства?  Перевернув вверх  дном все
старые представления об устройстве Вселенной и движений планет, он так
и продолжал считать,  что планета Меркурий властвует  над гадалками, а
Венера над парикмахерами...  Незадолго до смерти Коперника  с его слов
было  записано  его учеником  Ретиком:  «Мы  видим, что  все  монархии
начали свое  существование, когда центр эксцентрического  круга Солнца
находился на  вершине малого  круга. Римская империя  стала монархией,
когда эксцентриситет  Солнца был особенно  велик. А с  его уменьшением
эта империя,  старея, становилась  все более слабой  и в  конце концов
погибла... Этот маленький круг  представляет собой колесо счастья. Его
вращение вызывает появление и изменение  мировых империй. В этом круге
заключены  все события  мировой истории».  Астрологические взгляды  не
помешали  Копернику сделаться  одним из  величайших астрономов,  и это
понятно,  так  как  основной  принцип  астрологии  ---  взаимодействие
всех  планет  и  важность   математической  обработки  наблюдений  ---
стимулировал, а не мешал  развитию математической астрономии. Суеверия
же, связанные с астрологией,  постепенно отсеялись сами собой. Вредило
развитию астрономии лишь чисто  меристическое понимание мира и неверие
в мощь математики.

12.  Уже   изложенного,  казалось  бы,  достаточно,   чтобы  прийти  к
убеждению, что  система Коперника  вовсе не была  «вызовом», брошенным
великим   ученым  церковному   суеверию  согласно   хорошо  известному
выражению  Энгельса  (гл.  IV,  §1).  Учение  Коперника  не  имеет  ни
следа материалистического  или атеистического мировоззрения.  Но ввиду
исключительного   интереса  и   важности  вопроса   полезно  отношение
Коперника к  религии, и в  частности к католической  церкви, разобрать
детально.

Во  всей своей  деятельности  Коперник проявлял  себя  как верный  сын
католической  церкви,  но,  видимо,   не  проявлял  большого  интереса
специально к  теологическим вопросам. Мне неизвестно,  чтобы, при всем
разнообразии  его деятельности,  он  оставил специально  теологические
сочинения.  В   этом  он  отличается  от   активных  теологов  Николая
Кузанского  и Исаака  Ньютона. Ревзин  (1949, с.  361), как  и следует
ожидать  от  советского  писателя,  пытается  усомниться  в  том,  что
Коперник  был  «добрым  католиком».  Основанием  для  такого  сомнения
являются два соображения: 1)  Современниками Коперника были развратные
папы Иннокентий VIII (умер в 1492 г.) и знаменитый своим распутством и
вероломством Александр  VI Борджиа (1492--1503). По  Ревзину (с. 153):
«Жить  по  соседству  с  развратным, алчным,  лишенным  малейшей  тени
добродетели Ватиканом  и не  преисполниться скептическим  отношением к
церкви мог только ленивый разумом и чувствами человек. Коперник не был
таковым».  2)  Резкая  критика  Коперником  одного  из  отцов  церкви,
Лактанция,  что потом  вызвало гнев  инквизиторов. Ревзин  пишет (стр.
361): «Не поразительно ли, что  человек, проживший тридцать лет в тени
собора, сорок лет бывший каноником,  мог в таких выражениях отозваться
о  Лактанции,  этом  ``христианском Цицероне'',  ``Григории  Богослове
Западной церкви''? Как  можно после этого говорить о  Копернике, как о
``добром канонике'', что делают многие буржуазные исследователи? Он им
никогда не был». Разберем оба этих довода.

Конечно, распущенность  Римской церкви, длившаяся  столетиями, сыграла
немалую роль в кризисе католичества, но из этого вовсе не следует, что
только  ленивые  разумом и  чувствами  люди  могли оставаться  верными
католиками.  Когда мощное  идейное  направление  переживает кризис,  в
частности  из-за  недостойного  поведения руководящих  деятелей  этого
направления,  то   в  зависимости  от  темперамента   и  иных  качеств
современников можно  различить три  основных формы реакции  на кризис:
1)  Крайние заключают  из  недостойного  поведения руководителей,  что
все  это  направление основано  на  обмане  и подлежит  ликвидации.  В
случае с религиями оно  выражается в полной антирелигиозности. Пример:
знаменитый антирелигиозный  трактат Средневековья «О  трех обманщиках»
(Де трибус импосторибус), под которыми подразумевались Моисей, Христос
и Магомет.  2) Другие сохраняют  верность основам учения,  но считают,
что распущенность возникла как следствие отказа от «истинного» учения,
которое необходимо восстановить: разные формы реформации, или ревизии.
3) Наконец, многие считают, что дело  не в основах учения, которые они
не  считают нужным  ревизовать,  а в  организационных формах  общества
и  в  человеческих  слабостях,  с которыми  должна  идти  борьба...  И
это  ортодоксальное направление  опять распадается,  по крайней  мере,
на  три  ветви:  а)  Если считают  кризис  следствием  ревизионистских
попыток  и потакания  человеческим  слабостям, с  которыми надо  вести
беспощадную  борьбу в  интересах  спасения от  гибели большинства;  на
католической почве это направление в особенно яркой форме выразилось в
ордене доминиканцев, основатели и главные представители которого были,
несомненно,  людьми, твердо  верившими  в правоту  своих взглядов,  не
только гнавшими на  костер тех, кого они считали врагами  церкви, но и
сами готовые  бестрепетно войти на  костер во имя своих  убеждений, б)
Другое направление, родственное  первому, считает недостаточным борьбу
с инакомыслящими, но требует  крупных общественных реформ, не отступая
и от подлинно революционных мероприятий. Наиболее ярким примером здесь
является  доминиканец Савонарола,  в) И  наконец, третье  направление,
может быть, самое обширное, считает, что исправление недостатков может
быть достигнуто медленной и  упорной борьбой без эффектных жестокостей
и  без революционных  потрясений.  Светлым примером  этого рода  можем
считать уже  разобранного в конце четвертой  главы Николая Кузанского.
Кузанский  работал  и  как  философ,  и  как  богослов,  политический,
общественный  и научный  деятель.  Другие  сосредоточиваются на  более
узком диапазоне:  дела милосердия (Франциск Ассизский  и др.), научная
деятельность и проч. У Коперника, как мы видели, дел было по горло и в
общественной жизни.

Все перечисленные направлениям вполне  могут совмещаться с честностью,
т.е.  твердой  убежденностью  исповедника  тех  или  иных  взглядов  в
истине  своих  убеждений.  Но  наряду с  этим  имеется  большое  число
приспособленцев,   лицемеров  и   просто   жуликов.  Известна   фраза,
приписываемая папе  Льву X  Медичи: «Будем пользоваться  папством, раз
Господь  Бог нам  его  дал». Такие  фигуры, как  Александр  VI, Лев  X
и  другие,  конечно,  производят  впечатление, что  в  Ватикане  живут
только развратные, алчные, лишенные  малейшей добродетели деятели; как
я  постараюсь  дальше  показать,  это  первое  впечатление  совершенно
неверно. Всякая  власть благоприятствует развращению, и  если исходить
из  конкретных носителей  власти, то  легко прийти  к самому  крайнему
анархизму, но как  нас учит история, крайние анархисты  сплошь и рядом
обладают теми же пороками, что  и критикуемые ими носители власти. Как
бы то  ни было, внешние  и внутренние факторы  привели к тому,  что за
последующие столетия католическая церковь избавилась от многих пороков
времени  Возрождения  и  сейчас представляется  вполне  жизнеспособным
организмом.

13. Также  нет оснований  сомневаться в добром  католичестве Коперника
на  основании довода  «от  Лактанция». Лактанций  был ранний  богослов
(IV  века),  перешедший  в  христианство из  язычества.  Коперник  его
упоминает в  посвящении папе  Павлу III (Коперник,  с. 193):  «Не дабы
убедить  ученых  и  неученых  в  том,  что  я  не  боюсь  суждений,  я
посвящаю мои  исследования не  кому другому, как  твоему святейшеству,
досточтимому  и  в  обитаемом  мною отдаленном  уголке  земли  как  по
высокому твоему  званию, так  равно и  по любви  твоей к  математике и
прочим наукам,  в надежде, что  влияние и суждение твое  легко защитят
меня  от придирок  завистников,  хотя, по  пословице, против  укушения
интриганов и  нет средств.  Если бы  нашлись пустые  болтуны, которые,
хотя  вовсе не  сведущие в  математических науках,  дозволили бы  себе
осуждать или опровергать мое предприятие, намеренно искажая какое-либо
место  Священного Писания,  то я  не стану  на них  обращать внимания,
а,  напротив, буду  пренебрегать  подобным  неразумным суждением,  ибо
небезызвестно,  что   знаменитый  Лактанций,  не   особенно,  впрочем,
сведущий в  математике, довольно  ребячески рассуждал о  фигуре Земли,
насмехаясь над теми, которые ее  считали шаровидною. Потому люди науки
и  не  должны  удивляться  тому, что  мыслящие  таким  образом  станут
насмехаться  над  ними.  Математические  предметы  пишутся  для  одних
математиков,  а  последние,  если  я  не  совершенно  ошибаюсь,  будут
того  мнения,  что мои  исследования  могут  приносить пользу  церкви,
тобою  управляемой.  Ибо когда  несколько  лет  тому назад,  во  время
Льва X  рассуждалось на  Латеранском соборе об  исправлении церковного
летоисчисления, то задача эта осталась  в то время нерешенною и именно
по той  причине, что тогда  еще не  были в состоянии  точно определить
продолжительность года и  месяца, а равно и движение Солнца  и Луны. С
тех  пор, побуждаемый  к  тому досточтимым  Павлом Фоссомбрийским,  на
которого  было возложено  это дело,  я старался  подробнее исследовать
вопрос. Что удалось  мне сделать в этом  отношении, представляю судить
твоему святейшеству и прочим ученым математикам...»

Из этого  посвящения ясно,  что Коперник считает  Павла III  не только
почтенным папой, но  и ученым математиком, и вряд  ли последний эпитет
мог быть  направлен в виде  лести. Мы знаем, что  противники внедрения
математики  в  науке существуют  и  в  наше  время. С  другой  стороны
ясно,  что  если бы  критика  Лактанция  была указанием  на  «недоброе
католичество», то вряд ли умный Коперник мог совершить такую глупость,
чтобы поместить критику в  посвящении, адресованном главе католической
церкви. Но  кто такой  был Лактанций?  Как уже  было указано,  это был
ранний  богослов.  Как  новообращенный  христианин,  он,  естественно,
основную свою задачу видел в пропаганде богословских и этических основ
христианства.  В этом  отношении он  и его  сторонники возобновляли  в
еще  более решительной  форме мнение  Сократа, что  единственно ценная
философия  та,  которая учит  нас  нашим  нравственным обязанностям  и
религиозным надеждам  (Уэвель, 1867, с.  324) Лактанций и  считал, что
вся философия этих других людей суетная и ложная. «Разыскивать причины
естественных  вещей,  исследовать,  так  ли  велико  Солнце,  как  оно
кажется, выпукла  Луна или вогнута,  остаются ли звезды  неподвижны на
небе или  плавают свободно в  воздухе, как велико  небо и из  чего оно
сделано, остается  ли оно в покое  или движется, как велика  Земля, на
каких  основаниях она  повешена и  находится в  равновесии; спорить  и
делать предположения  об этих предметах  значит совершенно то  же, как
если бы  мы стали рассуждать,  что мы  думаем о каком-нибудь  городе в
отдаленной стране,  о котором  мы не знаем  ничего, кроме  жизни». Но,
как  было  показано  в  §53--57 четвертой  главы,  такой  обскурантизм
был  характерен не  для всей  христианской  церкви, а  лишь для  части
ее  представителей, причем  обскурантами  были  не только  религиозные
мыслители.

14.  Возьмем, например,  суждение Лактанция  об антиподах  (Уэвель, с.
326): «Возможно ли людям быть столь безумными, чтобы верить, что жатвы
и деревья на другой  стороне земли висят вниз и что  у людей ноги выше
их голов?  Если вы спросите  их, как они защищают  подобные нелепости?
Как вещи  не падают с  Земли на той стороне?  --- Они ответят,  что по
природе  вещей  тяжелые тела  стремятся  к  центру, подобные  ступицам
колеса, тогда как легкие тела, например, облака, дым, огонь, стремятся
от  центра  к небу  во  всех  сторонах.  Но  я действительно  не  могу
придумать,  что  сказать  о  людях,  которые,  впавши  раз  в  ошибку,
упрямо продолжают  заблуждаться и одно нелепое  мнение защищают другим
нелепым мнением».  Поразительно, как  совпадает мнение об  антиподах у
христианского богослова  и атеиста  Лукреция (см.  цитату в  §41 главы
четвертой). С другой стороны, даже у разных христианских богословов не
было  единого  мнения ни  о  форме  Земли,  ни  об антиподах.  Тот  же
Уэвель  (с. 327--328)  указывает, что,  например, Августин  не отрицал
шарообразность  Земли (а  авторитет  Августина  стоял неизмеримо  выше
авторитета Лактанция и заслуженно), но отрицал существование антиподов
просто  потому,  что  о  них  не упоминается  в  Писании,  а  Иероним,
видимо,  использовал  доводы  того  же Писания  (о  двух  херувимах  с
четырьмя лицами) в пользу существования антиподов. Спор продолжался, и
например, в VIII столетии Бонифаций, архиепископ Майнцский, узнав, что
епископ  Виргилий защищает  существование  антиподов, вознегодовал  на
это  мнение,  принимавшее  мир человеческих  существ,  которые  должны
находиться  вне спасения,  и  просил папу  Захария наказать  человека,
который держался столь  опасного учения. Но Виргилий не  только не был
лишен епископства, о чем неправильно  говорят Кеплер и другие новейшие
писатели  (Уэвель,  с.  329),  но даже  вместе  со  своим  обличителем
Бонифацием был  причислен к лику  святых (§60 главы  четвертой). Споры
продолжались  и  Тостат отмечает  мнение  о  шарообразности Земли  как
«небезопасное» учение еще только за  несколько лет до того, как Колумб
посетил другое полушарие  (Уэвель, с. 328). И  вот существование таких
фигур,  как  Тостат,  отнюдь  не  типичных  для  передовых  мыслителей
Средневековья,  и дало  повод  к легенде  (§76  главы четвертой),  что
шарообразность  Земли отрицалась  католической церковью  и подчиненным
ей  ученым.  В  науке  господствовала тогда  философия  Аристотеля,  а
Аристотель превосходно обосновал шарообразность Земли. Папский престол
занимали  люди  самого  разнообразного  характера, но  Павел  III  был
образованным  человеком  и в  обращении  к  нему ссылка  на  языческих
мыслителей  и критика  Лактанция  не представляла  чего-либо нового  и
сомнительного  с точки  зрения католической  ортодоксии. Существование
обскурантов, подобных Тостату, в XV  или XVI в. не вызывает удивления.
Гораздо более удивительно, что подобного рода обскуранты в середине XX
в. могут  быть достаточно  многочисленными и  влиятельными. Достаточно
сослаться  на  статью  А.И.Попова  (А.И.Попов, 1959,  с.  104--105)  о
возможности  приложения  математических  методов в  биологии  с  точки
зрения  диалектического материализма.  Были  советские  философы, а  к
ним  охотно примыкали  по  этому вопросу  и  многие биологи,  которые,
ссылаясь на  одну заметку Ф. Энгельса,  отрицали отсутствие каких-либо
возможностей для математических приложений в биологии и в соответствии
с этим  просто запрещали такие  работы. Интересующее нас  замечание Ф.
Энгельса  гласит так:  «Применение математики  в механике  твердых тел
абсолютное, в механике газов приблизительное, в механике жидкостей уже
труднее;  в физике  больше  в  виде попыток  и  относительно, в  химии
простейшие уравнения первой  степени, в биологии =  0» (Энгельс, 1949,
с.  218). По  этому  поводу А.  И. Попов  пишет:  «Все это  совершенно
соответствует  положению дела  с  применением  математики в  различных
отделах науки во времена Энгельса, и  даже далеко не в самом его конце
деятельности, так  как упомянутые  записки относятся, как  известно, к
началу-середине 70-х  годов, т.е. к  тому времени, когда  очень многие
отделы физики были еще мало  математизированы, а в биологии применение
математики действительно было равно нулю».

Можно заметить: 1) конечно, Ф. Энгельс и предвидеть не мог, что его не
по  разуму усердные  почитатели превратят  все его  высказывания, даже
совершенно отрывочные,  в догмат, не  подлежащий никакой критике,  и в
руководство к  действию на все  времена; 2) даже  наиболее продуманные
высказывания  Энгельса  нередко не  стоят  на  уровне современной  ему
науки; это в  особенности будет ясно из его высказываний  о Ньютоне, о
чем будет речь в свое время;  3) цитированное место Энгельса явно было
им  написало второпях,  так  как говорить  о том,  что  в физике  20-х
гг.  применение математики  было в  виде попыток  и относительно,  ---
значит просто закрывать глаза  на современное состояние науки (оптика,
акустика,  учение  об электричестве  и  магнетизме  и проч.);  поэтому
мнение  Попова, что  Энгельс  для своего  времени  был прав,  вызывает
только недоумение; 4) мнение Энгельса  было наверно и для биологии его
времени,  но  тут  ошибка  более  простительная,  так  как  во  второй
половине  XIX  в.  под  влиянием  взгляда на  биологию  как  на  чисто
историческую  дисциплину  (исключая физиологию)  многие  интереснейшие
попытки  применения математики  в  биологии (начиная  с Галилея)  были
основательны забыты, но нашему современнику непростительно писать, что
тогда применение математики в биологии было равно нулю, так как законы
Менделя  были  опубликованы  в  60-х  гг.  XIX  в.  Но  А.И.Попов  как
официальный  философ  не  решается  ревизовать  догмат  непогрешимости
Маркса и Энгельса и потому защищает возможность применения математики,
стараясь показать, что Энгельс не ошибался, но был неправильно понят.

Таким образом,  все сомнения  в «добром католичестве»  Коперника могут
считаться совершенно  необоснованными и я рад  констатировать, что это
правильное суждение  высказывают не  только так  называемые буржуазные
исследователи,  как полагает  Ревзин  (см.  его цитату  в  §12), но  и
компетентные советские  астрономы. Неоднократно  цитированный Идельсон
(с. 38) пишет: «Несомненно, что  он верный сын католической церкви», и
неужели Ревзин и  другие советские писатели полагают,  что они воздают
дань уважения Копернику, доказывая, что  он всю свою жизнь лицемерил и
был  приспособленцем? Конечно,  бывали случаи,  что выдающиеся  ученые
говорили не то, что они думали, но это явление в настоящее время имеет
место во всяком случае не реже, чем в XVI в.

15. Изложенное,  мне кажется, ясно  показывает, что Коперник  не питал
никаких опасений в отношении главы  католической церкви. Но может быть
Копернику и тут повезло и в момент опублик...

На этом рукопись обрывается. --- Ред.

ЛИТЕРАТУРА 1

(Цифры  после  каждого указания  означают:  римские  --- номер  главы,
арабские --- параграфы, где данное сочинение упоминается.)

Александров А.Д. Об  идеализме в математике // Природа, 1951,  № 7, с.
3--11. Ш: 17, 18, 20, 21, 23, 24                                     .

Александров А. Д. Ленинская диалектика  и математика // Природа, 1951,
№ 1, с. 5--15. III: 21.

Амбарцумян В.А.  Наука земная, наука  звездная // Лит. газета,  7. XI.
1961. IV: 1                                                          .

Аристотель. Аналитики, первая и вторая. ГИЗ, 1952. IV: 1. Аристофан.

Комедии. М., 1954. II: 6.

Веер де. См. ниже иностр. лит.

Бернал Дж. Наука в истории общества. ИЛ.  Москва, 1956. I: 2, 4, 8, 9.
II: 12, 14. III: 2,  4, 34, 35. IV: 4, 9, 11, 27,  29, 30, 63, 65, 66,
67.

Бляшке В. Греческая и наглядная геометрия. Математическое просвещение.
1957. В. 2. С. 111--130. III: 3, 36. IV: 28, 46                      .

Богомолов  С.А. Эволюция  геометрической мысли.  «Начатки знаний».  Л.
1928. III: 2, 3, 37                                                  .

БСЭ (Большая Сов. Энциклоп.) 2-е изд. Т. 31: 3, 4, 11, 22. IV: 63, 75.
Брехт Бертольд. Жизнь Галилея. Гос. изд. «Искусство». М., 1957. IV: 1.
Брокгауз-Ефрон. Энциклоп. словарь. Т. IV: 4, 15, 37, 70.

Бруно Джордано. О причине, начале и едином. Пер. и предисл. М. А.

Дынника. ГИЗ. 1934. IV: 76.

Вейль. См. ниже иностр. лит.

Веилъ  Г. О  философии  математики //  Пер.  А.П.Юшкевича, предисл  от
перевод. С.А.Яновской. Гос. тех-теор. изд.  1934. III: 16--18, 24, 26,
28, 37.

Веселовский  И.Н.Аристарх Самосский  ---  Коперник  античного мира  //
Историко-астроном. исследования. Вып. VII, 1961.  С. 13--70. IV: 3, 7,
9--13, 21, 35.

Виндельбанд Б. История древней философии // Перевод со 2-го нем. изд.

Москва. Типогр. русск. тов-ва. 1911. II: 2, 3, 6.

Винер Н. Беседа // Вопросы философии. 1960. № 9. С. 167. III: 16.

Галилей Галилео. Диалоги о двух  главнейших системах мира. (Ориг. изд.
1632  г.).  Перевод,  предисл.  (с. 3--18)  и  примеч.  (с.  357--377)
А.И.Долгова. ОГИЗ. М-Л. IV: 28.

Гарре. См. ниже иностр. лит.

Гегель. Лекции по истории философии. Кн. 2. М., 1932. IV: 51, 52.

Гегель. То же. Кн. 3. 1935. IV: 52, 58, 61.

Гейзенберг В. Открытие Планка  и основные философские проблемы атомной
теории. Успехи  физич. наук. Т. 66,  вып. 2, 1958. С.  163--175. I: 4.
IV: 47.

246 Литература

Гердер. См. ниже иностр. лит.

Гильберт Д.  и Аккерман В.  Основы теоретической логики.  Ред., вступ.
ст. и коммент. проф. С.А.Яновской. ИЛ, Москва. 1947. I: 2. III: 21   .

Голубев Г. Н. Улугбек. Жизнь замечательных людей, № 12 (302). Молод.

Гвардия, I960. IV: 63.

Госкин. См. ниже иностр. лит.

Жданов  А.А. Выступление  на  дискуссии по  книге  Г. Ф.  Александрова
«История зап.-европ. философии» 24 июня 1947. ОГИЗ. 1947. III: 21, 22.

Жебелев С. и Радлов 3.: Предисловие к полн. собр. Творений Платона. Т.
1. С. 1--9. Академия. Петербург, 1923. II: 9                         .

Зайцев Б.А. Избранные сочинения. Т.  I. Изд. Всесоюз. общ. полит, кат.
и ссыльн. пос. М. 1934. I: 6, II: 11                                 .

Ибервег-Гейнце. См. ниже иностр. лит.

Идельсон  Н.  Н.  Жизнь  и  творчество  Коперника.  Сборник  статей  к
400-летию со дня смерти Коперника. 1947. С. 5--42. IV: 17.

Идельсон Н. Н. Этюды по истории планетных теорий. Там же. С. 268--305.

IV: 9, 13, 15, 16, 32, 63.

Извольский Н.А, Синтетическая геометрия. Гос. уч. пед. изд.

Наркомпроса. М. 1941. III: 39.

История философии, под  ред. Г.Ф.Александрова, Б. Э.  Быховского, М Б.
Митина и  П.Ф.Юдина. Политиздат. 1941. Т.  I. II: 10. III:  5, 14. IV:
3--10, 15, 28, 39, 46, 53, 54--56, 61,  62. То же. Т. 2. 1941. IV: 72,
75.

Кара-Мурза Г.С.Тайпины. Изд. 3-е. Гос. уч.-пед. изд. М., 1957. I: 7.

Карно Лазарь. Размышления о метафизике исчисления бесконечно-малых.

ОН-ТИ. М.-Л. 1936. III: 7, 8.

Клейн Феликс. Элементарная математика с точки зрения высшей. ОНТИ. М.

Л., т. I, 1935. III: 37.

Коперник Николай. Об обращении небесных  сфер. Первые две главы. Сбор,
статей к  400-летию со  дня смерти.  М., АН  СССР. 1947.  С. 187--213.
Прим. Идельсона, с. 214--217. IV: 3, 7, 22.

Коперник Николай. Сборник статей к 410-летию со дня смерти. М., Изд.

АН СССР. 1955 IV: 1.

Краткий филос. словарь  / Под ред. М. Розенталя и  П. Юдина. Изд. 4-е,
дополн. и исправл. Госполитизд. 1954. III: 16. То же. 4-е изд., допол.
тираж. 1955. III: 16. IV: 61.

Кузнецов Б.Г. Эйнштейн. Изд. АН СССР. М. 1962. IV: 24.

Ленин В.И. Материализм и эмпириокритицизм. Соч. Т. XIII. I: 1.

Лукреций. О природе вещей. Перев., вступ. ст. и коммент. Ф. А.

Петровского. Изд. АН СССР. М. 1958.*IV: 40--44, 47, 50.

Луначарский А. В. Почему нельзя верить в бога. Госполитизд. М-, 1958.

I: 6. IV: 69.

Лурье С. Я. Механика Демокрита. Труды Инст. истории науки и техники АН
СССР. Сер. 1, вып. 7. С. 129--180. 1935. [: 9. II: 12. IV: 37.

Лурье С. Я.  Платон и Аристотель о точных  науках//Труды Инст. истории
науки и техники. Сер. 1, вып. 9, с. 303--313, 1936. III: 38.

Лурье С.Я. Архимед.  Изд. АН СССР, 1945.  I: 10. Ц: 1, 7,  8, 13. III:
3--6, 9--12, 26, 34. IV: 9, 11, 12.

Лурье С.Я.  Очерки по истории античной  науки. Изд. АН СССР,  1947. I:
19. II: 1--6, 12, 14. Ill: 1, 5, 9, 13--16, 29--38. IV: 6, 34, 38.

Лурье С.Я. Архимед. БСЭ, 2 изд. 1950. Т. 3. С. 184--186. IV: 11.

Лурье  С.  Я.  Три  этюда  к  Архимеду.  Учен.  Записки  Львовск.  Гос
Университ. Т. XXXVIII. Серия механико-матем.,  вып. 7, с. 7--50. 1956.
III: 10.

Магидович И. Колумб и его время: Статья // Путешествия Христофора

Колумба. Гос изд. геогр. лит. М. 1956, с. 3--53. IV: 76.

Материалисты  Древней  Греции.  Гераклит,  Демокрит  и  Эпикур.  Собр.
текстов  под ред.  М.А.Дынника. Гос.  изд.  полит, лит.  1955. IV:  6,
35--39, 46.

Мензбир М. Тайна Великого океана. М.: Изд. М- и С. Сабашниковых, 1922.

I: 6.

Науманн. См. ниже иностр. лит.

Николай Кузанский.  Избранные сочинения  / Пер.  С.А.Лопашова и  А. Ф.
Лосева. М.  1937. Очерк жизни,  сост. Лопашовым. Гос.  соц.-экон. изд.
IV: 70--74. С. 326--348,

Новиков П.С. Элементы  математич. логики. Гос. изд.  физ.-мат. лит. М.
1959. III: 22                                                        .

Ньютон И. Математические начала натуральной философии / Пер. с примеч.
А.Н.Крылова. Извест. Никол. Морск. Академии.  Вып. IV и V. 1915--1916.
III: 8. IV: 49                                                       .

Огородников К.Ф. Астрономия. БСЭ, 2 изд.  Т. 3, 1950. с. 311--326. IV:
3.

Петровский  Ф.А.  Поэма  Лукреция  «О природе  вещей»  //Лукреций.  «О
природе вещей». М. 1958. IV: 40--45.

Писарев Д.И. Сочинения: В 4 томах. Гос. изд. худ. лит. 1955--56. I: 6.

Платон. Творения  Платона. Перев. под  ред. С. Жебелева и  Э. Радлова.
Труды Петерб.  философ, общ. Т.  1. Академия. Петербург. 1923.  Т. IX.
1924. II: 9. IV: 21, 31, 33, 47                                      .

Платон. См. ниже иностр. лит.

Радемахер Г. и Теплиц О. Числа и фигуры. Изд. второе. ОНТИ. 1938. М.

Л. III: 5.

Рассел Бертран. Человеческое сознание, его сферы и границы. ИЛ. М.:

Вступ. ст. Кольмана. 1957. IV: 46.

Рассел Б. Почему я не христианин? / Пер. с англ. ИЛ. М. 1958. I: 6.

III: 23.

Рассел Б. История западной философии. ИЛ. М., 1959. I: 1--3, 8, 9. IV:
7, 15, 18, 19, 33, 46, 48--60.

Ревзин Г. Коперник. М.: Молод. Гвардия, 1949. IV: 8, 12, 25.

Реманэ. См. ниже иностр. лит.

Румянцев Н. Жил ли Иисус Христос? ОГИЗ. Гос. антирел. изд. 1937. Т: 5.
Рыбников К.А. История  математики. М.: Изд. МГУ. 1960.  Ill: 1--8, 13,
14, 16, 18, 26, 35, 37.

Серебренников Виктор: Примечания к книге: Платон. Гос. соц. экон изд .

М. Л. 1936. С. 171--184. III: 14.

Тарле Е.В. История Италии в  Средние века. Брок.-Ефрон, СПб. 1901. IV:
77. Тимирязев  К.А. Луи  Паскаль. Избр. соч.  К.А. Тимирязева.  Т. II.
Сельхозгиз. 1948. III: 36.

Томпсон. См ниже иностр. лит.

Тьюринг А. Может ли машина мыслить?  С прилож. статьи Дж. фон Неймана:
Общая и  логическая теория  автоматов / Пер.  с англ.  Ю.А.Данилова, с
предисл. С. А. Яновской. Гос. изд. физ.-мат. лит. 1960. IV: 50.

248 Литература

Уэвель Уильям. История  индукт. наук. Том I. Перев. с  3-го англ. изд.
М.А.Антоновича и  А.Н.Пыпина. СПб., 1867. Изд.  «Русской книж. торг.».
IV: 3, 4, 14, 15, 17, 20, 21, 23--27, 32, 33, 45, 58, 60--65, 71     .

Фесенков В. Г. Современные представления  о Вселенной. Изд. АН СССР. ]
949. IV: 14.

Чернышевский Н.Г. Избр.  филос. сочинения. ГИЗ. М., 1938.  II: 11. IV:
44, 45.

Шкловский И. С. Вселенная, жизнь, разум. Изд. АН СССР. 1962 IV: 73   .

Энгельс Ф. Диалектика природы. Госполитизд.  1949. II: 13. III: 3. IV:
I, 28. Юшкевич А.П. См. Вейль, 1934. III: 25.

Юшкевич А П. Идеи  обоснования математического анализа в восемнадцатом
веке. Вступ. статья к книге Карно, 1936. с. 11--76. III: 8, 9.

Яглом  И.  М.  Рецензия  на   книгу  В.  А.  Розёнфельда;  Неевклидова
геометрия. Успехи  математических наук. 1956.  Т. XI. Вып. 6  (72), С.
257. III: 19.

Яновская С.А. Предисловие к русскому переводу книги: Гильберт и

Аккерман (см. выше). С. 5--13. М., 1947. III: 21, 23.

Beer G. de. Reflections of Darwinian. Thomas Nelson and Sons, L 1962 .

IV: 51. Diogenes Laertius. IX. 40. II: 2.

Harre R. Philosophical  aspects of cosmology. British  J. Philos. Sci.
1962, 13, 104--119. IV: 72

Herder J. G. Hellas Veilchen. Chemnitz. 1801. I: 4.

Hoskln M. A Nature and Mathematics. The making of science. Ed. by

Rupart. Leicester. Univ. Press. 1960, 17--22. IV: 29.

Naumann H. Albert Einstein's politische Erbs. Wissenschaft und

Fortschritt. 1960, 121--124. I: 5.

Platon. Oeuvres  completes. Trad, par  Leon Robin. Bibliotheque  de la
pleiade. 1950, II: 9. IV: 20--25, 39.

Plato's Staat.  Obers. von Schleiermacher, erl.  von Kirchmann. Zweite
Auflage arb. von Th. Siegert. Philosoph. Bibliothek. Band 80, Leipzig,
1901. IV: 28.

Remane A. Die Grundlagen  des natiirlichen Systems, der vergleichenden
Anatomic und der Phylogenetik. 2 Auflage. Leipzig, Akad. 1956, I: 1.

Thompson D'Arcy W. On growth and form. Cambr. Univ. Press. 1942. I: 3.

Ill:

19.

Uberweg-Heinze Fr. Grundriss der Geschichte der Philosophie. 1 Teil, 8
Auflage. 1894; 2 Teil, 8 Auflage, 1898. IV: 7, 10, 53, 54, 56.

Weyl H. Symmetry. Princeton Univ. Press. 1952. Ill: 3.

% vim: foldmethod=marker ft=tex tw=70 lbr :

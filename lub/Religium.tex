НАУКА И РЕЛИГИЯ

ПОСТАНОВКА ВОПРОСА

Вопрос об отношении науки и религии имеет по крайней мере двухвековую давность,
а правильнее, может быть, даже двухтысячелетнюю, и то решение, которое
наметилось примерно полтора века тому назад, многим из современных
интеллигентов кажется окончательной истиной в последней инстанции. Оно всего
лучше отображено в знаменитом коротком разговоре великого ученого Лапласа и
выдающегося государственного деятеля Наполеона Бонапарта. Когда Наполеон
(который, как известно, был хорошо знаком с высшей математикой) ознакомился с
известным сочинением Лапласа о небесной механике, он задал ему вопрос, почему
тот не упоминает о Боге в этой книге. Наполеон, очевидно, намекал на Ньютона,
которые закончил свои великие «Математические начала натуральной философии»
идеологическими рассуждениями. «Государь, я не нуждался в этой гипотезе», ---
ответил Лаплас (цитирую по Энгельсу, Диалектика природы, с. 268). Наполеон
задал свой вопрос не потому, что он был верующим, но он перешел от
преследования католической церкви к конкордату\footnote{Конкордат Наполеона ---
соглашение между папой римским (Католической церковью) и Наполеоном
(Францией) заключенный 15 июля 1801 года, по которому Рим признавал
новую французскую власть, католицизм был объявлен религией большинства
французов. При этом свобода вероисповедания сохранялась.} с целью упрочить свое
положение.  Религия не нужна науке, но нужна даже атеистическому деспоту как
«незримая паутина» (прекрасное выражение Горького) для более легкого подчинения
угнетенных масс, как опиум для народа. Вспомним преследование инакомыслящих,
Галилея, Дж. Бруно, инквизицию, индекс запрещенных книг, сопротивление
эволюционной теории Дарвина, и как будто придется сделать заключение, что
религия не только бесполезна, но и вредна для развития науки. Мнение
просветителей конца XVIII и начала XIX в., казалось, торжествует в течение
всего XIX и начала XX в. В биологии телеологии был нанесен сокрушающий удар Ч.
Дарвином, в конце XIX в. был сильный процесс «дехристианизации» Франции,
Бисмарк объявил «культуркампф» против католической церкви в Германии, наконец,
в XX в. правительства почти одной трети человечества стали откровенно
атеистическими. Подрастающая молодежь этих стран практически ничего не знает о
религии. Идеи социализма и атеизма считаются неразрывно связанными, и поскольку
деятели социализма выставляют великолепный идеал осуществления справедливого
строя на земле (в противовес обещанному церковью Царствию Божию на небе после
смерти), то эти социалистические идеалы считаются морально обязательными, а
следовательно, морально обязательным является и атеизм, и всякая «поповщина»,
легко приводящая к тем или иным формам теологии, отметается «с порога». Поэтому
обязательность атеистического обучения молодежи, запрещение религиозной
пропаганды и свободного издания религиозных книг не кажется многим ограничением
свобод вообще и свободы совести в частности, так как для многих честных
образованных и умных людей всякая религия кажется пережитком прошлого, подобным
каннибализму, обычаю убивать стариков, колдовству, гаданиям по звездам,
«свободе» заводить обширные гаремы, геноциду, учению о низших расах и проч., и
восстановление совершенно недопустимых с моральной точки зрения обычаев:
истребление пленных, стариков, душевнобольных и вообще «неполноценных»,
проводимое гитлеровцами, конечно, не делает гитлеровскую Германию более
«свободной» страной, чем страны антигитлеровской коалиции, так как свобода
истреблять или развращать себе подобных (торговля наркотиками, порнографической
литературой и проч.) не может считаться свободой, достойной прогрессивного
человечества. Не всякая свобода и желательна, как говорится в одном коротком
диалоге еще дореволюционного времени: «Извозчик, ты свободен?» --- «Свободен».
--- «Кричи: „Да здравствует свобода!"» Свобода пропаганды суеверий,
распространения наркотиков и т. п. не есть та свобода, за которую есть смысл
бороться.

Неудивительно, что в нашей стране, отставшей в силу пережившего себя
общественного строя от западных стран, особенно силен был антирелигиозный дух
среди нашей интеллигенции, несмотря на то что до революции преподавание в
школах в значительной степени было проникнуто религиозным духом (разных
религий), в каждом паспорте обозначалась религиозная принадлежность носителя
паспорта (национальность в паспортах не фигурировала) и не допускалось, чтобы
кто-либо объявил себя атеистом. Протест против религии носил исключительно
резкий характер. Ленин считал религию вообще одним из отвратительнейших явлений
в обществе и всякую самую утонченную форму фидеизма он решительно отбрасывал «с
порога». Даже субъективный идеализм объявлялся «поповщиной», и по отношению к
самым близким людям Ленин не допускал никакого компромисса. Его собственный
отец был глубоко религиозным человеком, как и отец Чернышевского, и оба они
были очень почтенными людьми. Религия была тем более ненавистна нашим
революционным демократам, что она тормозила развитие народа. Поэтому все они
переходили от глубокой религиозности в юности к воинствующему атеизму зрелого
возраста (Салтыков-Щедрин, Писарев, Чернышевский и др.). Но воинствующий атеизм
был свойствен не только революционерам или марксистам, но и лицам, далеко
стоящим от революционной борьбы и марксизма. Мне известен один ученый
выдающихся умственных и моральных качеств, весьма скептически относившийся к
селекционизму (неодарвинизму, т. е. учению о ведущей роли естественного
отбора), но вместе с тем считавший, что дарвинизм принес пользу как мощное
антирелигиозное учение. Религия в глазах этого ученого и честного человека была
настолько отрицательным явлением, что для борьбы с ней можно было использовать
даже ложные учения. Подобно тому как в христианском песнопении поется про
Христа, что он «смертию смерть поправ», так и здесь ложью следует победить
другую, более страшную ложь. «Цель оправдывает средства» --- этот лозунг
употребляется и людьми, весьма отрицательно относящимися к иезуитам.

И не только у нас, но и среди ученых и мыслителей Запада сейчас есть немало
представителей, полностью солидаризующихся с тем мнением, что религия есть зло,
только мешавшее прогрессу человечества, и что хотя среди великих ученых и
философов мы знаем очень много истинно религиозных людей, они сделались
великими учеными не благодаря, а вопреки своей религиозности, и что пропасть
между религией и наукой непроходима. Такую точку зрения развивают и лица,
весьма враждебно относящиеся к коммунизму, например Бертран Рассел. Его
неприязнь к коммунизму доходила до того, что он (пока Запад обладал монополией
атомной бомбы) был не против того, чтобы использовать преимущество Запада для
политических целей, но он же (сейчас --- видный представитель движения за мир)
сотрудничает в нашем журнале «Наука и религия» и прямо заявляет (в полном
согласии, например, с нашим Луначарским), что ученый и прогрессивный человек
вообще не только может, но и должен быть антирелигиозным человеком. Рассел не
одинок в своем высказывании. В 1961 г. появился под общей редакцией Юлиана
Гексли сборник статей 27 авторов под названием «Структура гуманизма»,
подвергнутый подробной рецензии в британском журнале «Философия науки» (1963,
том XIV, № 53, с. 41--53) Аланом Стюартом. Рецензент приветствует этот сборник
как «новое откровение», как веху в деле эмансипации человеческого духа, как
обоснование нового эволюционного гуманизма, полностью отвергающего все старые
религиозные суеверия, всякую сверхъестественную религию. И несмотря на высокую
оценку сборника, Стюарт отмечает очень крупный дефект, заключающийся в том, что
эти новые гуманисты одиннадцать раз цитируют иезуита Тейяра де Шардена даже с
указанием на желательность знакомства с этим автором. Рецензент негодует (с.
52): «Многие образованные честные искатели истины, которые знают кое-что о
последних двух тысячах лет человеческой истории, не могут избежать связывать
„священника" (попа) с „поповщиной", а „поповщину" с „лживостью". Здесь же мы
видим испорченную книгу, которая могла бы быть превосходной. Потому что,
несмотря на все их изъявления о глядении в будущее, эти гуманисты, подобно жене
Лота, тоскующей по Содому, глядят назад». Я затруднялся в переводе слов
«priest» и «priestcraft» и вставил вместо «priestcraft» «поповщину». Последний
термин широко применяется в советской литературе, но он почему-то отсутствует в
трех новых словарях: русско-английском и русско-немецком 1952 г. и
русско-французском 1962 г. Но мы видим, что рецензент, предлагая нам смотреть
только вперед и не оглядываться, или забыл, или игнорирует известное положение
о диалектике, где новый синтез в известном смысле возвращается к старому.

Но наряду с таким резким отношением ко всякой «поповщине» мы имеем гораздо
более примирительное отношение. Добржанский, очень близкий по своим
биологическим взглядам к Ю. Гексли, заканчивает свою интересную книгу об
эволюции человечества цитатой из Тейяра де Шардена и считает, что эволюционная
идея иезуита является лучом надежды для человечества. Он считает, что его книга
содержит и науку, и метафизику, и теологию, и даже поэзию, испорченную
несколько в английском переводе. Предисловие к русскому переводу
книги Тейяра де Шардена «Феномен человека» написано выдающимся, одним из
наиболее культурных французских коммунистов, Роже Гароди, который, наряду с
критикой многих положений Шардена, высоко оценивает многие прогрессивные
стороны его взглядов. Исчезает положение «двух лагерей»: с одной стороны,
реакционный черный лагерь защитников религии, состоящий из честных невежд или
фанатиков и бесчестных эксплуататоров, а с другой --- светлый лагерь
«прогрессистов», куда относятся все умные и честные люди, «свободомыслящие»,
или, как это понималось на языке XIX в., атеисты и социалисты. Все
перемешалось. По отрицательному отношению к религии Ленин оказался в одной
компании с Б. Расселом и цитированным рецензентом Стюартом, более умеренный
Луначарский, говоривший о «религиозном атеизме», о «новой религии социализма»,
сочувствующий неодарвинист Добржанский и, наконец, коммунист Гароди: для
каждого оттенка отношения к религии мы можем подыскать представителей обоих
лагерей. И здесь, как может быть всюду, господствует в высокой степени
комбинационный принцип.

Сейчас в мировом масштабе происходят два встречных течения. Наряду с успехами
антирелигиозности в мировом масштабе, падением посещаемости храмов и проч. мы
имеем укрепление религиозных позиций в ряде культурных стран. Во Франции
возвращены изгнанные на пороге столетия иезуиты и во главе правительства стоит
верующий католик. Ряд правительственных партий во Франции, ФРГ, Италии, Бельгии
и других странах откровенно христианские. Можно ли это объяснить только
империалистической реакцией? Два выдающихся деятеля XX в. --- Ганди и Кеннеди ---
не скрывали своей религиозности, и оба пали от рук людей, сходных по идеологии
с фашизмом, империализмом, расизмом и прочими бесспорно ретроградными
идеологиями. Защитниками религии или «поповщины» в широком смысле слова (причем
самого разнообразного характера) выступают самые передовые ученые
современности: Эддингтон, Эйнштейн, Гейзенберг, Планк, Шредингер и многие
другие меньшего значения. Писатели Сент-Экзюпери, Веркор и другие настаивают на
необходимости синтеза, а не голого отрицания. Растет число идеалистов самых
разнообразных направлений, в то время как материалистическая философия скорее
обнаруживает явные признаки загнивания. Мы знаем, что в нашей стране под видом
борьбы с религией и идеализмом систематически боролись со всеми новыми
течениями в науках: теория относительности, принцип неопределенности, теория
расширяющейся Вселенной, теория резонанса в химии, настоящая генетика. Даже
там, где поддерживали то или иное здравое направление (например, учение об
условных рефлексах Павлова), так догматизировали учение, что оказали ему
медвежью услугу. Все же новшества, предложенные под видом истинно
материалистической науки, не выдержали испытания временем, и притом короткого
времени. Таким образом практически опровергнут подход: раз это ведет хотя бы в
слабой степени к «поповщине», это надо отвергнуть «с порога».

Не менее неожиданными были и события на этическом и политическом фронтах.
Ставилось в вину всем религиозным правительствам, что они наряду с
«запугиванием адом» имели на вооружении и смертную
казнь. Запугивание адом у нас исчезло, но смертная казнь фигурирует в нашем
Уголовном кодексе в таком числе статей, как, я думаю, нигде в капиталистических
странах, а некоторые капиталистические страны (мне известны: ФРГ, Мексика,
Уругвай, Израиль до процесса Эйхмана) вовсе не имеют смертной казни в мирное
время. В политике «и сольются в одно все народы в вольном царстве святого
труда» пока обернулось жесточайшим конфликтом между двумя крупнейшими
коммунистическими партиями: СССР и Китая.

Если пока между Китаем и СССР еще нет вооруженного конфликта, то только потому,
что Китай еще недостаточно силен и раздираем внутренними противоречиями, а у
нас неслыханная в царские времена милитаризация и достаточная удовлетворенность
последними захватами. Но разговор между двумя «социалистическими» странами по
тону вовсе не отличается от разговоров двух милитаристских империалистических
держав. Но вопросы этики и политики уже выходят за пределы темы настоящей
статьи.

Мы видим, таким образом, что в отношении религии сейчас наблюдаются два
противоположных процесса: 1) антирелигиозный, касающийся масс и среднего уровня
интеллигенции; 2) прорелигиозный, выражающийся в обильном числе всевозможных
направлений, часто затрагивающих самые высокие уровни передовых мыслителей
современности. Полезно выяснить, является ли это второе направление вызванным
какими-либо новыми открытиями или явлениями современности или можно найти и в
материалистическом мировоззрении начала XIX в. такие черты, которые ясно
указывали на его несовершенство. Так ли ясно все в отношении разговора Лапласа
с Наполеоном? К этому и перейдем.

Глава I. НАУКА --- ВРАГ СУЕВЕРИЙ И ЧУДЕСНОГО

Дать определение науки не так-то легко, для наших целей достаточно ограничиться
тем, что ясно сквозит в словах Лапласа. Наука --- враг всякого суеверия,
чудесного, а религия, напротив, основана на суевериях и принятии чудесного.
Всего лучше это выражено в знаменитых словах блаженного Тертуллиана: «кредо,
квиа абсурдум ест», т. е. «верю, потому что это абсурдно». Пока не будем
разбирать вопроса, является ли высказывание Тертуллиана ортодоксальным или
типичным для всех христиан. Ограничимся тем, что уже в раннем христианстве
обозначилось направление, резко противополагавшее религиозное учение тогдашней
классической науке (см., например, История философии, 1941, № 1, с. 387--388) и
в известном смысле сохранившееся и в более поздние времена. Каков смысл
высказывания Тертуллиана? В учении о познании Тертуллиан придерживался
вульгаризированного стоического материализма и считал, что все действительно
существующее телесно, в том числе Бог и бессмертная душа. Таким образом, его
взгляд вовсе не является выражением его идеализма, а напротив, теснейшим
образом связан с материалистическим характером его мировоззрения. Разумом он
был материалист, но, как христианин, принимал и такие явления, которые
материалистическому объяснению
совершенно не поддавались, но так как они были для него совершенно бесспорны,
то в них он верил как в абсурд с точки зрения материализма. Это была не слепая,
а сознательная вера, и он твердо проводил различие между верой и знанием. А так
как для него на первом плане стояли истины веры, то естественно, что к
человеческой мудрости он относился с меньшим уважением и порой даже с
презрением. Но посмотрим, свободен ли материализм XIX в. от принятия чудесного
и непонятного. Смысл науки в том понимании, которое нас сейчас интересует,
заключается в том, что она должна бороться с суевериями, что можно понимать в
пяти формах: 1) чудесное, т. е. непонятное, 2) абсурдное, 3) противоречивое, 4)
сверхъестественное и 5) невероятное. Разберем по очереди.

1. Непонятное, часто необычное

Все непривычное, неожиданное, непонятное нам кажется чудом, и мы и в практике
жизни, и в науке стремимся к тому, чтобы непонятное свести к чему-то понятному.
Антропоморфические религии сводят все «чудесное» к вмешательству существа,
подобного человеку, но невидимого. Так толковались изумительные приспособления
органического мира. Дарвин оставил телеологию, т. е. мнение, что основа в
биологии --- явление приспособления, но вместо невидимого бога ввел тоже
невидимое явление --- естественный отбор, приемлемый для материалистов. Задачей
науки оказалось дать простое объяснение сложным явлениям. Это направление имеет
большую давность. Уже цитированный Стюарт (Брит. Журн. филос. науки, 1963, т.
XIV, № 53, с. 53) по поводу нового, эволюционного гуманизма пишет: «Это ново,
ново в истории так называемой западной или греко-христианской цивилизации.
Конечно, были и раньше люди, которые мыслили свободно --- в общем так же, как и
сегодняшние гуманисты. В дохристианскую эру были Эпикур и Лукреций с их
последователями». Можно ли считать Эпикура одним из лидеров античной науки?
Вряд ли. Со взглядами Эпикура можно познакомиться по известному античному
историку философии Диогену Лаэрцию (я использовал немецкий перевод). Диоген
Лаэрций --- добросовестный, но малокритический историк, который сообщает разные
небылицы про многих философов, но Эпикуру он посвятил, как и Зенону-стоику,
наибольшее место в своей книге (по 68 страниц) и, в отличие от других
философов, привел много текстов. Хотя обычно Эпикура считают представителем
линии Демокрита, но он резко отличается от Демокрита ярко выраженным
индетерминизмом. Основа мировоззрения Эпикура: боги не вмешиваются в нашу
судьбу, не вознаграждают и не карают, смерть ни к чему плохому не приводит. Это
и есть основа подлинного эпикуреизма: безмятежность души, атараксия. У самого
Эпикура это не приводит к безнравственности: напротив, многие из его положений
сходны с таковыми стоиков, он обосновывает этику утилитарными соображениями,
что вовсе не так плохо, но что дает его мировоззрение науке? Здесь как будто
проявляется полное свободомыслие. В вопросе о величине Солнца, о движении
планет, знамениях, громе и других явлениях он выставляет различные гипотезы и
не дает предпочтения ни одной. Они все с его точки зрения равноценны, при
условии, чтобы не было мифов, связанных с религией. Но будучи совершенно
беззаботным в области научных гипотез, Эпикур крайне догматичен в онтологии или
метафизике и против всякой диалектики. Но, может быть, это и есть то полезное
ограничение свободы, которое необходимо для ученого, чтобы он сосредоточил свои
усилия, основываясь на определенных бесспорных аксиомах. Нет, ограничиваясь
объяснениями и не придавая им принудительного значения, Эпикур чрезвычайно
презрительно относится к терпеливым усилиям античных астрономов, старавшихся
длительными наблюдениями выяснить законы движения небесных тел, т. е.
положивших начало математическому описанию явлений природы. Переводчик и
комментатор Апельт правильно пишет, что если бы такая точка зрения
восторжествовала, естествознание никогда бы не вышло из исходного состояния.

Последователями Эпикура в смысле необходимости в первую очередь «объяснения», а
не математического «описания» явлений был, конечно, Лукреций и все эпикурейцы.
А так как в Римском государстве в философии господствовали стоики (обращавшие
внимание только на этику) и эпикурейцы, то становится понятным тот ничтожный
вклад в науку, который сделала могущественная Римская империя.

Но может быть, учения Эпикура (и Лукреция, который не дал, кажется, ничего
оригинального по сравнению с Эпикуром) были в политическом отношении
прогрессивны? Он был гуманный человек и рекомендовал доброе отношение к рабам,
но в смысле общественной идеологии был прототипом «премудрого пескаря»: не
стоит жениться и иметь детей, не занимайся государственной деятельностью, как
бы чего не вышло, пребывай в мудрой атараксии и не размышляй о возможности
преобразований.

Если мы посмотрим всю историю человеческой мысли, то убедимся, что решительно
все строители «утопий» были идеалисты, что же касается материалистов, атеистов
и антирелигиозников, то они или сами были тиранами (Критий), или защищали
абсолютизм под разными видами --- «просвещенного абсолютизма» и т. д. (Гоббс,
Дидро и пр.). Все идеологи революций вплоть до XIX в. работали под
идеалистическими знаменами и в общем сделали немало для преобразования
общества. Поэтому слова Маркса: «Философы только объясняли мир, а его
необходимо перестроить» (точно слова не помню) надо понимать:
«Материалистические философы только объясняли мир, а сейчас им надо приняться
за перестройку», и тогда он будет звучать более или менее правильно, если
оставить в стороне вопрос (сейчас оставим его без рассмотрения), в какой мере
революционный марксизм может считаться чисто материалистическим учением.
Несомненно, что современный селектогенез, т. е. эволюционное учение, восходящее
к Дарвину, совершенно пропитан эпикурейским духом. Основной императив
дарвинской морфологии: дай какое-нибудь «причинное» и «механическое» объяснение
структуре, которое можно свести к действию естественного отбора, с тем чтобы
устранить «поповщину», «платонизм» и прочие вредные учения. Неважно, что это
объяснение не является механическим или причинным в смысле точных наук, важно
заглушить сомнения
в отсутствии целеполагающих начал в природе. К математическому толкованию формы
и системы это направление, естественно, глубоко враждебно. Но это направление
поддерживают и выдающиеся математики! Верно, но об этом придется сказать
несколько слов в разделе о невероятном.

Простое объяснение есть низший этап развития научного мышления, и если это
направление доминирует, то оно притупляет то, что можно назвать научной
бдительностью, удивлением перед новыми фактами, и становится подлинным опиумом
для науки.

Только что рассмотренный пункт к Лапласу не относится, так как он был
представителем точной науки, и когда давал объяснения (например, в его
знаменитой космогонической гипотезе), он оговаривал ненадежность этой гипотезы,
так как ее математической теории он дать не мог.

2. Абсурдное, т. е. нелепое

А вот абсурда Лаплас не избежал. Он был убежден, что в мире существуют как
конечные реальности одни атомы в пустом пространстве и был убежденным
сторонником механического детерминизма. По нашему философскому словарю (1963,
статья «Детерминизм», с. 121), Лаплас считал, «что значения координат и
импульсов всех частиц во Вселенной в данный момент времени однозначно
определяют ее состояние в любой прошедший или будущий момент. Так понятый
детерминизм ведет к фатализму, принимает мистический характер и фактически
смыкается с верой в божественное предопределение». Как видим, наши официальные
философы находят поповщину и в классическом изречении Лапласа, но невозможно
понять, что они дают взамен. Но в мировоззрение Лапласа входит не только
механический детерминизм, но и, следом за Ньютоном, принятие принципа
всемирного тяготения, т. е. действия на расстоянии материальных тел. С точки
зрения механики это совершенный абсурд: как может тело действовать там, где его
нет? А как же Ньютон? Ньютон это отлично понимал. В письме к Бентлею, автору
лекций по опровержению атеизма, Ньютон пишет (С. И. Вавилов. Исаак Ньютон. М.,
1961. С. 129): «Предполагать, что тяготение является существенным, неразрывным
и врожденным свойством материи, так что тело может действовать на другое на
любом расстоянии в пустом пространстве, без посредства чего-либо передавая
действие и силу, --- это, по-моему, такой абсурд, который немыслим ни для кого,
умеющего достаточно разбираться в философских предметах. Тяготение должно
вызываться агентом, постоянно действующим по определенным законам. Является ли,
однако, этот агент материальным или нематериальным, решать это я предоставил
моим читателям». Как указывает С. И. Вавилов дальше (с. 130), для самого
Ньютона вопрос был совершенно ясен: тяготение объясняется заполнением
пространства Богом (предшественники: Отто фон Герике и иезуит Кирхер). В
несколько скрытой форме это мнение было высказано и в «Общем поучении»
знаменитых «Математических начал натуральной философии» (с. 590, перевод А. Н.
Крылова, 1915 г.): «Бог есть единый и тот же самый Бог всегда и везде. Он
вездесущ не по свойству только, но по самой сущности, ибо свойство не может
существовать без сущности. В нем все содержится и все вообще движется, но без
действия друг на друга. Бог не испытывает воздействия от движущихся тел,
движущиеся тела не испытывают сопротивления от вездесущия Божия». В примечании
Ньютон ссылается на древних авторов: Пифагора, Фалеса, Анаксагора, Филона,
Арата, а также приводит ряд текстов из Библии. Богословские взгляды Ньютона не
были каким-то странным привеском к его научным взглядам, они пронизывали и его
научные теории. Он не был деистом, принявшим только первый толчок, а затем
исключительное действие естественных законов, он был ближе к взглядам
Мальбранша, для которого все происходящее является сплошным чудом. Ко времени
Лапласа к абсурдному с точки зрения механики принципу всемирного тяготения
успели привыкнуть благодаря исключительной плодотворности этого принципа, а
привыкнуть можно к любому абсурду и к непонятным вещам. Если бы у Лапласа не
было «убеждения чувства» --- его механического материализма, то он должен бы
был ответить Наполеону так: «Государь, великий Ньютон ввел Бога в свою книгу
для объяснения принципа всемирного тяготения, который я полностью использовал в
своей работе. Но Ньютон, будучи свободомыслящим ученым, не навязывал своего
взгляда другим и не запрещал искать материальных агентов всемирного тяготения.
Я, правда, не нашел таких агентов, но, надеюсь, кому-нибудь это удастся,
поэтому я и не упоминал о нематериальных факторах в своей книге». Сейчас мы
знаем, что эту роль до известной степени выполнил Эйнштейн в его общей теории
относительности. Нет пустого пространства, оно все наполнено физическим полем
или физическими полями (всю жизнь Эйнштейн и стремился к тому, чтобы дать
единую теорию поля). Дальнодействие исчезло, а вместе с тем и абсурд. Стало ли
все понятным? На этот вопрос я ответа дать не решаюсь. Эйнштейн, как известно,
разобрался и в непонятном факте эквивалентности инертной и тяжелой массы. Этот
непонятный факт в силу своей привычности не смущал подавляющее большинство
физиков.

3. Противоречивое, антиномичное

Это тоже приближается к абсурду, когда об одном и том же мы можем высказать с
одинаковым правом два прямо противоположных суждения. Световой эфир, как
известно, обладал такими свойствами, которые не могут быть приписаны
одновременно никакому телу, и, однако, теория светового эфира была очень
полезной, многие физики считали его существование совершенно доказанным, и наш
великий ученый Д. И. Менделеев всерьез считал возможным рассматривать его как
один из элементов периодической системы. Сейчас, как известно, Эйнштейн
упразднил эфир в специальной теории относительности и до известной степени
реабилитировал это понятие в совершенно ином понимании (лишенном механических
свойств) в общей теории относительности. Но является ли непротиворечивость
обязательным свойством научного мышления? Мы знаем, что великий философ Кант в
свою
классическую «Критику чистого разума» включает «диалектику чистого разума», где
видное место занимают антиномии чистого разума, т. е. такие пары
противоположных суждений, где можно доказать нелепость каждой из
противоположностей. Две наиболее известных: о конечности и бесконечности
пространства и времени и о необходимости и свободе. Кант считал
сформулированные им антиномии принципиально неразрешимыми и в этом видел
границу человеческого разума, все же остальное познание он считал возможным
свести к непротиворечивому виду и полагал, что некоторые теории познания, в
частности формальная логика, уже этого уровня достигли. Он не сомневался в
абсолютной достоверности математических аксиом и теорий. Существование Бога он
считал недоказуемым и неопровержимым, но склонен был строить религию исходя из
этических соображений (Религия в пределах только разума).

Против критической философии Канта со всей решительностью выступил другой
великий немецкий философ Гегель. В своей речи 22 октября 1818 г. (соч. Гегеля,
т. I, 1929, с. 15) он пишет: «Наконец так называемая критическая философия дала
этому неведению вечного и божественного возможность придерживаться этой позиции
с чистой совестью, так как эта философия уверяет, будто ей удалось \emph{доказать}
(курсив Гегеля. --- \emph{А.Л.}), что мы ничего не можем знать относительно вечного и
божественного. Это мнимое познание даже дерзнуло присвоить себе название
философии, и ничего не могло быть желаннее для поверхностных умов и характеров,
ничто не было столь охотно принято ими, как это учение о незнании, благодаря
которому их собственная поверхность и пустота оказывались чем-то превосходным,
желанной целью и результатом всех интеллектуальных усилий. Что мы не знаем
истины и что нам дано знать одни случайные и преходящие, т. е. \emph{ничтожные},
явления, вот то \emph{ничтожное} учение, которое делало и делает наиболее шума и
которое господствует теперь в философии». Из этой цитаты ясно, что утверждение
о непознаваемости мира, которое наши казенные философы инкриминируют всем
идеалистам, явно неприложимо к такому выдающемуся философу, как Гегель,
который, конечно, не меньший, а больший идеалист, чем Кант.

Как же относится к антиномиям Гегель? Отрицает ли он их? Напротив (там же, с.
97), он упрекает Канта за то, что тот перечисляет только четыре антиномии,
тогда как, по мнению Гегеля, антиномии встречаются во всех предметах всякого
рода, во всех представлениях, понятиях и идеях. Разрешение противоречия состоит
в том, что оно принадлежит не предмету самому по себе, а лишь познающему
разуму. Следовательно, в своем развитии разум снимает противоречие (синтез) с
тем, чтобы перейти к новому противоречию, вновь снимаемому, и так далее на
бесконечном пути стремления к абсолютной истине.

По-видимому, общий взгляд Гегеля сейчас торжествует в науке. Аподиктическая
достоверность евклидовой геометрии разбита трудами Лобачевского, Римана и др.,
антиномия конечности и бесконечности пространства снята в общей теории
относительности, где пространство оказывается (в духе Римана) ни конечным, ни
бесконечным, а безграничным. Развитие математики и физики шло, по-видимому,
самостоятельным путем, независимо от Гегеля, но сейчас многие выдающиеся
мыслители, занимающиеся историей и философией науки, пришли к утверждениям,
очень сходным с цитированными мыслями Гегеля. Так, Дюгем в своей замечательной
книге «Физическая теория, ее цель и строение» утверждает, что «экспериментум
круцис» (эксперимент креста, где опровержение одной стороны доказывает
справедливость ей противоположной) вещь в физике невозможная, т. к. в истории
физики неоднократны случаи, где при споре по поводу какого-нибудь вопроса
оказывалось ложным не то, что оспаривала одна из спорящих сторон, а то, в чем
не сомневались обе спорящих стороны. Примерно то же показывает в биологии и
Радль в его замечательной истории биологических учений.

В этом и заключается отрицание безусловной значимости закона исключенного
третьего, одного из столпов формальной логики, подлинной диалектической
логикой. Отрицание закона исключенного третьего лежит и в основе математической
школы интуитивизма. Она опирается и на математические факты. Кажется очевидным,
что могут быть сходящиеся или расходящиеся ряды. А нашли такие ряды, которые не
являются ни сходящимися, ни расходящимися.

Совершенно прав Гегель, что антиномичность, противоречивость пронизывает все
наше мышление, но мы с этим не должны примиряться, а работать над преодолением
этих противоречий. Разумеется, из того, что все реальное противоречиво, не
значит, что все противоречивое реально и заслуживает рассмотрения. Надо
различать между противоречием и бессмыслицей: по-немецки это звучит видерштрейт
и видерзинн, но надо сказать, что отличить бессмыслицу от противоречия не
всегда бывает легко. Одним из излюбленных доказательств бессмыслицы религиозных
учений был догмат христианской церкви, что Бог един, но троичен в лицах. Явная
«бессмысленность» этого догмата послужила причиной возникновения многих ересей,
в частности как будто и причиной гибели Сервета. В самом деле: $3 \times 1 = 1$, явная
бессмыслица. Сейчас разрешение этого противоречия лежит в основе теории
множеств даже гениального Георга Кантора, который был убежденным католиком.
Приведенная формула нелепа только в области конечных величин. Для бесконечных
множеств соединение двух или нескольких множеств одинаковой мощности в одной
дает множество той же мощности. Например, множества всех четных и всех нечетных
вместе дают множество натуральных чисел. Наша таблица умножения неприменима к
бесконечному. Кстати, зачатки теории множеств имеются уже у Галилея.

4. Сверхъестественное

Это то, что не заключает внутренних противоречий, вполне понятно, но которое
выходит за пределы человеческого опыта. Недопущение сверхъестественного
является как будто основой всякого научного мышления. Но одним из как будто
совершенно достоверных выводов человеческого опыта будет: «Ничто не вечно под
луной», «все течет» по Гераклиту. И однако ученые и философы с каким-то
необыкновенным упорством ищут «покоящуюся ось в потоке явлений» будучи глубоко
убежденными (тут они сознательно или бессознательно следуют философу
Пармениду), что все истинно сущее неизменно. Одно из явных проявлений --- атомная
теория, принимающая, что, несмотря на непрерывные кажущиеся изменения тел, по
существу они неизменны. Но ведь это же противоречит всему нашему опыту. Мы
знаем, что самые твердые тела от трения изнашиваются, а тут мельчайшие частицы
двигаются нередко с большой быстротой, сталкиваются и вечно остаются
неизменными. Это совершенно сверхъестественно, но мы к этому привыкли, а
известно, что можно привыкнуть к любому абсурду. Не так думали великие
мыслители прошлого. Возьмем опять Ньютона (Каблуков. Ньютон как химик. «Под
знаменем марксизма», 1937, № 4, с. 205): «При размышлении о всех этих вещах, ---
говорит Ньютон, --- мне кажется вероятным, что вначале Бог сотворил материю в
виде твердых, непроницаемых, подвижных, обладающих массой частиц таких размеров
и форм, с такими свойствами и в таких относительных количествах, какие пригодны
для той цели, для которой он их создал; эти первоначальные твердые частицы
несравненно тверже, чем какое бы то ни было пористое тело, составленное из них;
они так тверды, что никогда не снашиваются и не раздробляются на части, ибо
обыкновенная сила не способна разделить то, что сам Бог сделал единым при
первом творении». Неудивительно, что творцы новой атомной теории или духовные
лица (священник Гассенди, иезуит Боскович), или искренне религиозные люди
(квакер Дальтон, протестант Ньютон, настроенный резко антикатолически) и в
античной философии идеалистические философы (Пифагор, Платон) отнюдь не
отрицали атомизма. Первая математическая атомная теория Босковича, конечно,
ближе к Пифагору, чем к Демокриту. А Демокрит? Разве не связан
материализм теснейшим образом с атомной теорией? Конечно нет, а лишь с
пониманием атомов и той ролью, которая им приписывается в мироздании. По
вопросу дискретного строения тел в античном мире, видимо, не было резких
противоречий. Возьмем того же Диогена Лаэрция. Демокрит, по Диогену Лаэрцию,
был почитателем пифагорейцев и Пифагора, и Трасилл, который свел все сочинения
Демокрита, собрал также и все сочинения Платона. Разница заключается в том, что
для Демокрита атомы были конечной реальностью, а для Платона лишь кирпичами
видимой реальности. Демокрит был совершенно чужд холизму (целое определяет
поведение частей): развитие мира --- следствия случая, а не имманентного закона;
отрицание того, что позднее называлось финальными и формальными причинами.
Позднейший механический материализм усвоил от Демокрита принятие механической
необходимости (в этом смысле он отрицает случайность в природе в противовес
индетерминисту Эпикуру) наряду с отрицанием финальной причинности (в этом
смысле, говоря словами Данте, «вот тот, кто мир случайным полагает, философ
знаменитый Демокрит»).

Вот эти положения, существенные для истинного материализма, действительно
лишены всякой «сверхъестественности», но признание инвариантных атомов могло
зародиться только на объективно идеалистической почве. А потом привыкли
считать, что атомная теория --- цитадель материализма.

Такую же религиозную основу имеет происхождение и другого великого инварианта ---
закона сохранения энергии. Предшественниками его, как известно, были Декарт и
Лейбниц, и оба исходили из того, что Бог вложил элемент своей неизменности в
природу в форме сохранения энергии (формулировали они его не так, как сейчас,
но эта мысль и двигала искание закона). В XIX в. к инвариантности элементов
мироздания привыкли, и из двух авторов закона сохранения энергии в современном
виде Гельмгольц во всяком случае не отличался религиозностью, а другой, Роберт
Майер, был глубоко религиозным человеком. Как известно, и Р. Майер и Гельмгольц
с трудом добились того, чтобы их работы по этому закону были напечатаны; но
даже когда этот закон был признан и на съезде в Инсбруке Р. Майер позволил себе
несколько фраз в религиозном смысле (в 1869 г.), этим воспользовались его
противники, и К. Фогт в газете, намекнул, что это говорит человек, выпущенный
из дома умалишенных, где он одно время действительно был, видимо, не без
содействия родственников (Тимирязев К. А. Избр. соч. в 4 т., т. I, 1949, с.
132). Вспомним, что и Ньютона многие считали сумасшедшим.

5. Невероятное

Невероятное, точнее, чрезвычайно маловероятное. Есть известный рассказ об одном
аббате, хорошо разбиравшемся в основах теории вероятности. Он вошел в таверну,
где играли в кости. Кто-то бросавший кости получил на всех трех костях три раза
подряд по шесть очков. «Кости фальшивые!» --- воскликнул аббат и оказался прав
(их наливали свинцом с одной стороны). Мог ли он ошибиться? Конечно, мог. В
данном случае при правильных костях вероятность такого результата равна
$1:6^9$, т. е. $1:10$ миллионов (приблизительно). Поэтому если бы такая серия
бросаний повторялась несколько миллионов раз, то результат не был бы
удивителен. Но практически мы с такими вероятностями не считаемся. В США
население около 200 миллионов, а погибает ежегодно от автомобилей 50 тысяч
человек, следовательно, средняя вероятность для американца погибнуть от
автомобиля в течение года примерно $1:4000$, а каждый день приблизительно
$1 : 1 500 000$. Но никто же из американцев не считается серьезно с опасностью
погибнуть в ближайший день. Ясно, что ни один здравомыслящий человек не будет
планировать свое поведение из ожидания исключительно маловероятных событий. И,
однако, в науке есть такие странные люди --- большинство механических
материалистов. Второй закон термодинамики является одним из очень важных
достижений физики XIX в., но в соединении с предположением о бесконечности
Вселенной в пространстве и времени он приводит к представлению о тепловой
смерти Вселенной, которая неизбежно должна бы произойти к настоящему времени,
если бы этот закон был абсолютен. А так как Вселенная отнюдь не находится в
состоянии тепловой смерти, то придется допустить или ограниченность ее
состояния во времени (часы Вселенной были когда-то заведены), или наличие
других процессов иной направленности. Но тогда второе начало термодинамики
теряет свою универсальность. Знаменитый физик Л. Больцман (1844--1906) предложил
третий выход. Основываясь на статистическом характере второго начала
термодинамики, он (Философский словарь, 1963, с. 53) «для преодоления
идеалистической гипотезы „тепловой смерти Вселенной" выдвинул свою
флуктуационную гипотезу, согласно которой общее равновесное состояние мира в
целом постоянно и неизбежно нарушается в отдельных областях гигантскими
флуктуациями (отклонениями), приводящими к неравновесному процессу развития
отдельных миров. По своему мировоззрению Больцман был убежденным материалистом,
критиковал энергетизм и махизм». Мы знаем, что Больцман высоко ценил Дарвина и
даже высказался, что наш (XIX век) есть век механического понимания природы,
век Дарвина. Мнение Больцмана поддерживал и наш известный, исключительно
образованный физик Хвольсон. В своей статье «Можно ли прилагать законы физики
ко Вселенной» он развивает аналогичную мысль, что так как законы физики по
крайней мере, как правило, являются статистическими законами, то всегда в
бесконечной Вселенной мы можем найти такой уголок, где законы эти в силу
случайных отклонений (флуктуации) оказываются неприложимыми, все там идет
навыворот. Весь наблюдаемый нами мир есть результат накопления колоссального
количества случайностей. Для области живого уже давно (Ауэрбах) сформулирован
принцип эктропизма --- концентрации, а не рассеяния энергии. Таким образом, вся
наша Метагалактика --- огромный невероятный кусок во Вселенной, где в силу
накопления случайностей второй закон термодинамики не соблюдается, и в этом
невероятном участке Вселенной имеется еще более невероятная область --- область
живого. Все построено на теории полной невероятности. Это дает объяснение и
тому, что многие выдающиеся ученые, представители точных наук, так высоко
ценили учение о естественном отборе. Против этого учения неоднократно
выдвигались серьезные возражения, отмечавшие, что процесс эволюции на основе
накопления случайных изменений совершенно невероятен: и не хватает материала
для отбора (в особенности полового отбора), нет никакого соответствия между
темпами размножения и темпами эволюции, и совершенно невозможно себе
представить, чтобы путем накопления поломок достаточно совершенного органа
можно было получить более совершенный орган. На это многие дарвинисты говорили,
что ряд математиков или математически образованных ученых принимают это учение,
значит, математические возражения несущественны. На это можно ответить: уже в
области физики и астрономии ученые, подобные Больцману, принимают абсолютно
невероятное. Аргументы о невероятности на таких людей подействовать не могут.
Это --- Тертуллианы наизнанку. Тот говорил: «Я верю (в догматы веры), потому что
это абсурд (противоречит данным материалистического мировоззрения)». Эти
говорят: «Мы принимаем, что наш мир совершенно невероятен; но мы готовы жить в
невероятном мире, чтобы не допустить проникновения в науку идеалистических
воззрений». Оба наиболее ценным считают не свободное мышление, а подчинение
определенным догматам, и в этом смысле оба --- представители людей, сознательно
верующих в невероятное или чудесное. А те лица, которые наивно думают, что
взгляды Больцмана, Дарвина и других материалистов целиком основаны на вполне
рациональных научных данных, являются выразителями слепой веры.

6. Антиномичность в науке и религии

От противоречий, абсурдов, невероятного не свободна оказывается и наука, мнящая
себя совершенно свободной от всякого суеверия. Как к этому относиться? Первый
путь --- примириться с неизбежным и считать, что антиномичность неразрешима. Для
определенной области так думал Кант. Один из комментаторов Канта, Файхингер,
опубликовал книгу под заглавием «Философия фикции» (ди философи дер «Альс об»),
где доказывал, что все основные понятия науки настолько противоречивы, что
могут считаться фикциями, но надо выбирать полезные фикции. Эта философия вовсе
не так редка, как может показаться на первый взгляд. Наш Пушкин сказал до
Файхингера: «Тьмы низких истин нам дороже нас возвышающий обман». У Горького мы
читаем: «Если к правде святой мир дорогу найти не сумеет, честь безумцу,
который навеет человечеству сон золотой». И, наконец, Ницше (вернее, Ницше
хронологически занимает середину между Пушкиным и Горьким) выразился так:
«Истина есть наиболее целесообразное заблуждение». Но в этом заключается как
будто различие между Тертуллианом и материалистически настроенными учеными.
Позиция Тертуллиана отдаляет его от науки, механический же материализм при его
слабой философской обоснованности дал, несомненно, очень много науке. Но
Тертуллиан, вероятно, бесплодный в науке, не был бесплоден в области этики. Ему
наряду с Августином принадлежит едва ли не первый протест против смертной
казни. В античные дохристианские времена никто как будто до этого не доходил.
Может быть, Тертуллиан, подобно Августину, не удержался на этой позиции, но тут
уж действовали иные причины.

Другой путь --- искать выход из антиномий в новом синтезе, то, что как было
указано выше, с особой отчетливостью выразил Гегель. И вот мы видим, что
позиция Тертуллиана вовсе не была единственной и даже типичной для христианской
церкви и что наряду с ней было другое, более мощное направление, стремившееся к
примирению противоречий, к синтезу всех источников знания, к примирению знания
и веры. Другой, более великий представитель патристики, Августин, определенно
указывает черты близости христианской и языческой философии (в частности,
Платона). Он был продолжателем направления одного из ранних апологетов
христианства, Юстина-мученика (казнен в Риме ок. 166 г.). Цитирую по «Истории
философии» (под ред. Г. Ф. Александрова и проч., 1941, том I, с. 385--387):
«Юстин доказывал, что почти все содержание христианского учения уже имеется в
языческой философии. И это потому, что у христианства и философии один и тот же
источник --- божественный логос, разлитый во всем мире. В Христе этот логос
только проявился во всей полноте. К христианам Юстин относил всех тех, кто
прожил свою жизнь „с логосом". Таковы из греков --- Гераклит и Сократ.
Теоретически логос признавали также стоики. Юстин имел большое влияние на
позднейших „отцов церкви" и на дальнейшее развитие христианской идеологии».
Здесь огромную роль сыграла знаменитая александрийская школа. Александрия в то
время была центром величайшей в античности точной
науки (Александрийский музей), там же был важнейший центр еврейской «диаспоры»,
в которой складывалась особая, иудейско-эллинистическая культура (виднейший
представитель --- Филон), там жил епископ Климент (ок. 150--215 гг.)
Александрийский, который «развил теорию объединения веры и знания, которая была
принята христианской церковью» (Ист. филос, с. 389), со своим преемником
Оригеном. Все это были последователи Платона. Был и ряд других вполне
ортодоксальных представителей этой школы. У нас обычно, когда вспоминают раннее
христианство, упоминают не о Клименте, а о Кирилле Александрийском,
представителе совсем другого направления, ожесточенном враге эллинской культуры
(гибель Гипатии, сожжение Александрийской библиотеки). Их не так трудно спутать
и по сходству имен, и по месту их деятельности, и потому, что оба причислены к
лику святых. Но не надо забывать, что Кирилл действовал в эпоху после неудачной
языческой реакции императора Юлиана Отступника, преследовавшего христиан.
Неудивительно, что реакцией на языческого императора, объявившего войну
христианству, было возникновение убеждения о принципиальной противности
христианского учения и языческой культуры. Это было повторение сходного
процесса. Основоположником направления, враждебного язычеству (наиболее яркий
пример --- Тертуллиан), был Татиан, который был слушателем Юстина-мученика.
«После казни Юстина Татиан перебрался в Сирию и отошел от церкви, усвоив
осужденные ею гностические воззрения» (Ист. филос, с. 387). И здесь казнь
почитаемого учителя заставила Татиана осудить вместе с палачами учителя всю
языческую культуру. Но победила в раннем христианстве линия Климента. Опять в
«Ист. филос», с. 389: «По Клименту, нет знания без веры и веры без знания.
Полная гармония их требует изучения всего круга человеческих знаний: „семи
свободных искусств". Никакой несовместимости между языческой философией и
христианским учением, согласно Клименту, нет: это как две ветви одного и того
же ствола. Истины христианства согласны с учением лучших из язычников.
Философия представляет собой как бы пропедевтику, преддверие христианства. В
философии истина содержится не целиком в одной какой-либо школе, а по частям во
всех. Хотя отличительным признаком подлинной науки является ее совпадение с
учением веры, однако, с другой стороны, истинное содержание самого писания
устанавливается только философским изучением. Главным приемом для введения
философии в христианство было у Климента, как и у филона, „аллегорическое"
объяснение Священного Писания».

Это аллегорическое толкование Писания через известного епископа миланского
Амвросия перешло к Августину (с. 392). В «Истории философии» отмечается, что
сочинение Августина «О граде божием», несмотря на наивность исторической
концепции, имело большое историческое значение как попытка дать обзор истории
человечества в целом, как попытка философии истории. На с. 396 читаем:
«Христианская церковь, искаженно отобразив, как в кривом зеркале, эллинскую
культуру, сберегла ее в этом виде до нового времени, когда постепенно античная
культура стала оживать в ее подлинном виде. Вклады самих патриотических
писателей в науку ничтожны и не могут идти ни в
какое сравнение с достижениями античного мира». Последнее замечание верно, но
есть ли здесь вина христианства? Подлинная античная, эллинская культура в своих
высших достижениях не была понятна солдафонскому Риму. Первый пожар
Александрийской библиотеки был во время взятия Александрии Юлием Цезарем
(несомненно, одним из культурнейших людей в римском понимании термина
культура). Убыток был до известной степени возмещен Антонием, который галантно
поднес Клеопатре огромное количество рукописей из разграбленного Антонием
Пергама. Но все это собрание после гибели Клеопатры отправилось с триумфальным
поездом в Рим, где и было сожжено. Государственные дотации Александрийскому
музею, позволявшие ученым спокойно работать, были прекращены. Наконец, с севера
надвигались варвары. И здесь мы читаем в «Истории философии», с. 394; «Учение
Августина о предопределении было религиозным фатализмом. Для христиан оно
служило идеологической опорой в тяжелой борьбе, какую начиная с V в. пришлось
вести западной церкви с нахлынувшими на Европу и Северную Африку варварскими
народами. Вера в предопределение и возведение к воле божества каждого действия
как отдельного христианина, так и всей церкви придавали христианской церкви
сплоченность и фанатическое упорство».

Экономическое потрясение Западной Европы и Северной Африки, вызванное
нашествием варваров и крушением Римской империи, было основной причиной того
упадка культуры, который характеризует «века мрака» (примерно до 1000 г.). Если
бы не было христианской церкви, сохранившейся благодаря своему «фанатизму»,
если бы в этой церкви и ее монастырях не тлели ростки, изучавшиеся лучшими
представителями культуры --- монахами, то и возрождение наук (которому
чрезвычайно помогли и сохраненные арабами и греками элементы эллинской
культуры) не было бы возможно. И потому конец первой книги «Истории философии»:
«Главное содержание патристики --- разработка религиозной идеологии --- имела для
научного прогресса человечества отрицательное значение, служила орудием
косности и застоя» --- совершенно не соответствует истине. Как могла бы
возродиться античная культура, если бы сохранилась только римская традиция?
Ведь средневековая философия (вершиной которой был доминиканец Фома Аквинат)
основывалась на Аристотеле (которого привели к согласию с христианской
идеологией), а Возрождение связано с именами Платона и Пифагора, а отнюдь не
Эпикура и Лукреция.

Но у Августина, как и Тертуллиана, есть как будто совершенно отрицательные
взгляды. Несмотря на то что оба они высказали прогрессивную мысль, что смертная
казнь несовместима с христианским учением, оба они не удержались на этой
позиции (История философии, с. 395): «Августин ревностно отстаивал право церкви
на принуждение в делах веры на том же основании, что принуждение к „истине"
вовсе не есть насилие, а забота о благе принуждаемого. Учение Августина
получило зловещий характер, превратившись в реальность церковной практики. Так
как всякий еретик будет вечно мучиться за гробом, то лучше ему претерпеть
сожжение здесь, на земле (хотя и это не вполне обеспечивает его от загробных
мук)».

Но ведь это как бы предвидение современных «прогрессивных» взглядов, что
насилие над вредными учениями не есть нарушение свободы. Как раньше боролись с
еретиками, так теперь борются с «ревизионистами», исходя из того положения, что
единодушие необходимо в трагические периоды человеческой истории. И смертная
казнь в мирное время, практически отсутствовавшая в дореволюционной России
(подвергались казни только покушавшиеся на царя), восстановлена революционной
властью в неслыханном размере и сохраняется даже тогда, когда можно сказать,
что трагические времена миновали.

Наука в Западной Европе в течение темного периода, последовавшего за разорением
Европы варварами, теплилась в монастырях, но не следует думать, что она велась
тайком в противовес официальной католической церкви. Нет, ведь начало подъема
цивилизации обычно считается совпадающим с началом XI в., когда на папский
престол вступил Герберт под именем Сильвестра II. Этот был ученейший муж своего
времени, получивший образование в мусульманском университете в Кордове.

В XII-XIII вв. именно католической церковью были основаны университеты. В XIII
в. францисканский монах Роджер Бэкон положил основание индуктивной логике.
Силлогистическая логика получила наибольшее свое развитие в системе Петра
Испанского, португальского ученого, который правил 8 месяцев под именем папы
Иоанна XXI (Минто. Индукт. и дедукт. логика, с. 18--19). Конечно, была всегда и
ретроградная оппозиция, восходящая идейно к Татиану.

Противник Герберта, епископ Оттон, уверял, что Герберт обязан своим высоким
положением только союзу со злыми духами (Уэвель. История индуктивных наук, т.
I, 1867, с. 582). Некоторые историки литературы видят в Герберте прообраз
Фауста. Роджер Бэкон то попадал в тюрьму при папах одного направления, то
освобождался при прогрессивных папах и все-таки умер на свободе в старости. Но
невежественная толпа видела во всех ученых колдунов и соучастников нечистой
силы, и от этого обвинения не были защищены даже служители религии: Фома
Аквинат, Роджер Бэкон, Михаил Скот, Роберт Гростет, епископ Линкольнский,
Альберт Великий, епископ Регенсбургский, папы Сильвестр II и Григорий VII
(Уэвель, там же, с. 381).

Как было указано, такими обвинителями были и некоторые духовные лица. Ясно, что
по отношению к дохристианским деятелям обвинение в колдовстве было еще более
естественным: сюда попали Аристотель, Соломон, Иосиф, Пифагор, Вергилий (там
же).

Тенденция синтеза науки и религии получила свое наибольшее выражение в трудах
Фомы Аквината, и мы знаем, что и сейчас философия Фомы (неотомизм, или просто
томизм) является далеко не исчезнувшим учением. Во время Возрождения борьба шла
не между наукой и религией (что будет еще разъяснено дальше), а между
консервативным направлением, связанным с Аристотелем, и новым, связанным с
именами Пифагора и Платона.

7. Игнорирование антиномий

В предыдущем параграфе было рассмотрено два пути преодоления антиномий:
признание их непреодолимыми, стремление к преодолению противоречия, примирение
разума с верой. Но есть еще третий путь --- полное их игнорирование. Он основан
на том гносеологическом постулате, что мы имеем уже очень много окончательно
установленных истин и что прогресс науки заключается в постепенном наращивании
таких абсолютных истин и в постепенном распространении их на неограниченно
большую область бытия. Это мнение выражено классическим представителем
детерминизма Лапласом во введении к «Аналитической теории вероятностей».
Цитирую по статье Елены Эйльштейн: «Лаплас, Энгельс и наши современники» в
сборнике «Закон, необходимость, вероятность». 1967, с. 235--236: «Все явления,
даже те, которые по своей незначительности как будто не зависят от великих
законов природы, суть столь же неизбежные следствия этих законов, как обращение
Солнца. Не зная уз, соединяющих их с системой мира в целом, их приписывают
конечным причинам или случаю, в зависимости от того, происходили ли и следовали
ли они одно за другим с известной правильностью или же без видимого порядка; но
эти мнимые причины отбрасывались по мере того, как расширялись границы нашего
знания, и совершенно исчезли перед здравой философией, которая видит в них лишь
проявление неведения. Мы должны рассматривать настоящее состояние Вселенной как
следствие ее предыдущего состояния и как причину последующего. Ум, которому
были бы известны для какого-либо данного момента все силы, одушевляющие
природу, и относительное положение всех ее составных частей, если бы вдобавок
он оказался достаточно обширным, чтобы подчинить эти данные анализу, обнял бы в
одной формуле движения величайших тел Вселенной наравне с движениями легчайших
атомов: не осталось бы ничего, что было бы недостоверно, и будущее, так же как
и прошедшее, предстало бы перед его взором. Ум человеческий в совершенстве,
которое он придал астрономии, дает нам представление о слабом наброске того
разума. Его открытия в механике и геометрии в соединении с открытием всемирного
тяготения сделали его способным понимать под одними и теми же аналитическими
выражениями прошедшие и будущие состояния мировой системы. Применяя тот же
метод к некоторым другим объектам знания, нашему разуму удалось подвести
наблюдаемые явления под общие законы и предвидеть явления, которые будут
вызваны данными условиями. Все усилия духа в поисках истины постоянно стремятся
приблизить его к разуму, о котором мы только что упоминали, но от которого он
останется всегда бесконечно далеким. Это стремление, свойственное роду
человеческому, возвышает его над животными; и успехи его в этом направлении
различают нации и века и составляют их истинную славу».

Детерминизм Лапласа вызывал самую разнообразную реакцию. Как я указывал в
параграфе (об абсурдном), наши ортодоксальные философы обвиняли Лапласа в
фатализме и даже в уклоне в мистицизм. Е.  Эйльштейн, из статьи которой взята
эта цитата, считает, что дело в неправильном понимании высказывания Энгельса.
Современные физики в большинстве своем полностью отрицают лапласовский
детерминизм и, склонны на элементарном уровне к индетерминизму. Но такой
выдающийся физик, как Альберт Эйнштейн, до конца своей жизни сохранил верность
лапласовскому детерминизму. Он заканчивает свою статью «Основы теоретической
физики» (Эйнштейн. Физика и реальность. Сборник статей. 1965, с. 76) следующими
словами: «Некоторые физики, в том числе и я сам, не могут поверить, что мы раз
и навсегда должны отказаться от идеи прямого изображения физической реальности
в пространстве и времени, или что мы должны согласиться с мнением, будто
явления в природе подобны игре случая. Каждому дозволено выбрать направление
приложения своих усилий, и каждый человек может найти утешение в прекрасном
изречении Лессинга, что поиск истины значительно ценнее, чем обладание ею». Мы
видим, что хотя Эйнштейн продолжает оставаться верным дифференциальному закону
(современное состояние полностью определяет последующее), но он не скрывает,
что это дело веры, субъективной индукции, и принципиально допускает иное
решение проблемы. В этом его отличие от Лапласа, который считал свой постулат
абсолютной истиной. Но, может быть, во времена Лапласа и не было оснований
сомневаться в верности его взглядов, а новые открытия заставили пересмотреть
вопрос? Нет, как совершенно правильно отмечает в статье Е. Эйльштейн,
детерминизм Лапласа имеет две стороны: онтологическую и гносеологическую, но
она дальше разбирает этот вопрос недостаточно широко.

В чем гносеологическая ценность высказывания Лапласа? В возможности
предвидеть явления. Эта возможность блестяще оправдалась в астрономии,
механике и других точных науках, и отсюда Лаплас делает заключение, что тем же
путем принципиально можно предвидеть все. Принципиально, но не фактически, так
как сам Лаплас отчетливо сознает, что этот его постулат недоказуем и
неопровержим, так как там, где нельзя произвести строгих математических
расчетов, предвидение событий невозможно. В этом и заключается, как
справедливо указывает, следуя многим авторам (с. 239), Е. Эйльштейн,
отличие лапласовского детерминизма (механического) от фатализма, другой
формы детерминизма (с. 329): «Фатализм утверждает, что каждое событие
детерминировано отдельно, одной и той же причиной, действующей вне сферы
материальной природы. Таким образом, фатализм провозглашает фактическую
независимость явлений --- механический же детерминизм Лапласа является
интегральным, основанным на взаимодействии всех решительно частиц
материального мира. Мы знаем, что фатализм --- очень древнее понимание мира и,
в частности, на фатализме основана вся астрология, составление гороскопов,
которую не отрицали и такие революционеры в науке, как Коперник и Кеплер.
Астрология основывалась на экстраполяции безусловных фактов: влияние луны и
солнца на морские приливы и астрологический принцип всемирного тяготения,
отвергнутый Галилеем и реабилитированный Ньютоном, использовался и
Лапласом, хотя он был и не механической природы. Астрологи приводили и
многие случаи удавшихся предсказаний, а неудачные приписывали или
злоупотреблениям, или недостаточному совершенству методов. Гадание,
предвидение будущего и сейчас
широко распространены и не только в капиталистическом мире. Вероятно, не
проделано еще достаточно большой работы, чтобы показать, что процент
удавшихся предсказаний именно таков, какой надо ожидать, если удачные
предсказания являются делом случая. Поэтому астрологические методы могут
быть опровергнуты, лапласовский же детерминизм неопровержим
непосредственно. Но косвенно он может быть опровергнут, и это опровержение
известно каждому, конечно, и самому Лапласу. Ведь детерминизм Лапласа
связан и с его онтологией: принципиальным отрицанием индетерминизма (случая в
подлинном аристотельском смысле: того, что может и не быть) и отрицанием
конечных причин. Но можем ли мы на основании конечных причин делать
прогнозы? Каждый из нас делает. Я приду к вам послезавтра в десять часов
утра и, как правило, этот прогноз осуществляется, причем независимо от
разнообразия условий. Предположим, для осуществления своего прогноза вы
намерены были выйти за час и ехать трамваем. Но вас кто-то задержал на
полчаса или вы проспали. Вместо трамвая вы едете на такси. Осуществляется
совершенно иной комплекс движения материальных частей, подчиненный конечной
причине: вашему желанию прийти вовремя. Ссылаясь на вышеприведенную цитату,
можно сказать, что фатализм, утверждающий, что каждое событие
детерминировано отдельной причиной, действующей независимо от материальной
природы, вовсе не глупость, так как мы этот фатализм осуществляем
ежедневно. А экстраполяция этого законного подхода к явлениям на
бесконечность не менее законна (вернее, не более беззаконна), чем
экстраполяция механических законов на всю Вселенную.

Но любопытно, что Лаплас делал ошибки с экстраполяцией положений,
справедливых в определенной области, и там, где он выступал не как
философ-дилетант, но и в тех областях, где он был бесспорный выдающийся
специалист. Кроме астрономии и механики, он оставил глубокий след в теории
вероятностей и в исчислении бесконечно малых. И вот современный математик Д.
Пойа в очень интересной книге «Математика и правдоподобные рассуждения»
(Москва, ИЛ, 1957), останавливается на попытке Лапласа связать индукцию с
вероятностью (с. 395--398). «Когда вероятность простого события неизвестна, то
можно предполагать ее равною всем числовым значениям от нуля до единицы», ---
говорит Лаплас в «Опыте философии теории вероятностей»; «Это равное
распределение незнания», --- насмехаются его оппоненты. Пойа указывает на
явно глупые применения этого принципа (с. 398). «Эти применения кажутся
глупыми, но нет ничего глупее следующего применения, принадлежащего самому
Лапласу. „Если отнести древнейшую историческую эпоху за пять тысяч лет, или за
1 826 213 дней, назад и принять во внимание, что солнце постоянно восходило за
этот промежуток времени при каждой смене суток, то будет 1 826 214 шансов
против одного за то, что оно взойдет и завтра"». Пойа прибавляет: «Я,
конечно, остерегся бы предложить такое пари норвежскому коллеге, который мог
бы для обоих устроить воздушное путешествие в какое-нибудь место за
полярным кругом». Лаплас, конечно, знал, что за полярным кругом Солнце
зимой не восходит, но он об этом просто позабыл.

Еще любопытнее другая «забывчивость» Лапласа, касающаяся непосредственно его
формулировки детерминизма. Из его формулировки вытекает, что если мы знаем
исходное состояние и все дифференциальные уравнения, описывающие
зависимость между элементами бытия, то на каждый следующий момент мы можем
получить только одно решение. Но на самом деле в ряде случаев мы имеем так
называемые особые точки, где происходит как бы разветвление и получается не
одно решение, а несколько. На это обратил внимание французский математик
Буссинеск в интересной книге под заглавием «Примирение подлинного
механического детерминизма с существованием жизни и нравственной свободой».

Эта книга --- изложение своеобразного дуализма. Вне жизни нет особых точек, все
развивается сообразно механическому детерминизму. А жизнь --- это область
приложения особых интегралов, где существуют особые точки, где механический
детерминизм недостаточен. Что же решает, каким путем пойдет процесс,
дошедший до особой точки? Буссинеск склонен принимать специфически
витальные направляющие силы, но, конечно, можно здесь предоставить роль и
чистому случаю. Таким образом исключение случая и конечных причин сделано
Лапласом не потому, что он был хорошим математиком, а потому что он был,
скажем мягко, не слишком выдающимся философом. В предисловии к книге
Буссинеска профессор Жаке указывает, что эта книга, освобождая нашу мысль от
пут механического детерминизма, делает необязательной тягостную дилемму:
выбирать между трансцендентальным идеализмом Канта и предустановленной
гармонией Лейбница.

Как отнеслись математики к книге Буссинеска? К ней хорошо отнесся
выдающийся математик А. Пуанкаре, а наша соотечественница, Софья
Ковалевская, которая была не только математиком, но и писателем, написала
роман (по-шведски; кажется, есть русский перевод) «Борьба за счастье», где в
популярной форме изложила мысль, что в каждой человеческой судьбе есть
«точки разветвления» и выбор пути не предопределен совершенно прошлым. Но,
может быть, особые интегралы стали известны после Лапласа? Были ли они
известны до Лапласа, не знаю, но хорошо известно, что в разработке учения об
особых интегралах особое значение имели как раз работы Лапласа. Он о
собственных достижениях позабыл в угоду любезной ему метафизики. Лаплас,
по-видимому, и не подозревал, что он просто принял одну из тез великой
антиномии необходимости и свободы, которая волновала мыслителей
средневековья и которая была отчетливо формулирована Кантом. В Средние века
богословы спорили: если мы отрицаем свободу воли человека, то почему мы его
считаем ответственным за его поступки (этот спор велся и среди евреев между
саддукеями и фарисеями), а если мы признаем его свободу, то признаем и
индетерминизм, а значит, ограничим всеведение Божие.

Лаплас встал на сторону детерминистов и, упразднив Бога как реальность, он
возвел его как фикцию в абсолют. Всякая иная метафизика казалась ему
совершенным сумасбродством. Вот как он пишет про гениального Кеплера
(цитирую по: Уэвель. Ист. индукт. наук, I, 1867, с. 533): «Печально за
человеческий ум видеть, как этот великий человек даже в своих последних
произведениях с наслаждением предается своим химерическим умствованиям и
считает их душой и жизнью астрономии». Уэвель на той же странице приводит без
указания авторов и другие аналогичные высказывания: «Этот успех (Кеплера)
может устрашить тех, кто привык считать опыт и строгую индукцию
единственным средством успешно исследовать природу»; «Удивительное счастье
Кеплера схватывать истину среди самых диких и нелепых теорий»; «Опасность
следовать его методу в отыскании истины». Сам Уэвель, хотя он и является
убежденным сторонником индукции и не разделяет метафизики Кеплера, однако,
судит более объективно, с. 537: «Мистические стороны его мнений, как,
например, его вера в астрологию, его убеждение, что земля есть животное, и
множество туманных нравственных духовных и материальных представлений о
силах, управлявших, по его мнению, Вселенной, --- по-видимому, не только не
мешали его открытию, но скорее подталкивали его изобретательность и
воодушевляли его труды. В самом деле, где ум обладает ясными научными
идеями об одном предмете, там мистицизм по другим предметам, кажется, вовсе не
бывает неблагоприятным для успешного хода исследования. Я полагаю поэтому, что
мы можем видеть в характере Кеплера общие черты характера научного
открывателя, хотя некоторые из них преувеличены, а другие проведены слишком
слабо». Не забудем, что Кеплера особенно ценили и Маркс и Энгельс.

Лаплас в данном случае пал жертвой своей неумеренной преданности весьма
посредственной материалистической французской философии конца XVIII в.
Кажется, сейчас никто не считает этих философов крупными. Даже
основоположники марксизма при всей своей приверженности к материализму
искали философскую основу своего учения в немецкой идеалистической
философии, пытаясь только (с каким успехом, не будем пока говорить)
перевернуть ее на голову. Но великолепную оценку французской
материалистической философии конца XVIII в. (бравшей монополию на
«просветительство») дал наш поэт М. Ю. Лермонтов в изумительном по красоте
стихотворении «Последнее новоселье». Конечно, и Лермонтов хватил через
край, обозвав весь французский народ «жалким и пустым» (включая всех
французских мыслителей и ученых), переоценил он и Наполеона (ошибка совсем
простительная, принимая во внимание, что в такую ошибку впадали три великих
немца: Бетховен, Гете и Гегель), но в отношении французского материализма
вполне справедливы его слова:

Ты жалок потому, что вера, слава, гений,

Все, все великое, священное земли,

С насмешкой глупою ребяческих сомнений

Тобой растоптано в пыли.

Из славы сделал ты игрушку лицемеръя,

Из вольности --- орудье палача,

И все заветные, отцовские поверья

Ты им рубил, рубил с плеча.

Поэтому отдадим должное Лапласу как великому ученому, достойному
продолжателю еще более великого Ньютона, и не будем придавать значения его
философствованиям, следствию его убеждениям чувства,
а не разума. В них он не подчиненный только доводам разума ученый, а слепо
верующий весьма сомнительным догмам человек. «Орлам случается и ниже кур
спускаться, да курам никогда до облак не подняться». Но многие куриные
философы больше всего восхищаются куриной частью мировоззрения Лапласа.

Глава 2. ВРЕД ОТ ЦЕРКВИ ДЛЯ НАУКИ

1. Общая постановка вопроса

Все, что написано в предыдущей главе, показывает, что не только религия, но и
наука в лице своих выдающихся представителей не чужда суеверий и что они часто
не сознают, что высказываемые ими как абсолютные истины положения по своей
обоснованности ничуть не лучше тех нелепых догматов, которые выставляются
религиями. На это можно ответить, что религиозные суеверия особенно опасны тем,
что господствующие церкви, связанные с правительствами классов-угнетателей,
применяют такие средства борьбы, которые угнетают свободное развитие науки.
Поэтому многие религиозные люди считают, что не религия как таковая, а
богословие, теология господствующих церквей органически связаны по своему
догматизму с систематической борьбой со свободой науки. Этой мыслью проникнута,
например, книга Уайта «История войны науки с теологией», в двух томах (первое
издание в 1896 г., второе в 1960 г.): \emph{White A.D. A history of the
warfare of science with theology in Christendom.}

Конечно, в этом утверждении заключается большая доля истины, догматизация
всегда приводит к преследованию инакомыслящих. Но в данном случае нас
интересует вопрос, в какой мере церковь, в особенности католическая,
преследовала науку и в какой мере она затормозила ее развитие. Мысль Уайта
(который в свое время был посланником США в С.-Петербурге) выражена ясно на с.
XII его предисловия: «Мое убеждение состоит в том, что наука, хотя она
очевидно и победила догматическое богословие, основанное на библейских
текстах и древнем способе мышления, будет идти рука об руку с религией и
что, хотя теологический контроль будет продолжать уменьшаться, религия,
понимаемая в признании "Силы во Вселенной вне нас, стремящейся к
справедливости", и из любви к Богу и к ближнему, будет все усиливаться и
усиливаться, не только в ученых учреждениях Америки, но и во всем свете».

Как это ни может показаться странным, но сходные мысли высказал и А.
Эйнштейн в статье «Религия и наука», опубликованной в сборнике «Мое
мировоззрение» (англ. перев. 1935 г.), с. 27: «Этическое поведение человека
должно быть основано на симпатии, воспитании и социальных связях;
религиозного основания не требуется. Печально было бы, если бы человек был
ограничен лишь страхом наказания и надеждой на посмертную награду. Отсюда
ясно, почему церкви всегда боролись с наукой и преследовали адептов науки. С
другой стороны, я утверждаю, что космическое религиозное чувство является
сильнейшим и благороднейшим побуждением к научным исследованиями. Только те,
кто
понимают огромные усилия и прежде всего преданность делу, которые требуются для
тех, кто прокладывает новые пути к теоретической науке, могут понять всю
силу эмоций, из которых может произойти такое дело, отдаленное от
непосредственных требований жизни. Какое глубокое убеждение в
рациональности Вселенной и какую жажду познать хотя бы слабое отражение
разума, обнаруживаемого в мире, должны были иметь Кеплер и Ньютон, чтобы они
могли потратить годы одинокого труда в распутывании основ небесной
механики. Те люди, знакомство которых с научными исследованиями связано
преимущественно с практическими результатами, легко приобретают совершенно
превратные представления о складе ума людей, которые, будучи окружены
скептическим миром, устанавливают связь с людьми подобного им мышления, как бы
они ни были рассеяны по земле и по столетиям. Только тот, кто посвятил свою
жизнь сходным целям, может живо представить себе, что вдохновляло этих
людей и давало им силу оставаться верными своему призванию несмотря на
многочисленные провалы. Это --- космическое религиозное чувство, которое
давало людям такую силу. Один современник сказал, и не без основания, что в
современную материалистическую эпоху серьезными учеными могут быть только
глубоко религиозные люди». Как увидим дальше, Эйнштейн несколько изменил
свои взгляды в последние годы своей жизни, но пока ограничимся этой
цитатой. Можно возразить, что понимание Эйнштейном религии не имеет ничего
общего с обычным пониманием и что отрицательное отношение его к церковной
религии выражено вполне ясно.

В числе многочисленных жертв святой инквизиции были, конечно, и ученые, но они
составляли небольшой процент общего числа. Наиболее знаменитые случаи ---
Джордано Бруно и Г. Галилей. Их и разберем.

2. Дело Джордано Бруно

Как известно, Бруно (1548--1600) был доминиканским монахом, проведшим долгие
годы в странствиях, выступавший с лекциями на диспутах по богословию,
философии, интересовавшийся и общественно-политическими вопросами.
Неосторожно вернувшись в Венецию, он был там захвачен агентами инквизиции и
после нескольких лет заточения в тюрьме сожжен в Риме. За что его сожгли? За
его научные работы? В точных науках он не дал ничего сколько-нибудь
существенного. Но он был пламенным пропагандистом Коперника и еще больше
Николая Кузанского, которого он ставит чрезвычайно высоко и сравнивает с
Пифагором.

Николай Кузанский (1401--1464) был сторонником бесконечности Вселенной,
движения Земли и множества населенных миров, где могли обитать существа,
более совершенные, чем человек. Бруно воспринял эти идеи Кузанского и в
этом отношении отличался от Коперника, который признавал ограниченность
Вселенной. И Кузанский и Бруно были склонны к пантеизму, и за это их обоих
упрекали противники. И Кузанский и Бруно были склонны к платонизму и
гилозоизму, т. е. учению об одухотворенности материи, оба являются
представителями
диалектического мышления с тезисом о совпадении противоположностей (см.
статья «Николай Кузанский», Философский словарь, 1963, с. 309). В области
натурфилософии и философии вряд ли можно указать что-либо существенно
новое, что внес Бруно по сравнению с Кузанским. Кузанский был почти забыт в
течение несколько сотен лет, и даже в трехтомной «Истории философии» Гегеля он
вовсе не упоминается, хотя его последователю, Бруно, отведено достаточно
места. Сейчас положение изменилось: сочинения Кузанского издаются и на
Западе и у нас (Избранные философские сочинения, Москва, Госсоцэкгиз,
1937), ему отведено достаточно места и в «Истории философии» (т. II, 1941, с.
42--44), БСЭ и других изданиях. Выдающееся значение этого мыслителя не
отрицает сейчас как будто никто. Какова же судьба Кузанского? Он был
епископом, кардиналом, последние года провел в Риме, где, по свидетельству
многих, играл роль вице-папы. Он отнюдь не замыкался в ученой деятельности. Он
деятельно работал по объединению Восточной и Западной церквей, принимал
участие в организации похода христианских государств против турок, боролся с
массовым паломничеством к святым местам. Боролся с богослужебным
формализмом, суевериями, верой в чудеса, с конкубинатом духовенства. Он
подвергался и репрессиям, но не со стороны духовных властей, а со стороны
герцога Тирольского Сигизмунда, который некоторое время держал его в тюрьме за
то (по жалобам монахинь), что Кузанский слишком энергично боролся с
распущенностью, царившей в женских монастырях («Очерк жизни Кузанского»,
написанный Лопашовым в «Избранных сочинениях»). Хотя против Кузанского его
противники выступали с обвинениями в пантеизме, это не имело для него
никаких последствий и в индексе запрещенных книг сочинения Николая
Кузанского никогда не фигурировали. И после сожжения Бруно в 1600 г. не
было объявлено об осуждении учения Коперника. Это было сделано лишь в 1616 г.,
и поводом к этому были не сочинения самого Коперника или Бруно, а сочинение
кармелита П. Фоскарини (1615), где тот пытался доказать отсутствие
противоречий между доктриной Коперника и доктриной церкви. «По декрету
конгрегации индекса запрещенных книг от 5/III 1616 г., сочинение Фоскарини
было "совершенно запрещено и осуждено"» (Идельсон. Галилей в истории
астрономии. Вопросы истории естествознания и техники, вып. 16, 1964, с.
62). Вот ослушание этому постановлению было со временем и поставлено в вину
Галилею (о чем будет речь дальше), но никак не могло быть поставлено в вину
Бруно. При всех своих ужасных законах инквизиция достаточно строго
придерживалась принципа, что закон обратной силы не имеет и по
несуществующим законам не судили (в отличие от некоторых современных
правителей).

Как это указывается и в предисловиях к советским изданиям сочинений Бруно, он
был казнен не за свою научную, а за свою общественную деятельность.

Он решительно осуждал господствовавшие тогда в католической церкви порядки,
есть мнение, что он даже перешел в протестантство, но и там не поладил с
кальвинистами. Обладая необузданным темпераментом, он выступал (в отличие от
Кузанского) очень резко и тем нажил огромное количество врагов. В биографии
его, написанной
А. Штекли (Жизнь замечательных людей, выпуск 21(395), 1964), он выставлен
почти как воинствующий атеист, который еще в доминиканском монастыре
выбросил иконы (с. 19) и всю свою жизнь высказывал резко антирелигиозные
суждения. Образ жизни он вел, судя по указанной биографии, тоже совсем не
благочестивый, но к этому греху в то время относились снисходительно.
Будучи очень образованным человеком и выдающимся оратором, он, конечно, был
опасен непосредственно для верхушки католической церкви, и скорее можно
удивляться, что процесс длился так долго (почти восемь лет), чем тому, что он
не избежал костра. На костер отправляли тогда и лиц, не имевших никаких
научных заслуг и вполне ортодоксально мысливших, если находили их опасными или
слишком неприятными для церкви, причем, конечно, суровость репрессии
соответствовала серьезности политической ситуации и в значительной степени
характеру руководителей церкви. Поэтому Дж. Бруно можно зачислить в ряды
мучеников борьбы со всякого рода деспотизмом, за улучшение общественного
строя, за устранение злоупотреблений власть имущих, туда же, куда отнесем
Савонаролу, целый ряд францисканцев и прочих лиц, но к числу мучеников
науки мы его причислить не можем.

3. Дело Галилея

Обратимся теперь к Галилею (1564--1642). Если спросить среднего
интеллигента, кто такой Галилей, то он, конечно, ответит: борец за
коперниково мировоззрение, осужденный за это инквизицией и принужденный
отречься от своих взглядов. Верно это? Конечно, верно, но только это ---
неполная истина, сходна по своему значению с такой, например, истиной: Кант ---
немец, никогда не выезжавший из Кенигсберга. И в «Философском словаре» 1963 г.,
с. 86, где дано краткое изложение его учения (в общем правильное),
приводится только один его труд в качестве основных: «Диалоги о двух
главнейших системах мира --- птолемеевой и коперниковой» (1632). В «Истории
философии» 1941 г., т. II, изложение гораздо полнее и более приближается к
полной истине. Там правильно указано другое фундаментальное сочинение
Галилея: «Беседы и математические доказательства, касающиеся двух новых
отраслей науки» (1638) (правильнее: двух новых наук). Эти две науки,
основание которых положил Галилей (вполне сознававший свое значение как
основоположника этих наук) --- динамика и сопротивление материалов.
Основанием динамики он встал рядом с Архимедом, основателем статики, и
заслужил справедливо звание одного из величайших пионеров точных наук. Он дал
толчок развитию экспериментального метода, настаивал на математизации наук
(«Измеряй все доступное измерению и делай неизмеримое измеримым»),
чрезвычайно много сделал по борьбе с догматизмом перипатетиков и является
основателем того механического материализма, видным представителем которого
является Лаплас (в большей степени, чем Ньютон). У него есть мысли,
послужившие для развития теории вероятностей и теории множеств. Что
касается астрономии, то он способствовал продвижению коперниковских идей как
своими теоретическими работами (принцип
относительности Галилея), так и наблюдениями в телескоп (солнечные пятна,
фазы Венеры, спутники Юпитера и т. д.), но в развитии математической теории
гелиоцентрической системы он не только не участвовал, но даже
противодействовал принятию нового крупнейшего шага в этом направлении,
работам не менее гениального Кеплера, своего современника, работы которого ему
были известны и с которым он был в переписке. Здесь сыграло роль именно его
основное стремление --- разработка идей по механике, которой он занимался всю
свою сознательную жизнь. Поэтому сейчас можно сказать с полной
уверенностью: если бы исчезли «Диалоги», то астрономия не претерпела бы
существенной потери, так как к моменту их напечатания (1632) великий Кеплер
(1571--1630) уже закончил свой жизненный путь, а по линии Кеплера Галилей не дал
ничего, а вот если бы исчезли «Беседы», то развитие механики несомненно
задержалось бы, так как гении, подобные Галилею, рождаются нечасто.
Механике Галилей посвятил всю свою жизнь. Первые его работы посвящены
механике и прикладным задачам (по военной архитектуре и фортификации):
Теоремы о центрах тяжести твердых тел (около 1585 г.), Гидростатические
весы (1586), Трактат о движении (1590), где впервые --- расхождение с
аристотелевской динамикой, Механика (между 1593 и 1599 г.), где впервые
применяется термин «момент» силы. Все перечисленные произведения не были
опубликованы при жизни Галилея (Идельсон. Галилей в истории астрономии.
Вопросы истории естествознания и техники, вып. 16, 1964, с. 52). Они
соответствуют возрасту Галилея 21--35 лет. В этом интервале он был
профессором математики у себя на родине, в Пизе (1589--1592), где и наблюдал
качание маятника (паникадила) в Пизанском соборе и падение тел со
знаменитой падающей башни. После Пизы он переехал на 18 лет в Падую, где
занимался разнообразными вопросами технического характера, читал лекции в
Падуанском университете на кафедре математики и имел как лектор огромный
успех. В лекциях по астрономии он упоминал о наличии ученых, считающих
Землю подвижной, придерживался, однако, точки зрения Аристотеля и Птолемея о
неподвижности Земли. Но, судя по письму Галилея Кеплеру от 4 августа 1597 г.,
написанному в ответ на получение книги Кеплера «Мистериум космографикум»
(1596), он уже тогда был скрытым коперниканцем. Это письмо крайне интересно
(привожу по Идельсону, с. 53): «Твою книгу я прочту с тем большей охотой, что
на точку зрения Коперника я встал уже много лет тому назад, и мне удалось на
основе ее найти объяснение многим явлениям природы, которые, без сомнения, не
могут найти объяснения на основе общепринятых положений. Я записал много
доказательств и много опровержений рассуждений, основанных на
противоположной точке зрения; но выпустить все это в свет я не решался,
устрашенный судьбой Коперника, нашего учителя, который хотя и заслужил себе
бессмертную славу у немногих, но со стороны бессчетного числа людей (ибо так
велико число глупцов) подвергся лишь насмешке и освисту. Я решился бы,
действительно, продолжать мои рассуждения, если бы существовало много
людей, подобных тебе, Кеплер, но их нет, и я откажусь от этих занятий». Это
письмо интересно во многих отношениях. Во-первых, из него ясно, что Галилей
интересовался астрономией давно и был
убежденным коперниканцем, но всерьез этим не занимался, так как и без того дел
у него было по горло. Во-вторых, к тому времени он уже испытал резкую
оппозицию своим работам. Его «Трактат о движении» (Де Моту гравиум, 1590)
вызвал против молодого профессора бурю негодования со стороны большинства
профессоров, сторонников Аристотеля (История философии, т. II, 1941, с.
65), и эта буря была основана на чисто перипатетической косности, вне
всякой связи с религией. Галилей, может быть, переоценил значение числа
«глупцов». В-третьих, ясно, что Галилей вовсе не считал себя обязанным идти
напролом и считал возможным, будучи коперниканцем, излагать астрономию по
Птолемею; и, наконец, в-четвертых, из письма ясно, что он еще не успел
прочесть присланной ему книги, вряд ли он согласился бы с ее содержанием.

Книга эта считается ошибкой Кеплера, но сейчас, ввиду усиления интереса к
истории науки, ее содержание передается многими. В частности, в «Истории
астрономии» Паннекука, 1966, с. 241--242, изложено ее содержание, а на с. 243
(рис. 31) дана модель Вселенной по этой книге Кеплера. Кеплер дал
объяснение строения планетной системы на основе «Платоновых тел», т. е.
правильных многогранников. «Если на каждом из шести планетных кругов
построена сфера, мы можем между каждой парой последовательных сфер, считая их
точно концентрическими, построить одно из правильных геометрических тел
таким образом, чтобы вершины его были расположены на внешней сфере, а
плоскости были касательными к внутренней сфере». Для некоторых расстояний
получилось хорошее совпадение. С удивлением читаем дальше в книге марксиста с
предисловием нашего известного астронома Кукаркина такие слова на той же
странице: «Однако совпадение слишком велико, чтобы быть случайным. Итак,
Кеплер, движимый более астрологическими идеями, расположил 5 геометрических тел
в последовательности, идущей от центра наружу: 18-, 20-, 12-, 4-,
6-гранники, между шестью планетными сферами... Раскрыв этот секрет строения
мира, Кеплер поднял теорию Коперника на уровень, значительно превосходящий
спорное мнение, основанное на неопределенном эмпиризме, и сделал ее
фундаментальной философской истиной». Хорошо известно, что эту книгу
Кеплера считают неудачным началом великого пути и славу Кеплер приобретет не
ею, а своими тремя законами, изложенными в более поздних сочинениях. Ясно, что
такая книга не могла привлечь Галилея, которому остался чужд и дальнейший
путь Кеплера. Галилей вовсе не был далек от пифагореизма (С. И. Вавилов.
Галилей. БСЭ, 2-е изд., 1952, т. 10; с. 126): «Мир был для Галилея открытой
книгой, написанной на математическом языке в виде треугольников, кругов и
других геометрических фигур», но астрологии и представлениям о связях между
планетами он был совершенно чужд. Кеплер ответил Галилею на его письмо
(Паннекук, с. 242): «Решайтесь, выступим одновременно. Дружными усилиями мы
сдвинем этот экипаж. Своими доказательствами Вы поможете тем из наших
сторонников, которые теперь еще придерживаются неправильных суждений. Я
думаю, что очень немногие из знаменитых математиков Европы будут против
нас, так как могущество истины бесспорно». Галилей все же обратился к
астрономии, но пошел своим путем (у Кеплера
было плохое зрение, и он не был хорошим наблюдателем). Потратив много труда,
Галилей сконструировал телескоп и в ночь 7 января 1610 г. (когда ему было уже
45 лет) в Падуе оказался первым человеком, смотревшим на небо вооруженным
взглядом (Идельсон, 1964, с. 54). Перед ним открылся совершенно новый мир и при
первых же наблюдениях вблизи Юпитера обнаружились какие-то звездочки --- спутники
Юпитера. «Действительно, было от чего прийти в экстаз и восхищение; вдаль
уходили схемы Аристотеля и Птолемея; как дым, рассеивалась мистическая
надстройка над Коперником, которую предполагал Кеплер; новый мир, реальный и
величественный, открывался перед человеком; материя этого мира представлялась в
богатстве и разнообразии, которое надлежало теперь осознать».

Кеплер, как мы знаем, был плохим наблюдателем и пользовался преимущественно
наблюдениями Тихо де Браге: он наблюдал Вселенную умственными (как говорили в
старину, «умными») очами и, как известно, таким способом сделал огромный шаг
вперед. Галилей же продолжал наблюдения до тех пор, пока болезнь глаз
(катаракта) в 1638 г. (когда ему уже было 74) не заставила его прекратить это
занятие (Идельсон, с. 60).

Галилей же не пытался совершенствовать теорию орбит солнечной системы, хотя при
его жизни были открыты Кеплером три его великих закона (два первых опубликованы
в 1609 г. в сочинении (Астрономия нова) «О движении Марса» и третий в 1619 г. в
«Мировой гармонии»). Галилей же в знаменитом «Диалоге», опубликованном в 1632
г., решительно отвергает и даже высмеивает Кеплера по поводу его мнения о
влиянии Луны и полностью игнорирует его великие законы. «И среди великих людей,
рассуждающих об этом удивительном явлении приливов, более всех других удивляет
меня Кеплер, который, будучи наделен умом свободным и острым и хорошо знакомым
с движениями, приписываемыми Земле, допускал особую власть Луны над водой,
сокровенные свойства и тому подобные ребячества» (Галилей. Диалоги о двух
главнейших системах мира --- птолемеевой и коперниковой. Перевод и примечания А.
И. Долгова, 1948, с. 326). В примечаниях к книге А. И. Долгов указывает, что
уже в эпоху Галилея указывалось вполне основательно, что приливы и отливы
достигают максимальной величины в эпоху равноденствий, а не солнцестояний (в
частности Фр. Бэконом), как следовало по теории Галилея. Свои теоретические
выводы Галилей не сравнивал с практическими данными, т. е. в данном случае
отступал от тех принципов экспериментального метода, которые он так энергично и
эффективно проводил в других своих работах. А. И. Долгов на той же странице
одобряет Галилея за его отрицание скрытых свойств, но вместе с тем указывает,
что в отношении Кеплера он был не прав. Это хорошо отмечено и А. Эйнштейном в
статье «О Галилее и его "Диалоге"» (Вопросы истории естествознания и техники,
вып. 16, 1964, с. 32). Эйнштейн не скрывает своего восхищения перед Галилеем и
вместе с тем пишет: «Подкреплением системы Коперника, сверх качественных
доводов, являлось бы только определение "истинных орбит" планет, а эта
проблема, казалось, почти непреодолимой трудности, была, однако, разрешена
Кеплером (при жизни Галилея) истинно гениальным способом. То, что это решающее
достижение
не оставило следа в работах, которым Галилей посвятил свою жизнь, иллюстрация
того факта, что творческие личности часто бывают невосприимчивы... Стремление
Галилея доказать механическое движение Земли ввело его в заблуждение при
создании им его ошибочной теории приливов и отливов. Блестящие рассуждения,
изложенные в последней беседе, сам Галилей признал бы бездоказательными, если
бы не его темперамент». В чем тут дело? Каким образом гениальный основоположник
механики мог сделать такие ошибки? Тут именно дело в «темпераменте», на что
указывает Эйнштейн, на его «убеждении чувства», а не разума. Механика была
лейтмотивом всей его жизни, и, экстраполируя (под влиянием уже убеждения
чувства, а не разума) применимость механики за пределы ее действительной
применимости, он был убежден, что его механическая, ошибочная теория приливов и
отливов (связывающая приливы и отливы с вращением Земли) поможет обосновать и
систему Коперника. Поэтому первоначальное название «Диалогов» было «О приливах
и отливах» и последняя, четвертая беседа, где теория приливов обосновывалась,
была завершающей, едва ли не самой важной с точки зрения Галилея. Название было
изменено по совету лиц из папской курии, продвигавших эту работу в печать.
Отсюда понятно, почему Галилей чрезвычайно упростил систему Коперника и
Птолемея, совершенно отказавшись от эпициклов, так как для его механической
теории приливов и отливов такая сложность системы была помехой, понятно также,
что он не делает отличия между системами Аристотеля и Птолемея, хотя, само
собой разумеется, не мог не знать их различий. Ведь у Аристотеля --- система
концентрических сфер, движение по правильным кругам, а у Птолемея --- большое
число эпициклов и Солнце движется не точно вокруг Земли, а вокруг эксцентра. Но
система Аристотеля удовлетворяла строгим философским требованиям --- совершенство
--- и не годилась для прогноза, система же Птолемея вызывала резкие возражения
перипатетиков, например знаменитого Ибн-Рушда (Аверроэса), но по ней можно было
вычислять с удовлетворительной точностью движения планет, затмения и проч. Для
Галилея важно было другое --- применить новую механику для механического
доказательства движения Земли вокруг Солнца при помощи приливов и отливов. Эта
сокровенная мысль изложена им в письме к кардиналу Орсини от 1616 г. (Идельсон.
Вопросы ист. естест. и тех., вып. 16, 1964, с. 64), и через 15 лет это письмо
составит самый нерв его «Диалога». Теория Галилея исключает действие какой бы
то ни было приливообразующей силы, и возможность таких сил Галилей резко
отрицает. «Мой рассудок не может приспособиться к тому, чтобы подписаться под
действием света, темперированного тепла или возбуждения явлений через скрытые
качества и тому подобными бреднями; все это не только не является, но и не
может явиться причиной прилива; скорее уж обратно, прилив в мозгах ведет здесь
к этой болтовне и к крикливым суждениям, а не к размышлениям над более
глубокими явлениями природы и к их исследованиям» (там же, с. 79). Как видим,
здесь Галилей покидает даже обычный для него спокойный тон рассуждения. В своих
выражениях Галилей объявил войну той доктрине Средневековья, которая
приписывала приливы таинственному влиянию Луны, доктрине, за которой
шел и Кеплер. Для Галилея мир «осязаем» и нет скрытых явлений. Эскиз
космической механики Галилея не нашел подтверждения в наблюдении и опыте. Там
же Идельсон пишет: «Ньютон в единой формуле дальнодействия, над раскрытием
сокровенного смысла которой человеческая мысль работает и по настоящее время,
объединил законы движения планет, их спутников и комет, приливы вод океана.
Явление приливов оказалось обусловленным именно притяжением Луны; сила
притяжения теперь снова появилась как орудие познания природы, будучи очищена
от той таинственной окраски, которую наложило на нее мистическое мышление
средних веков и которая так отталкивала Галилея. С этого момента стало
очевидным, что галилеево одновременное доказательство двойного движения Земли
ничтожно и что его теория приливов может в лучшем случае служить для пояснения
некоторых частностей явления, каким оно наблюдается на Земле. Небесная механика
Галилея вовсе не этап развития, а тупик, и небесная механика Ньютона ей
диаметрально противоположна». Но как и указывает Идельсон в примечании к той же
с. 79, и теория приливов Ньютона (статическая по своему характеру) не учитывает
инерции вод океана (начало динамической теории приливов положено Лапласом) и
«до сих пор неизвестно механическое явление, которым можно было бы доказать
одновременно оба движения Земли --- суточное и годичное. Для механического
доказательства первого из них служит маятник Фуко, второе обнаруживается только
в аберрации неподвижных звезд и в их годичном параллаксе. То, что не удалось
Галилею, не удалось и в последующие три столетия».

Чем же вызван такой бурный протест Галилея против Кеплера и других
предшественников Ньютона? Конечно, это результат того ожесточенного
сопротивления, которое он с молодых лет встречал у перипатетиков, сторонников
Аристотеля. Метод перипатетиков того времени, в значительной мере искаженный
толкователями Аристотеля, заключается в следующем (статья Д. Бобылева в
«Энциклопедическом словаре» Брокгауза и Эфрона, т. УПа (14), 1892, с. 896--898):
«Прежде всего, исходили из гипотез или положений, прямо почерпнутых из
сочинений Аристотеля и из них, путем силлогизмов, выводили заключение
относительно того, как должны происходить те или иные явления природы; к
проверке же этих заключений путем опыта не прибегали вовсе. Следуя такому пути,
перипатетики были, например, убеждены и учили других, что тело, весящее в
десять раз больше другого тела, падает в десять раз быстрее». В 1589 г. (25
лет) Галилей занял кафедру математики Пизанского университета и здесь открыл,
что скорость падения тела возрастает со временем и независима от веса тела. Это
вызвало неудовольствие перипатетиков, и те нашли повод к его удалению с кафедры
за неодобрительный отзыв, данный им относительно нелепого проекта какой-то
машины, поданного одним из побочных сыновей Козимо Медичи. Но по ходатайству
маркиза дель Монти он перешел на кафедру математики в Падую, где он пробыл 18
лет, с 1592 по 1610 гг. Так как перипатетики были его главными противниками
(они и выведены в «Диалогах» и «Беседах» под именем Симпличио), то такая
непрерывная борьба и окрасила эмоциональным элементом его рассуждения и привела
к возникновению ошибочных «убеждений
чувства». Но, борясь с перипатетиками, он отнюдь не отрицал всего Аристотеля.
По всей своей философской физиономии он был ярким выразителем того платонизма и
пифагореизма, которым характеризуется Возрождение. Будучи чрезвычайно
разносторонним и сделав много для усовершенствования стиля итальянской речи, он
читал, например, лекции и о Данте, и там, чтобы определить форму адской
воронки, он пользовался теоремами Архимеда и Альбрехта Дюрера, геометрическими
и архитектурно-механическими соображениями (Цейтлин. Галилей. 1935, с. 20). Он
был ревнителем механического понимания, но весьма далеким от материализма,
будучи почитателем Платона.

Галилей в «Диалоге», игнорируя эллипсы Кеплера, продолжает считать естественным
совершенное движение по кругу (Аристотель, сохранивший здесь платоновское
положение) и считает траекторией свободно падающего тела окружность, хотя им же
дано доказательство, что это парабола. В ранний период Галилеем была написана
работа «об ускоренном движении тела», вошедшая впоследствии в «Беседы о двух
новых науках», но в этих последних в гораздо более совершенной форме
доказывается о движении падающих тел по параболам (Галилей. Диалоги. Прим. А.
И. Долгова, с. 363).

То обстоятельство, что Галилей в своих последних «Беседах» (1638) значительно
усовершенствовал свои механические представления по сравнению с «Диалогами»
(1632), свидетельствует, что на старости лет, почти накануне смерти, он не
только изложил последовательно работу всей своей жизни (начиная примерно с
19-летнего возраста), но и продолжал творчески работать над своей теорией, уже
будучи 74-летним ослепшим стариком: редкий пример мыслителя, у которого вершина
творчества (акме) падает на последние годы его жизни. Успех «Бесед» заключается
и в том, что в них, оставив космические проблемы (которые еще не созрели для
механического истолкования), он обратился к чисто земным явлениям. «Все, что он
здесь говорит, безошибочно (в отношении механики. --- \emph{А.Л.}) и классично и по
праву вызвало ту восхищенную оценку, которую Лагранж дал этим открытиям
Галилея» (Идельсон, там же, с. 75). В «Диалогах» же Галилей, следуя своему
призванию механика и инженера, стал немедленно применять найденные им законы
движения твердых тел к мировой материи в целом.

Ошибочно судил Галилей и о кометах. Еще в своей блестящей работе «Пробирщик...»
(Салжиторе, 1623) он защищал неправильное положение, относя кометы к «подлунной
сфере», в противоположность Тихо де Браге, который, основываясь на исчезающе
малом параллаксе кометы 1572 г., пришел к совершенно правильному выводу о ее
большом расстоянии от Земли. (Галилей. Прим. А. И. Долгова, с. 360). Галилей
же, в данном случае следуя Аристотелю (который полагал, что только земному
свойственно возникновение и уничтожение и потому кометы не могут быть небесными
явлениями, а относятся к сфере Земли), считал, что кометы основаны на
испарениях от Земли вопреки фактам. И здесь экстраполяция твердых убеждений,
справедливых в определенной области, привела к игнорированию того самого
экспериментального метода, одним из блестящих основоположников которого он
являлся.

Мы видим, таким образом, что, по согласному мнению современных выдающихся
ученых, главная заслуга Галилея --- не его защита гелиоцентрической системы, а
создание основ механики. В знаменитом «Диалоге» много упрощений и ошибочных
мнений, в нем нет ни одной формулы, численные данные и результаты вычислений
даны редко и скупо. Все это, а также то, что она написана на итальянском, а не
на принятом в то время для ученых латинском языке создало широко
распространенное мнение, что это --- научно-популярная книга, написанная с целью
распространения коперникова учения. Галилей, конечно, стремился к широкому
распространению своих взглядов, будучи блестящим лектором и стилистом, но столь
же блестяще и тоже на итальянском языке написана его последняя книга «Беседы о
двух новых науках», посвященная целиком механике и сопротивлению материалов.
Его книги написаны, если так можно выразиться, «максимально понятно», принимая
во внимание трудность предмета, и для специалистов по астрономии и механике они
могут показаться популярными. Я лично читал их с удовольствием в хорошем
русском переводе, но никак не могу согласиться с мнением, что это --- популярные
книги. В «Диалоге» нет формул не потому, что они удалены (как это делается в
популярных книгах) для облегчения понимания, а потому что их вообще не
существует. Упростил Галилей теорию Птолемея и Коперника не для целей
популяризации, а для целей продвижения своих механических идей, но так как, как
указано выше, его механическая теория приливов была неверна, то никакой гений
не мог бы дать математическую теорию или (если бы он ввел какие-либо
дополнительные предположения) она оказалась бы совершенно невероятной
сложности. Упрощая структуру Солнечной системы, Галилей не погрешил против
научного метода, потому что всякий физический закон есть относительный закон.
Как пишет Дюгем (Физическая теория, ее цель и строение, 1910, глава V, § 3):
«Относительность закона не в том, что он верен для одного физика и не верен для
другого, а в том, что степень его приближения может быть достаточна для одного
применения и недостаточна для другого. Один и тот же физик один и тот же закон
то принимает, то отвергает в одной и той же работе: Реньо, например, при
исследовании сжимаемости газов старается заменить закон Мариотта формулой более
точной, но давление на высоте свободной поверхности в его манометре он
определяет по формуле Лапласа, в основе которой лежит закон Мариотта; здесь нет
противоречия, так как ошибка, внесенная в вычисления этим специальным
применением закона Мариотта, гораздо меньше, чем степень надежности
экспериментальных методов, которыми он пользуется». Погрешил Галилей как ученый
в том, что он пытался выдать за научный принцип чисто онтологический постулат
об универсальности механического истолкования явлений. Он стремился изгнать
всякие скрытые таинственные силы, которые принимал и Кеплер (Уайт, с. 152:
Кеплер принимал участие ангелов в планетных движениях), и из-за борьбы с ними
не принял участия в дальнейшем развитии гелиоцентрической теории. Ньютон
восстановил таинственные силы и сделал мощный шаг вперед, а многие
последователи Ньютона, подобно Лапласу, следуя политике страуса, сделали вид,
что таинственных сил они вообще не используют.

Этот краткий очерк показывает, что механика была лейтмотивом всей жизни Галилея
с 19-летнего возраста вплоть до его последних лет, когда он, уже ослепший,
диктовал страницы своего бессмертного последнего труда, и именно в последние
годы он довел до наилучшей проработанности основные свои идеи. Отклонения в
область астрономии хотя и сопровождались рядом блестящих успехов, но привели и
к ряду ошибок, связанных с неумеренной экстраполяцией. История Галилея поэтому
поучительна для всех наук в ряде отношений.

1) Недопустима экстраполяция положений и постулатов, оправданных в определенной
ограниченной области, на те области, где они недоступны опытной проверке. Этой
ошибкой грешат сейчас особенно в биологии, где ряд положений, оправданных в
области молекулярной биологии, генетики и проч., распространяется на область
макроэволюции при полном игнорировании вопиющих противоречий.

2) Развитие наук идет не путем накопления окончательно установленных истин, а
путем последовательных синтезов, причем при новом синтезе нередко возвращаются
к казалось бы окончательно отвергнутым положениям. Такое полное непонимание
духа истории науки показывают многие выдающиеся ученые, например Холден, не
понимающий антагонистичности взглядов Галилея и Кеплера и исключительности
значения Ньютона, объединившего в своем синтезе двух, казалось бы, непримиримых
ученых.

3) И из этого краткого очерка ясно, какие трудности представляли даже для
величайших умов крупные шаги в деле точной науки и как примитивен
вульгаризаторский подход писателей, именующих себя «прогрессивными», например
Бертольда Брехта, для которого все дело чрезвычайно просто: с одной стороны,
прогрессисты, с другой, мракобесы, не понимающие таких простых доводов, которые
способен понять самый простой, необразованный человек. Борьба геоцентрической и
гелиоцентрической систем вовсе не соответствовала разделу двух лагерей: 1)
прогрессивного, куда: материалисты, атеисты, новые религиозные движения,
выдвинутые Реформацией, и проч. 2) реакционного, куда: католическая церковь,
идеалисты, защитники феодализма и проч. Это всего лучше можно показать на ряде
сопоставлений, причем очень часто критическим пунктом является разногласие в
учении о приливах.

Френсис Бэкон считается основоположником материализма и в значительной степени
(тут сильно переоценен) индуктивного метода. Но он был противником теории
Коперника в некоторой степени потому, что он справедливо возражал против
ошибочной теории приливов Галилея. Противники Коперника охотно держались
старого астрологического толкования приливов и отливов действием Луны, так как
эта теория не предполагает вращения Земли (Дюгем, глава VII, § 2).

Гораздо позже, уже когда Ньютон реставрировал астрологический принцип
всемирного тяготения, выдающийся ученый Гюйгенс считал этот принцип абсурдным,
то же бы сказал и Декарт, а за ним и его последователи, картезианцы (Дюгем,
глава I, § IV). Примерно то же говорил и Лейбниц, и, как было показано выше,
сам Ньютон не отрицал абсурдности действия на расстоянии, но решительно
перешагнул через это, казалось бы, непреодолимое препятствие. Таким образом,
сопротивление Копернику, Галилею и другим строителям новой системы в
значительной (у многих ученых даже и решающей) степени было связано со вполне
рациональными доводами. Система Декарта связана с учением о вихрях, но законы
Кеплера нельзя было объяснить вихревым движением мировой материи (Паннекук, с.
268). Тот же Паннекук (там же) указывает, что восторженным почитателем Кеплера
(применившим его законы к объяснениям движения Луны) был юный английский
священник Иеремия Горроксю, рано умерший.

Протестантские церкви, подвергшие критике многие положения католичества, были
не менее решительны и в осуждении учения Коперника и продолжали в ряде случаев
это осуждение вплоть до второй половины XIX столетия (Уайт, с. 168).

Какое же отношение встретил Галилей среди своих соотечественников? Уже было
указано, что главными противниками его были догматические последователи
Аристотеля, перипатетики. Среди его сторонников сразу оказывается значительное
число представителей церкви: пизанский монах --- математик Кастелли,
неаполитанский богослов Фоскарини, знаменитый доминиканец Кампанелла, даже
генеральный проповедник Мараффи, Дини --- все они «игнорируя вопросы церковной
политики сегодняшнего дня, разрешают вопрос с точки зрения вечности» (М. Я.
Выгодский. Галилей и инквизиция. 1934, с. 127).

Астрономические открытия Галилея многими были встречены восторженно. Несмотря
на то что среди перипатетиков у него было много ожесточенных врагов и что
учение Аристотеля поддерживалось как неопровержимая истина католической
церковью, Галилей нашел себе сторонников в Риме среди высших лиц курии: таковы,
например, знаменитый богослов кардинал Беллармин и кардинал Маттео Барберини,
впоследствии папа Урбан VIII. В 1611 г. Галилея с триумфом встречают в Риме
ученые иезуиты из Коллегиум Романум; он находит восторженный прием при папском
дворе, становится другом князя Чези, основателя Академии ден Линчей
(Рысьеглазых) и сам делается членом этой Академии (С. И. Вавилов, статья в БСЭ,
2 изд., т. 10, 1952, с. 128). На запрос Беллармина, правильны ли наблюдения
Галилея, касающиеся скоплений неподвижных звезд, Сатурна, Венеры, Луны и
Юпитера, Римская коллегия за подписью четырех авторитетных членов Клавио,
Гринбергера, Малькотио и Лембо ответила утвердительно, хотя и с некоторыми
несущественными оговорками (предисловие А. И. Долгова к «Диалогам» Галилея, с.
9).

Галилей был в переписке с Маттео Барберини. Во время диспута о плавающих телах
Маттео Барберини принял сторону Галилея против перипатетиков, после сообщения о
солнечных пятнах принял его сторону против перипатетиков и иезуита Шейнера. В
1624 г., уже сделавшись папой Урбаном VIII, он подарил Галилею портрет, медали
и обещал стипендию сыну Винченцо. По некоторым данным, Барберини деятельно
защищал Галилея в 1616 г. и не вполне одобрял декрет от 5 марта, но, несмотря
на шесть бесед с Галилеем, не согласился отменить его в 1624 г.

Но в 1616 г. коперниково учение было официально осуждено, а к этому времени
Галилей уже опубликовал «Звездный Вестник» («Посланец от звездного мира»), где
содержалось изложение его астрономических наблюдений с неприкрытым одобрением
теории Коперника.
По приказанию папы Павла V Галилей был вызван во дворец кардинала Беллармина, и
здесь этот кардинал увещевал Галилея об ошибочности этого учения, а в протоколе
пленарного заседания конгрегации инквизиции в присутствии папы кардинал
Беллармин сообщил, что «математик Галилей, будучи предупрежден о приказании
инквизиции отойти от учения, которого он до сих пор придерживался, именно, что
Солнце есть центр сфер и неподвижно, а Земля движется, с этим согласился». На
том же заседании папа Павел V утвердил текст декрета, изданного 5 марта 1616
г., по которому книга Коперника была «задержана впредь до исправления», в то
время как написанное в «примиренческом» духе письмо Фоскарини было совершенно
«запрещено и осуждено». В перечне задержанных и осужденных книг ни «Посланец от
звездного мира», ни «Письма о солнечных пятнах» Галилея не значатся. Сам
Галилей сразу после опубликования декрета пишет, что вопрос идет только о
незначительных исправлениях книги Коперника, но что его, Галилея, враги
совершенно посрамлены. В письме от 12 марта 1616 г. Галилей описывает
милостивую аудиенцию, данную ему накануне папой Павлом V, и пишет: «Когда в
заключение я указал, что остаюсь в некотором беспокойстве, опасаясь возможности
постоянных преследований со стороны неумолимого коварства людей, папа утешил
меня словами, что я могу жить в спокойном настроении, так как обо мне у его
святейшества и у всей конгрегации остается такое мнение, что нелегко будет
прислушиваться к словам клеветников; так что пока он жив, я могу чувствовать
себя в безопасности» (Идельсон, 1964, с. 65).

Конечно, Галилей был недоволен, что учение Коперника все же было осуждено, но
это осуждение никогда не было безусловным. Требовалось лишь, чтобы это учение
выдавалось не за абсолютную истину, а за удобную математическую гипотезу. В
письме 1615 г. кардинал Беллармин пишет: «В действительности очень хорошо
поступает тот, кто говорит, что, предполагая Землю подвижной, а Солнце
неподвижным, мы гораздо лучше отдаем себе отчет во всех явлениях, чем это можно
было бы сделать при помощи эксцентрических кругов и эпициклов. Это не
представляет ни малейшей опасности и вполне достаточно для математики» (Дюгем,
глава III, § 2. См. также Идельсон, 1964, с. 62). И это подтверждается теми
несущественными изменениями, которые были внесены в сочинения Коперника,
переизданные в 1630 г., т. е. через четыре года после издания декрета.
Безусловное же осуждение сочинения Фоскарини было сделано потому, что там автор
стремился примирить учение Коперника и Священное Писание, а тем самым учение
Коперника возводилось в абсолютную истину (А. И. Долгов. Предисловие к
«Диалогам» Галилея. 1948, с. 11).

Естественно, что Галилей воспрянул духом, когда в 1623 г. кардинал Барберини
сделался папой Урбаном VIII. Полемическое сочинение Галилея «Пробирщик золота»
было выпущено в 1623 г. с посвящением Урбану VIII от имени Академии. Урбан VIII
даже писал стихи в честь Галилея. Галилей предпринял поездку в Рим в 1624 г. с
целью добиться смягчения декрета 1616 г., но, несмотря на ряд бесед с папой,
ему это не удалось. Галилей сообщал князю Чези: «Кардинал Целлер (Гогенцоллерн)
передал мне, что он имел беседу с его святейшеством относительно Коперника и
сказал папе, что, поскольку все еретики
(протестанты) придерживаются коперниканского учения, которое они считают
достовернейшим, то, принимая то или иное решение, здесь нужно действовать
весьма осмотрительно. На это папа ответил: "Св. церковь не осуждала и не
предполагает осуждать это учение как еретическое, но только как необоснованно
смелое; однако (добавил папа) не следует опасаться, что кому-нибудь удастся
доказать, что это учение есть необходимо истинное"». При отъезде Галилея из
Рима Урбан VIII снабдил Галилея так называемым бревэ на имя герцога
Тосканского, где Галилею расточались высокие похвалы: «Уже давно взираем мы с
отеческим благоволением на мужа, слава которого сияет в небесах и
распространяется по всей земле». Все это не помешало Урбану VIII через восемь
лет оказаться главным инициатором позорного процесса против Галилея, а после
осуждения его --- проявить себя мучителем великого старца до самой его смерти
(Идельсон, 1964, с. 67, примеч). Тот же Идельсон на с. 171 указывает, что
Галилей «слишком доверчиво отнесся к сообщению его любимого ученика Кастелли в
письме от 16 марта 1630 г. о том, что папа Урбан VIII сказал: "Это (запрещение
Коперника) никогда не было нашим намерением, и если бы зависело от нас, то
декрет не был бы издан"».

В изложении современных историков науки, компетентных в астрономии, мы видим, с
одной стороны, что никаких признаков неблаговоления со стороны пап (Павла V и
Урбана VIII) по отношению к Галилею не было вплоть до знаменитого процесса,
мало того, Галилей имел все основания считать Урбана VIII своим ценителем и
покровителем --- и вдруг такая перемена. Урбан превращается внезапно в гонителя и
мучителя, и это все преподается без малейшей попытки объяснения такой
удивительной перемены. Астроном изложил с точки зрения астрономии, а от
суждения вне пределов своей специальности воздержался. Я разберу этот вопрос
позже в связи с процессом Галилея, а пока остановлюсь еще на той категории
противников Галилея, каковым приписываются интриги против великого ученого,
именно на иезуитах. Этот вопрос хорошо разобран между прочим в книге Цейтлина,
который пишет (с. 89): «Здесь же заметим, что традиционное изображение
иезуитского и католического отношения к науке как всеобщего, голого и грубого
обскурантизма является в корне ложным». Легенда о том, что главными
преследователями Галилея были иезуиты, не является необоснованной. Сам Галилей
в письме к Дислати от 25 июня 1634 г. выражается следующим образом: «Вы видите,
значит, что я пострадал не за защиту того или иного взгляда, а потому что впал
в немилость у иезуитов» (Цейтлин, с. 131). Это же подтверждает и математик
коллегии иезуитов Гринбергер, который в беседе с одним из друзей Галилея в 1634
г. (т. е. уже после процесса Галилея) сказал: «Если бы Галилей сумел сохранить
расположение к себе со стороны этой коллегии, то жил бы он в славе на свете и
не случилось бы с ним ни одного из его несчастий, и он мог бы писать по желанию
о любом предмете, в том числе и о движении Земли, и т. д.» (Идельсон, 1964, с.
60). Понятно, что и знаменитейший и яростнейший противник иезуитов, великий
Паскаль имел право указывать на иезуитов как на главных виновников осуждения
Галилея. Но из вышеприведенных слов ясно, что Галилей «впал в немилость», «не
сумел сохранить расположения», так как раньше, это ясно и из первого приема
коллегией иезуитов в Риме, он пользовался милостью иезуитов. И верно, что он
нажил несколько крупных врагов среди иезуитов.

Такими были Грасси и Шейнер. Но эти противники держались очень осторожно, пока
генералом ордена иезуитов был Клавдий Аквавива, последний умер в 1616 г., и
тогда выступления противников Галилея стали смелее, так как на место Аквавивы
стал слабохарактерный «ангел мира» Муццион Виттелески (Цейтлин, с. 129). Ясно,
что Аквавива защищал Галилея от нападок, а Виттелески лишь не мешал нападать.
Но «Пробирщик» (Салжиторе) Галилея направлен в значительной степени против того
взгляда Грасси, что кометы принадлежат к сфере неба (что, в частности,
противоречило мнению Аристотеля). Здесь Галилей был не прав, а по свидетельству
Идельсона (с. 55 и 67), растянутая, почти непрерывная полемика действует на
читателя несколько утомительно. Значит, как это ни странно, в данном случае
иезуит защищал антиперипатетические взгляды, несмотря на поддержку Аристотеля
церковью.

Другого врага среди иезуитов, Шейнера, Галилей тоже нажил по поводу приоритета
открытия солнечных пятен, что вызвало ожесточенные споры. И здесь Галилей был
бесспорно прав в отношении объяснения солнечных пятен, но Шейнер несколько
опередил его в деле наблюдения, хотя обоих их опередил астроном Иоганн
Фабрициус, о котором не знал ни Галилей, ни Шейнер, но о котором знал Кеплер
(Идельсон, 1964, с. 60). После периода корректной переписки между Галилеем и
Шейнером возникла полемика. Галилей заострил ее в памфлете (Салжиторе), Шейнер
резко обрушился на Галилея. Галилей писал о Шейнере в 1636 г. в письме: «Этот
поросенок и лукавый осленок составляет теперь перечень того, что я не знал в
свое время...» (там же).

Но и по поводу объяснения солнечных пятен Галилей не избежал ошибок,
что хорошо показано в той же статье Идельсона (с. 53--59). В первых
статьях о солнечных пятнах он дал правильное объяснение своеобразному
и изменчивому для земного наблюдателя виду траекторий пятен по диску
Солнца, выяснив, что это есть следствие наклона плоскости солнечного
экватора к плоскости эклиптики под небольшим углом. Все это относится
к самым мастерским и блестящим моментам творчества Галилея. Но
позже, в эпоху создания «Диалога» (1630), Галилей считал возможным
извлечь из своего открытия нечто большее и здесь его рассуждение (на
этот раз ошибочное) приводит к тому, что Солнцу пришлось бы дать
«третье» движение. Как пишет Идельсон, эта ошибка Галилея тем более
удивительна, что он сам доказал ненужность «третьего» движения Земли в
системе Коперника. Тот же Идельсон замечает: от этого спора падает
тяжелая тень, омрачающая последние годы жизни Галилея. Мы видим, таким
образом, что в споре с Шейнером Галилей не был прав на сто процентов.
Но в этом споре примешивалась и политическая сторона. Шейнер был одним
из самых влиятельных и деятельных агентов испано-немецкой партии, как
и лидер испанской оппозиции Урбану VIII, кардинал Гаспар Борджиа, о
чем еще будет речь (Цейтлин, с. 131). Само собой разумеется, что
резкость полемики Галилея и ошибки, им сделанные, и в которых он
не хотел признаваться, привлекли на сторону Грасси и Шейнера и других членов
общества Иисуса.

Был и второстепенный объект спора: язык. Галилей писал свои работы на
итальянском языке. Это новшество вызвало возражения с разных сторон.

Как указано в книге Цейтлина (с. 37), историк Ольшки делает мелкое замечание,
что можно было искать у духовенства покровительства для самых смелых мыслей,
если они были выражены на латинском языке. С другой стороны, Кеплер жаловался
на это оскорбление человечества (и тут два гения разошлись во мнениях), а
столетие спустя не менее гениальный Лейбниц выступил в защиту латыни, так как
путаница, вызванная употреблением народных языков в науке, заставила его
почувствовать необходимость восстановления мирового языка науки или замены его
искусственным. Как своевременны эти мысли Лейбница, в особенности в наше время!

О восторженном приеме астрономических открытий Галилея иезуитами Римской
коллегии я уже писал, но, может быть, после процесса иезуиты, как и другие
ревностные католики, препятствовали развитию идей Галилея. Вредное влияние
католицизма на развитие науки утверждает, например, Идельсон, в той же статье,
с. 66: «...декрет 5 марта 1616 г. --- это удар отнюдь не по одному Галилею (на
предыдущей странице тот же Идельсон пишет, что в перечне осужденных книг
сочинения Галилея не числятся и сам Галилей был принят в милостивой аудиенции
папой Павлом V); это суровое испытание для науки и культуры в странах
католицизма, где развитие новой астрономии приостанавливается приблизительно на
200 лет; где оно искусственно и умышленно задерживается, чтобы дать некоторое
время безраздельно господствовать над умами представителям отживающих
мировоззрений». В примечании Идельсон указывает, что декрет 5/III 1616 г. был
впервые «опущен» в 1757 г. при составлении нового кодекса запрещенных книг при
папе Бенедикте XIV (при котором, см. Уайт, с. 155, негласно уже допускалось
учение Коперника. --- \emph{А.Л.}), и только в индексе издания 1835 г. не встречаются
имена Коперника, Дилана Астуника, Фоскарини, Галилея и Кеплера. Несколько
раньше, в 1822 г., декретом папы Пия VII публикация работ с изложением
гелиоцентрической теории была разрешена в Риме (Уайт, с. 156).

Слова Идельсона об искусственной приостановке католицизмом развития новой
астрономии на 200 лет совершенно не соответствуют истине. Совершенно иначе
пишет известный современный марксистский историк науки, крупный физик Дж.
Бернал в своей книге («Наука и история общества», 1956, с. 228): «Даже движение
контрреформации, которому удалось пресечь и повернуть вспять развитие
протестантизма в Европе, не оказало подобного влияния на науку. Руководившие
этим движением иезуиты (а им принадлежит главная роль в том, что католичество
оказалось полностью восстановленным в Польше, Венгрии, Австрии, Бельгии,
Баварии и Рейнской области, где протестантизм, казалось бы, одержал полную
победу --- см. Британск. энциклопедия, 1957, т. 13, с. 9--15) были достаточно
умны, чтобы понимать, что им легче будет покорить души, поощряя науку, а не
слепо противодействуя ей. В соответствии с этим они полностью включились в
научное движение, в частности в новую астрономию, и даже содействовали ее
распространению и созданию обсерваторий в Индии, Китае и Японии. В то же время
иезуиты действовали как сторожевые псы внутри науки, призванные ограждать
истинную религию от всевозможного вредного влияния со стороны этого движения и
тем самым они, сами того не желая, поставили деятелей науки в протестантских
странах, находившихся вне сферы их контроля, в более выгодное положение». Вряд
ли и это утверждение соответствует действительности. Уайт в неоднократно
цитированной книге пишет (с. 155): «В Германии, в особенности в протестантской
ее части, война (с коперниковым учением. --- \emph{А.Л.}) была еще более ожесточенной
(чем в католических странах) и длилась в течение первой половины XVIII
столетия. Выдающиеся лютеранские доктора богословия наводнили страну
трактатами, доказывающими, что доктрина Коперника не может быть согласована с
Писанием. В богословских семинариях и во многих университетах, где было сильно
клерикальное влияние, богословы, казалось бы, все сметали на пути; и однако в
середине столетия некоторые из наиболее ясномыслящих поняли, что их дело
проиграно». Таким ясномыслящим Уайт считает папу Бенедикта XIV, про которого
уже говорилось. Книга Уайта вышла первым изданием в 1896 г. и в предисловии,
написанном им в С.-Петербурге в 1895 г., Уайт жалуется на то засилье
духовенства, которое имело место в США и Англии еще во второй половине XIX в.,
и то влияние, которое играла принадлежность к тому или иному религиозному
направлению в деле назначения профессоров и которое стало исчезать именно в это
время. Вспомним, что и знаменитый «обезьяний процесс» в наше время имел место в
протестантской Америке, а не в католической стране.

И вот мы видим, что отношение католичества вообще, а иезуитов в частности к
теории Коперника строго соответствовало первоначальной позиции кардинала
иезуита Беллармина: развивать ее как полезную для описания явлений
математическую гипотезу, но не придавать ей абсолютного значения.

Так выразился и знаменитый иезуит Боскович, автор первой математически развитой
атомной теории (пифагорейского, а не демокритовского характера): «Что касается
до меня, то полностью уважая Священное Писание и декрет Священной Инквизиции, я
считаю Землю неподвижной; однако, для простоты объяснения, я рассуждаю, как
будто бы Земля была подвижной, так как доказано, что из двух гипотез последняя
была обоснована явлениями» (Уайт, с. 155).

В 1672 г. иезуит Риччиоли разобрал все доводы за и против коперниковой теории и
указал, что имеется 49 доводов в пользу Коперника и 77 против. Большинство
доводов, таким образом, было против Коперника (Уайт, с. 154), но мы знаем, что
в науке большинством голосов вопросы не решаются, и всякий внимательный
читатель мог убедиться в сравнительной силе доводов.

Во Франции иезуиты Лессер и Жакье издали в 1739--1742 гг. французский перевод
«Начал» Ньютона, т. е. популяризировали гелиоцентрическую систему в том
совершенном виде, который ей придал Ньютон, но в предисловии делают требуемую
католической церковью оговорку (Цейтлин, с. 261--262). Эти же два иезуита
выпустили трехтомное латинское издание «Начал» в 1760 г., снабдив его
подробными примечаниями (см. А. Крылов, предисловие к переводу книги: И.
Ньютон. Математические начала натуральной философии, 1916, с. VIII). Иезуиты
потратили много труда, чтобы способствовать распространению гелиоцентрической
теории в то время, когда книги Коперника и Галилея числились в списке
запрещенных книг, а автор «Начал» Ньютон был ожесточенным антипапистом и в
своей книге «Толкований пророка Даниила и Апокалипсиса» рассматривал папу как
апокалипсического зверя.

Иезуиты дали много выдающихся ученых и как раз в особенности в астрономии
(Лалинд насчитывает 42) и геофизике. Особенно крупным был (уже в XIX в.) Секки
(1818--1878), один из основоположников спектрального анализа и давший
классификацию звезд. Энгельс неоднократно цитирует Секки и так его
характеризует (Диалектика природы, 1949, с. 158): «Патер Секки хотя и воздает
ему (Богу) всякие канонические повести, тем не менее весьма категорически
выпроваживает его из Солнечной системы, разрешая ему творческий акт только в
отношении первоначальной туманности». И Секки не только работал успешно в
области астрономии, но и популяризовал гелиоцентрическую систему, поставив опыт
Фуко в Риме в 1852 г. в одной из церквей (Уайт, с. 157). Наши воинствующие
безбожники, считая себя страшно революционными, полагают, что, поставив опыт
Фуко в одном из соборов С.-Петербурга, они проводят радикальную атеистическую
пропаганду, а на самом деле они плетутся в хвосте у ученого иезуита,
поставившего этот опыт в середине XIX в. в самом центре католического мира.

Хорошо известно, что руководителями инквизиции были доминиканцы, а не иезуиты.
Конечно, и среди доминиканцев было много выдающихся умов, и среди них были
защитники коперникова учения, например знаменитый Кампанелла (не говоря о
Бруно), но по целому ряду философских, богословских и политических вопросов у
них было значительное расхождение. Характерно для соотношения сил, что в
инквизиционной конгрегации, подготовившей постановление 1616 г., было из 11
членов шесть доминиканцев, один иезуит, один августинец, один бенедиктинец и
два без обозначения принадлежности к ордену (Цейтлин, с. 123).

\textbf{Процесс Галилея и его значение.} Но если среди иезуитов были отдельные враги, а
не весь их орден, и если папа Урбан VIII во многом сочувствовал Галилею, чем же
объясняется громкий процесс 1632--1633 гг., после которого Галилей прожил еще
примерно девять лет?

Если возьмем такого добросовестного историка науки, как Уайт, неоднократно
цитированного, то там (с. 143) мы найдем такое красочное изображение последних
лет великого ученого. «До конца своей жизни --- нет, даже после окончания жизни,
преследование Галилея продолжалось. Он содержался в изоляции от своей семьи, от
своих друзей, от своих благородных занятий, и его заставляли строго держать
свое обещание не говорить о его теории. Когда, перенося интенсивные телесные
страдания от болезни и душевные муки от страданий в его семье, он просил
немного свободы, он был встречен угрозами заключения в темницу. Когда, наконец,
специальная комиссия доложила
духовным властям, что он ослеп и изнурен болезнью и печалью, ему дали несколько
больше свободы, но это немногое было стеснено строгим надзором. Он принужден
был в молчании выслушивать презрительные нападки на него и его дело; видеть,
что дружественно расположенные к нему люди были серьезно наказаны; патер
Кастелли изгнан; Риччиради, церемониймейстер Священного Двора (мастер оф те
Сакрел Палас), и Чиамполи, папский секретарь, уволены с постов папой Урбаном, а
инквизитор Флоренции получил выговор за то, что дал разрешение печатать труд
Галилея. Он дожил до того, что истины, им установленные, тщательно изгонялись
из всех церковных колледжей и университетов Европы, и когда в каком-то ученом
сочинении о нем упомянули как о "знаменитом" ("реноунд"), инквизиция заставила
заменить это словом "известный" ("ноториос")». "Такое жалкое состояние Галилея
в его последние годы описывает и известный писатель Бертольд Брехт в своей
драме «Галилей»: в нужде, тайно пишущий свои сочинения, окруженный шпионами и
проч.

Какие основания были для такой мрачной картины? Во-первых, письма самого
великого ученого, который нередко помечал их «из моей тюрьмы в Арчетри» (см.
Идельсон, 1964, с. 60). Во-вторых, официально опубликованный приговор, где
упоминалось и об угрозе пыткой, и о тюремном заключении, и о воспрещении
пропаганды своего учения (Цейтлин, с. 217). Но по приговору кроме тюремного
заключения на Галилея было наложено спасительное покаяние: «в течение трех лет
прочитывать один раз в неделю семь покаянных псалмов, оставляя за собой право
вышеназванные санкции и покаяния уменьшить, изменить полностью или отчасти» (с.
225). Что касается гонения на Галилея после его смерти, то под этим, очевидно,
подразумевается то, что папа Урбан не разрешил похоронить Галилея там, где тот
хотел быть погребенным. Вот те фактические данные, на основе которых создалась
широко распространенная легенда о последних годах жизни Галилея, активно
использованная для антирелигиозной и особенно антикатолической пропаганды. Но
дают ли сообщенные факты полную истину? Посмотрим.

Для человека, вся жизнь которого была посвящена науке, самым ужасным, является,
конечно, замалчивание результатов его работ. И однако мы видим, что как раз в
«заключении» в 1638 г. Галилей написал и опубликовал свое величайшее
произведение «Беседы о двух новых науках». Приведу слова Лагранжа из его
«Аналитической механики», 1811 (предисловие А. И. Долгова к книге Галилея
«Беседы», 1934, с. 9): «Эта наука (динамика) сполна создана в последнее время,
причем первые основы ее были заложены Галилеем... Это открытие составляет
теперь наиболее значительную и непререкаемую часть заслуг этого великого
человека. В самом деле, чтобы открыть спутников Юпитера, фазы Венеры, солнечные
пятна и т. д., требуется только телескоп и наблюдательность, но нужен
исключительный гений, чтобы установить законы природы на явлениях, которые
всегда были у всех перед глазами и тем не менее ускользали от внимания
философов». Как указывает Долгов далее, Галилей в этой книге заложил также
основы гидростатики, акустики и сопротивления материалов.

Но как же удалось Галилею преодолеть бдительность окружавших его шпионов (если
верить Б. Брехту) и опубликовать книгу? Сам Галилей пишет об этом в посвящении
к своей книге графу де Ноайль (французскому послу) (Галилей, с. 33). «Считаю
актом благодеяния с вашей стороны, досточтимый синьор, что вы соблаговолили
распорядиться моим настоящим сочинением, хотя я, как вам известно, смущенный и
напуганный несчастной судьбой других своих сочинений, принял решение не
выпускать более публично своих трудов и, чтобы не оставлять их вовсе под
спудом, сохранять лишь рукописные копии таковых в месте, доступном, по крайней
мере, для лиц, достаточно знакомых с трактуемыми мной предметами». Поэтому
Галилей и передал рукопись графу, который обещал хранить ее и ознакомить
некоторых друзей из Франции, «показав тем, что хотя я и молчу, но провожу жизнь
не совсем праздно».

Галилей хотел приступить к изготовлению других копий для Германии, Фландрии,
Англии, Испании и некоторых мест Италии, как вдруг неожиданно был извещен
известной фирмой «Эльзевир», что произведения Галилея готовы к печатанию и
желательно кому-либо их посвятить. Галилей был радостно взволнован,
естественно, посвятил книгу графу де Ноайль, так великолепно распорядившемуся
переданной ему рукописью. Он пишет: «сделать это побуждает меня не только
сознание всего того, чем я вам обязан, но и готовность ваша, да позволено мне
будет так выразиться, защищать репутацию ото всех желающих запятнать ее. Вы
опять воодушевили меня на борьбу с противниками». Опубликование нового
сочинения Галилея, признание им самим, что он передал ее иностранцу и что он не
препятствует напечатанию и готов снова бороться с противником, не повлекло
никаких репрессий для Галилея. Но из его посвящения мы видим, что он не был
строго изолирован. Эта «изоляция» относится тоже к чисто легендарной истории.
Об этой «изоляции» Цейтлин пишет, с. 249: «Многие лица посещали Галилея, не
спрашивая инквизиции. Так, Галилея посетили знаменитый английский писатель
Мильтон, философ Гоббс и, возможно, Декарт. Фердинанд II Медичи, его брат
Леопольд и другие члены дома Медичи неоднократно бывали у Галилея. В 1636 г.
некоторое время у Галилея жил его ученик Кавальери (настоятель католического
монастыря ордена иеронимитов, по существу введший понятие определенного
интеграла (Рыбников. История математики. I, 1960, с. 158. --- \emph{А.Л.}), в 1637 г. ---
математик Пери. Постоянными помощниками Галилея были три члена ордена:
Микелини, патер Клеменс и патер Амброзиус (Брехт изображает живущих с Галилеем
монахов сыщиками: плохо же они работали! --- \emph{А.Л.}). С 1640 г. у Галилея жили:
временно --- Геньери, а до самой смерти --- Нивиани и знаменитый физик Торичелли,
ученик Кастелли (ученика Галилея. --- \emph{А.Л.})».

Несмотря на существование декрета инквизиции, запрещавшего печатать какие бы то
ни было сочинения Галилея, ему удалось напечатать за границей кроме «Бесед»
латинский перевод «Диалога» (та же фирма «Эльзевир» в 1635 г.) с приложением
письма Галилея к Христине Лотарингской.

Но ведь Галилей был приговорен к тюремному заключению и сам писал «из тюрьмы»?
В настоящей тюрьме Галилей не провел ни одного
дня. По прибытии в Рим 13 февраля 1633 г. Галилей прожил два месяца у
тосканского посланника Никколини и лишь две недели провел в здании
инквизиционного трибунала, но не в тюрьме, а на квартире фискала --- прокурора
инквизиции Синяере, а потом снова у Никколини. В конце июня Галилея направили в
Сиену под надзор его друга и ученика, архиепископа сиенского Асканис
Никколомини (в механике он сделал первый шаг, отрицая мнение Аристотеля, что
тяжелые и легкие тела падают с разной скоростью (Долгов, с. 31)). Французский
поэт Сен-Аман, посетивший осенью Галилея, описывает богато меблированное и
обитое шелком жилище ученого, полное книг и рукописей, где Галилей приступил к
работе над давно задуманным сочинением «Беседы» (Долгов, с. 15--16; Цейтлин, с.
235). Ясно, что Галилей не терял времени и в самый год знаменитого процесса
приступил к своей величайшей работе. Но уже в декабре 1633 г. Галилею разрешили
поселиться на собственной вилле в Арчетри, в миле от Флоренции, под названием
«Драгоценность». Эту-то «Драгоценность» Галилей в письмах и называет «тюрьмой».
А на что же жил в «тюрьме» Галилей?

У Цейтлина читаем (с. 94): пенсия Урбана VIII, первоначально предназначенная
для сына Галилея (он от нее отказался, так как условием было выполнение
некоторых монашеских обязанностей), была передана Галилею, и он ее продолжал
получать и после своего осуждения.

А как с разлукой с семьей? Галилей вовсе не был женат, и законной семьи у него
не было. Он не был аскетом, а был сыном своего времени и эпохи Возрождения, у
него была длительная связь с венецианкой Маритой Гамба (с 1599 г. лет десять),
от которой у него был сын Винченцо и две дочери, Виргиния и Ливия, сделавшиеся
потом монахинями. Многие биографы считают, что одной из причин переезда во
Флоренцию было желание порвать эту незаконную связь. Он не обидел Гамбу, и она
при его денежной поддержке благополучно вышла замуж за Бартодуцци (Цейтлин, с.
94--95). Сам Галилей указывает в письмах, что он покинул Падую потому, что нравы
и обычаи республики не дают условий, благоприятных для научной работы. «Эти
желаемые условия я не могу получить ни от кого, кроме абсолютного князя», ---
писал Галилей. Вопреки изображению Брехта, Галилей изысканно одевался, любил
общество и был большим знатоком вин (Цейтлин, с. 22). Скончался Галилей 78 лет
8 января 1642 г. в присутствии сына Винченцо с женой, учеников Вивиани и
Торичелли и перед смертью он получил генеральное отпущение --- причащение и
благословение от самого папы Урбана VIII.

Галилей не мог не понимать, что он оставляет талантливых и преданных учеников и
последователей его идей: достаточно назвать три таких славных имени, как
Торичелли, Кавальери и Норелли, а своему биографу, Вивиани, он уже слепой
диктовал свои труды.

Получается любопытный факт, что официальное описание процесса гораздо строже
того, что фактически имело место, и это расхождение ни у кого не вызывает
сейчас сомнения. Вот как пишет Дж. Бернал в книге «Наука в истории общества»,
ИЛ, 1956, с. 236--237: «Такие горячие приверженцы Коперника, как Бруно и
Кампанелла (1568--1639), уже сделали из нового знания выводы, угрожавшие устоям
церкви, правительства, общественной морали и самой собственности (с. 183).
Бруно был сожжен на костре, Кампанелла заключен в тюрьму на долгие годы; однако
с Галилеем дело обстояло иначе: у него был большой научный авторитет и
влиятельные друзья, его католицизм не подвергался сомнению, и, кроме как в
науке, он вовсе не был революционером. Судебный процесс, как и следовало
ожидать, велся в рамках представлений и образа мышления церкви, а не Галилея, и
потому результат его был предопределен. Однако интересен тот факт, что
протоколы суда держались в секрете, по всей вероятности потому, что опасались,
как бы их обнародование не разоблачило не суровость, а относительную
снисходительность судей. Папа и его курия больше всего боялись возможной
реакции со стороны твердолобых фанатиков церкви, чем со стороны ученых. Галилей
был осужден и вынужден сделать свое знаменитое отречение, однако он подвергся
только условному заключению во дворце одного из своих друзей. Находясь в
уединении, он смог закончить свой труд о динамике и статике и опубликовать его
в последние годы своей жизни. Судебный процесс над Галилеем ознаменовал собой
целую эпоху, ибо драматизировал конфликт между наукой и религиозной догмой.
Своим фактическим провалом, ибо приговор был весьма отрицательно принят почти
всем ученым миром, даже в католических странах, процесс этот неизмеримо поднял
престиж новой революционной экспериментальной науки, особенно в тех странах,
которые уже свергли у себя власть римской церкви. Достижение Галилея выглядит
как высшая точка наступления на старую космологию. С этого момента от нее
молчаливо отказались, и астрономы-практики стали пользоваться созданной
Коперником и Кеплером теорией Солнечной системы. Сорок лет спустя законы,
выведенные Кеплером путем наблюдений, были объединены с открытыми Галилеем
законами динамики в ньютоновской теории всемирного тяготения».

Из этой цитаты ясно, во-первых, что приговор был настолько снисходителен, что
папа его не опубликовал из нежелания раздражать фанатиков церкви. Ясно также,
что процесс не только не затормозил развития науки, но вызвал еще больший
интерес к новой астрономии и тем самым стимулировал ее развитие, но Бернал
все-таки думает, что основной причиной процесса была ортодоксальность папской
курии, а фактическая снисходительность --- результат влияния друзей и проч. Такой
результат особенно ясно показывает, что процесс не мог быть следствием личной
неприязни Урбана VIII к Галилею, основанной якобы на том, что под видом
Симпличио (простака) Галилей вывел самого папу. Известно, что сам папа отвергал
эту глупую версию, так как она не имеет ни малейшего правдоподобия. Симпличио
вовсе не выведен дураком или простаком, а местами дает своим противникам весьма
умные возражения. Слово Симпличио не выдумано Галилеем, а есть фамилия
знаменитого комментатора Аристотеля, о чем пишет сам Галилей (он, следуя
Платону, вводит в свои диалоги реальных лиц, а не выдуманных). (Кстати, слово
«простак» по-итальянски симпличино, а не симпличио). И наконец, Симпличио вновь
появляется в «Беседах», написанных Галилеем в «тюрьме». Со стороны старого,
пережившего тревоги Галилея было бы неслыханным, чисто мальчишеским озорством
снова использовать это имя, если бы Галилей имел малейшее основание думать, что
папа был обижен этим именем. Ведь Урбан VIII пережил Галилея и послал ему перед
смертью папское благословение. Но говорят, Урбан преследовал Галилея и после
смерти, не позволив родственникам и ученикам Галилея похоронить его согласно
его желанию. Конечно, возмущение такой «жестокостью» кажется несколько странным
в устах наших казенных писак, которые стараются если не оправдать, то по
крайней мере замолчать судьбу сотен тысяч людей, не только не похороненных
согласно их желанию, но вообще похороненных неизвестно где в нашем «единственно
прогрессивном» государстве. Ясно, что поведение Урбана по отношению к Галилею
заключалось в том, чтобы проявить якобы наибольшую строгость к строптивому
ученому, а на деле, поступить максимально снисходительно. Разгадка заключается
в том, что процесс Галилея был вовсе не научным спором, а чисто политическим.
Это ясно вытекает из многих сводок, опубликованных в советской литературе, в
особенности ясно это изложено в неоднократно цитированной обстоятельной книге
Цейтлина (1935). Правда, этот вывод сразу не бросается в глаза при
поверхностном просмотре введения, оглавления и иллюстраций книги, но содержание
всей книги не оставляет никакого сомнения, хотя, конечно, оно становится
совершенно ясным лишь при внимательном чтении.

У нас часто говорят о необходимости рассматривать развитие науки и вообще
культуры в связи с общественными явлениями, но по принятой схеме стараются
отыскать классовую подоплеку тех или иных явлений. Сомневаюсь, чтобы такой
подход был удачен в деле Галилея, но классовый подход далеко не единственный из
возможных, и в деле Галилея сыграли большую роль национальный и
общественно-политический моменты. Дело ведь происходило в самый разгар
трагической Тридцатилетней войны (1618--1648), последней грандиозной попытки
вооруженной рукой покорить протестантизм. Она закончилась, как известно,
Вестфальским миром, после которого в первом приближении установилось по крайней
мере в большинстве стран мирное сосуществование разных религиозных идеологий ---
начало полной религиозной свободы, просуществовавшей до XX в. Но кроме этого
основного антагонизма существовали два других, не менее, а даже более
могущественных. Во-первых, национальный: испанский и немецкий (возглавляемый
Австрийской империей, бывшей тогда гегемоном немецкого мира), с одной стороны,
и итало-французский --- с другой. Во-вторых, социально-политический:
независимость церкви от государства и более или менее полное подчинение церкви
государству, в крайнем своем выражении делающееся цезаропапизмом, т. е. когда
глава государства является одновременно и главой церкви. А так как «принципиум
дивизионис» был не один, а по крайней мере три, то тут не было двух лагерей, а
была комбинация принципов трех разных антитез. Папа римский, как глава
католической церкви, естественно, должен был быть на стороне католических
армий. Но будучи главой католической церкви, он одновременно был решительным
противником цезаропапизма. По известной доктрине «двух мечей» (духовного и
светского) папа римский объединял и светскую и духовную власть в
Папской области, но с гегемонией духовной власти. При цезаропапизме было
наоборот: при объединенных в одних руках «двух мечах» гегемония была на стороне
светской власти. В начале Тридцатилетней войны католические армии шли от успеха
к успеху и дело протестантизма казалось проигранным по крайней мере в немецких
государствах, пока вмешательство чемпиона протестантства, шведского короля
Густава Адольфа (которого военные историки считают одним из величайших
полководцев всех времен и народов), не внесло крутого изменения в войну. Гибель
Густава Адольфа при Лгоцене в 1632 г. остановила его блестящие успехи, но его
преемники, шведские полководцы, сохраняли перевес шведов в войне до самого ее
конца. Урбан VIII был не первым папой, понявшим, что гегемония стран
католического абсолютизма может быть вредной для католической церкви. Еще Сикст
V понял необходимость отказа от традиционного союза папства с Испанией.
Посланный им во Францию известный кардинал иезуит Беллармин отклонил
домогательства лигистов (Священная испано-французская католическая лига борьбы
против французского короля Генриха IV, долгое время бывшего вождем гугенотов)
воздействовать на папу в их пользу. Преемник Сикста V Григорий XIV решил стать
на сторону лигистов, и тогда Беллармин демонстративно оставил Париж, за что был
вознагражден саном кардинала новым папой Климентом VIII.

Но Густава Адольфа субсидировал не только папа, но и другой видный деятель
католического мира, известный кардинал Ришелье, фактический диктатор Франции
при Людовике XIII. В своем государстве он решительно боролся с гугенотами (хотя
по дикости приемов борьбы его сильно перещеголял Людовик XIV), и он же
заключил, наряду с некоторыми другими странами (например, Венецией), союз с
Густавом Адольфом в 1631 г. при посредничестве папского нунция. Это было
сделано из государственных соображений для ослабления конкурентов --- Испании и
Австрии, полная победа которых привела бы к резкому нарушению европейского
равновесия.

Поддержка Густава Адольфа длилась недолго: в 1632 г. шведский король погиб в
бою, Ришелье был серьезно болен в 1633 и 1634 гг. Но внутри самой Австрии было
неспокойно, и главнокомандующий католической армией Валленштейн был убит в 1634
г. по приказу австрийского императора, так как его подозревали в чрезмерном
честолюбии и сношениях с Францией. Под давлением испано-немецкой партии папа
был вынужден давать субсидии католическим монархам, в чем он ранее отказывал.
Но дело защиты протестантизма было сделано, опасность полной победы
католических армий уже была исключена. По словам Григоровиуса, Урбан VIII был
последним значительным политическим деятелем на папском престоле, пытавшимся
вести независимую политику. После его смерти в 1644 г. ясно выявилась победа
испано-немецкой оппозиции и папой был выбран кардинал Памфили под именем
Иннокентия X.

Но Урбан VIII был не только главой католической церкви, но и итальянцем, и это
обостряло его неприязнь к Испании. Известно, что значительная часть Италии,
имевшая общее название «Королевство обеих Сицилий» (куда входили южная часть
полуострова южнее Папской области со столицей Неаполем и вся Сицилия), долгое
время находилась под властью Испании. Несмотря на родство испанцев и итальянцев
и религиозное единство, испанское владычество над южной Италией было
несравненно более тягостным, чем, скажем, господство Польши над Украиной.
Всякий, кто бывал в южной Италии (я там прожил три месяца в 1909 г.), а потом
переезжает в северную, поражается контрасту этих двух частей современной Италии
(как будто переезжаешь из Африки в Европу). Само собой разумеется, что
передовые люди стремились к освобождению южной Италии от испанского гнета.
Среди наиболее выдающихся упоминаем знаменитого доминиканца Кампанеллу, который
был автором картины утопического социализма «Государство солнца» и вместе с тем
решительным сторонником Коперника. Урбан VIII помог Кампанелле бежать из
неаполитанской тюрьмы во Францию, где Кампанелла и жил остаток жизни под
покровительством Ришелье.

А какую сторону в этих политических спорах принимали иезуиты? «Единодушия
мнений» у них не было, потому что этот орден был интернациональным, но
национальная принадлежность членов ордена не могла не иметь влияния.
Основателями ордена были в подавляющем большинстве испанцы, испанцами же были
три первых генерала (Лойола, Лайнец и Борджиа), но в 1581 г. генералом был
избран Клавдий Аквавива, неаполитанец по происхождению, принадлежавший к
французской ориентации. Во время Тридцатилетней войны, естественно, испанские
иезуиты и часть немецких поддерживали Испанию и Австрию. Поэтому карикатура
времен Тридцатилетней войны, изображающая, как «Северный лев (Густав Адольф)
разрывает тенета, которыми иезуиты окружали центральную Европу», и помещенная в
книге Цейтлина, не является искажением действительности, но ее неполным
отображением, так как большинство иезуитов помогали Северному льву. Видную роль
в деле Галилея играли как раз представитель немецко-испанской партии иезуит
Шейнер, о ком речь была выше, и лидер испанской оппозиции, посол Испании при
папском дворе кардинал Гаспар Борджиа.

А какую политическую позицию занимали видные ученые того времени? Они, конечно,
имели те или иные политические симпатии, но это практически не сказывалось на
их поведении. От Галилея как итальянца естественно ожидать поддержки
французской ориентации. Этому соответствует и список его друзей, хорошие
отношения с верхушкой французской партии. Есть даже легенда, что его лекции
слушал инкогнито сам Густав Адольф.

Что касается Кеплера, протестанта и во многом, как было показано, противника
Галилея, то его поведение скорее показывает его индифферентность к
происходившей в то время политической борьбе. Он долго и успешно работал у
императора Рудольфа II, одного из лидеров «католической реакции», а потом
состоял штатным астрологом у главнокомандующего католическими армиями
Валленштейна.

В окружении Урбана VIII создалась исключительно напряженная ситуация. В начале
1632 г. можно было ожидать полного разрыва между папой и Испанией и, может
быть, даже императором (австрийским, который был тогда номинально императором
всей германской
нации). Послы императора вместе с некоторыми кардиналами протестовали против
политики папы. Папа заявил, что данную войну не считает религиозной войной, так
как Густав Адольф выступает лишь против слишком возросшей силы Австрии.
Кардинал Борджиа выступил представителем ряда лиц с открытым протестом,
поднялся бурный спор. Папа готов был бы предать Борджиа суду, но испанский
король в Неаполе мог выступить с оружием на защиту Борджиа. Три года папа
добивался отозвания Борджиа как испанского посла из Рима, но испанское
правительство соглашалось на отзыв лишь ценой разрыва папы с Францией, на что в
конце концов папе пришлось пойти. Несомненно, что жизнь Урбана VIII была в
прямой опасности от возможных заговорщиков: астрологи в 1630 г. выпустили ряд
предсказаний о предстоящей смерти папы, явно выражавших намерения
испано-немецкой партии. Папа арестовал ряд «предсказателей», в том числе
некоторых лиц, связанных с Галилеем. Опасность для власти и даже жизни Урбана
продолжалась не только до смерти Галилея в 1642 г., но и до самой смерти папы.
Смерть Галилея совпала с началом войны за Кастро. Пармский герцог Одоардо
Фарнезе был в январе 1642 г. отлучен от церкви и лишен своих владетельных прав;
в ответ он во главе 3000 всадников вторгся в Церковную область и начал быстро
продвигаться к Риму при бурных аплодисментах испанцев, как говорит один
историк. Война продолжалась до 1644 г., когда умиравший Урбан VIII вынужден был
заключить тяжелый для его авторитета мирный договор с Кастро; подписав этот
договор, папа лишился чувств.

Вот на фоне каких событий происходил процесс Галилея. Судьба Урбана VIII
гораздо более печальна, чем судьба Галилея. Папа отнюдь не был фанатическим
католиком, заклятым врагом протестантизма, он мудро стремился ограничить власть
испанской и австрийской деспотий и стремился к прогрессивной цели --- изгнанию
испанцев из Италии. Перед смертью он мог думать, что эти его планы рухнули, так
как испано-немецкая партия восторжествовала.

Испанцев, конечно, выгнали из Италии, но гораздо позже.

Но совпадение процесса Галилея с трагическими событиями Тридцатилетней войны не
было случайным. В той же книге Цейтлина прекрасно показано, что напечатание
книги Галилея якобы с одобрения папы было хитро задуманной провокацией
испано-немецкой партии, во главе с Гаспаром Борджиа и личным секретарем папы
Чиамполи (Цейтлин, с. 191 и 192). Последний примкнул к испано-немецкой партии,
так как был обойден папой при назначении кардиналов (с. 198). Судя по всему,
они якобы с одобрения папы опубликовали знаменитые «Диалоги» во Флоренции. Цель
их была ясна: скомпрометировать папу перед фанатическим католическим миром, как
единомышленника учения, осужденного в 1616 г. католической церковью и добиться
его смещения (прецеденты смещения пап были, вспомним знаменитого Иоанна XXIII,
не только лишенного папского звания, но и вычеркнутого впоследствии из списка
пап, отчего в наши дни под именем Иоанна XXIII правил другой папа, оставивший
по себе самую светлую память). Чиамполи делал ставку на решительную победу
испано-немецкой партии, которая фактически запоздала всего лишь на десять лет,
и Чиамполи не дожил до нее, так как умер за год до смерти Урбана VIII.
Цейтлин приводит убедительнейшие факты в пользу такого понимания. Если бы само
напечатание книги Галилея было преступлением, то, конечно, должен был бы в
первую очередь пострадать цензор инквизиции и индекса доминиканец Никколо
Риккарди, которого называли «падро монстро» по причине необычайной учености,
красноречия и чудовищной толщины и который поставил свое «имприматур» (подлежит
печатанию). Но Риккарди нисколько не пострадал и остался папским цензором до
своей смерти в 1639 г., а Чиамполи после процесса Галилея был смещен со своего
поста, выслан из Рима в глухие местечки и умер опальным в 1643 г. Для спасения
своей власти, а может быть, и жизни Урбан VIII был принужден инсценировать
процесс над Галилеем, несмотря на безусловное свое расположение к Галилею.
Совершенно освободить Галилея от наказания значило бы для Урбана VIII признать
собственную ответственность за публикацию «Диалога».

Папа проявил такую «объективность», что поставил во главе списка судей одного
из своих заклятых врагов, кардинала Гаспара Борджиа, но для обеспечения
надежного следствия назначил в декабре 1632 г. нового генерального комиссара
инквизиции --- патера Фиренцуолу, сменившего прежнего, замешанного в деле
Борджиа. Этот патер Фиренцуола был главным военным инженером папы и прославился
усовершенствованием укреплений в римской крепости-замке Св. Ангела и постройкой
форта на острове Мальта. Фортификационные чертежи Фиренцуолы были часто на
столе папы наряду с новейшими поэтическими произведениями. Галилей предстал на
допросе не перед фанатическим богословом, а перед математиком и инженером в
монашеской рясе. Может быть, папа оказался недоволен слишком мягким результатом
следствия и суда? Нет. Фиренцуола сохранил полное благоволение папы: после
смерти Риккарди в 1639 г. он сделался главным цензором, а в 1643 г. возведен в
сан кардинала-епископа и архиепископа. Недоволен был Гаспар Борджиа. Это ясно
из того, что его подпись (наряду с подписями двух других лиц) отсутствует под
приговором, вынесенным Галилею. Лидер испано-немецкой оппозиции тем самым
бросил в лицо Урбану VIII вызов, что не столько Галилей сильно подозрителен в
ереси, сколько сам глава вселенской, католической и апостольской церкви. Не
Галилея надо было подвергнуть строгому испытанию, а собором лишенного папской
тиары Маттео Барберини --- вот о чем говорит отсутствие подписи кардинала
Борджиа. Любопытно, что на отсутствие подписи Борджиа и двух других лиц впервые
обратил внимание известный историк математики Мориц Кантор, хотя, казалось бы,
это прежде всего должно было интересовать обычных историков.

Но, может быть, по закону Галилея и нельзя было присудить к более тяжелому
наказанию? Мы знаем, что папа в то время был самодержавным государем Церковной
области, а самодержцы всегда не слишком считаются с законами, но в данном
случае Галилей мог бы быть присужден к самому тяжкому наказанию --- сожжению на
вполне законном основании. Ведь осуждение учения Коперника было в 1616 г., и
тогда кардиналу Беллармину поручено было уговорить Галилея отказаться от этой
ереси; кардинал имел беседу с Галилеем, и от открытой защиты учения Коперника
Галилей в общем воздерживался до опубликования «Диалогов». Поэтому считалось,
что Галилей
«раскаялся» уже в 1616 г., подчинившись указаниям Беллармина. Но если он в 1616
г. уже раскаялся, а в 1632 г. вновь согрешил, то он мог бы по закону быть
объявлен неисправимым еретиком (херетикус релапсус) и передан светской власти
для казни «со всей кротостью и без проливания крови», т. е. сожжению. Цейтлин
указывает, что даже в таком крайнем случае Урбан VIII нашел бы пути и средства,
чтобы избавить Галилея от этой страшной участи, но чрезвычайных мер не
потребовалось, так как Галилей предъявил письмо кардинала Беллармина, из коего
явствовало, что разговор с Беллармином не привел к отречению. Поэтому суд имел
право рассматривать Галилея не как неисправимого еретика, а как человека,
впервые попавшего на суд инквизиции, тогда в случае раскаяния ему смертная
казнь не угрожала. Но Беллармин в 1632 г. был уже в могиле, и закрытый суд,
если бы он хотел погубить Галилея, легко мог бы просто уничтожить письмо
Беллармина. Важна была инсценировка громкого процесса и, возможно, более мягкий
приговор. Как известно, существует легенда, что Галилей после торжественного
отречения на коленях от доктрины Коперника встал с колен и, топнув ногою,
сказал: «Э пур си муове» --- а все-таки она движется. Уэвель (1867, т. I, с 505)
справедливо замечает: «Эти слова представляются иногда героическим изречением
человека, преданного своему убеждению и истине наперекор преследованиям: я
думаю, что мы можем естественнее представить себе эти слова сказанными в виде
шуточной эпиграммы на ухо кардинальскому секретарю, с полной уверенностью, что
они будут непосредственно переданы его господину». Разумеется, громко таких
слов Галилей не мог произнести, но если эти слова он сказал на ухо, например,
патеру Фиренцуоле, то тот, наверное, ответил бы ему: «Тише, тише, Галилей, как
бы Борджиа не услышал».

Но ведь все-таки Галилей отрекся от своих убеждений, и это отречение должно
было быть ему очень мучительным. Но совершенно несомненно, что Галилей никогда
не защищал коперниковской теории как абсолютной истины, и этого нет в
«Диалогах». Вот что сам Галилей пишет на с. 22 своего труда (Галилей. Диалоги.
1948): «Благоразумному читателю. В последние годы в Риме был издан спасительный
эдикт, который для прекращения опасных споров нашего времени своевременно
наложил запрет на пифагорейское мнение о подвижности Земли... Ради этой цели я
взял на себя в беседах роль сторонника системы Коперника и излагаю ее сначала
как чисто математическую гипотезу, стараясь далее при помощи разных
искусственных приемов доказать ее превосходство не над учением о неподвижности
Земли вообще, а над тем, которое защищается людьми, являющимися перипатетиками
по профессии, ложно носящими это имя, ибо они довольствуются безоговорочным
почитанием тени и, не пытаясь размышлять самостоятельно, держатся лишь за
заученные на память, но плохо понятые четыре принципа» (Четырьмя принципами или
видами причинности Аристотеля являлись, как известно, форма, материя, движущая
причина и цель). Но все изложение «Диалогов» таково, что учение Коперника
представляется гораздо более убедительным, и поэтому совершенно ясно, что
Галилея можно было обвинить не в открытой защите осужденного католической
церковью учения, а в неискренности. Метод
Галилея был использован позднее многими, в частности французскими
энциклопедистами XVIII в., которые защищали, например, атеизм в своей
«Энциклопедии», как будто его опровергая. Думаю, что и Цейтлин в некоторой
степени использовал этот метод для усыпления бдительности сталинской цензуры,
так как некоторые выражения в его прекрасной книге наводят на такую мысль, о
чем будет упомянуто дальше.

То, что Галилей в своей книге еще до отречения принял указания Беллармина
(теория Коперника --- вполне допустимая математическая гипотеза, но не абсолютная
истина), как правило, неизвестно пишущим о Галилее, и многие «принципиальные»
люди вроде Брехта обвиняют Галилея в измене своим убеждениям. Цейтлин на такие
обвинения пишет на с. 232--233: «Некоторые старые и новые ученые и неученые
лицемеры поднимают ханжеские вопли, порицая поведение Галилея на процессе. С их
точки зрения, Галилею следовало бы гордо взойти на костер инквизиции подобно
Бруно и доставить большое удовольствие верным псам господа бога. Разумеется,
нужно воздать хвалу героизму Бруно и всех гигантов учености, духа и характера
эпохи Ренессанса. Но надо быть справедливым к Галилею. Его поведение на
процессе и отречение только тогда можно было бы назвать позорной трусостью,
если бы Галилей изменил себе, изменил той стратегии и тактике, которой он
придерживался всю жизнь и в целесообразности которой был непоколебимо убежден».
Но дальше почему-то Цейтлин пишет, что «из тщательной и всесторонней оценки
реального соотношения сил он пришел к выводу, что наилучший метод борьбы с
феодально-католической реакцией, это метод рыси, лицемерно поднявшей глаза к
небу и раздирающей когтями... трехглавого цербера папизма». Это последнее
совершенно, по-моему, неверно. Никакой цели борьбы с феодально-католической
реакцией Галилей не ставил, так как был чужд политике и придерживался
консервативных взглядов, и он вовсе не стремился бороться с папизмом. Как
правильно указывает Выгодский (Галилей и инквизиция, ч. I, 1934), Галилей и его
сторонники вовсе не были настроены антирелигиозно и антиклерикально.
Аполитичность Галилея ясна из того, что главным стимулом для переезда из Падуи
во Флоренцию он считает то, что нравы и обычаи республики не создают достаточно
условий для научной работы и что их он найдет у абсолютного князя (Цейтлин,
1935). Галилей был великолепным представителем беспартийной и аполитичной
науки. Эту точку зрения защищал в своей книге о Галилее наш выдающийся
математик Стеклов, за что на него обрушился тот же Выгодский, видимо,
удовлетворяя требованиям времени и не приводя серьезных возражений. Книгу
Стеклова я, к сожалению, не мог достать в библиотеке. Науке Галилей не изменил
и весь свой великий талант развил полностью. Те требования, которые ему
предъявлялись, не мешали развитию науки, так как даже коперниканская доктрина,
как было показано выше, допускалась к развитию как «математическая гипотеза»,
но не как абсолютная истина. Мы знаем, кроме того, что развиваемые им взгляды
встречали наиболее ожесточенное сопротивление со стороны перипатетиков, которые
препятствовали развитию его основных работ по механике, к которым католическая
церковь никаких претензий не предъявляла. Но по отношению к главным врагам
прогрессивной науки того
времени, догматическим и фанатическим перипатетикам, Галилей не сделал ни
малейшей уступки, сохраняя вместе с тем уважение к Аристотелю, как того и
заслуживал этот выдающийся философ.

Разумеется, политиканствующие писатели типа Бертольда Брехта не могут этого
понять. Для них прогрессивный ученый обязательно материалист, атеист и
сторонник революционных методов борьбы с отживающим классом. Если в Средние
века было распространено мнение «философия --- служанка богословия», то
современные сторонники партийной и политической культуры считают: «наука и
философия --- служанки политики и атеизма». Так (вопреки всему содержанию своей
книги) пытается изобразить дело и Цейтлин, с. 263: «Борьба Галилея за
революционное материалистическое учение Коперника --- это сущность и ядро его
жизни и деятельности. Эту борьбу Галилей вынес перед лицом широкой аудитории,
популяризуя коперниканские антирелигиозные идеи». Учение Коперника может быть
названо революционным в двух смыслах: 1) как подлинная революция в науке, 2) по
названию «ле революционис орбиум целестиум», но слово «революцио» по-латински
(в классической латыни этого слова, видимо, нет, но производится оно от глагола
«револьно») переводят «Об обращениях», что никакого отношения к политическим
революциям не имеет. Кроме того, совершенно ясно, что ядром деятельности
Галилея было создание новой механики, чего он успешно и достиг.

Но Урбан VIII вовсе не был представителем одного из двух пресловутых «лагерей»:
феодально-католического и буржуазно-капиталистического. Как папа, он, конечно,
должен был бы идти по линии феодально-католической реакции, а он на деле был
сторонником французской партии, которая отображала в известной степени
прогрессивные тенденции буржуазии, относительно заинтересованные тогда в
развитии науки.

Но можно ли сказать, что Ришелье, фактический глава Франции, был действительно
сторонником буржуазии? Насколько мне известно, Реформация вообще и гугеноты в
частности и были наиболее яркими представителями буржуазии, но с гугенотами
Ришелье вел беспощадную войну. В интересах какого класса? В интересах создания
из Франции централизованного абсолютистского государства, и эта линия
продолжалась и дальше. Несмотря на страшную революционность французских
якобинцев, лозунг, унаследованный от Ришелье, «Франция --- единая и неделимая»
проводился ими с несравненно большей решительностью и последовательностью, чем
подлинно великий лозунг Революции: «свобода, равенство и братство». Линия
Ришелье привела к созданию многочисленного класса дворянства --- подлинных
паразитов народа, что и привело в конце концов к великой трагедии. В Англии
феодализм не искоренен до сего дня (лендлорды и палата лордов). Германия
объединилась только во второй половине XIX в., и это объединение привело тоже к
ужасному результату.

Но Франция, которая обладала культурной гегемонией в XVII и XVIII вв., потеряла
ее и во многих отношениях отстала от Англии, Германии и Австрии. Конечно,
победа Австрии в Тридцатилетней войне была нежелательной во многих отношениях
для прогрессивного человечества, но это не значит, что австрийский
государственный строй во
всех отношениях был хуже французского и являлся отражением наиболее
реакционного класса. К. А. Тимирязев в своей работе «Источники азота растений»
(Избр. соч. в четырех томах, т. 2, 1948, с. 147) сравнивал двух монархов,
оставивших противоположную славу. Один, Фридрих II, сошел со сцены,
сопровождаемый восторженным удивлением современников и потомства. Другой, Иосиф
II, еще при жизни вынес горькое сознание, что все его благие намерения
потерпели крушение, встретив отпор, главным образом, в своекорыстии правящих
классов, и сошел в могилу, оставив по себе память коронованного неудачника. «Но
преследовавшие его в течение всей его жизни неудачи не помешали, конечно,
беспристрастной истории видеть в нем одного из просвещеннейших и передовых
представителей своего века, как не помешали воспоминанию о нем сохраниться в
благодарной памяти австрийского крестьянина, освобожденного им от крепостной
зависимости. Если всякому знакомо изображение Фридриха II на коне, окруженного
сонмом полководцев, то теперь еще, в глухих уголках Австрии, можно встретить
популярную гравюру, изображающую Иосифа II, пашущего плугом». Дальше Тимирязев
указывает, что постоянная забота Иосифа II о насущных потребностях крестьян
выразилась в том, что в 1784 г. он возвел в дворянское достоинство Иоганна
Христиана Шубарта, дав ему титул фон Клеефельд. Поводом к этому были не
какие-нибудь выдающиеся подвиги на поле брани, не блестящая деятельность на
поприще дипломатии или администрации, --- нет, вся заслуга Шубарта заключалась в
том, что он деятельно, печатным словом и примером, пропагандировал возделывание
клевера и тем, по словам историка земледелия, "положил краеугольный камень
благосостоянию немецкого поселянина". Пожалуй, вряд ли можно считать Иосифа II
неудачником, если, конечно, не придерживаться моральной чумы современности ---
ультрапатриотизма, видящего основную задачу правителей в расширении границ. Но
даже и тут Франция отстает перед своими соседями Англией и Испанией. Франция
еще до Революции потеряла Канаду, а воинственный Наполеон, проливший огромное
количество крови, продал Соединенным Штатам обширную провинцию Луизиана
(значительно превышающую одноименный штат Луизиана). Лица, говорящие
по-английски, занимают огромную территорию на Земле, английский язык уже
фактически сделался международным, на нем печатается больше половины научных
трудов. Что касается Испании, то, конечно, и сейчас она сильно отстает от
Франции в культурном и общественном отношении. Но испанский народ вместе с
португальским, наряду с многими ужасами, внесенными ими в историю (а кто не
вносил ужасов?), выполнили и великую историческую миссию в Латинской Америке.
Этот комплект стран, по размерам и по населению примерно равный СССР,
представляет удивительное разрешение той расовой проблемы, которая так
потрясает англосаксонские страны. По недавно выпущенным у нас справочникам
«Зарубежные страны» (1957) и «Латинская Америка» (1962) можно составить себе
представление о племенном составе 20 латиноамериканских стран. За пять лет
население этих стран увеличилось со 170 миллионов до 206 миллионов. Состав
населения чрезвычайно колеблется, и наряду со странами, где население состоит
почти исключительно из потомков европейцев (Аргентина, Уругвай), есть страны,
где белые почти отсутствуют или все перемешано. Данные двух справочников иногда
сильно расходятся; например, для Мексики (второй по величине стране Латинской
Америки, население которой, по новейшим данным, приближается к 50 миллионам)
«Зарубежные страны» дают: испано-индейские метисы --- более 50\%, индейцы --- около
33\%, креолы, испанцы и др. 16--17\%, по «Латинской Америке» --- метисы около 79\%,
до 20\% --- индейцы, и выходит, что лиц чисто испанского происхождения почти нет.
По-видимому, испанцы и индейцы так сильно перемешались, что совершенно
невозможно установить точную границу между краснокожими и «бледнолицыми». В
других странах имеется значительная примесь негров или мулатов. Всего цветных
по «Латинской Америке» 121,7 миллиона, или 59,0\%, белых --- 84,6 миллиона, или
41,0\%. По «Зарубежным странам» (данные на пять лет раньше) цветных 98,5
миллиона, или 57,9\%, белых 71,7 миллиона, или 42,1\%. При такой чересполосице
самых разнообразных племен можно было бы ожидать значительного расового
антагонизма, но об этом ничего не слышно. Несмотря на отсутствие
государственного единства, имеется единство культуры (португальский язык в
Бразилии, испанский в остальных), единство католической религии, верной
хранительницы великих принципов интернационализма и антирасизма.

Не только злое принесли Испания и Португалия в историю человечества, хотя,
несомненно, они принесли много злого. И пусть те политики и политиканы,
которые, достигнув определенного (иногда спорного) прогресса, при помощи злых
средств, примут в соображение эти показания в пользу латинских стран. «Каким
судом вы судите, тем и вас будут судить». Сейчас страны Латинской Америки
несомненно вступили на путь стойкого культурного прогресса, и испанский язык
сделался одним из мировых языков, конкурируя с французским. Поэтому,
рассматривая процессы истории ретроспективно, трудно резко провести разделение
на козлищ и овец. В некоторых странах Латинской Америки (мне известно это для
Мексики и Уругвая) смертная казнь полностью отсутствует, как она отсутствует
полностью и в ФРГ (отсутствовала и в Израиле до процесса Эйхмана). Мне
неизвестны социалистические страны, где бы этот позор юриспруденции
отсутствовал. Мне известно также, что Фидель Кастро за открытое восстание
против диктатора Батисты был подвергнут лишь тюремному заключению, а сейчас
ввел смертную казнь даже за воровство (конечно квалифицированное). Где же
козлища, где же овцы?

Возвращаясь к Галилею и Тридцатилетней войне, можно сказать, что политический
характер процесса Галилея совершенно ясен, но признать здесь ведущими классовые
влияния можно только тогда, когда находишься в плену устарелых догматов.

Подведем итог судьбе Галилея. Можно ли в целом считать его несчастным
человеком, заслужившим тяжелыми лишениями и страданиями свое право на
бессмертие? Так его рисует, например, Брехт, который считает, что последние
годы он мучительно казнил себя за «отступничество». Несомненно, что Галилей
очень строго реагировал на ту борьбу, которую он вел в течение всей своей жизни
с перипатетиками и был омрачен судьбой своих последних лет. Темперамент его
был очень живой, и он явно выражен даже там, где Галилей встречал справедливые
возражения. Он разочаровался в Урбане VIII. Зная истинные взгляды этого умного
и образованного человека, он несколько наивно думал, что с его восшествием на
папский престол все прогрессивные идеи в науке получат полное развитие, и был в
этом горько разочарован. Видимо, будучи далеким от политики, он не понимал
истинного смысла процесса. Но взглянем на судьбу Галилея объективно. Родился он
в 1564 г. и уже в 1589 г. был профессором в родном городе Пизе. Там у него
вышли конфликты с перипатетиками, но он перешел на такую же должность
профессора математики в Падую, где и пробыл с 1592 по 1610 гг. Оттуда он
совершенно добровольно перешел во Флоренцию, куда великий герцог Тосканский
пригласил его с большим по тому времени содержанием и званием первого
математика и философа его высочества. Падуанский период Галилей потом считал
счастливейшим периодом своей жизни. При отъезде во Флоренцию он расстался, как
известно, со своей невенчанной женой, и, очевидно, у него пропал вкус к
семейной жизни, так как он новой семьи не завел, несмотря на свое возросшее во
Флоренции экономическое благосостояние. Во Флоренции он пробыл вплоть до
процесса 1633 года, но и после процесса работал и закончил свое величайшее
произведение. Он был блестящим лектором и стилистом, пользовался огромной
популярностью и известностью, жизнь вел отнюдь не аскетическую, вполне в духе
своей эпохи. Неужели такого человека можно назвать несчастным? Есть великая
мудрость в древнем поверье, изложенном в старой легенде о Поликрате (поликратов
перстень):

Здесь вечны блага не бывали,

И никогда нам без печали

Не доставалися они.

Как это ни странно, но сходную мысль высказывает и такой представитель
современной позитивной науки, как Карл Пирсон (Грамматика науки. Рус. пер.,
изд. «Шиповник», с. 13). Он указывает, что новые идеи, связанные с Дарвином,
медленно проникают в наши общие представления. «Эта медлительность не должна
нас обескураживать, ибо одним из важнейших фактов социальной устойчивости
является инертность, или, скорее, даже активная вражда, с которой человеческие
общества встречают всякие новые идеи. Это --- горнило, в котором шлаки отделяются
от чистого металла и которое спасает социальное тело от ряда бесполезных и,
может быть, даже гибельных потрясений. То, что реформатор часто должен также
быть и жертвой, есть, может быть, не слишком дорогая цена за осторожность, с
которой должно двигаться общество как целое».

И если сравним судьбу Галилея с судьбой многих других пионеров мысли, то придем
к заключению, что за свою великую и заслуженную славу Галилей заплатил совсем
дешево.

Ну а теперь, прежде чем распрощаться с делом Галилея, нам остается рассмотреть
один или, вернее, два пункта, наиболее важных с точки зрения философии науки. А
что бы случилось, если бы могущественная власть признала правоту Галилея и кто
же был прав в споре Галилея с Беллармином?

Спор ведь шел не о том, чтобы вовсе осудить систему Коперника, а о том,
признать ли ее как удобную математическую гипотезу или как абсолютную истину.
Что бы случилось, если бы могущественная власть, взявшая на себя право
руководить всей жизнью страны, как экономической, так и культурной, «утвердила»
бы «Диалоги» Галилея и (как это свойственно могущественной власти) стала бы
проводить в жизнь свое утверждение? Разумеется, так как власти несвойственно
быть хорошо знакомой с отдельными науками, она не позволила бы выделить кое-что
из «Диалогов» как ошибочное, напротив, всякая критика отдельных мест
рассматривалась бы как принципиально недопустимый ревизионизм. А так как
Галилей критиковал Кеплера, то, значит, дальнейший этап коперниковой теории был
бы запрещен. Так как Галилей не допускал и считал совершенным суеверием влияние
небесных светил на Землю, то не позволено было бы появиться трудам Ньютона;
развитие гелиоцентрической теории было бы полностью остановлено. Напротив, мы
видели, что отказ Галилея считать гелиоцентрическую теорию за абсолютную истину
никакой задержки в развитии астрономии не вызвал.

А отсюда получается ответ и на другой вопрос. Нельзя сказать, чтобы в споре
Галилея с Беллармином Галилей был полностью прав, а Беллармин полностью не
прав. Конечно, Беллармин исходил из религиозных догматов, но он делал такие
уступки свободному научному мышлению, которые не стеснили развития науки, и в
этом он подошел фактически к современной научной гносеологии, связанной с
именами Кирхгофа, Маха, Дюгема, Пуанкаре и других и которой мы обязаны в
значительной степени тому феноменальному прогрессу, который мы наблюдаем в
физике и других точных науках в XX в. Один из лидеров этого направления, П.
Дюгем, крупнейший историк физики, и формулирует основное положение: «Всякий
физический закон ни правилен, ни неправилен, а только приблизителен»,
«Экспериментум круцис вещь в физике невозможная» и др. Дюгем открыто объявляет
о своем сочувствии идеям Э. Маха, и потому наши казенные философы считают его
совершенно неприемлемым. Не так думал Ленин. Он в ряде мест «Материализма и
эмпириокритицизма» критикует Дюгема за его махизм, но вместе с тем он указывает
(1948, с. 294), что «в целом ряде мест он (Дюгем) вплотную подходит к
диалектическому материализму». Ленин там же цитирует слова Дюгема: «Борьба
между реальностью и законами физики будет длиться бесконечно; всякому закону,
который сформулирует физика, реальность противопоставит, рано или поздно,
грубое опровержение --- опровержение посредством факта; но физика будет неутомимо
ретушировать, видоизменять, усложнять опровергнутый закон» --- и прибавляет: «Это
было бы совершенно правильным изложением диалектического материализма, если бы
только автор твердо держался за существование, независимое от человечества,
этой объективной реальности». Таким образом, в области гносеологии Ленин не
находит разницы между махистскими взглядами Дюгема и диалектическим
материализмом. Различие заключается лишь в онтологии, учении о истинно сущем.
Ленин (в данном случае совсем не диалектически, а догматически) решает его в
смысле материалистическом, а как решают махисты? Никак, так как махисты считали
главной заслугой махизма полное изгнание метафизических признаков, полную
ненужность онтологии (обычно ее называют также метафизика, но этот термин имеет
слишком много смыслов). Это --- свойство всякой позитивной философии, включая
Конта, Спенсера и др. Конт и различал три периода в развитии человеческой
культуры: теологический, метафизический и научный. В научном периоде «наука ---
сама себе философия» и никакой особой философии не нужно. Сейчас многие
выдающиеся ученые отказываются от этой точки зрения. Назову два имени:
Эйнштейна и Гейзенберга, но оба, критикуя махизм (заслуги которого они
признают), в онтологии приходят к объективному идеализму (разных смыслов), но
не к материализму. Огромный рост разных идеалистических направлений в наше
время (при полной или почти полной стагнации материализма) показывает, что и в
области онтологии Беллармин не может считаться совершенно устаревшим. Недаром
такую известность приобрел в наше время выдающийся иезуит Тейяр де Шарден.
Любопытно, что в той же книге Цейтлина есть указания, что и сам Галилей имел
некоторый уклон к идеалистическому формализму. Цейтлин даже знаменитый лозунг
Ньютона «Не делаю гипотез» считает иезуитским, равноценным принципиальному
отказу от познания сущности истинных причин явлений природы; и Декарта
(который, как известно, получил воспитание в иезуитской школе) Цейтлин считает
представителем философского иезуитизма, избравшим своим девизом: «Хорошо живет
тот, кто хорошо скрывается». Резкое расхождение, казалось бы, двух непримиримых
взглядов стерлось, и общий вывод можно сделать тот, что никогда не следует
возводить в абсолютный догмат никакое, казалось бы, самое прочно обоснованное
положение. Но это не означает абсолютного скептицизма, а лишь правильный
пробабилистический подход, и Дюгем неоднократно цитирует великого мыслителя
Паскаля: «Мы обнаруживаем бессилие в доказательстве --- бессилие, которое никакой
догматизм победить не может; у нас есть идея истинного, которую весь пирронизм
победить не может» и «Когда он (разум) слишком превозносит себя, я принижаю
его; когда он слишком унижается, я превозношу его».

А теперь взглянем немного на прошлое глазами современника. Процесс
Галилея считался одним из самых блестящих доказательств вреда религии
вообще и католической церкви в частности, и в XIX в. этот аргумент
имел известное основание, так как в этом веке, в особенности после
окончания наполеоновских войн (так называемая «викторианская эра»),
господствовала идея о непрерывном монолитном прогрессе человечества,
вечный мир казался целью, которую можно достичь без специальных
усилий, и в смысле политической и религиозной свободы было достигнуто
положение, не имевшее прецедента в истории человечества. Были
люди самых разнообразных взглядов, предупреждавшие, что достигнутое
благополучие иллюзорно, что целый ряд особенностей общественного строя
Европы (парламентаризм, свобода, отсутствие расовых предрассудков и т.
д.) непрочны, но это, как правило, было гласом вопиющего в пустыне. Но
пришел XX в., и возникли такие явления, о которых не могли и подумать
в ХIХ в.

НАУКА И РЕЛИГИЯ

ПОСТАНОВКА ВОПРОСА

Вопрос об отношении науки и религии имеет по крайней мере двухвековую давность,
а правильнее, может быть, даже двухтысячелетнюю, и то решение, которое
наметилось примерно полтора века тому назад, многим из современных
интеллигентов кажется окончательной истиной в последней инстанции. Оно всего
лучше отображено в знаменитом коротком разговоре великого ученого Лапласа и
выдающегося государственного деятеля Наполеона Бонапарта. Когда Наполеон
(который, как известно, был хорошо знаком с высшей математикой) ознакомился с
известным сочинением Лапласа о небесной механике, он задал ему вопрос, почему
тот не упоминает о Боге в этой книге. Наполеон, очевидно, намекал на Ньютона,
которые закончил свои великие «Математические начала натуральной философии»
идеологическими рассуждениями. «Государь, я не нуждался в этой гипотезе», ---
ответил Лаплас (цитирую по Энгельсу, Диалектика природы, с. 268). Наполеон
задал свой вопрос не потому, что он был верующим, но он перешел от
преследования католической церкви к конкордату с целью упрочить свое положение.
Религия не нужна науке, но нужна даже атеистическому деспоту как «незримая
паутина» (прекрасное выражение Горького) для более легкого подчинения
угнетенных масс, как опиум для народа. Вспомним преследование инакомыслящих,
Галилея, Дж. Бруно, инквизицию, индекс запрещенных книг, сопротивление
эволюционной теории Дарвина, и как будто придется сделать заключение, что
религия не только бесполезна, но и вредна для развития науки. Мнение
просветителей конца XVIII и начала XIX в., казалось, торжествует в течение
всего XIX и начала XX в. В биологии телеологии был нанесен сокрушающий удар Ч.
Дарвином, в конце XIX в. был сильный процесс «дехристианизации» Франции,
Бисмарк объявил «культуркампф» против католической церкви в Германии, наконец,
в XX в. правительства почти одной трети человечества стали откровенно
атеистическими. Подрастающая молодежь этих стран практически ничего не знает о
религии. Идеи социализма и атеизма считаются неразрывно связанными, и поскольку
деятели социализма выставляют великолепный идеал осуществления справедливого
строя на земле (в противовес обещанному церковью Царствию Божию на небе после
смерти), то эти социалистические идеалы считаются морально обязательными, а
следовательно, морально обязательным является и атеизм, и всякая «поповщина»,
легко приводящая к тем или иным формам теологии, отметается «с порога». Поэтому
обязательность атеистического обучения молодежи, запрещение религиозной
пропаганды и свободного издания религиозных книг не кажется многим ограничением
свобод вообще и свободы совести в частности, так как для многих честных
образованных и умных людей всякая религия кажется пережитком прошлого, подобным
каннибализму, обычаю убивать стариков, колдовству, гаданиям по звездам,
«свободе» заводить обширные гаремы, геноциду, учению о низших расах и проч., и
восстановление совершенно недопустимых с моральной точки зрения обычаев:
истребление пленных, стариков, душевнобольных и вообще «неполноценных»,
проводимое гитлеровцами, конечно, не делает гитлеровскую Германию более
«свободной» страной, чем страны антигитлеровской коалиции, так как свобода
истреблять или развращать себе подобных (торговля наркотиками, порнографической
литературой и проч.) не может считаться свободой, достойной прогрессивного
человечества. Не всякая свобода и желательна, как говорится в одном коротком
диалоге еще дореволюционного времени: «Извозчик, ты свободен?» --- «Свободен». ---
«Кричи: „Да здравствует свобода!"» Свобода пропаганды суеверий, распространения
наркотиков и т. п. не есть та свобода, за которую есть смысл бороться.

Неудивительно, что в нашей стране, отставшей в силу пережившего себя
общественного строя от западных стран, особенно силен был антирелигиозный дух
среди нашей интеллигенции, несмотря на то что до революции преподавание в
школах в значительной степени было проникнуто религиозным духом (разных
религий), в каждом паспорте обозначалась религиозная принадлежность носителя
паспорта (национальность в паспортах не фигурировала) и не допускалось, чтобы
кто-либо объявил себя атеистом. Протест против религии носил исключительно
резкий характер. Ленин считал религию вообще одним из отвратительнейших явлений
в обществе и всякую самую утонченную форму фидеизма он решительно отбрасывал «с
порога». Даже субъективный идеализм объявлялся «поповщиной», и по отношению к
самым близким людям Ленин не допускал никакого компромисса. Его собственный
отец был глубоко религиозным человеком, как и отец Чернышевского, и оба они
были очень почтенными людьми. Религия была тем более ненавистна нашим
революционным демократам, что она тормозила развитие народа. Поэтому все они
переходили от глубокой религиозности в юности к воинствующему атеизму зрелого
возраста (Салтыков-Щедрин, Писарев, Чернышевский и др.). Но воинствующий атеизм
был свойствен не только революционерам или марксистам, но и лицам, далеко
стоящим от революционной борьбы и марксизма. Мне известен один ученый
выдающихся умственных и моральных качеств, весьма скептически относившийся к
селекционизму (неодарвинизму, т. е. учению о ведущей роли естественного
отбора), но вместе с тем считавший, что дарвинизм принес пользу как мощное
антирелигиозное учение. Религия в глазах этого ученого и честного человека была
настолько отрицательным явлением, что для борьбы с ней можно было использовать
даже ложные учения. Подобно тому как в христианском песнопении поется про
Христа, что он «смертию смерть поправ», так и здесь ложью следует победить
другую, более страшную ложь. «Цель оправдывает средства» --- этот лозунг
употребляется и людьми, весьма отрицательно относящимися к иезуитам.

И не только у нас, но и среди ученых и мыслителей Запада сейчас есть немало
представителей, полностью солидаризующихся с тем мнением, что религия есть зло,
только мешавшее прогрессу человечества, и что хотя среди великих ученых и
философов мы знаем очень много истинно религиозных людей, они сделались
великими учеными не благодаря, а вопреки своей религиозности, и что пропасть
между религией и наукой непроходима. Такую точку зрения развивают и лица,
весьма враждебно относящиеся к коммунизму, например Бертран Рассел. Его
неприязнь к коммунизму доходила до того, что он (пока Запад обладал монополией
атомной бомбы) был не против того, чтобы использовать преимущество Запада для
политических целей, но он же (сейчас --- видный представитель движения за мир)
сотрудничает в нашем журнале «Наука и религия» и прямо заявляет (в полном
согласии, например, с нашим Луначарским), что ученый и прогрессивный человек
вообще не только может, но и должен быть антирелигиозным человеком. Рассел не
одинок в своем высказывании. В 1961 г. появился под общей редакцией Юлиана
Гексли сборник статей 27 авторов под названием «Структура гуманизма»,
подвергнутый подробной рецензии в британском журнале «Философия науки» (1963,
том XIV, № 53, с. 41-53) Аланом Стюартом. Рецензент приветствует этот сборник
как «новое откровение», как веху в деле эмансипации человеческого духа, как
обоснование нового эволюционного гуманизма, полностью отвергающего все старые
религиозные суеверия, всякую сверхъестественную религию. И несмотря на высокую
оценку сборника, Стюарт отмечает очень крупный дефект, заключающийся в том, что
эти новые гуманисты одиннадцать раз цитируют иезуита Тейяра де Шардена даже с
указанием на желательность знакомства с этим автором. Рецензент негодует (с.
52): «Многие образованные честные искатели истины, которые знают кое-что о
последних двух тысячах лет человеческой истории, не могут избежать связывать
„священника" (попа) с „поповщиной", а „поповщину" с „лживостью". Здесь же мы
видим испорченную книгу, которая могла бы быть превосходной. Потому что,
несмотря на все их изъявления о глядении в будущее, эти гуманисты, подобно жене
Лота, тоскующей по Содому, глядят назад». Я затруднялся в переводе слов
«priest» и «priestcraft» и вставил вместо «priestcraft» «поповщину». Последний
термин широко применяется в советской литературе, но он почему-то отсутствует в
трех новых словарях: русско-английском и русско-немецком 1952 г. и
русско-французском 1962 г. Но мы видим, что рецензент, предлагая нам смотреть
только вперед и не оглядываться, или забыл, или игнорирует известное положение
о диалектике, где новый синтез в известном смысле возвращается к старому.

Но наряду с таким резким отношением ко всякой «поповщине» мы имеем гораздо
более примирительное отношение. Добржанский, очень близкий по своим
биологическим взглядам к Ю. Гексли, заканчивает свою интересную книгу об
эволюции человечества цитатой из Тейяра де Шардена и считает, что эволюционная
идея иезуита является лучом надежды для человечества. Он считает, что его книга
содержит и науку, и метафизику, и теологию, и даже поэзию, испорченную
несколько в английском переводе. Предисловие к русскому переводу
книги Тейяра де Шардена «Феномен человека» написано выдающимся, одним из
наиболее культурных французских коммунистов, Роже Гароди, который, наряду с
критикой многих положений Шардена, высоко оценивает многие прогрессивные
стороны его взглядов. Исчезает положение «двух лагерей»: с одной стороны,
реакционный черный лагерь защитников религии, состоящий из честных невежд или
фанатиков и бесчестных эксплуататоров, а с другой --- светлый лагерь
«прогрессистов», куда относятся все умные и честные люди, «свободомыслящие»,
или, как это понималось на языке XIX в., атеисты и социалисты. Все
перемешалось. По отрицательному отношению к религии Ленин оказался в одной
компании с Б. Расселом и цитированным рецензентом Стюартом, более умеренный
Луначарский, говоривший о «религиозном атеизме», о «новой религии социализма»,
сочувствующий неодарвинист Добржанский и, наконец, коммунист Гароди: для
каждого оттенка отношения к религии мы можем подыскать представителей обоих
лагерей. И здесь, как может быть всюду, господствует в высокой степени
комбинационный принцип.

Сейчас в мировом масштабе происходят два встречных течения. Наряду с успехами
антирелигиозности в мировом масштабе, падением посещаемости храмов и проч. мы
имеем укрепление религиозных позиций в ряде культурных стран. Во Франции
возвращены изгнанные на пороге столетия иезуиты и во главе правительства стоит
верующий католик. Ряд правительственных партий во Франции, ФРГ, Италии, Бельгии
и других странах откровенно христианские. Можно ли это объяснить только
империалистической реакцией? Два выдающихся деятеля XX в. --- Ганди и Кеннеди ---
не скрывали своей религиозности, и оба пали от рук людей, сходных по идеологии
с фашизмом, империализмом, расизмом и прочими бесспорно ретроградными
идеологиями. Защитниками религии или «поповщины» в широком смысле слова (причем
самого разнообразного характера) выступают самые передовые ученые
современности: Эддингтон, Эйнштейн, Гейзенберг, Планк, Шредингер и многие
другие меньшего значения. Писатели Сент-Экзюпери, Веркор и другие настаивают на
необходимости синтеза, а не голого отрицания. Растет число идеалистов самых
разнообразных направлений, в то время как материалистическая философия скорее
обнаруживает явные признаки загнивания. Мы знаем, что в нашей стране под видом
борьбы с религией и идеализмом систематически боролись со всеми новыми
течениями в науках: теория относительности, принцип неопределенности, теория
расширяющейся Вселенной, теория резонанса в химии, настоящая генетика. Даже
там, где поддерживали то или иное здравое направление (например, учение об
условных рефлексах Павлова), так догматизировали учение, что оказали ему
медвежью услугу. Все же новшества, предложенные под видом истинно
материалистической науки, не выдержали испытания временем, и притом короткого
времени. Таким образом практически опровергнут подход: раз это ведет хотя бы в
слабой степени к «поповщине», это надо отвергнуть «с порога».

Не менее неожиданными были и события на этическом и политическом фронтах.
Ставилось в вину всем религиозным правительствам, что они наряду с
«запугиванием адом» имели на вооружении и смертную
казнь. Запугивание адом у нас исчезло, но смертная казнь фигурирует в нашем
Уголовном кодексе в таком числе статей, как, я думаю, нигде в капиталистических
странах, а некоторые капиталистические страны (мне известны: ФРГ, Мексика,
Уругвай, Израиль до процесса Эйхмана) вовсе не имеют смертной казни в мирное
время. В политике «и сольются в одно все народы в вольном царстве святого
труда» пока обернулось жесточайшим конфликтом между двумя крупнейшими
коммунистическими партиями: СССР и Китая.

Если пока между Китаем и СССР еще нет вооруженного конфликта, то только потому,
что Китай еще недостаточно силен и раздираем внутренними противоречиями, а у
нас неслыханная в царские времена милитаризация и достаточная удовлетворенность
последними захватами. Но разговор между двумя «социалистическими» странами по
тону вовсе не отличается от разговоров двух милитаристских империалистических
держав. Но вопросы этики и политики уже выходят за пределы темы настоящей
статьи.

Мы видим, таким образом, что в отношении религии сейчас наблюдаются два
противоположных процесса: 1) антирелигиозный, касающийся масс и среднего уровня
интеллигенции; 2) прорелигиозный, выражающийся в обильном числе всевозможных
направлений, часто затрагивающих самые высокие уровни передовых мыслителей
современности. Полезно выяснить, является ли это второе направление вызванным
какими-либо новыми открытиями или явлениями современности или можно найти и в
материалистическом мировоззрении начала XIX в. такие черты, которые ясно
указывали на его несовершенство. Так ли ясно все в отношении разговора Лапласа
с Наполеоном? К этому и перейдем.

Глава I. НАУКА --- ВРАГ СУЕВЕРИЙ И ЧУДЕСНОГО

Дать определение науки не так-то легко, для наших целей достаточно ограничиться
тем, что ясно сквозит в словах Лапласа. Наука --- враг всякого суеверия,
чудесного, а религия, напротив, основана на суевериях и принятии чудесного.
Всего лучше это выражено в знаменитых словах блаженного Тертуллиана: «кредо,
квиа абсурдум ест», т. е. «верю, потому что это абсурдно». Пока не будем
разбирать вопроса, является ли высказывание Тертуллиана ортодоксальным или
типичным для всех христиан. Ограничимся тем, что уже в раннем христианстве
обозначилось направление, резко противополагавшее религиозное учение тогдашней
классической науке (см., например, История философии, 1941, № 1, с. 387---388) и
в известном смысле сохранившееся и в более поздние времена. Каков смысл
высказывания Тертуллиана? В учении о познании Тертуллиан придерживался
вульгаризированного стоического материализма и считал, что все действительно
существующее телесно, в том числе Бог и бессмертная душа. Таким образом, его
взгляд вовсе не является выражением его идеализма, а напротив, теснейшим
образом связан с материалистическим характером его мировоззрения. Разумом он
был материалист, но, как христианин, принимал и такие явления, которые
материалистическому объяснению
совершенно не поддавались, но так как они были для него совершенно бесспорны,
то в них он верил как в абсурд с точки зрения материализма. Это была не слепая,
а сознательная вера, и он твердо проводил различие между верой и знанием. А так
как для него на первом плане стояли истины веры, то естественно, что к
человеческой мудрости он относился с меньшим уважением и порой даже с
презрением. Но посмотрим, свободен ли материализм XIX в. от принятия чудесного
и непонятного. Смысл науки в том понимании, которое нас сейчас интересует,
заключается в том, что она должна бороться с суевериями, что можно понимать в
пяти формах: 1) чудесное, т. е. непонятное, 2) абсурдное, 3) противоречивое, 4)
сверхъестественное и 5) невероятное. Разберем по очереди.

1. Непонятное, часто необычное

Все непривычное, неожиданное, непонятное нам кажется чудом, и мы и в практике
жизни, и в науке стремимся к тому, чтобы непонятное свести к чему-то понятному.
Антропоморфические религии сводят все «чудесное» к вмешательству существа,
подобного человеку, но невидимого. Так толковались изумительные приспособления
органического мира. Дарвин оставил телеологию, т. е. мнение, что основа в
биологии --- явление приспособления, но вместо невидимого бога ввел тоже
невидимое явление --- естественный отбор, приемлемый для материалистов. Задачей
науки оказалось дать простое объяснение сложным явлениям. Это направление имеет
большую давность. Уже цитированный Стюарт (Брит. Журн. филос. науки, 1963, т.
XIV, № 53, с. 53) по поводу нового, эволюционного гуманизма пишет: «Это ново,
ново в истории так называемой западной или греко-христианской цивилизации.
Конечно, были и раньше люди, которые мыслили свободно --- в общем так же, как и
сегодняшние гуманисты. В дохристианскую эру были Эпикур и Лукреций с их
последователями». Можно ли считать Эпикура одним из лидеров античной науки?
Вряд ли. Со взглядами Эпикура можно познакомиться по известному античному
историку философии Диогену Лаэрцию (я использовал немецкий перевод). Диоген
Лаэрций --- добросовестный, но малокритический историк, который сообщает разные
небылицы про многих философов, но Эпикуру он посвятил, как и Зенону-стоику,
наибольшее место в своей книге (по 68 страниц) и, в отличие от других
философов, привел много текстов. Хотя обычно Эпикура считают представителем
линии Демокрита, но он резко отличается от Демокрита ярко выраженным
индетерминизмом. Основа мировоззрения Эпикура: боги не вмешиваются в нашу
судьбу, не вознаграждают и не карают, смерть ни к чему плохому не приводит. Это
и есть основа подлинного эпикуреизма: безмятежность души, атараксия. У самого
Эпикура это не приводит к безнравственности: напротив, многие из его положений
сходны с таковыми стоиков, он обосновывает этику утилитарными соображениями,
что вовсе не так плохо, но что дает его мировоззрение науке? Здесь как будто
проявляется полное свободомыслие. В вопросе о величине Солнца, о движении
планет, знамениях, громе и других явлениях он выставляет различные гипотезы и
не дает предпочтения ни одной. Они все с его точки зрения равноценны, при
условии, чтобы не было мифов, связанных с религией. Но будучи совершенно
беззаботным в области научных гипотез, Эпикур крайне догматичен в онтологии или
метафизике и против всякой диалектики. Но, может быть, это и есть то полезное
ограничение свободы, которое необходимо для ученого, чтобы он сосредоточил свои
усилия, основываясь на определенных бесспорных аксиомах. Нет, ограничиваясь
объяснениями и не придавая им принудительного значения, Эпикур чрезвычайно
презрительно относится к терпеливым усилиям античных астрономов, старавшихся
длительными наблюдениями выяснить законы движения небесных тел, т. е.
положивших начало математическому описанию явлений природы. Переводчик и
комментатор Апельт правильно пишет, что если бы такая точка зрения
восторжествовала, естествознание никогда бы не вышло из исходного состояния.

Последователями Эпикура в смысле необходимости в первую очередь «объяснения», а
не математического «описания» явлений был, конечно, Лукреций и все эпикурейцы.
А так как в Римском государстве в философии господствовали стоики (обращавшие
внимание только на этику) и эпикурейцы, то становится понятным тот ничтожный
вклад в науку, который сделала могущественная Римская империя.

Но может быть, учения Эпикура (и Лукреция, который не дал, кажется, ничего
оригинального по сравнению с Эпикуром) были в политическом отношении
прогрессивны? Он был гуманный человек и рекомендовал доброе отношение к рабам,
но в смысле общественной идеологии был прототипом «премудрого пескаря»: не
стоит жениться и иметь детей, не занимайся государственной деятельностью, как
бы чего не вышло, пребывай в мудрой атараксии и не размышляй о возможности
преобразований.

Если мы посмотрим всю историю человеческой мысли, то убедимся, что решительно
все строители «утопий» были идеалисты, что же касается материалистов, атеистов
и антирелигиозников, то они или сами были тиранами(Критий), или защищали
абсолютизм под разными видами --- «просвещенного абсолютизма» и т. д. (Гоббс,
Дидро и пр.). Все идеологи революций вплоть до XIX в. работали под
идеалистическими знаменами и в общем сделали немало для преобразования
общества. Поэтому слова Маркса: «Философы только объясняли мир, а его
необходимо перестроить» (точно слова не помню) надо понимать:
«Материалистические философы только объясняли мир, а сейчас им надо приняться
за перестройку», и тогда он будет звучать более или менее правильно, если
оставить в стороне вопрос (сейчас оставим его без рассмотрения), в какой мере
революционный марксизм может считаться чисто материалистическим учением.
Несомненно, что современный селектогенез, т. е. эволюционное учение, восходящее
к Дарвину, совершенно пропитан эпикурейским духом. Основной императив
дарвинской морфологии: дай какое-нибудь «причинное» и «механическое» объяснение
структуре, которое можно свести к действию естественного отбора, с тем чтобы
устранить «поповщину», «платонизм» и прочие вредные учения. Неважно, что это
объяснение не является механическим или причинным в смысле точных наук, важно
заглушить сомнения
в отсутствии целеполагающих начал в природе. К математическому толкованию формы
и системы это направление, естественно, глубоко враждебно. Но это направление
поддерживают и выдающиеся математики! Верно, но об этом придется сказать
несколько слов в разделе о невероятном.

Простое объяснение есть низший этап развития научного мышления, и если это
направление доминирует, то оно притупляет то, что можно назвать научной
бдительностью, удивлением перед новыми фактами, и становится подлинным опиумом
для науки.

Только что рассмотренный пункт к Лапласу не относится, так как он был
представителем точной науки, и когда давал объяснения (например, в его
знаменитой космогонической гипотезе), он оговаривал ненадежность этой гипотезы,
так как ее математической теории он дать не мог.

2. Абсурдное, т. е. нелепое

А вот абсурда Лаплас не избежал. Он был убежден, что в мире существуют как
конечные реальности одни атомы в пустом пространстве и был убежденным
сторонником механического детерминизма. По нашему философскому словарю (1963,
статья «Детерминизм», с. 121), Лаплас считал, «что значения координат и
импульсов всех частиц во Вселенной в данный момент времени однозначно
определяют ее состояние в любой прошедший или будущий момент. Так понятый
детерминизм ведет к фатализму, принимает мистический характер и фактически
смыкается с верой в божественное предопределение». Как видим, наши официальные
философы находят поповщину и в классическом изречении Лапласа, но невозможно
понять, что они дают взамен. Но в мировоззрение Лапласа входит не только
механический детерминизм, но и, следом за Ньютоном, принятие принципа
всемирного тяготения, т. е. действия на расстоянии материальных тел. С точки
зрения механики это совершенный абсурд: как может тело действовать там, где его
нет? А к як. лее Ньютон? Ньютон это отлично понимал. В письме к Бентлею, автору
лекций по опровержению атеизма, Ньютон пишет (С. И. Вавилов. Исаак Ньютон. М.,
1961. С. 129): «Предполагать, что тяготение является существенным, неразрывным
и врожденным свойством материи, так что тело может действовать на другое на
любом расстоянии в пустом пространстве, без посредства чего-либо передавая
действие и силу, --- это, по-моему, такой абсурд, который немыслим ни для кого,
умеющего достаточно разбираться в философских предметах. Тяготение должно
вызываться агентом, постоянно действующим по определенным законам. Является ли,
однако, этот агент материальным или нематериальным, решать это я предоставил
моим читателям». Как указывает С. И. Вавилов дальше (с. 130), для самого
Ньютона вопрос был совершенно ясен: тяготение объясняется заполнением
пространства Богом (предшественники: Отто фон Герике и иезуит Кирхер). В
несколько скрытой форме это мнение было высказано и в «Общем поучении»
знаменитых «Математических начал натуральной философии» (с. 590, перевод А. Н.
Крылова, 1915 г.): «Бог есть единый и тот же самый Бог всегда и везде. Он
вездесущ не по свойству только, но по самой сущности, ибо свойство не может
существовать без сущности. В нем все содержится и все вообще движется, но без
действия друг на друга. Бог не испытывает воздействия от движущихся тел,
движущиеся тела не испытывают сопротивления от вездесущия Божия». В примечании
Ньютон ссылается на древних авторов: Пифагора, Фалеса, Анаксагора, Филона,
Арата, а также приводит ряд текстов из Библии. Богословские взгляды Ньютона не
были каким-то странным привеском к его научным взглядам, они пронизывали и его
научные теории. Он не был деистом, принявшим только первый толчок, а затем
исключительное действие естественных законов, он был ближе к взглядам
Мальбранша, для которого все происходящее является сплошным чудом. Ко времени
Лапласа к абсурдному с точки зрения механики принципу всемирного тяготения
успели привыкнуть благодаря исключительной плодотворности этого принципа, а
привыкнуть можно к любому абсурду и к непонятным вещам. Если бы у Лапласа не
было «убеждения чувства» --- его механического материализма, то он должен бы
был ответить Наполеону так: «Государь, великий Ньютон ввел Бога в свою книгу
для объяснения принципа всемирного тяготения, который я полностью использовал в
своей работе. Но Ньютон, будучи свободомыслящим ученым, не навязывал своего
взгляда другим и не запрещал искать материальных агентов всемирного тяготения.
Я, правда, не нашел таких агентов, но, надеюсь, кому-нибудь это удастся,
поэтому я и не упоминал о нематериальных факторах в своей книге». Сейчас мы
знаем, что эту роль до известной степени выполнил Эйнштейн в его общей теории
относительности. Нет пустого пространства, оно все наполнено физическим полем
или физическими полями (всю жизнь Эйнштейн и стремился к тому, чтобы дать
единую теорию поля). Дальнодействие исчезло, а вместе с тем и абсурд. Стало ли
все понятным? На этот вопрос я ответа дать не решаюсь. Эйнштейн, как известно,
разобрался и в непонятном факте эквивалентности инертной и тяжелой массы. Этот
непонятный факт в силу своей привычности не смущал подавляющее большинство
физиков.

3. Противоречивое, антиномичное

Это тоже приближается к абсурду, когда об одном и том же мы можем высказать с
одинаковым правом два прямо противоположных суждения. Световой эфир, как
известно, обладал такими свойствами, которые не могут быть приписаны
одновременно никакому телу, и, однако, теория светового эфира была очень
полезной, многие физики считали его существование совершенно доказанным, и наш
великий ученый Д. И. Менделеев всерьез считал возможным рассматривать его как
один из элементов периодической системы. Сейчас, как известно, Эйнштейн
упразднил эфир в специальной теории относительности и до известной степени
реабилитировал это понятие в совершенно ином понимании (лишенном механических
свойств) в общей теории относительности. Но является ли непротиворечивость
обязательным свойством научного мышления? Мы знаем, что великий философ Кант в
свою

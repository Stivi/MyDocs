НАУКА И РЕЛИГИЯ

ПОСТАНОВКА ВОПРОСА

Вопрос об отношении науки и религии имеет по крайней мере двухвековую давность,
а правильнее, может быть, даже двухтысячелетнюю, и то решение, которое
наметилось примерно полтора века тому назад, многим из современных
интеллигентов кажется окончательной истиной в последней инстанции. Оно всего
лучше отображено в знаменитом коротком разговоре великого ученого Лапласа и
выдающегося государственного деятеля Наполеона Бонапарта. Когда Наполеон
(который, как известно, был хорошо знаком с высшей математикой) ознакомился с
известным сочинением Лапласа о небесной механике, он задал ему вопрос, почему
тот не упоминает о Боге в этой книге. Наполеон, очевидно, намекал на Ньютона,
которые закончил свои великие «Математические начала натуральной философии»
идеологическими рассуждениями. «Государь, я не нуждался в этой гипотезе», ---
ответил Лаплас (цитирую по Энгельсу, Диалектика природы, с. 268). Наполеон
задал свой вопрос не потому, что он был верующим, но он перешел от
преследования католической церкви к конкордату с целью упрочить свое положение.
Религия не нужна науке, но нужна даже атеистическому деспоту как «незримая
паутина» (прекрасное выражение Горького) для более легкого подчинения
угнетенных масс, как опиум для народа. Вспомним преследование инакомыслящих,
Галилея, Дж. Бруно, инквизицию, индекс запрещенных книг, сопротивление
эволюционной теории Дарвина, и как будто придется сделать заключение, что
религия не только бесполезна, но и вредна для развития науки. Мнение
просветителей конца XVIII и начала XIX в., казалось, торжествует в течение
всего XIX и начала XX в. В биологии телеологии был нанесен сокрушающий удар Ч.
Дарвином, в конце XIX в. был сильный процесс «дехристианизации» Франции,
Бисмарк объявил «культуркампф» против католической церкви в Германии, наконец,
в XX в. правительства почти одной трети человечества стали откровенно
атеистическими. Подрастающая молодежь этих стран практически ничего не знает о
религии. Идеи социализма и атеизма считаются неразрывно связанными, и поскольку
деятели социализма выставляют великолепный идеал осуществления справедливого
строя на земле (в противовес обещанному церковью Царствию Божию на небе после
смерти), то эти социалистические идеалы считаются морально обязательными, а
следовательно, морально обязательным является и атеизм, и всякая «поповщина»,
легко приводящая к тем или иным формам теологии, отметается «с порога». Поэтому
обязательность атеистического обучения молодежи, запрещение религиозной
пропаганды и свободного издания религиозных книг не кажется многим ограничением
свобод вообще и свободы совести в частности, так как для многих честных
образованных и умных людей всякая религия кажется пережитком прошлого, подобным
каннибализму, обычаю убивать стариков, колдовству, гаданиям по звездам,
«свободе» заводить обширные гаремы, геноциду, учению о низших расах и проч., и
восстановление совершенно недопустимых с моральной точки зрения обычаев:
истребление пленных, стариков, душевнобольных и вообще «неполноценных»,
проводимое гитлеровцами, конечно, не делает гитлеровскую Германию более
«свободной» страной, чем страны антигитлеровской коалиции, так как свобода
истреблять или развращать себе подобных (торговля наркотиками, порнографической
литературой и проч.) не может считаться свободой, достойной прогрессивного
человечества. Не всякая свобода и желательна, как говорится в одном коротком
диалоге еще дореволюционного времени: «Извозчик, ты свободен?» --- «Свободен». ---
«Кричи: „Да здравствует свобода!"» Свобода пропаганды суеверий, распространения
наркотиков и т. п. не есть та свобода, за которую есть смысл бороться.

Неудивительно, что в нашей стране, отставшей в силу пережившего себя
общественного строя от западных стран, особенно силен был антирелигиозный дух
среди нашей интеллигенции, несмотря на то что до революции преподавание в
школах в значительной степени было проникнуто религиозным духом (разных
религий), в каждом паспорте обозначалась религиозная принадлежность носителя
паспорта (национальность в паспортах не фигурировала) и не допускалось, чтобы
кто-либо объявил себя атеистом. Протест против религии носил исключительно
резкий характер. Ленин считал религию вообще одним из отвратительнейших явлений
в обществе и всякую самую утонченную форму фидеизма он решительно отбрасывал «с
порога». Даже субъективный идеализм объявлялся «поповщиной», и по отношению к
самым близким людям Ленин не допускал никакого компромисса. Его собственный
отец был глубоко религиозным человеком, как и отец Чернышевского, и оба они
были очень почтенными людьми. Религия была тем более ненавистна нашим
революционным демократам, что она тормозила развитие народа. Поэтому все они
переходили от глубокой религиозности в юности к воинствующему атеизму зрелого
возраста (Салтыков-Щедрин, Писарев, Чернышевский и др.). Но воинствующий атеизм
был свойствен не только революционерам или марксистам, но и лицам, далеко
стоящим от революционной борьбы и марксизма. Мне известен один ученый
выдающихся умственных и моральных качеств, весьма скептически относившийся к
селекционизму (неодарвинизму, т. е. учению о ведущей роли естественного
отбора), но вместе с тем считавший, что дарвинизм принес пользу как мощное
антирелигиозное учение. Религия в глазах этого ученого и честного человека была
настолько отрицательным явлением, что для борьбы с ней можно было использовать
даже ложные учения. Подобно тому как в христианском песнопении поется про
Христа, что он «смертию смерть поправ», так и здесь ложью следует победить
другую, более страшную ложь. «Цель оправдывает средства» --- этот лозунг
употребляется и людьми, весьма отрицательно относящимися к иезуитам.

И не только у нас, но и среди ученых и мыслителей Запада сейчас есть немало
представителей, полностью солидаризующихся с тем мнением, что религия есть зло,
только мешавшее прогрессу человечества, и что хотя среди великих ученых и
философов мы знаем очень много истинно религиозных людей, они сделались
великими учеными не благодаря, а вопреки своей религиозности, и что пропасть
между религией и наукой непроходима. Такую точку зрения развивают и лица,
весьма враждебно относящиеся к коммунизму, например Бертран Рассел. Его
неприязнь к коммунизму доходила до того, что он (пока Запад обладал монополией
атомной бомбы) был не против того, чтобы использовать преимущество Запада для
политических целей, но он же (сейчас --- видный представитель движения за мир)
сотрудничает в нашем журнале «Наука и религия» и прямо заявляет (в полном
согласии, например, с нашим Луначарским), что ученый и прогрессивный человек
вообще не только может, но и должен быть антирелигиозным человеком. Рассел не
одинок в своем высказывании. В 1961 г. появился под общей редакцией Юлиана
Гексли сборник статей 27 авторов под названием «Структура гуманизма»,
подвергнутый подробной рецензии в британском журнале «Философия науки» (1963,
том XIV, № 53, с. 41-53) Аланом Стюартом. Рецензент приветствует этот сборник
как «новое откровение», как веху в деле эмансипации человеческого духа, как
обоснование нового эволюционного гуманизма, полностью отвергающего все старые
религиозные суеверия, всякую сверхъестественную религию. И несмотря на высокую
оценку сборника, Стюарт отмечает очень крупный дефект, заключающийся в том, что
эти новые гуманисты одиннадцать раз цитируют иезуита Тейяра де Шардена даже с
указанием на желательность знакомства с этим автором. Рецензент негодует (с.
52): «Многие образованные честные искатели истины, которые знают кое-что о
последних двух тысячах лет человеческой истории, не могут избежать связывать
„священника" (попа) с „поповщиной", а „поповщину" с „лживостью". Здесь же мы
видим испорченную книгу, которая могла бы быть превосходной. Потому что,
несмотря на все их изъявления о глядении в будущее, эти гуманисты, подобно жене
Лота, тоскующей по Содому, глядят назад». Я затруднялся в переводе слов
«priest» и «priestcraft» и вставил вместо «priestcraft» «поповщину». Последний
термин широко применяется в советской литературе, но он почему-то отсутствует в
трех новых словарях: русско-английском и русско-немецком 1952 г. и
русско-французском 1962 г. Но мы видим, что рецензент, предлагая нам смотреть
только вперед и не оглядываться, или забыл, или игнорирует известное положение
о диалектике, где новый синтез в известном смысле возвращается к старому.

Но наряду с таким резким отношением ко всякой «поповщине» мы имеем гораздо
более примирительное отношение. Добржанский, очень близкий по своим
биологическим взглядам к Ю. Гексли, заканчивает свою интересную книгу об
эволюции человечества цитатой из Тейяра де Шардена и считает, что эволюционная
идея иезуита является лучом надежды для человечества. Он считает, что его книга
содержит и науку, и метафизику, и теологию, и даже поэзию, испорченную
несколько в английском переводе. Предисловие к русскому переводу
книги Тейяра де Шардена «Феномен человека» написано выдающимся, одним из
наиболее культурных французских коммунистов, Роже Гароди, который, наряду с
критикой многих положений Шардена, высоко оценивает многие прогрессивные
стороны его взглядов. Исчезает положение «двух лагерей»: с одной стороны,
реакционный черный лагерь защитников религии, состоящий из честных невежд или
фанатиков и бесчестных эксплуататоров, а с другой --- светлый лагерь
«прогрессистов», куда относятся все умные и честные люди, «свободомыслящие»,
или, как это понималось на языке XIX в., атеисты и социалисты. Все
перемешалось. По отрицательному отношению к религии Ленин оказался в одной
компании с Б. Расселом и цитированным рецензентом Стюартом, более умеренный
Луначарский, говоривший о «религиозном атеизме», о «новой религии социализма»,
сочувствующий неодарвинист Добржанский и, наконец, коммунист Гароди: для
каждого оттенка отношения к религии мы можем подыскать представителей обоих
лагерей. И здесь, как может быть всюду, господствует в высокой степени
комбинационный принцип.

Сейчас в мировом масштабе происходят два встречных течения. Наряду с успехами
антирелигиозности в мировом масштабе, падением посещаемости храмов и проч. мы
имеем укрепление религиозных позиций в ряде культурных стран. Во Франции
возвращены изгнанные на пороге столетия иезуиты и во главе правительства стоит
верующий католик. Ряд правительственных партий во Франции, ФРГ, Италии, Бельгии
и других странах откровенно христианские. Можно ли это объяснить только
империалистической реакцией? Два выдающихся деятеля XX в. --- Ганди и Кеннеди ---
не скрывали своей религиозности, и оба пали от рук людей, сходных по идеологии
с фашизмом, империализмом, расизмом и прочими бесспорно ретроградными
идеологиями. Защитниками религии или «поповщины» в широком смысле слова (причем
самого разнообразного характера) выступают самые передовые ученые
современности: Эддингтон, Эйнштейн, Гейзенберг, Планк, Шредингер и многие
другие меньшего значения. Писатели Сент-Экзюпери, Веркор и другие настаивают на
необходимости синтеза, а не голого отрицания. Растет число идеалистов самых
разнообразных направлений, в то время как материалистическая философия скорее
обнаруживает явные признаки загнивания. Мы знаем, что в нашей стране под видом
борьбы с религией и идеализмом систематически боролись со всеми новыми
течениями в науках: теория относительности, принцип неопределенности, теория
расширяющейся Вселенной, теория резонанса в химии, настоящая генетика. Даже
там, где поддерживали то или иное здравое направление (например, учение об
условных рефлексах Павлова), так догматизировали учение, что оказали ему
медвежью услугу. Все же новшества, предложенные под видом истинно
материалистической науки, не выдержали испытания временем, и притом короткого
времени. Таким образом практически опровергнут подход: раз это ведет хотя бы в
слабой степени к «поповщине», это надо отвергнуть «с порога».

Не менее неожиданными были и события на этическом и политическом фронтах.
Ставилось в вину всем религиозным правительствам, что они наряду с
«запугиванием адом» имели на вооружении и смертную
казнь. Запугивание адом у нас исчезло, но смертная казнь фигурирует в нашем
Уголовном кодексе в таком числе статей, как, я думаю, нигде в капиталистических
странах, а некоторые капиталистические страны (мне известны: ФРГ, Мексика,
Уругвай, Израиль до процесса Эйхмана) вовсе не имеют смертной казни в мирное
время. В политике «и сольются в одно все народы в вольном царстве святого
труда» пока обернулось жесточайшим конфликтом между двумя крупнейшими
коммунистическими партиями: СССР и Китая.

Если пока между Китаем и СССР еще нет вооруженного конфликта, то только потому,
что Китай еще недостаточно силен и раздираем внутренними противоречиями, а у
нас неслыханная в царские времена милитаризация и достаточная удовлетворенность
последними захватами. Но разговор между двумя «социалистическими» странами по
тону вовсе не отличается от разговоров двух милитаристских империалистических
держав. Но вопросы этики и политики уже выходят за пределы темы настоящей
статьи.

Мы видим, таким образом, что в отношении религии сейчас наблюдаются два
противоположных процесса: 1) антирелигиозный, касающийся масс и среднего уровня
интеллигенции; 2) прорелигиозный, выражающийся в обильном числе всевозможных
направлений, часто затрагивающих самые высокие уровни передовых мыслителей
современности. Полезно выяснить, является ли это второе направление вызванным
какими-либо новыми открытиями или явлениями современности или можно найти и в
материалистическом мировоззрении начала XIX в. такие черты, которые ясно
указывали на его несовершенство. Так ли ясно все в отношении разговора Лапласа
с Наполеоном? К этому и перейдем.

Глава I. НАУКА --- ВРАГ СУЕВЕРИЙ И ЧУДЕСНОГО

Дать определение науки не так-то легко, для наших целей достаточно ограничиться
тем, что ясно сквозит в словах Лапласа. Наука --- враг всякого суеверия,
чудесного, а религия, напротив, основана на суевериях и принятии чудесного.
Всего лучше это выражено в знаменитых словах блаженного Тертуллиана: «кредо,
квиа абсурдум ест», т. е. «верю, потому что это абсурдно». Пока не будем
разбирать вопроса, является ли высказывание Тертуллиана ортодоксальным или
типичным для всех христиан. Ограничимся тем, что уже в раннем христианстве
обозначилось направление, резко противополагавшее религиозное учение тогдашней
классической науке (см., например, История философии, 1941, № 1, с. 387-388) и
в известном смысле сохранившееся и в более поздние времена. Каков смысл
высказывания Тертуллиана? В учении о познании Тертуллиан придерживался
вульгаризированного стоического материализма и считал, что все действительно
существующее телесно, в том числе Бог и бессмертная душа. Таким образом, его
взгляд вовсе не является выражением его идеализма, а напротив, теснейшим
образом связан с материалистическим характером его мировоззрения. Разумом он
был материалист, но, как христианин, принимал и такие явления, которые
материалистическому объяснению
совершенно не поддавались, но так как они были для него совершенно бесспорны,
то в них он верил как в абсурд с точки зрения материализма. Это была не слепая,
а сознательная вера, и он твердо проводил различие между верой и знанием. А так
как для него на первом плане стояли истины веры, то естественно, что к
человеческой мудрости он относился с меньшим уважением и порой даже с
презрением. Но посмотрим, свободен ли материализм XIX в. от принятия чудесного
и непонятного. Смысл науки в том понимании, которое нас сейчас интересует,
заключается в том, что она должна бороться с суевериями, что можно понимать в
пяти формах: 1) чудесное, т. е. непонятное, 2) абсурдное, 3) противоречивое, 4)
сверхъестественное и 5) невероятное. Разберем по очереди.

1. Непонятное, часто необычное

Все непривычное, неожиданное, непонятное нам кажется чудом, и мы и в практике
жизни, и в науке стремимся к тому, чтобы непонятное свести к чему-то понятному.
Антропоморфические религии сводят все «чудесное» к вмешательству существа,
подобного человеку, но невидимого. Так толковались изумительные приспособления
органического мира. Дарвин оставил телеологию, т. е. мнение, что основа в
биологии --- явление приспособления, но вместо невидимого бога ввел тоже
невидимое явление --- естественный отбор, приемлемый для материалистов. Задачей
науки оказалось дать простое объяснение сложным явлениям. Это направление имеет
большую давность. Уже цитированный Стюарт (Брит. Журн. филос. науки, 1963, т.
XIV, № 53, с. 53) по поводу нового, эволюционного гуманизма пишет: «Это ново,
ново в истории так называемой западной или греко-христианской цивилизации.
Конечно, были и раньше люди, которые мыслили свободно --- в общем так же, как и
сегодняшние гуманисты. В дохристианскую эру были Эпикур и Лукреций с их
последователями». Можно ли считать Эпикура одним из лидеров античной науки?
Вряд ли. Со взглядами Эпикура можно познакомиться по известному античному
историку философии Диогену Лаэрцию (я использовал немецкий перевод). Диоген
Лаэрций --- добросовестный, но малокритический историк, который сообщает разные
небылицы про многих философов, но Эпикуру он посвятил, как и Зенону-стоику,
наибольшее место в своей книге (по 68 страниц) и, в отличие от других
философов, привел много текстов. Хотя обычно Эпикура считают представителем
линии Демокрита, но он резко отличается от Демокрита ярко выраженным
индетерминизмом. Основа мировоззрения Эпикура: боги не вмешиваются в нашу
судьбу, не вознаграждают и не карают, смерть ни к чему плохому не приводит. Это
и есть основа подлинного эпикуреизма: безмятежность души, атараксия. У самого
Эпикура это не приводит к безнравственности: напротив, многие из его положений
сходны с таковыми стоиков, он обосновывает этику утилитарными соображениями,
что вовсе не так плохо, но что дает его мировоззрение науке? Здесь как будто
проявляется полное свободомыслие. В вопросе о величине Солнца, о движении
планет, знамениях, громе и других явлениях он выставляет различные гипотезы и
не дает предпочтения ни одной. Они все с его точки зрения равноценны, при
условии, чтобы не было мифов, связанных с религией. Но будучи совершенно
беззаботным в области научных гипотез, Эпикур крайне догматичен в онтологии или
метафизике и против всякой диалектики. Но, может быть, это и есть то полезное
ограничение свободы, которое необходимо для ученого, чтобы он сосредоточил свои
усилия, основываясь на определенных бесспорных аксиомах. Нет, ограничиваясь
объяснениями и не придавая им принудительного значения, Эпикур чрезвычайно
презрительно относится к терпеливым усилиям античных астрономов, старавшихся
длительными наблюдениями выяснить законы движения небесных тел, т. е.
положивших начало математическому описанию явлений природы. Переводчик и
комментатор Апельт правильно пишет, что если бы такая точка зрения
восторжествовала, естествознание никогда бы не вышло из исходного состояния.

Последователями Эпикура в смысле необходимости в первую очередь «объяснения», а
не математического «описания» явлений был, конечно, Лукреций и все эпикурейцы.
А так как в Римском государстве в философии господствовали стоики (обращавшие
внимание только на этику) и эпикурейцы, то становится понятным тот ничтожный
вклад в науку, который сделала могущественная Римская империя.

Но может быть, учения Эпикура (и Лукреция, который не дал, кажется, ничего
оригинального по сравнению с Эпикуром) были в политическом отношении
прогрессивны? Он был гуманный человек и рекомендовал доброе отношение к рабам,
но в смысле общественной идеологии был прототипом «премудрого пескаря»: не
стоит жениться и иметь детей, не занимайся государственной деятельностью, как
бы чего не вышло, пребывай в мудрой атараксии и не размышляй о возможности
преобразований.

Если мы посмотрим всю историю человеческой мысли, то убедимся, что решительно
все строители «утопий» были идеалисты, что же касается материалистов, атеистов
и антирелигиозников, то они или сами были тиранами (Критий), или защищали
абсолютизм под разными видами --- «просвещенного абсолютизма» и т. д. (Гоббс,
Дидро и пр.). Все идеологи революций вплоть до XIX в. работали под
идеалистическими знаменами и в общем сделали немало для преобразования
общества. Поэтому слова Маркса: «Философы только объясняли мир, а его
необходимо перестроить» (точно слова не помню) надо понимать:
«Материалистические философы только объясняли мир, а сейчас им надо приняться
за перестройку», и тогда он будет звучать более или менее правильно, если
оставить в стороне вопрос (сейчас оставим его без рассмотрения), в какой мере
революционный марксизм может считаться чисто материалистическим учением.
Несомненно, что современный селектогенез, т. е. эволюционное учение, восходящее
к Дарвину, совершенно пропитан эпикурейским духом. Основной императив
дарвинской морфологии: дай какое-нибудь «причинное» и «механическое» объяснение
структуре, которое можно свести к действию естественного отбора, с тем чтобы
устранить «поповщину», «платонизм» и прочие вредные учения. Неважно, что это
объяснение не является механическим или причинным в смысле точных наук, важно
заглушить сомнения
в отсутствии целеполагающих начал в природе. К математическому толкованию формы
и системы это направление, естественно, глубоко враждебно. Но это направление
поддерживают и выдающиеся математики! Верно, но об этом придется сказать
несколько слов в разделе о невероятном.

Простое объяснение есть низший этап развития научного мышления, и если это
направление доминирует, то оно притупляет то, что можно назвать научной
бдительностью, удивлением перед новыми фактами, и становится подлинным опиумом
для науки.

Только что рассмотренный пункт к Лапласу не относится, так как он был
представителем точной науки, и когда давал объяснения (например, в его
знаменитой космогонической гипотезе), он оговаривал ненадежность этой гипотезы,
так как ее математической теории он дать не мог.

2. Абсурдное, т. е. нелепое

А вот абсурда Лаплас не избежал. Он был убежден, что в мире существуют как
конечные реальности одни атомы в пустом пространстве и был убежденным
сторонником механического детерминизма. По нашему философскому словарю (1963,
статья «Детерминизм», с. 121), Лаплас считал, «что значения координат и
импульсов всех частиц во Вселенной в данный момент времени однозначно
определяют ее состояние в любой прошедший или будущий момент. Так понятый
детерминизм ведет к фатализму, принимает мистический характер и фактически
смыкается с верой в божественное предопределение». Как видим, наши официальные
философы находят поповщину и в классическом изречении Лапласа, но невозможно
понять, что они дают взамен. Но в мировоззрение Лапласа входит не только
механический детерминизм, но и, следом за Ньютоном, принятие принципа
всемирного тяготения, т. е. действия на расстоянии материальных тел. С точки
зрения механики это совершенный абсурд: как может тело действовать там, где его
нет? А как же Ньютон? Ньютон это отлично понимал. В письме к Бентлею, автору
лекций по опровержению атеизма, Ньютон пишет (С. И. Вавилов. Исаак Ньютон. М.,
1961. С. 129): «Предполагать, что тяготение является существенным, неразрывным
и врожденным свойством материи, так что тело может действовать на другое на
любом расстоянии в пустом пространстве, без посредства чего-либо передавая
действие и силу, --- это, по-моему, такой абсурд, который немыслим ни для кого,
умеющего достаточно разбираться в философских предметах. Тяготение должно
вызываться агентом, постоянно действующим по определенным законам. Является ли,
однако, этот агент материальным или нематериальным, решать это я предоставил
моим читателям». Как указывает С. И. Вавилов дальше (с. 130), для самого
Ньютона вопрос был совершенно ясен: тяготение объясняется заполнением
пространства Богом (предшественники: Отто фон Герике и иезуит Кирхер). В
несколько скрытой форме это мнение было высказано и в «Общем поучении»
знаменитых «Математических начал натуральной философии» (с. 590, перевод А. Н.
Крылова, 1915 г.): «Бог есть единый и тот же самый Бог всегда и везде. Он
вездесущ не по свойству только, но по самой сущности, ибо свойство не может
существовать без сущности. В нем все содержится и все вообще движется, но без
действия друг на друга. Бог не испытывает воздействия от движущихся тел,
движущиеся тела не испытывают сопротивления от вездесущия Божия». В примечании
Ньютон ссылается на древних авторов: Пифагора, Фалеса, Анаксагора, Филона,
Арата, а также приводит ряд текстов из Библии. Богословские взгляды Ньютона не
были каким-то странным привеском к его научным взглядам, они пронизывали и его
научные теории. Он не был деистом, принявшим только первый толчок, а затем
исключительное действие естественных законов, он был ближе к взглядам
Мальбранша, для которого все происходящее является сплошным чудом. Ко времени
Лапласа к абсурдному с точки зрения механики принципу всемирного тяготения
успели привыкнуть благодаря исключительной плодотворности этого принципа, а
привыкнуть можно к любому абсурду и к непонятным вещам. Если бы у Лапласа не
было «убеждения чувства» --- его механического материализма, то он должен бы
был ответить Наполеону так: «Государь, великий Ньютон ввел Бога в свою книгу
для объяснения принципа всемирного тяготения, который я полностью использовал в
своей работе. Но Ньютон, будучи свободомыслящим ученым, не навязывал своего
взгляда другим и не запрещал искать материальных агентов всемирного тяготения.
Я, правда, не нашел таких агентов, но, надеюсь, кому-нибудь это удастся,
поэтому я и не упоминал о нематериальных факторах в своей книге». Сейчас мы
знаем, что эту роль до известной степени выполнил Эйнштейн в его общей теории
относительности. Нет пустого пространства, оно все наполнено физическим полем
или физическими полями (всю жизнь Эйнштейн и стремился к тому, чтобы дать
единую теорию поля). Дальнодействие исчезло, а вместе с тем и абсурд. Стало ли
все понятным? На этот вопрос я ответа дать не решаюсь. Эйнштейн, как известно,
разобрался и в непонятном факте эквивалентности инертной и тяжелой массы. Этот
непонятный факт в силу своей привычности не смущал подавляющее большинство
физиков.

3. Противоречивое, антиномичное

Это тоже приближается к абсурду, когда об одном и том же мы можем высказать с
одинаковым правом два прямо противоположных суждения. Световой эфир, как
известно, обладал такими свойствами, которые не могут быть приписаны
одновременно никакому телу, и, однако, теория светового эфира была очень
полезной, многие физики считали его существование совершенно доказанным, и наш
великий ученый Д. И. Менделеев всерьез считал возможным рассматривать его как
один из элементов периодической системы. Сейчас, как известно, Эйнштейн
упразднил эфир в специальной теории относительности и до известной степени
реабилитировал это понятие в совершенно ином понимании (лишенном механических
свойств) в общей теории относительности. Но является ли непротиворечивость
обязательным свойством научного мышления? Мы знаем, что великий философ Кант в
свою
классическую «Критику чистого разума» включает «диалектику чистого разума», где
видное место занимают антиномии чистого разума, т. е. такие пары
противоположных суждений, где можно доказать нелепость каждой из
противоположностей. Две наиболее известных: о конечности и бесконечности
пространства и времени и о необходимости и свободе. Кант считал
сформулированные им антиномии принципиально неразрешимыми и в этом видел
границу человеческого разума, все же остальное познание он считал возможным
свести к непротиворечивому виду и полагал, что некоторые теории познания, в
частности формальная логика, уже этого уровня достигли. Он не сомневался в
абсолютной достоверности математических аксиом и теорий. Существование Бога он
считал недоказуемым и неопровержимым, но склонен был строить религию исходя из
этических соображений (Религия в пределах только разума).

Против критической философии Канта со всей решительностью выступил другой
великий немецкий философ Гегель. В своей речи 22 октября 1818 г. (соч. Гегеля,
т. I, 1929, с. 15) он пишет: «Наконец так называемая критическая философия дала
этому неведению вечного и божественного возможность придерживаться этой позиции
с чистой совестью, так как эта философия уверяет, будто ей удалось \textit{доказать}
(курсив Гегеля. --- \textit{А.Л.}), что мы ничего не можем знать относительно вечного и
божественного. Это мнимое познание даже дерзнуло присвоить себе название
философии, и ничего не могло быть желаннее для поверхностных умов и характеров,
ничто не было столь охотно принято ими, как это учение о незнании, благодаря
которому их собственная поверхность и пустота оказывались чем-то превосходным,
желанной целью и результатом всех интеллектуальных усилий. Что мы не знаем
истины и что нам дано знать одни случайные и преходящие, т. е. \textit{ничтожные},
явления, вот то \textit{ничтожное} учение, которое делало и делает наиболее шума и
которое господствует теперь в философии». Из этой цитаты ясно, что утверждение
о непознаваемости мира, которое наши казенные философы инкриминируют всем
идеалистам, явно неприложимо к такому выдающемуся философу, как Гегель,
который, конечно, не меньший, а больший идеалист, чем Кант.

Как же относится к антиномиям Гегель? Отрицает ли он их? Напротив (там же, с.
97), он упрекает Канта за то, что тот перечисляет только четыре антиномии,
тогда как, по мнению Гегеля, антиномии встречаются во всех предметах всякого
рода, во всех представлениях, понятиях и идеях. Разрешение противоречия состоит
в том, что оно принадлежит не предмету самому по себе, а лишь познающему
разуму. Следовательно, в своем развитии разум снимает противоречие (синтез) с
тем, чтобы перейти к новому противоречию, вновь снимаемому, и так далее на
бесконечном пути стремления к абсолютной истине.

По-видимому, общий взгляд Гегеля сейчас торжествует в науке. Аподиктическая
достоверность евклидовой геометрии разбита трудами Лобачевского, Римана и др.,
антиномия конечности и бесконечности пространства снята в общей теории
относительности, где пространство оказывается (в духе Римана) ни конечным, ни
бесконечным, а безграничным. Развитие математики и физики шло, по-видимому,
самостоятельным путем, независимо от Гегеля, но сейчас многие выдающиеся
мыслители, занимающиеся историей и философией науки, пришли к утверждениям,
очень сходным с цитированными мыслями Гегеля. Так, Дюгем в своей замечательной
книге «Физическая теория, ее цель и строение» утверждает, что «экспериментум
круцис» (эксперимент креста, где опровержение одной стороны доказывает
справедливость ей противоположной) вещь в физике невозможная, т. к. в истории
физики неоднократны случаи, где при споре по поводу какого-нибудь вопроса
оказывалось ложным не то, что оспаривала одна из спорящих сторон, а то, в чем
не сомневались обе спорящих стороны. Примерно то же показывает в биологии и
Радль в его замечательной истории биологических учений.

В этом и заключается отрицание безусловной значимости закона исключенного
третьего, одного из столпов формальной логики, подлинной диалектической
логикой. Отрицание закона исключенного третьего лежит и в основе математической
школы интуитивизма. Она опирается и на математические факты. Кажется очевидным,
что могут быть сходящиеся или расходящиеся ряды. А нашли такие ряды, которые не
являются ни сходящимися, ни расходящимися.

Совершенно прав Гегель, что антиномичность, противоречивость пронизывает все
наше мышление, но мы с этим не должны примиряться, а работать над преодолением
этих противоречий. Разумеется, из того, что все реальное противоречиво, не
значит, что все противоречивое реально и заслуживает рассмотрения. Надо
различать между противоречием и бессмыслицей: по-немецки это звучит видерштрейт
и видерзинн, но надо сказать, что отличить бессмыслицу от противоречия не
всегда бывает легко. Одним из излюбленных доказательств бессмыслицы религиозных
учений был догмат христианской церкви, что Бог един, но троичен в лицах. Явная
«бессмысленность» этого догмата послужила причиной возникновения многих ересей,
в частности как будто и причиной гибели Сервета. В самом деле: 3x1 = 1, явная
бессмыслица. Сейчас разрешение этого противоречия лежит в основе теории
множеств даже гениального Георга Кантора, который был убежденным католиком.
Приведенная формула нелепа только в области конечных величин. Для бесконечных
множеств соединение двух или нескольких множеств одинаковой мощности в одной
дает множество той же мощности. Например, множества всех четных и всех нечетных
вместе дают множество натуральных чисел. Наша таблица умножения неприменима к
бесконечному. Кстати, зачатки теории множеств имеются уже у Галилея.

4. Сверхъестественное

Это то, что не заключает внутренних противоречий, вполне понятно, но которое
выходит за пределы человеческого опыта. Недопущение сверхъестественного
является как будто основой всякого научного мышления. Но одним из как будто
совершенно достоверных выводов человеческого опыта будет: «Ничто не вечно под
луной», «все течет» по Гераклиту. И однако ученые и философы с каким-то
необыкновенным упорством ищут «покоящуюся ось в потоке явлений» будучи глубоко
убежденными (тут они сознательно или бессознательно следуют философу
Пармениду), что все истинно сущее неизменно. Одно из явных проявлений --- атомная
теория, принимающая, что, несмотря на непрерывные кажущиеся изменения тел, по
существу они неизменны. Но ведь это же противоречит всему нашему опыту. Мы
знаем, что самые твердые тела от трения изнашиваются, а тут мельчайшие частицы
двигаются нередко с большой быстротой, сталкиваются и вечно остаются
неизменными. Это совершенно сверхъестественно, но мы к этому привыкли, а
известно, что можно привыкнуть к любому абсурду. Не так думали великие
мыслители прошлого. Возьмем опять Ньютона (Каблуков. Ньютон как химик. «Под
знаменем марксизма», 1937, № 4, с. 205): «При размышлении о всех этих вещах, ---
говорит Ньютон, --- мне кажется вероятным, что вначале Бог сотворил материю в
виде твердых, непроницаемых, подвижных, обладающих массой частиц таких размеров
и форм, с такими свойствами и в таких относительных количествах, какие пригодны
для той цели, для которой он их создал; эти первоначальные твердые частицы
несравненно тверже, чем какое бы то ни было пористое тело, составленное из них;
они так тверды, что никогда не снашиваются и не раздробляются на части, ибо
обыкновенная сила не способна разделить то, что сам Бог сделал единым при
первом творении». Неудивительно, что творцы новой атомной теории или духовные
лица (священник Гассенди, иезуит Боскович), или искренне религиозные люди
(квакер Дальтон, протестант Ньютон, настроенный резко антикатолически) и в
античной философии идеалистические философы (Пифагор, Платон) отнюдь не
отрицали атомизма. Первая математическая атомная теория Босковича, конечно,
ближе к Пифагору, чем к Демокриту. А Демокрит? Разве не связан
материализм теснейшим образом с атомной теорией? Конечно нет, а лишь с
пониманием атомов и той ролью, которая им приписывается в мироздании. По
вопросу дискретного строения тел в античном мире, видимо, не было резких
противоречий. Возьмем того же Диогена Лаэрция. Демокрит, по Диогену Лаэрцию,
был почитателем пифагорейцев и Пифагора, и Трасилл, который свел все сочинения
Демокрита, собрал также и все сочинения Платона. Разница заключается в том, что
для Демокрита атомы были конечной реальностью, а для Платона лишь кирпичами
видимой реальности. Демокрит был совершенно чужд холизму (целое определяет
поведение частей): развитие мира --- следствия случая, а не имманентного закона;
отрицание того, что позднее называлось финальными и формальными причинами.
Позднейший механический материализм усвоил от Демокрита принятие механической
необходимости (в этом смысле он отрицает случайность в природе в противовес
индетерминисту Эпикуру) наряду с отрицанием финальной причинности (в этом
смысле, говоря словами Данте, «вот тот, кто мир случайным полагает, философ
знаменитый Демокрит»).

Вот эти положения, существенные для истинного материализма, действительно
лишены всякой «сверхъестественности», но признание инвариантных атомов могло
зародиться только на объективно идеалистической почве. А потом привыкли
считать, что атомная теория --- цитадель материализма.

Такую же религиозную основу имеет происхождение и другого великого инварианта ---
закона сохранения энергии. Предшественниками его, как известно, были Декарт и
Лейбниц, и оба исходили из того, что Бог вложил элемент своей неизменности в
природу в форме сохранения энергии (формулировали они его не так, как сейчас,
но эта мысль и двигала искание закона). В XIX в. к инвариантности элементов
мироздания привыкли, и из двух авторов закона сохранения энергии в современном
виде Гельмгольц во всяком случае не отличался религиозностью, а другой, Роберт
Майер, был глубоко религиозным человеком. Как известно, и Р. Майер и Гельмгольц
с трудом добились того, чтобы их работы по этому закону были напечатаны; но
даже когда этот закон был признан и на съезде в Инсбруке Р. Майер позволил себе
несколько фраз в религиозном смысле (в 1869 г.), этим воспользовались его
противники, и К. Фогт в газете, намекнул, что это говорит человек, выпущенный
из дома умалишенных, где он одно время действительно был, видимо, не без
содействия родственников (Тимирязев К. А. Избр. соч. в 4 т., т. I, 1949, с.
132). Вспомним, что и Ньютона многие считали сумасшедшим.

5. Невероятное

Невероятное, точнее, чрезвычайно маловероятное. Есть известный рассказ об одном
аббате, хорошо разбиравшемся в основах теории вероятности. Он вошел в таверну,
где играли в кости. Кто-то бросавший кости получил на всех трех костях три раза
подряд по шесть очков. «Кости фальшивые!» --- воскликнул аббат и оказался прав
(их наливали свинцом с одной стороны). Мог ли он ошибиться? Конечно, мог. В
данном случае при правильных костях вероятность такого результата равна 1 :
6 в 9, т. е. 1 : 10 миллионов (приблизительно). Поэтому если бы такая серия
бросаний повторялась несколько миллионов раз, то результат не был бы
удивителен. Но практически мы с такими вероятностями не считаемся. В США
население около 200 миллионов, а погибает ежегодно от автомобилей 50 тысяч
человек, следовательно, средняя вероятность для американца погибнуть от
автомобиля в течение года примерно 1 : 4000, а каждый день приблизительно 1 : 1
500 000. Но никто же из американцев не считается серьезно с опасностью
погибнуть в ближайший день. Ясно, что ни один здравомыслящий человек не будет
планировать свое поведение из ожидания исключительно маловероятных событий. И,
однако, в науке есть такие странные люди --- большинство механических
материалистов. Второй закон термодинамики является одним из очень важных
достижений физики XIX в., но в соединении с предположением о бесконечности
Вселенной в пространстве и времени он приводит к представлению о тепловой
смерти Вселенной, которая неизбежно должна бы произойти к настоящему времени,
если бы этот закон был абсолютен. А так как Вселенная отнюдь не находится в
состоянии тепловой смерти, то придется допустить или ограниченность ее
состояния во времени (часы Вселенной были когда-то заведены), или наличие
других процессов иной направленности. Но тогда второе начало термодинамики
теряет свою универсальность. Знаменитый физик Л. Больцман (1844-1906) предложил
третий выход. Основываясь на статистическом характере второго начала
термодинамики, он (Философский словарь, 1963, с. 53) «для преодоления
идеалистической гипотезы „тепловой смерти Вселенной" выдвинул свою
флуктуационную гипотезу, согласно которой общее равновесное состояние мира в
целом постоянно и неизбежно нарушается в отдельных областях гигантскими
флуктуациями (отклонениями), приводящими к неравновесному процессу развития
отдельных миров. По своему мировоззрению Больцман был убежденным материалистом,
критиковал энергетизм и махизм». Мы знаем, что Больцман высоко ценил Дарвина и
даже высказался, что наш (XIX век) есть век механического понимания природы,
век Дарвина. Мнение Больцмана поддерживал и наш известный, исключительно
образованный физик Хвольсон. В своей статье «Можно ли прилагать законы физики
ко Вселенной» он развивает аналогичную мысль, что так как законы физики по
крайней мере, как правило, являются статистическими законами, то всегда в
бесконечной Вселенной мы можем найти такой уголок, где законы эти в силу
случайных отклонений (флуктуации) оказываются неприложимыми, все там идет
навыворот. Весь наблюдаемый нами мир есть результат накопления колоссального
количества случайностей. Для области живого уже давно (Ауэрбах) сформулирован
принцип эктропизма --- концентрации, а не рассеяния энергии. Таким образом, вся
наша Метагалактика --- огромный невероятный кусок во Вселенной, где в силу
накопления случайностей второй закон термодинамики не соблюдается, и в этом
невероятном участке Вселенной имеется еще более невероятная область --- область
живого. Все построено на теории полной невероятности. Это дает объяснение и
тому, что многие выдающиеся ученые, представители точных наук, так высоко
ценили учение о естественном отборе. Против этого учения неоднократно
выдвигались серьезные возражения, отмечавшие, что процесс эволюции на основе
накопления случайных изменений совершенно невероятен: и не хватает материала
для отбора (в особенности полового отбора), нет никакого соответствия между
темпами размножения и темпами эволюции, и совершенно невозможно себе
представить, чтобы путем накопления поломок достаточно совершенного органа
можно было получить более совершенный орган. На это многие дарвинисты говорили,
что ряд математиков или математически образованных ученых принимают это учение,
значит, математические возражения несущественны. На это можно ответить: уже в
области физики и астрономии ученые, подобные Больцману, принимают абсолютно
невероятное. Аргументы о невероятности на таких людей подействовать не могут.
Это --- Тертуллианы наизнанку. Тот говорил: «Я верю (в догматы веры), потому что
это абсурд (противоречит данным материалистического мировоззрения)». Эти
говорят: «Мы принимаем, что наш мир совершенно невероятен; но мы готовы жить в
невероятном мире, чтобы не допустить проникновения в науку идеалистических
воззрений». Оба наиболее ценным считают не свободное мышление, а подчинение
определенным догматам, и в этом смысле оба --- представители людей, сознательно
верующих в невероятное или чудесное. А те лица, которые наивно думают, что
взгляды Больцмана, Дарвина и других материалистов целиком основаны на вполне
рациональных научных данных, являются выразителями слепой веры.

6. Антиномичность в науке и религии

От противоречий, абсурдов, невероятного не свободна оказывается и наука, мнящая
себя совершенно свободной от всякого суеверия. Как к этому относиться? Первый
путь --- примириться с неизбежным и считать, что антиномичность неразрешима. Для
определенной области так думал Кант. Один из комментаторов Канта, Файхингер,
опубликовал книгу под заглавием «Философия фикции» (ди философи дер «Альс об»),
где доказывал, что все основные понятия науки настолько противоречивы, что
могут считаться фикциями, но надо выбирать полезные фикции. Эта философия вовсе
не так редка, как может показаться на первый взгляд. Наш Пушкин сказал до
Файхингера: «Тьмы низких истин нам дороже нас возвышающий обман». У Горького мы
читаем: «Если к правде святой мир дорогу найти не сумеет, честь безумцу,
который навеет человечеству сон золотой». И, наконец, Ницше (вернее, Ницше
хронологически занимает середину между Пушкиным и Горьким) выразился так:
«Истина есть наиболее целесообразное заблуждение». Но в этом заключается как
будто различие между Тертуллианом и материалистически настроенными учеными.
Позиция Тертуллиана отдаляет его от науки, механический же материализм при его
слабой философской обоснованности дал, несомненно, очень много науке. Но
Тертуллиан, вероятно, бесплодный в науке, не был бесплоден в области этики. Ему
наряду с Августином принадлежит едва ли не первый протест против смертной
казни. В античные дохристианские времена никто как будто до этого не доходил.
Может быть, Тертуллиан, подобно Августину, не удержался на этой позиции, но тут
уж действовали иные причины.

Другой путь --- искать выход из антиномий в новом синтезе, то, что как было
указано выше, с особой отчетливостью выразил Гегель. И вот мы видим, что
позиция Тертуллиана вовсе не была единственной и даже типичной для христианской
церкви и что наряду с ней было другое, более мощное направление, стремившееся к
примирению противоречий, к синтезу всех источников знания, к примирению знания
и веры. Другой, более великий представитель патристики, Августин, определенно
указывает черты близости христианской и языческой философии (в частности,
Платона). Он был продолжателем направления одного из ранних апологетов
христианства, Юстина-мученика (казнен в Риме ок. 166 г.). Цитирую по «Истории
философии» (под ред. Г. Ф. Александрова и проч., 1941, том I, с. 385-387):
«Юстин доказывал, что почти все содержание христианского учения уже имеется в
языческой философии. И это потому, что у христианства и философии один и тот же
источник --- божественный логос, разлитый во всем мире. В Христе этот логос
только проявился во всей полноте. К христианам Юстин относил всех тех, кто
прожил свою жизнь „с логосом". Таковы из греков --- Гераклит и Сократ.
Теоретически логос признавали также стоики. Юстин имел большое влияние на
позднейших „отцов церкви" и на дальнейшее развитие христианской идеологии».
Здесь огромную роль сыграла знаменитая александрийская школа. Александрия в то
время была центром величайшей в античности точной
науки (Александрийский музей), там же был важнейший центр еврейской «диаспоры»,
в которой складывалась особая, иудейско-эллинистическая культура (виднейший
представитель --- Филон), там жил епископ Климент (ок. 150-215 гг.)
Александрийский, который «развил теорию объединения веры и знания, которая была
принята христианской церковью» (Ист. филос, с. 389), со своим преемником
Оригеном. Все это были последователи Платона. Был и ряд других вполне
ортодоксальных представителей этой школы. У нас обычно, когда вспоминают раннее
христианство, упоминают не о Клименте, а о Кирилле Александрийском,
представителе совсем другого направления, ожесточенном враге эллинской культуры
(гибель Гипатии, сожжение Александрийской библиотеки). Их не так трудно спутать
и по сходству имен, и по месту их деятельности, и потому, что оба причислены к
лику святых. Но не надо забывать, что Кирилл действовал в эпоху после неудачной
языческой реакции императора Юлиана Отступника, преследовавшего христиан.
Неудивительно, что реакцией на языческого императора, объявившего войну
христианству, было возникновение убеждения о принципиальной противности
христианского учения и языческой культуры. Это было повторение сходного
процесса. Основоположником направления, враждебного язычеству (наиболее яркий
пример --- Тертуллиан), был Татиан, который был слушателем Юстина-мученика.
«После казни Юстина Татиан перебрался в Сирию и отошел от церкви, усвоив
осужденные ею гностические воззрения» (Ист. филос, с. 387). И здесь казнь
почитаемого учителя заставила Татиана осудить вместе с палачами учителя всю
языческую культуру. Но победила в раннем христианстве линия Климента. Опять в
«Ист. филос», с. 389: «По Клименту, нет знания без веры и веры без знания.
Полная гармония их требует изучения всего круга человеческих знаний: „семи
свободных искусств". Никакой несовместимости между языческой философией и
христианским учением, согласно Клименту, нет: это как две ветви одного и того
же ствола. Истины христианства согласны с учением лучших из язычников.
Философия представляет собой как бы пропедевтику, преддверие христианства. В
философии истина содержится не целиком в одной какой-либо школе, а по частям во
всех. Хотя отличительным признаком подлинной науки является ее совпадение с
учением веры, однако, с другой стороны, истинное содержание самого писания
устанавливается только философским изучением. Главным приемом для введения
философии в христианство было у Климента, как и у филона, „аллегорическое"
объяснение Священного Писания».

Это аллегорическое толкование Писания через известного епископа миланского
Амвросия перешло к Августину (с. 392). В «Истории философии» отмечается, что
сочинение Августина «О граде божием», несмотря на наивность исторической
концепции, имело большое историческое значение как попытка дать обзор истории
человечества в целом, как попытка философии истории. На с. 396 читаем:
«Христианская церковь, искаженно отобразив, как в кривом зеркале, эллинскую
культуру, сберегла ее в этом виде до нового времени, когда постепенно античная
культура стала оживать в ее подлинном виде. Вклады самих патриотических
писателей в науку ничтожны и не могут идти ни в
какое сравнение с достижениями античного мира». Последнее замечание верно, но
есть ли здесь вина христианства? Подлинная античная, эллинская культура в своих
высших достижениях не была понятна солдафонскому Риму. Первый пожар
Александрийской библиотеки был во время взятия Александрии Юлием Цезарем
(несомненно, одним из культурнейших людей в римском понимании термина
культура). Убыток был до известной степени возмещен Антонием, который галантно
поднес Клеопатре огромное количество рукописей из разграбленного Антонием
Пергама. Но все это собрание после гибели Клеопатры отправилось с триумфальным
поездом в Рим, где и было сожжено. Государственные дотации Александрийскому
музею, позволявшие ученым спокойно работать, были прекращены. Наконец, с севера
надвигались варвары. И здесь мы читаем в «Истории философии», с. 394; «Учение
Августина о предопределении было религиозным фатализмом. Для христиан оно
служило идеологической опорой в тяжелой борьбе, какую начиная с V в. пришлось
вести западной церкви с нахлынувшими на Европу и Северную Африку варварскими
народами. Вера в предопределение и возведение к воле божества каждого действия
как отдельного христианина, так и всей церкви придавали христианской церкви
сплоченность и фанатическое упорство».

Экономическое потрясение Западной Европы и Северной Африки, вызванное
нашествием варваров и крушением Римской империи, было основной причиной того
упадка культуры, который характеризует «века мрака» (примерно до 1000 г.). Если
бы не было христианской церкви, сохранившейся благодаря своему «фанатизму»,
если бы в этой церкви и ее монастырях не тлели ростки, изучавшиеся лучшими
представителями культуры --- монахами, то и возрождение наук (которому
чрезвычайно помогли и сохраненные арабами и греками элементы эллинской
культуры) не было бы возможно. И потому конец первой книги «Истории философии»:
«Главное содержание патристики --- разработка религиозной идеологии --- имела для
научного прогресса человечества отрицательное значение, служила орудием
косности и застоя» --- совершенно не соответствует истине. Как могла бы
возродиться античная культура, если бы сохранилась только римская традиция?
Ведь средневековая философия (вершиной которой был доминиканец Фома Аквинат)
основывалась на Аристотеле (которого привели к согласию с христианской
идеологией), а Возрождение связано с именами Платона и Пифагора, а отнюдь не
Эпикура и Лукреция.

Но у Августина, как и Тертуллиана, есть как будто совершенно отрицательные
взгляды. Несмотря на то что оба они высказали прогрессивную мысль, что смертная
казнь несовместима с христианским учением, оба они не удержались на этой
позиции (История философии, с. 395): «Августин ревностно отстаивал право церкви
на принуждение в делах веры на том же основании, что принуждение к „истине"
вовсе не есть насилие, а забота о благе принуждаемого. Учение Августина
получило зловещий характер, превратившись в реальность церковной практики. Так
как всякий еретик будет вечно мучиться за гробом, то лучше ему претерпеть
сожжение здесь, на земле (хотя и это не вполне обеспечивает его от загробных
мук)».

Но ведь это как бы предвидение современных «прогрессивных» взглядов, что
насилие над вредными учениями не есть нарушение свободы. Как раньше боролись с
еретиками, так теперь борются с «ревизионистами», исходя из того положения, что
единодушие необходимо в трагические периоды человеческой истории. И смертная
казнь в мирное время, практически отсутствовавшая в дореволюционной России
(подвергались казни только покушавшиеся на царя), восстановлена революционной
властью в неслыханном размере и сохраняется даже тогда, когда можно сказать,
что трагические времена миновали.

Наука в Западной Европе в течение темного периода, последовавшего за разорением
Европы варварами, теплилась в монастырях, но не следует думать, что она велась
тайком в противовес официальной католической церкви. Нет, ведь начало подъема
цивилизации обычно считается совпадающим с началом XI в., когда на папский
престол вступил Герберт под именем Сильвестра II. Этот был ученейший муж своего
времени, получивший образование в мусульманском университете в Кордове.

В XII-XIII вв. именно католической церковью были основаны университеты. В XIII
в. францисканский монах Роджер Бэкон положил основание индуктивной логике.
Силлогистическая логика получила наибольшее свое развитие в системе Петра
Испанского, португальского ученого, который правил 8 месяцев под именем папы
Иоанна XXI (Минто. Индукт. и дедукт. логика, с. 18-19). Конечно, была всегда и
ретроградная оппозиция, восходящая идейно к Татиану.

Противник Герберта, епископ Оттон, уверял, что Герберт обязан своим высоким
положением только союзу со злыми духами (Уэвель. История индуктивных наук, т.
I, 1867, с. 582). Некоторые историки литературы видят в Герберте прообраз
Фауста. Роджер Бэкон то попадал в тюрьму при папах одного направления, то
освобождался при прогрессивных папах и все-таки умер на свободе в старости. Но
невежественная толпа видела во всех ученых колдунов и соучастников нечистой
силы, и от этого обвинения не были защищены даже служители религии: Фома
Аквинат, Роджер Бэкон, Михаил Скот, Роберт Гростет, епископ Линкольнский,
Альберт Великий, епископ Регенсбургский, папы Сильвестр II и Григорий VII
(Уэвель, там же, с. 381).

Как было указано, такими обвинителями были и некоторые духовные лица. Ясно, что
по отношению к дохристианским деятелям обвинение в колдовстве было еще более
естественным: сюда попали Аристотель, Соломон, Иосиф, Пифагор, Вергилий (там
же).

Тенденция синтеза науки и религии получила свое наибольшее выражение в трудах
Фомы Аквината, и мы знаем, что и сейчас философия Фомы (неотомизм, или просто
томизм) является далеко не исчезнувшим учением. Во время Возрождения борьба шла
не между наукой и религией (что будет еще разъяснено дальше), а между
консервативным направлением, связанным с Аристотелем, и новым, связанным с
именами Пифагора и Платона.

7. Игнорирование антиномий

В предыдущем параграфе было рассмотрено два пути преодоления антиномий:
признание их непреодолимыми, стремление к преодолению противоречия, примирение
разума с верой. Но есть еще третий путь --- полное их игнорирование. Он основан
на том гносеологическом постулате, что мы имеем уже очень много окончательно
установленных истин и что прогресс науки заключается в постепенном наращивании
таких абсолютных истин и в постепенном распространении их на неограниченно
большую область бытия. Это мнение выражено классическим представителем
детерминизма Лапласом во введении к «Аналитической теории вероятностей».
Цитирую по статье Елены Эйльштейн: «Лаплас, Энгельс и наши современники» в
сборнике «Закон, необходимость, вероятность». 1967, с. 235-236: «Все явления,
даже те, которые по своей незначительности как будто не зависят от великих
законов природы, суть столь же неизбежные следствия этих законов, как обращение
Солнца. Не зная уз, соединяющих их с системой мира в целом, их приписывают
конечным причинам или случаю, в зависимости от того, происходили ли и следовали
ли они одно за другим с известной правильностью или же без видимого порядка; но
эти мнимые причины отбрасывались по мере того, как расширялись границы нашего
знания, и совершенно исчезли перед здравой философией, которая видит в них лишь
проявление неведения. Мы должны рассматривать настоящее состояние Вселенной как
следствие ее предыдущего состояния и как причину последующего. Ум, которому
были бы известны для какого-либо данного момента все силы, одушевляющие
природу, и относительное положение всех ее составных частей, если бы вдобавок
он оказался достаточно обширным, чтобы подчинить эти данные анализу, обнял бы в
одной формуле движения величайших тел Вселенной наравне с движениями легчайших
атомов: не осталось бы ничего, что было бы недостоверно, и будущее, так же как
и прошедшее, предстало бы перед его взором. Ум человеческий в совершенстве,
которое он придал астрономии, дает нам представление о слабом наброске того
разума. Его открытия в механике и геометрии в соединении с открытием всемирного
тяготения сделали его способным понимать под одними и теми же аналитическими
выражениями прошедшие и будущие состояния мировой системы. Применяя тот же
метод к некоторым другим объектам знания, нашему разуму удалось подвести
наблюдаемые явления под общие законы и предвидеть явления, которые будут
вызваны данными условиями. Все усилия духа в поисках истины постоянно стремятся
приблизить его к разуму, о котором мы только что упоминали, но от которого он
останется всегда бесконечно далеким. Это стремление, свойственное роду
человеческому, возвышает его над животными; и успехи его в этом направлении
различают нации и века и составляют их истинную славу».

Детерминизм Лапласа вызывал самую разнообразную реакцию. Как я
указывал в параграфе (об абсурдном), наши ортодоксальные философы
обвиняли Лапласа в фатализме и даже в уклоне в мистицизм. Е.
Эйльштейн, из статьи которой взята эта цитата, считает, что дело в
непра-

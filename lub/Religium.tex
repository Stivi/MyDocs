НАУКА И РЕЛИГИЯ

ПОСТАНОВКА ВОПРОСА

Вопрос об отношении науки и религии имеет по крайней мере двухвековую давность,
а правильнее, может быть, даже двухтысячелетнюю, и то решение, которое
наметилось примерно полтора века тому назад, многим из современных
интеллигентов кажется окончательной истиной в последней инстанции. Оно всего
лучше отображено в знаменитом коротком разговоре великого ученого Лапласа и
выдающегося государственного деятеля Наполеона Бонапарта. Когда Наполеон
(который, как известно, был хорошо знаком с высшей математикой) ознакомился с
известным сочинением Лапласа о небесной механике, он задал ему вопрос, почему
тот не упоминает о Боге в этой книге. Наполеон, очевидно, намекал на Ньютона,
которые закончил свои великие «Математические начала натуральной философии»
идеологическими рассуждениями. «Государь, я не нуждался в этой гипотезе», ---
ответил Лаплас (цитирую по Энгельсу, Диалектика природы, с. 268). Наполеон
задал свой вопрос не потому, что он был верующим, но он перешел от
преследования католической церкви к конкордату\footnote{Конкордат Наполеона ---
соглашение между папой римским (Католической церковью) и Наполеоном
(Францией) заключенный 15 июля 1801 года, по которому Рим признавал
новую французскую власть, католицизм был объявлен религией большинства
французов. При этом свобода вероисповедания сохранялась.} с целью упрочить свое
положение. Религия не нужна науке, но нужна даже атеистическому деспоту как
«незримая паутина» (прекрасное выражение Горького) для более легкого подчинения
угнетенных масс, как опиум для народа. Вспомним преследование инакомыслящих,
Галилея, Дж. Бруно, инквизицию, индекс запрещенных книг, сопротивление
эволюционной теории Дарвина, и как будто придется сделать заключение, что
религия не только бесполезна, но и вредна для развития науки. Мнение
просветителей конца XVIII и начала XIX в., казалось, торжествует в течение
всего XIX и начала XX в. В биологии телеологии был нанесен сокрушающий удар Ч.
Дарвином, в конце XIX в. был сильный процесс «дехристианизации» Франции,
Бисмарк объявил «культуркампф» против католической церкви в Германии, наконец,
в XX в. правительства почти одной трети человечества стали откровенно
атеистическими. Подрастающая молодежь этих стран практически ничего не знает о
религии. Идеи социализма и атеизма считаются неразрывно связанными, и поскольку
деятели социализма выставляют великолепный идеал осуществления справедливого
строя на земле (в противовес обещанному церковью Царствию Божию на небе после
смерти), то эти социалистические идеалы считаются морально обязательными, а
следовательно, морально обязательным является и атеизм, и всякая «поповщина»,
легко приводящая к тем или иным формам теологии, отметается «с порога». Поэтому
обязательность атеистического обучения молодежи, запрещение религиозной
пропаганды и свободного издания религиозных книг не кажется многим ограничением
свобод вообще и свободы совести в частности, так как для многих честных
образованных и умных людей всякая религия кажется пережитком прошлого, подобным
каннибализму, обычаю убивать стариков, колдовству, гаданиям по звездам,
«свободе» заводить обширные гаремы, геноциду, учению о низших расах и проч., и
восстановление совершенно недопустимых с моральной точки зрения обычаев:
истребление пленных, стариков, душевнобольных и вообще «неполноценных»,
проводимое гитлеровцами, конечно, не делает гитлеровскую Германию более
«свободной» страной, чем страны антигитлеровской коалиции, так как свобода
истреблять или развращать себе подобных (торговля наркотиками, порнографической
литературой и проч.) не может считаться свободой, достойной прогрессивного
человечества. Не всякая свобода и желательна, как говорится в одном коротком
диалоге еще дореволюционного времени: «Извозчик, ты свободен?» --- «Свободен».
--- «Кричи: „Да здравствует свобода!"» Свобода пропаганды суеверий,
распространения наркотиков и т. п. не есть та свобода, за которую есть смысл
бороться.

Неудивительно, что в нашей стране, отставшей в силу пережившего себя
общественного строя от западных стран, особенно силен был антирелигиозный дух
среди нашей интеллигенции, несмотря на то что до революции преподавание в
школах в значительной степени было проникнуто религиозным духом (разных
религий), в каждом паспорте обозначалась религиозная принадлежность носителя
паспорта (национальность в паспортах не фигурировала) и не допускалось, чтобы
кто-либо объявил себя атеистом. Протест против религии носил исключительно
резкий характер. Ленин считал религию вообще одним из отвратительнейших явлений
в обществе и всякую самую утонченную форму фидеизма он решительно отбрасывал «с
порога». Даже субъективный идеализм объявлялся «поповщиной», и по отношению к
самым близким людям Ленин не допускал никакого компромисса. Его собственный
отец был глубоко религиозным человеком, как и отец Чернышевского, и оба они
были очень почтенными людьми. Религия была тем более ненавистна нашим
революционным демократам, что она тормозила развитие народа. Поэтому все они
переходили от глубокой религиозности в юности к воинствующему атеизму зрелого
возраста (Салтыков-Щедрин, Писарев, Чернышевский и др.). Но воинствующий атеизм
был свойствен не только революционерам или марксистам, но и лицам, далеко
стоящим от революционной борьбы и марксизма. Мне известен один ученый
выдающихся умственных и моральных качеств, весьма скептически относившийся к
селекционизму (неодарвинизму, т. е. учению о ведущей роли естественного
отбора), но вместе с тем считавший, что дарвинизм принес пользу как мощное
антирелигиозное учение. Религия в глазах этого ученого и честного человека была
настолько отрицательным явлением, что для борьбы с ней можно было использовать
даже ложные учения. Подобно тому как в христианском песнопении поется про
Христа, что он «смертию смерть поправ», так и здесь ложью следует победить
другую, более страшную ложь. «Цель оправдывает средства» --- этот лозунг
употребляется и людьми, весьма отрицательно относящимися к иезуитам.

И не только у нас, но и среди ученых и мыслителей Запада сейчас есть немало
представителей, полностью солидаризующихся с тем мнением, что религия есть зло,
только мешавшее прогрессу человечества, и что хотя среди великих ученых и
философов мы знаем очень много истинно религиозных людей, они сделались
великими учеными не благодаря, а вопреки своей религиозности, и что пропасть
между религией и наукой непроходима. Такую точку зрения развивают и лица,
весьма враждебно относящиеся к коммунизму, например Бертран Рассел. Его
неприязнь к коммунизму доходила до того, что он (пока Запад обладал монополией
атомной бомбы) был не против того, чтобы использовать преимущество Запада для
политических целей, но он же (сейчас --- видный представитель движения за мир)
сотрудничает в нашем журнале «Наука и религия» и прямо заявляет (в полном
согласии, например, с нашим Луначарским), что ученый и прогрессивный человек
вообще не только может, но и должен быть антирелигиозным человеком. Рассел не
одинок в своем высказывании. В 1961 г. появился под общей редакцией Юлиана
Гексли сборник статей 27 авторов под названием «Структура гуманизма»,
подвергнутый подробной рецензии в британском журнале «Философия науки» (1963,
том XIV, № 53, с. 41--53) Аланом Стюартом. Рецензент приветствует этот сборник
как «новое откровение», как веху в деле эмансипации человеческого духа, как
обоснование нового эволюционного гуманизма, полностью отвергающего все старые
религиозные суеверия, всякую сверхъестественную религию. И несмотря на высокую
оценку сборника, Стюарт отмечает очень крупный дефект, заключающийся в том, что
эти новые гуманисты одиннадцать раз цитируют иезуита Тейяра де Шардена даже с
указанием на желательность знакомства с этим автором. Рецензент негодует (с.
52): «Многие образованные честные искатели истины, которые знают кое-что о
последних двух тысячах лет человеческой истории, не могут избежать связывать
„священника" (попа) с „поповщиной", а „поповщину" с „лживостью". Здесь же мы
видим испорченную книгу, которая могла бы быть превосходной. Потому что,
несмотря на все их изъявления о глядении в будущее, эти гуманисты, подобно жене
Лота, тоскующей по Содому, глядят назад». Я затруднялся в переводе слов
«priest» и «priestcraft» и вставил вместо «priestcraft» «поповщину». Последний
термин широко применяется в советской литературе, но он почему-то отсутствует в
трех новых словарях: русско-английском и русско-немецком 1952 г. и
русско-французском 1962 г. Но мы видим, что рецензент, предлагая нам смотреть
только вперед и не оглядываться, или забыл, или игнорирует известное положение
о диалектике, где новый синтез в известном смысле возвращается к старому.

Но наряду с таким резким отношением ко всякой «поповщине» мы имеем гораздо
более примирительное отношение. Добржанский, очень близкий по своим
биологическим взглядам к Ю. Гексли, заканчивает свою интересную книгу об
эволюции человечества цитатой из Тейяра де Шардена и считает, что эволюционная
идея иезуита является лучом надежды для человечества. Он считает, что его книга
содержит и науку, и метафизику, и теологию, и даже поэзию, испорченную
несколько в английском переводе. Предисловие к русскому переводу
книги Тейяра де Шардена «Феномен человека» написано выдающимся, одним из
наиболее культурных французских коммунистов, Роже Гароди, который, наряду с
критикой многих положений Шардена, высоко оценивает многие прогрессивные
стороны его взглядов. Исчезает положение «двух лагерей»: с одной стороны,
реакционный черный лагерь защитников религии, состоящий из честных невежд или
фанатиков и бесчестных эксплуататоров, а с другой --- светлый лагерь
«прогрессистов», куда относятся все умные и честные люди, «свободомыслящие»,
или, как это понималось на языке XIX в., атеисты и социалисты. Все
перемешалось. По отрицательному отношению к религии Ленин оказался в одной
компании с Б. Расселом и цитированным рецензентом Стюартом, более умеренный
Луначарский, говоривший о «религиозном атеизме», о «новой религии социализма»,
сочувствующий неодарвинист Добржанский и, наконец, коммунист Гароди: для
каждого оттенка отношения к религии мы можем подыскать представителей обоих
лагерей. И здесь, как может быть всюду, господствует в высокой степени
комбинационный принцип.

Сейчас в мировом масштабе происходят два встречных течения. Наряду с успехами
антирелигиозности в мировом масштабе, падением посещаемости храмов и проч. мы
имеем укрепление религиозных позиций в ряде культурных стран. Во Франции
возвращены изгнанные на пороге столетия иезуиты и во главе правительства стоит
верующий католик. Ряд правительственных партий во Франции, ФРГ, Италии, Бельгии
и других странах откровенно христианские. Можно ли это объяснить только
империалистической реакцией? Два выдающихся деятеля XX в. --- Ганди и Кеннеди ---
не скрывали своей религиозности, и оба пали от рук людей, сходных по идеологии
с фашизмом, империализмом, расизмом и прочими бесспорно ретроградными
идеологиями. Защитниками религии или «поповщины» в широком смысле слова (причем
самого разнообразного характера) выступают самые передовые ученые
современности: Эддингтон, Эйнштейн, Гейзенберг, Планк, Шредингер и многие
другие меньшего значения. Писатели Сент-Экзюпери, Веркор и другие настаивают на
необходимости синтеза, а не голого отрицания. Растет число идеалистов самых
разнообразных направлений, в то время как материалистическая философия скорее
обнаруживает явные признаки загнивания. Мы знаем, что в нашей стране под видом
борьбы с религией и идеализмом систематически боролись со всеми новыми
течениями в науках: теория относительности, принцип неопределенности, теория
расширяющейся Вселенной, теория резонанса в химии, настоящая генетика. Даже
там, где поддерживали то или иное здравое направление (например, учение об
условных рефлексах Павлова), так догматизировали учение, что оказали ему
медвежью услугу. Все же новшества, предложенные под видом истинно
материалистической науки, не выдержали испытания временем, и притом короткого
времени. Таким образом практически опровергнут подход: раз это ведет хотя бы в
слабой степени к «поповщине», это надо отвергнуть «с порога».

Не менее неожиданными были и события на этическом и политическом фронтах.
Ставилось в вину всем религиозным правительствам, что они наряду с
«запугиванием адом» имели на вооружении и смертную
казнь. Запугивание адом у нас исчезло, но смертная казнь фигурирует в нашем
Уголовном кодексе в таком числе статей, как, я думаю, нигде в капиталистических
странах, а некоторые капиталистические страны (мне известны: ФРГ, Мексика,
Уругвай, Израиль до процесса Эйхмана) вовсе не имеют смертной казни в мирное
время. В политике «и сольются в одно все народы в вольном царстве святого
труда» пока обернулось жесточайшим конфликтом между двумя крупнейшими
коммунистическими партиями: СССР и Китая.

Если пока между Китаем и СССР еще нет вооруженного конфликта, то только потому,
что Китай еще недостаточно силен и раздираем внутренними противоречиями, а у
нас неслыханная в царские времена милитаризация и достаточная удовлетворенность
последними захватами. Но разговор между двумя «социалистическими» странами по
тону вовсе не отличается от разговоров двух милитаристских империалистических
держав. Но вопросы этики и политики уже выходят за пределы темы настоящей
статьи.

Мы видим, таким образом, что в отношении религии сейчас наблюдаются два
противоположных процесса: 1) антирелигиозный, касающийся масс и среднего уровня
интеллигенции; 2) прорелигиозный, выражающийся в обильном числе всевозможных
направлений, часто затрагивающих самые высокие уровни передовых мыслителей
современности. Полезно выяснить, является ли это второе направление вызванным
какими-либо новыми открытиями или явлениями современности или можно найти и в
материалистическом мировоззрении начала XIX в. такие черты, которые ясно
указывали на его несовершенство. Так ли ясно все в отношении разговора Лапласа
с Наполеоном? К этому и перейдем.

Глава I. НАУКА --- ВРАГ СУЕВЕРИЙ И ЧУДЕСНОГО

Дать определение науки не так-то легко, для наших целей достаточно ограничиться
тем, что ясно сквозит в словах Лапласа. Наука --- враг всякого суеверия,
чудесного, а религия, напротив, основана на суевериях и принятии чудесного.
Всего лучше это выражено в знаменитых словах блаженного Тертуллиана: «кредо,
квиа абсурдум ест», т. е. «верю, потому что это абсурдно». Пока не будем
разбирать вопроса, является ли высказывание Тертуллиана ортодоксальным или
типичным для всех христиан. Ограничимся тем, что уже в раннем христианстве
обозначилось направление, резко противополагавшее религиозное учение тогдашней
классической науке (см., например, История философии, 1941, № 1, с. 387--388) и
в известном смысле сохранившееся и в более поздние времена. Каков смысл
высказывания Тертуллиана? В учении о познании Тертуллиан придерживался
вульгаризированного стоического материализма и считал, что все действительно
существующее телесно, в том числе Бог и бессмертная душа. Таким образом, его
взгляд вовсе не является выражением его идеализма, а напротив, теснейшим
образом связан с материалистическим характером его мировоззрения. Разумом он
был материалист, но, как христианин, принимал и такие явления, которые
материалистическому объяснению
совершенно не поддавались, но так как они были для него совершенно бесспорны,
то в них он верил как в абсурд с точки зрения материализма. Это была не слепая,
а сознательная вера, и он твердо проводил различие между верой и знанием. А так
как для него на первом плане стояли истины веры, то естественно, что к
человеческой мудрости он относился с меньшим уважением и порой даже с
презрением. Но посмотрим, свободен ли материализм XIX в. от принятия чудесного
и непонятного. Смысл науки в том понимании, которое нас сейчас интересует,
заключается в том, что она должна бороться с суевериями, что можно понимать в
пяти формах: 1) чудесное, т. е. непонятное, 2) абсурдное, 3) противоречивое, 4)
сверхъестественное и 5) невероятное. Разберем по очереди.

1. Непонятное, часто необычное

Все непривычное, неожиданное, непонятное нам кажется чудом, и мы и в практике
жизни, и в науке стремимся к тому, чтобы непонятное свести к чему-то понятному.
Антропоморфические религии сводят все «чудесное» к вмешательству существа,
подобного человеку, но невидимого. Так толковались изумительные приспособления
органического мира. Дарвин оставил телеологию, т. е. мнение, что основа в
биологии --- явление приспособления, но вместо невидимого бога ввел тоже
невидимое явление --- естественный отбор, приемлемый для материалистов. Задачей
науки оказалось дать простое объяснение сложным явлениям. Это направление имеет
большую давность. Уже цитированный Стюарт (Брит. Журн. филос. науки, 1963, т.
XIV, № 53, с. 53) по поводу нового, эволюционного гуманизма пишет: «Это ново,
ново в истории так называемой западной или греко-христианской цивилизации.
Конечно, были и раньше люди, которые мыслили свободно --- в общем так же, как и
сегодняшние гуманисты. В дохристианскую эру были Эпикур и Лукреций с их
последователями». Можно ли считать Эпикура одним из лидеров античной науки?
Вряд ли. Со взглядами Эпикура можно познакомиться по известному античному
историку философии Диогену Лаэрцию (я использовал немецкий перевод). Диоген
Лаэрций --- добросовестный, но малокритический историк, который сообщает разные
небылицы про многих философов, но Эпикуру он посвятил, как и Зенону-стоику,
наибольшее место в своей книге (по 68 страниц) и, в отличие от других
философов, привел много текстов. Хотя обычно Эпикура считают представителем
линии Демокрита, но он резко отличается от Демокрита ярко выраженным
индетерминизмом. Основа мировоззрения Эпикура: боги не вмешиваются в нашу
судьбу, не вознаграждают и не карают, смерть ни к чему плохому не приводит. Это
и есть основа подлинного эпикуреизма: безмятежность души, атараксия. У самого
Эпикура это не приводит к безнравственности: напротив, многие из его положений
сходны с таковыми стоиков, он обосновывает этику утилитарными соображениями,
что вовсе не так плохо, но что дает его мировоззрение науке? Здесь как будто
проявляется полное свободомыслие. В вопросе о величине Солнца, о движении
планет, знамениях, громе и других явлениях он выставляет различные гипотезы и
не дает предпочтения ни одной. Они все с его точки зрения равноценны, при
условии, чтобы не было мифов, связанных с религией. Но будучи совершенно
беззаботным в области научных гипотез, Эпикур крайне догматичен в онтологии или
метафизике и против всякой диалектики. Но, может быть, это и есть то полезное
ограничение свободы, которое необходимо для ученого, чтобы он сосредоточил свои
усилия, основываясь на определенных бесспорных аксиомах. Нет, ограничиваясь
объяснениями и не придавая им принудительного значения, Эпикур чрезвычайно
презрительно относится к терпеливым усилиям античных астрономов, старавшихся
длительными наблюдениями выяснить законы движения небесных тел, т. е.
положивших начало математическому описанию явлений природы. Переводчик и
комментатор Апельт правильно пишет, что если бы такая точка зрения
восторжествовала, естествознание никогда бы не вышло из исходного состояния.

Последователями Эпикура в смысле необходимости в первую очередь «объяснения», а
не математического «описания» явлений был, конечно, Лукреций и все эпикурейцы.
А так как в Римском государстве в философии господствовали стоики (обращавшие
внимание только на этику) и эпикурейцы, то становится понятным тот ничтожный
вклад в науку, который сделала могущественная Римская империя.

Но может быть, учения Эпикура (и Лукреция, который не дал, кажется, ничего
оригинального по сравнению с Эпикуром) были в политическом отношении
прогрессивны? Он был гуманный человек и рекомендовал доброе отношение к рабам,
но в смысле общественной идеологии был прототипом «премудрого пескаря»: не
стоит жениться и иметь детей, не занимайся государственной деятельностью, как
бы чего не вышло, пребывай в мудрой атараксии и не размышляй о возможности
преобразований.

Если мы посмотрим всю историю человеческой мысли, то убедимся, что решительно
все строители «утопий» были идеалисты, что же касается материалистов, атеистов
и антирелигиозников, то они или сами были тиранами (Критий), или защищали
абсолютизм под разными видами --- «просвещенного абсолютизма» и т. д. (Гоббс,
Дидро и пр.). Все идеологи революций вплоть до XIX в. работали под
идеалистическими знаменами и в общем сделали немало для преобразования
общества. Поэтому слова Маркса: «Философы только объясняли мир, а его
необходимо перестроить» (точно слова не помню) надо понимать:
«Материалистические философы только объясняли мир, а сейчас им надо приняться
за перестройку», и тогда он будет звучать более или менее правильно, если
оставить в стороне вопрос (сейчас оставим его без рассмотрения), в какой мере
революционный марксизм может считаться чисто материалистическим учением.
Несомненно, что современный селектогенез, т. е. эволюционное учение, восходящее
к Дарвину, совершенно пропитан эпикурейским духом. Основной императив
дарвинской морфологии: дай какое-нибудь «причинное» и «механическое» объяснение
структуре, которое можно свести к действию естественного отбора, с тем чтобы
устранить «поповщину», «платонизм» и прочие вредные учения. Неважно, что это
объяснение не является механическим или причинным в смысле точных наук, важно
заглушить сомнения
в отсутствии целеполагающих начал в природе. К математическому толкованию формы
и системы это направление, естественно, глубоко враждебно. Но это направление
поддерживают и выдающиеся математики! Верно, но об этом придется сказать
несколько слов в разделе о невероятном.

Простое объяснение есть низший этап развития научного мышления, и если это
направление доминирует, то оно притупляет то, что можно назвать научной
бдительностью, удивлением перед новыми фактами, и становится подлинным опиумом
для науки.

Только что рассмотренный пункт к Лапласу не относится, так как он был
представителем точной науки, и когда давал объяснения (например, в его
знаменитой космогонической гипотезе), он оговаривал ненадежность этой гипотезы,
так как ее математической теории он дать не мог.

2. Абсурдное, т. е. нелепое

А вот абсурда Лаплас не избежал. Он был убежден, что в мире существуют как
конечные реальности одни атомы в пустом пространстве и был убежденным
сторонником механического детерминизма. По нашему философскому словарю (1963,
статья «Детерминизм», с. 121), Лаплас считал, «что значения координат и
импульсов всех частиц во Вселенной в данный момент времени однозначно
определяют ее состояние в любой прошедший или будущий момент. Так понятый
детерминизм ведет к фатализму, принимает мистический характер и фактически
смыкается с верой в божественное предопределение». Как видим, наши официальные
философы находят поповщину и в классическом изречении Лапласа, но невозможно
понять, что они дают взамен. Но в мировоззрение Лапласа входит не только
механический детерминизм, но и, следом за Ньютоном, принятие принципа
всемирного тяготения, т. е. действия на расстоянии материальных тел. С точки
зрения механики это совершенный абсурд: как может тело действовать там, где его
нет? А как же Ньютон? Ньютон это отлично понимал. В письме к Бентлею, автору
лекций по опровержению атеизма, Ньютон пишет (С. И. Вавилов. Исаак Ньютон. М.,
1961. С. 129): «Предполагать, что тяготение является существенным, неразрывным
и врожденным свойством материи, так что тело может действовать на другое на
любом расстоянии в пустом пространстве, без посредства чего-либо передавая
действие и силу, --- это, по-моему, такой абсурд, который немыслим ни для кого,
умеющего достаточно разбираться в философских предметах. Тяготение должно
вызываться агентом, постоянно действующим по определенным законам. Является ли,
однако, этот агент материальным или нематериальным, решать это я предоставил
моим читателям». Как указывает С. И. Вавилов дальше (с. 130), для самого
Ньютона вопрос был совершенно ясен: тяготение объясняется заполнением
пространства Богом (предшественники: Отто фон Герике и иезуит Кирхер). В
несколько скрытой форме это мнение было высказано и в «Общем поучении»
знаменитых «Математических начал натуральной философии» (с. 590, перевод А. Н.
Крылова, 1915 г.): «Бог есть единый и тот же самый Бог всегда и везде. Он
вездесущ не по свойству только, но по самой сущности, ибо свойство не может
существовать без сущности. В нем все содержится и все вообще движется, но без
действия друг на друга. Бог не испытывает воздействия от движущихся тел,
движущиеся тела не испытывают сопротивления от вездесущия Божия». В примечании
Ньютон ссылается на древних авторов: Пифагора, Фалеса, Анаксагора, Филона,
Арата, а также приводит ряд текстов из Библии. Богословские взгляды Ньютона не
были каким-то странным привеском к его научным взглядам, они пронизывали и его
научные теории. Он не был деистом, принявшим только первый толчок, а затем
исключительное действие естественных законов, он был ближе к взглядам
Мальбранша, для которого все происходящее является сплошным чудом. Ко времени
Лапласа к абсурдному с точки зрения механики принципу всемирного тяготения
успели привыкнуть благодаря исключительной плодотворности этого принципа, а
привыкнуть можно к любому абсурду и к непонятным вещам. Если бы у Лапласа не
было «убеждения чувства» --- его механического материализма, то он должен бы
был ответить Наполеону так: «Государь, великий Ньютон ввел Бога в свою книгу
для объяснения принципа всемирного тяготения, который я полностью использовал в
своей работе. Но Ньютон, будучи свободомыслящим ученым, не навязывал своего
взгляда другим и не запрещал искать материальных агентов всемирного тяготения.
Я, правда, не нашел таких агентов, но, надеюсь, кому-нибудь это удастся,
поэтому я и не упоминал о нематериальных факторах в своей книге». Сейчас мы
знаем, что эту роль до известной степени выполнил Эйнштейн в его общей теории
относительности. Нет пустого пространства, оно все наполнено физическим полем
или физическими полями (всю жизнь Эйнштейн и стремился к тому, чтобы дать
единую теорию поля). Дальнодействие исчезло, а вместе с тем и абсурд. Стало ли
все понятным? На этот вопрос я ответа дать не решаюсь. Эйнштейн, как известно,
разобрался и в непонятном факте эквивалентности инертной и тяжелой массы. Этот
непонятный факт в силу своей привычности не смущал подавляющее большинство
физиков.

3. Противоречивое, антиномичное

Это тоже приближается к абсурду, когда об одном и том же мы можем высказать с
одинаковым правом два прямо противоположных суждения. Световой эфир, как
известно, обладал такими свойствами, которые не могут быть приписаны
одновременно никакому телу, и, однако, теория светового эфира была очень
полезной, многие физики считали его существование совершенно доказанным, и наш
великий ученый Д. И. Менделеев всерьез считал возможным рассматривать его как
один из элементов периодической системы. Сейчас, как известно, Эйнштейн
упразднил эфир в специальной теории относительности и до известной степени
реабилитировал это понятие в совершенно ином понимании (лишенном механических
свойств) в общей теории относительности. Но является ли непротиворечивость
обязательным свойством научного мышления? Мы знаем, что великий философ Кант в
свою
классическую «Критику чистого разума» включает «диалектику чистого разума», где
видное место занимают антиномии чистого разума, т. е. такие пары
противоположных суждений, где можно доказать нелепость каждой из
противоположностей. Две наиболее известных: о конечности и бесконечности
пространства и времени и о необходимости и свободе. Кант считал
сформулированные им антиномии принципиально неразрешимыми и в этом видел
границу человеческого разума, все же остальное познание он считал возможным
свести к непротиворечивому виду и полагал, что некоторые теории познания, в
частности формальная логика, уже этого уровня достигли. Он не сомневался в
абсолютной достоверности математических аксиом и теорий. Существование Бога он
считал недоказуемым и неопровержимым, но склонен был строить религию исходя из
этических соображений (Религия в пределах только разума).

Против критической философии Канта со всей решительностью выступил другой
великий немецкий философ Гегель. В своей речи 22 октября 1818 г. (соч. Гегеля,
т. I, 1929, с. 15) он пишет: «Наконец так называемая критическая философия дала
этому неведению вечного и божественного возможность придерживаться этой позиции
с чистой совестью, так как эта философия уверяет, будто ей удалось \emph{доказать}
(курсив Гегеля. --- \emph{А.Л.}), что мы ничего не можем знать относительно вечного и
божественного. Это мнимое познание даже дерзнуло присвоить себе название
философии, и ничего не могло быть желаннее для поверхностных умов и характеров,
ничто не было столь охотно принято ими, как это учение о незнании, благодаря
которому их собственная поверхность и пустота оказывались чем-то превосходным,
желанной целью и результатом всех интеллектуальных усилий. Что мы не знаем
истины и что нам дано знать одни случайные и преходящие, т. е. \emph{ничтожные},
явления, вот то \emph{ничтожное} учение, которое делало и делает наиболее шума и
которое господствует теперь в философии». Из этой цитаты ясно, что утверждение
о непознаваемости мира, которое наши казенные философы инкриминируют всем
идеалистам, явно неприложимо к такому выдающемуся философу, как Гегель,
который, конечно, не меньший, а больший идеалист, чем Кант.

Как же относится к антиномиям Гегель? Отрицает ли он их? Напротив (там же, с.
97), он упрекает Канта за то, что тот перечисляет только четыре антиномии,
тогда как, по мнению Гегеля, антиномии встречаются во всех предметах всякого
рода, во всех представлениях, понятиях и идеях. Разрешение противоречия состоит
в том, что оно принадлежит не предмету самому по себе, а лишь познающему
разуму. Следовательно, в своем развитии разум снимает противоречие (синтез) с
тем, чтобы перейти к новому противоречию, вновь снимаемому, и так далее на
бесконечном пути стремления к абсолютной истине.

По-видимому, общий взгляд Гегеля сейчас торжествует в науке. Аподиктическая
достоверность евклидовой геометрии разбита трудами Лобачевского, Римана и др.,
антиномия конечности и бесконечности пространства снята в общей теории
относительности, где пространство оказывается (в духе Римана) ни конечным, ни
бесконечным, а безграничным. Развитие математики и физики шло, по-видимому,
самостоятельным путем, независимо от Гегеля, но сейчас многие выдающиеся
мыслители, занимающиеся историей и философией науки, пришли к утверждениям,
очень сходным с цитированными мыслями Гегеля. Так, Дюгем в своей замечательной
книге «Физическая теория, ее цель и строение» утверждает, что «экспериментум
круцис» (эксперимент креста, где опровержение одной стороны доказывает
справедливость ей противоположной) вещь в физике невозможная, т. к. в истории
физики неоднократны случаи, где при споре по поводу какого-нибудь вопроса
оказывалось ложным не то, что оспаривала одна из спорящих сторон, а то, в чем
не сомневались обе спорящих стороны. Примерно то же показывает в биологии и
Радль в его замечательной истории биологических учений.

В этом и заключается отрицание безусловной значимости закона исключенного
третьего, одного из столпов формальной логики, подлинной диалектической
логикой. Отрицание закона исключенного третьего лежит и в основе математической
школы интуитивизма. Она опирается и на математические факты. Кажется очевидным,
что могут быть сходящиеся или расходящиеся ряды. А нашли такие ряды, которые не
являются ни сходящимися, ни расходящимися.

Совершенно прав Гегель, что антиномичность, противоречивость пронизывает все
наше мышление, но мы с этим не должны примиряться, а работать над преодолением
этих противоречий. Разумеется, из того, что все реальное противоречиво, не
значит, что все противоречивое реально и заслуживает рассмотрения. Надо
различать между противоречием и бессмыслицей: по-немецки это звучит видерштрейт
и видерзинн, но надо сказать, что отличить бессмыслицу от противоречия не
всегда бывает легко. Одним из излюбленных доказательств бессмыслицы религиозных
учений был догмат христианской церкви, что Бог един, но троичен в лицах. Явная
«бессмысленность» этого догмата послужила причиной возникновения многих ересей,
в частности как будто и причиной гибели Сервета. В самом деле: $3 \times 1 = 1$, явная
бессмыслица. Сейчас разрешение этого противоречия лежит в основе теории
множеств даже гениального Георга Кантора, который был убежденным католиком.
Приведенная формула нелепа только в области конечных величин. Для бесконечных
множеств соединение двух или нескольких множеств одинаковой мощности в одной
дает множество той же мощности. Например, множества всех четных и всех нечетных
вместе дают множество натуральных чисел. Наша таблица умножения неприменима к
бесконечному. Кстати, зачатки теории множеств имеются уже у Галилея.

4. Сверхъестественное

Это то, что не заключает внутренних противоречий, вполне понятно, но которое
выходит за пределы человеческого опыта. Недопущение сверхъестественного
является как будто основой всякого научного мышления. Но одним из как будто
совершенно достоверных выводов человеческого опыта будет: «Ничто не вечно под
луной», «все течет» по Гераклиту. И однако ученые и философы с каким-то
необыкновенным упорством ищут «покоящуюся ось в потоке явлений» будучи глубоко
убежденными (тут они сознательно или бессознательно следуют философу
Пармениду), что все истинно сущее неизменно. Одно из явных проявлений --- атомная
теория, принимающая, что, несмотря на непрерывные кажущиеся изменения тел, по
существу они неизменны. Но ведь это же противоречит всему нашему опыту. Мы
знаем, что самые твердые тела от трения изнашиваются, а тут мельчайшие частицы
двигаются нередко с большой быстротой, сталкиваются и вечно остаются
неизменными. Это совершенно сверхъестественно, но мы к этому привыкли, а
известно, что можно привыкнуть к любому абсурду. Не так думали великие
мыслители прошлого. Возьмем опять Ньютона (Каблуков. Ньютон как химик. «Под
знаменем марксизма», 1937, № 4, с. 205): «При размышлении о всех этих вещах, ---
говорит Ньютон, --- мне кажется вероятным, что вначале Бог сотворил материю в
виде твердых, непроницаемых, подвижных, обладающих массой частиц таких размеров
и форм, с такими свойствами и в таких относительных количествах, какие пригодны
для той цели, для которой он их создал; эти первоначальные твердые частицы
несравненно тверже, чем какое бы то ни было пористое тело, составленное из них;
они так тверды, что никогда не снашиваются и не раздробляются на части, ибо
обыкновенная сила не способна разделить то, что сам Бог сделал единым при
первом творении». Неудивительно, что творцы новой атомной теории или духовные
лица (священник Гассенди, иезуит Боскович), или искренне религиозные люди
(квакер Дальтон, протестант Ньютон, настроенный резко антикатолически) и в
античной философии идеалистические философы (Пифагор, Платон) отнюдь не
отрицали атомизма. Первая математическая атомная теория Босковича, конечно,
ближе к Пифагору, чем к Демокриту. А Демокрит? Разве не связан
материализм теснейшим образом с атомной теорией? Конечно нет, а лишь с
пониманием атомов и той ролью, которая им приписывается в мироздании. По
вопросу дискретного строения тел в античном мире, видимо, не было резких
противоречий. Возьмем того же Диогена Лаэрция. Демокрит, по Диогену Лаэрцию,
был почитателем пифагорейцев и Пифагора, и Трасилл, который свел все сочинения
Демокрита, собрал также и все сочинения Платона. Разница заключается в том, что
для Демокрита атомы были конечной реальностью, а для Платона лишь кирпичами
видимой реальности. Демокрит был совершенно чужд холизму (целое определяет
поведение частей): развитие мира --- следствия случая, а не имманентного закона;
отрицание того, что позднее называлось финальными и формальными причинами.
Позднейший механический материализм усвоил от Демокрита принятие механической
необходимости (в этом смысле он отрицает случайность в природе в противовес
индетерминисту Эпикуру) наряду с отрицанием финальной причинности (в этом
смысле, говоря словами Данте, «вот тот, кто мир случайным полагает, философ
знаменитый Демокрит»).

Вот эти положения, существенные для истинного материализма, действительно
лишены всякой «сверхъестественности», но признание инвариантных атомов могло
зародиться только на объективно идеалистической почве. А потом привыкли
считать, что атомная теория --- цитадель материализма.

Такую же религиозную основу имеет происхождение и другого великого инварианта ---
закона сохранения энергии. Предшественниками его, как известно, были Декарт и
Лейбниц, и оба исходили из того, что Бог вложил элемент своей неизменности в
природу в форме сохранения энергии (формулировали они его не так, как сейчас,
но эта мысль и двигала искание закона). В XIX в. к инвариантности элементов
мироздания привыкли, и из двух авторов закона сохранения энергии в современном
виде Гельмгольц во всяком случае не отличался религиозностью, а другой, Роберт
Майер, был глубоко религиозным человеком. Как известно, и Р. Майер и Гельмгольц
с трудом добились того, чтобы их работы по этому закону были напечатаны; но
даже когда этот закон был признан и на съезде в Инсбруке Р. Майер позволил себе
несколько фраз в религиозном смысле (в 1869 г.), этим воспользовались его
противники, и К. Фогт в газете, намекнул, что это говорит человек, выпущенный
из дома умалишенных, где он одно время действительно был, видимо, не без
содействия родственников (Тимирязев К. А. Избр. соч. в 4 т., т. I, 1949, с.
132). Вспомним, что и Ньютона многие считали сумасшедшим.

5. Невероятное

Невероятное, точнее, чрезвычайно маловероятное. Есть известный рассказ об одном
аббате, хорошо разбиравшемся в основах теории вероятности. Он вошел в таверну,
где играли в кости. Кто-то бросавший кости получил на всех трех костях три раза
подряд по шесть очков. «Кости фальшивые!» --- воскликнул аббат и оказался прав
(их наливали свинцом с одной стороны). Мог ли он ошибиться? Конечно, мог. В
данном случае при правильных костях вероятность такого результата равна
$1:6^9$, т. е. $1:10$ миллионов (приблизительно). Поэтому если бы такая серия
бросаний повторялась несколько миллионов раз, то результат не был бы
удивителен. Но практически мы с такими вероятностями не считаемся. В США
население около 200 миллионов, а погибает ежегодно от автомобилей 50 тысяч
человек, следовательно, средняя вероятность для американца погибнуть от
автомобиля в течение года примерно $1:4000$, а каждый день приблизительно
$1 : 1 500 000$. Но никто же из американцев не считается серьезно с опасностью
погибнуть в ближайший день. Ясно, что ни один здравомыслящий человек не будет
планировать свое поведение из ожидания исключительно маловероятных событий. И,
однако, в науке есть такие странные люди --- большинство механических
материалистов. Второй закон термодинамики является одним из очень важных
достижений физики XIX в., но в соединении с предположением о бесконечности
Вселенной в пространстве и времени он приводит к представлению о тепловой
смерти Вселенной, которая неизбежно должна бы произойти к настоящему времени,
если бы этот закон был абсолютен. А так как Вселенная отнюдь не находится в
состоянии тепловой смерти, то придется допустить или ограниченность ее
состояния во времени (часы Вселенной были когда-то заведены), или наличие
других процессов иной направленности. Но тогда второе начало термодинамики
теряет свою универсальность. Знаменитый физик Л. Больцман (1844--1906) предложил
третий выход. Основываясь на статистическом характере второго начала
термодинамики, он (Философский словарь, 1963, с. 53) «для преодоления
идеалистической гипотезы „тепловой смерти Вселенной" выдвинул свою
флуктуационную гипотезу, согласно которой общее равновесное состояние мира в
целом постоянно и неизбежно нарушается в отдельных областях гигантскими
флуктуациями (отклонениями), приводящими к неравновесному процессу развития
отдельных миров. По своему мировоззрению Больцман был убежденным материалистом,
критиковал энергетизм и махизм». Мы знаем, что Больцман высоко ценил Дарвина и
даже высказался, что наш (XIX век) есть век механического понимания природы,
век Дарвина. Мнение Больцмана поддерживал и наш известный, исключительно
образованный физик Хвольсон. В своей статье «Можно ли прилагать законы физики
ко Вселенной» он развивает аналогичную мысль, что так как законы физики по
крайней мере, как правило, являются статистическими законами, то всегда в
бесконечной Вселенной мы можем найти такой уголок, где законы эти в силу
случайных отклонений (флуктуации) оказываются неприложимыми, все там идет
навыворот. Весь наблюдаемый нами мир есть результат накопления колоссального
количества случайностей. Для области живого уже давно (Ауэрбах) сформулирован
принцип эктропизма --- концентрации, а не рассеяния энергии. Таким образом, вся
наша Метагалактика --- огромный невероятный кусок во Вселенной, где в силу
накопления случайностей второй закон термодинамики не соблюдается, и в этом
невероятном участке Вселенной имеется еще более невероятная область --- область
живого. Все построено на теории полной невероятности. Это дает объяснение и
тому, что многие выдающиеся ученые, представители точных наук, так высоко
ценили учение о естественном отборе. Против этого учения неоднократно
выдвигались серьезные возражения, отмечавшие, что процесс эволюции на основе
накопления случайных изменений совершенно невероятен: и не хватает материала
для отбора (в особенности полового отбора), нет никакого соответствия между
темпами размножения и темпами эволюции, и совершенно невозможно себе
представить, чтобы путем накопления поломок достаточно совершенного органа
можно было получить более совершенный орган. На это многие дарвинисты говорили,
что ряд математиков или математически образованных ученых принимают это учение,
значит, математические возражения несущественны. На это можно ответить: уже в
области физики и астрономии ученые, подобные Больцману, принимают абсолютно
невероятное. Аргументы о невероятности на таких людей подействовать не могут.
Это --- Тертуллианы наизнанку. Тот говорил: «Я верю (в догматы веры), потому что
это абсурд (противоречит данным материалистического мировоззрения)». Эти
говорят: «Мы принимаем, что наш мир совершенно невероятен; но мы готовы жить в
невероятном мире, чтобы не допустить проникновения в науку идеалистических
воззрений». Оба наиболее ценным считают не свободное мышление, а подчинение
определенным догматам, и в этом смысле оба --- представители людей, сознательно
верующих в невероятное или чудесное. А те лица, которые наивно думают, что
взгляды Больцмана, Дарвина и других материалистов целиком основаны на вполне
рациональных научных данных, являются выразителями слепой веры.

6. Антиномичность в науке и религии

От противоречий, абсурдов, невероятного не свободна оказывается и наука, мнящая
себя совершенно свободной от всякого суеверия. Как к этому относиться? Первый
путь --- примириться с неизбежным и считать, что антиномичность неразрешима. Для
определенной области так думал Кант. Один из комментаторов Канта, Файхингер,
опубликовал книгу под заглавием «Философия фикции» (ди философи дер «Альс об»),
где доказывал, что все основные понятия науки настолько противоречивы, что
могут считаться фикциями, но надо выбирать полезные фикции. Эта философия вовсе
не так редка, как может показаться на первый взгляд. Наш Пушкин сказал до
Файхингера: «Тьмы низких истин нам дороже нас возвышающий обман». У Горького мы
читаем: «Если к правде святой мир дорогу найти не сумеет, честь безумцу,
который навеет человечеству сон золотой». И, наконец, Ницше (вернее, Ницше
хронологически занимает середину между Пушкиным и Горьким) выразился так:
«Истина есть наиболее целесообразное заблуждение». Но в этом заключается как
будто различие между Тертуллианом и материалистически настроенными учеными.
Позиция Тертуллиана отдаляет его от науки, механический же материализм при его
слабой философской обоснованности дал, несомненно, очень много науке. Но
Тертуллиан, вероятно, бесплодный в науке, не был бесплоден в области этики. Ему
наряду с Августином принадлежит едва ли не первый протест против смертной
казни. В античные дохристианские времена никто как будто до этого не доходил.
Может быть, Тертуллиан, подобно Августину, не удержался на этой позиции, но тут
уж действовали иные причины.

Другой путь --- искать выход из антиномий в новом синтезе, то, что как было
указано выше, с особой отчетливостью выразил Гегель. И вот мы видим, что
позиция Тертуллиана вовсе не была единственной и даже типичной для христианской
церкви и что наряду с ней было другое, более мощное направление, стремившееся к
примирению противоречий, к синтезу всех источников знания, к примирению знания
и веры. Другой, более великий представитель патристики, Августин, определенно
указывает черты близости христианской и языческой философии (в частности,
Платона). Он был продолжателем направления одного из ранних апологетов
христианства, Юстина-мученика (казнен в Риме ок. 166 г.). Цитирую по «Истории
философии» (под ред. Г. Ф. Александрова и проч., 1941, том I, с. 385--387):
«Юстин доказывал, что почти все содержание христианского учения уже имеется в
языческой философии. И это потому, что у христианства и философии один и тот же
источник --- божественный логос, разлитый во всем мире. В Христе этот логос
только проявился во всей полноте. К христианам Юстин относил всех тех, кто
прожил свою жизнь „с логосом". Таковы из греков --- Гераклит и Сократ.
Теоретически логос признавали также стоики. Юстин имел большое влияние на
позднейших „отцов церкви" и на дальнейшее развитие христианской идеологии».
Здесь огромную роль сыграла знаменитая александрийская школа. Александрия в то
время была центром величайшей в античности точной
науки (Александрийский музей), там же был важнейший центр еврейской «диаспоры»,
в которой складывалась особая, иудейско-эллинистическая культура (виднейший
представитель --- Филон), там жил епископ Климент (ок. 150--215 гг.)
Александрийский, который «развил теорию объединения веры и знания, которая была
принята христианской церковью» (Ист. филос, с. 389), со своим преемником
Оригеном. Все это были последователи Платона. Был и ряд других вполне
ортодоксальных представителей этой школы. У нас обычно, когда вспоминают раннее
христианство, упоминают не о Клименте, а о Кирилле Александрийском,
представителе совсем другого направления, ожесточенном враге эллинской культуры
(гибель Гипатии, сожжение Александрийской библиотеки). Их не так трудно спутать
и по сходству имен, и по месту их деятельности, и потому, что оба причислены к
лику святых. Но не надо забывать, что Кирилл действовал в эпоху после неудачной
языческой реакции императора Юлиана Отступника, преследовавшего христиан.
Неудивительно, что реакцией на языческого императора, объявившего войну
христианству, было возникновение убеждения о принципиальной противности
христианского учения и языческой культуры. Это было повторение сходного
процесса. Основоположником направления, враждебного язычеству (наиболее яркий
пример --- Тертуллиан), был Татиан, который был слушателем Юстина-мученика.
«После казни Юстина Татиан перебрался в Сирию и отошел от церкви, усвоив
осужденные ею гностические воззрения» (Ист. филос, с. 387). И здесь казнь
почитаемого учителя заставила Татиана осудить вместе с палачами учителя всю
языческую культуру. Но победила в раннем христианстве линия Климента. Опять в
«Ист. филос», с. 389: «По Клименту, нет знания без веры и веры без знания.
Полная гармония их требует изучения всего круга человеческих знаний: „семи
свободных искусств". Никакой несовместимости между языческой философией и
христианским учением, согласно Клименту, нет: это как две ветви одного и того
же ствола. Истины христианства согласны с учением лучших из язычников.
Философия представляет собой как бы пропедевтику, преддверие христианства. В
философии истина содержится не целиком в одной какой-либо школе, а по частям во
всех. Хотя отличительным признаком подлинной науки является ее совпадение с
учением веры, однако, с другой стороны, истинное содержание самого писания
устанавливается только философским изучением. Главным приемом для введения
философии в христианство было у Климента, как и у филона, „аллегорическое"
объяснение Священного Писания».

Это аллегорическое толкование Писания через известного епископа миланского
Амвросия перешло к Августину (с. 392). В «Истории философии» отмечается, что
сочинение Августина «О граде божием», несмотря на наивность исторической
концепции, имело большое историческое значение как попытка дать обзор истории
человечества в целом, как попытка философии истории. На с. 396 читаем:
«Христианская церковь, искаженно отобразив, как в кривом зеркале, эллинскую
культуру, сберегла ее в этом виде до нового времени, когда постепенно античная
культура стала оживать в ее подлинном виде. Вклады самих патриотических
писателей в науку ничтожны и не могут идти ни в
какое сравнение с достижениями античного мира». Последнее замечание верно, но
есть ли здесь вина христианства? Подлинная античная, эллинская культура в своих
высших достижениях не была понятна солдафонскому Риму. Первый пожар
Александрийской библиотеки был во время взятия Александрии Юлием Цезарем
(несомненно, одним из культурнейших людей в римском понимании термина
культура). Убыток был до известной степени возмещен Антонием, который галантно
поднес Клеопатре огромное количество рукописей из разграбленного Антонием
Пергама. Но все это собрание после гибели Клеопатры отправилось с триумфальным
поездом в Рим, где и было сожжено. Государственные дотации Александрийскому
музею, позволявшие ученым спокойно работать, были прекращены. Наконец, с севера
надвигались варвары. И здесь мы читаем в «Истории философии», с. 394; «Учение
Августина о предопределении было религиозным фатализмом. Для христиан оно
служило идеологической опорой в тяжелой борьбе, какую начиная с V в. пришлось
вести западной церкви с нахлынувшими на Европу и Северную Африку варварскими
народами. Вера в предопределение и возведение к воле божества каждого действия
как отдельного христианина, так и всей церкви придавали христианской церкви
сплоченность и фанатическое упорство».

Экономическое потрясение Западной Европы и Северной Африки, вызванное
нашествием варваров и крушением Римской империи, было основной причиной того
упадка культуры, который характеризует «века мрака» (примерно до 1000 г.). Если
бы не было христианской церкви, сохранившейся благодаря своему «фанатизму»,
если бы в этой церкви и ее монастырях не тлели ростки, изучавшиеся лучшими
представителями культуры --- монахами, то и возрождение наук (которому
чрезвычайно помогли и сохраненные арабами и греками элементы эллинской
культуры) не было бы возможно. И потому конец первой книги «Истории философии»:
«Главное содержание патристики --- разработка религиозной идеологии --- имела для
научного прогресса человечества отрицательное значение, служила орудием
косности и застоя» --- совершенно не соответствует истине. Как могла бы
возродиться античная культура, если бы сохранилась только римская традиция?
Ведь средневековая философия (вершиной которой был доминиканец Фома Аквинат)
основывалась на Аристотеле (которого привели к согласию с христианской
идеологией), а Возрождение связано с именами Платона и Пифагора, а отнюдь не
Эпикура и Лукреция.

Но у Августина, как и Тертуллиана, есть как будто совершенно отрицательные
взгляды. Несмотря на то что оба они высказали прогрессивную мысль, что смертная
казнь несовместима с христианским учением, оба они не удержались на этой
позиции (История философии, с. 395): «Августин ревностно отстаивал право церкви
на принуждение в делах веры на том же основании, что принуждение к „истине"
вовсе не есть насилие, а забота о благе принуждаемого. Учение Августина
получило зловещий характер, превратившись в реальность церковной практики. Так
как всякий еретик будет вечно мучиться за гробом, то лучше ему претерпеть
сожжение здесь, на земле (хотя и это не вполне обеспечивает его от загробных
мук)».

Но ведь это как бы предвидение современных «прогрессивных» взглядов, что
насилие над вредными учениями не есть нарушение свободы. Как раньше боролись с
еретиками, так теперь борются с «ревизионистами», исходя из того положения, что
единодушие необходимо в трагические периоды человеческой истории. И смертная
казнь в мирное время, практически отсутствовавшая в дореволюционной России
(подвергались казни только покушавшиеся на царя), восстановлена революционной
властью в неслыханном размере и сохраняется даже тогда, когда можно сказать,
что трагические времена миновали.

Наука в Западной Европе в течение темного периода, последовавшего за разорением
Европы варварами, теплилась в монастырях, но не следует думать, что она велась
тайком в противовес официальной католической церкви. Нет, ведь начало подъема
цивилизации обычно считается совпадающим с началом XI в., когда на папский
престол вступил Герберт под именем Сильвестра II. Этот был ученейший муж своего
времени, получивший образование в мусульманском университете в Кордове.

В XII-XIII вв. именно католической церковью были основаны университеты. В XIII
в. францисканский монах Роджер Бэкон положил основание индуктивной логике.
Силлогистическая логика получила наибольшее свое развитие в системе Петра
Испанского, португальского ученого, который правил 8 месяцев под именем папы
Иоанна XXI (Минто. Индукт. и дедукт. логика, с. 18--19). Конечно, была всегда и
ретроградная оппозиция, восходящая идейно к Татиану.

Противник Герберта, епископ Оттон, уверял, что Герберт обязан своим высоким
положением только союзу со злыми духами (Уэвель. История индуктивных наук, т.
I, 1867, с. 582). Некоторые историки литературы видят в Герберте прообраз
Фауста. Роджер Бэкон то попадал в тюрьму при папах одного направления, то
освобождался при прогрессивных папах и все-таки умер на свободе в старости. Но
невежественная толпа видела во всех ученых колдунов и соучастников нечистой
силы, и от этого обвинения не были защищены даже служители религии: Фома
Аквинат, Роджер Бэкон, Михаил Скот, Роберт Гростет, епископ Линкольнский,
Альберт Великий, епископ Регенсбургский, папы Сильвестр II и Григорий VII
(Уэвель, там же, с. 381).

Как было указано, такими обвинителями были и некоторые духовные лица. Ясно, что
по отношению к дохристианским деятелям обвинение в колдовстве было еще более
естественным: сюда попали Аристотель, Соломон, Иосиф, Пифагор, Вергилий (там
же).

Тенденция синтеза науки и религии получила свое наибольшее выражение в трудах
Фомы Аквината, и мы знаем, что и сейчас философия Фомы (неотомизм, или просто
томизм) является далеко не исчезнувшим учением. Во время Возрождения борьба шла
не между наукой и религией (что будет еще разъяснено дальше), а между
консервативным направлением, связанным с Аристотелем, и новым, связанным с
именами Пифагора и Платона.

7. Игнорирование антиномий

В предыдущем параграфе было рассмотрено два пути преодоления антиномий:
признание их непреодолимыми, стремление к преодолению противоречия, примирение
разума с верой. Но есть еще третий путь --- полное их игнорирование. Он основан
на том гносеологическом постулате, что мы имеем уже очень много окончательно
установленных истин и что прогресс науки заключается в постепенном наращивании
таких абсолютных истин и в постепенном распространении их на неограниченно
большую область бытия. Это мнение выражено классическим представителем
детерминизма Лапласом во введении к «Аналитической теории вероятностей».
Цитирую по статье Елены Эйльштейн: «Лаплас, Энгельс и наши современники» в
сборнике «Закон, необходимость, вероятность». 1967, с. 235--236: «Все явления,
даже те, которые по своей незначительности как будто не зависят от великих
законов природы, суть столь же неизбежные следствия этих законов, как обращение
Солнца. Не зная уз, соединяющих их с системой мира в целом, их приписывают
конечным причинам или случаю, в зависимости от того, происходили ли и следовали
ли они одно за другим с известной правильностью или же без видимого порядка; но
эти мнимые причины отбрасывались по мере того, как расширялись границы нашего
знания, и совершенно исчезли перед здравой философией, которая видит в них лишь
проявление неведения. Мы должны рассматривать настоящее состояние Вселенной как
следствие ее предыдущего состояния и как причину последующего. Ум, которому
были бы известны для какого-либо данного момента все силы, одушевляющие
природу, и относительное положение всех ее составных частей, если бы вдобавок
он оказался достаточно обширным, чтобы подчинить эти данные анализу, обнял бы в
одной формуле движения величайших тел Вселенной наравне с движениями легчайших
атомов: не осталось бы ничего, что было бы недостоверно, и будущее, так же как
и прошедшее, предстало бы перед его взором. Ум человеческий в совершенстве,
которое он придал астрономии, дает нам представление о слабом наброске того
разума. Его открытия в механике и геометрии в соединении с открытием всемирного
тяготения сделали его способным понимать под одними и теми же аналитическими
выражениями прошедшие и будущие состояния мировой системы. Применяя тот же
метод к некоторым другим объектам знания, нашему разуму удалось подвести
наблюдаемые явления под общие законы и предвидеть явления, которые будут
вызваны данными условиями. Все усилия духа в поисках истины постоянно стремятся
приблизить его к разуму, о котором мы только что упоминали, но от которого он
останется всегда бесконечно далеким. Это стремление, свойственное роду
человеческому, возвышает его над животными; и успехи его в этом направлении
различают нации и века и составляют их истинную славу».

Детерминизм Лапласа вызывал самую разнообразную реакцию. Как я указывал в
параграфе (об абсурдном), наши ортодоксальные философы обвиняли Лапласа в
фатализме и даже в уклоне в мистицизм. Е. Эйльштейн, из статьи которой взята
эта цитата, считает, что дело в неправильном понимании высказывания Энгельса.
Современные физики в большинстве своем полностью отрицают лапласовский
детерминизм и, склонны на элементарном уровне к индетерминизму. Но такой
выдающийся физик, как Альберт Эйнштейн, до конца своей жизни сохранил верность
лапласовскому детерминизму. Он заканчивает свою статью «Основы теоретической
физики» (Эйнштейн. Физика и реальность. Сборник статей. 1965, с. 76) следующими
словами: «Некоторые физики, в том числе и я сам, не могут поверить, что мы раз
и навсегда должны отказаться от идеи прямого изображения физической реальности
в пространстве и времени, или что мы должны согласиться с мнением, будто
явления в природе подобны игре случая. Каждому дозволено выбрать направление
приложения своих усилий, и каждый человек может найти утешение в прекрасном
изречении Лессинга, что поиск истины значительно ценнее, чем обладание ею». Мы
видим, что хотя Эйнштейн продолжает оставаться верным дифференциальному закону
(современное состояние полностью определяет последующее), но он не скрывает,
что это дело веры, субъективной индукции, и принципиально допускает иное
решение проблемы. В этом его отличие от Лапласа, который считал свой постулат
абсолютной истиной. Но, может быть, во времена Лапласа и не было оснований
сомневаться в верности его взглядов, а новые открытия заставили пересмотреть
вопрос? Нет, как совершенно правильно отмечает в статье Е. Эйльштейн,
детерминизм Лапласа имеет две стороны: онтологическую и гносеологическую, но
она дальше разбирает этот вопрос недостаточно широко.

В чем гносеологическая ценность высказывания Лапласа? В возможности
предвидеть явления. Эта возможность блестяще оправдалась в астрономии,
механике и других точных науках, и отсюда Лаплас делает заключение, что тем же
путем принципиально можно предвидеть все. Принципиально, но не фактически, так
как сам Лаплас отчетливо сознает, что этот его постулат недоказуем и
неопровержим, так как там, где нельзя произвести строгих математических
расчетов, предвидение событий невозможно. В этом и заключается, как
справедливо указывает, следуя многим авторам (с. 239), Е. Эйльштейн,
отличие лапласовского детерминизма (механического) от фатализма, другой
формы детерминизма (с. 329): «Фатализм утверждает, что каждое событие
детерминировано отдельно, одной и той же причиной, действующей вне сферы
материальной природы. Таким образом, фатализм провозглашает фактическую
независимость явлений --- механический же детерминизм Лапласа является
интегральным, основанным на взаимодействии всех решительно частиц
материального мира. Мы знаем, что фатализм --- очень древнее понимание мира и,
в частности, на фатализме основана вся астрология, составление гороскопов,
которую не отрицали и такие революционеры в науке, как Коперник и Кеплер.
Астрология основывалась на экстраполяции безусловных фактов: влияние луны и
солнца на морские приливы и астрологический принцип всемирного тяготения,
отвергнутый Галилеем и реабилитированный Ньютоном, использовался и
Лапласом, хотя он был и не механической природы. Астрологи приводили и
многие случаи удавшихся предсказаний, а неудачные приписывали или
злоупотреблениям, или недостаточному совершенству методов. Гадание,
предвидение будущего и сейчас
широко распространены и не только в капиталистическом мире. Вероятно, не
проделано еще достаточно большой работы, чтобы показать, что процент
удавшихся предсказаний именно таков, какой надо ожидать, если удачные
предсказания являются делом случая. Поэтому астрологические методы могут
быть опровергнуты, лапласовский же детерминизм неопровержим
непосредственно. Но косвенно он может быть опровергнут, и это опровержение
известно каждому, конечно, и самому Лапласу. Ведь детерминизм Лапласа
связан и с его онтологией: принципиальным отрицанием индетерминизма (случая в
подлинном аристотельском смысле: того, что может и не быть) и отрицанием
конечных причин. Но можем ли мы на основании конечных причин делать
прогнозы? Каждый из нас делает. Я приду к вам послезавтра в десять часов
утра и, как правило, этот прогноз осуществляется, причем независимо от
разнообразия условий. Предположим, для осуществления своего прогноза вы
намерены были выйти за час и ехать трамваем. Но вас кто-то задержал на
полчаса или вы проспали. Вместо трамвая вы едете на такси. Осуществляется
совершенно иной комплекс движения материальных частей, подчиненный конечной
причине: вашему желанию прийти вовремя. Ссылаясь на вышеприведенную цитату,
можно сказать, что фатализм, утверждающий, что каждое событие
детерминировано отдельной причиной, действующей независимо от материальной
природы, вовсе не глупость, так как мы этот фатализм осуществляем
ежедневно. А экстраполяция этого законного подхода к явлениям на
бесконечность не менее законна (вернее, не более беззаконна), чем
экстраполяция механических законов на всю Вселенную.

Но любопытно, что Лаплас делал ошибки с экстраполяцией положений,
справедливых в определенной области, и там, где он выступал не как
философ-дилетант, но и в тех областях, где он был бесспорный выдающийся
специалист. Кроме астрономии и механики, он оставил глубокий след в теории
вероятностей и в исчислении бесконечно малых. И вот современный математик Д.
Пойа в очень интересной книге «Математика и правдоподобные рассуждения»
(Москва, ИЛ, 1957), останавливается на попытке Лапласа связать индукцию с
вероятностью (с. 395--398). «Когда вероятность простого события неизвестна, то
можно предполагать ее равною всем числовым значениям от нуля до единицы», ---
говорит Лаплас в «Опыте философии теории вероятностей»; «Это равное
распределение незнания», --- насмехаются его оппоненты. Пойа указывает на
явно глупые применения этого принципа (с. 398). «Эти применения кажутся
глупыми, но нет ничего глупее следующего применения, принадлежащего самому
Лапласу. „Если отнести древнейшую историческую эпоху за пять тысяч лет, или за
1 826 213 дней, назад и принять во внимание, что солнце постоянно восходило за
этот промежуток времени при каждой смене суток, то будет 1 826 214 шансов
против одного за то, что оно взойдет и завтра"». Пойа прибавляет: «Я,
конечно, остерегся бы предложить такое пари норвежскому коллеге, который мог
бы для обоих устроить воздушное путешествие в какое-нибудь место за
полярным кругом». Лаплас, конечно, знал, что за полярным кругом Солнце
зимой не восходит, но он об этом просто позабыл.

Еще любопытнее другая «забывчивость» Лапласа, касающаяся непосредственно его
формулировки детерминизма. Из его формулировки вытекает, что если мы знаем
исходное состояние и все дифференциальные уравнения, описывающие
зависимость между элементами бытия, то на каждый следующий момент мы можем
получить только одно решение. Но на самом деле в ряде случаев мы имеем так
называемые особые точки, где происходит как бы разветвление и получается не
одно решение, а несколько. На это обратил внимание французский математик
Буссинеск в интересной книге под заглавием «Примирение подлинного
механического детерминизма с существованием жизни и нравственной свободой».

Эта книга --- изложение своеобразного дуализма. Вне жизни нет особых точек, все
развивается сообразно механическому детерминизму. А жизнь --- это область
приложения особых интегралов, где существуют особые точки, где механический
детерминизм недостаточен. Что же решает, каким путем пойдет процесс,
дошедший до особой точки? Буссинеск склонен принимать специфически
витальные направляющие силы, но, конечно, можно здесь предоставить роль и
чистому случаю. Таким образом исключение случая и конечных причин сделано
Лапласом не потому, что он был хорошим математиком, а потому что он был,
скажем мягко, не слишком выдающимся философом. В предисловии к книге
Буссинеска профессор Жаке указывает, что эта книга, освобождая нашу мысль от
пут механического детерминизма, делает необязательной тягостную дилемму:
выбирать между трансцендентальным идеализмом Канта и предустановленной
гармонией Лейбница.

Как отнеслись математики к книге Буссинеска? К ней хорошо отнесся
выдающийся математик А. Пуанкаре, а наша соотечественница, Софья
Ковалевская, которая была не только математиком, но и писателем, написала
роман (по-шведски; кажется, есть русский перевод) «Борьба за счастье», где в
популярной форме изложила мысль, что в каждой человеческой судьбе есть
«точки разветвления» и выбор пути не предопределен совершенно прошлым. Но,
может быть, особые интегралы стали известны после Лапласа? Были ли они
известны до Лапласа, не знаю, но хорошо известно, что в разработке учения об
особых интегралах особое значение имели как раз работы Лапласа. Он о
собственных достижениях позабыл в угоду любезной ему метафизики. Лаплас,
по-видимому, и не подозревал, что он просто принял одну из тез великой
антиномии необходимости и свободы, которая волновала мыслителей
средневековья и которая была отчетливо формулирована Кантом. В Средние века
богословы спорили: если мы отрицаем свободу воли человека, то почему мы его
считаем ответственным за его поступки (этот спор велся и среди евреев между
саддукеями и фарисеями), а если мы признаем его свободу, то признаем и
индетерминизм, а значит, ограничим всеведение Божие.

Лаплас встал на сторону детерминистов и, упразднив Бога как реальность, он
возвел его как фикцию в абсолют. Всякая иная метафизика казалась ему
совершенным сумасбродством. Вот как он пишет про гениального Кеплера
(цитирую по: Уэвель. Ист. индукт. наук, I, 1867, с. 533): «Печально за
человеческий ум видеть, как этот великий человек даже в своих последних
произведениях с наслаждением предается своим химерическим умствованиям и
считает их душой и жизнью астрономии». Уэвель на той же странице приводит без
указания авторов и другие аналогичные высказывания: «Этот успех (Кеплера)
может устрашить тех, кто привык считать опыт и строгую индукцию
единственным средством успешно исследовать природу»; «Удивительное счастье
Кеплера схватывать истину среди самых диких и нелепых теорий»; «Опасность
следовать его методу в отыскании истины». Сам Уэвель, хотя он и является
убежденным сторонником индукции и не разделяет метафизики Кеплера, однако,
судит более объективно, с. 537: «Мистические стороны его мнений, как,
например, его вера в астрологию, его убеждение, что земля есть животное, и
множество туманных нравственных духовных и материальных представлений о
силах, управлявших, по его мнению, Вселенной, --- по-видимому, не только не
мешали его открытию, но скорее подталкивали его изобретательность и
воодушевляли его труды. В самом деле, где ум обладает ясными научными
идеями об одном предмете, там мистицизм по другим предметам, кажется, вовсе не
бывает неблагоприятным для успешного хода исследования. Я полагаю поэтому, что
мы можем видеть в характере Кеплера общие черты характера научного
открывателя, хотя некоторые из них преувеличены, а другие проведены слишком
слабо». Не забудем, что Кеплера особенно ценили и Маркс и Энгельс.

Лаплас в данном случае пал жертвой своей неумеренной преданности весьма
посредственной материалистической французской философии конца XVIII в.
Кажется, сейчас никто не считает этих философов крупными. Даже
основоположники марксизма при всей своей приверженности к материализму
искали философскую основу своего учения в немецкой идеалистической
философии, пытаясь только (с каким успехом, не будем пока говорить)
перевернуть ее на голову. Но великолепную оценку французской
материалистической философии конца XVIII в. (бравшей монополию на
«просветительство») дал наш поэт М. Ю. Лермонтов в изумительном по красоте
стихотворении «Последнее новоселье». Конечно, и Лермонтов хватил через
край, обозвав весь французский народ «жалким и пустым» (включая всех
французских мыслителей и ученых), переоценил он и Наполеона (ошибка совсем
простительная, принимая во внимание, что в такую ошибку впадали три великих
немца: Бетховен, Гете и Гегель), но в отношении французского материализма
вполне справедливы его слова:

Ты жалок потому, что вера, слава, гений,

Все, все великое, священное земли,

С насмешкой глупою ребяческих сомнений

Тобой растоптано в пыли.

Из славы сделал ты игрушку лицемеръя,

Из вольности --- орудье палача,

И все заветные, отцовские поверья

Ты им рубил, рубил с плеча.

Поэтому отдадим должное Лапласу как великому ученому, достойному
продолжателю еще более великого Ньютона, и не будем придавать значения его
философствованиям, следствию его убеждениям чувства,
а не разума. В них он не подчиненный только доводам разума ученый, а слепо
верующий весьма сомнительным догмам человек. «Орлам случается и ниже кур
спускаться, да курам никогда до облак не подняться». Но многие куриные
философы больше всего восхищаются куриной частью мировоззрения Лапласа.

Глава 2. ВРЕД ОТ ЦЕРКВИ ДЛЯ НАУКИ

1. Общая постановка вопроса

Все, что написано в предыдущей главе, показывает, что не только религия, но и
наука в лице своих выдающихся представителей не чужда суеверий и что они часто
не сознают, что высказываемые ими как абсолютные истины положения по своей
обоснованности ничуть не лучше тех нелепых догматов, которые выставляются
религиями. На это можно ответить, что религиозные суеверия особенно опасны тем,
что господствующие церкви, связанные с правительствами классов-угнетателей,
применяют такие средства борьбы, которые угнетают свободное развитие науки.
Поэтому многие религиозные люди считают, что не религия как таковая, а
богословие, теология господствующих церквей органически связаны по своему
догматизму с систематической борьбой со свободой науки. Этой мыслью проникнута,
например, книга Уайта «История войны науки с теологией», в двух томах (первое
издание в 1896 г., второе в 1960 г.): \emph{White A.D. A history of the
warfare of science with theology in Christendom.}

Конечно, в этом утверждении заключается большая доля истины, догматизация
всегда приводит к преследованию инакомыслящих. Но в данном случае нас
интересует вопрос, в какой мере церковь, в особенности католическая,
преследовала науку и в какой мере она затормозила ее развитие. Мысль Уайта
(который в свое время был посланником США в С.-Петербурге) выражена ясно на с.
XII его предисловия: «Мое убеждение состоит в том, что наука, хотя она
очевидно и победила догматическое богословие, основанное на библейских
текстах и древнем способе мышления, будет идти рука об руку с религией и
что, хотя теологический контроль будет продолжать уменьшаться, религия,
понимаемая в признании "Силы во Вселенной вне нас, стремящейся к
справедливости", и из любви к Богу и к ближнему, будет все усиливаться и
усиливаться, не только в ученых учреждениях Америки, но и во всем свете».

Как это ни может показаться странным, но сходные мысли высказал и А.
Эйнштейн в статье «Религия и наука», опубликованной в сборнике «Мое
мировоззрение» (англ. перев. 1935 г.), с. 27: «Этическое поведение человека
должно быть основано на симпатии, воспитании и социальных связях;
религиозного основания не требуется. Печально было бы, если бы человек был
ограничен лишь страхом наказания и надеждой на посмертную награду. Отсюда
ясно, почему церкви всегда боролись с наукой и преследовали адептов науки. С
другой стороны, я утверждаю, что космическое религиозное чувство является
сильнейшим и благороднейшим побуждением к научным исследованиями. Только те,
кто
понимают огромные усилия и прежде всего преданность делу, которые требуются для
тех, кто прокладывает новые пути к теоретической науке, могут понять всю
силу эмоций, из которых может произойти такое дело, отдаленное от
непосредственных требований жизни. Какое глубокое убеждение в
рациональности Вселенной и какую жажду познать хотя бы слабое отражение
разума, обнаруживаемого в мире, должны были иметь Кеплер и Ньютон, чтобы они
могли потратить годы одинокого труда в распутывании основ небесной
механики. Те люди, знакомство которых с научными исследованиями связано
преимущественно с практическими результатами, легко приобретают совершенно
превратные представления о складе ума людей, которые, будучи окружены
скептическим миром, устанавливают связь с людьми подобного им мышления, как бы
они ни были рассеяны по земле и по столетиям. Только тот, кто посвятил свою
жизнь сходным целям, может живо представить себе, что вдохновляло этих
людей и давало им силу оставаться верными своему призванию несмотря на
многочисленные провалы. Это --- космическое религиозное чувство, которое
давало людям такую силу. Один современник сказал, и не без основания, что в
современную материалистическую эпоху серьезными учеными могут быть только
глубоко религиозные люди». Как увидим дальше, Эйнштейн несколько изменил
свои взгляды в последние годы своей жизни, но пока ограничимся этой
цитатой. Можно возразить, что понимание Эйнштейном религии не имеет ничего
общего с обычным пониманием и что отрицательное отношение его к церковной
религии выражено вполне ясно.

В числе многочисленных жертв святой инквизиции были, конечно, и ученые, но они
составляли небольшой процент общего числа. Наиболее знаменитые случаи ---
Джордано Бруно и Г. Галилей. Их и разберем.

2. Дело Джордано Бруно

Как известно, Бруно (1548--1600) был доминиканским монахом, проведшим долгие
годы в странствиях, выступавший с лекциями на диспутах по богословию,
философии, интересовавшийся и общественно-политическими вопросами.
Неосторожно вернувшись в Венецию, он был там захвачен агентами инквизиции и
после нескольких лет заточения в тюрьме сожжен в Риме. За что его сожгли? За
его научные работы? В точных науках он не дал ничего сколько-нибудь
существенного. Но он был пламенным пропагандистом Коперника и еще больше
Николая Кузанского, которого он ставит чрезвычайно высоко и сравнивает с
Пифагором.

Николай Кузанский (1401--1464) был сторонником бесконечности Вселенной,
движения Земли и множества населенных миров, где могли обитать существа,
более совершенные, чем человек. Бруно воспринял эти идеи Кузанского и в
этом отношении отличался от Коперника, который признавал ограниченность
Вселенной. И Кузанский и Бруно были склонны к пантеизму, и за это их обоих
упрекали противники. И Кузанский и Бруно были склонны к платонизму и
гилозоизму, т. е. учению об одухотворенности материи, оба являются
представителями
диалектического мышления с тезисом о совпадении противоположностей (см.
статья «Николай Кузанский», Философский словарь, 1963, с. 309). В области
натурфилософии и философии вряд ли можно указать что-либо существенно
новое, что внес Бруно по сравнению с Кузанским. Кузанский был почти забыт в
течение несколько сотен лет, и даже в трехтомной «Истории философии» Гегеля он
вовсе не упоминается, хотя его последователю, Бруно, отведено достаточно
места. Сейчас положение изменилось: сочинения Кузанского издаются и на
Западе и у нас (Избранные философские сочинения, Москва, Госсоцэкгиз,
1937), ему отведено достаточно места и в «Истории философии» (т. II, 1941, с.
42--44), БСЭ и других изданиях. Выдающееся значение этого мыслителя не
отрицает сейчас как будто никто. Какова же судьба Кузанского? Он был
епископом, кардиналом, последние года провел в Риме, где, по свидетельству
многих, играл роль вице-папы. Он отнюдь не замыкался в ученой деятельности. Он
деятельно работал по объединению Восточной и Западной церквей, принимал
участие в организации похода христианских государств против турок, боролся с
массовым паломничеством к святым местам. Боролся с богослужебным
формализмом, суевериями, верой в чудеса, с конкубинатом духовенства. Он
подвергался и репрессиям, но не со стороны духовных властей, а со стороны
герцога Тирольского Сигизмунда, который некоторое время держал его в тюрьме за
то (по жалобам монахинь), что Кузанский слишком энергично боролся с
распущенностью, царившей в женских монастырях («Очерк жизни Кузанского»,
написанный Лопашовым в «Избранных сочинениях»). Хотя против Кузанского его
противники выступали с обвинениями в пантеизме, это не имело для него
никаких последствий и в индексе запрещенных книг сочинения Николая
Кузанского никогда не фигурировали. И после сожжения Бруно в 1600 г. не
было объявлено об осуждении учения Коперника. Это было сделано лишь в 1616 г.,
и поводом к этому были не сочинения самого Коперника или Бруно, а сочинение
кармелита П. Фоскарини (1615), где тот пытался доказать отсутствие
противоречий между доктриной Коперника и доктриной церкви. «По декрету
конгрегации индекса запрещенных книг от 5/III 1616 г., сочинение Фоскарини
было "совершенно запрещено и осуждено"» (Идельсон. Галилей в истории
астрономии. Вопросы истории естествознания и техники, вып. 16, 1964, с.
62). Вот ослушание этому постановлению было со временем и поставлено в вину
Галилею (о чем будет речь дальше), но никак не могло быть поставлено в вину
Бруно. При всех своих ужасных законах инквизиция достаточно строго
придерживалась принципа, что закон обратной силы не имеет и по
несуществующим законам не судили (в отличие от некоторых современных
правителей).

Как это указывается и в предисловиях к советским изданиям сочинений Бруно, он
был казнен не за свою научную, а за свою общественную деятельность.

Он решительно осуждал господствовавшие тогда в католической церкви порядки,
есть мнение, что он даже перешел в протестантство, но и там не поладил с
кальвинистами. Обладая необузданным темпераментом, он выступал (в отличие от
Кузанского) очень резко и тем нажил огромное количество врагов. В биографии
его, написанной
А. Штекли (Жизнь замечательных людей, выпуск 21(395), 1964), он выставлен
почти как воинствующий атеист, который еще в доминиканском монастыре
выбросил иконы (с. 19) и всю свою жизнь высказывал резко антирелигиозные
суждения. Образ жизни он вел, судя по указанной биографии, тоже совсем не
благочестивый, но к этому греху в то время относились снисходительно.
Будучи очень образованным человеком и выдающимся оратором, он, конечно, был
опасен непосредственно для верхушки католической церкви, и скорее можно
удивляться, что процесс длился так долго (почти восемь лет), чем тому, что он
не избежал костра. На костер отправляли тогда и лиц, не имевших никаких
научных заслуг и вполне ортодоксально мысливших, если находили их опасными или
слишком неприятными для церкви, причем, конечно, суровость репрессии
соответствовала серьезности политической ситуации и в значительной степени
характеру руководителей церкви. Поэтому Дж. Бруно можно зачислить в ряды
мучеников борьбы со всякого рода деспотизмом, за улучшение общественного
строя, за устранение злоупотреблений власть имущих, туда же, куда отнесем
Савонаролу, целый ряд францисканцев и прочих лиц, но к числу мучеников
науки мы его причислить не можем.

3. Дело Галилея

Обратимся теперь к Галилею (1564--1642). Если спросить среднего
интеллигента, кто такой Галилей, то он, конечно, ответит: борец за
коперниково мировоззрение, осужденный за это инквизицией и принужденный
отречься от своих взглядов. Верно это? Конечно, верно, но только это ---
неполная истина, сходна по своему значению с такой, например, истиной: Кант ---
немец, никогда не выезжавший из Кенигсберга. И в «Философском словаре» 1963 г.,
с. 86, где дано краткое изложение его учения (в общем правильное),
приводится только один его труд в качестве основных: «Диалоги о двух
главнейших системах мира --- птолемеевой и коперниковой» (1632). В «Истории
философии» 1941 г., т. II, изложение гораздо полнее и более приближается к
полной истине. Там правильно указано другое фундаментальное сочинение
Галилея: «Беседы и математические доказательства, касающиеся двух новых
отраслей науки» (1638) (правильнее: двух новых наук). Эти две науки,
основание которых положил Галилей (вполне сознававший свое значение как
основоположника этих наук) --- динамика и сопротивление материалов.
Основанием динамики он встал рядом с Архимедом, основателем статики, и
заслужил справедливо звание одного из величайших пионеров точных наук. Он дал
толчок развитию экспериментального метода, настаивал на математизации наук
(«Измеряй все доступное измерению и делай неизмеримое измеримым»),
чрезвычайно много сделал по борьбе с догматизмом перипатетиков и является
основателем того механического материализма, видным представителем которого
является Лаплас (в большей степени, чем Ньютон). У него есть мысли,
послужившие для развития теории вероятностей и теории множеств. Что
касается астрономии, то он способствовал продвижению коперниковских идей как
своими теоретическими работами (принцип
относительности Галилея), так и наблюдениями в телескоп (солнечные пятна,
фазы Венеры, спутники Юпитера и т. д.), но в развитии математической теории
гелиоцентрической системы он не только не участвовал, но даже
противодействовал принятию нового крупнейшего шага в этом направлении,
работам не менее гениального Кеплера, своего современника, работы которого ему
были известны и с которым он был в переписке. Здесь сыграло роль именно его
основное стремление --- разработка идей по механике, которой он занимался всю
свою сознательную жизнь. Поэтому сейчас можно сказать с полной
уверенностью: если бы исчезли «Диалоги», то астрономия не претерпела бы
существенной потери, так как к моменту их напечатания (1632) великий Кеплер
(1571--1630) уже закончил свой жизненный путь, а по линии Кеплера Галилей не дал
ничего, а вот если бы исчезли «Беседы», то развитие механики несомненно
задержалось бы, так как гении, подобные Галилею, рождаются нечасто.
Механике Галилей посвятил всю свою жизнь. Первые его работы посвящены
механике и прикладным задачам (по военной архитектуре и фортификации):
Теоремы о центрах тяжести твердых тел (около 1585 г.), Гидростатические
весы (1586), Трактат о движении (1590), где впервые --- расхождение с
аристотелевской динамикой, Механика (между 1593 и 1599 г.), где впервые
применяется термин «момент» силы. Все перечисленные произведения не были
опубликованы при жизни Галилея (Идельсон. Галилей в истории астрономии.
Вопросы истории естествознания и техники, вып. 16, 1964, с. 52). Они
соответствуют возрасту Галилея 21--35 лет. В этом интервале он был
профессором математики у себя на родине, в Пизе (1589--1592), где и наблюдал
качание маятника (паникадила) в Пизанском соборе и падение тел со
знаменитой падающей башни. После Пизы он переехал на 18 лет в Падую, где
занимался разнообразными вопросами технического характера, читал лекции в
Падуанском университете на кафедре математики и имел как лектор огромный
успех. В лекциях по астрономии он упоминал о наличии ученых, считающих
Землю подвижной, придерживался, однако, точки зрения Аристотеля и Птолемея о
неподвижности Земли. Но, судя по письму Галилея Кеплеру от 4 августа 1597 г.,
написанному в ответ на получение книги Кеплера «Мистериум космографикум»
(1596), он уже тогда был скрытым коперниканцем. Это письмо крайне интересно
(привожу по Идельсону, с. 53): «Твою книгу я прочту с тем большей охотой, что
на точку зрения Коперника я встал уже много лет тому назад, и мне удалось на
основе ее найти объяснение многим явлениям природы, которые, без сомнения, не
могут найти объяснения на основе общепринятых положений. Я записал много
доказательств и много опровержений рассуждений, основанных на
противоположной точке зрения; но выпустить все это в свет я не решался,
устрашенный судьбой Коперника, нашего учителя, который хотя и заслужил себе
бессмертную славу у немногих, но со стороны бессчетного числа людей (ибо так
велико число глупцов) подвергся лишь насмешке и освисту. Я решился бы,
действительно, продолжать мои рассуждения, если бы существовало много
людей, подобных тебе, Кеплер, но их нет, и я откажусь от этих занятий». Это
письмо интересно во многих отношениях. Во-первых, из него ясно, что Галилей
интересовался астрономией давно и был
убежденным коперниканцем, но всерьез этим не занимался, так как и без того дел
у него было по горло. Во-вторых, к тому времени он уже испытал резкую
оппозицию своим работам. Его «Трактат о движении» (Де Моту гравиум, 1590)
вызвал против молодого профессора бурю негодования со стороны большинства
профессоров, сторонников Аристотеля (История философии, т. II, 1941, с.
65), и эта буря была основана на чисто перипатетической косности, вне
всякой связи с религией. Галилей, может быть, переоценил значение числа
«глупцов». В-третьих, ясно, что Галилей вовсе не считал себя обязанным идти
напролом и считал возможным, будучи коперниканцем, излагать астрономию по
Птолемею; и, наконец, в-четвертых, из письма ясно, что он еще не успел
прочесть присланной ему книги, вряд ли он согласился бы с ее содержанием.

Книга эта считается ошибкой Кеплера, но сейчас, ввиду усиления интереса к
истории науки, ее содержание передается многими. В частности, в «Истории
астрономии» Паннекука, 1966, с. 241--242, изложено ее содержание, а на с. 243
(рис. 31) дана модель Вселенной по этой книге Кеплера. Кеплер дал
объяснение строения планетной системы на основе «Платоновых тел», т. е.
правильных многогранников. «Если на каждом из шести планетных кругов
построена сфера, мы можем между каждой парой последовательных сфер, считая их
точно концентрическими, построить одно из правильных геометрических тел
таким образом, чтобы вершины его были расположены на внешней сфере, а
плоскости были касательными к внутренней сфере». Для некоторых расстояний
получилось хорошее совпадение. С удивлением читаем дальше в книге марксиста с
предисловием нашего известного астронома Кукаркина такие слова на той же
странице: «Однако совпадение слишком велико, чтобы быть случайным. Итак,
Кеплер, движимый более астрологическими идеями, расположил 5 геометрических тел
в последовательности, идущей от центра наружу: 18-, 20-, 12-, 4-,
6-гранники, между шестью планетными сферами... Раскрыв этот секрет строения
мира, Кеплер поднял теорию Коперника на уровень, значительно превосходящий
спорное мнение, основанное на неопределенном эмпиризме, и сделал ее
фундаментальной философской истиной». Хорошо известно, что эту книгу
Кеплера считают неудачным началом великого пути и славу Кеплер приобретет не
ею, а своими тремя законами, изложенными в более поздних сочинениях. Ясно, что
такая книга не могла привлечь Галилея, которому остался чужд и дальнейший
путь Кеплера. Галилей вовсе не был далек от пифагореизма (С. И. Вавилов.
Галилей. БСЭ, 2-е изд., 1952, т. 10; с. 126): «Мир был для Галилея открытой
книгой, написанной на математическом языке в виде треугольников, кругов и
других геометрических фигур», но астрологии и представлениям о связях между
планетами он был совершенно чужд. Кеплер ответил Галилею на его письмо
(Паннекук, с. 242): «Решайтесь, выступим одновременно. Дружными усилиями мы
сдвинем этот экипаж. Своими доказательствами Вы поможете тем из наших
сторонников, которые теперь еще придерживаются неправильных суждений. Я
думаю, что очень немногие из знаменитых математиков Европы будут против
нас, так как могущество истины бесспорно». Галилей все же обратился к
астрономии, но пошел своим путем (у Кеплера
было плохое зрение, и он не был хорошим наблюдателем). Потратив много труда,
Галилей сконструировал телескоп и в ночь 7 января 1610 г. (когда ему было уже
45 лет) в Падуе оказался первым человеком, смотревшим на небо вооруженным
взглядом (Идельсон, 1964, с. 54). Перед ним открылся совершенно новый мир и при
первых же наблюдениях вблизи Юпитера обнаружились какие-то звездочки --- спутники
Юпитера. «Действительно, было от чего прийти в экстаз и восхищение; вдаль
уходили схемы Аристотеля и Птолемея; как дым, рассеивалась мистическая
надстройка над Коперником, которую предполагал Кеплер; новый мир, реальный и
величественный, открывался перед человеком; материя этого мира представлялась в
богатстве и разнообразии, которое надлежало теперь осознать».

Кеплер, как мы знаем, был плохим наблюдателем и пользовался преимущественно
наблюдениями Тихо де Браге: он наблюдал Вселенную умственными (как говорили в
старину, «умными») очами и, как известно, таким способом сделал огромный шаг
вперед. Галилей же продолжал наблюдения до тех пор, пока болезнь глаз
(катаракта) в 1638 г. (когда ему уже было 74) не заставила его прекратить это
занятие (Идельсон, с. 60).

Галилей же не пытался совершенствовать теорию орбит солнечной системы, хотя при
его жизни были открыты Кеплером три его великих закона (два первых опубликованы
в 1609 г. в сочинении (Астрономия нова) «О движении Марса» и третий в 1619 г. в
«Мировой гармонии»). Галилей же в знаменитом «Диалоге», опубликованном в 1632
г., решительно отвергает и даже высмеивает Кеплера по поводу его мнения о
влиянии Луны и полностью игнорирует его великие законы. «И среди великих людей,
рассуждающих об этом удивительном явлении приливов, более всех других удивляет
меня Кеплер, который, будучи наделен умом свободным и острым и хорошо знакомым
с движениями, приписываемыми Земле, допускал особую власть Луны над водой,
сокровенные свойства и тому подобные ребячества» (Галилей. Диалоги о двух
главнейших системах мира --- птолемеевой и коперниковой. Перевод и примечания А.
И. Долгова, 1948, с. 326). В примечаниях к книге А. И. Долгов указывает, что
уже в эпоху Галилея указывалось вполне основательно, что приливы и отливы
достигают максимальной величины в эпоху равноденствий, а не солнцестояний (в
частности Фр. Бэконом), как следовало по теории Галилея. Свои теоретические
выводы Галилей не сравнивал с практическими данными, т. е. в данном случае
отступал от тех принципов экспериментального метода, которые он так энергично и
эффективно проводил в других своих работах. А. И. Долгов на той же странице
одобряет Галилея за его отрицание скрытых свойств, но вместе с тем указывает,
что в отношении Кеплера он был не прав. Это хорошо отмечено и А. Эйнштейном в
статье «О Галилее и его "Диалоге"» (Вопросы истории естествознания и техники,
вып. 16, 1964, с. 32). Эйнштейн не скрывает своего восхищения перед Галилеем и
вместе с тем пишет: «Подкреплением системы Коперника, сверх качественных
доводов, являлось бы только определение "истинных орбит" планет, а эта
проблема, казалось, почти непреодолимой трудности, была, однако, разрешена
Кеплером (при жизни Галилея) истинно гениальным способом. То, что это решающее
достижение
не оставило следа в работах, которым Галилей посвятил свою жизнь, иллюстрация
того факта, что творческие личности часто бывают невосприимчивы... Стремление
Галилея доказать механическое движение Земли ввело его в заблуждение при
создании им его ошибочной теории приливов и отливов. Блестящие рассуждения,
изложенные в последней беседе, сам Галилей признал бы бездоказательными, если
бы не его темперамент». В чем тут дело? Каким образом гениальный основоположник
механики мог сделать такие ошибки? Тут именно дело в «темпераменте», на что
указывает Эйнштейн, на его «убеждении чувства», а не разума. Механика была
лейтмотивом всей его жизни, и, экстраполируя (под влиянием уже убеждения
чувства, а не разума) применимость механики за пределы ее действительной
применимости, он был убежден, что его механическая, ошибочная теория приливов и
отливов (связывающая приливы и отливы с вращением Земли) поможет обосновать и
систему Коперника. Поэтому первоначальное название «Диалогов» было «О приливах
и отливах» и последняя, четвертая беседа, где теория приливов обосновывалась,
была завершающей, едва ли не самой важной с точки зрения Галилея. Название было
изменено по совету лиц из папской курии, продвигавших эту работу в печать.
Отсюда понятно, почему Галилей чрезвычайно упростил систему Коперника и
Птолемея, совершенно отказавшись от эпициклов, так как для его механической
теории приливов и отливов такая сложность системы была помехой, понятно также,
что он не делает отличия между системами Аристотеля и Птолемея, хотя, само
собой разумеется, не мог не знать их различий. Ведь у Аристотеля --- система
концентрических сфер, движение по правильным кругам, а у Птолемея --- большое
число эпициклов и Солнце движется не точно вокруг Земли, а вокруг эксцентра. Но
система Аристотеля удовлетворяла строгим философским требованиям --- совершенство
--- и не годилась для прогноза, система же Птолемея вызывала резкие возражения
перипатетиков, например знаменитого Ибн-Рушда (Аверроэса), но по ней можно было
вычислять с удовлетворительной точностью движения планет, затмения и проч. Для
Галилея важно было другое --- применить новую механику для механического
доказательства движения Земли вокруг Солнца при помощи приливов и отливов. Эта
сокровенная мысль изложена им в письме к кардиналу Орсини от 1616 г. (Идельсон.
Вопросы ист. естест. и тех., вып. 16, 1964, с. 64), и через 15 лет это письмо
составит самый нерв его «Диалога». Теория Галилея исключает действие какой бы
то ни было приливообразующей силы, и возможность таких сил Галилей резко
отрицает. «Мой рассудок не может приспособиться к тому, чтобы подписаться под
действием света, темперированного тепла или возбуждения явлений через скрытые
качества и тому подобными бреднями; все это не только не является, но и не
может явиться причиной прилива; скорее уж обратно, прилив в мозгах ведет здесь
к этой болтовне и к крикливым суждениям, а не к размышлениям над более
глубокими явлениями природы и к их исследованиям» (там же, с. 79). Как видим,
здесь Галилей покидает даже обычный для него спокойный тон рассуждения. В своих
выражениях Галилей объявил войну той доктрине Средневековья, которая
приписывала приливы таинственному влиянию Луны, доктрине, за которой
шел и Кеплер. Для Галилея мир «осязаем» и нет скрытых явлений. Эскиз
космической механики Галилея не нашел подтверждения в наблюдении и опыте. Там
же Идельсон пишет: «Ньютон в единой формуле дальнодействия, над раскрытием
сокровенного смысла которой человеческая мысль работает и по настоящее время,
объединил законы движения планет, их спутников и комет, приливы вод океана.
Явление приливов оказалось обусловленным именно притяжением Луны; сила
притяжения теперь снова появилась как орудие познания природы, будучи очищена
от той таинственной окраски, которую наложило на нее мистическое мышление
средних веков и которая так отталкивала Галилея. С этого момента стало
очевидным, что галилеево одновременное доказательство двойного движения Земли
ничтожно и что его теория приливов может в лучшем случае служить для пояснения
некоторых частностей явления, каким оно наблюдается на Земле. Небесная механика
Галилея вовсе не этап развития, а тупик, и небесная механика Ньютона ей
диаметрально противоположна». Но как и указывает Идельсон в примечании к той же
с. 79, и теория приливов Ньютона (статическая по своему характеру) не учитывает
инерции вод океана (начало динамической теории приливов положено Лапласом) и
«до сих пор неизвестно механическое явление, которым можно было бы доказать
одновременно оба движения Земли --- суточное и годичное. Для механического
доказательства первого из них служит маятник Фуко, второе обнаруживается только
в аберрации неподвижных звезд и в их годичном параллаксе. То, что не удалось
Галилею, не удалось и в последующие три столетия».

Чем же вызван такой бурный протест Галилея против Кеплера и других
предшественников Ньютона? Конечно, это результат того ожесточенного
сопротивления, которое он с молодых лет встречал у перипатетиков, сторонников
Аристотеля. Метод перипатетиков того времени, в значительной мере искаженный
толкователями Аристотеля, заключается в следующем (статья Д. Бобылева в
«Энциклопедическом словаре» Брокгауза и Эфрона, т. УПа (14), 1892, с. 896--898):
«Прежде всего, исходили из гипотез или положений, прямо почерпнутых из
сочинений Аристотеля и из них, путем силлогизмов, выводили заключение
относительно того, как должны происходить те или иные явления природы; к
проверке же этих заключений путем опыта не прибегали вовсе. Следуя такому пути,
перипатетики были, например, убеждены и учили других, что тело, весящее в
десять раз больше другого тела, падает в десять раз быстрее». В 1589 г. (25
лет) Галилей занял кафедру математики Пизанского университета и здесь открыл,
что скорость падения тела возрастает со временем и независима от веса тела. Это
вызвало неудовольствие перипатетиков, и те нашли повод к его удалению с кафедры
за неодобрительный отзыв, данный им относительно нелепого проекта какой-то
машины, поданного одним из побочных сыновей Козимо Медичи. Но по ходатайству
маркиза дель Монти он перешел на кафедру математики в Падую, где он пробыл 18
лет, с 1592 по 1610 гг. Так как перипатетики были его главными противниками
(они и выведены в «Диалогах» и «Беседах» под именем Симпличио), то такая
непрерывная борьба и окрасила эмоциональным элементом его рассуждения и привела
к возникновению ошибочных «убеждений
чувства». Но, борясь с перипатетиками, он отнюдь не отрицал всего Аристотеля.
По всей своей философской физиономии он был ярким выразителем того платонизма и
пифагореизма, которым характеризуется Возрождение. Будучи чрезвычайно
разносторонним и сделав много для усовершенствования стиля итальянской речи, он
читал, например, лекции и о Данте, и там, чтобы определить форму адской
воронки, он пользовался теоремами Архимеда и Альбрехта Дюрера, геометрическими
и архитектурно-механическими соображениями (Цейтлин. Галилей. 1935, с. 20). Он
был ревнителем механического понимания, но весьма далеким от материализма,
будучи почитателем Платона.

Галилей в «Диалоге», игнорируя эллипсы Кеплера, продолжает считать естественным
совершенное движение по кругу (Аристотель, сохранивший здесь платоновское
положение) и считает траекторией свободно падающего тела окружность, хотя им же
дано доказательство, что это парабола. В ранний период Галилеем была написана
работа «об ускоренном движении тела», вошедшая впоследствии в «Беседы о двух
новых науках», но в этих последних в гораздо более совершенной форме
доказывается о движении падающих тел по параболам (Галилей. Диалоги. Прим. А.
И. Долгова, с. 363).

То обстоятельство, что Галилей в своих последних «Беседах» (1638) значительно
усовершенствовал свои механические представления по сравнению с «Диалогами»
(1632), свидетельствует, что на старости лет, почти накануне смерти, он не
только изложил последовательно работу всей своей жизни (начиная примерно с
19-летнего возраста), но и продолжал творчески работать над своей теорией, уже
будучи 74-летним ослепшим стариком: редкий пример мыслителя, у которого вершина
творчества (акме) падает на последние годы его жизни. Успех «Бесед» заключается
и в том, что в них, оставив космические проблемы (которые еще не созрели для
механического истолкования), он обратился к чисто земным явлениям. «Все, что он
здесь говорит, безошибочно (в отношении механики. --- \emph{А.Л.}) и классично и по
праву вызвало ту восхищенную оценку, которую Лагранж дал этим открытиям
Галилея» (Идельсон, там же, с. 75). В «Диалогах» же Галилей, следуя своему
призванию механика и инженера, стал немедленно применять найденные им законы
движения твердых тел к мировой материи в целом.

Ошибочно судил Галилей и о кометах. Еще в своей блестящей работе «Пробирщик...»
(Салжиторе, 1623) он защищал неправильное положение, относя кометы к «подлунной
сфере», в противоположность Тихо де Браге, который, основываясь на исчезающе
малом параллаксе кометы 1572 г., пришел к совершенно правильному выводу о ее
большом расстоянии от Земли. (Галилей. Прим. А. И. Долгова, с. 360). Галилей
же, в данном случае следуя Аристотелю (который полагал, что только земному
свойственно возникновение и уничтожение и потому кометы не могут быть небесными
явлениями, а относятся к сфере Земли), считал, что кометы основаны на
испарениях от Земли вопреки фактам. И здесь экстраполяция твердых убеждений,
справедливых в определенной области, привела к игнорированию того самого
экспериментального метода, одним из блестящих основоположников которого он
являлся.

Мы видим, таким образом, что, по согласному мнению современных выдающихся
ученых, главная заслуга Галилея --- не его защита гелиоцентрической системы, а
создание основ механики. В знаменитом «Диалоге» много упрощений и ошибочных
мнений, в нем нет ни одной формулы, численные данные и результаты вычислений
даны редко и скупо. Все это, а также то, что она написана на итальянском, а не
на принятом в то время для ученых латинском языке создало широко
распространенное мнение, что это --- научно-популярная книга, написанная с целью
распространения коперникова учения. Галилей, конечно, стремился к широкому
распространению своих взглядов, будучи блестящим лектором и стилистом, но столь
же блестяще и тоже на итальянском языке написана его последняя книга «Беседы о
двух новых науках», посвященная целиком механике и сопротивлению материалов.
Его книги написаны, если так можно выразиться, «максимально понятно», принимая
во внимание трудность предмета, и для специалистов по астрономии и механике они
могут показаться популярными. Я лично читал их с удовольствием в хорошем
русском переводе, но никак не могу согласиться с мнением, что это --- популярные
книги. В «Диалоге» нет формул не потому, что они удалены (как это делается в
популярных книгах) для облегчения понимания, а потому что их вообще не
существует. Упростил Галилей теорию Птолемея и Коперника не для целей
популяризации, а для целей продвижения своих механических идей, но так как, как
указано выше, его механическая теория приливов была неверна, то никакой гений
не мог бы дать математическую теорию или (если бы он ввел какие-либо
дополнительные предположения) она оказалась бы совершенно невероятной
сложности. Упрощая структуру Солнечной системы, Галилей не погрешил против
научного метода, потому что всякий физический закон есть относительный закон.
Как пишет Дюгем (Физическая теория, ее цель и строение, 1910, глава V, § 3):
«Относительность закона не в том, что он верен для одного физика и не верен для
другого, а в том, что степень его приближения может быть достаточна для одного
применения и недостаточна для другого. Один и тот же физик один и тот же закон
то принимает, то отвергает в одной и той же работе: Реньо, например, при
исследовании сжимаемости газов старается заменить закон Мариотта формулой более
точной, но давление на высоте свободной поверхности в его манометре он
определяет по формуле Лапласа, в основе которой лежит закон Мариотта; здесь нет
противоречия, так как ошибка, внесенная в вычисления этим специальным
применением закона Мариотта, гораздо меньше, чем степень надежности
экспериментальных методов, которыми он пользуется». Погрешил Галилей как ученый
в том, что он пытался выдать за научный принцип чисто онтологический постулат
об универсальности механического истолкования явлений. Он стремился изгнать
всякие скрытые таинственные силы, которые принимал и Кеплер (Уайт, с. 152:
Кеплер принимал участие ангелов в планетных движениях), и из-за борьбы с ними
не принял участия в дальнейшем развитии гелиоцентрической теории. Ньютон
восстановил таинственные силы и сделал мощный шаг вперед, а многие
последователи Ньютона, подобно Лапласу, следуя политике страуса, сделали вид,
что таинственных сил они вообще не используют.

Этот краткий очерк показывает, что механика была лейтмотивом всей жизни Галилея
с 19-летнего возраста вплоть до его последних лет, когда он, уже ослепший,
диктовал страницы своего бессмертного последнего труда, и именно в последние
годы он довел до наилучшей проработанности основные свои идеи. Отклонения в
область астрономии хотя и сопровождались рядом блестящих успехов, но привели и
к ряду ошибок, связанных с неумеренной экстраполяцией. История Галилея поэтому
поучительна для всех наук в ряде отношений.

1) Недопустима экстраполяция положений и постулатов, оправданных в определенной
ограниченной области, на те области, где они недоступны опытной проверке. Этой
ошибкой грешат сейчас особенно в биологии, где ряд положений, оправданных в
области молекулярной биологии, генетики и проч., распространяется на область
макроэволюции при полном игнорировании вопиющих противоречий.

2) Развитие наук идет не путем накопления окончательно установленных истин, а
путем последовательных синтезов, причем при новом синтезе нередко возвращаются
к казалось бы окончательно отвергнутым положениям. Такое полное непонимание
духа истории науки показывают многие выдающиеся ученые, например Холден, не
понимающий антагонистичности взглядов Галилея и Кеплера и исключительности
значения Ньютона, объединившего в своем синтезе двух, казалось бы, непримиримых
ученых.

3) И из этого краткого очерка ясно, какие трудности представляли даже для
величайших умов крупные шаги в деле точной науки и как примитивен
вульгаризаторский подход писателей, именующих себя «прогрессивными», например
Бертольда Брехта, для которого все дело чрезвычайно просто: с одной стороны,
прогрессисты, с другой, мракобесы, не понимающие таких простых доводов, которые
способен понять самый простой, необразованный человек. Борьба геоцентрической и
гелиоцентрической систем вовсе не соответствовала разделу двух лагерей: 1)
прогрессивного, куда: материалисты, атеисты, новые религиозные движения,
выдвинутые Реформацией, и проч. 2) реакционного, куда: католическая церковь,
идеалисты, защитники феодализма и проч. Это всего лучше можно показать на ряде
сопоставлений, причем очень часто критическим пунктом является разногласие в
учении о приливах.

Френсис Бэкон считается основоположником материализма и в значительной степени
(тут сильно переоценен) индуктивного метода. Но он был противником теории
Коперника в некоторой степени потому, что он справедливо возражал против
ошибочной теории приливов Галилея. Противники Коперника охотно держались
старого астрологического толкования приливов и отливов действием Луны, так как
эта теория не предполагает вращения Земли (Дюгем, глава VII, § 2).

Гораздо позже, уже когда Ньютон реставрировал астрологический принцип
всемирного тяготения, выдающийся ученый Гюйгенс считал этот принцип абсурдным,
то же бы сказал и Декарт, а за ним и его последователи, картезианцы (Дюгем,
глава I, § IV). Примерно то же говорил и Лейбниц, и, как было показано выше,
сам Ньютон не отрицал абсурдности действия на расстоянии, но решительно
перешагнул через это, казалось бы, непреодолимое препятствие. Таким образом,
сопротивление Копернику, Галилею и другим строителям новой системы в
значительной (у многих ученых даже и решающей) степени было связано со вполне
рациональными доводами. Система Декарта связана с учением о вихрях, но законы
Кеплера нельзя было объяснить вихревым движением мировой материи (Паннекук, с.
268). Тот же Паннекук (там же) указывает, что восторженным почитателем Кеплера
(применившим его законы к объяснениям движения Луны) был юный английский
священник Иеремия Горроксю, рано умерший.

Протестантские церкви, подвергшие критике многие положения католичества, были
не менее решительны и в осуждении учения Коперника и продолжали в ряде случаев
это осуждение вплоть до второй половины XIX столетия (Уайт, с. 168).

Какое же отношение встретил Галилей среди своих соотечественников? Уже было
указано, что главными противниками его были догматические последователи
Аристотеля, перипатетики. Среди его сторонников сразу оказывается значительное
число представителей церкви: пизанский монах --- математик Кастелли,
неаполитанский богослов Фоскарини, знаменитый доминиканец Кампанелла, даже
генеральный проповедник Мараффи, Дини --- все они «игнорируя вопросы церковной
политики сегодняшнего дня, разрешают вопрос с точки зрения вечности» (М. Я.
Выгодский. Галилей и инквизиция. 1934, с. 127).

Астрономические открытия Галилея многими были встречены восторженно. Несмотря
на то что среди перипатетиков у него было много ожесточенных врагов и что
учение Аристотеля поддерживалось как неопровержимая истина католической
церковью, Галилей нашел себе сторонников в Риме среди высших лиц курии: таковы,
например, знаменитый богослов кардинал Беллармин и кардинал Маттео Барберини,
впоследствии папа Урбан VIII. В 1611 г. Галилея с триумфом встречают в Риме
ученые иезуиты из Коллегиум Романум; он находит восторженный прием при папском
дворе, становится другом князя Чези, основателя Академии ден Линчей
(Рысьеглазых) и сам делается членом этой Академии (С. И. Вавилов, статья в БСЭ,
2 изд., т. 10, 1952, с. 128). На запрос Беллармина, правильны ли наблюдения
Галилея, касающиеся скоплений неподвижных звезд, Сатурна, Венеры, Луны и
Юпитера, Римская коллегия за подписью четырех авторитетных членов Клавио,
Гринбергера, Малькотио и Лембо ответила утвердительно, хотя и с некоторыми
несущественными оговорками (предисловие А. И. Долгова к «Диалогам» Галилея, с.
9).

Галилей был в переписке с Маттео Барберини. Во время диспута о плавающих телах
Маттео Барберини принял сторону Галилея против перипатетиков, после сообщения о
солнечных пятнах принял его сторону против перипатетиков и иезуита Шейнера. В
1624 г., уже сделавшись папой Урбаном VIII, он подарил Галилею портрет, медали
и обещал стипендию сыну Винченцо. По некоторым данным, Барберини деятельно
защищал Галилея в 1616 г. и не вполне одобрял декрет от 5 марта, но, несмотря
на шесть бесед с Галилеем, не согласился отменить его в 1624 г.

Но в 1616 г. коперниково учение было официально осуждено, а к этому времени
Галилей уже опубликовал «Звездный Вестник» («Посланец от звездного мира»), где
содержалось изложение его астрономических наблюдений с неприкрытым одобрением
теории Коперника.
По приказанию папы Павла V Галилей был вызван во дворец кардинала Беллармина, и
здесь этот кардинал увещевал Галилея об ошибочности этого учения, а в протоколе
пленарного заседания конгрегации инквизиции в присутствии папы кардинал
Беллармин сообщил, что «математик Галилей, будучи предупрежден о приказании
инквизиции отойти от учения, которого он до сих пор придерживался, именно, что
Солнце есть центр сфер и неподвижно, а Земля движется, с этим согласился». На
том же заседании папа Павел V утвердил текст декрета, изданного 5 марта 1616
г., по которому книга Коперника была «задержана впредь до исправления», в то
время как написанное в «примиренческом» духе письмо Фоскарини было совершенно
«запрещено и осуждено». В перечне задержанных и осужденных книг ни «Посланец от
звездного мира», ни «Письма о солнечных пятнах» Галилея не значатся. Сам
Галилей сразу после опубликования декрета пишет, что вопрос идет только о
незначительных исправлениях книги Коперника, но что его, Галилея, враги
совершенно посрамлены. В письме от 12 марта 1616 г. Галилей описывает
милостивую аудиенцию, данную ему накануне папой Павлом V, и пишет: «Когда в
заключение я указал, что остаюсь в некотором беспокойстве, опасаясь возможности
постоянных преследований со стороны неумолимого коварства людей, папа утешил
меня словами, что я могу жить в спокойном настроении, так как обо мне у его
святейшества и у всей конгрегации остается такое мнение, что нелегко будет
прислушиваться к словам клеветников; так что пока он жив, я могу чувствовать
себя в безопасности» (Идельсон, 1964, с. 65).

Конечно, Галилей был недоволен, что учение Коперника все же было осуждено, но
это осуждение никогда не было безусловным. Требовалось лишь, чтобы это учение
выдавалось не за абсолютную истину, а за удобную математическую гипотезу. В
письме 1615 г. кардинал Беллармин пишет: «В действительности очень хорошо
поступает тот, кто говорит, что, предполагая Землю подвижной, а Солнце
неподвижным, мы гораздо лучше отдаем себе отчет во всех явлениях, чем это можно
было бы сделать при помощи эксцентрических кругов и эпициклов. Это не
представляет ни малейшей опасности и вполне достаточно для математики» (Дюгем,
глава III, § 2. См. также Идельсон, 1964, с. 62). И это подтверждается теми
несущественными изменениями, которые были внесены в сочинения Коперника,
переизданные в 1630 г., т. е. через четыре года после издания декрета.
Безусловное же осуждение сочинения Фоскарини было сделано потому, что там автор
стремился примирить учение Коперника и Священное Писание, а тем самым учение
Коперника возводилось в абсолютную истину (А. И. Долгов. Предисловие к
«Диалогам» Галилея. 1948, с. 11).

Естественно, что Галилей воспрянул духом, когда в 1623 г. кардинал Барберини
сделался папой Урбаном VIII. Полемическое сочинение Галилея «Пробирщик золота»
было выпущено в 1623 г. с посвящением Урбану VIII от имени Академии. Урбан VIII
даже писал стихи в честь Галилея. Галилей предпринял поездку в Рим в 1624 г. с
целью добиться смягчения декрета 1616 г., но, несмотря на ряд бесед с папой,
ему это не удалось. Галилей сообщал князю Чези: «Кардинал Целлер (Гогенцоллерн)
передал мне, что он имел беседу с его святейшеством относительно Коперника и
сказал папе, что, поскольку все еретики
(протестанты) придерживаются коперниканского учения, которое они считают
достовернейшим, то, принимая то или иное решение, здесь нужно действовать
весьма осмотрительно. На это папа ответил: "Св. церковь не осуждала и не
предполагает осуждать это учение как еретическое, но только как необоснованно
смелое; однако (добавил папа) не следует опасаться, что кому-нибудь удастся
доказать, что это учение есть необходимо истинное"». При отъезде Галилея из
Рима Урбан VIII снабдил Галилея так называемым бревэ на имя герцога
Тосканского, где Галилею расточались высокие похвалы: «Уже давно взираем мы с
отеческим благоволением на мужа, слава которого сияет в небесах и
распространяется по всей земле». Все это не помешало Урбану VIII через восемь
лет оказаться главным инициатором позорного процесса против Галилея, а после
осуждения его --- проявить себя мучителем великого старца до самой его смерти
(Идельсон, 1964, с. 67, примеч). Тот же Идельсон на с. 171 указывает, что
Галилей «слишком доверчиво отнесся к сообщению его любимого ученика Кастелли в
письме от 16 марта 1630 г. о том, что папа Урбан VIII сказал: "Это (запрещение
Коперника) никогда не было нашим намерением, и если бы зависело от нас, то
декрет не был бы издан"».

В изложении современных историков науки, компетентных в астрономии, мы видим, с
одной стороны, что никаких признаков неблаговоления со стороны пап (Павла V и
Урбана VIII) по отношению к Галилею не было вплоть до знаменитого процесса,
мало того, Галилей имел все основания считать Урбана VIII своим ценителем и
покровителем --- и вдруг такая перемена. Урбан превращается внезапно в гонителя и
мучителя, и это все преподается без малейшей попытки объяснения такой
удивительной перемены. Астроном изложил с точки зрения астрономии, а от
суждения вне пределов своей специальности воздержался. Я разберу этот вопрос
позже в связи с процессом Галилея, а пока остановлюсь еще на той категории
противников Галилея, каковым приписываются интриги против великого ученого,
именно на иезуитах. Этот вопрос хорошо разобран между прочим в книге Цейтлина,
который пишет (с. 89): «Здесь же заметим, что традиционное изображение
иезуитского и католического отношения к науке как всеобщего, голого и грубого
обскурантизма является в корне ложным». Легенда о том, что главными
преследователями Галилея были иезуиты, не является необоснованной. Сам Галилей
в письме к Дислати от 25 июня 1634 г. выражается следующим образом: «Вы видите,
значит, что я пострадал не за защиту того или иного взгляда, а потому что впал
в немилость у иезуитов» (Цейтлин, с. 131). Это же подтверждает и математик
коллегии иезуитов Гринбергер, который в беседе с одним из друзей Галилея в 1634
г. (т. е. уже после процесса Галилея) сказал: «Если бы Галилей сумел сохранить
расположение к себе со стороны этой коллегии, то жил бы он в славе на свете и
не случилось бы с ним ни одного из его несчастий, и он мог бы писать по желанию
о любом предмете, в том числе и о движении Земли, и т. д.» (Идельсон, 1964, с.
60). Понятно, что и знаменитейший и яростнейший противник иезуитов, великий
Паскаль имел право указывать на иезуитов как на главных виновников осуждения
Галилея. Но из вышеприведенных слов ясно, что Галилей «впал в немилость», «не
сумел сохранить расположения», так как раньше, это ясно и из первого приема
коллегией иезуитов в Риме, он пользовался милостью иезуитов. И верно, что он
нажил несколько крупных врагов среди иезуитов.

Такими были Грасси и Шейнер. Но эти противники держались очень осторожно, пока
генералом ордена иезуитов был Клавдий Аквавива, последний умер в 1616 г., и
тогда выступления противников Галилея стали смелее, так как на место Аквавивы
стал слабохарактерный «ангел мира» Муццион Виттелески (Цейтлин, с. 129). Ясно,
что Аквавива защищал Галилея от нападок, а Виттелески лишь не мешал нападать.
Но «Пробирщик» (Салжиторе) Галилея направлен в значительной степени против того
взгляда Грасси, что кометы принадлежат к сфере неба (что, в частности,
противоречило мнению Аристотеля). Здесь Галилей был не прав, а по свидетельству
Идельсона (с. 55 и 67), растянутая, почти непрерывная полемика действует на
читателя несколько утомительно. Значит, как это ни странно, в данном случае
иезуит защищал антиперипатетические взгляды, несмотря на поддержку Аристотеля
церковью.

Другого врага среди иезуитов, Шейнера, Галилей тоже нажил по поводу приоритета
открытия солнечных пятен, что вызвало ожесточенные споры. И здесь Галилей был
бесспорно прав в отношении объяснения солнечных пятен, но Шейнер несколько
опередил его в деле наблюдения, хотя обоих их опередил астроном Иоганн
Фабрициус, о котором не знал ни Галилей, ни Шейнер, но о котором знал Кеплер
(Идельсон, 1964, с. 60). После периода корректной переписки между Галилеем и
Шейнером возникла полемика. Галилей заострил ее в памфлете (Салжиторе), Шейнер
резко обрушился на Галилея. Галилей писал о Шейнере в 1636 г. в письме: «Этот
поросенок и лукавый осленок составляет теперь перечень того, что я не знал в
свое время...» (там же).

Но и по поводу объяснения солнечных пятен Галилей не избежал ошибок,
что хорошо показано в той же статье Идельсона (с. 53--59). В первых
статьях о солнечных пятнах он дал правильное объяснение своеобразному
и изменчивому для земного наблюдателя виду траекторий пятен по диску
Солнца, выяснив, что это есть следствие наклона плоскости солнечного
экватора к плоскости эклиптики под небольшим углом. Все это относится
к самым мастерским и блестящим моментам творчества Галилея. Но
позже, в эпоху создания «Диалога» (1630), Галилей считал возможным
извлечь из своего открытия нечто большее и здесь его рассуждение (на
этот раз ошибочное) приводит к тому, что Солнцу пришлось бы дать
«третье» движение. Как пишет Идельсон, эта ошибка Галилея тем более
удивительна, что он сам доказал ненужность «третьего» движения Земли в
системе Коперника. Тот же Идельсон замечает: от этого спора падает
тяжелая тень, омрачающая последние годы жизни Галилея. Мы видим, таким
образом, что в споре с Шейнером Галилей не был прав на сто процентов.
Но в этом споре примешивалась и политическая сторона. Шейнер был одним
из самых влиятельных и деятельных агентов испано-немецкой партии, как
и лидер испанской оппозиции Урбану VIII, кардинал Гаспар Борджиа, о
чем еще будет речь (Цейтлин, с. 131). Само собой разумеется, что
резкость полемики Галилея и ошибки, им сделанные, и в которых он
не хотел признаваться, привлекли на сторону Грасси и Шейнера и других членов
общества Иисуса.

Был и второстепенный объект спора: язык. Галилей писал свои работы на
итальянском языке. Это новшество вызвало возражения с разных сторон.

Как указано в книге Цейтлина (с. 37), историк Ольшки делает мелкое замечание,
что можно было искать у духовенства покровительства для самых смелых мыслей,
если они были выражены на латинском языке. С другой стороны, Кеплер жаловался
на это оскорбление человечества (и тут два гения разошлись во мнениях), а
столетие спустя не менее гениальный Лейбниц выступил в защиту латыни, так как
путаница, вызванная употреблением народных языков в науке, заставила его
почувствовать необходимость восстановления мирового языка науки или замены его
искусственным. Как своевременны эти мысли Лейбница, в особенности в наше время!

О восторженном приеме астрономических открытий Галилея иезуитами Римской
коллегии я уже писал, но, может быть, после процесса иезуиты, как и другие
ревностные католики, препятствовали развитию идей Галилея. Вредное влияние
католицизма на развитие науки утверждает, например, Идельсон, в той же статье,
с. 66: «...декрет 5 марта 1616 г. --- это удар отнюдь не по одному Галилею (на
предыдущей странице тот же Идельсон пишет, что в перечне осужденных книг
сочинения Галилея не числятся и сам Галилей был принят в милостивой аудиенции
папой Павлом V); это суровое испытание для науки и культуры в странах
католицизма, где развитие новой астрономии приостанавливается приблизительно на
200 лет; где оно искусственно и умышленно задерживается, чтобы дать некоторое
время безраздельно господствовать над умами представителям отживающих
мировоззрений». В примечании Идельсон указывает, что декрет 5/III 1616 г. был
впервые «опущен» в 1757 г. при составлении нового кодекса запрещенных книг при
папе Бенедикте XIV (при котором, см. Уайт, с. 155, негласно уже допускалось
учение Коперника. --- \emph{А.Л.}), и только в индексе издания 1835 г. не встречаются
имена Коперника, Дилана Астуника, Фоскарини, Галилея и Кеплера. Несколько
раньше, в 1822 г., декретом папы Пия VII публикация работ с изложением
гелиоцентрической теории была разрешена в Риме (Уайт, с. 156).

Слова Идельсона об искусственной приостановке католицизмом развития новой
астрономии на 200 лет совершенно не соответствуют истине. Совершенно иначе
пишет известный современный марксистский историк науки, крупный физик Дж.
Бернал в своей книге («Наука и история общества», 1956, с. 228): «Даже движение
контрреформации, которому удалось пресечь и повернуть вспять развитие
протестантизма в Европе, не оказало подобного влияния на науку. Руководившие
этим движением иезуиты (а им принадлежит главная роль в том, что католичество
оказалось полностью восстановленным в Польше, Венгрии, Австрии, Бельгии,
Баварии и Рейнской области, где протестантизм, казалось бы, одержал полную
победу --- см. Британск. энциклопедия, 1957, т. 13, с. 9--15) были достаточно
умны, чтобы понимать, что им легче будет покорить души, поощряя науку, а не
слепо противодействуя ей. В соответствии с этим они полностью включились в
научное движение, в частности в новую астрономию, и даже содействовали ее
распространению и созданию обсерваторий в Индии, Китае и Японии. В то же время
иезуиты действовали как сторожевые псы внутри науки, призванные ограждать
истинную религию от всевозможного вредного влияния со стороны этого движения и
тем самым они, сами того не желая, поставили деятелей науки в протестантских
странах, находившихся вне сферы их контроля, в более выгодное положение». Вряд
ли и это утверждение соответствует действительности. Уайт в неоднократно
цитированной книге пишет (с. 155): «В Германии, в особенности в протестантской
ее части, война (с коперниковым учением. --- \emph{А.Л.}) была еще более ожесточенной
(чем в католических странах) и длилась в течение первой половины XVIII
столетия. Выдающиеся лютеранские доктора богословия наводнили страну
трактатами, доказывающими, что доктрина Коперника не может быть согласована с
Писанием. В богословских семинариях и во многих университетах, где было сильно
клерикальное влияние, богословы, казалось бы, все сметали на пути; и однако в
середине столетия некоторые из наиболее ясномыслящих поняли, что их дело
проиграно». Таким ясномыслящим Уайт считает папу Бенедикта XIV, про которого
уже говорилось. Книга Уайта вышла первым изданием в 1896 г. и в предисловии,
написанном им в С.-Петербурге в 1895 г., Уайт жалуется на то засилье
духовенства, которое имело место в США и Англии еще во второй половине XIX в.,
и то влияние, которое играла принадлежность к тому или иному религиозному
направлению в деле назначения профессоров и которое стало исчезать именно в это
время. Вспомним, что и знаменитый «обезьяний процесс» в наше время имел место в
протестантской Америке, а не в католической стране.

И вот мы видим, что отношение католичества вообще, а иезуитов в частности к
теории Коперника строго соответствовало первоначальной позиции кардинала
иезуита Беллармина: развивать ее как полезную для описания явлений
математическую гипотезу, но не придавать ей абсолютного значения.

Так выразился и знаменитый иезуит Боскович, автор первой математически развитой
атомной теории (пифагорейского, а не демокритовского характера): «Что касается
до меня, то полностью уважая Священное Писание и декрет Священной Инквизиции, я
считаю Землю неподвижной; однако, для простоты объяснения, я рассуждаю, как
будто бы Земля была подвижной, так как доказано, что из двух гипотез последняя
была обоснована явлениями» (Уайт, с. 155).

В 1672 г. иезуит Риччиоли разобрал все доводы за и против коперниковой теории и
указал, что имеется 49 доводов в пользу Коперника и 77 против. Большинство
доводов, таким образом, было против Коперника (Уайт, с. 154), но мы знаем, что
в науке большинством голосов вопросы не решаются, и всякий внимательный
читатель мог убедиться в сравнительной силе доводов.

Во Франции иезуиты Лессер и Жакье издали в 1739--1742 гг. французский перевод
«Начал» Ньютона, т. е. популяризировали гелиоцентрическую систему в том
совершенном виде, который ей придал Ньютон, но в предисловии делают требуемую
католической церковью оговорку (Цейтлин, с. 261--262). Эти же два иезуита
выпустили трехтомное латинское издание «Начал» в 1760 г., снабдив его
подробными примечаниями (см. А. Крылов, предисловие к переводу книги: И.
Ньютон. Математические начала натуральной философии, 1916, с. VIII). Иезуиты
потратили много труда, чтобы способствовать распространению гелиоцентрической
теории в то время, когда книги Коперника и Галилея числились в списке
запрещенных книг, а автор «Начал» Ньютон был ожесточенным антипапистом и в
своей книге «Толкований пророка Даниила и Апокалипсиса» рассматривал папу как
апокалипсического зверя.

Иезуиты дали много выдающихся ученых и как раз в особенности в астрономии
(Лалинд насчитывает 42) и геофизике. Особенно крупным был (уже в XIX в.) Секки
(1818--1878), один из основоположников спектрального анализа и давший
классификацию звезд. Энгельс неоднократно цитирует Секки и так его
характеризует (Диалектика природы, 1949, с. 158): «Патер Секки хотя и воздает
ему (Богу) всякие канонические повести, тем не менее весьма категорически
выпроваживает его из Солнечной системы, разрешая ему творческий акт только в
отношении первоначальной туманности». И Секки не только работал успешно в
области астрономии, но и популяризовал гелиоцентрическую систему, поставив опыт
Фуко в Риме в 1852 г. в одной из церквей (Уайт, с. 157). Наши воинствующие
безбожники, считая себя страшно революционными, полагают, что, поставив опыт
Фуко в одном из соборов С.-Петербурга, они проводят радикальную атеистическую
пропаганду, а на самом деле они плетутся в хвосте у ученого иезуита,
поставившего этот опыт в середине XIX в. в самом центре католического мира.

Хорошо известно, что руководителями инквизиции были доминиканцы, а не иезуиты.
Конечно, и среди доминиканцев было много выдающихся умов, и среди них были
защитники коперникова учения, например знаменитый Кампанелла (не говоря о
Бруно), но по целому ряду философских, богословских и политических вопросов у
них было значительное расхождение. Характерно для соотношения сил, что в
инквизиционной конгрегации, подготовившей постановление 1616 г., было из 11
членов шесть доминиканцев, один иезуит, один августинец, один бенедиктинец и
два без обозначения принадлежности к ордену (Цейтлин, с. 123).

\textbf{Процесс Галилея и его значение.} Но если среди иезуитов были отдельные враги, а
не весь их орден, и если папа Урбан VIII во многом сочувствовал Галилею, чем же
объясняется громкий процесс 1632--1633 гг., после которого Галилей прожил еще
примерно девять лет?

Если возьмем такого добросовестного историка науки, как Уайт, неоднократно
цитированного, то там (с. 143) мы найдем такое красочное изображение последних
лет великого ученого. «До конца своей жизни --- нет, даже после окончания жизни,
преследование Галилея продолжалось. Он содержался в изоляции от своей семьи, от
своих друзей, от своих благородных занятий, и его заставляли строго держать
свое обещание не говорить о его теории. Когда, перенося интенсивные телесные
страдания от болезни и душевные муки от страданий в его семье, он просил
немного свободы, он был встречен угрозами заключения в темницу. Когда, наконец,
специальная комиссия доложила
духовным властям, что он ослеп и изнурен болезнью и печалью, ему дали несколько
больше свободы, но это немногое было стеснено строгим надзором. Он принужден
был в молчании выслушивать презрительные нападки на него и его дело; видеть,
что дружественно расположенные к нему люди были серьезно наказаны; патер
Кастелли изгнан; Риччиради, церемониймейстер Священного Двора (мастер оф те
Сакрел Палас), и Чиамполи, папский секретарь, уволены с постов папой Урбаном, а
инквизитор Флоренции получил выговор за то, что дал разрешение печатать труд
Галилея. Он дожил до того, что истины, им установленные, тщательно изгонялись
из всех церковных колледжей и университетов Европы, и когда в каком-то ученом
сочинении о нем упомянули как о "знаменитом" ("реноунд"), инквизиция заставила
заменить это словом "известный" ("ноториос")». "Такое жалкое состояние Галилея
в его последние годы описывает и известный писатель Бертольд Брехт в своей
драме «Галилей»: в нужде, тайно пишущий свои сочинения, окруженный шпионами и
проч.

Какие основания были для такой мрачной картины? Во-первых, письма самого
великого ученого, который нередко помечал их «из моей тюрьмы в Арчетри» (см.
Идельсон, 1964, с. 60). Во-вторых, официально опубликованный приговор, где
упоминалось и об угрозе пыткой, и о тюремном заключении, и о воспрещении
пропаганды своего учения (Цейтлин, с. 217). Но по приговору кроме тюремного
заключения на Галилея было наложено спасительное покаяние: «в течение трех лет
прочитывать один раз в неделю семь покаянных псалмов, оставляя за собой право
вышеназванные санкции и покаяния уменьшить, изменить полностью или отчасти» (с.
225). Что касается гонения на Галилея после его смерти, то под этим, очевидно,
подразумевается то, что папа Урбан не разрешил похоронить Галилея там, где тот
хотел быть погребенным. Вот те фактические данные, на основе которых создалась
широко распространенная легенда о последних годах жизни Галилея, активно
использованная для антирелигиозной и особенно антикатолической пропаганды. Но
дают ли сообщенные факты полную истину? Посмотрим.

Для человека, вся жизнь которого была посвящена науке, самым ужасным, является,
конечно, замалчивание результатов его работ. И однако мы видим, что как раз в
«заключении» в 1638 г. Галилей написал и опубликовал свое величайшее
произведение «Беседы о двух новых науках». Приведу слова Лагранжа из его
«Аналитической механики», 1811 (предисловие А. И. Долгова к книге Галилея
«Беседы», 1934, с. 9): «Эта наука (динамика) сполна создана в последнее время,
причем первые основы ее были заложены Галилеем... Это открытие составляет
теперь наиболее значительную и непререкаемую часть заслуг этого великого
человека. В самом деле, чтобы открыть спутников Юпитера, фазы Венеры, солнечные
пятна и т. д., требуется только телескоп и наблюдательность, но нужен
исключительный гений, чтобы установить законы природы на явлениях, которые
всегда были у всех перед глазами и тем не менее ускользали от внимания
философов». Как указывает Долгов далее, Галилей в этой книге заложил также
основы гидростатики, акустики и сопротивления материалов.

Но как же удалось Галилею преодолеть бдительность окружавших его шпионов (если
верить Б. Брехту) и опубликовать книгу? Сам Галилей пишет об этом в посвящении
к своей книге графу де Ноайль (французскому послу) (Галилей, с. 33). «Считаю
актом благодеяния с вашей стороны, досточтимый синьор, что вы соблаговолили
распорядиться моим настоящим сочинением, хотя я, как вам известно, смущенный и
напуганный несчастной судьбой других своих сочинений, принял решение не
выпускать более публично своих трудов и, чтобы не оставлять их вовсе под
спудом, сохранять лишь рукописные копии таковых в месте, доступном, по крайней
мере, для лиц, достаточно знакомых с трактуемыми мной предметами». Поэтому
Галилей и передал рукопись графу, который обещал хранить ее и ознакомить
некоторых друзей из Франции, «показав тем, что хотя я и молчу, но провожу жизнь
не совсем праздно».

Галилей хотел приступить к изготовлению других копий для Германии, Фландрии,
Англии, Испании и некоторых мест Италии, как вдруг неожиданно был извещен
известной фирмой «Эльзевир», что произведения Галилея готовы к печатанию и
желательно кому-либо их посвятить. Галилей был радостно взволнован,
естественно, посвятил книгу графу де Ноайль, так великолепно распорядившемуся
переданной ему рукописью. Он пишет: «сделать это побуждает меня не только
сознание всего того, чем я вам обязан, но и готовность ваша, да позволено мне
будет так выразиться, защищать репутацию ото всех желающих запятнать ее. Вы
опять воодушевили меня на борьбу с противниками». Опубликование нового
сочинения Галилея, признание им самим, что он передал ее иностранцу и что он не
препятствует напечатанию и готов снова бороться с противником, не повлекло
никаких репрессий для Галилея. Но из его посвящения мы видим, что он не был
строго изолирован. Эта «изоляция» относится тоже к чисто легендарной истории.
Об этой «изоляции» Цейтлин пишет, с. 249: «Многие лица посещали Галилея, не
спрашивая инквизиции. Так, Галилея посетили знаменитый английский писатель
Мильтон, философ Гоббс и, возможно, Декарт. Фердинанд II Медичи, его брат
Леопольд и другие члены дома Медичи неоднократно бывали у Галилея. В 1636 г.
некоторое время у Галилея жил его ученик Кавальери (настоятель католического
монастыря ордена иеронимитов, по существу введший понятие определенного
интеграла (Рыбников. История математики. I, 1960, с. 158. --- \emph{А.Л.}), в 1637 г. ---
математик Пери. Постоянными помощниками Галилея были три члена ордена:
Микелини, патер Клеменс и патер Амброзиус (Брехт изображает живущих с Галилеем
монахов сыщиками: плохо же они работали! --- \emph{А.Л.}). С 1640 г. у Галилея жили:
временно --- Геньери, а до самой смерти --- Нивиани и знаменитый физик Торичелли,
ученик Кастелли (ученика Галилея. --- \emph{А.Л.})».

Несмотря на существование декрета инквизиции, запрещавшего печатать какие бы то
ни было сочинения Галилея, ему удалось напечатать за границей кроме «Бесед»
латинский перевод «Диалога» (та же фирма «Эльзевир» в 1635 г.) с приложением
письма Галилея к Христине Лотарингской.

Но ведь Галилей был приговорен к тюремному заключению и сам писал «из тюрьмы»?
В настоящей тюрьме Галилей не провел ни одного
дня. По прибытии в Рим 13 февраля 1633 г. Галилей прожил два месяца у
тосканского посланника Никколини и лишь две недели провел в здании
инквизиционного трибунала, но не в тюрьме, а на квартире фискала --- прокурора
инквизиции Синяере, а потом снова у Никколини. В конце июня Галилея направили в
Сиену под надзор его друга и ученика, архиепископа сиенского Асканис
Никколомини (в механике он сделал первый шаг, отрицая мнение Аристотеля, что
тяжелые и легкие тела падают с разной скоростью (Долгов, с. 31)). Французский
поэт Сен-Аман, посетивший осенью Галилея, описывает богато меблированное и
обитое шелком жилище ученого, полное книг и рукописей, где Галилей приступил к
работе над давно задуманным сочинением «Беседы» (Долгов, с. 15--16; Цейтлин, с.
235). Ясно, что Галилей не терял времени и в самый год знаменитого процесса
приступил к своей величайшей работе. Но уже в декабре 1633 г. Галилею разрешили
поселиться на собственной вилле в Арчетри, в миле от Флоренции, под названием
«Драгоценность». Эту-то «Драгоценность» Галилей в письмах и называет «тюрьмой».
А на что же жил в «тюрьме» Галилей?

У Цейтлина читаем (с. 94): пенсия Урбана VIII, первоначально предназначенная
для сына Галилея (он от нее отказался, так как условием было выполнение
некоторых монашеских обязанностей), была передана Галилею, и он ее продолжал
получать и после своего осуждения.

А как с разлукой с семьей? Галилей вовсе не был женат, и законной семьи у него
не было. Он не был аскетом, а был сыном своего времени и эпохи Возрождения, у
него была длительная связь с венецианкой Маритой Гамба (с 1599 г. лет десять),
от которой у него был сын Винченцо и две дочери, Виргиния и Ливия, сделавшиеся
потом монахинями. Многие биографы считают, что одной из причин переезда во
Флоренцию было желание порвать эту незаконную связь. Он не обидел Гамбу, и она
при его денежной поддержке благополучно вышла замуж за Бартодуцци (Цейтлин, с.
94--95). Сам Галилей указывает в письмах, что он покинул Падую потому, что нравы
и обычаи республики не дают условий, благоприятных для научной работы. «Эти
желаемые условия я не могу получить ни от кого, кроме абсолютного князя», ---
писал Галилей. Вопреки изображению Брехта, Галилей изысканно одевался, любил
общество и был большим знатоком вин (Цейтлин, с. 22). Скончался Галилей 78 лет
8 января 1642 г. в присутствии сына Винченцо с женой, учеников Вивиани и
Торичелли и перед смертью он получил генеральное отпущение --- причащение и
благословение от самого папы Урбана VIII.

Галилей не мог не понимать, что он оставляет талантливых и преданных учеников и
последователей его идей: достаточно назвать три таких славных имени, как
Торичелли, Кавальери и Норелли, а своему биографу, Вивиани, он уже слепой
диктовал свои труды.

Получается любопытный факт, что официальное описание процесса гораздо строже
того, что фактически имело место, и это расхождение ни у кого не вызывает
сейчас сомнения. Вот как пишет Дж. Бернал в книге «Наука в истории общества»,
ИЛ, 1956, с. 236--237: «Такие горячие приверженцы Коперника, как Бруно и
Кампанелла (1568--1639), уже сделали из нового знания выводы, угрожавшие устоям
церкви, правительства, общественной морали и самой собственности (с. 183).
Бруно был сожжен на костре, Кампанелла заключен в тюрьму на долгие годы; однако
с Галилеем дело обстояло иначе: у него был большой научный авторитет и
влиятельные друзья, его католицизм не подвергался сомнению, и, кроме как в
науке, он вовсе не был революционером. Судебный процесс, как и следовало
ожидать, велся в рамках представлений и образа мышления церкви, а не Галилея, и
потому результат его был предопределен. Однако интересен тот факт, что
протоколы суда держались в секрете, по всей вероятности потому, что опасались,
как бы их обнародование не разоблачило не суровость, а относительную
снисходительность судей. Папа и его курия больше всего боялись возможной
реакции со стороны твердолобых фанатиков церкви, чем со стороны ученых. Галилей
был осужден и вынужден сделать свое знаменитое отречение, однако он подвергся
только условному заключению во дворце одного из своих друзей. Находясь в
уединении, он смог закончить свой труд о динамике и статике и опубликовать его
в последние годы своей жизни. Судебный процесс над Галилеем ознаменовал собой
целую эпоху, ибо драматизировал конфликт между наукой и религиозной догмой.
Своим фактическим провалом, ибо приговор был весьма отрицательно принят почти
всем ученым миром, даже в католических странах, процесс этот неизмеримо поднял
престиж новой революционной экспериментальной науки, особенно в тех странах,
которые уже свергли у себя власть римской церкви. Достижение Галилея выглядит
как высшая точка наступления на старую космологию. С этого момента от нее
молчаливо отказались, и астрономы-практики стали пользоваться созданной
Коперником и Кеплером теорией Солнечной системы. Сорок лет спустя законы,
выведенные Кеплером путем наблюдений, были объединены с открытыми Галилеем
законами динамики в ньютоновской теории всемирного тяготения».

Из этой цитаты ясно, во-первых, что приговор был настолько снисходителен, что
папа его не опубликовал из нежелания раздражать фанатиков церкви. Ясно также,
что процесс не только не затормозил развития науки, но вызвал еще больший
интерес к новой астрономии и тем самым стимулировал ее развитие, но Бернал
все-таки думает, что основной причиной процесса была ортодоксальность папской
курии, а фактическая снисходительность --- результат влияния друзей и проч. Такой
результат особенно ясно показывает, что процесс не мог быть следствием личной
неприязни Урбана VIII к Галилею, основанной якобы на том, что под видом
Симпличио (простака) Галилей вывел самого папу. Известно, что сам папа отвергал
эту глупую версию, так как она не имеет ни малейшего правдоподобия. Симпличио
вовсе не выведен дураком или простаком, а местами дает своим противникам весьма
умные возражения. Слово Симпличио не выдумано Галилеем, а есть фамилия
знаменитого комментатора Аристотеля, о чем пишет сам Галилей (он, следуя
Платону, вводит в свои диалоги реальных лиц, а не выдуманных). (Кстати, слово
«простак» по-итальянски симпличино, а не симпличио). И наконец, Симпличио вновь
появляется в «Беседах», написанных Галилеем в «тюрьме». Со стороны старого,
пережившего тревоги Галилея было бы неслыханным, чисто мальчишеским озорством
снова использовать это имя, если бы Галилей имел малейшее основание думать, что
папа был обижен этим именем. Ведь Урбан VIII пережил Галилея и послал ему перед
смертью папское благословение. Но говорят, Урбан преследовал Галилея и после
смерти, не позволив родственникам и ученикам Галилея похоронить его согласно
его желанию. Конечно, возмущение такой «жестокостью» кажется несколько странным
в устах наших казенных писак, которые стараются если не оправдать, то по
крайней мере замолчать судьбу сотен тысяч людей, не только не похороненных
согласно их желанию, но вообще похороненных неизвестно где в нашем «единственно
прогрессивном» государстве. Ясно, что поведение Урбана по отношению к Галилею
заключалось в том, чтобы проявить якобы наибольшую строгость к строптивому
ученому, а на деле, поступить максимально снисходительно. Разгадка заключается
в том, что процесс Галилея был вовсе не научным спором, а чисто политическим.
Это ясно вытекает из многих сводок, опубликованных в советской литературе, в
особенности ясно это изложено в неоднократно цитированной обстоятельной книге
Цейтлина (1935). Правда, этот вывод сразу не бросается в глаза при
поверхностном просмотре введения, оглавления и иллюстраций книги, но содержание
всей книги не оставляет никакого сомнения, хотя, конечно, оно становится
совершенно ясным лишь при внимательном чтении.

У нас часто говорят о необходимости рассматривать развитие науки и вообще
культуры в связи с общественными явлениями, но по принятой схеме стараются
отыскать классовую подоплеку тех или иных явлений. Сомневаюсь, чтобы такой
подход был удачен в деле Галилея, но классовый подход далеко не единственный из
возможных, и в деле Галилея сыграли большую роль национальный и
общественно-политический моменты. Дело ведь происходило в самый разгар
трагической Тридцатилетней войны (1618--1648), последней грандиозной попытки
вооруженной рукой покорить протестантизм. Она закончилась, как известно,
Вестфальским миром, после которого в первом приближении установилось по крайней
мере в большинстве стран мирное сосуществование разных религиозных идеологий ---
начало полной религиозной свободы, просуществовавшей до XX в. Но кроме этого
основного антагонизма существовали два других, не менее, а даже более
могущественных. Во-первых, национальный: испанский и немецкий (возглавляемый
Австрийской империей, бывшей тогда гегемоном немецкого мира), с одной стороны,
и итало-французский --- с другой. Во-вторых, социально-политический:
независимость церкви от государства и более или менее полное подчинение церкви
государству, в крайнем своем выражении делающееся цезаропапизмом, т. е. когда
глава государства является одновременно и главой церкви. А так как «принципиум
дивизионис» был не один, а по крайней мере три, то тут не было двух лагерей, а
была комбинация принципов трех разных антитез. Папа римский, как глава
католической церкви, естественно, должен был быть на стороне католических
армий. Но будучи главой католической церкви, он одновременно был решительным
противником цезаропапизма. По известной доктрине «двух мечей» (духовного и
светского) папа римский объединял и светскую и духовную власть в
Папской области, но с гегемонией духовной власти. При цезаропапизме было
наоборот: при объединенных в одних руках «двух мечах» гегемония была на стороне
светской власти. В начале Тридцатилетней войны католические армии шли от успеха
к успеху и дело протестантизма казалось проигранным по крайней мере в немецких
государствах, пока вмешательство чемпиона протестантства, шведского короля
Густава Адольфа (которого военные историки считают одним из величайших
полководцев всех времен и народов), не внесло крутого изменения в войну. Гибель
Густава Адольфа при Лгоцене в 1632 г. остановила его блестящие успехи, но его
преемники, шведские полководцы, сохраняли перевес шведов в войне до самого ее
конца. Урбан VIII был не первым папой, понявшим, что гегемония стран
католического абсолютизма может быть вредной для католической церкви. Еще Сикст
V понял необходимость отказа от традиционного союза папства с Испанией.
Посланный им во Францию известный кардинал иезуит Беллармин отклонил
домогательства лигистов (Священная испано-французская католическая лига борьбы
против французского короля Генриха IV, долгое время бывшего вождем гугенотов)
воздействовать на папу в их пользу. Преемник Сикста V Григорий XIV решил стать
на сторону лигистов, и тогда Беллармин демонстративно оставил Париж, за что был
вознагражден саном кардинала новым папой Климентом VIII.

Но Густава Адольфа субсидировал не только папа, но и другой видный деятель
католического мира, известный кардинал Ришелье, фактический диктатор Франции
при Людовике XIII. В своем государстве он решительно боролся с гугенотами (хотя
по дикости приемов борьбы его сильно перещеголял Людовик XIV), и он же
заключил, наряду с некоторыми другими странами (например, Венецией), союз с
Густавом Адольфом в 1631 г. при посредничестве папского нунция. Это было
сделано из государственных соображений для ослабления конкурентов --- Испании и
Австрии, полная победа которых привела бы к резкому нарушению европейского
равновесия.

Поддержка Густава Адольфа длилась недолго: в 1632 г. шведский король погиб в
бою, Ришелье был серьезно болен в 1633 и 1634 гг. Но внутри самой Австрии было
неспокойно, и главнокомандующий католической армией Валленштейн был убит в 1634
г. по приказу австрийского императора, так как его подозревали в чрезмерном
честолюбии и сношениях с Францией. Под давлением испано-немецкой партии папа
был вынужден давать субсидии католическим монархам, в чем он ранее отказывал.
Но дело защиты протестантизма было сделано, опасность полной победы
католических армий уже была исключена. По словам Григоровиуса, Урбан VIII был
последним значительным политическим деятелем на папском престоле, пытавшимся
вести независимую политику. После его смерти в 1644 г. ясно выявилась победа
испано-немецкой оппозиции и папой был выбран кардинал Памфили под именем
Иннокентия X.

Но Урбан VIII был не только главой католической церкви, но и итальянцем, и это
обостряло его неприязнь к Испании. Известно, что значительная часть Италии,
имевшая общее название «Королевство обеих Сицилий» (куда входили южная часть
полуострова южнее Папской области со столицей Неаполем и вся Сицилия), долгое
время находилась под властью Испании. Несмотря на родство испанцев и итальянцев
и религиозное единство, испанское владычество над южной Италией было
несравненно более тягостным, чем, скажем, господство Польши над Украиной.
Всякий, кто бывал в южной Италии (я там прожил три месяца в 1909 г.), а потом
переезжает в северную, поражается контрасту этих двух частей современной Италии
(как будто переезжаешь из Африки в Европу). Само собой разумеется, что
передовые люди стремились к освобождению южной Италии от испанского гнета.
Среди наиболее выдающихся упоминаем знаменитого доминиканца Кампанеллу, который
был автором картины утопического социализма «Государство солнца» и вместе с тем
решительным сторонником Коперника. Урбан VIII помог Кампанелле бежать из
неаполитанской тюрьмы во Францию, где Кампанелла и жил остаток жизни под
покровительством Ришелье.

А какую сторону в этих политических спорах принимали иезуиты? «Единодушия
мнений» у них не было, потому что этот орден был интернациональным, но
национальная принадлежность членов ордена не могла не иметь влияния.
Основателями ордена были в подавляющем большинстве испанцы, испанцами же были
три первых генерала (Лойола, Лайнец и Борджиа), но в 1581 г. генералом был
избран Клавдий Аквавива, неаполитанец по происхождению, принадлежавший к
французской ориентации. Во время Тридцатилетней войны, естественно, испанские
иезуиты и часть немецких поддерживали Испанию и Австрию. Поэтому карикатура
времен Тридцатилетней войны, изображающая, как «Северный лев (Густав Адольф)
разрывает тенета, которыми иезуиты окружали центральную Европу», и помещенная в
книге Цейтлина, не является искажением действительности, но ее неполным
отображением, так как большинство иезуитов помогали Северному льву. Видную роль
в деле Галилея играли как раз представитель немецко-испанской партии иезуит
Шейнер, о ком речь была выше, и лидер испанской оппозиции, посол Испании при
папском дворе кардинал Гаспар Борджиа.

А какую политическую позицию занимали видные ученые того времени? Они, конечно,
имели те или иные политические симпатии, но это практически не сказывалось на
их поведении. От Галилея как итальянца естественно ожидать поддержки
французской ориентации. Этому соответствует и список его друзей, хорошие
отношения с верхушкой французской партии. Есть даже легенда, что его лекции
слушал инкогнито сам Густав Адольф.

Что касается Кеплера, протестанта и во многом, как было показано, противника
Галилея, то его поведение скорее показывает его индифферентность к
происходившей в то время политической борьбе. Он долго и успешно работал у
императора Рудольфа II, одного из лидеров «католической реакции», а потом
состоял штатным астрологом у главнокомандующего католическими армиями
Валленштейна.

В окружении Урбана VIII создалась исключительно напряженная ситуация. В начале
1632 г. можно было ожидать полного разрыва между папой и Испанией и, может
быть, даже императором (австрийским, который был тогда номинально императором
всей германской
нации). Послы императора вместе с некоторыми кардиналами протестовали против
политики папы. Папа заявил, что данную войну не считает религиозной войной, так
как Густав Адольф выступает лишь против слишком возросшей силы Австрии.
Кардинал Борджиа выступил представителем ряда лиц с открытым протестом,
поднялся бурный спор. Папа готов был бы предать Борджиа суду, но испанский
король в Неаполе мог выступить с оружием на защиту Борджиа. Три года папа
добивался отозвания Борджиа как испанского посла из Рима, но испанское
правительство соглашалось на отзыв лишь ценой разрыва папы с Францией, на что в
конце концов папе пришлось пойти. Несомненно, что жизнь Урбана VIII была в
прямой опасности от возможных заговорщиков: астрологи в 1630 г. выпустили ряд
предсказаний о предстоящей смерти папы, явно выражавших намерения
испано-немецкой партии. Папа арестовал ряд «предсказателей», в том числе
некоторых лиц, связанных с Галилеем. Опасность для власти и даже жизни Урбана
продолжалась не только до смерти Галилея в 1642 г., но и до самой смерти папы.
Смерть Галилея совпала с началом войны за Кастро. Пармский герцог Одоардо
Фарнезе был в январе 1642 г. отлучен от церкви и лишен своих владетельных прав;
в ответ он во главе 3000 всадников вторгся в Церковную область и начал быстро
продвигаться к Риму при бурных аплодисментах испанцев, как говорит один
историк. Война продолжалась до 1644 г., когда умиравший Урбан VIII вынужден был
заключить тяжелый для его авторитета мирный договор с Кастро; подписав этот
договор, папа лишился чувств.

Вот на фоне каких событий происходил процесс Галилея. Судьба Урбана VIII
гораздо более печальна, чем судьба Галилея. Папа отнюдь не был фанатическим
католиком, заклятым врагом протестантизма, он мудро стремился ограничить власть
испанской и австрийской деспотий и стремился к прогрессивной цели --- изгнанию
испанцев из Италии. Перед смертью он мог думать, что эти его планы рухнули, так
как испано-немецкая партия восторжествовала.

Испанцев, конечно, выгнали из Италии, но гораздо позже.

Но совпадение процесса Галилея с трагическими событиями Тридцатилетней войны не
было случайным. В той же книге Цейтлина прекрасно показано, что напечатание
книги Галилея якобы с одобрения папы было хитро задуманной провокацией
испано-немецкой партии, во главе с Гаспаром Борджиа и личным секретарем папы
Чиамполи (Цейтлин, с. 191 и 192). Последний примкнул к испано-немецкой партии,
так как был обойден папой при назначении кардиналов (с. 198). Судя по всему,
они якобы с одобрения папы опубликовали знаменитые «Диалоги» во Флоренции. Цель
их была ясна: скомпрометировать папу перед фанатическим католическим миром, как
единомышленника учения, осужденного в 1616 г. католической церковью и добиться
его смещения (прецеденты смещения пап были, вспомним знаменитого Иоанна XXIII,
не только лишенного папского звания, но и вычеркнутого впоследствии из списка
пап, отчего в наши дни под именем Иоанна XXIII правил другой папа, оставивший
по себе самую светлую память). Чиамполи делал ставку на решительную победу
испано-немецкой партии, которая фактически запоздала всего лишь на десять лет,
и Чиамполи не дожил до нее, так как умер за год до смерти Урбана VIII.
Цейтлин приводит убедительнейшие факты в пользу такого понимания. Если бы само
напечатание книги Галилея было преступлением, то, конечно, должен был бы в
первую очередь пострадать цензор инквизиции и индекса доминиканец Никколо
Риккарди, которого называли «падро монстро» по причине необычайной учености,
красноречия и чудовищной толщины и который поставил свое «имприматур» (подлежит
печатанию). Но Риккарди нисколько не пострадал и остался папским цензором до
своей смерти в 1639 г., а Чиамполи после процесса Галилея был смещен со своего
поста, выслан из Рима в глухие местечки и умер опальным в 1643 г. Для спасения
своей власти, а может быть, и жизни Урбан VIII был принужден инсценировать
процесс над Галилеем, несмотря на безусловное свое расположение к Галилею.
Совершенно освободить Галилея от наказания значило бы для Урбана VIII признать
собственную ответственность за публикацию «Диалога».

Папа проявил такую «объективность», что поставил во главе списка судей одного
из своих заклятых врагов, кардинала Гаспара Борджиа, но для обеспечения
надежного следствия назначил в декабре 1632 г. нового генерального комиссара
инквизиции --- патера Фиренцуолу, сменившего прежнего, замешанного в деле
Борджиа. Этот патер Фиренцуола был главным военным инженером папы и прославился
усовершенствованием укреплений в римской крепости-замке Св. Ангела и постройкой
форта на острове Мальта. Фортификационные чертежи Фиренцуолы были часто на
столе папы наряду с новейшими поэтическими произведениями. Галилей предстал на
допросе не перед фанатическим богословом, а перед математиком и инженером в
монашеской рясе. Может быть, папа оказался недоволен слишком мягким результатом
следствия и суда? Нет. Фиренцуола сохранил полное благоволение папы: после
смерти Риккарди в 1639 г. он сделался главным цензором, а в 1643 г. возведен в
сан кардинала-епископа и архиепископа. Недоволен был Гаспар Борджиа. Это ясно
из того, что его подпись (наряду с подписями двух других лиц) отсутствует под
приговором, вынесенным Галилею. Лидер испано-немецкой оппозиции тем самым
бросил в лицо Урбану VIII вызов, что не столько Галилей сильно подозрителен в
ереси, сколько сам глава вселенской, католической и апостольской церкви. Не
Галилея надо было подвергнуть строгому испытанию, а собором лишенного папской
тиары Маттео Барберини --- вот о чем говорит отсутствие подписи кардинала
Борджиа. Любопытно, что на отсутствие подписи Борджиа и двух других лиц впервые
обратил внимание известный историк математики Мориц Кантор, хотя, казалось бы,
это прежде всего должно было интересовать обычных историков.

Но, может быть, по закону Галилея и нельзя было присудить к более тяжелому
наказанию? Мы знаем, что папа в то время был самодержавным государем Церковной
области, а самодержцы всегда не слишком считаются с законами, но в данном
случае Галилей мог бы быть присужден к самому тяжкому наказанию --- сожжению на
вполне законном основании. Ведь осуждение учения Коперника было в 1616 г., и
тогда кардиналу Беллармину поручено было уговорить Галилея отказаться от этой
ереси; кардинал имел беседу с Галилеем, и от открытой защиты учения Коперника
Галилей в общем воздерживался до опубликования «Диалогов». Поэтому считалось,
что Галилей
«раскаялся» уже в 1616 г., подчинившись указаниям Беллармина. Но если он в 1616
г. уже раскаялся, а в 1632 г. вновь согрешил, то он мог бы по закону быть
объявлен неисправимым еретиком (херетикус релапсус) и передан светской власти
для казни «со всей кротостью и без проливания крови», т. е. сожжению. Цейтлин
указывает, что даже в таком крайнем случае Урбан VIII нашел бы пути и средства,
чтобы избавить Галилея от этой страшной участи, но чрезвычайных мер не
потребовалось, так как Галилей предъявил письмо кардинала Беллармина, из коего
явствовало, что разговор с Беллармином не привел к отречению. Поэтому суд имел
право рассматривать Галилея не как неисправимого еретика, а как человека,
впервые попавшего на суд инквизиции, тогда в случае раскаяния ему смертная
казнь не угрожала. Но Беллармин в 1632 г. был уже в могиле, и закрытый суд,
если бы он хотел погубить Галилея, легко мог бы просто уничтожить письмо
Беллармина. Важна была инсценировка громкого процесса и, возможно, более мягкий
приговор. Как известно, существует легенда, что Галилей после торжественного
отречения на коленях от доктрины Коперника встал с колен и, топнув ногою,
сказал: «Э пур си муове» --- а все-таки она движется. Уэвель (1867, т. I, с 505)
справедливо замечает: «Эти слова представляются иногда героическим изречением
человека, преданного своему убеждению и истине наперекор преследованиям: я
думаю, что мы можем естественнее представить себе эти слова сказанными в виде
шуточной эпиграммы на ухо кардинальскому секретарю, с полной уверенностью, что
они будут непосредственно переданы его господину». Разумеется, громко таких
слов Галилей не мог произнести, но если эти слова он сказал на ухо, например,
патеру Фиренцуоле, то тот, наверное, ответил бы ему: «Тише, тише, Галилей, как
бы Борджиа не услышал».

Но ведь все-таки Галилей отрекся от своих убеждений, и это отречение должно
было быть ему очень мучительным. Но совершенно несомненно, что Галилей никогда
не защищал коперниковской теории как абсолютной истины, и этого нет в
«Диалогах». Вот что сам Галилей пишет на с. 22 своего труда (Галилей. Диалоги.
1948): «Благоразумному читателю. В последние годы в Риме был издан спасительный
эдикт, который для прекращения опасных споров нашего времени своевременно
наложил запрет на пифагорейское мнение о подвижности Земли... Ради этой цели я
взял на себя в беседах роль сторонника системы Коперника и излагаю ее сначала
как чисто математическую гипотезу, стараясь далее при помощи разных
искусственных приемов доказать ее превосходство не над учением о неподвижности
Земли вообще, а над тем, которое защищается людьми, являющимися перипатетиками
по профессии, ложно носящими это имя, ибо они довольствуются безоговорочным
почитанием тени и, не пытаясь размышлять самостоятельно, держатся лишь за
заученные на память, но плохо понятые четыре принципа» (Четырьмя принципами или
видами причинности Аристотеля являлись, как известно, форма, материя, движущая
причина и цель). Но все изложение «Диалогов» таково, что учение Коперника
представляется гораздо более убедительным, и поэтому совершенно ясно, что
Галилея можно было обвинить не в открытой защите осужденного католической
церковью учения, а в неискренности. Метод
Галилея был использован позднее многими, в частности французскими
энциклопедистами XVIII в., которые защищали, например, атеизм в своей
«Энциклопедии», как будто его опровергая. Думаю, что и Цейтлин в некоторой
степени использовал этот метод для усыпления бдительности сталинской цензуры,
так как некоторые выражения в его прекрасной книге наводят на такую мысль, о
чем будет упомянуто дальше.

То, что Галилей в своей книге еще до отречения принял указания Беллармина
(теория Коперника --- вполне допустимая математическая гипотеза, но не абсолютная
истина), как правило, неизвестно пишущим о Галилее, и многие «принципиальные»
люди вроде Брехта обвиняют Галилея в измене своим убеждениям. Цейтлин на такие
обвинения пишет на с. 232--233: «Некоторые старые и новые ученые и неученые
лицемеры поднимают ханжеские вопли, порицая поведение Галилея на процессе. С их
точки зрения, Галилею следовало бы гордо взойти на костер инквизиции подобно
Бруно и доставить большое удовольствие верным псам господа бога. Разумеется,
нужно воздать хвалу героизму Бруно и всех гигантов учености, духа и характера
эпохи Ренессанса. Но надо быть справедливым к Галилею. Его поведение на
процессе и отречение только тогда можно было бы назвать позорной трусостью,
если бы Галилей изменил себе, изменил той стратегии и тактике, которой он
придерживался всю жизнь и в целесообразности которой был непоколебимо убежден».
Но дальше почему-то Цейтлин пишет, что «из тщательной и всесторонней оценки
реального соотношения сил он пришел к выводу, что наилучший метод борьбы с
феодально-католической реакцией, это метод рыси, лицемерно поднявшей глаза к
небу и раздирающей когтями... трехглавого цербера папизма». Это последнее
совершенно, по-моему, неверно. Никакой цели борьбы с феодально-католической
реакцией Галилей не ставил, так как был чужд политике и придерживался
консервативных взглядов, и он вовсе не стремился бороться с папизмом. Как
правильно указывает Выгодский (Галилей и инквизиция, ч. I, 1934), Галилей и его
сторонники вовсе не были настроены антирелигиозно и антиклерикально.
Аполитичность Галилея ясна из того, что главным стимулом для переезда из Падуи
во Флоренцию он считает то, что нравы и обычаи республики не создают достаточно
условий для научной работы и что их он найдет у абсолютного князя (Цейтлин,
1935). Галилей был великолепным представителем беспартийной и аполитичной
науки. Эту точку зрения защищал в своей книге о Галилее наш выдающийся
математик Стеклов, за что на него обрушился тот же Выгодский, видимо,
удовлетворяя требованиям времени и не приводя серьезных возражений. Книгу
Стеклова я, к сожалению, не мог достать в библиотеке. Науке Галилей не изменил
и весь свой великий талант развил полностью. Те требования, которые ему
предъявлялись, не мешали развитию науки, так как даже коперниканская доктрина,
как было показано выше, допускалась к развитию как «математическая гипотеза»,
но не как абсолютная истина. Мы знаем, кроме того, что развиваемые им взгляды
встречали наиболее ожесточенное сопротивление со стороны перипатетиков, которые
препятствовали развитию его основных работ по механике, к которым католическая
церковь никаких претензий не предъявляла. Но по отношению к главным врагам
прогрессивной науки того
времени, догматическим и фанатическим перипатетикам, Галилей не сделал ни
малейшей уступки, сохраняя вместе с тем уважение к Аристотелю, как того и
заслуживал этот выдающийся философ.

Разумеется, политиканствующие писатели типа Бертольда Брехта не могут этого
понять. Для них прогрессивный ученый обязательно материалист, атеист и
сторонник революционных методов борьбы с отживающим классом. Если в Средние
века было распространено мнение «философия --- служанка богословия», то
современные сторонники партийной и политической культуры считают: «наука и
философия --- служанки политики и атеизма». Так (вопреки всему содержанию своей
книги) пытается изобразить дело и Цейтлин, с. 263: «Борьба Галилея за
революционное материалистическое учение Коперника --- это сущность и ядро его
жизни и деятельности. Эту борьбу Галилей вынес перед лицом широкой аудитории,
популяризуя коперниканские антирелигиозные идеи». Учение Коперника может быть
названо революционным в двух смыслах: 1) как подлинная революция в науке, 2) по
названию «ле революционис орбиум целестиум», но слово «революцио» по-латински
(в классической латыни этого слова, видимо, нет, но производится оно от глагола
«револьно») переводят «Об обращениях», что никакого отношения к политическим
революциям не имеет. Кроме того, совершенно ясно, что ядром деятельности
Галилея было создание новой механики, чего он успешно и достиг.

Но Урбан VIII вовсе не был представителем одного из двух пресловутых «лагерей»:
феодально-католического и буржуазно-капиталистического. Как папа, он, конечно,
должен был бы идти по линии феодально-католической реакции, а он на деле был
сторонником французской партии, которая отображала в известной степени
прогрессивные тенденции буржуазии, относительно заинтересованные тогда в
развитии науки.

Но можно ли сказать, что Ришелье, фактический глава Франции, был действительно
сторонником буржуазии? Насколько мне известно, Реформация вообще и гугеноты в
частности и были наиболее яркими представителями буржуазии, но с гугенотами
Ришелье вел беспощадную войну. В интересах какого класса? В интересах создания
из Франции централизованного абсолютистского государства, и эта линия
продолжалась и дальше. Несмотря на страшную революционность французских
якобинцев, лозунг, унаследованный от Ришелье, «Франция --- единая и неделимая»
проводился ими с несравненно большей решительностью и последовательностью, чем
подлинно великий лозунг Революции: «свобода, равенство и братство». Линия
Ришелье привела к созданию многочисленного класса дворянства --- подлинных
паразитов народа, что и привело в конце концов к великой трагедии. В Англии
феодализм не искоренен до сего дня (лендлорды и палата лордов). Германия
объединилась только во второй половине XIX в., и это объединение привело тоже к
ужасному результату.

Но Франция, которая обладала культурной гегемонией в XVII и XVIII вв., потеряла
ее и во многих отношениях отстала от Англии, Германии и Австрии. Конечно,
победа Австрии в Тридцатилетней войне была нежелательной во многих отношениях
для прогрессивного человечества, но это не значит, что австрийский
государственный строй во
всех отношениях был хуже французского и являлся отражением наиболее
реакционного класса. К. А. Тимирязев в своей работе «Источники азота растений»
(Избр. соч. в четырех томах, т. 2, 1948, с. 147) сравнивал двух монархов,
оставивших противоположную славу. Один, Фридрих II, сошел со сцены,
сопровождаемый восторженным удивлением современников и потомства. Другой, Иосиф
II, еще при жизни вынес горькое сознание, что все его благие намерения
потерпели крушение, встретив отпор, главным образом, в своекорыстии правящих
классов, и сошел в могилу, оставив по себе память коронованного неудачника. «Но
преследовавшие его в течение всей его жизни неудачи не помешали, конечно,
беспристрастной истории видеть в нем одного из просвещеннейших и передовых
представителей своего века, как не помешали воспоминанию о нем сохраниться в
благодарной памяти австрийского крестьянина, освобожденного им от крепостной
зависимости. Если всякому знакомо изображение Фридриха II на коне, окруженного
сонмом полководцев, то теперь еще, в глухих уголках Австрии, можно встретить
популярную гравюру, изображающую Иосифа II, пашущего плугом». Дальше Тимирязев
указывает, что постоянная забота Иосифа II о насущных потребностях крестьян
выразилась в том, что в 1784 г. он возвел в дворянское достоинство Иоганна
Христиана Шубарта, дав ему титул фон Клеефельд. Поводом к этому были не
какие-нибудь выдающиеся подвиги на поле брани, не блестящая деятельность на
поприще дипломатии или администрации, --- нет, вся заслуга Шубарта заключалась в
том, что он деятельно, печатным словом и примером, пропагандировал возделывание
клевера и тем, по словам историка земледелия, "положил краеугольный камень
благосостоянию немецкого поселянина". Пожалуй, вряд ли можно считать Иосифа II
неудачником, если, конечно, не придерживаться моральной чумы современности ---
ультрапатриотизма, видящего основную задачу правителей в расширении границ. Но
даже и тут Франция отстает перед своими соседями Англией и Испанией. Франция
еще до Революции потеряла Канаду, а воинственный Наполеон, проливший огромное
количество крови, продал Соединенным Штатам обширную провинцию Луизиана
(значительно превышающую одноименный штат Луизиана). Лица, говорящие
по-английски, занимают огромную территорию на Земле, английский язык уже
фактически сделался международным, на нем печатается больше половины научных
трудов. Что касается Испании, то, конечно, и сейчас она сильно отстает от
Франции в культурном и общественном отношении. Но испанский народ вместе с
португальским, наряду с многими ужасами, внесенными ими в историю (а кто не
вносил ужасов?), выполнили и великую историческую миссию в Латинской Америке.
Этот комплект стран, по размерам и по населению примерно равный СССР,
представляет удивительное разрешение той расовой проблемы, которая так
потрясает англосаксонские страны. По недавно выпущенным у нас справочникам
«Зарубежные страны» (1957) и «Латинская Америка» (1962) можно составить себе
представление о племенном составе 20 латиноамериканских стран. За пять лет
население этих стран увеличилось со 170 миллионов до 206 миллионов. Состав
населения чрезвычайно колеблется, и наряду со странами, где население состоит
почти исключительно из потомков европейцев (Аргентина, Уругвай), есть страны,
где белые почти отсутствуют или все перемешано. Данные двух справочников иногда
сильно расходятся; например, для Мексики (второй по величине стране Латинской
Америки, население которой, по новейшим данным, приближается к 50 миллионам)
«Зарубежные страны» дают: испано-индейские метисы --- более 50\%, индейцы --- около
33\%, креолы, испанцы и др. 16--17\%, по «Латинской Америке» --- метисы около 79\%,
до 20\% --- индейцы, и выходит, что лиц чисто испанского происхождения почти нет.
По-видимому, испанцы и индейцы так сильно перемешались, что совершенно
невозможно установить точную границу между краснокожими и «бледнолицыми». В
других странах имеется значительная примесь негров или мулатов. Всего цветных
по «Латинской Америке» 121,7 миллиона, или 59,0\%, белых --- 84,6 миллиона, или
41,0\%. По «Зарубежным странам» (данные на пять лет раньше) цветных 98,5
миллиона, или 57,9\%, белых 71,7 миллиона, или 42,1\%. При такой чересполосице
самых разнообразных племен можно было бы ожидать значительного расового
антагонизма, но об этом ничего не слышно. Несмотря на отсутствие
государственного единства, имеется единство культуры (португальский язык в
Бразилии, испанский в остальных), единство католической религии, верной
хранительницы великих принципов интернационализма и антирасизма.

Не только злое принесли Испания и Португалия в историю человечества, хотя,
несомненно, они принесли много злого. И пусть те политики и политиканы,
которые, достигнув определенного (иногда спорного) прогресса, при помощи злых
средств, примут в соображение эти показания в пользу латинских стран. «Каким
судом вы судите, тем и вас будут судить». Сейчас страны Латинской Америки
несомненно вступили на путь стойкого культурного прогресса, и испанский язык
сделался одним из мировых языков, конкурируя с французским. Поэтому,
рассматривая процессы истории ретроспективно, трудно резко провести разделение
на козлищ и овец. В некоторых странах Латинской Америки (мне известно это для
Мексики и Уругвая) смертная казнь полностью отсутствует, как она отсутствует
полностью и в ФРГ (отсутствовала и в Израиле до процесса Эйхмана). Мне
неизвестны социалистические страны, где бы этот позор юриспруденции
отсутствовал. Мне известно также, что Фидель Кастро за открытое восстание
против диктатора Батисты был подвергнут лишь тюремному заключению, а сейчас
ввел смертную казнь даже за воровство (конечно квалифицированное). Где же
козлища, где же овцы?

Возвращаясь к Галилею и Тридцатилетней войне, можно сказать, что политический
характер процесса Галилея совершенно ясен, но признать здесь ведущими классовые
влияния можно только тогда, когда находишься в плену устарелых догматов.

Подведем итог судьбе Галилея. Можно ли в целом считать его несчастным
человеком, заслужившим тяжелыми лишениями и страданиями свое право на
бессмертие? Так его рисует, например, Брехт, который считает, что последние
годы он мучительно казнил себя за «отступничество». Несомненно, что Галилей
очень строго реагировал на ту борьбу, которую он вел в течение всей своей жизни
с перипатетиками и был омрачен судьбой своих последних лет. Темперамент его
был очень живой, и он явно выражен даже там, где Галилей встречал справедливые
возражения. Он разочаровался в Урбане VIII. Зная истинные взгляды этого умного
и образованного человека, он несколько наивно думал, что с его восшествием на
папский престол все прогрессивные идеи в науке получат полное развитие, и был в
этом горько разочарован. Видимо, будучи далеким от политики, он не понимал
истинного смысла процесса. Но взглянем на судьбу Галилея объективно. Родился он
в 1564 г. и уже в 1589 г. был профессором в родном городе Пизе. Там у него
вышли конфликты с перипатетиками, но он перешел на такую же должность
профессора математики в Падую, где и пробыл с 1592 по 1610 гг. Оттуда он
совершенно добровольно перешел во Флоренцию, куда великий герцог Тосканский
пригласил его с большим по тому времени содержанием и званием первого
математика и философа его высочества. Падуанский период Галилей потом считал
счастливейшим периодом своей жизни. При отъезде во Флоренцию он расстался, как
известно, со своей невенчанной женой, и, очевидно, у него пропал вкус к
семейной жизни, так как он новой семьи не завел, несмотря на свое возросшее во
Флоренции экономическое благосостояние. Во Флоренции он пробыл вплоть до
процесса 1633 года, но и после процесса работал и закончил свое величайшее
произведение. Он был блестящим лектором и стилистом, пользовался огромной
популярностью и известностью, жизнь вел отнюдь не аскетическую, вполне в духе
своей эпохи. Неужели такого человека можно назвать несчастным? Есть великая
мудрость в древнем поверье, изложенном в старой легенде о Поликрате (поликратов
перстень):

Здесь вечны блага не бывали,

И никогда нам без печали

Не доставалися они.

Как это ни странно, но сходную мысль высказывает и такой представитель
современной позитивной науки, как Карл Пирсон (Грамматика науки. Рус. пер.,
изд. «Шиповник», с. 13). Он указывает, что новые идеи, связанные с Дарвином,
медленно проникают в наши общие представления. «Эта медлительность не должна
нас обескураживать, ибо одним из важнейших фактов социальной устойчивости
является инертность, или, скорее, даже активная вражда, с которой человеческие
общества встречают всякие новые идеи. Это --- горнило, в котором шлаки отделяются
от чистого металла и которое спасает социальное тело от ряда бесполезных и,
может быть, даже гибельных потрясений. То, что реформатор часто должен также
быть и жертвой, есть, может быть, не слишком дорогая цена за осторожность, с
которой должно двигаться общество как целое».

И если сравним судьбу Галилея с судьбой многих других пионеров мысли, то придем
к заключению, что за свою великую и заслуженную славу Галилей заплатил совсем
дешево.

Ну а теперь, прежде чем распрощаться с делом Галилея, нам остается рассмотреть
один или, вернее, два пункта, наиболее важных с точки зрения философии науки. А
что бы случилось, если бы могущественная власть признала правоту Галилея и кто
же был прав в споре Галилея с Беллармином?

Спор ведь шел не о том, чтобы вовсе осудить систему Коперника, а о том,
признать ли ее как удобную математическую гипотезу или как абсолютную истину.
Что бы случилось, если бы могущественная власть, взявшая на себя право
руководить всей жизнью страны, как экономической, так и культурной, «утвердила»
бы «Диалоги» Галилея и (как это свойственно могущественной власти) стала бы
проводить в жизнь свое утверждение? Разумеется, так как власти несвойственно
быть хорошо знакомой с отдельными науками, она не позволила бы выделить кое-что
из «Диалогов» как ошибочное, напротив, всякая критика отдельных мест
рассматривалась бы как принципиально недопустимый ревизионизм. А так как
Галилей критиковал Кеплера, то, значит, дальнейший этап коперниковой теории был
бы запрещен. Так как Галилей не допускал и считал совершенным суеверием влияние
небесных светил на Землю, то не позволено было бы появиться трудам Ньютона;
развитие гелиоцентрической теории было бы полностью остановлено. Напротив, мы
видели, что отказ Галилея считать гелиоцентрическую теорию за абсолютную истину
никакой задержки в развитии астрономии не вызвал.

А отсюда получается ответ и на другой вопрос. Нельзя сказать, чтобы в споре
Галилея с Беллармином Галилей был полностью прав, а Беллармин полностью не
прав. Конечно, Беллармин исходил из религиозных догматов, но он делал такие
уступки свободному научному мышлению, которые не стеснили развития науки, и в
этом он подошел фактически к современной научной гносеологии, связанной с
именами Кирхгофа, Маха, Дюгема, Пуанкаре и других и которой мы обязаны в
значительной степени тому феноменальному прогрессу, который мы наблюдаем в
физике и других точных науках в XX в. Один из лидеров этого направления, П.
Дюгем, крупнейший историк физики, и формулирует основное положение: «Всякий
физический закон ни правилен, ни неправилен, а только приблизителен»,
«Экспериментум круцис вещь в физике невозможная» и др. Дюгем открыто объявляет
о своем сочувствии идеям Э. Маха, и потому наши казенные философы считают его
совершенно неприемлемым. Не так думал Ленин. Он в ряде мест «Материализма и
эмпириокритицизма» критикует Дюгема за его махизм, но вместе с тем он указывает
(1948, с. 294), что «в целом ряде мест он (Дюгем) вплотную подходит к
диалектическому материализму». Ленин там же цитирует слова Дюгема: «Борьба
между реальностью и законами физики будет длиться бесконечно; всякому закону,
который сформулирует физика, реальность противопоставит, рано или поздно,
грубое опровержение --- опровержение посредством факта; но физика будет неутомимо
ретушировать, видоизменять, усложнять опровергнутый закон» --- и прибавляет: «Это
было бы совершенно правильным изложением диалектического материализма, если бы
только автор твердо держался за существование, независимое от человечества,
этой объективной реальности». Таким образом, в области гносеологии Ленин не
находит разницы между махистскими взглядами Дюгема и диалектическим
материализмом. Различие заключается лишь в онтологии, учении о истинно сущем.
Ленин (в данном случае совсем не диалектически, а догматически) решает его в
смысле материалистическом, а как решают махисты? Никак, так как махисты считали
главной заслугой махизма полное изгнание метафизических признаков, полную
ненужность онтологии (обычно ее называют также метафизика, но этот термин имеет
слишком много смыслов). Это --- свойство всякой позитивной философии, включая
Конта, Спенсера и др. Конт и различал три периода в развитии человеческой
культуры: теологический, метафизический и научный. В научном периоде «наука ---
сама себе философия» и никакой особой философии не нужно. Сейчас многие
выдающиеся ученые отказываются от этой точки зрения. Назову два имени:
Эйнштейна и Гейзенберга, но оба, критикуя махизм (заслуги которого они
признают), в онтологии приходят к объективному идеализму (разных смыслов), но
не к материализму. Огромный рост разных идеалистических направлений в наше
время (при полной или почти полной стагнации материализма) показывает, что и в
области онтологии Беллармин не может считаться совершенно устаревшим. Недаром
такую известность приобрел в наше время выдающийся иезуит Тейяр де Шарден.
Любопытно, что в той же книге Цейтлина есть указания, что и сам Галилей имел
некоторый уклон к идеалистическому формализму. Цейтлин даже знаменитый лозунг
Ньютона «Не делаю гипотез» считает иезуитским, равноценным принципиальному
отказу от познания сущности истинных причин явлений природы; и Декарта
(который, как известно, получил воспитание в иезуитской школе) Цейтлин считает
представителем философского иезуитизма, избравшим своим девизом: «Хорошо живет
тот, кто хорошо скрывается». Резкое расхождение, казалось бы, двух непримиримых
взглядов стерлось, и общий вывод можно сделать тот, что никогда не следует
возводить в абсолютный догмат никакое, казалось бы, самое прочно обоснованное
положение. Но это не означает абсолютного скептицизма, а лишь правильный
пробабилистический подход, и Дюгем неоднократно цитирует великого мыслителя
Паскаля: «Мы обнаруживаем бессилие в доказательстве --- бессилие, которое никакой
догматизм победить не может; у нас есть идея истинного, которую весь пирронизм
победить не может» и «Когда он (разум) слишком превозносит себя, я принижаю
его; когда он слишком унижается, я превозношу его».

А теперь взглянем немного на прошлое глазами современника. Процесс
Галилея считался одним из самых блестящих доказательств вреда религии
вообще и католической церкви в частности, и в XIX в. этот аргумент
имел известное основание, так как в этом веке, в особенности после
окончания наполеоновских войн (так называемая «викторианская эра»),
господствовала идея о непрерывном монолитном прогрессе человечества,
вечный мир казался целью, которую можно достичь без специальных
усилий, и в смысле политической и религиозной свободы было достигнуто
положение, не имевшее прецедента в истории человечества. Были
люди самых разнообразных взглядов, предупреждавшие, что достигнутое
благополучие иллюзорно, что целый ряд особенностей общественного строя
Европы (парламентаризм, свобода, отсутствие расовых предрассудков и т.
д.) непрочны, но это, как правило, было гласом вопиющего в пустыне. Но
пришел XX в., и возникли такие явления, о которых не могли и подумать
в ХIХ в.
Преследование науки и свободомыслия проводилось в ряде культурных стран с таким
ожесточением и с таким размахом, который оставляет совершенно в тени процесс
Галилея. Уже не предполагается выбор между трактовкой того или иного учения как
абсолютной истины или математической гипотезы. Принималась абсолютная
достоверность того или иного учения, причем за критерий достоверности прежде
всего принимались его отношения с теми или иными политическими принципами и с
атеизмом. Если Беллармин и его единомышленники говорили: «Это противоречит
Священному Писанию; поэтому недопустима защита этого положения, как абсолютной
истины, но вполне допустима как математической гипотезы», то современные
атеистические аналоги Беллармина говорят: «Это противоречит марксизму, это
приводит к поповщине; поэтому учение должно быть отвергнуто "с порога", без
рассмотрения доводов». Мы знаем, каков был результат. Но несмотря на
потрясающую серию провалов в «руководстве культурой», современные фельдфебели,
данные нам в вольтеры, продолжают утверждать, что они представители
единственного честного, научного и прогрессивного течения мысли, а все их
противники --- пережитки капитализма, приказчики буржуазии, мракобесы и т. д. и
все, что фельдфебели делают, считается не противоречащим конституции,
гарантировавшей свободу мысли и слова.

Я разобрал подробно процесс Галилея, так как до сих пор этот процесс в
популярной и художественной литературе и в учебниках излагается совершенно
неправильно. Я не использовал никаких неопубликованных источников, и
большинство цитируемых мной книг изданы по-русски в наших издательствах. Но из
разговоров со многими научными работниками, часто даже весьма высокой
квалификации, я увидел, что ученые совершенно не знают подробностей этого
процесса и Галилей по-прежнему фигурирует как пример ожесточенного гонения
науки со стороны католической церкви.

4. Общее значение церковных преследований

Дело Галилея было мной подробно разобрано потому, что, во-первых, это самый
известный случай столкновения науки с господствующей церковью и, во-вторых,
потому, что по этому делу сохранилось много документов, позволяющих
восстановить и сейчас истинное значение процесса. Но мне могут возразить, что
ведь жертв-то инквизиции было очень много и можно думать, что в числе многих
безымянных жертв были и ученые, о которых почти ничего или ничего не
сохранилось. Этого отрицать, конечно, невозможно, и поэтому полезно разобрать,
во имя чего же (если не против науки непосредственно) сжигались на кострах
многие тысячи жертв. Мотивы борьбы достаточно хорошо известны, и жертвы
инквизиции можно разделить на четыре главных категории.

а) Еретики

Это, конечно, главный официальный пункт обвинения со стороны инквизиции и
других религиозных организаций, возникший очень
скоро по проникновении и особенно после торжества христианства. Прекрасно
сказано у А. К. Толстого (перевод «Коринфской невесты» Гёте).

И богов веселых рой родимый

Новой веры сила изгнала,

И теперь один царит Незримый,

Одному Распятому хвала.

Агнцы боле тут

Жертвой не падут,

Но людские жертвы без числа!

А несколько раньше:

Где за веру спор,

Там, как ветром сор,

И любовь и дружба сметены.

Преследование инаковерующих было во все времена у всех народов, и большая или
меньшая интенсивность борьбы была связана как с большей или меньшей силой
возникшего еретического учения, так, конечно, и с влиянием посторонних
факторов. Мы знаем, что и Вселенские соборы созывались для борьбы с ересями, и
Символ веры в значительной степени содержит отрицание тех или иных положений,
выдвигавшихся ересиархами. Особенное значение и особенное распространение в
раннее христианство приобрела знаменитая ересь Ария, и борьба между
ортодоксальными христианами и арианами нередко принимала характер кровавых
столкновений.

Через некоторое время ортодоксия на Западе восторжествовала. Но новая страшная
религиозная война возникла в конце XII и начале XIII в. на юге Франции
(Лангедок) в связи с альбигойской ересью. Как пишет Гегель (Философия истории,
соч., т. 8, 1935, с. 370): «Несколько крестовых походов, которые могут только
возбуждать отвращение, было предпринято и против южной Франции. Там развилась
прекрасная культура: благодаря трубадурам процветала свобода нравов вроде той,
которая существовала при Гогенштауфенах в Германии, но с тем различием, что в
первой оказывалось нечто напыщенное, а вторая была искреннее. Но как в северной
Италии, так и в южной Франции распространились мечтательные представления о
чистоте; поэтому папы стали проповедовать крестовый поход против этой страны.
Святой Доминик отправился туда с многочисленными войсками, которые беспощадно
грабили и убивали виновных и невиновных и совершенно опустошили эту цветущую
страну». Именно в начале XIII в. были основаны ордена доминиканцев (1215, по
имени св. Доминика, 1170--1221), францисканцев (Франциск Ассизский, 1182--1226),
а также инквизиция, приобретшая столь громкую и печальную известность.

Известна «установка» Симона де Монфора старшего (не смешивать с младшим того же
имени, добившимся учреждения парламента в Англии) после штурма одного из
альбигойских городов: истребить все население. Ему возразили: «Но там есть и
католики, зачем же их
истреблять?» --- «Бей всех! --- возразил не имеющий сомнений рыцарь. --- На том свете
разберут, и те, кто убиты невинно, сразу попадут в Царствие небесное: они
ничего не потеряют».

Совершенно ясно, что такие картины могут вызвать только отвращение, тем более
что доминиканцы видели свою заслугу именно в беспощадном искоренении всяких
ересей. Они использовали даже игру слов и охотно производили название своего
ордена не от имени Доминика, а читали «доминиканес» как «Домини канес», т. е.
«псы Господа».

Но если бы доминиканцы были только или в основном «псы Господа», беспощадно
терзавшие всех, кого они считали еретиками, то как могло случиться, что этот
орден существует до настоящего времени, что в числе его членов мы видим такие
имена, как Альберт Великий (1205--1280?, даты сомнительны), его ученик Фома
Аквинат (1224--1274), Джироламо Савонарола (1452--1498), Джордано Бруно
(1548--1600), Томазио Кампанелла (1568--1639), и что в наше время знаменитый
французский архитектор Корбюзье, пытавшийся принимать участие в строительстве
Москвы, кончил тем, что проектировал церкви доминиканцев? Неужели все это
только «псы Господа»? Кто были эти люди?

Альберт Великий (родился в 1193 или 1205, умер в 1280 г.), германский
провинциал (по современным понятиям, областной секретарь) ордена,
следовательно, занимал он высокий пост в ордене. Цитирую Гегеля (Истор.
философии, кн. 3, соч., т. XI, 1935, с. 139). «Под его наукой понимали тогда
главным образом колдовство, ибо, хотя схоластике, в собственном смысле этого
слова, колдовство было совершенно чуждо и она была совершенно слепа к природе,
он все же занимался явлениями природы и между прочим изготовил говорящую
машину, увидя которую его ученик Фома Аквинский испугался, а затем разбил ее
вдребезги, ибо видел в ней создание дьявола. Также и тот факт, что он принял и
угостил Вильгельма Голландского глубокой зимой в цветущем саду, тоже приводится
как колдовское дело, между тем как мы находим совершенно естественным зимний
сад у Фауста». Альберт Великий оставил много трудов, дошедших до наших дней,
которые сейчас изучаются, так как интерес к истории науки сейчас чрезвычайно
велик.

Ученик Альберта, знаменитый Фома Аквинат (1224--1274) и до сего времени
пользуется влиянием как основатель томизма, что в основном сводится к
примирению Аристотеля с христианством. Менее известно, что Фома Аквинат был
также выдающимся политическим мыслителем. Он, следуя в этом отношении, конечно,
античной традиции Платона и Аристотеля, был решительным противником всякой
тирании. Вот что мы читаем у Чичерина (Политические мыслители древнего и нового
мира, вып. I, 1879, с. 93): «Вопрос ставится таким образом: есть ли возмущение
смертный грех? Возмущение, отвечает Фома, противополагается единству народа,
живущего в государстве: но народом, по определению Цицерона, которое приводит и
Августин, называется не всякая толпа, а собрание людей, связанное согласием
права и общением пользы; следовательно, единство, которому противополагается
возмущение, есть единство права и общей пользы, а потому возмущение есть
смертный грех. Но, с другой стороны, на том же самом основании,
восстание в защиту общей пользы не должно называться возмущением и считаться
грехом... Тираническое управление само но себе неправедно, ибо оно
устанавливается не для общей пользы, а для частной; поэтому ниспровержение
подобного правления не имеет значения возмущения, разве оно совершается так
беспорядочно, что народ более терпит от восстания, нежели от тирании. В
тираническом правлении возмутитель скорее сам тиран, который поддерживает
раздоры в народе, чтобы властвовать безопаснее. Фома допускает даже
тираноубийство. По поводу возражения, что Цицерон хвалил убийц Юлия Цезаря, он
говорит: это относится к тому случаю, когда кто-то насильственно захватил
власть и нет прибежища к высшему; тогда тот, кто для освобождения отечества
убивает тирана, получает похвалу и награду».

Старое изречение ап. Павла «Несть бо власть аще не от Бога», которое обычно
толкуют как безоговорочное подчинение всякой власти, уже у первых христиан
подверглось ревизии и притом не у еретиков, а у ортодоксов. В том же сочинении,
с. 94, читаем: «У греческих отцов Церкви является уже зародыш учения, что от
Бога происходит только существо власти, а не та или другая ее форма и не
принадлежность ее тому или другому лицу. У Фомы Аквинского мы встречаем
дальнейшее развитие этих начал, которых последствия мы увидим ниже. От Бога, по
мнению знаменитого схоластика, проистекает только общий порядок подчинения и
управления, а не приобретение власти тем или другим лицом и не то или другое ее
употребление. Все это --- человеческое дело, которое может быть праведно и
неправедно. Поэтому подданные отнюдь не обязаны безусловным повиновением
властям. Повеления правителей для них обязательны настолько, насколько этого
требует порядок правды. Но если мы спросим: кто же судья праведности
предписаний? и не превратится ли общественный порядок в чистую анархию, если
каждому подданному дано будет право отказывать властям в повиновении, когда он
считает известное повеление несогласным с общим благом? --- то на эти вопросы мы
не найдем ответа. Фома требует только, чтобы восстание не приносило больше
вреда, нежели пользы, --- ограничение весьма неопределенное. Понятно, какой
простор подобное учение оставляло и свободе подданных, и вмешательству
церковной власти. Мы увидим впоследствии, каким образом из этих начал
выводились чисто демократические теории». Чичерин дальше указывает, что сам
Фома был далек от пристрастия к демократии, но он ограничивал княжескую власть
и давал политические права подданным в весьма умеренных размерах. Наилучшим
устройством он считал правление у ветхозаветных евреев, с. 96: «Такое
правление, смешанное из монархии, аристократии и демократии, было именно
установлено у евреев законом Божиим. Моисей и его преемники властвовали как
монархи; старшины избирались по добродетели; наконец, демократическое начало
состояло в том, что они брались из всего народа. Против этого можно возразить,
что наилучшее правление --- царство, ибо оно более всех других представляет собою
управление мира единым Богом. Действительно, говорит Фома, царство --- наилучшее
правление, когда оно не извращается, но вследствие великой власти, которая
вручается царю, оно легко превращается в тиранию.
Только вполне добродетельные люди способны выносить великое счастье, а
совершенная добродетель находится у немногих».

Мы видим у Аквината и развитие идеи общественного договора (восходящей к
Платону), сыгравшей потом такую революционную роль, и понятие народа как
совокупности граждан, связанных этим договором, и резкий протест против всякой
тирании. Как видно, схоластики занимались не только вопросом о том, сколько
демонов может поместиться на конце иглы, как это нам старались внушить, в связи
с чем само слово «схоластик», как и «талмудист», приняло исключительно
оскорбительное значение.

Перейдем к следующему великому доминиканцу --- Джироламо Савонароле (1452--1498),
сожженному при папе Александре VI Борджиа. Этот пламенный проповедник силой
одного своего слова добился власти в цветущей, но достаточно распущенной
Флоренции и установил там «святую жизнь», но он был слишком требователен к
своей пастве, и врагам его (прежде всего самому папе, который был бессилен
помешать ему захватить власть) удалось его погубить. Научных заслуг как будто
Савонарола не имеет, и его гибель целиком связана с его общественной
деятельностью.

Как было указано выше, та же общественная деятельность погубила и знаменитого
Джордано Бруно (1548--1600), но так как Бруно был не только богослов и
общественный деятель, но и философ, то и создавалась легенда, что причиной
гибели были его философские воззрения.

Не менее известен и Томазио Кампанелла (1568--1639), один из основоположников
утопического социализма, сторонник теории Коперника, борец за освобождение
южной Италии от испанского господства. Эта общественная деятельность (в данном
случае поддержанная папой Урбаном VIII) привела его к длительному тюремному
заключению, от которого он избавился благодаря помощи папы.

Разобранные пять великих доминиканцев оставили крупный след в развитии мысли и
общественной деятельности, и в тех случаях, где они занимались преимущественно
мыслительной деятельностью, как Альберт и Фома, они умерли в почете и никаким
преследованиям со стороны властей не подвергались, несмотря на то что их
деятельность вызывала подозрение суеверной толпы или они защищали весьма
прогрессивные политические взгляды.

Можно упомянуть и великих францисканцев: Дунса Скота (1265--1308) и Роджера
Бэкона (ок. 1214--1292). Бэкон --- один из величайших алхимиков (Философский
словарь, 1963, с. 16). Ему приписывали изобретение пороха, зеркала, телескопа
(Гегель, т. II, с. 152), подозревали, конечно, в сношениях с дьяволом. Он же
основоположник индуктивной логики, обличал феодальные нравы, идеологию и
политику, за что много лет сидел в тюрьме.

Дуне Скот --- преподавал в Оксфордском и Парижском университетах и, по словам
Маркса, «заставил самое теологию проповедовать материализм» (Философский
словарь, 1963, с. 140). Выступил с резкой критикой томизма (системы Фомы
Аквината), один из виднейших представителей номинализма. Пользовался
колоссальным успехом как лектор: иногда имел до 30 тысяч слушателей (Гегель, т.
II, с. 136).
Был принят в Париже с большой торжественностью, но вскоре умер от удара.

Перечисленные семь лиц двух орденов, основанных в начале XIII в., уже
показывают, какой огромный диапазон деятельности показывали эти два ордена. И
эта деятельность проходила не только в глуши монашеских келий и совсем не была
характерна тем «единодушием», догматизмом, который сейчас склонны приписывать
самому расцвету Средневековья. Руководители инквизиции, доминиканцы (и в
меньшей степени францисканцы) вовсе не были чем-то подобным «прогрессивному
войску» опричников, созданному только для выметания и выгрызания крамолы, и
деятельность их была вызвана какими-то более общими причинами. Прекрасно пишет
об этом Гегель (том 8, с. 371). Он указывает на упадок церкви, развращение
общества, накопление роскоши господствующих классов. Реакцией и явилось
возникновение нищенствующих монашеских орденов, главным образом францисканцев и
доминиканцев. «Нищенствующие монахи в невероятных количествах распространялись
по всему христианскому миру; они являлись, с одной стороны, постоянной
апостольской армией папы; с другой стороны, они же резко восставали против его
светскости. Францисканцы оказывали сильную поддержку Людовику Баварскому в его
борьбе против папских притязаний; им приписывалось и утверждение, что
вселенский церковный собор стоит выше папы; но впоследствии и они погрязли в
апатии и невежестве».

Идейная база для возникновения нищенствующих орденов была таким образом весьма
широка и независима в значительной степени как от духовной, так и от светской
власти, отчего многие доминиканцы и францисканцы и погибли на костре. Можно
различить по крайней мере шесть сторон их деятельности: 1) борьба против
светскости, обмирщения церкви; 2) борьба за чистоту нравов; 3) против
чрезмерной роскоши; 4) против чрезмерных притязаний папства на власть; 5) за
осуществление справедливого общества; 6) против ересей. Так как монахи, в
особенности основатели орденов, были глубоко верующими католиками, то они,
естественно, в пороках современного общества видели проявление козней дьявола,
выразившихся прежде всего в возникновении ересей. Само собой разумеется, что
такая широта подхода была свойственна только крупнейшим представителям орденов,
многие же схватывали лишь одну или несколько сторон, и вероятно, что такой
выдающийся представитель доминиканцев, как Торквемада, был почти исключительно
«пес Господа». Все указанные шесть сторон идеологии нищенствующих орденов могут
считаться в большей или меньшей степени положительными мотивами. Но, конечно,
всегда и прежде всего в альбигойских войнах примешивались, а у многих даже
доминировали низменные мотивы: корысть, честолюбие, мстительность. Рыцари
северной Франции в крестовом походе на Лангедок, конечно, стремились не столько
к спасению своей души в борьбе за богоугодное дело, сколько к богатой добыче
цветущего Лангедока. Но это --- общее свойство всех крупных идейных движений: с
самого начала к сравнительно небольшому числу идейных руководителей движения
примешивается значительное число усваивающих только часть смысла
движения, а затем при успехе движения примешивается все больше и больше простой
сволочи. Прекрасно сказано у С. Есенина:

Розу белую с черной жабой

Я хотел на земле повенчать.

Белых роз становится все меньше и меньше, их съедают черные жабы, и
поверхностному историку может показаться, что и с самого начала ничего, кроме
черных жаб, не было.

Мы знаем, что альбигойские войны были не единственным ужасным событием. Италия
в XIII и XIV вв. раздиралась междуусобной войной гвельфов (сторонников папы) и
гибеллинов (сторонников императора Священно-Римской Империи германской нации),
идущей с ужасным ожесточением. Великий итальянский поэт Данте Алигьери
(1265--1313) был в философском отношении последователем Фомы Аквината и,
безусловно, верующим католиком. Сначала он был гвельфом, потом сделался
гибеллином и, несмотря на свой католицизм, враждовал с партией сторонников
папы. Он влагал в эту борьбу весь свой темперамент и поместил своих
политических врагов в разные отделения своего Ада. Он не только обрекает их на
вечные муки, он ставит их вне приложения человеческих добродетелей, к ним не
относится жалость, верность слову и проч. Он сначала обещает одному из
грешников немного облегчить его страдания, а потом насмехается над ним, что тот
поверил ему, и такое вероломство ставит себе в добродетель. «Потому что нет
большего негодяя, чем тот, кто жалеет осужденного Богом». При этом он забывает,
что в начале своего путешествия в Ад он сам был подобным «негодяем», так как от
жалости к Франческо-да-Римини (которая тоже была осуждена Богом) упал на землю.
Такого накала достигли политические страсти даже там, где обе стороны были
верующими католиками и никакое религиозное различие их не разделяло. Как
указывает Тарле и другие историки, страшные жестокости, которые в умах
большинства связаны с названием Средневековья, возникли в конце Средневековья;
и в Италии, в частности, такая ожесточенность была связана с борьбой гвельфов и
гибеллинов. Эта последняя борьба отвлекает нас в сторону, указывает на примесь
чисто политических различий, о чем подробнее будет речь дальше.

Постараемся теперь выяснить вопрос об удельном весе еретиков в общем количестве
жертв инквизиции и других религиозных организаций. Беру цифры жертв инквизиции
по Боклю (История цивилизации в Англии, том I, с. 174). В 1546 г. венецианский
посланник при дворе императора Карла V указывает в официальном отчете своему
правительству, что в Голландии и Фрисландии более 30 тысяч человек было убито
по суду за анабаптистскую ересь. В Испании инквизиция за 18 лет правления
Торквемады наказала, по самым низким оценкам, до 105 тысяч человек, из коих
8800 было сожжено. В одной Андалузии за один год инквизиция убила 2000 евреев,
не считая лиц, понесших менее суровые наказания.

Эти кошмарные цифры могут быть, казалось бы, использованы как указание на то,
что с появлением инквизиции и возникли кошмары. Это неверно. Страшные еврейские
погромы были и до появления инквизиции, как вообще многие ужасы ей
предшествовали. Вот как рисует начало крестовых походов (конец XI в., а
инквизиция появилась в XIII в.) Гегель (Философия истории, соч., том VIII, с.
368): «Крестовые походы начались тотчас же непосредственно на самом Западе;
многие тысячи евреев были убиты и ограблены, --- и после этого ужасного начала
христианское войско двинулось: монах Петр Пустынник из Амьена шел впереди
огромной толпы сброда. Толпа прошла в величайшем беспорядке через Венгрию,
повсюду разбойничая и грабя, но сама толпа затем очень поредела, и лишь
немногие добрались до Константинополя. Ведь не могло быть речи о разумных
основаниях; толпа верила, что Бог непосредственно будет вести и охранять ее. О
том, что энтузиазм довел народы почти до безумия, лучше всего свидетельствует
то обстоятельство, что впоследствии толпы детей бежали от своих родителей и
отправились в Марсель, чтобы оттуда на кораблях поехать в святую землю.
Немногие прибыли туда, а остальные были проданы купцами в рабство сарацинам.
Наконец с большим трудом и с огромными потерями более дисциплинированные войска
достигли своей цели: они овладели всеми знаменитыми святыми местами: Вифлеемом,
Гефсиманией, Голгофой и даже Гробом Господним. Во всем движении, во всех
поступках христиан обнаруживался этот вообще проявлявшийся чудовищный контраст,
выражавшийся в том, что от крайней разнузданности и насилий христианское войско
вновь переходило к величайшему сокрушению и смирению. Еще обагренные кровью
убитого населения Иерусалима, христиане пали ниц у Гроба Спасителя и обратились
к нему с горячей мольбой».

С какой великолепной объективностью Гегель изображает контрасты такого великого
исторического события, какими были крестовые походы, и как мало историков,
которые смогли удержаться на такой высоте. Одни выпячивают энтузиазм и высокие
побуждения руководителей походов и стараются замалчивать ужасы их. Другие
правильно изображают ужасы, а отсюда заключают, что весь энтузиазм и все
благородные побуждения фальшивы и что с самого начала в крестовых походах
ничего, кроме грабежа, и не было. Но великая антиномия истории и заключается в
том, что во всех великих движениях, вплоть до наших дней, белые розы и черные
жабы неизменно перепутаны.

Преследования еретиков и иноверцев вообще, изложенные выше, связаны, как было
отмечено, не только с инквизицией, но вообще с католической церковью. Значит ли
это, что и другие церкви, в частности протестантская, главный оппонент
католичества, были от этого свободны? Нет, если обратимся к тому же Боклю, то
увидим, что и протестанты действовали в том же духе. Показания Бокля ценны
особенно потому, что он сам протестант и считает католичество низшей формой
религии, отягощенной суевериями.

Кровавая реакция против католиков в Англии возникла при воцарении Елизаветы,
сменившей на троне фанатическую католичку Марию Тюдор (Кровавую). Но, как
указывает Бокль (т. 2, с. 48), хотя много лиц было казнено несомненно по
религиозным мотивам, никто не указывал религию как причину казни, а всегда
приводили политические обвинения. После Елизаветы два человека были публично
сожжены в 1611 г. англиканскими епископами за еретические взгляды (с. 54).
Сделавшись господствующей, англиканская церковь вела суровую борьбу с
диссентерами. Количество жертв Бокль приводит на с. 121 т. 2. За период между
1660 и 1688 г. составлен список не менее 60 тысяч пострадавших, из коих не
менее пяти тысяч умерли в тюрьме. Как указывает Бокль (т. 2, с. 105),
единственный раз англиканская церковь вступила в борьбу с королем, и это тогда,
когда король издал указ о свободе совести. Конечно, Иаков II руководился при
этом не симпатией к диссентерам, а симпатией к католикам, что в значительной
степени и послужило причиной его свержения с престола.

Нетерпимость кальвинизма хорошо известна. Кальвин сжег на медленном огне
Сервета из-за расхождения о догмате Троицы. Кстати, Сервет является одним из
бесспорных случаев того, что погиб крупный ученый (ему приписывают открытие
кровообращения) из-за религиозных догматических разногласий.

Бокль считает (т. 1, с. 244--245), что католическая церковь более суеверна и
более нетерпима, чем протестантская, но приводит и исключения. Французы, по его
мнению, более свободны от суеверия и нетерпимости, чем большинство
протестантов, и, в частности, у шведов нетерпимость такова, что ее можно
считать вдвойне позорной для народа, который, по его утверждению, основывает
свою религию на праве личных убеждений. В XVII столетии шведская церковь издала
указ, утвержденный правительством, что если шведский подданный сменит религию,
то он изгоняется из королевства и теряет всякое право на наследство, как для
себя, так и для своих потомков. До 1781 г. римские католики не имели права
осуществлять религиозные службы в Швеции.

Указание Бокля на большую терпимость французов кажется странным, если вспомнить
кровавые события во Франции, связанные с преследованием гугенотов
(Варфоломеевская ночь, отмена Нантского эдикта, драгонады) и приведшие в конце
концов к почти полному исчезновению гугенотов во Франции (большинство
эмигрировало). Трагедия гугенотов естественно вызывает к ним симпатию; это тем
более понятно, что последний акт этой трагедии во Франции (отмена Нантского
эдикта, драгонады) отвратителен по выполнению, совершенно не оправдан никакими
политическими соображениями и связан с омерзительной смесью ханжества и
деспотизма стареющего изверга Людовика XIV. Но тот же Бокль показывает общий
характер борьбы Ришелье (нанесшего главный удар гугенотам), и из этого видно,
что в разгар борьбы гугеноты были гораздо более нетерпимыми, чем католики, и
что полностью сокрушив политическое значение гугенотов, Ришелье сохранил полную
свободу вероисповедания. Как особенно любопытный факт отметим, что в числе
военачальников Людовика XIII девять были протестантами, в том числе и
прославившийся уже при Людовике XIV знаменитый маршал Тюренн, которого военные
специалисты считают одним из величайших полководцев всех времен и народов
(Бокль, т. 2, с. 239). На с. 246 Бокль дает превосходное резюме: «Это один из
случаев, которые показывают, как поверхностен взгляд тех писателей, которые
полагают, что протестантская религия обязательно более либеральна, чем
католическая. Если бы те, кто принимают этот взгляд, постарались исследовать
историю Европы по первоисточникам, они бы
нашли, что либерализм каждой секты зависит не от признаваемых ею догматов, а от
обстоятельств, в которые она поставлена, и от объема авторитета духовенства.
Протестантская религия большей частью более терпима, чем католическая, просто
потому, что события, которые вызвали протестантизм, в то же самое время
увеличили роль интеллекта и тем самым снизили роль духовенства. Но когда мы
читаем труды выдающихся кальвинистских богословов, и особенно когда мы изучаем
их историю, то мы видим, что в шестнадцатом и семнадцатом столетиях желание
преследовать своих противников так же пламенно горело у них, как у католиков в
худшие времена папского господства. ...И даже теперь в низших слоях общества
шотландских протестантов больше суеверия, больше фанатизма и меньше милосердия
истинной религии, чем среди низших слоев французских католиков».

Это сравнение Бокля оказалось поистине пророческим. Где сейчас максимальный
накал расизма? В Южной Африке и притом как раз среди низших слоев белого
населения (интеллигенция и духовенство в большинстве антирасистское), где
господствует кальвинизм (голландские кальвинисты с известной примесью потомков
французских гугенотов: Кронье, Жубер, Деллярей и др.). А в католической Южной
Америке о ярком расизме не слышно: мирное сосуществование белой, черной и
красной рас, а в КуКлуксКлан католиков вовсе не принимают.

Если мы сравним теперь Русскую Православную Церковь с церквями Запада, то
увидим как будто более благополучное состояние. У нас не было организацией,
подобных инквизиции. Сожжение еретиков, конечно, было (секта жидовствующих,
протопоп Аввакум и другие), но по масштабам сильно уступало Западу. Судьба
евреев была не лучше, чем на Западе: при взятии Полоцка Иоанном Грозным они
были утоплены в реке (кажется, те, которые не захотели креститься). Страшные
эксцессы на Украине в отношении евреев и католиков общеизвестны (Уманьская
резня и проч.), но последние акты были действиями повстанцев, а не
правительства. Но в России было гораздо более распространено другое явление:
самосожжение сектантов. Видимо, количество самосожженцев достигало тысяч 17 или
более, т. е. по масштабам шло в сравнение с количеством сожженных в Испании.
Какой-то страшный рок требовал большого количества жертв, и если они не
приносились на алтарь нового Молоха правительством, жертвы шли на алтарь
добровольно. Самосожжение имело место у нас главным образом в XVII и XVIII вв.
(в связи с преследованием раскольников), но отдельные случаи были и в конце XIX
в. в связи с переписью. Сейчас на Руси, кажется, никто себя не сжигает, но
самосожженцы появились в XX в. вновь: буддисты во Вьетнаме, один квакер и еще
из другой секты даже в США! Поразительно, но факт: одна из самых ужасных
смертей принимается добровольно героическими людьми!

Инквизиции будто не было в магометанских государствах, несмотря на
обилие всевозможных сект и на религиозные войны. Блестящие цивилизации
ислама сами собой пришли в упадок без всякого содействия инквизиции.
Связать наличие инквизиции с упадком культуры невозможно. Инквизиция и
ее методы были свойственны не одной Испании. Почему же испанская
особенно известна? Потому что именно в Испании смертные приговоры
обычно проводились в торжественной
обстановке при большом стечении сочувствовавшего инквизиции народа. «Толпа
смотрела на сожжение еретиков с такой жадностью, перед которой меркло
наслаждение боем быков, и если грешники в слишком большом числе каялись после
приговора и отделывались удушением, вместо того чтобы сгореть на костре,
зрители начинали роптать» (Фейхтвангер, Гойя, 1955, с. 144). Не думаю, чтобы
смотревшие на сожжение еретиков испанцы были особенно плохие люди. Но они не
имели сомнения в том, что сжигаемые --- слуги дьявола, следовательно, подлинные
враги христиан, и, как сказал Данте, «нет большего негодяя, чем тот, кто жалеет
осужденного Богом», и они имели неограниченное доверие к руководителям
инквизиции. В своей личной жизни такие руководители инквизиции, как святой
Доминик и Торквемада, были, по-видимому, безупречны и в своей страшной
деятельности руководились желаниями спасти человечество и даже отдельных лиц от
адских мучений (см. превосходную трагедию В. Гюго «Торквемада»).

Но все-таки эти кошмары связаны с религией, значит, религия и виновата! Это
возражение имело силу во второй половине XIX в., сейчас оно ее потеряло, потому
что в XX в. кошмары, не уступающие инквизиции, выросли не на религиозной, а на
антирелигиозной, квазинаучной почве. В нашей стране еретиков называют
ревизионистами и так же беспощадны к ним, как и к еретикам. Процесс
реабилитации жертв сталинской инквизиции захватил огромное количество лиц, но
он не коснулся тех, кто позволил себе отклониться от генеральной линии партии.
О количестве точно мы не можем судить, но оно, конечно, не уступает количеству
уничтоженных еретиков, и в качестве лидеров борьбы с ревизионизмом и с
иноверующими вообще можно указать таких лиц, которые вполне подобны Доминику и
Торквемаде по безупречной личной жизни и по полному отсутствию сомнений в
справедливости избранного ими пути и средств. Такими у нас можно назвать
Дзержинского и Свердлова, санкционировавших такие страшные акты, как, например,
расстрел нескольких сот ни в чем не повинных людей после покушения на Ленина.
«Как невинных? --- они люди того же класса (в этот широко понимаемый класс
входили такие люди, как еврейка Фанни Каплан и протоиерей Казанского собора
Философ Орнатский), они несут коллективную ответственность». И надо сказать,
что испанский и советский результаты были во многом сходны: «Одна паства, один
пастырь, одна вера, один властитель, один меч» --- как пел в Испании поэт Эонандо
Акунья (Фейхтвангер, Гойя, с. 141). У нас тоже было почти осуществлено
единомыслие по Козьме Пруткову. Испания избежала гражданских войн, на
Пиренейском полуострове было полное единомыслие, у нас тоже результаты
голосования в пользу единственных правительственных кандидатов превышают 99\%.
Правда, сейчас в международном масштабе прежнего единства уже не наблюдается,
хотя достижение единства по-прежнему провозглашается главной целью политики.

Это стремление к единству любой ценой проявлялось, как известно, не только у
нас. Хорошо известно достигнутое при Гитлере единство в Германии. А несколько
раньше европейски образованный Энвер-паша в Турции сильно усовершенствовал
старые мусульманские методы борьбы с гяурами. Если раньше, в предыдущие века,
время от времени
происходили кровавые погромы армян, болгар, греков и других христиан,
вызывавшие всеобщее возмущение, то это не мешало армянам и другим размножаться,
богатеть и играть все большую роль в экономической и культурной жизни Турции. С
приходом к власти младотурок армяне получили равноправие, но в период Первой
мировой войны была предпринята первая попытка геноцида (здесь бесспорный
приоритет Энвера и Талаата перед Гитлером), к сожалению, со значительным
успехом в отношении армян. В этом деле играл роль исключительный турецкий
этатизм, а не ислам.

Мы видим, что и на антирелигиозной, квазинаучной почве может возникнуть такая
же нетерпимость, такое же страшное преследование инакомыслящих, как и на самой
фанатической религиозной почве. Что же общего: абсолютная уверенность в
собственной непогрешимости, отсутствие сомнений. Прекрасно сказано у Бокля (т.
2, с. 61): «Необходимо научиться сомневаться, прежде чем сделаться терпимым;
люди должны признать погрешимость собственных мнений, прежде чем они начнут
уважать мнения своих противников». И история показывает, что уверенность в
собственной непогрешимости дает особенно страшные результаты тогда, когда она
совмещается с высокими личными моральными качествами.

б) Ведьмы и колдуньи

Второй и, пожалуй, самой важной категорией жертв церковных преследований были
ведьмы. В русском языке слова «ведьма» и «колдунья» можно считать синонимами, и
на других языках они часто обозначаются одним словом, но нет слова мужского
рода того же корня, что ведьма. Может быть, это не случайно, так как хотя
колдуны тоже преследовались и подвергались сожжению, но масштаб преследования
колдунов не идет ни в какое сравнение с преследованием колдуний или ведьм.
Колоссальный размах преследований ведьм обычно тоже относят к Средневековью, но
почти четырехсотлетняя вспышка преследований ведьм относится уже к концу
Средневековья и захватывает и новейшую историю. Канторович (Короленко, статья:
Я. Канторович. Средневековые процессы о ведьмах. Соч. в 10 т. 1955, т. 8, с.
305 и далее) указывает, что «с конца XIV в. до второй половины XVIII в. в
течение почти четырех столетий во всех странах Европы не переставали полыхать
костры, раздуваемые невежеством, фанатизмом и суеверием, и сотни тысяч невинных
людей, после страшных мучений пытки, обрекались на смерть по обвинению в связи
с дьяволом и в разных чудовищных преступлениях колдовства». Как единичные
случаи, процессы ведьм сохранились до конца XVIII в., и в Швейцарии одна ведьма
была публично сожжена в 1780 г. (Гегель. Философия истории, с. 397), а в
католической Мексике пять женщин были обвинены в колдовстве по всем правилам
судопроизводства и 20 августа 1877 г. сожжены на костре (Короленко, с. 305).

Руководство по преследованию ведьм под названием «Молот ведьм» было составлено
в 1489 г. инквизиторами Шпренгером и Кремером с благословения папы и одобрения
кельнского теологического факультета. Книга начинается текстом папской буллы,
объявлявшей еретиком
всякого сомневающегося в наличии колдовства и существовании ведьм. В течение
только XVI и XVIII столетий в одной Германии было сожжено свыше ста тысяч
ведьм, а по всей Европе за период XIV--XVIII вв. насчитывается свыше миллиона
жертв! Некоторые насчитывают их до четырех, даже до девяти миллионов
(Короленко, 306). В некоторых деревнях Трирского епископства все местные
женщины подпали обвинению в колдовстве, так что в одной деревне осталась в
живых всего одна женщина.

Как видим, если в процессах еретиков (помимо погибших в религиозных войнах) шла
речь о десятках тысяч жертв, то в деле о ведьмах речь идет о сотнях тысяч, а
может быть, и о миллионах.

Вспышка преследований ведьм возникла тогда, когда вся Западная Европа была
католической, и потому преследования первоначально велись только католиками, но
преследования не прекратились и в протестантских странах. Чем это объясняется?
Да тем, что вплоть до конца XVII в. убеждение в существовании ведьм было
всеобщим. Оно имеет корни в древнейших представлениях о людях, обладающих
особой магической силой. Бокль даже указывает достаточно точно тот момент,
когда в Англии мнение большинства переменилось (т. 2, с. 71): это произошло
между Реставрацией Стюартов и Революцией (Вильгельма III), т. е. в 1660 г.
большинство образованных людей верило в ведьм, а в 1688 г. большинство
перестало верить. Последние ведьмы были повешены в Англии в 1682 г., а
возможно, даже в 1705 и 1712 гг.

Передовые люди XVI и XVII столетия верили в существование ведьм. Например, Жан
Болен (1530--1596), депутат третьего сословия, выступавший в защиту
веротерпимости и нападавший на религиозный фанатизм (он был сторонником
французского Генриха IV, защищавшего религиозный индифферентизм) и обвиненный
даже в атеизме, защищал, однако, мысли о реальности ведьм и необходимости их
преследования (История философии, т. 2, 1941, с. 75--76).

В России, как правильно пишет Короленко (с. 308), преследование ведьм не
развилось в такую повальную болезнь изуверства и безумия, как в Западной
Европе. Но в допетровский период погибло и у нас немало людей, благодаря
выкликаниям кликуш; в указах царя Федора Алексеевича учителям магии и
чернокнижия полагается сожжение, в артикулах воинского устава Петра Великого
говорится об идолопоклоннике, чернокнижце, ружьязаговорителе --- что таковой
«весьма сожжен быть имеет» (там же). Таким образом, и в России идеология в
отношении колдовства даже весьма просвещенного Петра не отличается от идеологии
«Молота ведьм», но исторические условия (о чем постараемся сказать дальше) не
позволили в России развиться этой идеологии.

Нельзя сказать, чтобы не было протестов против процессов ведьм, но первые века
это были протесты против доказанности всех процессов и призывы к милосердию. Но
эти первые протесты (Корнелий Агриппа Нетесгеймский в 1531 г., Иоганн Вейер из
Мозеля в 1523 г., Адам Таннер из Инсбрука в 1626 г.) не отрицали реальности
колдовских явлений (Короленко, с. 307).

По-видимому, первым, кто выступил с отрицанием вообще процессов ведьм, был
иезуит Фридрих фон Спе (или Шпе) (1591--1635), профессор философии в Кельне (он,
кстати, был также и поэтом; см. Гегель.
Философия истории, с. 397 и 463, Короленко, с. 307). «Будучи назначен в 1627 г.
исповедником приговоренных к сожжению так называемых "ведьм" в Юрсбурге, где в
1627--1628 гг. было сожжено 158 человек в том числе две 9-летние девочки, Спе
убедился, что никто из сожженных ни в чем не виновен. После этого он выступил
против возбуждения процесса по таким нелепым поводам в специальном сочинении.
Хотя инквизиторы добились удаления Спе из Юрсбурга, но его книга имела успех, и
Филипп фон Шенборг, курфюрст Майнцский запретил всякие розыски ведьм» (Гегель,
с. 463). «К вам, судьи, обращаюсь я и спрашиваю, --- писал в 1631 г. благородный
Фридрих фон Шпе, поседевший преждевременно от самого вида страданий невинно
казненных, --- зачем вы так тщательно ищете повсюду ведьм и колдунов? Я вам
укажу, где они находятся. Возьмите первого капуцинского монаха, первого
иезуита, первого священника, подвергните его пытке, и он признается, он
непременно признается... Возьмите прелатов, кардиналов, возьмите самого папу!
Они признаются, уверяю вас, они признаются!...» (Короленко, с. 308). Следом за
Спе выступил Томазий, профессор в Галле (1655--1728, Гегель, с. 397, 465),
усилия этих двух деятелей и их последователей и привели в конце концов к
ликвидации процессов ведьм.

Но естественно задать вопрос, почему сомнение в существовании ведьм возникло с
таким запозданием? От «Молота ведьм» (1489) до книги Спе (1631) прошло 142 г.,
а после учреждения ордена иезуитов около ста лет. Неужели раньше не могли в
этом разобраться? Для этого надо немного разобраться в понятии ведьм и в
отличии ведьм от колдунов и колдуний: хотя слова «ведьма» и «колдунья», как уже
было указано, считаются практически синонимами, но это не совсем так.

Уходящие в глубокую древность понятия колдунов, колдуний, кудесников, ведьм (от
слова «ведать»? «вещие женки»), знахарей, шаманов и прочих обозначали прежде
всего то, что можно назвать первобытной интеллигенцией, т. е. прежде всего
примитивных представителей медицины, технологии и проч. Мы знаем, что и в
России, например, мельники (особенно водяные) подозревались в колдовстве, так
как без помощи сверхъестественных сил считалось невозможным соорудить такое
сложное строение, как водяную мельницу. Мы знаем, что знахари собрали немало
полезных сведений, используемых и сейчас в медицине. Полезность знахарей, ведьм
в первоначальном смысле слова и приводила к тому, что хотя они время от времени
и подвергались преследованиям вплоть до сожжения (погром Пифагорейского союза в
Древней Греции имеет ту же причину), но на полное искоренение колдунов и
колдуний не шли, так как в них реально нуждались. Подозрения же в колдовстве и
в сношениях с нечистой силой распространялись практически на всех выдающихся
людей того времени, как это уже упоминалось: Альберт Великий, папа Сильвестр
II, Роджер Бэкон, Фома Аквинат и прочие подвергались подозрениям невежественной
толпы, но многие из них занимали видное положение и никак не преследовались.

За период Средневековья был достигнут весьма значительный прогресс в развитии
самой разнообразной техники. Это хорошо показано у Бернала (Наука в истории
общества, 1956, с. 184--193 и 196--197).
Очень значительная часть этих открытий, по-видимому, заимствована с Востока,
прежде всего Китая (с. 184): хомут лошади, часы, компас, кормовой руль, порох,
бумага, книгопечатание. Но, как указывает Бернал (с. 185), в Китае, в меньшей
степени в Индии и мусульманских странах, этот технический прогресс в XV в.
совершенно прекратился. По мнению Нидхэма (Бернал, с. 185), этот застой
восточных цивилизаций объясняется ростом литературно образованной бюрократии ---
мандаринов. В Европе, конечно, были соответствующие «мандарины» в лице
перипатетиков, но они не сумели остановить прогресса, и все зерна с Востока
пали на благодатную почву. Как ясно из таблицы 3, приведенной Берналом (с.
196--197), многие изобретения вводились в самые темные годы Средневековья.
Колесный плуг и трехпольная система севооборота введены в VIII в., улучшение
викингами морских судов --- в IX. Тогда же в Европе распространились хомут,
подковы и стремянка, вывезенные из Китая. В X в. широко распространились
водяные мельницы, в XI --- ветряные мельницы в Персии, в XII --- во Франции стали
употреблять линзы, начали перегонять спирт. В XII --- бумага в Испании, цветное
стекло, в XIII --- компас, порох, часы, в XIV --- очки, пушки на войне, кормовой
руль. Алхимики наряду с фантастическими проектами получения благородных
металлов сделали множество интересных открытий, а усовершенствование
производства стекла способствовало успехам примитивной химии, основы будущей
научной химии.

Выдающиеся деятели Средневековья не стояли в стороне от потока изобретений. Как
пишет Бернал (с. 192): «Гроссетест (епископ), Роджер Бэкон и Дитрих из
Фрейсбурга внесли свои вклады в науку, объяснив действие линз как в случае
фокусирования световых лучей, так и в случае увеличения изображений». Но ведь
этот значительный прогресс прямого отношения к процессам ведьм не имеет, так
как участие женщин в развитии науки и техники было ничтожным в те времена, если
вообще было. Мне лично не известно ни одно женское имя, относящееся к науке и
технике периода Средневековья. В прошлом (античные времена) известны имена
математички Гипатии, убитой фанатическими христианами в Александрии, и еврейки
Марии --- почти легендарной фигуры, предполагаемой изобретательницы водяной бани
(Бернал, с. 131). Поэтому можно с достаточной уверенностью сказать, что
процессы ведьм никак не были связаны с преследованием науки. Чем же они
вызывались? Обвинения, предъявляемые к ведьмам, сводились прежде всего к связи
с дьяволом, посещению шабашей верхом на метле и пр., служении дьяволу в форме
«черных месс» и, конечно, использовании разных магических средств для
причинения вреда другим людям. Как сказано у Жуковского (перевод баллады Саути
«Как одна старушка ехала на черном коне»):

Я кровь младенцев проливала,

Власы невест в огне волшебном жгла

И кости мертвых похищала.

Были ли все эти преступления выдуманными? Конечно нет. Среди знахарок были,
конечно, и преступницы, а мы знаем, что и в настоящее
время суеверия имеют самое широкое распространение и не только за рубежом;
несомненно, что были случаи использования в преступных целях средств народной
медицины и гипноза (самая реальность которого долгое время отрицалась) и,
наконец, черные мессы служились и в XIX в.: одна из них в карикатурном виде
изображена в замечательном рассказе М. Горького «Сторож». Разумеется, в
подавляющем большинстве случаев признания ведьм были вынуждены пыткой
(Короленко, с. 307), «изуверские суды считали признание "более очевидным, чем
сама очевидность", и потому на кострах гибли не раз ведьмы, обвиненные в
умерщвлении детей, которые, однако, оказывались живы и невредимы». Но все ли
показания были вынуждены пыткой? У того же Короленко читаем там же: «Около 1484
г. повальная истерия охватила монахинь одного монастыря, --- они мяукали, лаяли,
катались по земле. После напрасных попыток изгнать дьявола --- все монахини были
сожжены». «Зараза достигала таких пределов, что один из наиболее жестоких
судей, Реми или Ремигиус, автор книги "Демонолатрия", сжегший в течение своей
15-летней практики свыше 900 ведьм, --- в конце своей жизни вообразил себя
одержимым сатаной и дал себя сжечь на костре». Сохранились многочисленные
показания ведьм, которые с восторгом признавались в своей связи с сатаной.

Все это показатели массового психоза, охватившего эти времена. Я уже указывал,
что предшествовавшие гонениям на ведьм крестовые походы (конец XI --- середина
XIII в.) тоже во многих отношениях характеризуются наличием массовых психозов.
Женщины, как известно, подвержены в гораздо более сильной степени, чем мужчины,
истерии (кликушество в России и пр.), и потому, естественно, этот массовый
психоз коснулся прежде всего женщин, и он естественно развивался на почве тех
огромных потрясений, которые были характерны для Запада: крестовые походы,
альбигойские войны, гвельфы и гибеллины и проч. О связи процессов ведьм с
истерией говорил, между прочим, Эразм Дарвин, дед знаменитого Чарльза Дарвина,
врач по профессии. Известен один из методов, применявшихся при исследовании
ведьм: испытуемую бросали в воду: если она тонула, значит, она не ведьма, и за
нее служили панихиду, а если не тонула, значит, ее поддерживает дьявол: ее
обрекали на сожжение. Любопытно, что Эразм Дарвин дает этому процессу вполне
научное толкование (Э. Дарвин. Храм природы, 2-е изд., 1960, с. 170): «Он
читает книгу о ведьмах, которые не тонут в воде. Он объясняет это просто. У
истеричных ипохондриков, болезни которых возникают от страха и волнения,
расстраивается пищеварение, желудок и кишечник наполняются воздухом, и больные,
будучи погружены в воду, всплывают на поверхность».

Принимая все это во внимание, можно предположить, что гораздо меньшее развитие
процессов ведьм в России отчасти объясняется меньшим техническим прогрессом и
тем, что у нас не было явления, подобного крестовым походам, приведшим (и
отчасти и вызванным) к массовым психозам. Массовые психозы встречались у нас,
но они распространялись на гонимое меньшинство и привели к таким достаточно
массовым актам, как самосожжение.

Резюмируя, можно сказать, что преследование ведьм в течение четырех столетий
было связано с психопатологическими явлениями,
которые истолковывались как следствие влияния нечистой силы, и до тех пор пока
эти психозы не спали, немыслим был и крупный протест против них. К моменту
выступления Спе уже все ведьмы делали свои признания под пыткой, отсюда можно
было сделать вывод, что процессы ведьм являются вообще недопустимыми. Но так
как толкование психозов как доказательств связи с дьяволом делалось на
религиозной почве, то выходит, что религия виновата в данном случае если не в
преследовании науки, то в преследовании душевнобольных людей. Ясно, что таких
преследований мы не вправе ожидать там, где религия не является основой
юриспруденции. XX в. опроверг это положение. Процессы ведьм в иной форме
появились вновь; правда, преследуют больше колдунов, а не ведьм. Прекрасно
написано у Н. Винера в книге «Я --- математик» (1964, с. 295): «Во всяком случае,
мне с самого начала было ясно, что отныне ученым придется сталкиваться с
двойственным отношением к себе. Ибо современное общество в соответствии с
традицией, возникшей еще в незапамятные времена, с одной стороны, относилось к
нам как к магам и волшебникам, а с другой, считало, что мы вполне достойны быть
принесенными в жертву богам. Вместе со взрывом атомной бомбы возник страшный
призрак охоты за ведьмами, мучавший нас следующие восемь лет, так что все нами
пережитое было лишь воплощением предначертаний, запечатленных в страшном
облаке, поднявшемся над Хиросимой». Конечно, на Западе борьба с ведьмами заняла
не все страны (в Англии ее как будто не было), продолжалась сравнительно
недолго, и даже в США она не носила смертельного для жертв охоты за ведьмами
характера. Наибольший шабаш ведьм имел место, конечно, в совершенно
атеистической стране, сталинской России. Конечно, за связь с дьяволом не
преследовали. Но как в биологии место всемогущего Бога занял всемогущий (но не
милосердный) естественный отбор, так аналогом дьявола у нас был классовый враг,
буржуй, который и имел типичные для него черты (цилиндр, сигара и проч.), и
обвинения в связи с этим новым дьяволом часто носили столь же нелепый характер,
как и в старые времена преследования ведьм. То, что называлось в старину
колдунами, у нас приобрело термин вредителя, и количество невинно погубленных
«вредителей», конечно, колоссально.

Другую форму, также не связанную с религией, дьявол принял у нацистов ---
представитель чужой, низшей расы (что многими лицами, далеко не чуждыми науки,
связывалось и с теорией Дарвина, и с учением о естественном отборе).

Несомненно было и наличие психозов вовсе не на религиозной почве. Всякое
крупное массовое общественное движение так или иначе связано с нарушением
нормальной психики. В революциях имеет место то, что называется оргористией, т.
е. болезненной уверенностью в осуществлении желаемого. Иногда она приводит к
поразительным результатам, но, как правило, ведет к колоссальному количеству
совершенно ненужных жертв.

Любопытно, что и методы судопроизводства в XX в. нередко напоминают таковые в
отдаленные времена преследования ведьм. Во времена процессов ведьм признание
обвиняемого, полученное под пыткой, считалось достаточным доказательством его
виновности (см. выше,
с. 139). Спе справедливо указывал, что признание, вынужденное пыткой, не имеет
никакого значения, и с тех пор этот принцип, что признание обвиняемого не
является доказательством виновности, прочно вошел в юридическое сознание всех
культурных стран. Мы отвергли существовавшую юриспруденцию как «классовую»,
«буржуазную», и что же мы дали взамен этой устаревшей дисциплины? Идеологом
судебных процессов сталинского периода стал Вышинский, сам юрист по
образованию. Он полностью реабилитировал практику времен процессов ведьм (ведь
он был и академик АН СССР): как бы ни было получено признание, его достаточно
для того, чтобы счесть доказанной вину обвиняемого. Сейчас начинают
восстанавливаться нормы старого, дореволюционного правосудия, но далеко не вся
сталинщина еще выкорчевана.

в) Политические противники

В первых двух разделах обвинение лиц, преследовавшихся инквизицией,
соответствовало характеру преступлений, истинных или мнимых. Вина обвиняемых
состояла или в исповедовании верований, считавшихся ложными, или в совершении
актов, свидетельствующих об отступничестве от Бога и служении дьяволу. Но во
многих случаях такие или подобные обвинения были только предлогом для борьбы со
своими противниками.

Так как часто формулировка обвинений не соответствовала истинной причине
преследования, то возникло мнение, что всегда религиозные преследования были
лишь ширмой, а истинная основа всегда была политическая, классовая и т. д. Это
мнение, конечно, вполне допустимо в тех случаях, когда мы имеем смешанный
характер причин преследования, но, несомненно, крайние примеры показывают или
вполне идеологические основы борьбы, или, напротив, прикрытие религиозными
основаниями чисто политической борьбы.

Несомненно, когда Николай чудотворец «заушил» (т. е. нанес пощечину) Арию на
Вселенском соборе, он действовал только в силу своего возмущения тем
принижением значения Иисуса Христа, которое было в учении Ария. Кальвин сжег
Сервета, руководствуясь исключительно догматическими расхождениями о Св.
Троице.

Но можно ли сказать, например, что восстания запорожских казаков против
польского правительства, с такой художественной силой изображенные Гоголем в
«Тарасе Бульбе», были войной за веру, как думал сам Гоголь. «Известно, какова в
русской земле война, поднятая за веру: нет силы, сильнее веры» (Тарас Бульба,
гл. XII).

Конечно, основа борьбы была чисто политическая --- стремление украинцев сохранить
свою национальность, и мы знаем, что это основание вполне достаточно для
возникновения чрезвычайного энтузиазма. Наша современная история это прекрасно
доказывает. Но какие же религиозные различия отличали католиков --- поляков и
православных --- казаков? Вряд ли казаки разбирались в таких догматических
тонкостях, как исхождение Св. Духа от Отца или от Отца и Сына. Критерии
религиозных различий у них были иные. Это показано у
того же Гоголя в его поистине страшной «Страшной мести». Доблестный казак
Данило говорит с женой о своем тесте (глава IV): «живет около месяца и хоть бы
раз развеселился, как добрый казак! Не захотел выпить меду!.. Горилки даже не
пьет, экая пропасть! Мне кажется, пани Катерина, что он и в господа Христа не
верует... поганые католики дюже падки до водки; одни только турки не пьют».
Тесть не любил галушек. «Знаю, что тебе лучше жидовская лапша», --- подумал про
себя Данило. Тесть не любил свинины. «Для чего же не любишь свинины? --- сказал
Данило, --- одни турки и жиды не едят свинины». По продуманным религиозным
критериям доблестного казака, его тесть одновременно был и магометанином и
евреем, так как евреи, как известно, от алкогольных напитков не отказываются.

В пандан к рассуждениям пана Данилы можно сказать, что его критерии
принадлежности к религии сохранились даже до наших дней. Несколько лет тому
назад в «Известиях» появилось сообщение, что, помнится, в Кривом Роге одного
безупречного работника стали преследовать за то, что он --- скрытый сектант, и
доводами в пользу такого «обвинения» (а почему нельзя быть сектантом?) было то,
что он не пьет, не курит и не сквернословит --- безусловные признаки сектанта.
Его довели до того, что он ушел с работы и долго не мог устроиться, так как
критерии сектантства у современных доблестных безбожников были достаточно
общеприняты. Атеистическая религия так же нетерпима, как и православная, и
притом наши атеисты забывают, что для революционеров периода подпольной борьбы
были как раз более характерны эти самые признаки, которые сейчас свойственны
преимущественно сектантам. Многое перешло у нас в свою противоположность даже в
отношении идеологии, и такой «ревизионизм» нисколько не преследуется.

Возвращаясь к инквизиции, в частности испанской, можно сказать, что там
огромную роль играл политический момент, в особенности борьба с маврами. Борьба
могла вестись и на чисто национальной почве, но по свойствам того времени ей
был придан религиозный характер, отчего получились и своеобразные следствия.
Когда большая часть Испании была освобождена, то на территории Испании
оказалось значительное число мавров (и евреев), из которых многие искренне или
лицемерно захотели ассимилироваться с испанцами. В более позднее время
(например, при опрусачении и обрусении поляков в имперской Германии или царской
России) для того, чтобы получить все права, достаточно было изменить язык и
сохранить веру, хотя, конечно, в старой России широко было распространено
отождествление всех католиков с поляками. В старые времена для ассимиляции
требовалось принятие веры, и, несомненно, во многих случаях такая смена веры
была лицемерной, и новые адепты продолжали тайно исполнять обряды старой
религии. Дело кончилось, как известно, массовым выселением (тут уже гуманнее:
не сожжением!) мавров и евреев из Испании. Пришли к заключению, что смена веры,
как правило, является лицемерной.

Полная аналогия этому процессу имела место и в сталинские времена. Преследовали
националистов и лиц, подозреваемых в деятельности в пользу соседних государств.
Были полностью выселены (в период ежовщины) корейцы из Амурской области (в
Среднюю Азию), а
во время последней войны депортация приняла огромные размеры: немцы
Поволжья, крымские татары, многие народы Северного Кавказа: калмыки,
карачаевцы, ингуши, чеченцы и проч. Депортация во всех случаях была
тотальной, т. е. высылали независимо от классовой или партийной
принадлежности. Депортация была не только, так сказать, физическая, но
и идейная: не только представители национальности были высланы далеко
за пределы своей родины, но само упоминание о национальностях было
вычеркнуто из официальных справочников. Калмыцкая автономная область
исчезла с географических карт, столица ее Элиста была переименована
в город Степной (см., например, Атлас СССР 1954 г., 41--42),
переименовано было большое число названий крымских населенных пунктов
и, например, в статьях о Кавказе в БСЭ, 2-е изд., изгнанные нации
(ингуши, чеченцы, карачаевцы и др.) даже не упоминались, как будто они
вовсе никогда не существовали. Сейчас уже кое-что восстановилось
и многие жители вернулись на свои прежние места, возродились и
исчезнувшие географические понятия. С маврами испанцы поступали
все-таки поделикатнее. Никто не пытался доказать, что их никогда в
Испании не было, и переименование, если и было, то в виде исключения.
Старинные арабские названия, как Альказар, Альгамбра, Алехезирас и
другие, до сего времени фигурируют на географических картах. О евреях
скажем несколько слов в следующем разделе.

г) Социальные противники

В поздние Средние века и позже очень много выдающихся деятелей
пострадали за то, что они выступали обличителями общественных язв
того времени: бедствий народа, распущенности духовенства и знати,
злоупотреблений властью и т. д. Иногда такой протест облекался в форму
той или иной ереси (например, у анабаптистов), но очень часто у
противников официальных кругов не было ни малейших догматических
отличий от их преследователей. Одну из древнейших фигур этого рода мы
видим в образе пламенного оратора, Иоанна Златоуста, который кончил
свою жизнь в ссылке на Колхиде: от более печальной участи его спас
лишь его несравненный моральный авторитет. Я уже упоминал о таких
фигурах, как Роджер Бэкон, Николай Кузанский, Савонарола, Кампанелла.
Сюда же относятся многочисленные представители нищенствующих орденов,
прежде всего францисканцев, многие из которых погибли на костре за
свою общественно-политическую деятельность. В России был совершенно
параллельный процесс борьбы так называемых нестяжателей (последователи
Нила Сорского) и иосифлян (последователей Иосифа Волоцкого). Первые
были вполне подобны нищенствующим орденам, но, может быть, были более
последовательны и отрицали право церкви на владение землей. Вторые
защищали церковные земли, которые составляли весьма значительную
часть территории России. Власть долгое время колебалась, какую линию
поддержать. Нестяжатели ей нравились, так как давали возможность без
нарушения закона захватить обширные церковные земли, но у них было
неприятное свойство: они, будучи в лице своих лучших представителей
высокоидейными людьми, готовыми пострадать за свои убеждения,
отказывались поддерживать полностью самодержавную власть. Иосифляне же
поддерживали самодержавие. Пришлось московскому правительству стать на
сторону иосифлян, нестяжателей посажали по тюрьмам, повысылали. Если и
были казни, то в незначительном количестве. Правда, потом, когда
самодержавная власть вполне укрепилась и церковные земли были в
значительной степени конфискованы (конфликт Екатерины II с епископом
Арсением Мацеевичем).

Возвращаясь к Западу, мы знаем, что массовая депортация в Испании коснулась не
только мавров, но также и евреев. Евреи в то время не имели
самостоятельного государства, и они были опасны не своим политическим, а
экономическим влиянием. Изгнание евреев и мавров из Испании рассматривается
многими как пример особой нетерпимости католической церкви вообще и
испанской в частности и как одна из важнейших причин последовавшего
экономического упадка Испании.

Если бы мавры составляли столь важную часть экономического актива
Испании, то при возвращении их на свою прежнюю родину они должны
были бы способствовать ее развитию. Но мы знаем, что мусульманские
государства в целом шли неуклонно к своей деградации. Что касается
евреев, то у нас забывают, что не Испания начала изгнание евреев.
Первой на этот путь вступила Англия еще во времена крестовых походов,
дальше пошли другие государства, Испания долгое время принимала к себе
евреев. В Англию евреи были вновь допущены только при Кромвеле, в XVII
в. Изгнанные из Испании евреи после долгих мытарств в конце концов
осели в католической Польше при правлении, по-видимому, действительно
крупного государя Казимира Великого. Для меня, малознакомого с
историей евреев, непонятно, почему они выработали свой «идиш»,
представляющий собой диалект немецкого языка: очевидно, они достаточно
долго жили в Германии и были в общем оттуда изгнаны. Сейчас,
конечно, преследования евреев в католической Испании кажутся буквально
пустяками по сравнению с судьбой их в нацистской Германии. До
такого радикального решения еврейского вопроса ни Средневековье, ни
инквизиторы не додумались.

Ну а как в СССР? Как сказала бывший посол Израиля в СССР (а потом министр
иностранных дел Израиля) Голда Мейер, в СССР нет антисемитизма, а есть
антисионизм, что большой разницы не показывает. Перед самой смертью Сталина
ходили упорные слухи о массовой депортации евреев из всех городов СССР в
Биробиджан. Составлялись списки всех евреев (мне известно по Киеву и
Ленинграду), не считаясь с их общественным и партийным значением. Смерть
Сталина прекратила это мероприятие, но вероятность его ясна из кошмарного (и
совершенно фальшивого) дела о врачах-отравителях, где был замешан цвет
медицинской корпорации Москвы, было очень много евреев и определенно
указано на наличие сионистских организаций. Антисионизм и сейчас
господствует в официальной идеологии, и как Сталин дружил с
«национал-социалистом» Гитлером, так и нынешнее руководство дружит с
современным «национал-социалистом» Абдель-Насером. Сталинская линия в
иностранной политике осталась в полной неприкосновенности. Сейчас в
еврейском вопросе выступила и политическая сторона.

Настоящими социальными врагами у нас были буржуи, кулаки, подкулачники.
Ввиду крайней неопределенности последнего понятия, оно обнимало почти все
крестьянство, которое и поплатилось страшно в период коллективизации ---
примерно 8--10 миллионов жертв.

д) Общее о церковных преследованиях

Я выделил четыре главных категории лиц, подвергшихся церковным
преследованиям, и показал, что все они имеют аналоги в современных
преследованиях, не имеющих ничего общего с религией:

еретики --- ревизионисты

ведьмы и колдуны --- вредители

политические противники --- империалисты

социальные противники --- буржуи, кулаки и т. д.

Но все эти категории охватывают лишь факторы, так сказать, безличного
характера, когда врагами оказываются группы населения, а не отдельные лица. Но
всегда во всех правительствах, и особенно тиранических, играли огромную
роль личные моменты: властолюбие, корыстолюбие, косность, соперничество на
самой разнообразной почве, зависть, приспособленчество. Временами причина
конфликта такова, что все факторы перемешаны. Например, в нашем конфликте с
Китаем все перепуталось. Протестуя против ревизионизма, против ревизии
устаревших представлений марксизма (неизбежность мировой революции и взрыва
капиталистического мира, свобода рабочего класса от националистических и
расовых предрассудков, необходимость синтеза новых идей и Ганди и пр.), мы,
однако, успешно ревизовали ряд почтенных положений марксизма. Старая
история России подверглась в значительной степени полной реабилитации (я уже
не говорю о господствовавшем во времена Сталина возвеличении такого
изверга, как Иван Грозный), и все завоевания царской России сейчас
рассматриваются как законное приобретение, и ревизия границ принципиально не
допускается (в отличие от империалистических государств, в особенности
Англии, которая почти полностью ревизовала свою империю и продолжает давать
свободу жалким остаткам великой империи) --- одна из причин конфликта с
Китаем и Румынией. Эмиграция и иммиграция в капиталистических странах
продолжает оставаться весьма значительной, но мы принципиально отказываемся от
иммигрантов соседнего социалистического Китая и отказываемся объединяться в
единое социалистическое государство, т. е. сделать крупный шаг по пути
величественной программы Рабочей Марсельезы:

И сольются в одно все народы

В вольном царстве святого труда.

В социалистических странах пышно расцвел культ личности и режим личной
диктатуры, совершенно не предусмотренный марксизмом: у нас --- Сталин и
Хрущев, в Китае --- Мао Цзэдун, на Кубе --- Фидель Кастро.

Колоссальную роль в нашей истории сыграло властолюбие и зависть к своим
товарищам по партии. Они привели к тому, что Сталин погубил весь цвет
партии, следуя в этом отношении примеру античного тирана (сбивать колоски,
торчащие над остальными). Совершенная недопустимость даже малейшей критики
правительства не предусмотрена Марксом и Энгельсом.

Можно заключить эту главу словами, что утверждение об особой вредности
религии, церкви, и особенно католической церкви, совершенно несостоятельно. Это
аргумент имел силу в «либеральный» XIX в., но в XX в. сам термин
«либерализм» в нашей стране приобрел оскорбительный характер. Все
преступления, которые приписывались католической церкви как католической
церкви, связаны не с религией, а с догматизмом, общенародными суевериями,
политическими, социальными и личными мотивами, которые действуют не с
меньшей, а может быть, даже с большей силой при полном изгнании всяких
следов религиозной идеологии.

Глава 3. ПОЛЬЗА ОТ ЦЕРКВИ И РЕЛИГИИ ДЛЯ НАУКИ

1. Польза католической церкви для
науки

В прошлой главе я постарался разобрать главные обвинения, предъявляемые
главному обвиняемому --- католической церкви, и указать на то, что может
служить в ее защиту. Но, разобрав вред, причиненный церковью науке, и
показав, что вина далеко не может считаться доказанной, разберем теперь
свидетелей защиты, говорящих, что католическая церковь принесла и немалую
пользу развитию науки.

Прежде всего неоспорим факт, что вся средневековая наука, как и культура,
была полностью в руках католического духовенства. Очень любопытно пишет
марксистский историк науки Дж. Бернал (1956, с. 179): «Это длинное
теологическое, философское вступление к средневековой науке необходимо
потому, что даже самые небольшие научные исследования того времени
предпринимались исключительно для религиозных целей и представителями
духовенства --- священниками, монахами или членами какого-либо ордена. В этом
отношении условия ее развития заметно отличаются от условий развития
мусульманской науки, где мало кто из ученых имел религиозное призвание, а
большинство руководствовалось откровенно утилитарными целями». Бернал
возражает против современной «моды», предпринятой рядом современных ученых, ---
превозносить науку Средневековья в ущерб науке Возрождения. Он указывает, что
даже Роджер Бэкон (ок. 1235--1315) в своих раздраженных и извращающих
правду обличениях --- он называет великих св. Альберта и Фому
«невежественными мальчишками» --- никогда не подверг бы сомнению, что главной
целью науки была поддержка откровения. «Его единственным отличием от них
было то, что он искал подтверждения своих положений в опыте, а не в разуме.
Средневековые ученые были вполне компетентны в научных рассуждениях,
замыслах и выполнении опытов. Эти эксперименты были, однако, изолированными и,
подобно арабским и греческим, продолжали оставаться в основном
демонстрациями, не ведущими к каким-либо научным революциям. Какой бы
похвалы ни заслуживала за свои достижения горсточка средневековых
экспериментаторов, они фактически мало прибегали к использованию этих
методов для исследования природы и еще меньше --- для управления ею. У них не
было стимула это делать, но много поводов за то, чтобы не делать. Будучи
духовными лицами, они имели множество других занятий: Герберт (ок.
930--1003), первый из западных ученых, стал папой; Роберт Грассетест (ок.
1168--1253), наиболее талантливый из них, был епископом и президентом
Оксфордского университета; св. Альберт Великий был архиепископом
Доминиканского ордена, причем власть его распространялась на всю Германию;
такую же должность занимал Дитрих из Фрейбурга (1390), лучший из
экспериментаторов. Даже наиболее смелый мыслитель позднего Средневековья
Николай Кузанский (1401--1464) подпал под влияние папской пропаганды и стал
епископом в Бриксене. Все, что они делали для науки, они делали в свободное
время».

«Исключения --- Роджер Бэкон и таинственный Петр Пилигрим --- подтверждение
правила. Роджер Бэкон затратил крупное состояние на научные изыскания и,
несмотря на папское благословение, был за свои труды заключен в тюрьму.
Согласно его поклоннику Р. Бэкону, „Петр не заботился о речах и словесных
битвах, но занимался мудрыми делами и находил мир в них".

Весь итог средневековых достижений в естественных науках сводится к
нескольким заметкам св. Альберта по естественной истории "и минералам;
трактату императора Фридриха II об охотничьих птицах; Петр Пилигрим был
пионером в экспериментальном исследовании магнетизма, о котором он
опубликовал одно короткое письмо в дополнениях к оптике Альгазена,
сделанных Дитрихом из Фрейбурга и Витело, включая объяснение радуги,
оставшееся неизменным до Ньютона, и в некоторых не очень оригинальных
критических замечаниях Буридана и Эразма по теории движений Аристотеля.
Основываясь на этом, теперь утверждают, что датой начала научной революции
следует считать ХIII в. и что св. Альберт, несколько запоздало
канонизированный в 1931 г., имеет право считаться святым покровителем
науки».

Эта длинная выписка из работы Бернала, крупного физика, компетентного в
истории науки и придерживающегося марксистских взглядов, заслуживает
детального комментария. Коснусь основных положений цитаты Бернала, с тем
чтобы дальше коснуться некоторых из них более подробно.

1) Бернал (с. 181) признает, что «вклад средневекового христианства в науку
был, быть может, несправедливо забыт в прошлом», но сейчас он опасается, как
бы он не был преувеличен, а действительно новейшие работы по истории науки
открывают все больше и больше. Для Бернала в Средневековье работала горстка
экспериментаторов, давшая несколько блестящих имен. Но немногие блестящие
имена --- это те, которые сохранились, а они, конечно, работали наряду с
большим количеством оставшихся безвестными или чисто легендарными учеными
(например, знаменитый монах Бертольд Шварц, один из многих, кому
приписывают изобретение пороха и притом на основе чисто религиозных
представлений. Один вариант по поводу чтения жития великомученицы
Екатерины, которая была казнена за отступничество собственным отцом,
который был тут же убит молнией: «душа мученицы была чиста как селитра,
душа отца черна как уголь, и небесная сера (молния) пожгла нечестивца».
Другой вариант: попытка вызвать дьявола, который черен как уголь, едок как
селитра и пахнет серой).

2) Важно то, что приведенные Берналом блестящие имена --- за немногими
исключениями --- не темные монахи, работавшие в безвестной тиши монастырей, а
магнаты церкви, пользовавшиеся огромным авторитетом и не подвергавшиеся
никаким преследованиям.

3) Поразительный контраст. На фоне довольно длинного списка духовных
деятелей, внесших солидный вклад в науку, фигурирует только одна светская
фигура --- императора Священной Римской Империи Фридриха II Гогенштауфена. Это
был один из немногих атеистов тех времен, и он был в свое время отлучен
папой. Он был, как было свойственно в те времена всем монархам, страстным
соколиным охотником, написал о соколиной охоте книгу и, будучи
наблюдательным человеком, отметил, что птицы того же вида северного
происхождения крупнее птиц с юга. В ХIХ в. это правило было вновь открыто
Бергманном и получило название «правила Бергманна». Вот едва ли не
единственное открытие, которым осчастливили науку средневековые феодалы
разных рангов, имевшие достаточно свободного времени, но которые считали
занятиями, достойными рыцарей, лишь войну и охоту.

4) Противопоставление христианства мусульманству несомненно заслуживает
внимания, но его требуется уточнить. Верно то, что мусульманские ученые
практически не были связаны с религией и несомненно также, что в исламе
больше выпячивали практическое значение науки. Но занятие чистой наукой в
исламе было делом гораздо более опасным, чем в христианстве (вспомним
судьбу великого астронома Улугбека), и «потолок» науки в исламе был гораздо
ниже, чем в христианстве. Что же касается практической направленности, то она
очень сильно была выражена и в христианстве, отнюдь не поглощая собой
теоретическую работу. Я уже цитировал указания самого Бернала на
значительное число изобретений (отчасти заимствованных у других народов), и в
списке лиц, занимавшихся изобретательством, указано много таких, которые
фигурируют и в списке крупных ученых вообще (Альберт Великий, Роджер Бэкон и
проч.). Исследования по вкладу Средневековья в технику все продолжаются, и
список изобретений все увеличивается. Очень интересная статья опубликована во
французском марксистском журнале «Мысль»: Magalhaes Vilhene. Progres technique
et blocage sociale dans la cit\'{e} antique. La Pens\'{e}e, 1962, № 102, April, р. 103--120.

В этой статье есть указания на ряд работ, показывающих удивительное
разнообразие технических изобретений Средневековья. Из этой статьи ясно, что
сейчас нет спора о выдающихся успехах техники Средневековья, спор идет о
том, в какой степени превосходят технические достижения Средневековья
таковые античности и являются ли они продолжением достижений античности.
Любопытно, что здесь подчеркивается огромная роль Платона даже в развитии
античной техники в отличие от обычного мнения.

5) Само собой разумеется, что деятельность ученых и мыслителей
Средневековья не ограничивается тем скромным списком, который приводит
Бернал, но верно также, что главная роль столь презираемых схоластиков
заключалась в изощрении мысли, развитии того мыслительного аппарата,
который принес столь блестящие плоды в деле возрождения науки.

6) Датой начала научной революции Бернал считает XIII в., т. е. позднее
Средневековье. Смотря по точке зрения, это и слишком ранняя и слишком
поздняя дата. Если под началом научной революции следует понимать
достаточно отчетливую формулировку новых взглядов, то ее придется связать с
именем Николая Кузанского, предшественника Коперника. А если началом
считать создание первого организационного центра зародыша Возрождения, то
придется такую дату отнести значительно раньше, примерно к VIII в.,
образованию Бенедиктинского монастыря в Монте Кассино с его знаменитой
библиотекой. К. А. Тимирязева нельзя упрекнуть в особом пристрастии к
католичеству и монахам, а, однако, и он упоминает Монте Кассино как образец
учреждения, где люди вдали от мирской суеты занимаются наукой. И, как он
пишет, Институт Пастера в Париже многие называли «светским монастырем». А
VIII в. --- это самая гуща «веков мрака», как недавно любили называть
Средневековье. Организационного расцвета средневековая наука достигла с
организацией университетов, о чем полезно сказать несколько слов.

2. Средневековые университеты

Эти учреждения не были, конечно, вполне оригинальными, было несомненное
влияние, например, знаменитого магометанского университета в Кордове, где
учился и будущий папа Сильвестр II (Герберт). Прообразом университета
являлась, конечно, Академия Платона (Бернал, с. 121), и неудивительно, что и
схема преподавания первых университетов была по Платону: тривиум и
квадривиум. Из потребности дать образование духовенству родились соборные
школы, а затем университеты (Бернал, с. 176). Основание первого и наиболее
известного университета, Парижского, относится к 1160 г. Почти одновременно с
Парижским, если не раньше, --- Болонский университет, Оксфордский,
практически как филиал Парижского 1167, Кембриджский 1209. Затем Падуя
1222, Неаполь 1224, Саламанка 1227, Прага 1347, Краков 1364, Вена 1367 и
Сент-Эндрьюс 1410.

Университеты возникли в самых разнообразных странах (Франция, Италия,
Англия, Испания, Чехия, Польша, Австрия) и составляли некоторое единое
целое, так как языком науки в то время был один язык --- латинский. Это
чрезвычайно облегчало и передвижение ученых из одной страны в другую. Мне
неизвестно, возникло ли деление на четыре факультета (теологический,
философский, медицинский и юридический) с самого начала, но установилось оно
очень давно и сохранилось и до настоящего времени, несмотря на то что
философский факультет, в средневековом смысле обнимающий все светское
знание, с тех времен чрезвычайно разросся. Основы обучения: тривиум
(грамматика, риторика и логика) и квадривиум (арифметика, геометрия,
астрономия и музыка) восходили к Платону. Дальше уже сообразно факультету. Как
пишет там же Бернал, основное обучение было не только светским, но и
научным; в этом оно было построено по мусульманскому образцу.

Университеты с самого начала были местом развития научной мысли. Конечно,
главная задача была поставлена еще до появления университетов. «Уже в XI в.,
до того как влияние арабского учения стало полностью ощущаться, центральной
проблемой схоластических диспутов явилось создание основы для веры в разум, или
более узко, для примирения священных писаний и сочинений духовных отцов с
логикой греков» (Бернал, с. 177). Задача поставлена была: получить синтез, а
не подавить ниспровергнутое языческое учение. Знаменитый Абеляр (1079--1142) в
своем сочинении «Да и нет» представил значительное количество выдержек из
произведений духовных отцов, выражающих противоположные мнения почти по
каждому существенно важному вопросу (Бернал, там же), и вот горячие споры по
самым разнообразным вопросам и проходят красной нитью через всю схоластику.
Возможно, что были и споры о том, сколько демонов может поместиться на
конце иглы (что нам в свое время приводили для посрамления схоластики), но
споры шли и по гораздо более важным вопросам. Даже когда выдающийся ум Фомы
Аквината дал законченную систему, примирившую христианство с Аристотелем,
споры не прекратились, как было уже указано выше в отношении Роджера Бэкона.

Возникли такие глубокие философские споры, как спор номиналистов и
реалистов, позднее по радикальному вопросу о предопределении и свободе воли и
очень много других.

Но, может быть, споры эти происходили в очень узком кругу и интеллигенция
того времени была малочисленна? Нет, хорошо известно, что интеллигенция
тогда совсем не была так малочисленна, как кажется. Наш профессор Шимкевич
(который как раз всегда любил вспоминать на лекциях о числе демонов на
конце иглы) указывает в «Популярных биологических очерках»: «Жажда знаний
была в массах. Монашеские ордена учреждают школы, из которых в
бенедиктинской школе в Монте Кассино возникают первые медицинские курсы,
возникают университеты Парижский, Оксфордский, Болонский и др. и
переполняются слушателями. В XI и XII вв. латинский квартал составлял 1/3
всего Парижа, а Альберт Великий мог читать только на Плас Мебер, ибо ни
одна аудитория не могла вместить всех желающих его слушать. Число
слушателей Михаила Скотуса доходило до 30 тысяч».

Но, может быть, вся атмосфера средневековых университетов была скована
строжайшей дисциплиной и неусыпным надзором со стороны? Нет, известно, что
средневековые университеты пользовались реальной автономией, само это понятие
возникло тогда. Студенты играли значительную роль в управлении, и если
профессор не собрал известного числа студентов в своей аудитории, то
подвергался взысканию. Недавние беспорядки в Сорбонне, Парижском университете,
который некогда был твердыней ортодоксии, отчасти вызваны тем, что студенты
Сорбонны стремятся восстановить некоторые средневековые обычаи, связанные с
большей ролью студентов в управлении (это движение среди студентов Запада
коснулось не только Франции). Характерное признание о свободе, господствовавшей
в средневековых университетах, сделал советский критик А. Парфенов во
вступительной статье к собранию сочинений Кристофера Марло (Москва, 1961) по
поводу Кембриджского университета, который ко времени Марло был уже, конечно,
протестантским. Университет в конце XVI в. представлял в значительной мере
светское учреждение, исчезли толпы монахов, схоластическому богословию пришлось
потесниться, но «исчезла и ничем не ограничиваемая свобода средневекового
студенчества». Все было регламентировано, но вместе с тем ересь и самое
страшное зло --- католицизм выслеживались с полицейской зоркостью.

Но как совместить эту свободу с преследованием еретиков? Конечно, еретиков
преследовали, но надо иметь в виду, что догматы церкви были не так
многочисленны и вовсе не касались всего человеческого знания. Даже в
богословии наряду с догматами были так называемые «теологумены», где
дискуссии разрешались. Многие существенные вопросы (например, о
предопределении и свободе воли) до сего времени получают различное решение в
разных католических школах, ввиду того что различные и часто соперничающие
организации (в отношении свободы воли такими являются доминиканцы и
иезуиты) придерживаются разных взглядов, и папам, заинтересованным в
деятельности обеих организаций, часто бывает нежелательно решительно
становиться на ту или другую сторону. Наконец, в Средние века удалось
добиться по отношению ко многим вопросам позиции так называемой
двойственной истины. Этот термин (Философский словарь, 1963, с. 113)
обозначает «учение о взаимной независимости истин философии и богословия,
возникшее в эпоху Средневековья и направленное на высвобождение науки от пут
религии». Это учение развивал особенно арабский философ Ибн-Рушд
(Аверроэс), но его развивали и представители номинализма (Дуне Скот, Уильям
Оккам).

Мы видим, что учение о двойственной истине развивали и арабские и
христианские мыслители, но почему-то мусульманская культура угасла
решительно во всех мусульманских государствах, а христианская стала на путь
блестящего развития. Учение Ибн-Рушда (аверроизм) жестоко преследовалась
мусульманской ортодоксией (Философский словарь, с. 156). Разница в том, что
наука ислама всегда носила исключительно светский характер (Бернал, с.
162): «Этим светским и торговым фоном наука ислама резко отличалась вместе с
тем от христианской науки, которая носила почти исключительно религиозный
характер. В этом отношении наука арабов в известной степени напоминала
науку Возрождения. Именно такое покровительство двора и богатых меценатов
дало возможность врачам и астрономам ислама ставить опыты и делать
наблюдения. Пока покровительство продолжали оказывать, оно защищало ученых от
гнева религиозных фанатиков, которые понимали, что вся эта премудрость так или
иначе может поколебать веру в Аллаха. Союз науки с царями, богатыми купцами и
знатью был вначале источником ее силы, а в конечном счете стал источником ее
слабости, поскольку с течением времени она оторвалась
от народа, который считал, что ученые советники сильных мира сего ни к чему
хорошему не приведут. Это делало ученых легкой добычей религиозного
фанатизма». В христианском же мире церковь стремилась дать синтез веры и
знания, ученые в массе своей были духовные, а духовенство пользовалось
доверием народа, несмотря на свои прегрешения. Бернал правильно пишет (с.
163): «Полная неудача попыток примирить науку с устойчивыми особенностями
мусульманской религии была, очевидно, главной причиной увядания науки в
последние века существования ислама, который в культурном и
интеллектуальном отношении переживал застой». Разумеется, другой причиной
упадка мусульманской науки был ее практицизм. «Чистая» наука особенно не
пользовалась уважением и среди магометан. «Только гяуры могут заниматься
подобными пустяками: порядочный мусульманин никогда не будет тратить время на
подобную чепуху», как сказал преемник Махди, когда узнал, что снятый с
убитого аиста цилиндрик с отметкой «Аскания Нова» просто был прикреплен к
птице для выяснения путей пролета птиц.

Великолепную оценку вредного влияния практицизма и общего значения церкви для
развития науки дал основоположник кибернетики Норберт Винер (Я --- математик.
1964, с. 345--346):

«Мы живем в эпоху, когда соображения выгоды играют настолько исключительную
роль, что подавляют все остальные... Открытие, которое, может быть, только
через пятьдесят лет даст что-нибудь практике, почти не имеет шансов
оказаться выгодным для тех, кто оплачивал всю работу, проделанную ради
того, чтобы оно совершилось; в то же время не стремиться к такого рода
открытиям и жить тем, что уже достигнуто, --- значит предавать свое
собственное будущее и будущее своих детей и внуков».

«Научные традиции, как рощи секвойи, могут существовать тысячи лет;
древесина, которую мы потребляем сейчас, --- результат вложений, сделанных
солнцем и дождем много веков тому назад. Прибыль от этих вложений налицо, но
какая часть денег и ценных бумаг остается в одних и тех же руках на
протяжении хотя бы одного только столетия? Вот почему, измеряя долгую жизнь
секвойи денежными единицами, обладающими лишь скоропреходящей ценностью, мы не
можем считать секвойю сельскохозяйственной культурой. В мире, связанном
стремлением к выгоде, мы вынуждены эксплуатировать рощи секвойи, как шахты, не
оставляя будущему ничего, кроме опустошенной земли».

«Некоторые научные идеи, возникшие еще во времена Лейбница, т. е. около
двухсот пятидесяти лет тому назад, только сейчас нашли применение в
промышленности. Может ли какая-нибудь фирма или правительственное
учреждение с их вечными заботами о немедленном создании нового оружия
охватить такой период времени?»

«Огромная роща науки должна быть передана на попечение долгоживущим
организациям, способным создавать и поддерживать долгоживущие ценности. \emph{В
прошлом одной из таких организаций была церковь, и хотя сейчас она в какой-то
степени лишилась былого величия, в свое время именно ее усилиями были
созданы университеты, академии и другие научные учреждения,
просуществовавшие уже многие века».} (Выделено мной. --- \emph{А.Л.})

«Эти долгоживущие учреждения не могут требовать и не требуют немедленного
превращения надежд и идеалов в мелкую разменную монету сегодняшнего дня. Они
существуют благодаря вере в то, что совершенствование знаний --- благо,
которое в конце концов должно принести пользу всему человечеству».

Замечания Винера имеют ценность и потому, что его никак нельзя заподозрить в
семейной симпатии к католической церкви. Он --- еврейского происхождения, и его
дед издавал газету на идиш в Белостоке.

Какие же другие организации, кроме католической церкви, способствовали
развитию чистой, не прикладной науки? Оставляя в стороне по незнакомству
Индию, можно сказать, что в области Средиземноморья в широком смысле
таковыми были: жречество Халдеи (Месопотамии) и Египта, орфическая религия,
перешедшая в пифагореизм, иудейская религия. Пифагорейский союз, Академия
Платона, Александрийский музей, христианская церковь, в первую очередь
католичество. Идейная связь всех этих организаций не вызывает никакого
сомнения. Что общего между всеми этими организациями? Их лучше всего
назвать пифагорейскими религиями, хотя, видимо, они существовали и до
Пифагора.

3. Пифагорейские религии

Характерно для них --- понимание Вселенной как Космоса (гармония, красота), а не
как Хаоса. Этот Космос подчинен Логосу, поэтому законы природы выражены в
математических законах, доступных пониманию человека. По-видимому,
совершенно ясным представителем такого понимания был мудрый еврейский царь
Соломон. Цитирую по книжке А. Цингера (Нач. физ. I ступ., 1918, с. 457):
«Отходя ко сну, откройте Библию. Среди вечных страниц божественного
назидания вы найдете вдохновенные хвалы чарующей красоте и дивной гармонии
Божьего мира, найдете благоговейное преклонение перед тайной мироздания, из
века в век манящего к себе пытливый разум человека... Откройте книгу, за
тридцать веков до нас написанную царственным мудрецом, возлюбившим
Премудрость более скипетров и престолов, более драгоценных камней и злата,
более здравия и красоты. Какая горячая, безграничная вера в могущество
познающего разума и в незыблемую закономерность внешнего мира, в котором
Всемогущий все расположил мерою и числом и весом».

Отрывок славянского текста из книги Премудрости: «Сей бо даде мне о сущих
познание неложное, познати составление мира, и действие стихий, начало и
конец и средину времен, возвратов перемены, и изменение времен, лет круги, и
звезд расположение, естество животных и гнев зверей, ветров усиление, и
помышление человеков, разнство леторослем и силы корений. И елика суть
скрыта и явна, познах: всех бо художница научи мя премудрость».

Трагическая судьба еврейского народа не позволила в полной мере развить эту
науку в государственных рамках, но влияние ее на эллинскую культуру (сходство с
платонизмом) всерьез рассматривалось, например, Августином в его «Граде
божием». Несомненно, что «линия Соломона» сохранилась и в более поздние
времена. Поразительный пример пытливого понимания того, что на первый взгляд
кажется чудом, я прочел у Эфроимсона, «Введение в медицинскую генетику», 1964,
с. 125: «Уже составители Талмуда настолько хорошо знали о „матриархальном"
наследовании гемофилии по женской линии, что это нашло отражение в совершенно
точном указании: смерть от кровотечения при обрезании освобождала от обрезания
всех родных братьев и двоюродных братьев погибшего, притом братьев двоюродных
именно только по женской линии, а не по линии мужской. Таким образом,
наблюдения над наследованием гемофилии имеют тысячелетнюю давность». Это
указание представляет совершенно исключительный интерес. Ведь в данном случае
смерть происходила при совершении священнейшего для иудеев обряда, которым
мальчик и причислялся к избранному народу, и, оказывается, не все евреи по
закону являются обрезанными. Если бы религия имела такое вредное влияние и если
бы религиозные люди имели обычай все исключительное относить к
непосредственному вмешательству Бога, то рассуждение религиозного еврея в
случае гибели мальчика при обрезании могло бы иметь только такой характер:
гибель мальчика при совершении обязательного обряда есть очевидная кара Бога за
какое-то преступление, совершенное предками. Вместо этого столь презираемые
нашими «прогрессистами» талмудисты начинают искать естественные причины
исключительного явления (вспомним определение Августином чуда: «чудо не против
природы, а против того, что нам известно о природе») и, собрав наблюдения,
очевидно, значительного числа случаев, приходят к рекомендации, совершенно
понятной с точки зрения современной генетики. А всякий, кто пробовал объяснять
схему наследования гемофилии лицам, далеким от генетики, знает, как трудно
понимается этот удивительный случай наследственности: сын больного отца никогда
не бывает болен, он наследует болезнь от деда через внешне вполне здоровую
мать.

Еврейский народ сохранил соломоновскую традицию и в том поразительном отношении
к учению (хотя бы к талмудическому) и к учителям. Лично в период 1919--1921 гг.
за время работы в Крымском университете я много сталкивался с евреями как в
качестве ассистента по гистологии (у проф. А. Г. Гурвича) на медицинском
факультете (а там 85\% студентов были евреи), так и в качестве репетитора в ряде
еврейских семей. Меня поражало совершенно исключительное отношение ко мне со
стороны евреев-учеников и их родителей, принадлежавших по имущественному
положению к весьма разнообразным слоям общества. Разъяснение этого я получил
гораздо позже во время последней войны, когда мне случилось давать консультации
по биометрии одному профессору еврею. Как-то при его уходе я хотел подать ему
пальто. «Что вы, что вы, --- возразил он, --- учитель не может подавать пальто
ученику». --- «Какой же я учитель: я вам дал всего немного советов». --- «Учителем,
рабби, является всякий, кто сообщил что-то полезное, а в талмуде сказано: „если
твой отец и твой учитель оказались в тюрьме, постарайся добиться освобождения
сначала учителя, а потом
отца, потому что отец дал тебе жизнь, а учитель научил тебя мудрости"». Это и
другие события моей жизни заставили меня заинтересоваться столь презираемым
нашими прогрессивными безбожниками Талмудом, и мне удалось прочесть первые два
тома (из четырех) «Агады» (сказания, притчи, изречения Талмуда и Мидрашей),
составленных И. X. Равницким и X. Н. Бяликом в русском переводе С. Г. Фруга
(Изд. Зальцман, Берлин, 1922). Я там нашел много интересного.

Ученость ценилась так высоко, что даже вероотступник Ахер не был лишен райского
блаженства за свою ученость по молитве своего преданного ученика Мейра.

Известно, что у евреев никогда не было «единодушия» во мнениях, и это
разногласие не осуждалось, но одобрялись кроткие спорщики (часть вторая, с.
71). «Три года продолжался спор между школой Шаммая и школой Гиллеля. Каждая
настаивала на правильности своих толкований закона. Но раздался Бат Кол
(небесный голос): «Устами и тех и других гласит слово Бога Живого, но поступать
следует по толкованиям школы Гиллеля».

«Но если учение и тех и других есть слово Бога Живого, чем же удостоилась
преимущества школа Гиллеля?» --- «Тем, что ее последователи отличались духом
кротости и смиренномудрия и наряду со своим учением преподавали также учение
школы Шаммая. Мало того, мнение школы Шаммая они излагали всегда ранее
собственных толкований».

Споры талмудистов иногда доходили до Бога, и вот какие бывали результаты
(Агада, часть вторая, с. 99): Элиэзер воскликнул: «Пусть, наконец, само небо
подтвердит мою правоту!» Раздался Бат Кол (небесный голос): «Зачем противитесь
вы словам Элиэзера? Закон всегда на его стороне». Встал рави Иошуа и говорит:
«Не в небесах Тора. Мы и Бат Колу не подчинимся». Встретился после этого рави
Натан с Илией пророком и спрашивает: «Как отнеслись на небе к этому спору?»
Отвечает Илия: «Улыбкой озарились уста Всевышнего --- Господь говорил: „Победили
меня сыны мои, победили меня"». Евреи всерьез принимали слова библейского
сказания, что Иаков боролся с Богом, за что и получил имя Израиль (богоборец).
Еврейская религия не признает абсолютных авторитетов. Евреи предъявляли
огромные требования к образованию своих «руководящих товарищей». Члены
Синедриона обязаны были знать языки всех народов. Мардохей был знаком со всеми
семьюдесятью языками и наречиями. Не знаю, много ли членов современных
синедрионов отвечают таким требованиям.

В толкованиях Торы талмудисты шли значительно дальше автора Торы, Моисея (с.
116). Когда раби Акиба говорил, то сам Моисей не мог понять, о каком законе
идет речь, так как, по мнению Моисея, он ничего подобного не говорил. На что
раби Акиба возразил, что его толкование вытекает из принципов, установленных
Моисеем на Синае. И наградой за великую ученость раби Акиба была мученическая
смерть (при императоре Адриане).

Вот верность великим традициям Торы и Талмуда и составляет главное основание
изумительной живучести культурной традиции еврейского народа, вызывающей страх
у юдофобов и уважение у всех культурных людей. Здесь дело вовсе не в каких-то
особых способностях
еврейского народа (юдофобы, как всякие --- фобы, т. е. боящиеся, нередко
чрезвычайно преувеличивают способности евреев), а именно в непрерывной
культурной традиции, приведшей к подлинному чуду --- возрождению государства
Израиль в самых невозможных условиях.

Соломонов дух таким образом очень близок к пифагорейскому и, может быть, надо
было бы взять термин «соломонов» по принципу приоритета, но, во-первых,
подлинное развитие этот дух получил только в Элладе, а во-вторых, на приоритет
могут претендовать еще более древние религии: халдейская и египетская. Там тоже
относились к чудесам не как к тому, что принципиально противно природе, а лишь
как к тому, что противно известному нам о природе! Возьмем затмения, солнечные
и лунные. Со стороны всякого примитивного человека всякое затмение есть чудо,
рассматриваемое часто как предупреждение, «Знамение от Бога», как поет князь
Игорь. Но ведь закономерность затмений установлена, насколько мне известно,
египетскими жрецами задолго до князя Игоря. Ведь достаточно было в течение
сотен лет (а для этого требуется стойкая организация жрецов, защищенная от
вмешательства профанов) тщательно регистрировать эти в общем редкие явления, а
потом, руководясь только идеей закономерности Вселенной, искать и найти эту
закономерность.

4. Непифагорейские религии

Обратимся теперь к религиям иного сорта, которые не верят в первоначальную
закономерность и гармонию Вселенной и потому, естественно, не поддерживают
стремлений к просвещению и науке. Мне кажется, крайним примером таких религий
является религия некоторых курдских племен, так называемые езиды, если верить
сообщениям миссионера Гардзони по его заметке, приложенной А. С. Пушкиным к его
«Путешествию в Арзрум». Из всех сект, возникших в Месопотамии среди мусульман
после смерти их пророка, нет ни одной, которая была бы столь ненавистна для
всех прочих, как секта езидов. Основатель секты, шейх Езид, заклятый враг рода
Али: смесь манихейства, магометанства и верований древних персов. Священных
книг нет, запрещено обучаться чтению и письму. Первое правило езидов ---
заручиться дружбой дьявола и с мечом в руках вставать на его защиту. Они
признают всех пророков и всех святых, которых чтут христиане и имена которых
носят монастыри, расположенные по соседству. Особенно сильно, по их мнению,
дьявол проявился в Моисее, Иисусе Христе и Магомете. Они думают, что Бог
повелевает, но выполнение своих велений поручает дьяволу. Уважают христианские
монастыри, целуют руку патриарху и епископу. Но избегают входить в турецкие
мечети.

Самое могущественное из племен на горе Синджар между Моссулом и рекой Хибур.
Грабят караваны, выдержали несколько войн против пашей Моссула и Багдада.
Наводят ужас своей жестокостью: убивают всех ограбленных. Если же среди них
находятся шерифы, потомки Магомета, или мусульманские законоучители, то убивают
особенно жестоким способом и с особенным удовольствием, полагая, что это
составляет большую заслугу. Султан терпит их в своих владениях,
потому что, по мнению магометанских ученых, следует признавать правоверным
всякого, исповедующего основной догмат: «Нет Бога, кроме Аллаха, и Магомет ---
пророк его», хотя бы он не соблюдал ни одного из остальных предписаний
мусульманского закона. Магометане твердо убеждены, что всякий погибший от руки
езидов умирает мучеником. Князь Анадийский поэтому держит при себе палача из
езидов для исполнения смертных приговоров над турками. Полная взаимность:
турок, убивающий езида, совершает богоугодное дело, езид, убивающий турка,
угождает великому шейху, т. е. дьяволу. Персы и все магометане из секты Али не
терпят езидов и им запрещено оставлять их в живых. Туркам разрешено обращать в
рабство жен и детей езидов при войнах с ними: езиды при войне с турками убивают
всех.

Достаточно сумбурная религия езидов --- откровенно сатанинская, и пока такая
идеология господствует, ни о каком прогрессе говорить невозможно, какие бы
талантливые люди ни нарождались среди езидов.

Не столь сумбурна, но совершенно чужда пифагорейского духа религия античных
римлян. Господствовала идея государства, никакой «мистики» вроде Платоновых
идей. С богами были запросто: императоры после смерти, а многие и при жизни
причислялись к сонму богов. Никакой чистой науки и попыток решения космических
проблем. Вся философия --- богов нет или они не интересуются людьми, живи
безмятежно в свое удовольствие. В практическом отношении кое-что сделано:
прекрасные дороги, бани, канализация, лучшие военные машины, недурная
организация управления и даже решение вопроса о пролетариате: бесплатные пайки
и билеты в цирк, где без всякого «абстрактного гуманизма» творец истории,
римский народ, может получать удовольствие от взаимного истребления гладиаторов
или еще более пикантно, --- от поедания христиан зверьми.

Трудно сказать ясно о культуре огромного Китая. С одной стороны
--- глубокая древность, огромное количество выдающихся изобретений,
сохранение высокой технической культуры до настоящего времени,
большое значение, придававшееся образованию, --- и, с другой стороны,
несомненный застой в течение нескольких столетий и уродливые зигзаги
современности. Вряд ли Китай был вовсе чужд пифагореизма, но
господствующим пифагореизм никогда не был. Вероятно, прав Бернал
в своем суждении (с. 116): «Предпочтение, которое Платон отдавал
математике, обеспечило наличие по крайней мере одной научной
дисциплины в обучении, которое иначе могло бы быть чисто литературным.
Конфуций, чье влияние на китайское образование было почти столь
длительным, как влияние Платона на Западе, совершенно не занимался
математикой. Этому (вероятно, «это». --- \emph{А.Л.}) возможно, в
значительной степени способствовала сравнительная отсталость китайской
науки».

Таким образом, застойность культуры Китая, мне думается, можно объяснить
недостаточным значением пифагорейского духа: искание законов Вселенной
математического характера, чистая наука.

Культура мусульманства мне очень малоизвестна. Весьма вероятно, там были
пифагорейские струи, но то, что пифагорейские элементы не поддерживались
официальной религией, и привело их к удушению противными течениями: этатизм,
практицизм и проч.

5. Борьба течений в христианстве

Огромная роль платоновско-пифагорейского духа в раннем христианстве не подлежит
ни малейшему сомнению. Значение Логоса (Слова) ясно выражено в начале Евангелия
Иоанна Богослова: «В начале было Слово». Ясна также идейная связь с Гераклитом
и Филоном Александрийским. Соломон и у христиан считается мудрейшим из людей.
Но через все христианство проходит длительная борьба между пифагорейцами и
антипифагорейцами. Несомненно, значительная часть дельных замечаний о вреде
церкви по отношению к науке объясняется и тем, что многие деятели христианства
или даже целые национальные церкви отказались от мудрых мнений многих отцов
церкви и многие решения современных духовных деятелей кажутся вынужденными
новейшими успехами науки. Например, в 1948 г. папа Пий XII заявил ex cathedra,
что первую главу книги «Бытия» следует понимать в аллегорическом смысле
(Бернал, с. 371), что означает конец полемики по вопросу о приемлемости для
католиков теории эволюции. Но в данном случае папа Пий XII только вернулся к
пониманию первых отцов церкви, как Климент Александрийский и др., но вместо
духа Климента Александрийского в значительной степени возобладал дух Кирилла
Александрийского, враждебного всей античной науке.

Можно привести взгляды ортодокса из ортодоксов Василия Великого. Укоренившееся
за последние времена понятие чуда обозначает нечто совершенно
сверхъестественное, для которого невозможно никакое научное объяснение. Но я не
раз цитировал мнение, приписываемое Августину: «Чудо не против природы, но
против того, что нам известно о природе». Одним из казалось бы абсолютно
сверхъестественных явлений было рождение Христа без отца, или, как говорят,
партено-генетически. И вот в своей знаменитой книге «Беседы на шестоднев» (т.
е. о шести днях творения) Василий Великий говорит, что птицы развиваются без
оплодотворения. Исходя из этого факта, Василий считает неосновательными
возражения язычников, что Христос не мог родиться без отца. Конечно, Василий
Великий ошибался, предполагая, что птицы развиваются без отца, хотя и тут он
ошибался меньше, чем думали недавно, так как в новейшей литературе описаны
случаи партеногенеза у птиц, но важно то, что он пытается дать вполне
рациональное обоснование такому сверхъестественному акту, как рождение Христа.

Критики христианства опираются на многие факты искажения первоначального духа
христианства и забвение его основной черты --- гуманности. Как сказал в свое
время В. Соловьев: «Бесчеловеческий Бог породил безбожного человека». Другое
искажение христианства связано с тем, что так сильно повредило мусульманству:
подчинение, иногда прямо порабощение духовных властей гражданскими,
использовавших религию для подчинения подданных. В наиболее сильном развитии
это выражается в цезаропапизме, когда духовная власть целиком подчинена
светской и идет на службу государства. Это случилось, например, в России, где
введенный Петром Великим духовный регламент требует от священников выдачи
политических тайн, сообщенных на исповеди, что по точному учению церкви
является преступлением против Святого Духа, т. е. самым высшим из возможных
преступлений. Значение православия как орудия империализма и подчинения
подданных ясно сознавалось и даже такими деятелями церкви, которые, казалось
бы, целиком поддерживали цезаропапизм. В дореволюционные времена видным
деятелем крайнего правого крыла Государственной Думы был епископ Евлогий. Он во
время революции эмигрировал, но во времена второй мировой войны, живя в Париже,
выступал с речами, благословлявшими победу России (хотя бы и Советской) и
приветствовал возвращение русских эмигрантов на родину: за все это его в
советской печати (забывая его старые грехи) именовали мудрым старцем. В одной
из статей, напечатанных до революции, епископ Евлогий указывает, что среди
православных христиан есть две категории: одна, для которых вера и религия
стоят на первом месте, и они заботятся прежде всего об интересах Церкви;
другая, которых Евлогий называет «русскими римлянами», рассматривает церковь
лишь как орудие государства, и к таким он относит, например, злого гения России
К. Победоносцева. Но не забудем, что Достоевский был в самых дружеских
отношениях с Победоносцевым, поэтому даже Достоевского (с его неистовой
поддержкой русского империализма) можно до известной степени причислить к
«русским римлянам». Название «русские римляне» великолепно: оно вроде
соответствует знаменитой теории «Москва --- третий Рим»: «Два Рима падоша (первый
Рим --- настоящий Рим, второй --- Византия --- Константинополь, впавший в унию с
Римом и завоеванный турками), третий --- Москва стоит, а четвертому не быть
вовеки». Русские римляне могут с торжеством отметить, что империалистические
традиции Третьего Рима сохраняются и в современной безбожной Москве, почему ее
и поддерживают такие видные деятели правого крыла, как епископ Евлогий и
Шульгин.

Историческая победа Москвы и знаменовала крушение пифагорейского начала в
Православной Церкви, а оно несомненно было на нашем севере, в Новгородской
республике (также Пскове и Твери). Новгородский митрополит был совершенно
независим от князя и посадника и имел даже свою собственную дружину. Он
выступал не с воинственными, а миролюбивыми призывами, и когда «буйное
новгородское вече» делалось слишком уж буйным, владыка с крестным ходом шел на
мост, где больше всего бушевали, и мирил споривших. И сейчас, только в XX в.,
через шесть веков после покорения Новгорода отатарившейся Москвой, начинают
понимать, какая великолепная культура была раздавлена задницей московского
медведя. Высокая бытовая культура: мостовые, все в сапогах, а не в лаптях,
широкое развитие письменности в народе (берестяные грамоты), живопись и
архитектура изумительной высоты. Сейчас север России стал местом своеобразного
паломничества и не уничтоженные еще московскими «прогрессивными» варварами
церкви и иконы начинают бережно сохраняться. Была широкая связь не только с
Западом, но и с Востоком (город Мангазея, сейчас раскапываемый), связь была в
общем мирная. Полное отсутствие милитаризма и погубило Новгород, так как для
роковой битвы на Шелони Новгород мог выслать лишь наспех подготовленное войско.
Сыграла, конечно, большую роль и роковая рознь православных и
католиков, так как сторонники независимости Новгорода принуждены были призывать
на помощь «латинскую» Литву и Польшу. История России, вероятно, была бы гораздо
более культурной, если бы старая Русь получила христианство не от Византии, а
от Рима.

Разгром Новгорода, Пскова и Твери был произведен подлинными дикарями, не
понимавшими, на что они поднимают руку. Было бы если не простительно, то до
известной степени понятно, если бы громившие были люди иного племени, иного
языка и иной веры. Здесь громилы по всем признакам были тождественны с
громимыми, но характер погрома был, пожалуй, единственным в мировой истории.
Может быть, остатки пифагорейского духа еще найдутся. В старых постановлениях о
преследовании инакомыслящих (например, нестяжателей) есть указания, что
обвиняемые занимались «эллинской ересью» (Платоном и Аристотелем), высокое
понимание живописи сохранилось вплоть до Стоглавого собора, руководимого
просвещенным митрополитом Макарием (например, все рублевские иконы считались
чудотворными, хотя Рублев сам не был святой: так высоко понимали значение
искусства), сейчас интерес к истории культуры чрезвычайно возрос и, может быть,
в отдаленных северных уголках еще найдутся покрытые пылью веков хартии высокой
культурной ценности.

Борьба течений в христианстве имела место, конечно, и на Западе, но там
католическая церковь, сохранившая независимость от светской власти и подлинный
интернационализм, сумела сохранить в лице лучших своих представителей подлинно
пифагорейский дух. Это принужден признать и неоднократно цитированный Бокль,
который как протестант особо неприязнен к католичеству, а как свободомыслящий
человек борется с клерикализмом вообще. Бокль много указывает на искажение
исторических анналов католическими деятелями раннего Средневековья (т. 2, с.
18, 23), но даже в эти давние времена духовенство приносило не только вред (т.
2, с. 203): «Во Франции длинная цепь событий, о чем скажу после, давала
духовенству раннего периода большую долю власти, чем та, которой они обладали в
Англии. Для того времени эти результаты были положительно благотворными,
поскольку церковь сдерживала беззакония варварского времени и обеспечивала
убежище для слабых и угнетенных. Но поскольку шел прогресс знаний во Франции,
духовные авторитеты, сделавшие много для укрощения страстей, начали сильно
давить на гениев и сдерживать их развитие. Та же церковная власть, которая для
невежественного века является беспримесным благодеянием, для более
просвещенного делается серьезным злом».

Сравнивая Англию и Францию, Бокль пишет (с. 204): «Для обеих стран в их детстве
церковь оказала большие благодеяния, так как всегда была готова защищать народ
от притеснения короны и знати». Но потом народы стали способны сами
сопротивляться и стало необходимым ограничить власть духовенства, которое
мешало прогрессу знаний. В Англии дело дошло до разрыва с католичеством, во
Франции католичество сохранилось и вред его делался все больше вплоть до
Революции.

Протестантизм в Англии привел к движению и религиозную мысль. Оно привело к
отделению теологии от этики в конце XVII столетия и
от политики в середине XVIII столетия. Обе эти перемены возглавлялись
духовенством (т. 2, с. 127). Кумберланд, епископ в Петерборо, был первым, кто
попытался построить систему морали без помощи теологии. Варбуртон, епископ
Глочестера, первый заявил, что государство должно рассматривать религию не в
связи с откровением, а с целесообразностью; «оно должно покровительствовать той
или иной вере не в соотношении с истиной, а только в связи с ее общей
полезностью». Бокль указывает, что то, что такого человека сделали епископом,
было большим достижением даже для XVIII века и было бы невозможно для XVII.
Мнения Кумберланда были развиты потом Юмом и Палеем, который довел утилитаризм
до последнего предела. О Палее же (его логичности) высоко отзывался и Ч.
Дарвин.

6. Эволюция религий. Мысли Эйнштейна

В предыдущем я старался показать, что отношение религии к науке весьма
разнообразно и нельзя говорить, что всякая религия полезна или всякая вредна,
но, вероятно, всегда сосуществовали разные формы религий. Вспомним и нашего
Горького, который хорошо говорил, что у его дедушки и бабушки Бог был
совершенно различен. Несомненно, в истории соотношение разных форм религии
менялось и в общем более примитивные формы сменялись более современными, хотя
несомненно существовали и регрессивные явления, как во всякой эволюции. Много
интересных мыслей по этому поводу высказал наш великий современник Альберт
Эйнштейн. Его высказывания по самым разнообразным вопросам собраны в двух
сборниках. Английское издание этих сборников и использовано мной.

1. Einstein A. The world as I see it. London, 1935, 1941. 214 p.

2. Einstein A. Out of my later years. N. Y., 1950. 282 p.

Эйнштейн различает три источника религии. У первобытного человека религиозные
эмоции вызываются в первую очередь чувством страха --- страх голода, диких
зверей, болезней, смерти (Эйнштейн, 1935, с. 24). Так же он говорит о втором
источнике религии --- общественные чувства: «Желание руководства, любви и
поддержки побуждает человека сконструировать общественное или моральное
понимание Бога. Это Бог Провидения, который защищает, управляет, вознаграждает
или наказывает, Бог, который, смотря по широте воззрений, любит или поощряет
жизнь племени или человеческой расы или даже жизнь как таковую, утешитель
печали и неудовлетворенных желаний, сохраняющий души после смерти. В этом
заключается социальное или моральное понимание Бога».

«Еврейские книги иллюстрируют это развитие от религии страха к моральной
религии, что продолжено и в Новом Завете. Религии всех цивилизованных народов,
в особенности народов Востока, являются первично моральными религиями. Развитие
религии от религии страха к моральной религии --- большой шаг в жизни нации. Мы
должны остерегаться того предрассудка, что примитивные религии целиком
основаны на страхе, а религии цивилизованных народов целиком на морали. Истина
заключается в том, что все они являются переходными,
с той оговоркой, что на более высоких уровнях преобладает религия
нравственности».

«Общим для всех этих типов является антропоморфический характер понятия Бога.
Только индивиды исключительной одаренности и исключительно возвышенные духом
общества могут подняться выше этого уровня. Но имеется третье состояние
религиозного опыта, которое принадлежит всем уровням, но которое встречается
редко в чистой форме и которое я называю космическим религиозным чувством.
Трудно объяснить это ощущение тем, кто его совершенно лишен, в особенности
потому, что ему не соответствует никакого антропоморфического представления
Бога».

«Индивид чувствует ничтожество человеческих желаний и целей и возвышенность и
чудесный порядок, которые открываются как в природе, так и в мире мысли. Он
смотрит на индивидуальное существование как на род тюрьмы и стремится постичь
Вселенную как единое значимое целое. Начало космического религиозного чувства
появляется на ранних этапах развития, например, во многих псалмах Давида и у
некоторых пророков. Буддизм, поскольку мы с ним знакомы в особенности из
удивительных сочинений Шопенгауэра, содержит значительно больший элемент этого
чувства».

«Религиозные гении всех времен выделялись этим сортом религиозного ощущения,
которое не знает догматов и Бога по образу человека; так что не может быть
церкви, центральное учение которой было бы построено на этом. Поэтому именно
среди еретиков всех времен мы встречаем людей, преисполненных высшим видом
религиозного чувства и которые во многих случаях рассматривались их
современниками как атеисты, а иногда также как святые. Рассматриваемые в этом
свете люди, подобные Демокриту, Франциску Ассизскому и Спинозе, очень
родственны друг другу».

«Как космическая религия может передаваться от одного человека к другому, если
она не приводит к определенному понятию Бога и не имеет теологии? По-моему,
наиболее важной функцией искусства и науки является пробуждение этого чувства и
поддержание его у тех людей, которые к нему способны».

«Мы приходим к пониманию отношения науки и религии, очень отличному от
обычного. Если рассматривать историю вопроса, то легко склониться к мнению, что
наука и религия --- непримиримые антагонисты и по вполне понятной причине тот,
кто полностью убежден в универсальности закона причинности, не может даже на
момент представить себе идею существа, которое вмешивается в течение событий, ---
конечно, если он серьезно принимает гипотезу причинности. Ему нет необходимости
в религии страха и также мало в религии социальной или моральной. Бог, который
награждает и наказывает, непонятен для него по простой причине, что
человеческие действия определяются внешней или внутренней необходимостью, так
что в глазах Бога он не может быть ответствен в большей степени, чем
неодушевленный предмет может отвечать за движения, которые он проделывает.
Поэтому науку обвиняли в подрыве нравственности, но обвинение несправедливо.
Этическое поведение человека должно быть эффективно обосновано на симпатии,
воспитании и социальных связях; религиозная основа не
нужна». Дальше я уже цитировал на с. 34 (конец § 1, второй главы), а затем
Эйнштейн говорит о религиозности науки (с. 28):

«Вы вряд ли найдете среди более глубокого разряда научных умов человека без
свойственного ему особенного религиозного чувства. Но оно отлично от религии
наивного человека. Для последнего Бог является существом, от которого он
надеется получить выгоду или наказания которого он боится; возвышение чувства,
сходного с чувством сына к отцу, существу, с которым каждый стоит до известной
степени в личном отношении, как бы оно ни было окрашено благоговением».

«Но ученый исполнен чувством универсальной причинности. Для него будущее так же
необходимо и детерминировано, как прошедшее. Нет ничего божественного в морали,
это чисто человеческое дело. Его религиозное чувство принимает форму
восторженного изумления перед гармонией естественного закона, которая открывает
разум такой высоты, что по сравнению с ним все систематическое мышление и
поведение человеческих существ является крайне несущественным отражением. Это
ощущение есть ведущий принцип его жизни и деятельности, поскольку он успевает
освобождаться от оков эгоистических желаний. Это чувство несомненно родственно
тому, которым обладали религиозные гении всех времен».

Эйнштейн не боится таинственного, мистического, с. 4--5.

«Прекраснейшее, что мы можем испытать, есть таинственное. Это --- основное
ощущение, стоящее у колыбели истинного искусства и истинной науки. Тот, кто не
знает этого и не может больше удивляться, чувствовать изумления, может
считаться мертвым, задутой свечкой. Ощущение тайны --- даже смешанное со страхом
--- породило религию. Знание существования чего-то, во что мы не можем
проникнуть, проявлений глубочайшего разума и самой лучезарной красоты,
доступной нашему разуму лишь в самой элементарной форме --- это знание и это
ощущение и составляют истинно религиозное мироощущение; в этом смысле, и только
в этом, я являюсь глубоко религиозным человеком. Я не могу понять Бога, кто
вознаграждает и наказывает свои создания, или обладает волей того типа, который
я осознаю сам в себе. За пределами моего понимания находится и индивид, который
переживает свою физическую смерть, и я совсем не хочу, чтобы было иначе;
подобные представления соответствуют лишь страхам и эгоизму слабых душ. Для
меня достаточно тайны вечности жизни и намека на чудесную структуру реальности,
вместе с прямодушным стремлением понять частицу, как бы мала она ни была,
разума, который проявляется в природе».

Очень любопытны высказывания Эйнштейна о понятии «научная истина» (1935, с.
131):

1) Трудно придать точное значение термину «научная истина». Так, различно
значение слова «истина» смотря по тому, имеем ли дело с фактом из нашего опыта,
математическим предложением или научной теорией. Совершенно неясно для меня,
что такое «религиозная истина».

2) Научное исследование может сократить суеверие, стимулируя людей думать и
исследовать вещи в понятиях причины и следствия. Несомненно, что убеждение,
родственное религиозному чувству о рациональности и познаваемости мира, стоит
за каждым научным делом высшего порядка.

3) Твердое убеждение, связанное с глубоким чувством, в наличии высшего разума,
открывающегося в мире опыта, представляет собой мое понимание Бога, который на
обычном языке может быть назван «пантеистическим» (Спиноза).

4) Сектантские традиции я могу рассматривать только с исторической и
психологической точки зрения: другого значения они для меня не имеют.

На той же с. 131 Эйнштейн, касаясь метода теоретической физики, пишет: «Если вы
хотите узнать у теоретических физиков о методах, которые они применяют, я
советую вам твердо придерживаться принципа: не слушайте их слов, обращайте
внимание на их дела. Для открывателя на этом поприще продукты его воображения
кажутся столь необходимыми и естественными, что он смотрит на них, и хотел бы,
чтобы другие на них так смотрели, не как на творения мысли, а как на
реальности». И наконец, в том же томе по поводу Кеплера он пишет на с. 146:
«Думается, что человеческий разум должен независимо сконструировать формы
прежде, чем он найдет их в вещах. Поразительные достижения Кеплера являются
особенно изящным примером той истины, что познание не может возникать только из
опыта, а только путем сравнения изобретений интеллекта с наблюдаемыми фактами».

Ряд интересных мыслей о соотношении науки и религии имеется и в позднем
сборнике работ Эйнштейна 1951 г. В статье о «моральном разложении» он пишет (с.
9): «Все религии, искусства и науки являются ветвями одного дерева. Все они
стремятся к облагорожению человеческой жизни и возвышению ее от сферы чисто
физического существования к движению индивида на пути к свободе. Вовсе не
случайно наши старые университеты развились из церковных школ. И церкви и
университеты --- поскольку они остаются верными их истинной роли --- служат
облагорожению индивидуума. Они стремятся выполнить это великое задание,
распространяя моральное и культурное понимание, отказываясь от применения
грубой силы».

«Существенное единство церковных и светских культурных учреждений было утеряно
в XIX в., дойдя до точки бессмысленной враждебности. Однако никогда не было
сомнения в стремлении к культуре. Никто не сомневался в святости цели. Спор был
только о подходе к цели».

Грозные явления современности заставляют сказать (с. 10): «Кто может
сомневаться, что Моисей был лучшим лидером человечества, чем Макиавелли?»

С. 23: «Высочайшие принципы для наших стремлений и суждений даны нам в
еврейско-христианской религиозной традиции. Это очень высокая цель, которую,
при наших слабых силах, мы можем достичь очень несовершенно, но которая дает
нам прочное основание для наших стремлений и оценок. Если снять религиозную
форму с этой цели и посмотреть только на ее чисто человеческую сторону, то ее
можно определить примерно так: свободное и ответственное развитие индивидуума,
так что он может использовать свои силы свободно и радостно на службу всего
человечества».

«Нет места для обожествления нации, класса, не говоря уже об индивидууме. Разве
мы не все дети одного отца, как сказано на религиозном языке? Даже
обожествление человечества как абстрактного целого не соответствует духу этого
идеала. Душа дана только индивидууму».

На с. 24--26 --- чрезвычайно интересные суждения о соотношении науки и религии.

«Не трудно прийти к соглашению о том, что мы понимаем под наукой. Наука есть
многовековое стремление связать путем систематического мышления воспринимаемые
феномены мира в возможно более непротиворечивую ассоциацию. Смелее, это попытка
апостериорной реконструкции бытия путем процесса концептуализации. Но когда
меня спросят, что такое религия, я не могу найти так легко ответ. И даже найдя
ответ, который меня в данный момент удовлетворяет, я остаюсь убежденным, что ни
при каких обстоятельствах я не смогу свести в единое целое даже в слабой
степени всех мыслителей, которые серьезно разбирали этот вопрос».

«Сначала вместо того, чтобы спрашивать, что такое религия, я
предпочту спросить, что характеризует стремления человека, который
производит на меня впечатление религиозного: человек религиозно
просвещенный кажется мне человеком, который, использовав все свои
способности, освободился от оков эгоистических желаний и занят
чувствами, мыслями и стремлениями, оцениваемыми им в силу их
сверхперсонального значения. Мне кажется, что наибольшее значение
имеет сила этого сверхперсонального содержания и глубина убеждения в
огромной значимости этого содержания, независимо от того, делается
ли попытка связать это содержание с божественным Существом, так
как иначе невозможно было бы считать Будду и Спинозу религиозными
личностями. Следовательно, религиозная личность благочестива в смысле
того, что у ней нет сомнения в значимости и возвышенности этих
сверхперсональных объектов и целей, которые не требуют и недоступны
для рационального обоснования. Они существуют с той же необходимостью
и реальностью, как он сам. В этом смысле религия является вековым
стремлением человечества получить совершенно ясное и полное сознание
этих ценностей и целей и постоянно усиливать и расширять их эффект.
Если понимать так религию и науку, то конфликт между ними делается
невозможным. Потому что наука может установить только то, что
\emph{есть}, но не то, что \emph{должно быть}, и за ее пределами
остаются необходимыми оценочные суждения разнообразных родов. Религия,
с другой стороны, касается только оценок человеческих мыслей и
действий: ей не позволительно говорить о фактах и отношениях между
фактами. Согласно этой интерпретации хорошо известные конфликты между
религией и наукой в прошлом должны быть приписаны недоразумению в
понимании только что описанной ситуации». Эйнштейн приводит известные
примеры с Галилеем и Дарвином.

С. 26. «Однако, хотя области науки и религии ясно отграничены друг от друга,
тем не менее между ними существует взаимное родство и зависимость. Хотя религия
может быть тем, что определяет цель, она тем не менее научается от науки в
самом широком смысле слова, какие средства могут быть использованы для
достижения цели. Но наука может быть создана лишь теми, кто целиком проникнуты
стремлением к истине и пониманию. Но этот источник ощущения возникает в сфере
религии. Сюда же относится вера в возможность того, что правила, пригодные для
существующего мира, являются рациональными, т. е. доступными разуму. Я не могу
себе представить истинного ученого без этой глубокой веры. Положение может быть
представлено образно: \emph{наука без религии хром\'{а}, религия без науки слепа}»
(выделено мной. --- \emph{А.Л.}). Это последнее мнение Эйнштейна напоминает
изречение Канта (ни за авторство, ни за точность текста не ручаюсь): «философия
без науки пуста, наука без философии слепа» и изречение его же, что роль науки
как служанки, освещающей своим факелом путь теологии, вовсе не заслуживает
неуважения.

7. Комментарии к мыслям Эйнштейна

Я думаю, что читатель не посетует на меня, что я весь предыдущий параграф свел
к длинным выпискам из мыслей Эйнштейна. Исключительный авторитет Эйнштейна как
ученого и гуманиста заслуживает того, чтобы его мысли по философским вопросам
знали бы больше людей. В его высказываниях по любому поводу много оригинального
и ценного, но это не значит, что к ним надо относиться без всякой критики. Это
не соответствовало бы и мыслям Эйнштейна, протестовавшего против обожествления
человека. Для интересующего нас вопроса важно, что Эйнштейн не только считает
недоразумением конфликт науки и религии, но подчеркивает и большую пользу,
которую принесла науке такая ненавистная для многих, мнящих себя
«прогрессистами» людей, организация, как католическая церковь. Здесь мнение
Эйнштейна полностью совпадает с уже цитированным мнением Норберта Винера,
основоположника кибернетики. Полная независимость суждений обоих ученых в этом
деле ясна уже из того факта, что оба они --- евреи по происхождению. Несмотря на
благородство и красоту высказываний Эйнштейна, их нельзя считать недоступными
критике, и вот что могут возразить материалисты, противники всякой религии, в
том числе и космической религии.

а) О гармонии и красоте

Для Эйнштейна основа космической религии --- ощущение гармонии и красоты
мироздания. Так ведь это же --- чистая иллюзия. Никакой исконной гармонии нет,
то, что нам кажется гармонией, --- это эпифеномен борьбы, хаоса на всех уровнях
бытия: 1) законы физики вроде кинетической теории газов --- следствие
столкновений беспорядочно мятущихся атомов; 2) дивные приспособления
органического мира --- следствие борьбы за существование и естественного отбора;
3) все достижения общества --- науки, мораль, общественное устройство --- следствие
ожесточенной классовой борьбы. И тут не надо быть марксистом. Сказал же Шиллер:
«Природу к жизни побуждает голод
и любовь»: любовь здесь понимается не в платоническом смысле, а в грубом
животном смысле борьбы за самку. И утверждение, что гармония дана как основа
миру, а не добывается тяжелой борьбой, есть не только неверное, но и вредное
учение, так как убаюкивает человека и заставляет его примиряться с
существующими мерзостями.

Назовем эти два мировоззрения: в основе гармония или в основе борьба ---
гармоническим и полемическим (от полемос --- борьба, война). Эйнштейн мог бы
возразить: эти упреки, может быть, имели значение в конце XIX и начале XX в. в
период расцвета полемического мировоззрения, но не сейчас. Сейчас ясно, что и в
физике строение элементарных частиц обнаруживает удивительную гармонию. В
атомах можно найти то закономерное расположение орбит электронов, которое
древние пифагорейцы и Кеплер искали в расположении орбит Солнечной системы, и
много других закономерностей, вовсе не выводимых путем борьбы и столкновений,
вывела современная физика. Мы знаем, что весь прогресс точных наук (прежде
всего развитие гелиоцентрической системы) был связан с гармоническим
мировоззрением, что давало право рассматривать гармонизм по крайней мере как
очень полезную фикцию. Сейчас мы вправе ему придавать и не только фиктивное
значение. В биологии внешне как будто благополучно для селекционизма (учение о
ведущей роли естественного отбора), но это благополучие целиком основано на
некритическом восприятии механистического мировоззрения как обязательного для
биолога. Наконец, в социологии лучшим доказательством того, что марксизм
колеблется между состояниями кризиса и явного загнивания, является то, что в
странах, где марксизм официально господствует десятилетиями, его приходится
удерживать только силой штыков, вернее, танков. Не отрицая важного значения
борьбы, полемоса, и наличия беспорядка, хаоса во Вселенной, мы не вправе
считать обязательным постулат, что вначале не было ни Космоса (красоты), ни
Логоса. И намек на синтез мы вправе видеть уже в том, что один из древнейших
мудрецов, Гераклит Эфесский (а учеником Гераклита был Кратил, один из первых
учителей Платона), в своем учении говорил и о Логосе, как основе бытия, и о
Полемосе, как движущем начале Бытия.

б) Определение религии и религиозных людей

Но тут можно указать, что сам Эйнштейн так определяет религию и религиозных
людей, что практически исключает необходимость введения понятия Логос. В
качестве близких ему мыслителей он приводит Демокрита и Спинозу. Но ведь «линия
Демокрита» всегда противополагалась «линии Платона» как защита материализма и
отрицание религии. Спинозу же всегда трактуют как вежливого или несознательного
атеиста. Все это связано, конечно, с философией детерминизма. Здесь, как мы
знаем, Эйнштейн, в отличие от подавляющего большинства современных физиков,
продолжает стоять на позиции Лапласа: принятие одной единственно возможной
истории всего мирового процесса.

Наличие подавляющего большинства никогда не является аргументом в пользу того
или иного воззрения, но привязанность Эйнштейна
к детерминизму есть лишь одно лишнее доказательство того, что даже величайшие
умы оказываются бессильными преодолеть некоторые убеждения чувств. Эйнштейн
свою верность детерминизму связывает с нежеланием отказаться от принципа или,
как он называет в другом месте, гипотезы причинности, но ведь причинность-то
бывает разная. Классическое схоластическое, восходящее к Аристотелю понимание
причинности обнимает причинность четырех родов. Если мы наталкиваемся на
красивое здание, то задаемся вопросом: 1) какая конечная причина здания, т. е.
какая цель постройки и кто поставил эту цель; 2) какая формальная причина: кто
составил план постройки (архитектор); 3) какая материальная причина: из каких
материалов здание сооружено и 4) наконец, кто фактически построил здание:
рабочие, прорабы и проч. Было время, когда такой же широкий подход был и в
отношении произведений природы, т. е. организмов, и когда конечные причины
играли важную роль. Одно из значений материализма --- отрицание конечных причин,
значит, материализм имеет смысл не утверждения причинности, а ограничения ее.
На практике при постройке зданий и сейчас имеют вполне реальное, а отнюдь не
какое-то «мистическое» значение все четыре рода причин: 1) организатор, 2)
архитектор, 3) материалы и 4) работники. В нашей повседневной жизни мы часто
ограничиваемся формулированием конечной цели нашего поведения (я пришел к вам,
чтобы пригласить завтра пойти на экскурсию, а не потому, что мои мускулы
работали так, что я в силу единственно возможного хода мировой истории должен
был прийти к вам), и отнюдь не нелепо мнение, что дальнейшее развитие биологии
приведет к восстановлению вполне реальных конечных причин. Поэтому нельзя
утверждать ни того, что признание принципа причинности обязательно связано с
детерминизмом, ни того, что признание всех четырех форм причинности требует
обязательно индетерминизма на каком-либо уровне. Возможных мировоззрений
гораздо больше, чем обычно думают.

в) О науке и этике

Эйнштейн защищает науку от обвинения в безнравственности, так как детерминизм,
который Эйнштейн почему-то связывает обязательно с наукой, как бы освобождает
человека от ответственности, и представление о карающем и награждающем Боге
наука отвергает. Но он считает, что нет ничего божественного в морали, это
чисто человеческое дело. Выходит, что этика не связана ни с наукой, ни с
религией. А с чем же она связана? По мнению Эйнштейна, «этическое поведение
человека должно быть эффективно обосновано на симпатии, воспитании и социальных
связях; религиозная основа не нужна». А в чем же тогда роль религии? Религия в
самом широком смысле, по Эйнштейну, является вековым стремлением человечества
получить совершенно ясное и полное осознание сверхперсональных ценностей и
целей и постоянно усиливать и расширять их эффект. При таком понимании не может
быть конфликта между религией и наукой, так как наука устанавливает то, что
есть, но не то, что должно быть. Но установление целей есть тоже область этики
и притом высшей этики. На синайских
скрижалях были записаны почти исключительно ограничения человеческой
деятельности: не сотвори кумира, не работай в субботу, не кради, не убий, не
прелюбодействуй, не лжесвидетельствуй, не завидуй, --- и ничего не говорится о
целях человеческого поведения. В Нагорной проповеди, наоборот, нет осуждения
преступлениям, а восхваляется стремление к благочестивым целям: алкание правды,
миротворцы, кроткие, чистые сердцем, изгнанные правды ради, нищие духом,
подвергающиеся преследованиям за правое дело. И для достижения совершенства
необходимо самоотречение. «Если хочешь быть совершенен, раздай имущество свое
нищим и вслед за мной гряди». «Больше сея любви никто же имать, да кто душу
свою положит за други своя». Дальнейшее развитие этики мы имеем в замечательной
формальной этике Канта, ничего не говорящей о конкретных поступках. «Поступай
так, как если бы правила твоего поведения могли стать всеобщим законом
природы». «Человеческая личность может быть только целью, но отнюдь не
средством». Вот уж тут без науки никак не обойтись. Надо поразмыслить о целях
мирового процесса и действовать так, чтобы твое поведение в той или иной
степени содействовало этому процессу и мешало процессу, так сказать, сойти с
рельсов. При такой высокой оценке науки в деле выяснения морального поведения
человека, неудивительно, что Кант пытался построить религию в пределах одного
разума, не этику базировать на религии, а наоборот, исходя из определенных
этических норм прийти к обоснованию религии.

Попытки обосновать этику на научных, правильнее, рациональных началах восходят
к глубокой древности. Великая философская триада Эллады: Сократ, Платон и
Аристотель, несмотря на расхождения по другим вопросам, была едина в том, что
разум есть сам по себе высшая, а может быть, единственная добродетель человека,
и истинно разумный человек не может быть безнравственным.

В древнем Израиле два главенствующих богословско-философских направления ---
фарисеи и саддукеи --- связывали отношение к человеческим поступкам, в частности
преступлениям, со своими философскими взглядами. Саддукеи (куда относилось, как
правило, высшее духовенство и аристократия) были индетерминистами и поэтому
считали человека полностью ответственным за свои поступки: отсюда полное
соблюдение суровых законов Моисея, очень часто осуждавших на квалифицированную
смертную казнь --- побиение камнями (вспомним «Грешницу»). Фарисеи были
детерминистами и потому считали возможным быть более снисходительными, раз
человек в сущности является игрушкой неведомых сил. Они стремились истолковать
старый принцип «око за око, зуб за зуб» так, чтобы возмездие не могло по своему
характеру превосходить преступления и в значительной степени под их влиянием
применение смертной казни значительно сократилось. Конечно, связь между
детерминизмом и милосердием вовсе не является обязательной. В христианстве мы,
пожалуй, имеем обратное: детерминистами являются пуритане и к нему склонны
доминиканцы --- защитники более сурового правосудия, напротив, наиболее
снисходительными оказываются молинисты --- иезуиты, в наибольшей степени
защищающие постулат свободной воли.

На уровне обычной житейской этики вполне имеет право на существование
утилитаризм, обосновывающий этику на выгодности быть моральным. Крайняя степень
утилитаризма --- этика Молчалина, угождавшего всем, вплоть до собаки дворника.
Почему же утилитаризм пользуется и малым распространением, и незавидной
репутацией? Потому что в природе человека заложены такие стремления и чувства,
которые приводят к неодинаковому для всех людей пониманию счастья. Есть люди,
любящие мирное житие, а другие без драки не могут себе представить
существования. Одни не гонятся за славой, а другие, подобно Герострату, готовы
положить свою душу не «за други своя», а только за то, чтобы получить более или
менее значительную известность. Такие геростратовские эмоции чисто рациональным
путем подавить невозможно, и вот тут приходит на помощь этика, основанная на
религиозных императивах.

Почему же в настоящее время (и этого придерживается и Эйнштейн) так
распространено мнение, что наука и этика имеют совершенно различное
обоснование? Мне думается, это является следствием того, что были попытки
обосновать этику на научных основаниях, и эти попытки вызвали естественный
протест. На основе постулата ведущей роли естественного отбора, принимаемого и
сейчас очень многими за бесспорную истину, оправдывались многие отвратительные
явления человеческого общества: истребление «низших» рас (В. Зайцев, Писарев),
эксплуатация экономически слабых, свобода от всяких этических норм для высшей
расы, сверхчеловека (Ницше, К. Гамсун, нацизм), рекомендация стимулировать
размножение элиты (Р. Фишер) и т. д. Не будучи в состоянии разбить основу такой
этики, селекционизм, лица, не приемлющие выводов социал-дарвинизма,
предпочитают восстановить положение о независимости науки и этики. А поскольку
социология, политика теснейшим образом связаны с этикой, то делались и
квазинаучные попытки видеть цель человечества в создании общества, подобного
термитнику или муравейнику с полным подавлением человеческой личности, и надо
сказать, что такие попытки проводились и не только фашистами.

Как указал наш выдающийся современник Тейяр де Шарден, нельзя отрицать того,
что в будущем совершенном строе человеческого общества будет существенный
элемент тоталитаризма, но первые две попытки тоталитаристических обществ
(Гитлера и Сталина) следует признать бесспорно неудавшимися. Идеал: «полная
свобода частей при совершенном единстве целого», и следует не жалеть усилий по
разработке основ этого идеального общества, где социология и этика будут
связаны в одно гармоническое целое. И здесь следует вспомнить великого Спинозу
не за его детерминизм, а за самую попытку построить этику геометрическим
методом, т. е. так, чтобы она вся была единым целым, лишенным внутренних
противоречий. Легко показать, что некоторые этические системы легко
опровергаются выводами из них самих. Возьмем, например, доктрину крайнего
индивидуалистического анархизма: «все дозволено, никаких этических норм: жги,
убивай и проч.». Эта доктрина легко опровергается при принятии
трудноопровержимого постулата: творец любой этической системы не вправе
оспаривать применимость этой системы к нему самому. Раз дозволено убить
человека, значит, дозволено убить и данного анархиста.

Вряд ли можно считать справедливым и мнение Эйнштейна о том, что «этическое
поведение человека должно быть эффективно обосновано на симпатии, воспитании и
социальных связях». Если основывать на социальных связях (классовых,
национальных, расовых, религиозных), то никак нельзя выработать
общечеловеческую этику, благоприятствующую прогрессу человечества, так как
классовые этики, например, будут оспариваться на основе постулата о
незыблемости классовой структуры, следовательно, все такие частные этики и их
основы должны быть обязательно пересмотрены, ревизованы.

Несомненно, и симпатии могут быть предметом воспитания. Браки в крайних
степенях родства (родные братья и сестры) существовали кое-где (указания в
«Одиссее», династия Птолемеев в Египте), но, возможно, на основе появления
частых дефективных детей при тесных браках (примитивная генетика) большинство
народов не только отвергает такие браки, но и относится к ним с чрезвычайным
отвращением.

г) Об антагонизме науки и религии

Сам Эйнштейн не остается на своей позиции независимости науки и религии,
напротив, он защищает тезис: «наука без религии хрома, религия без науки слепа»
и считает, что все конфликты прошлого основаны в общем на недоразумении и что
их могло и не быть (как это при принятии им абсолютного детерминизма и
единственности хода мировой истории они могли бы не быть?). Здесь мы
сталкиваемся с общим вопросом при оценке тех или иных исторических событий и
всего мирового процесса. Вся история полна конфликтами, часто носящими ужасный
характер. Могла ли быть история без ужасных конфликтов? Есть философы, которые
считают, что наш мир --- наилучший из возможных (Лейбниц), в поэтической форме ---
Александр Поп (все, что существует, справедливо), и лучшей истории мира, чем
наша, не придумаешь. И насилия необходимы: «насилие есть повивальная бабка
истории» (Маркс). Но если мы откажемся от принятия единственного хода мировой
истории, то придем к заключению, что история могла бы быть лучше
действительной. Мы видим, что сходные проблемы в разных государствах решаются с
применением неодинакового количества насилия, и следовательно, в ряде случаев
исторические деятели перестарались без нужды. Я полагаю поэтому, что мыслима
прогрессивная история без грубого насилия и к построению именно такого хода
мирового процесса мы и должны стремиться, но это не значит, что мыслим прогресс
без всякой борьбы и без всяких конфликтов. Мы знаем хорошо, что конфликты имели
место не только между такими обширными областями человеческой культуры, как
наука и религия, но и в пределах науки между консервативными и прогрессивными
учеными. Ведь главными противниками гелиоцентрической системы были как раз
перипатетики. Консерватизм в области мышления, как и во всех других областях,
не является ни абсолютным злом, ни абсолютным добром. В умеренной степени он
предохраняет
от чрезмерно быстрых и неоправданных перестроек. В крайней степени ---
задерживает прогресс. Соблюсти нужную норму консервативности учит нас история
вообще и история науки в частности, и мы можем с удовлетворением
констатировать, что в области точных наук вредный консерватизм исчез, но и это
положительное состояние было достигнуто только после долгой борьбы.

Другим источником конфликтов было (и в известной степени остается всегда)
взаимодействие разных отраслей культуры и быта, роль темпераментов и проч.
Крайнее развитие пифагорейского направления приводит к игнорированию запросов
повседневности, что, естественно, вызывает протест и способствует развитию
непифагорейских видов религии или протесту против религии вообще. В известном
прологе к трагедии «Фауст» Гете архангелы на небе восторженно поют хвалу Богу и
говорят, что все его творения прекрасны, как в первый день. Но Мефистофель
резонно указывает, что на Земле не все уж так прекрасно, и симпатии гуманного
человека невольно склоняются на сторону духа зла, который, может быть только в
пику архангелам, проявляет большее сочувствие страждущему человечеству, чем не
замечающие этих страданий архангелы. «Праведных души не знают ни скорби, ни
злобы» (А. К. Толстой). И вот это отсутствие скорби у торжествующих и вызывает
естественный протест, часто идущий чрезмерно далеко.

С другой стороны, при эволюции религий, так же как и при эволюции организмов,
наблюдались чисто регрессивные девиации. Совершенно прав Эйнштейн, что одной из
основных эмоций была эмоция страха, но она могла иметь разные выражения в
религии. Один страх --- страх перед грозным и жестоким деспотом, которого нужно
умилосердить жертвами, вплоть до человеческих. Индусская богиня Кали,
финикийский Молох и многие другие божества требовали человеческих жертв. На
базе такого страха вряд ли возможна эволюция. Но есть другая форма страха ---
страх Божий, боязнь оскорбить строгого, но праведного Судию. В великолепном
рассказе Лескова «На краю света» показана высокая мораль некрещеного язычника ---
зырянина, который не только спас епископа, но, чтобы рассчитаться с неизвестным
ему хозяином за взятый им медвежий окорок, оставил в залог свою шапку, а сам по
морозу шел с непокрытой головой. Епископ спрашивает, зачем же он оставил
хозяину шапку. «Чтобы он дурно, бачка, не думал». --- «Да ведь этот хозяин тебя
не знает». --- «Этот, бачка, не знает, а другой знает». --- «Который другой?» --- «А
тот хозяин, который сверху смотрит... Он, бачка, не любит, кто худо делает».
Тут мы видим переход от религии страха к религии высокой морали. Что такое
худо? И что хорошо? Вопрос поставлен во всю широту в малоизвестном диалоге
Платона «Евтифрон», где Сократ спрашивает (10, А): «Потому ли боги любят
благочестивое, что оно благочестиво, или оно благочестиво потому, что его любят
боги?» Критерий добра и зла уже поставлен выше компетенции богов и подчинен
человеческому разуму, но это такая трудная задача, что решить ее не удалось и
по настоящее время.

8. Заключительные замечания

Я взял для разбора мысли одного Эйнштейна не только в силу высокого его
авторитета, но и потому, что по многим своим высказываниям он справедливо
считается близким к материализму и его пантеизм спинозовского типа многими
может быть сочтен за разновидность атеизма. Существенно то, что настоящие
атеисты лишены космического религиозного чувства, которое и является основой
для веры в исконную гармоничность природы, так сильно защищаемую Эйнштейном.
Это космическое религиозное чувство сопровождало и направляло первые шаги
настоящей, чистой науки, ставившей себе целью не удовлетворение текущих
потребностей, а проникновение в тайны природы. Первая точная наука --- астрономия
имела своих предшественниц в лице астролатрии и астрологии (см.
соответствующие статьи в старом Энциклопедическом словаре Брокгауза и Ефрона).
Астролатрия --- поклонение звездам. Как указано в статье «Астрология», «почти у
всех народов, достигших некоторой степени научного знания о небе, мы встречаем
астрологию как необходимую переходную стадию от астролатрии к астрономии.
Астрология, алхимия и магия тесно связаны между собой, представляя ступени
развития мысли человека, через которые оно, по-видимому, должно было непременно
пройти, прежде чем достичь научной астрономии, химии и других физических наук».

Лица, придерживающиеся доктрины исторического материализма, полагают, что
источником астрономии были практические потребности. Да, для ориентировки при
передвижении по земле, для счета времени необходимы некоторые знания о
расположении созвездий и пр., и эти знания имеются у примитивнейших народов и
остаются на примитивном уровне неограниченное время, не показывая ни малейших
признаков превращения в настоящую науку. Пытались связать изучение звезд с
сельскохозяйственной практикой: например, в «Георгиках» Вергилия есть указания
на то, при каком расположении звезд следует начинать посев и возделывать почву,
но я что-то не слыхал, чтобы эти «практические» указания получили подтверждение
и дальнейшее развитие в современной агрономии. Но все примитивные (и не только
примитивные) народы обожествляли Солнце, Луну и прочие светила: в их
неизменности видели проявление божественного начала. Поэтому первые
астрономические таблицы и прочие развивались жрецами Вавилона или Халдеи.
Марксистский историк астрономии Паннекук (История астрономии, 1966) посвящает
пять глав вавилонской и ассирийской астрологии и по отношению связи астрономии
с земледелием говорит (с. 25): «Можно задать вопрос, для чего была необходима
такая точность, если она намного превосходила потребности земледелия,
зависящего от колебаний погоды и носящего поэтому нерегулярный характер. Однако
следует иметь в виду, что в те времена сельскохозяйственные работы
сопровождались религиозными церемониями». «В богослужении не допускалось ни
малейшей небрежности и требовалось точное соблюдение ритуала. Хронологической
основой ритуала являлся календарь». Мы знаем, что связь календаря с религией
сохранилась и в христианстве и недаром современный календарь называется
григорианским, так как он был введен во всех католических
странах (позже его ввели и протестантские и даже атеистические страны) по
приказу папы Григория XIII в 1582 г., осуществившего постановления Тридентского
собора.

Пока астрономия была тесно связана с астролатрией и находилась под
покровительством религии, ей не было надобности в защите светских властей. Но
связанные с наблюдениями над божественными светилами жрецы нашли и
периодичность затмений (в первую очередь лунных), обратили внимание и на связь
небесных явлений с земными, прежде всего на влияние Луны и Солнца на приливы и
прочее (менструации, лунатики), а в связи с широко распространенным мнением
(выраженным и Гомером), что судьба человека определена в момент его рождения
(Лаплас пошел гораздо дальше --- она определена в расположении атомов
первичного хаоса), легко перейти к мысли, что будущее может быть прочтено в
звездах. Родившаяся астрология покровительствовала изучению светил и исканию
закономерностей и вместе с тем она приобрела «практический характер»,
заинтересовавший и светских властителей. Поэтому в Риме, где чистая наука была
не в почете, астрология процветала в противоположность античной Элладе.
Императоры боялись астрологов (их называли халдеями и иногда математиками) и
порой изгоняли их (Клавдий, Веспасиан, Диоклетиан, Юстиниан), но не
отказывались от их услуг: тот же Веспасиан, Тит, Домициан, а при Тиберии был
штатным астрологом Тразилл, собравший все труды Демокрита и Платона.
Наибольшего расцвета астрология достигла в XIII--XVI вв., и творцы новой
астрономии Коперник и Кеплер не чуждались астрологии. Самому Кеплеру астрология
давала средства к существованию (он был штатным астрологом Валленштейна) и
осуществлению его гениальных работ. Он это ясно сознавал: «Астрология есть,
конечно, глупая дочка, но Боже мой, что бы стало с ее многоразумной матерью
астрономией, если бы она не имела этой дочки, ибо и вообще жалованье
математиков так невелико, что мать, наверное, страдала бы от голода, если бы
дочь ее не зарабатывала что-нибудь» (из той же статьи «Астрология»). Из слов о
«глупой дочке» вовсе не следует, что Кеплер всю астрологию считал глупостью. На
этот путь стал Галилей с тем результатом, что отверг не только астрологические
элементы в учении Кеплера, но и его новые законы. Об этом в главе о Галилее
сказано достаточно. Ньютон восстановил «рациональное зерно» астрологии ---
притяжение небесных тел и создал, как известно, великолепный синтез, долгое
время считавшийся абсолютной истиной в последней инстанции.

Таким образом, не только общее космическое религиозное чувство стимулировало
развитие точного знания, но даже ошибки, несбыточные мечты, не только не
мешали, а помогали развитию науки. «Если к правде святой мир дорогу найти не
сумеет, честь безумцу, который навеет человечеству сон золотой». «Золотой сон»
в буквальном смысле (мечта добыть золото из простых металлов) в значительной
мере руководил и предшественниками (а вовсе не антагонистами) научной химии
алхимиками, и в погоне за этим сном они создали огромный эмпирический фундамент
научной химии. Поэтому утверждение о том, что религии любых сортов всегда
противодействовали развитию науки, есть просто невежественный вздор. Но многие
мыслители полагают, что, сыграв в свое время положительную роль, религия
исчерпала себя, что сейчас наука ни в какой религии и ни в какой философии не
нуждается или, как думают марксисты, она нуждается в философии диалектического
и исторического материализма. Пример Эйнштейна показывает, что и это положение
далеко не бесспорно, но сначала разберем, какую пользу и какой вред причинили
науке разные формы антирелигиозной, атеистической и антиклерикальной идеологии.

Глава 4. ПОЛЬЗА И ВРЕД АНТИРЕЛИГИОЗНЫХ НАПРАВЛЕНИЙ ДЛЯ НАУКИ

1. Торжество механицизма. Роль Ньютона

Развитие науки шло до XVIII в. под идеалистическим влиянием вплоть до Ньютона.
Мощные механистические, материалистические, а затем антиклерикальные и
атеистические влияния особенно сильно проявились в XVIII в., веке Просвещения,
и с тех пор развитие науки идет все ускоряющимися темпами параллельно с ростом
разнообразных течений, противных идеализму. В этом многие видят доказательство
монополии материализма как эвристического орудия. Но сопутствие двух явлений
вовсе не обязательно доказывает причинную связь этих двух явлений.

Мы знаем, что в России мощный подъем культуры во всех областях (наука,
литература, музыка) характеризует первую половину XIX в., время царствования
Александра I и Николая I. Вряд ли мы можем их считать обоих покровителями
культуры (пожалуй, это можно допустить для Александра I в отношении архитектуры
и Николая I в отношении астрономии --- постройка Пулковской обсерватории):
последняя половина царствования Александра прошла под знаком Аракчеева и
Магницкого, а Николай I больше всего заботился о военной выправке своих солдат.
Еще любопытнее пример можно извлечь из истории Англии: Бокль правильно
указывает, что Карл II --- один из наиболее неудачных королей Англии, и однако
его царствование ознаменовалось крупным прогрессом в деле усиления политической
свободы в Англии.

Необходимо, кроме того, делать различие между механистической методологией и
материалистическим мировоззрением. Одним из крупнейших проводников
механистической методологии был Декарт, но его никак нельзя назвать
материалистом, так как он был ярко выраженным дуалистом. Из бесспорнейшего
факта сознания он выводил существование Бога, но вне человека старался все
подчинить чисто механическим законам. Механистический подход к явлениям не
только неорганического мира, но и органической природы, законен и необходим.
Ошибка заключается в утверждении, что он является единственно допустимым в
науке подходом и в абсолютизации гипотез, имеющих определенную, но ограниченную
ценность. Вред от того или иного мировоззрения заключается всегда в
догматизации его.

Материалистическое мировоззрение является чрезвычайно древним, оно даже
наиболее древнее: первобытный человек мыслит материалистически: для него и Бог
и душа --- материальные образования. Так мыслил и далеко не первобытный
Тертуллиан.

Давал ли толчок к научному исследованию античный материализм в лице Эпикура,
Лукреция? Нет, как показано в первом параграфе первой главы, они стремились
лишь к созданию «объяснений», делавших ненужным какое-либо сверхъестественное
непонятное вмешательство. От более раннего Демокрита остались, как известно,
только фрагменты. Это был несомненно крупный мыслитель, но ни он, ни
представители материализма --- Эпикур и его последователи --- никакого участия
в развитии гелиоцентрической системы мира не принимали: даже представление о
шарообразности Земли (выдвинутое впервые Пифагором) было им чуждо. Все
прогрессивное развитие астрономии вплоть до Ньютона совершенно не связано с
материализмом. Результат работы Ньютона производил на ученых впечатление полной
достоверности вплоть до появления Эйнштейна. Вот как его характеризовали в
статье «Астрономия» (Энц. слов. Брокгауза и Ефрона, с. 392, относится к 1890
г.): «В настоящее время закон Ньютона может считаться не подлежащим ни
малейшему сомнению и трудно сказать какие-нибудь научные обобщения,
достоверность которых может сравниться с достоверностью закона тяготения». Сам
Ньютон более самокритически относился к собственным достижениям и считал, что
то, чего он достиг, во-первых, составляет ничтожно малую часть возможного
знания, а во-вторых, что он достиг своего лишь став на плечи своих великих
предшественников. Но так как он дал мощный толчок развитию механики, то
естественно, что его последователи, не удержавшись на критической высоте своего
учителя, попытались сделать из механизма мировоззрение и распространить
механизм на все области бытия. Это не значит, что все его последователи так
поступали. Как это ни странно, но сторонники астрологии отнюдь не считали, что
с Ньютоном их псевдонауке пришел конец. «Первые таблицы планет, вычисленные
Руже-де-Лиллем на основании теории тяготения Ньютона, были в то же время
последними астрологическими таблицами, так как они были составлены ввиду
вычисления прогнозов» (статья «Астрология», Энц. слов. Брокгауза и Ефрона).

Также с установлением теории тяготения Ньютона не исчезли и представления о
гармонии во Вселенной, и на основании этих представлений был сделан даже
удачный прогноз. Я имею в виду так называемый закон Тициуса--Бодэ о планетных
расстояниях в Солнечной системе. Вопросом планетных расстояний, как я уже
указывал, занимался всерьез и Кеплер. У Тициуса--Бодэ был чисто эмпирический
подход, но при этом оказался один незаполненный промежуток между Марсом и
Юпитером. Бодэ указывал, что в этом промежутке должна находиться планета для
пополнения гармонии в распределении тел Солнечной системы, и такая планета была
найдена: вернее, целый рой мелких планет, занимающий указанный им промежуток
(статья «Астрономия», Энц. слов. Брокгауза и Ефрона).

Основным результатом великого дела Ньютона в методологическом смысле было,
помимо стимулирования механического понимания природы, утверждение старого
убеждения, что наука состоит в добывании абсолютно достоверных истин и теорий,
«окончательных истин в последней инстанции», по выражению Е. Дюринга. Именно
эта достоверность науки в противоположность противоречивости и разнообразию
религиозных доктрин и делала науку чем-то принципиально отличным от религии.
Конечно, сам Ньютон так дела не понимал. Окончательная достоверность многого
того, что нам уже известно, принималась и такими выдающимися философами, как
Кант, который отличал аналитику чистого разума как учение о достоверном, от
диалектики как учения о противоречивом (антиномии). Критичность Канта и
заключается в том, что он не считал возможным получить абсолютно достоверное
знание об основных сущностях бытия. В таком самоограничении многие не в меру
усердные последователи Ньютона видели проявление робости или уступки
общественному мнению и склонны были считать всякий агностицизм «стыдливым
материализмом». Поэтому среди ученых XVIII и XIX вв. были широко распространены
две установки: 1) всякая поддержка религии является вредной для науки, и если
какой-либо ученый высказывает гипотезу, как бы поддерживающую то или иное
религиозное (в особенности христианское) учение, то этого одного уже
достаточно, чтобы отвергнуть эту гипотезу; 2) многие положения науки обладают
абсолютной достоверностью, и это решение уже окончательно. Коснемся этого
несколько более подробно.

2. Просветительный вандализм

Это выражение принадлежит оригинальному ученому Хладни, сумевшему преодолеть
сопротивление ученых, не допускавших возможности существования метеоритов
(Фесенков В. Г. Современные представления о Вселенной, 1949, с. 124).
«Недоверие общества, вернее, его наиболее просвещенной части (дело шло в конце
XVIII и начале XIX в. --- \emph{А.Л.}) заходило настолько далеко, что, по
словам Хладни, большая часть старинных метеоритов, хранившихся в общественных
собраниях и церквах, была из них выброшена людьми из опасения быть ославленными
как невежды, поддерживавшие вредное суеверие. Хладни перечисляет целый ряд
метеоритов, упавших в разных странах и в разное время, которые были выброшены и
уничтожены, как он выражается, из-за „просветительного вандализма"». Все дело в
том, что в Библии упоминается о «каменном дожде», сыгравшем даже роль в одном
из сражений, и всякое подтверждение библейской легенды считалось уже
обскурантизмом.

Выдающийся деятель просвещения, Вольтер, заслуженно пользующийся репутацией
борца за истину (реабилитация невинно погубленных религиозными фанатиками
людей), в суждениях по научным вопросам не избежал просветительного вандализма.
В его время в палеонтологии уже стало господствовать правильное суждение о том,
что ископаемые остатки прежде живущих организмов являются свидетелями того, что
там, где сейчас суша, в раннее время было море. А так как морские раковины
находились в Альпах, то, значит, на уровне Альп было море. Так ведь это как
будто доказательство Всемирного потопа (подымания горных цепей Вольтер в расчет
не принимал). Чтобы не дать церковникам права использовать эти факты для
подтверждения библейской легенды, Вольтер настаивал, что раковины были случайно
обронены пилигримами, переходившими Альпы в своем путешествии в Рим.

Не всегда аргументы Вольтера были так наивны. По поводу совместного нахождения
костей тигра, бегемота и северного оленя он говорил, что невозможно выдумать
климат, где бы жили одновременно такие разнородные животные: житель тропиков
тигр, житель тундр северный олень и житель болот и водоемов бегемот. Проще
допустить, писал Вольтер, что эти кости попали в одно место от какого-нибудь
старьевщика. Во времена Вольтера его аргументация звучала довольно
убедительно, так как современные ему французы, видимо, и не подозревали, что
самые крупные тигры и сейчас водятся в снегах Восточной Сибири и что в
Уссурийском крае тигр может охотиться за северным оленем. Найден сейчас в
Африке и карликовый бегемот, менее связанный с водой, чем его крупный
популярный сородич.

Другая ошибка Вольтера имела более серьезный характер. В его время обсуждался
вопрос о едином или множественном происхождении человека: дело шло не об
эволюционной теории, а о том, имеет ли человек одну пару прародителей или много
--- гипотезы моно- и полигенизма. Вольтер решительно встал на сторону
полигенизма просто потому, что учение моногенизма поддерживалось Библией:
утверждение, что все люди --- потомки одной пары --- Адама и Евы. Разумеется,
можно быть полигенистом и не быть расистом, но и в то время, и в наши дни
полигенизм особенно усердно защищают люди, утверждающие, что «низшие расы»
имеют совсем иное происхождение и не заслуживают такого же обращения, как
настоящие люди. Напомним, что глава католической церкви, один из лидеров
«католической реакции» папа Павел III решительно высказался за то, что
представители абсолютно всех человеческих рас --- настоящие люди, достойные
быть христианами и могущие заслужить вечное блаженство, --- мощный удар по
расизму, сохраняющий свою силу даже в настоящее время.

Аргументация «от поповщины» была широко распространена в начале XX в. и в конце
XIX в. Наш выдающийся ученый ботаник академик И. П. Бородин выступил с
публичным докладом в защиту витализма, учения, которое считалось «окончательно
опровергнутым» большинством биологов. Вместо критических возражений по существу
профессор (потом тоже академик) В. М. Шимкевич счел достаточным, чтобы громко
сказать: «Значит, по-старому, миром Господу помолимся».

Не сумел сохранить позицию объективного ученого и Фесенков в упомянутой уже
книге. В примечании к с. 252 он пишет:

«Из этого явления красного смещения идеалистически настроенные зарубежные
астрономы делают зачастую нелепые, ни на чем не основанные выводы, которые не
заслуживают даже опровержения. Упомянем, например, аббата Леметра, сделавшего
научную карьеру доказательством неустойчивости Вселенной по Эйнштейну, которое
в исчерпывающей форме было найдено за восемь лет до того советским математиком
А. А. Фридманом. Леметр использует эти теоретические построения как
доказательство сотворения мира несколько миллиардов лет тому назад и
фантазирует, что первоначально Вселенная была сотворена в виде единственного
атома, заключавшего всю ее массу порядка $10^{54}$ г, соответствующую по размерам
расстоянию Земли от Солнца. Этот атом, по мнению Леметра, несколько миллиардов
лет назад взорвался и дал начало современным галактикам...

С другой стороны, это же самое явление красного смещения, согласно Бонди, Голлу
и Хоулу, вовсе не указывает на начало Вселенной. Напротив, по мнению этих
авторов, творение материи из ничего происходит будто бы в природе и в настоящее
время.

На этом примере видно, что все подобные заключения обусловлены исключительно
идеалистическими настроениями их авторов и по существу не имеют никакого
отношения к науке».

Сам Фесенков считает, что следует, по-видимому, признать реальность разбегания
внегалактических туманностей, но он, естественно (для книги, напечатанной в
1949 г.), не указывает, что в идеализме обвиняли само утверждение о разбегании
и, конечно, работу А. А. Фридмана. При этом, однако, обходили молчанием вопрос,
как справиться с таким противоречием, далеко не единственным противоречием
космологии конца XIX в. С одной стороны, признавалась абсолютная значимость
2-го закона термодинамики, который приводит к «тепловой смерти» Вселенной, а с
другой --- также абсолютную значимость постулата о бесконечности пространства и
времени. Если Вселенная существует бесконечно и в ней постоянно действует
второе начало термодинамики, то Вселенная давно должна бы прийти в состояние
тепловой смерти. Как я уже указывал в разделе 5 главы 1, Больцман пытался выйти
из затруднения гипотезой, что мы живем в совершенно невероятном участке
Вселенной. Указанные две теории, однако, дают выход из затруднения. Теория
Леметра (ее обычно считают теорией русского эмигранта Гамова) есть теория
внезапного творения, теория Бонди и проч. --- теория непрерывного творения, и
современные астрономы, как указывает, например, Барбоур (1966, с. 366 и 367), и
придерживаются одной из этих двух теорий. Теория внезапного (мгновенного)
творения принимает конечность времени, теория непрерывного творения
ограничивает закон сохранения энергии-материи, но в обоих случаях получается
примирение со вторым началом термодинамики. Как раз гипотеза непрерывного
творения избегает метафизических и теологических постулатов, а некоторые
христианские авторы считают, что гипотеза мгновенного творения более
согласуется с библейским учением о творении. Если так широко понимать
«согласование», то можно сказать, что гипотеза непрерывного творения
соответствует учению Августина о творении «в потенции». Правильнее будет
сказать, что никакая натурфилософия не может избежать априорных постулатов,
которые приходится принимать на веру, и вера в бесконечность пространства и
времени никаких преимуществ перед другими не представляет. Лично я не могу себе
представить актуальную бесконечность материи, и безграничность (но не
бесконечность) Вселенной мне кажется априорно совершенно очевидной.

3. Отношение марксизма к вопросу

Если мы возьмем классиков марксизма, в особенности Энгельса, оставившего такие
труды, как «Анти-Дюринг» и «Диалектика природы», то легко можем найти ряд
указаний на их сочувствие полной свободе мысли. Энгельс решительно возражает
Дюрингу в попытках последнего установить «окончательные истины в последней
инстанции». В предисловии ко второму изданию «Анти-Дюринга» (с. 9) Энгельс
принимает то правило, что «по отношению к своему противнику я обязан не
исправлять ничего, раз он ничего не может исправить» и далее: «Впрочем, я тем
более должен соблюдать по отношению к нему все правила чести, принятые в
литературной борьбе, что уже после издания моей книги берлинский университет
поступил с ним постыдно несправедливо. Правда, университет был за это
достаточно наказан. Университет, который идет на то, чтобы при известных всем
обстоятельствах лишить г. Дюринга свободы преподавания, не вправе удивляться,
если ему, при столь же известных всем обстоятельствах, навязывают г.
Швенингера».

Эти слова Энгельса заслуживают особого внимания. Дюринг --- один из главных
противников Энгельса, он был подвергнут им жесточайшей критике, и однако
Энгельс считает необходимым сохранить за Дюрингом не только свободу печатания,
но и свободу преподавания! Как далек он от многих «марксистов».

В главе IX (Мораль и право, Вечные истины) в споре с Дюрингом Энгельс в вопросе
о суверенности человеческого мышления пишет прекрасные слова о мышлении, что
оно: «Суверенно и неограничено по своей природе, призванию, возможности,
исторической конечной цели; несуверенно и ограничено по отдельному
осуществлению, по данной в то или иное время действительности». Дальше в
классификации наук он указывает, что именно в области точных наук «кануло в
вечность девственное состояние абсолютной значимости и неопровержимой
доказанности всего математического» и что как раз в астрономии, физике и других
науках «окончательные истины в последней инстанции становятся здесь с течением
времени удивительно редкими». Энгельс верно подметил тенденцию точных наук ---
полный отказ от абсолютных истин (раньше называвшихся аксиомами). Мы знаем, что
Лобачевский и Больяи усомнились в абсолютной ценности пятого постулата Эвклида,
а после Римана все аксиомы считают не абсолютными истинами, а просто
положениями, которые берутся за основу построения математики, причем вместо
одной системы аксиом можно взять и другие. Но Энгельс, очевидно, не был знаком
с развитием неэвклидовой геометрии (хотя он писал во времена Гельмгольца,
популяризовавшего эти достижения), так как там же признает, что существуют
вечные истины, окончательные истины в последней инстанции, всякое сомнение в
которых представляется нам равнозначным сумасшествию, и в качестве одной из
таких приводит положение, что сумма углов треугольника равна двум прямым. Но мы
знаем, что в геометрии Лобачевского она меньше двух прямых, а в геометрии
Римана (которая наиболее подходит, видимо, к реальным условиям Вселенной) она
больше двух прямых. Мало того, в характеристике наук о неживой природе,
доступных в большей или меньшей степени математической обработке, он пишет:
«Если кому-нибудь доставляет удовольствие применять большие слова к весьма
простым вещам, то можно сказать, что \emph{некоторые} результаты этих наук
представляют собой вечные истины, окончательные истины в последней инстанции,
почему эти науки и называются \emph{точными}. Однако далеко еще не все результаты
имеют такой характер» (курсив Энгельса. --- \emph{А.Л.}). Мы видим, таким
образом, что Энгельс отличается от Дюринга вовсе не в отрицании окончательных
истин в последней инстанции, а лишь в том, что он признает их меньше, чем
Дюринг. Но связывать обладание вечными истинами с правом для данной науки
именовать себя точной --- крупная ошибка Энгельса, смешивающего понятия
точности и достоверности. «Земля есть замкнутое тело» --- утверждение
достоверное, но не точное. «Земля имеет форму шара» --- утверждение точное, но
не верное, так как на самом деле Земля имеет форму эллипсоида или даже геоида.
Мы измеряем точность весов по их чувствительности (вешает с точностью до
миллиграмма), хотя если весы не вполне равноплечи, результат может быть и
неверный (верный результат получится от двух взвешиваний с переменой чашек), а
правильные весы могут быть весьма неточными. И точными науками называются не те
науки, которые в наибольшей степени обладают вечными истинами, а те, которые
пользуются точными математическими теориями, позволяющими количественно
измерить расхождение между предвидимыми теорией и наблюдаемыми результатами.
Так как в силу ошибок опыта совпадение теоретически предвидимого и наблюдаемого
никогда не бывает абсолютным, то сейчас и принимает большинство ученых точных
наук, что ни одна совершенно строгая теория не может быть абсолютно доказана,
но может быть опровергнута, если расхождение ожидаемого и наблюдаемого слишком
велико. Но удивительно, что, определяя точные науки наличием в них вечных
истин, Энгельс и в третьей, наименее совершенной группе исторических наук
находит вечные истины, в сущности, в неограниченном количестве (с. 84):
«Поэтому, кто здесь погонится за окончательными истинами в последней инстанции,
за подлинными, вообще неизменными истинами, тот немногим поживится, --- разве
только банальностями и общими местами худшего сорта, вроде того, что люди в
общем не могут жить не трудясь, что они до сих пор делились большей частью на
господствующих и подчиненных, что Наполеон умер 5 мая 1821 г. и т. д.» (раньше
он приводил еще пример: Париж находится во Франции). Отметим прежде всего, что
«вечная истина» --- «Наполеон умер 5 мая 1821 г.» сейчас уже оспаривается. В
ряде советских журналов появились весьма аргументированные заметки, что вместо
Наполеона на острове Св. Елены умер какой-то заменивший его двойник. Наполеону
удалось бежать, и умер он неизвестно где и когда. Но если простые факты:
такой-то человек умер тогда-то (вряд ли Энгельс был монархистом и считал, что
только в отношении высоких особ такие факты заслуживают внимания) считать
примером «истины», то ведь таких же и подобных им утверждений можно назвать
бесконечное множество, и так как история каждого государства обладает большим
количеством «вечных истин» (например, смены, даты царствования и проч.
монархов), то по обилию вечных истин исторические науки сильно превзойдут
точные и, следовательно, их тоже следует назвать точными. Ошибка Энгельса в
смешении точности и достоверности уже используется некоторыми нашими не по
разуму усердными марксистами, которые нередко утверждают, что сейчас сгладилась
разница между точными и общественными науками, так как многие положения
марксизма (учение о прибавочной стоимости, теория ведущей роли классовой борьбы
в истории человечества и проч.) уже являются абсолютно точными (т. е.
достоверными) положениями, не подлежащими ревизии. Мне случалось говорить с
одним очень культурным марксистом (Владислав Вильгельмович Рудаш), который
считал, что без признания некоторых наших положений за абсолютные истины, не
может быть обоснован энтузиазм для борьбы с врагом (дело было во время Второй
мировой войны). С этой точки зрения Энгельс был не Анти-Дюрингом, а
Полу-Дюрингом. Полное отрицание познания абсолютных истин хотя, вообще говоря,
и совместимо с материализмом, но встречается гораздо чаще у идеалистов и служит
часто обвинением в идеализме. Наиболее далеко в отрицании абсолютных истин
пошла та группа марксистов, которая примкнула к махистам и эмпириокритикам (А.
А. Богданов, Базаров, Луначарский и др.) и которая подверглась жестокой критике
Ленина. Но не следует думать, что Ленин критиковал Богданова так резко потому,
что тот был большевиком, и что меньшевики относились иначе. Резкую критику А.
А. Богданова давал и Г. В. Плеханов и при этом называл его не товарищем, а
господином. Богданов считал такое обращение со стороны товарища по партии
оскорблением. На что Плеханов отвечал (Воинствующий материализм, Ответ г.
Богданову, письмо первое, с. 342, по сборнику «Против философского
ревизионизма», 1935), что именование Богданова товарищем не входит в число его
социал-демократических обязанностей. «А не товарищ Вы мне потому, что \emph{мы
с вами являемся представителями двух прямо противоположных
миросозерцаний} (курсив Плеханова. --- \emph{А.Л.}). И поскольку речь идет для
меня о защите моего миросозерцания, Вы являетесь по отношению ко мне не
товарищем, а самым решительным и самым непримиримым противником». Ленин к
Богданову, Луначарскому и другим махистам из большевистской партии относился
несравненно более терпимо не потому, что он терпеливо относился к религии, а
потому, что считался с субъективным отношением Богданова к религии. В конце
главы «Партии в философии» «Материализма и эмпириокритицизма», критикуя
Луначарского, Ленин пишет: «Позорные вещи, до которых опустился Луначарский,
--- не исключение, а порождение эмпириокритицизма и русского и немецкого.
Нельзя защищать их „хорошими намерениями" автора, „особым смыслом" его слов;
будь это прямой и обычный, т. е. непосредственно фидеистический смысл, мы не
стали бы и разговаривать с автором, ибо не" нашлось бы, наверное, ни одного
марксиста, для которого подобные заявления \emph{не} приравнивали бы
\emph{всецело} Анатолия Луначарского Петру Струве. Если этого нет (а этого
\emph{еще} нет), то исключительно потому, что мы видим „особый" смысл и
\emph{воюем, пока еще есть почва} для товарищеской войны... Надо быть слепым,
чтобы не видеть идейного родства между „обожествлением высших человеческих
потенций" Луначарского и „всеобщей подстановкой" психического под всю
физическую природу Богданова» (курсив Ленина. --- \emph{А.Л.}). Таким образом,
Ленин не менее непримирим к религии, фидеизму, идеализму, чем Плеханов, но он
более терпелив по отношению к товарищам по партии и считает возможным их
урезонивать, так как понимает и верит им, что субъективно они против религии.
Там, где речь идет не о товарищах но партии, Ленин чрезвычайно груб и
несправедлив. Например, наш выдающийся физик, Хвольсон, написал очень
остроумную и вполне культурную критическую брошюру против знаменитой книги Э.
Геккеля «Мировые загадки». Эту брошюру Ленин называет «подлой черносотенной».
Этот недопустимый тон был отмечен и в рецензиях на книгу Ленина, например Л.
Аксельрод (Ортодокс).

Возвращаясь к Энгельсу, можем себе задать вопрос: можно ли такие утверждения,
как «Париж находится во Франции», «Наполеон умер 5 мая 1821 г.» называть
«истинами». Это --- факты, а не истины, в понятие истины обязательно входит
какое-то обобщение из фактов, а не сами изолированные факты, и критика понятия
«абсолютная истина» заключается в том, что ни одно из положений, претендующих
на звание вечной истины, никогда не бывает одновременно полно, достоверно и
точно, и точные науки отличаются от неточных не тем, что они более достоверны,
а тем, что они знают меру своей неточности.

Но тогда, может быть, факты могут претендовать на значение «вечных» и абсолютно
достоверных? Кто когда-либо может оспорить факт, что Париж находится во
Франции? Да, таких лично я не знаю, но ведь такие утверждения, как «Наполеон
умер 5 мая 1821 г.», «Христос был распят при Понтии Пилате», «Александр I умер
в 1825 г. в Таганроге», «Жанна д'Арк была сожжена по приговору суда» и прочие
тоже не оспаривались в течение многих сотен лет, а сейчас есть мнение среди
компетентных историков, что Александр I сделался старцем Федором Кузьмичом и
умер в Сибири, что Жанна д'Арк была приговорена к сожжению, но не была сожжена.
А наши современные марксисты изо всех сил стараются доказать, что абсолютной
истиной в последней инстанции было не то, что Христос был распят, а то, что
такой личности никогда не было. А для того чтобы какое-то положение было
признано «вечным», нужно, чтобы оно выдержало любое испытание временем. Для
Парижа срок испытания еще не наступил, а что будет через тысячу лет, мы не
знаем. Кажется, никто не сомневается в том, что библейский Иерусалим находится
в Палестине, а вот наш талантливый шлиссельбуржец Н. А. Морозов в своем
удивительно интересно написанном «Христе» доказывает, что библейский Иерусалим
не что иное, как римская Помпея, и что плач Иеремии о разрушенном городе ---
описание вовсе не завоевания города, а разрушения города при извержении. Хотя
меня Морозов не убедил, но я согласен, что приводимый им текст из пророка
Иеремии более соответствует извержению и связанному с ним местному
землетрясению, чем завоеванию города. А если возьмем уже совсем современные
факты, то на картах сталинских времен мы не найдем, например, Элисты, столицы
Калмыкской автономной области, и в энциклопедиях нет упоминания о калмыках,
чеченцах, ингушах и прочих, как будто их никогда не существовало в природе.
Сейчас карты исправлены потому только, что умер Сталин.


\tableofcontents{}

Столетие со  дня рождения одного из  крупнейших генетиков, профессора,
доктора сельскохозяйственных  наук Александра  Александровича Любищева
отмечает  общественность  5 апреля  1990  года.  Более 20  лет  провел
профессор в  нашем городе,  заведовал кафедрой зоологии  в Ульяновском
государственном  ордена  «Знак  Почета» педагогическом  институте  им.
И.Н.Ульянова. Здесь  им написаны  многие значительные труды,  здесь он
воспитал  учеников,  среди  которых профессор  Н.  Н.  Благовещенская,
доценты Р. В. Наумов, Л. А. Грюкова.

Занимаясь естественными науками, А. А. Любищев глубоко интересовался
проблемами философии, этики, эстетики, вел активную переписку с
крупнейшими учеными. Все научные труды и обширная переписка профессора
составляют огромный архив. Материалы только неопубликованных работ
содержат более трехсот рукописей общим объемом около 10.000
машинописных страниц (примерно 400 печатных листов).

Часть рукописей была любезно предоставлена дочерью профессора Е. А.
Равдель редакции ульяновского еженедельника «Политическая агитация»,
где в № 21, 22, 23 и 24 была напечатана впервые. Настоящий сборник
является первым отдельным изданием работ А. А. Любищева в нашем
городе, а фрагмент «О морали, браке, любви» публикуется впервые.


\section{Каким быть? (Мое пожелание молодежи).}


Ответы на вопросы журналиста С. Е. Романовского.

На этот вопрос дают ответы многие видные писатели, академики,
художники, артисты, политические деятели.

И первое,  что я себе  сам задаю вопрос: имею  ли я право  выступать в
этой  компании? Все  опрашиваемые лица  занимают определенное  место и
ведущее положение в  той или иной области нашей культуры.  А кто я? По
образованию  я  биолог, но  ученой  степени  по  биологии не  имею,  я
являюсь доктором  сельскохозяйственных наук.  Но агрономом я  также не
являюсь и как  сельскохозяйственный энтомолог я также  не признан. Мои
работы  получают отзывы  со  стороны видных  энтомологов,  что они  не
имеют конструктивного  характера и  что это  --- работы  экономиста, а
не  биолога. Может  быть,  я  экономист? Тоже  нет,  так  как знают  и
считают, что мои работы касаются технической стороны повреждений, а не
экономической. Когда я делаю доклады, то часто получаю возражения, что
я  рассуждаю не  как  биолог,  а как  математик.  Верно,  что с  точки
зрения большинства  наших биологов (не исключая  и академиков, которые
совершенно невежественны  в математике), я, конечно,  математик, но, к
моему  великому сожалению,  я не  могу согласиться  со справедливостью
этого чрезвычайно почетного обвинения. В математике я только дилетант,
т. е.  действительно люблю математику,  но знаю только  столько, чтобы
понимать мое глубокое невежество в этой блистательной науке, «царице и
служанке  всех  остальных  наук».  Есть  у  меня  работы  и  по  общей
биологии, но их тоже считают не биологическими, а плохими философскими
с  соответствующими  ярлыками:   виталист,  антидарвинист,  менделист,
платонист,  махист, идеалист,  короче ---  аммовид\footnote{Непонятным
оказалось слово “аммовид”. Запросил объяснения. Академик ответил: “...
Вы  просите  разъяснить Вам  смысл  термина  “аммовид”. Это,  конечно,
не  насекомое,  хотя  очень  напоминает “аммофил”  -  симпатичных  ос.
Аммовид - животное  во всех отношениях вредное  и неприятное. Название
дано ему  по принципу  соединения начальных  букв в  следующих словах:
антимичуринец-менделист-морганист-вейсманист-идеалист.   Очевидно,  Вы
мало практиковались в решении ребусов, шарад и кроссвордов...”}.

Справедливы ли эти ярлыки? Если рассматривать их как обозначение
некритического преклонения перед авторитетами, которое так характерно
для советских философов, то нет, так как у меня нет абсолютных
авторитетов ни на небе, ни на земле: «Не сотвори себе кумира, ни
всякого подобия». Но если эти термины обозначают признание за деятелем
культуры его заслуг, то я одновременно являюсь и антидарвинистом (так
как считаю дарвинизм устаревшим учением) и дарвинистом, так как высоко
ценю Дарвина как тип ученого. Но я не меньше, а больше уважаю
Пифагора, Сократа, Платона, Гегеля, Маха, Менделя, Пастера, Иоганнеса
Мюллера, Бэра и вовсе не уважаю Лысенко и свору поддерживающих его
философов.

Кто же я такой? Если каждый из вошедших в сонм людей, достойных
поучать молодежь, может быть назван Некто, в той или иной области, то
я никто в любой из затронутой мною областей, по крайней мере, с
номенклатурной точки зрения.

А все-таки я, несмотря на многие научные и жизненные неудачи,
счастливее многих, если не большинства этих Некто, и вот как
счастливый человек, доживший до 65 лет и закончивший свою официальную
деятельность, я и обращаюсь к молодежи. Опять вопрос: какое право
имеет пенсионер, ушедший в отставку не по болезни, а при вполне
приличном для моего возраста состоянии здоровья, дезертир с нивы
общественного труда советовать молодежи? Почтенны только те, кто
умирает в должности, хотя бы из них песок сыпался. Нет, я не дезертир.
Напротив, я ушел с работы в педагогическом институте, хотя меня
упрашивали остаться, так как убедился, что затрачиваемый мною труд на
обучение молодежи непродуктивен не по моей вине и что смогу принести
гораздо большую пользу обществу, работая в полную, оставшуюся еще во
мне силу там, где будет больше всего полезно: по борьбе с невежеством,
обскурантизмом, шкурничеством и подхалимством в науке и по проведению
в порядок тех мыслей, которые накопились за долгие годы научной
работы. Многие, уйдя от привычной рутинной работы, скучают, а мне
понятие скуки и слова «нечего делать» незнакомы и непонятны.

И вот я думаю, что я все-таки имею право дать совет молодым людям на
их жизнь. Мой совет: будьте независимы, и этот совет можно понимать в
разных смыслах:

1. Независимость от окружающих: «Ты сам свой высший суд». Это вовсе не
означает презрения к людям или культа собственной личности. Это просто
означает, что высшим арбитром в решении спорных вопросов должны быть
собственные разум и совесть;

2. Независимость от условий среды. Очень часто в оправдание
обывательского загнивания приводят слова: «среда заела», говорят о
скуке и однообразии тех или иных условий существования. Но мы знаем
множество примеров, когда человек, усиленно работая над собой,
преодолевая самые неблагоприятные условия, не только достигал хорошего
среднего уровня, но подымался по своим результатам и по своему
умственному развитию высоко над окружающим уровнем;

3. Независимость от узкой специализации. Специализация необходима и
неизбежна, но это не влечет за собой разделения всего человечества на
узких специалистов. Есть хорошее изречение: надо знать все о кое-чем и
кое-что обо всем. Умственная культура человека должна строиться не в
одном направлении и не в одной плоскости, а по крайней мере в
двух-трех, взаимноперпендикулярных направлениях; такой принцип
осуществляет связь разных специальностей и обусловливает целостность и
прочность всего мировоззрения человека;

4. Независимость от догматов любого сорта: если можно говорить о
бесспорном выводе из истории человеческой культуры, то этим выводом
будет утверждение, что любое, самое прогрессивное учение, переходя в
неподлежащий критике догмат, из стимула развития превращается в тормоз
развития.

\section{ОСНОВНОЙ ПОСТУЛАТ ЭТИКИ}

В данной статье я делаю попытку обосновать тот основной постулат
этики, который может быть положен как отправной пункт для развития
социалистической и коммунистической морали, причем этот основной
постулат должен удовлетворять трем требованиям:

1) он должен быть универсальным;

2) он должен быть научно обоснованным;

3) он должен синтезировать положительные стороны всех этических систем
прошлого.

На первый взгляд, такая попытка может показаться невозможной или
ненужной. Позволю поэтому показать с самого начала несущественность
тех возражений, которые выдвигаются против возможности универсальной,
научной и синтетической этики.

Давно известно необычайное разнообразие этических норм в разные
времена и у разных народов. Яркий пример можно найти этому уже у
Геродота. Как можно попытаться строить универсальную этику при таком
разнообразии? Прихотливость и противоречивость разнообразных этических
норм находится как будто в вопиющем противоречии с тем, что нам
известно в науках, --- постоянством и незыблемостью устанавливаемых
науками законов природы. Но, как известно, законы природы
осуществляются только в определенных условиях, и мы должны отличать
законы природы от правил не потому, что правила имеют исключения, а
законы не имеют, а потому, что для законов природы мы можем точно
указывать область их применения. Старая механика Галилея сменила
предшествующую механику не потому, что она справедлива в обычных
условиях. Напротив, в обычных условиях старое положение, что скорость
пропорциональна силе, более справедливо, чем новое положение, что
ускорение пропорционально силе. Но в обычных условиях имеют место
различные виды сопротивления (сопротивление воздуха, трение и т. д.),
которые и объясняют практическую неприменимость законов Галилея.
Поэтому, если мы сумеем вывести отклонения от морали из единого общего
принципа, то это позволит нам построить универсальную этику.

Резкую критику универсальности морали проводили классики марксизма,
утверждавшие, что нет единой общечеловеческой морали, она
соответствует классовой структуре общества. Многие это поняли так, что
коммунизм отрицает всякую мораль. Мы знаем, что это положение неверно,
и сейчас коммунистическая мораль не только признается, но проведение
ее в жизнь считается одним из важнейших условий для построения
коммунистического общества. Однако мнение об отсутствии
коммунистической морали имело известные основания не только в критике
классиков марксизма в отрицании общечеловеческой морали, но и в ряде
других обстоятельств. И поэтому вполне естественно, что некоторые не
по разуму усердные коммунисты действительно склонны были в период
революции отрицать коммунистическую мораль, вернее, была ходячая
формула: «Морально то, что полезно для революции, а конкретизация этой
формулы считалась ненужной. Поэтому и возникли в свое время такие
извращения, как: 1) полное отрицание обязательности стыдливости; была
даже Лига «Долой стыд» и лигисты ходили совершенно голыми. Появление
однажды демонстрации этой Лиги на улице вызвало резкий протест
красноармейцев;

2) развивалась так называемая «теория стакана воды» (Коллонтай),
считавшая, что вступление в связь с мужчиной для женщины есть акт не
большего значения, чем поглощение стакана воды;

3) появились рассказы, где доказывалось, что жалость есть чисто
буржуазный предрассудок и даже развивались соображения о возможности
вивисекции над приговоренными к смертной казни;

4) понятия долга и чести считались буржуазными предрассудками. И
Макаренко в своей книге «Педагогическая поэма» указывает, что в
Наркомпросс Украины его ругали за то, что он поддерживает эти
буржуазные предрассудки;

5) наконец, такие вопросы, как смешанное или раздельное воспитание
детей, старались тоже связывать с классовой структурой и получалось
так, что даже школа колеблется между двумя способами до сего времени.

Сейчас, конечно, наблюдается полный поворот и даже проявляется
некоторая вспышка коммунистического пуританизма. В частности, с весьма
курьезными проявлениями: в ресторане «Фонтан» во Фрунзе на картине
Неффа двум купальщицам сначала приделали бюстгалтеры, а потом и вовсе
одели. По отношению к браку мы снова вернулись к представлению о
прочности брака и о недопустимости его расторжения без серьезных
оснований.

Во всем этом мы видим частный случай общего процесса восстановления
многих отвергнутых ценностей, отвергнутых потому, что их считали
наследием буржуазного общества. А такое восстановление многих
отвергнутых положений можно истолковать лишь так, что и в этике мы
имеем гораздо больше неклассовых, ненадстроечных элементов, чем думали
раньше.

Таким образом, старое отношение наших марксистов в этике может быть
объяснено следующими обстоятельствами:

а) ни Маркс, ни Энгельс не отрицали важности этических основ
социализма, но они показали, что их совершенно недостаточно для
осуществления справедливых реформ, а многие исследователи решили, что
раз для обоснования возможности построения социализма нет надобности в
этике, то значит, она вовсе не нужна;

б) революционный марксизм, большевизм принял возможность
социалистической революции в обществе, где пролетариат не имеет еще
большинства. Он теснейшим образом связан с признанием необходимости
революции, которая в свою очередь связана с самопожертвованием многих
тысяч революционеров. Старый строй использовал существенную мораль для
собственной защиты, например, выполнение присяги считалось
обязательным долгом, какие бы преступления ни требовали совершить
агенты правительства, которому присягали. Необходимо было разрушить
ряд моральных устоев и отрицательная сторона отступила перед
положительной. Естественно поэтому, что и была создана ходячая
формула: «Морально то, что полезно для революции», формула, являющаяся
частным случаем старой иезуитской формулы: «Цель оправдывает
средства»;

в) выработка коммунистической морали считалась ненужной, так как в
период гражданской войны и некоторое время после нее искренне
полагали, что все преступления суть порождение буржуазного общества.
Поэтому для неисправимых классовых врагов широко применяли смертную
казнь, а для тех, кого считали исправимыми, давали заключение не более
10 лет, т. е. считали, что через десять лет никаких тюрем не
потребуется. Известно, что одно время были уничтожены юридические
факультеты, т. к. полагали, что с ликвидацией классов надобности в
юристах не будет; Искренне верили в то, что:



«Мы раздуем пожар мировой,

Церкви и тюрьмы сравняем с землей».



Это мнение, в сущности, является, как это ни кажется странным,
перепевом старого мифа о грехопадении. Как Адама и Еву, первоначально
безгрешных, соблазнил дьявол и человечество могло быть спасено только
Искупителем (хотя, в сущности, спаслась только малая часть), так и
первобытно невинный человек впал в грех благодаря развитию классового
общества, развившего собственнические интересы. Социалистический
строй, устраняя эксплуатацию, возвращает человека в первобытное
безгрешное состояние.

Этим и объясняется возрождение совершенно нелепых идей Жан Жака Руссо,
которое привело к педагогическим извращениям, одно время
господствовавшим. Не следует ни в чем стеснять ребенка: естественно,
он будет хорошим. Сейчас от всех этих извращений отказались. Но в силу
некоторой инерции, по теории так называемого социалистического
реализма, в ряде пьес стараются разводить маниловщину о необыкновенно
благородном типичном советском человеке. Надо вернуться к правильному
утверждению Аристотеля, что хотя большинство преступлений вызывается
экономическими мотивами, самые тяжкие не зависят от экономики;

г) наконец, ненужность выработки коммунистической этики была
естественна для основоположников марксизма, прежде всего Ленина. Ленин
и другие подлинные деятели-коммунисты имеют подлинный категорический
императив: они не мыслили жизни без неустанной борьбы за освобождение
человечества. Это их делало моральными без какой-либо ясно осознанной
ими морали. Свои качества они распространили на всех революционеров и
тем самым впали в ту же ошибку, как и Кант, который считал наличие
категорического императива свойственным всем людям. Только под
давлением фактов он принужден был понять, что категорический императив
к совершению благородных действий свойственен далеко не всем людям.

У большинства революционеров, вернее, тех, которые стремятся быть
вождями, главным стимулом к деятельности является месть, любовь к
приключениям, стремление к власти, корысть, подражательность. Мы
знаем поэтому, как интенсивно переживал Ленин известие о том, что
бывший редактор «Правды» Малиновский, которого он считал истинным
большевиком, оказался провокатором, и как много из лиц, занимавших
виднейшие посты в коммунистической партии, оказались по решению самой
партии недостойными звания члена партии и даже преступниками против
партии. Когда же революционные партии достигают власти, то в них в
значительном количестве вливаются люди, которые раньше были верными
слугами старого режима: молчалины, Фамусовы, карьеристы и рвачи.

Объяснять возникновение моральных дефектов только классовым влиянием
--- значит, впадать в то же преувеличение, в какое впал монах в басне
Крылова, оправдывавший свой способ печения яйца тем, что его научил
этому дьявол.

Конечно, классовая структура приводит ко многим преступлениям, но и в
бесклассовом обществе источники преступления не исчезнут. А отсюда
необходимость морального воздействия не ограничивается созданием
правил достижения коммунистического строя. Этика будет играть роль не
только в переходный период к коммунизму, но и при самом совершенном
коммунизме.

Что касается возможности научного обоснования этики, то, как известно,
героическая попытка такого обоснования сделана в свое время Спинозой,
который прямо называл свою этику обоснованной геометрическим методом.
«Критика практического разума» и другие сочинения Канта также пытаются
вывести общие принципы морали. Для многих; кажется, что эти попытки не
только устарели, но и принципиально безнадежны. Я думаю, что это
неверно, и полагаю, что новая попытка научного обоснования этики может
быть сделана, если мы примем во внимание ту эволюцию научной мысли,
которая имела место после Канта.

Не зря Спиноза назвал свою этику геометрически обоснованной. Он
полагал, что как существует единственная геометрия Эвклида, также
может существовать и единственная этика, выведенная путем построения
ряда теорем из небольшого числа аксиом, т. е. таких истин, которые не
нуждаются в доказательствах в силу своей явной очевидности. Кант точно
так же считал геометрию Эвклида единственно возможной и из
аподиктичности математики вывел свое учение о пространстве и времени,
как формах нашего восприятия.

Как известно, решительный удар такому представлению был нанесен
Лобачевским, и сейчас в математике совершенно стерлось различие между
аксиомами (истинами, не нуждающимися в доказательствах) и постулатами
(истины, которые мы принимаем как необходимые для дальнейших
рассуждений, хотя они не могут считаться вполне очевидными, но
доказать которые мы не можем). Современная математика считает, что и
аксиомы и постулаты есть такие положения, которые мы кладем за основу
развития определенной отрасли науки и которые мы доказать не можем..
Но современная математика допускает возможность построения совершенно
новых систем теорем на основе постулатов, отличных от эвклидовых. И
таких неэвклидовых геометрий сейчас построено значительное количество.

И в этике мы можем считать возможным построение нескольких этических
систем, исходя из разных комплексов и постулатов. Но тогда выходит,
что будет не одна универсальная этика, а несколько этик. Это неверно.
Мы знаем, что хотя существует несколько неэвклидовых геометрий, но
реальное пространство подчиняется какой-то одной, а если в разных
областях пространства могут применяться разные геометрии, то,
очевидно, необходимо озаботиться созданием такой геометрии, частными
случаями которой являются разные неэвклидовые геометрии. Например,
если для отображения реального пространства пригодна геометрия Римана,
а не Эвклида, то это не является отрицанием геометрии Эвклида, т. к.
при огромных размерах радиуса кривизны пространства практически
геометрия Эвклида для огромных протяжений сохраняет полную силу, т. к.
она одна может считаться частным случаем геометрии Римана, если радиус
кривизны пространства равен бесконечности.

Таким образом, и в области этики мы можем рассматривать разные
этические системы в смысле удовлетворения их ряду критериев: 1)
критерию практики, 2) последовательности, 3) гибкости и т. д.

Очень многие этические системы приводят к абсурду при определенном их
понимании и, наоборот, совпадают с другими системами при другом их
понимании. Например, эпикуризм в вульгарном понимании означает, что
человек лучше всего себя будет чувствовать, если будет гоняться за
самыми материальными и низменными наслаждениями. Даже не говоря о том,
что большинство таких вульгарных эпикурейцев расстраивает свое
здоровье, в их жизни почти всегда наблюдается большая
неудовлетворенность, скука, необходимость все большего разнообразия
наслаждений и разочарование, ведущее сплошь и рядом к самоубийству.

Таким образом, погоня за наслаждениями в грубом понимании приводит к
обратному результату: внутреннему противоречию. Мы знаем, что сам
Эпикур отнюдь не был сторонником грубых наслаждений. Под истинным
наслаждением он подразумевал высокие духовные наслаждения и был сам
человек со всех точек зрения моральный, и оказалось, что подлинные
эпикурейцы и стоики в практическом поведении часто вовсе не
расходятся. Такое совпадение практических результатов лиц,
исповедующих как будто диаметрально противоположные взгляды,
заставляет думать, что они на самом деле исходят не от различных
систем постулатов, а фактически от тех же самых. И что только их
система изложения различна. Мы знаем, что и в геометрии за элемент
пространства мы можем рассматривать точку и прямую как линию,
соединяющую две точки. Но можно также элементом считать прямую и
рассматривать точку как следствие пересечения двух прямых.

Можно, таким образом, постараться и в разных этических системах найти
общую основу и их все рассматривать, как неполные, часто искаженные и
несовершенные выражения общей универсальной этики. В одном из
рассказов Анатоля Франса дьявол говорит, что защитники разных
философий защищают только один из цветов общего спектра, а что на
самом деле Истина белая, она синтезирует все цвета разных теорий.
Можно попытаться и найти такой синтез, вдумываясь в положения разных
философских систем.

Для этого полезно оттолкнуться от одной из замечательных систем этики
прошлого, этики Канта. Как известно, он дал три знаменитых положения:
1) действуй так, чтобы максима твоей воли могла одновременно служить
принципом всеобщего законодательства; 2) рассматривай всегда в своем
поведении всякого представителя человечества, в твоем ли собственном
лице или в лице других, только как цель и никогда не используй его как
средство; 3) действуй, имея идею твоей воли как универсальное
законодательство.

Мы знаем, что Кант в практическом применении своей морали пришел к
ряду противоречий и даже курьезов. Например, он резко отделяет мораль
от удовольствия и считает, что существом подлинно добродетельным не
является человек, который получает удовольствие в добродетели и жизнь
которого проходит без искушений. Добродетельным человеком является
тот, кто побеждает жестокие искушения благодаря своему характеру и
воле. Добродетель не есть порождение природы, она является победой над
природой. Делать добро не по склонности к добру, а по долгу --- в этом
особенно проявляется моральная ценность характера. Шиллер высмеял
Канта в превосходном стихотворении. Один человек жалуется, что он
любит делать добро людям и потому не может считаться добродетельным,
т. к. истинно добродетельный человек делает добро по долгу, а не по
склонности. Ему отвечают: поступи очень просто, постарайся
возненавидеть людей и потом делай им добро уже без всякой склонности.

Мы видим, таким образом, что Кант формулирует положение диаметрально
противоположное таковому у Руссо. По Руссо человек, следуя природе,
будет доброжелателен, а по Канту --- только борясь с природой, человек
может быть добродетелен. Нетрудно видеть, в чем ошибка Канта. Он
считает, что человек, любящий добродетель, и человек, живущий без
усилий и искушений, одно и то же лицо. На самом деле человек может
искренне любить добро, но у него могут быть искушения в виде лени,
жадности и пр. Сумев их преодолеть, он творит добро с наслаждением, и
очень много подлинно добродетельных людей творят добро с наслаждением.
Это сказано и Христом: блаженнее давать, нежели получать. Почти то же
самое можно найти и у Аристотеля.

Совершенно верно, что могут быть люди, творящие одно и то же дело,
однако мы им дадим разную моральную оценку. Таких категорий не две, а
гораздо больше: 1) любящие добродетель и не имеющие искушений,
например, смелые по природе; 2) делающие добро по долгу; 3) делающие
добро из страха, подражания или из-за других низменных мотивов
(фарисеи, ханжи и пр.).

Но однако в цитированных положениях Канта есть много ценного, что
необходимо использовать. Ценно утверждение, что человеческая личность
не может рассматриваться как средство, но, очевидно, это благородное
положение может быть только положением предельно развитой морали, и
сам Кант, признавая смертную казнь, явно этому противоречит. Положение
о том, что человек в своем поведении имел в виду всеобщее,
великолепно, но оно не дает никаких конкретных указаний для поведения,
т. к. неясны цели тех самых всеобщих законов, по которым должно
равняться поведение индивидуума. Поэтому я полагаю возможным
формулировать такую общую формулу этики, которая может обнять все
существующие формулы этики великих мыслителей.

{\large \textbf{Поступай так, чтобы твое поведение способствовало
прогрессу человечества, выражающемуся в победе духа над материей.}}

Такое положение является, по-моему, действительно синтетическим.

В самом деле: 1) оно является синтезом материальной и формальной
этики. Материальной потому, что ставит определенную цель поведением.
Но оно же является и положением формальной этики, т. к. может
считаться выражением категорического императива и служит для всего
человечества, а не только для определенных индивидуумов или классов,
оно отвечает положению стоиков, что следует жить согласно природе
человека, а так как отличительным свойством человека является разум,
то надо жить сообразно разуму. Но оно не означает, что мы должны
преодолевать природу в том смысле, что не должны ограничиваться уже
достигнутым, а должны продолжать борьбу за торжество духа над
материей. Эта тенденция не противна истинному смыслу природной
эволюции. По прекрасным словам Бэра, мы не можем сказать, откуда
произошел дух, но все развитие жизни есть непрерывная борьба духа с
материей со все более обозначающейся победой духа. Способствуя победе
духа над материей, мы одновременно осуществляем лозунг: жить согласно
истинному смыслу природы.

\section{«ДВУХ СТАНОВ НЕ БОЕЦ...»}

\textbf{«Двух станов не боец, но только гость случайный,}

\textbf{За правду я бы рад поднять мой добрый меч,}

\textbf{Но спор с обоими --- досель мой жребий тайный,}

\textbf{И к клятве ни один не мог меня привлечь:}

\textbf{Союза полного не будет между нами ---}

\textbf{Не купленный никем, под чье б ни стал я знамя,}

\textbf{Пристрастной ревности друзей не в силах снесть,}

\textbf{Я знамени врага отстаивал бы честь!..».}

\textbf{А. К. ТОЛСТОЙ.}

(К. В. Беклемишеву)

«...Очень тебе признателен за разъяснение твоей позиции по поводу моей
статьи о внутривидовой борьбе. Я с удивлением констатирую, что по
целому ряду вопросов этического характера у меня довольно значительные
расхождения с рядом лиц, даже очень близких мне по воззрениям. Ты
затронул один, очень важный вопрос, что нельзя смешивать научных и
политических споров, в последнем случае нельзя быть вполне
откровенными и вполне объективно излагать доводы «про» и «контра».

Лозунгом моей деятельности очень часто является замечательное
стихотворение А. К. Толстого --- «Двух станов не боец».

Начнем с основного постулата: следует ли всегда говорить только правду
или иногда можно сфальшивить? Я вовсе не являюсь ригористом, считаю,
что иезуиты правильно формулировали основной этический постулат
(практической морали): «цель оправдывает средства». Я толкую этот
постулат вовсе не как нечто, однозначное другому положению: «в борьбе
все средства хороши», а так, что решающей в оценке того или иного
средства является высота поставленных целей, а не ригоризм в выборе
средств. Иначе говоря, если соблюдая строгую мораль средств, мы
рискуем совершить худшее преступление, чем нарушение формальной
морали, то эта формальная мораль должна быть нарушена. Конкретный
пример: должно ли соблюдать честное слово, присягу и проформы
торжественных обещаний? Конечно, должно, за исключением тех случаев,
где соблюдение их приводит к худшему преступлению, чем соблюдение этих
моральных требований. Мы справедливо осуждаем царя Ирода, который
очень нехотя исполнил данное Саломее обещание, ибо в данном случае
было бы более морально нарушить клятву. То же касается и присяги: воин
должен не щадить своей жизни для защиты отечества, но если
правительство использует его для угнетения его братьев, присяга не
действительна: это --- иродова клятва. И я согласен, что бывают
случаи, когда человек не только должен скрывать истину, но прямо лгать
во имя более высоких целей. Если человек во время войны попал в плен,
то следует считать почтенным, если ему удастся обмануть врага и
внушить ему совершенно превратные представления о планах собственного
командования. Простой отказ от дачи показаний, может быть, и будет
более героическим (так как часто влечет истязания и смерть), но меньше
достигает цели, и потому обман в данном случае следует предпочесть.

Но хотя и я признаю примат цели перед средствами, практически я
считаю, что этим иезуитским постулатом следует пользоваться крайне
редко, не только потому, что он все-таки является уступкой более
высокой морали, но и по целесообразности его широкого применения.
Например, меня возмущает широко распространенное толкование
медицинской этики, по которой надо скрывать правду от больного, даже
совершенно безнадежного... Я считаю, что и в случаях безнадежных
больных эта этика неуместна, так как серьезный человек воспримет
приговор как необходимость окончить свои дела и возможно полнее
использовать оставшееся время, а кроме того, может использовать это
время для обращения к деятелям неофициальной науки, которая иногда
помогает лучше официальной (есть данные, что больные раком
излечивались после сильных укусов пчел: я лично, если буду знать, что
у меня злокачественная опухоль, безнадежная с точки зрения официальной
науки, буду систематически лазить в ульи, чтобы пчелы меня кусали, и
буду производить другие опыты с целью бороться за жизнь всеми
средствами). Если же врач говорит, что его случай не безнадежен, он
этим демобилизует больного.

Перейдем теперь к случаю политической, а не научной борьбы. Я думаю, и
при политической борьбе и вообще при политической деятельности
максимальная откровенность желательна и почтенна. Возьмем очень
распространенный вопрос о престиже власти. Из крупных государственных
деятелей, как говорят, римский император Тит никогда не отменял
сделанного им распоряжения, даже в том случае, если убеждался, что
решил неправильно. Для правителя это, конечно, очень удобно, так как
подчиненные, зная его обычай, не станут докучать ему просьбами об
отмене решений. Но мне лично гораздо более импонирует наш великий
Петр, который на своем собственном указе потом наложил резолюцию:
«отменить указ, потому что дуростью был учинен». И все случаи
требования дипломатии в политической борьбе почти всегда сводятся к
борьбе за престиж. Полезно припомнить, что само слово «самокритика»
было пущено Лениным по вопросу о споре за престиж партии.
Внутрипартийную критику, проводимую в открытой печати, осуждали как
подрывающую престиж партии, и Ленин тогда заявил, что самокритика
(употребленная в смысле открытой внутрипартийной критики), хотя и
может вызвать злорадство врага, на деле укрепляет, а не ослабляет
партию.

Аналогичные споры были повсюду и у нас во второй половине XIX века. А.
К. Толстому крепко попало за «Поток-богатырь» и за стихотворение
«Порой веселой мая...» от Салтыкова-Щедрина и других прогрессивных
деятелей, которые для себя допускали издевательство даже над
почтенными вещами, такими, как самоотверженность и верность
(«Самоотверженный заяц», «Верный Трезор»), но не допускали мысли,
чтобы над деятелями их лагеря была возможна насмешка. То же и с
Писаревым: как он громил цензуру и проч., но, когда Лесков написал
романы антинигилистического характера, то этот свободолюбец заявил,
что теперь его ни один порядочный редактор в свой журнал не пустит.
Недавно я читал воспоминания Тургенева о той встрече, которую сделали
представители нашего прогрессивного общества его роману «Отцы и дети».
Образ Базарова считался клеветой на современника, хотя в данном случае
Писарев вступился за Тургенева и правильно заявил, что это не клевета;
а дифирамб новому поколению.

Был ли спор 1948 года научным или политическим? Ни то, ни другое, а
нечто третье, так как совершенно нелепо все сводить к политике, но так
же нелепо устанавливать всегда «или-или»... Для тебя вопрос ясен: с
одной стороны мракобесы, с другой --- представители света и, очевидно,
если один из представителей света проронит даже слабое слово, что не
во всем представители света правы, то этим он уже учинит как бы
предательство правому делу.

Для меня вопрос гораздо сложнее: добро и зло, свет и тень переплетены
самым сложным образом и провести такое разделение очень
затруднительно. Если уж говорить о недопустимой уступке мракобесам, то
она заключается в признании Ш., и других практических достижений Л.,
что и давало ему в руки действительно огромный козырь и на что они не
имели даже формального права, так как в практической деятельности они
ни хрена не понимают. А кроме того, что такое мракобесы? По-моему, все
те, кто запрещает высказывать или отрицает право на существование
определенных взглядов без достаточного к тому основания. Последняя
прибавка необходима, так как при достаточном основании мы вправе
налагать запрет на высказывания и пропаганду каннибализма и публикацию
рецептов изготовления бифштексов из мягких частей младенцев, технику
вскрытия квартир и прочее. В викторианский период считали, что никто
из здравомыслящих людей подобного идиотизма защищать не будет и потому
полагали допустимым неограниченную свободу высказываний. Двадцатый век
разрушил эту иллюзию уже якобы достигнутого высокого разума. Появились
теории, проповедующие необходимость восстановления действия
естественного (вероятно, «искусственного?» --- Е. Р.) отбора с их
практическими последствиями --- полным уничтожением целых наций. И все
эти, абсолютно мракобеснические теории и действия ссылаются на как
будто научно доказанные факты о связи наследственности с хромосомами и
о полной ненаучности вопроса о наследовании приобретенных свойств, а
отсюда --- о необходимости организации человеководства по принципу
скотного двора (наш Серебровский). И эти люди совершенно нетерпимы к
сторонникам ламаркистских взглядов: Кольцов после моего доклада на 1-м
съезде зоологов заявил: «Я вас не понимаю и не желаю понимать!».
Добржанский пишет в одной статье: «Нет никакого смысла проверять
данные о наследственности приобретенных свойств, так как это
совершенно ненаучное дело». Точь-в-точь как говорили Галилею
перипатетики его времени. При всем моем восхищении положительными
достижениями менделизма, я вынужден признать, что в экстраполяции
морганических и менделистических взглядов на всю эволюцию мракобесии
хоть отбавляй. И единственным выходом из затянувшегося положения я
считаю совершенно хладнокровное, абсолютно независящее ни от каких
внешних соображений размышление об общих положениях биологии...

Позиция «двух станов не боец» вызывает решительное осуждение, как
отсутствие твердых убеждений. Я склонен думать наоборот: именно
сознательное или бессознательное надевание шор означает нетвердость
собственных убеждений о безусловной спасительности рационализма,
боязнь уступки «лукавому разуму».

Но даже принимая как первое приближение, что в период решительных
переворотов позиция «двух станов не боец» недопустима, мы должны
вспомнить старую пословицу: «всякому овощу свое время». Убежденность с
отрицанием , «с порога» всякого инакомыслия, нетерпимость даже
фанатизм, могут быть полезны в период крупных переворотов, но
превращаются в безусловный вред в период эволюционного прогресса после
завершения переворота, так как тогда они стремятся остановить развитие
мысли. Остановка культуры Китая --- слишком большое уважение к
прошлому...

\section{ПАРТИЙНОСТЬ КУЛЬТУРЫ}

В выступлении М. А. Шолохова на втором съезде писателей СССР есть
такие слова: «О нас, советских писателях, злобствующие враги за
рубежом говорят, будто мы пишем по указке партии. Дело обстоит
несколько иначе: каждый из нас пишет по указке своего сердца, а сердца
наши принадлежат партии и родному народу, которому мы служим своим
искусством» (Лит. газ. от 26 декабря, 1954, с. 2).

Но как понимать подлинную партийность культуры?

Она, на мой взгляд, состоит в пламенном сердце для осуществления
идеалов социализма, холодном разуме и суровой дисциплине. Идеалы
социализма --- высокие цели. Следовательно, четыре элемента: высокие
цели, пламенное сердце, суровая дисциплина и главное --- холодный
разум.

Если следовать только указке сердца, то настоящей коммунистической
партийности не получится. Во все времена было достаточно материала для
негодования, всякая власть основана на насилии, следуя велению только
сердца, мы неизбежно придем к отрицанию всякой власти, к полному
анархизму. И мы ценим тех анархистов, которые к анархии пришли не от
болезненного развития индивидуальности культа собственной личности,
как Штирнер, а на основе пламенного протеста против несправедливости
общественного строя. Элиза Реклю, Кропоткин, Лев Толстой. Не забудем,
что и Сакко и Ванцетти были анархистами. И люди юга, испанцы и
итальянцы, особенно склонны к анархии.

Очевидно, заслуга марксизма и заключается в том, что он показал
необходимость материального прогресса для возможности осуществления
идеалов социализма: устранение всякой эксплуатации человека человеком,
устранение резкого имущественного неравенства (резкое имущественное
неравенство есть замаскированная форма эксплуатации), полное отрицание
всякой формы расизма, стремление к вечному миру, равенству полов,
развитие духовной культуры --- науки и искусства, осуществление
идеалов гармоничного человека. Вот в утверждении постулатов
социализма, осуществимость которых в каждый данный момент проверяется
критикой холодного разума, и в отрицании противоположных постулатов,
наиболее четко выявленных фашистами, и заключается партийность
культуры: заключается и на том кончается.. Всякая же попытка
предлагать какой-либо догмат в качестве окончательного решения той или
иной философской, научной или эстетической задачи --- уже насилие над
культурой, и это --- ложная партийность.

К сожалению, за время развития Коммунистической партии такие попытки
были. Иногда они диктовались пламенным сердцем, как было у Ленина с
его характеристикой брошюры Хвольсона как «подлой и черносотенной», и
его суровостью в критике элементов «поповщины».

В последние годы они диктовались доверием к шарлатанам и невеждам
(Лысенко) со стороны Сталина и прямыми преступниками (серия министров
ГБ), неверным пониманием дисциплины, как холопской исполнительности, и
материальной заинтересованностью.

Исчезли и высокие цели, и пламенное сердце, и холодный разум. Высокие
цели ставятся только в смысле достижения возможно высоких доходов,
сердце пламенеет только в отношении к особам противоположного (еще
хорошо, если противоположного) пола, а холодный честный разум сменился
лукавым разумом: «плетью обуха не перешибешь», «не я первый», «я
исполнитель и не отвечаю» и т.д. Суровая же дисциплина подчас
превратилась в восторженно-холопское подчинение. И это называется
партийностью культуры? Дисциплина, как и всякая добродетель, подчинена
закону: при крайнем выражении она переходит в порок. Как не запомнить
совет бабушки Горького:



«Злых приказов не слушаться,

За чужую совесть не прятаться».



Ученый, преподающий то, в ложности чего он уверен, только потому, что
это приказано ЦК, представитель не суровой дисциплины, а холопской
дисциплины.

Но какое право могут присваивать себе ученые, выступая против
коллектива? Право свободного разума. Право крота критиковать орла в
басне Крылова «Орел и крот». И те многочисленные кроты нашей науки,
которые в свое время критиковали партийных орлов и были временами
растоптаны «орлами» типа Берия, по справедливости могут считаться не
нарушителями дисциплины, а мучениками культуры.

Вот поэтому сейчас особенно важно уточнить понятие партийности, так
как многим очень выгодно сохранить ходячее понимание в смысле
беспрекословного подчинения всему, что высказано в постановлениях ЦК и
партийных органах «Правда» и «Коммунист».

Но помимо правильно понимаемой партийности надо задавать вопрос: вся
ли литература должна быть партийна, как и вся культура? Не так давно
ответ не вызывал сомнения. Поскольку установление подлинного
социализма --- пока только мы на его пороге --- отвечает нашим
эстетическим запросам, безнравственно заниматься чем-то другим, кроме
того, что требуется для достижения этого идеала, все остальное ---
вредно. В период подготовки революции и в период гражданской войны это
приводило к социалистическому аскетизму. Идеалом считался Рахметов,
который строго ограничивал себя в еде и даже отказался от любви вполне
достойной его женщины, подвергая свою плоть истязаниям. Всякое
франтовство преследовалось и даже влюбленность у комсомольцев и
коммунистов считалась подозрительной, а уж поцелуй руки у любимой
женщины грозил исключением из комсомола. Развивался советский
пуританизм с изгнанием обнаженного тела и всякой лирики. Все это было
наивно, неэффективно, но по-своему последовательно и трогательно. В
сущности полный параллелизм раннему христианству, когда тоже ждали
скорого второго пришествия и считали, что нечего особенно заботиться о
грешной земле.

При нэпе разрешалось «грешить» только беспартийным, а партийным был
установлен партмаксимум. Но оказалось, что дело построения социализма
более трудно, и одним энтузиазмом его не построишь, и аскетизм
партийный постигла та же судьба, что и аскетизм христианский. Сейчас
не только комфорт, но даже роскошь не считается преступлением и
получает неодобрение, только когда производится не из личных средств.

Я считаю моральной обязанностью человека по мере своих сил
способствовать исчезновению нищеты, голода и угнетения, но
способствовать производству креп-жоржета, тюля, дорогих мехов, табака,
водки не может входить в обязанности человека. А если удовлетворение
таких потребностей, которые могут не входить в понятие «партийности»
культуры, нисколько не отражается на нашем отношении к человеку, то
почему следует считать вредным, если человек любит стихи Анны
Ахматовой, Мандельштама, Пастернака? Мне думается, что если в порядке
отдыха человек занимается беспартийными стихами, то это гораздо менее
вредно, чем если он занимается одинаково беспартийными предметами
роскоши. Потому что занятие стихами не вызывает чувства зависти и не
может толкнуть на преступление, а чрезмерное франтовство и роскошь при
недостаточной обеспеченности огромных масс населения развращают
человека и приводят его к разложению.

Еще более это касается так называемой «чистой» науки. Почему тратить
время на шахматы, вышивание, карточную игру можно, это не вызывает
протеста, а занятие «чистой» наукой вызывает протест? И опять
сравнение отнюдь не в пользу шахмат и вышивания. Я не знаю, какие
практические общеполезные результаты дали шахматы и вышивание
(пожалуй, больше всего дали человечеству азартные игры, так как они
послужили толчком к развитию теории вероятности). А ведь чистейшая
наука давала часто совершенно неожиданно для ее творцов грандиознейшие
практические последствия. И не повторяет ли гонение на чистую науку
басню Крылова «Свинья под дубом вековым». Вспомним и Тимирязева с его
Луи Пастером.

Но говорят, что чистые искусства и науки отвлекут человечество в
сторону. Странно, почему в капиталистическом мире они не отвлекали, а
в социалистическом отвлекут. Ведь XIX век, положивший основу огромному
развитию техники и тем создавший технические возможности построения
социализма, прошел под знаком пышного расцвета чистой науки и высокого
к ней уважения, и это нисколько не помешало развитию прикладной науки.
То же было и в XX веке.

Так почему же, если в капиталистическом мире чистая наука нисколько не
мешала прикладной, а напротив ее оплодотворяла, почему этого не будет
в социалистическом мире? Неужели буржуазная идеология так
привлекательна, что она будет соблазном и в социалистическом мире, как
думают в «Клопе» Маяковского. Не является ли чудовищным лицемерием, с
одной стороны, утверждать, что природа советского человека уже сейчас
так высока, что он стоит неизмеримо выше презренного представителя
мира буржуазного, а, с другой стороны, считать его более подверженным
соблазнам, чем средний буржуа? Напротив, в социалистическом строе,
кроме тех стимулов, которые всегда существовали (интерес к решению
научных проблем, тем более интенсивный, чем труднее проблема,
благородное честолюбие, материальные выгоды), присоединяется и то, что
отсутствует при капитализме: результаты работ не пойдут на укрепление
строя эксплуатации человека человеком или остальных наций, а послужат
только для пользы человечества.

\section{О МОРАЛИ, БРАКЕ, ЛЮБВИ}

Сейчас, в связи с колоссальным ростом разводов, упадком семейных основ
возникают дискуссии по поводу моральных основ брака, любви, разводов.

Одной из важных статей является статья Юрия Рюрикова «Зло, или
избавление от зла?» (ЛГ от 21 января 1970 г. № 4, с. 13) и небольшое
письмо Иванниковой «Долг и любовь» (там же). Рюриков в значительной
степени отвечает на статью Н. Юркевича (ЛГ № 40, 1969 г.).

Смысл статьи Рюрикова --- указание на то, что развод не только зло
(как думает большинство участников дискуссии), но и избавление от зла.
В книге Рюрикова «Три влечения» нет заклинаний типа «должен», а в
главе «Позор или благодеяние?» идет прямой спор с теми моралистами,
которые говорят, что за смертью любви обязательно должен идти развод.
Рюриков цитирует Энгельса: «Есть много справедливого в словах
Энгельса, что раз любовь у людей иссякает или вытеснена новой любовью,
то развод становится благодеянием как для обеих сторон, так и для
общества». «Надо только, --- добавлял он, --- избавить людей от
необходимости брести через ненужную грязь бракоразводного процесса».

Любопытно, что Рюриков смешивает «грязь бракоразводного процесса» в
дореволюционные времена (когда требовались свидетели для установления
супружеской измены) с современным советским бракоразводным процессом.
Он пишет: «Интересно, что грязь бракоразводного процесса удерживала от
развода многие внутренне распавшиеся семьи, и когда четыре года назад
формальности развода были смягчены, число разводов подскочило почти
вдвое --- с 360 тысяч в 1965 году до 646 тысяч в 1966 году».

Возражает Рюриков и против Юркевича, считавшего, на основе ответов,
что Долг в качестве мотива сохранения семьи встречается достаточно
часто, Рюриков: «уж, конечно, не абстрактный долг сдерживает
большинство людей, у которых прошла любовь, их, видимо, чаще скрепляют
другие причины: кого привязанность, уважение, любовь к детям, кого
привычка, нежелание причинить горе близким, кого боязнь неудобств,
молвы, суда, домостроевские взгляды. Именно эти живые чувства (или
мертвые предрассудки) связывают тех, у кого нет любви» (Рюриков
позабыл, что партийных от развода часто удерживает партийная
организация).

Из письма П. Иванниковой: «Должна признаться, что когда читаю
высказывания людей, делающих упор на то, что «брак без любви всегда
аморален», усмехаюсь и, надеюсь, безошибочно устанавливаю, что автор
весьма молод. Конечно, безнравственно выходить замуж (жениться) без
любви. Но люди, счастливо прожившие в браке многие годы, знают, что
спустя десятилетия приходит другое, очень важное чувство --- уважение
к супругу, родственная привязанность». При всем благородстве
утверждения, что брак без любви аморален, в нем есть нечто очень
опасное. Если мысль эта будет руководить всеми, то любая ссора в семье
найдет немедленное оправдание. Конец письма Иванниковой: «Из всего
сказанного, мне думается, напрашивается такой вывод: люди, которые
решили образовать семью, должны знать, как ответственен их шаг и
делать его осмотрительно. А потому всю жизнь они должны иметь долг ---
перед самим собой, перед другим супругом, перед детьми. В этом Н.
Юркевич был абсолютно прав». Рюриков кончает статью обычными
марксистскими размышлениями, что когда семья как хозяйственная ячейка
умрет, на смену ей придет новая семья --- свободный союз женщины и
мужчины.

Вся эта дискуссия --- отражение полной путаницы, возникшей от
разрушения старой морали и полного отсутствия новой. Сейчас поэтому
хватаются за обрывки старой морали и то провозглашают полную свободу
брачных отношений, а то снова стараются установить неразрушимый брак.

Само утверждение «брак без любви аморален» просто глупо. Это какое-то
нелепое ультраромантическое возвеличение любви. Аморальным является
то, что связано с наличием обманов и невыполнением своих обязательств
перед человечеством. В старой и известной картине «Неравный брак»
венчается старый сановник с молодой девушкой --- очевидно, жертвой
экономического принуждения. Сановник очевидно любит девушку, иначе он
не пошел бы на брак, девушка своего жениха не любит. Значит, для
сановника этот брак по любви --- морален, а для девушки без любви ---
аморален. Но это явно нелепо, сановник вероятно знает, что его невеста
не любит, но использует экономическое насилие, чтобы получить ее в
жены. А девушка (подобно Соне Мармеладовой) приносит себя в жертву
своей семье. Кто из них аморален?

И еще нелепее: раз прошла любовь, так значит супруг обязан прекратить
брак. Наверное, исходят из положения, что целью брака является
удовлетворение любви. Дети появляются в статьях очень редко, но ведь
всякому ясно, что брак именно потому выделяется из всех других
отношений, что он является основой продолжения человеческого рода, и
это --- главное, что нужно иметь в виду при суждении о разводе. Если
брак оказался бездетным, то, конечно, никаких обязательств он не
налагает и взаимное прекращение любви --- достаточное обоснование для
развода. А если есть дети, то задача сохранения семьи гораздо выше
задачи удовлетворения своих любовных чувств, если вообще этой
последней задаче можно приписывать какое-либо моральное значение.

Моральным долгом является рождение и воспитание детей и это главное.
По всем старым этикам (библейская, древнегреческая и др.) человек,
уклоняющийся от брачной жизни и от труда по воспитанию детей,
совершает антиобщественный поступок и потому подлежит осуждению. Даже
если он бездетен не по своей воле, то это считается позором. По Библии
допускались все меры по обеспечению потомства (дочери Лота, Авраам и
Агарь, Иаков и Балла, Онан), среди них Левират: вступление в брак с
вдовой брата было обязательством. Все исходило из великих заповедей:
«Растите, множитесь и наполняйте землю». Поэтому и брачные отношения
допускались только до тех пор, пока женщина была еще способна рожать,
а сами по себе брачные отношения считались грехом. Тут конечно было
преувеличение, и никогда эта строгая мораль во всей строгости не
соблюдалась, но очень многое (недопустимость абортов и
противозачаточных мер, сознание долга перед супругом даже при угасании
любви) соблюдалось и оно привело к тому, что многие народы не только
не исчезли, но, наоборот, размножились, несмотря на исключительно
тяжелую судьбу. Так называемый «гуманизм» или, вернее,
ультраромантизм, отказавшись принимать любовь, как смертный грех (это
правильно), ударился в другую крайность --- возвеличение любви как
величайшего качества человечества и как высокий долг. О детях почти
позабыли, аборт разрешен, при разводе дети играют второстепенную роль.

Совершенная нелепость: «брак без любви аморален». Если несчастная
одинокая девушка (оказавшаяся за бортом нормальной семейной жизни в
силу большого дефицита мужчин) идет без всякой любви на хотя бы
временной союз для получения ребенка --- это видите ли аморально, а
когда Анна Каренина в силу вспыхнувшей запоздалой чувственности
бросает и мужа, и сына, не интересуется и новым ребенком, не хочет
новых --- это видите ли героиня. До какого морального отупления надо
дойти, чтобы защищать подобную чушь?

Проституция не потому непочтенна, что проститутка отдается без любви:
это ее право, здесь она не совершает ни насилия, ни обмана, а честно
продает свое тело. Но она уклоняется от долга женщины и только в этом
ее проступок. Но такое же осуждение мы должны вынести мужчине, если
он, подобно одному герою Мопассана, порхает от одной женщины к другой,
не желая связывать себя семьей хотя бы в каком-либо отдельном случае,
или, подобно легендарному Дон Жуану, в каждой своей связи испытывал
искреннюю любовь ко всем.

Исключение от обязанности иметь семью допускалось только для высших
целей (монашество, наука, философия) и всегда сопровождалось
аскетическим образом жизни.

% vim: foldmethod=marker ft=tex tw=70 lbr :

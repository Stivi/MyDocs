Глава первая

Два пути постренессансной музыки

1. На  далекой периферии  западной цивилизации, скрытно  от городского
мира, на протяжении нескольких столетий шло формирование своеобразного
искусства, подобного которому не было  ни в одной другой точке земного
шара. Музыка,  первоначально вывезенная  невольниками из  Африки, дала
неожиданные ростки в  северном Новом Свете и  породила неведомые ранее
виды, получившие обобщенное название «афро-американская музыка».

До  середины  XIX столетия  эта  музыка  упорно пряталась  в  глубоких
подпочвенных  слоях   культуры  США.  Если   американцам  европейского
происхождения  и приходилось  слышать ее,  то они  не воспринимали  ее
как  явление  искусства.  И  уж  совсем не  проникала  она  в  Европу.
Только в 70-х годах  прошлого века негритянские спиричуэлс, неожиданно
появившиеся  на  концертных  эстрадах мира,  пробудили  в  музыкальных
кругах  сознание того,  что  в среде  черных  жителей США  развивается
своеобразное искусство. Отмеченное редкой и оригинальной красотой, оно
вместе с  тем было в  полной мере доступно слушателям,  воспитанным на
традиционных эстетических критериях и общепринятой музыкальной логике.
Однако следов влияния этой музыки  на художественную жизнь Европы в те
годы обнаружить нельзя.

И   вдруг   произошло    непостижимое.   В   период,   непосредственно
последовавший  за  первой  мировой   войной,  мир  оказался  буквально
заполоненным  пришедшей из-за  океана музыкой  явно афро-американского
(или,  как  тогда   ошибочно  полагали,  африканского)  происхождения.
Казалось,  она   бросала  вызов  всем  представлениям   о  прекрасном,
которые   веками   господствовали   в  композиторском   творчестве   и
фольклоре Запада.  Встреченная в штыки  представителями консервативной
психологии,  эта  необычная  музыка,  под столь  же  странно  звучащим
названием   «джаз»,  с   неимоверной  быстротой   распространялась  по
городам   Америки  и   Европы.  Никакие   формы  сопротивления   этому
дерзкому  пришельцу  из  Нового   Света  не  могли  приостановить  его
победоносное  шествие.  Ныне,  спустя   более  шестидесяти  лет  после
того,   как  джаз   впервые   вынырнул   из  негритянского   подполья,
шокируя  меломанов  европейского  воспитания, он  успел  разрастись  в
самостоятельную, широко разветвленную культуру, охватывающую множество
школ   и  направлений.   Ее   воздействию  подчинена   многомиллионная
аудитория  на   всем  земном   шаре.  По   масштабу  влияния   она  не
только соперничает  с оперно-симфоническим творчеством\footnote{Термин
«оперно-симфоническое творчество» используется  здесь условно. Под ним
подразумевается профессиональное композиторское творчество европейской
традиции,  которое в  постренессансную эпоху  представлено ярче  всего
музыкальной  драмой  и  симфонической культурой.},  но,  по  существу,
далеко  превосходит его.  И  если  XX век  вошел  в  историю как  «век
электричества» или «век атомной энергии», то с равным правом его можно
охарактеризовать  и  как «век  джаза».  Джаз  открыл в  музыке  особую
выразительную  сферу, чуждую  многовековому  европейскому опыту.  Этот
важнейший пласт современной культуры  проложил в искусстве новые пути,
выражая  на  языке  звуков некоторые  фундаментально  важные  духовные
устремления нашей эпохи.

Долгое  время  не  только  в  Европе,  но  и  на  самой  родине  джаза
ломали  голову  над  происхождением  этого  «невероятного»  искусства.
Одновременно  варварское  и   урбанистическое,  опирающееся,  с  одной
стороны,  на  дикие, хаотические,  резко  диссонирующие  на слух  того
времени  созвучия, а  с  другой стороны,  на инструменты  европейского
происхождения, оно поражало  необычностью своего образного содержания.
Музыка явно  легкожанрового плана, пронизанная  танцевальными ритмами,
она  обладала  вместе  с   тем  мощным  физиологическим  воздействием,
выходящим  далеко  за  рамки   возможностей  бальной  и  вообще  чисто
развлекательной  музыки.  Не   укладывались  в  европейскую  пикантную
легкожанровость также  налет грубой  чувствительности и  дух развязной
насмешливости,  свойственные джазу  20-х годов.  Но быть  может, самым
непостижимым в нем была  громадная сила его воздействия. Поразительным
образом музыкальный язык, казалось  бы совершенно чуждый европейскому,
оказался понятным  многим представителям «послевоенного  поколения» за
пределами негритянских кругов. Отчего музыка легкожанровой эстрады так
волновала, опьяняла, будоражила душу?  Почему формы выражения, далекие
от тех,  в русле  которых проявляла  себя музыка  западной цивилизации
последних   трех  столетий,   обладали  столь   яркими  эмоциональными
ассоциациями?  Заметим,  кстати,  что  в  джазе  не  ощущалось  прямой
преемственности и с образной системой спиричуэлс.

Более  полувека   длились  (и  поныне  продолжаются)   поиски  фактов,
которые  позволили   бы  хотя   отчасти  заполнить  «белое   пятно»  в
проблеме  происхождения  джаза  и  его  распространения  в  культурной
городской  среде.  Сегодня  в  этом  пустом  пространстве  уже  многое
проступило.  От Африки  до Бродвея,  от Латинской  Америки до  Чикаго,
от  африканских танцев  на  площади  Конго\footnote{Площадь Конго  или
Конгосквер  (Congo  Square)  —  прозвище,  утвердившееся  за  площадью
в  Новом   Орлеане  (подлинное   название  —  Place   de  Beauregard),
где  на  протяжении  нескольких   столетий  собирались  для  танцев  и
музицирования  негры   из  низших  социальных  прослоек.}   и  духовых
негритянских  оркестров  Нового Орлеана  до  кекуоков  и фокстротов  в
салонах  северо-востока  США,  от креольских  песнопений  до  сельских
блюзов, от «шаутс» и «холлерс»\footnote{«Шаутс» («shouts») — свободные
экстатические возгласы религиозного характера, широко распространенные
в  негритянской среде.  «Холлерс» («hollers»)  — характерные  выкрики,
сопровождающие трудовые  процессы у  негров.} в  рабочих лагерях  и на
верфях  Миссисипи до  пьяного бренчания  на фортепиано  в юго-западных
притонах и  т. д.  и т. п.  разбросаны источники  музыкальных явлений,
от  которых тянутся  нити  к джазовому  стилю.  Эта сторона  джазового
искусства  освещена  ныне  довольно подробно.  Каждый,  интересующийся
истоками  джаза, может  почерпнуть многое  из существующей  литературы
о  нем\footnote{На  русском  языке  эта литература  пока  очень  слабо
представлена.}.  При  этом   многочисленные  труды,  весьма  различные
по  отбору  и характеру  информации,  по  глубине разработки  вопроса,
по  интеллектуальному  уровню  авторов,  совпадают в  одном.  Все  они
опираются   на  как   бы  само   собой  разумеющееся   убеждение,  что
художественная неповторимость  джаза \emph{обусловлена оригинальностью
его музыкального языка, восходящей в Африке.}

Было бы  нелепо оспаривать справедливость этого  взгляда. Самобытность
джаза  в  самом  деле,  в   решающей  степени,  обусловлена  тем,  что
в  нем  скрестились  далекие фольклорные  культуры  как  европейского,
так  и  внеевропейского   происхождения.  Они  породили  интереснейший
интонационный  синтез,  немыслимый ни  в  одной  другой точке  земного
шара, кроме  США. Однако  признание этого не  дает ответа  на главные,
центральные  вопросы, возникающие  вокруг  джазового феномена.  Откуда
универсальность его воздействия?  Почему язык афро-американской музыки
подчиняется в джазе иной логике, чем  в спиричуэлс? Почему только в XX
веке  слух  европейца  (или  американца,  воспитанного  в  европейских
музыкальных традициях)  стал восприимчивым  к музыкально-выразительной
системе,  олицетворенной  джазом?  Почему,  наконец,  джаз,  возникший
в  глубокой  «американской  провинции»,   стал  одним  из  выразителей
мироощущения миллионов  людей нашей  эпохи? Информация,  относящаяся к
музыкальным истокам  джаза и  к среде бытования  его афро-американских
предшественников,  лишь  отчасти  помогает в  решении  этих  важнейших
проблем.

Цель  нашей  книги  —  не опровергнуть  установившуюся  точку  зрения,
но  дополнить ее.  Нам  представляется,  что особенности  музыкального
языка  джаза и  негритянской среды,  в которой  он рождался,  образуют
одну  сторону проблемы  —  очень  важную, быть  может  главную, но  не
исчерпывающую всей ее сложности. Разумеется, мелодика, ритм, гармония,
тембр  и  формообразующие черты  джаза  как  в  отдельности, так  и  в
непривычных  для  европейца  взаимоотношениях  в  очень  большой  мере
объясняют  свежесть, новизну,  богатство его  звучаний. В  этом смысле
джаз  совпадает  с  одной  из важнейших  общих  тенденций  музыкальной
психологии   XX  века   —   стремлением   к  радикальному   обновлению
выразительных  средств  эпохи  романтизма  и  классицизма.  Но  одними
элементами музыкальной  речи не объяснить образную  систему искусства,
столь глубоко подчинившего себе духовный мир большой части современной
молодежи.  Почему ни  в  Бразилии, ни  на  Кубе, ни  на  Ямайке, ни  в
Португалии,  ни  в одном  другом  месте,  где черные  жители  образуют
значительную  часть населения,  не появилось  в те  годы (или  раньше)
джаза  или   близкого  ему   явления?  В  традиционной   музыке  стран
тропической Африки  также не обнаруживается ничего,  что напоминало бы
джаз, несмотря на бесспорную  общность отдельных элементов музыкальной
речи.

Ответ на эти  вопросы коренится в более широких  связях и предпосылках
джаза,  обусловленных  особенностями  истории,  социальной  структуры,
общих духовных традиций породившей его страны. Очень существенную роль
при этом играют исторические моменты, восходящие к Европе.

И  то,  что США  возникли  как  «колониальный  филиал» Англии;  и  то,
что  они являют  собой  «страну  чистого капитализма»;  и  то, что  на
протяжении  трех столетий  пуританство  подчиняло  своему влиянию  все
внешние  проявления деятельности  американского народа;  и то,  что до
конца XIX  столетия огромное  место в  жизни страны  занимало общество
«нравственных  изгоев»;  и  то,  что  в  результате  экспансионистских
устремлений латинские  земли, вошедшие в  США, привнесли с  собой ярко
чувственный антипуританский колорит; и,  наконец, то, что над историей
и общественной жизнью этой страны  господствовала система рабства с ее
сложнейшими  психологическими  последствиями  —  все  это  в  огромной
степени предопределило эстетический облик джаза.

Только совокупность ряда аспектов предыстории джаза — узкомузыкального
и  общекультурного,   африканского  и  европейского,   фольклорного  и
профессионального — может подвести к решению поставленной проблемы.

2.  Картина   постепенного  формирования  музыкального   языка  джаза,
по-видимому,  навсегда  останется  «закрытой книгой».  Начиная  с  XVI
столетия невольники из самых разных районов Африки привозили с собой в
Новый Свет  песни и танцы  своих стран. В зависимости  от национальных
и  социальных  условий,  в   которые  попадали  бывшие  африканцы,  их
музыка  оставалась  либо  более   или  менее  неприкосновенной,  либо,
скрещиваясь с  местными фольклорными видами, подчас  видоизменялась до
неузнаваемости. Вот несколько общеизвестных примеров, типизирующих эти
крайние тенденции.  Так, с одной  стороны, на бывшей  латинской земле,
ставшей  лишь  в  1803  году североамериканским  штатом  Луизиана,  на
площади Конго  в Новом  Орлеане почти до  конца XIX  века продолжалась
интенсивная  жизнь  африканских  танцев  и  музыки  в  их  практически
первозданном виде.  Уже в  наше время  на некоторых  островах, лежащих
вблизи бывших английских земель на  Юге США, долгое время остававшихся
оторванными от главного потока жизни на «большой земле», был обнаружен
негритянский  фольклор,  сохранивший  в   нетронутом  виде  почти  все
характерные  черты африканской  музыки.  С  другой стороны,  участники
первой научной  экспедиции по изучению музыки  черных невольников были
изумлены их «высокоцивилизованным обликом»  и близостью к европейскому
фольклору.  Также и  в  широко известном  дневнике английской  актрисы
Кембл, из  которого по  сей день продолжают  черпать сведения  о жизни
рабовладельческого Юга,  содержатся наблюдения  о явном  родстве между
напевами  черных  невольников   и  англокельтскими  мелодиями  Старого
Света. Дворжак  же вообще услышал  в спиричуэлс все  основные качества
симфонической темы.

Негритянская  музыка  в  США   принимала  самые  разнообразные  формы.
Широко  известно,  какое  великое  разнообразие  народов  Африки  было
представлено  в   живом  товаре,  привозимом  с   черного  континента.
Каждый из  них имел  свою лингвистическую  культуру, свои  особые виды
музицирования\footnote{В  одной из  самых первых  книг о  работорговле
автор,   мореплаватель  и   работорговец,  называет   десяток  племен,
представленных  в  его «товаре»  (Fon  Yorube,  I bo,  Fauli,  Fulani,
Ashanti, Jolof, Mandingo, Baoule). Из другого источника мы узнаем, что
в  определенный  период  XVIII  столетия в  танцах  на  площади  Конго
участвовали  африканцы из  шести  разных племен  (К  гае Is,  Minakes,
Conger  и  Madringas,  Gangas,  Hiboas и  Fieue).}.  Это  многогранное
африканское  наследие  вступало в  соприкосновение  в  одном случае  с
протестантским  хоралом,  в  другом   —  с  шотландским  и  ирландским
фольклором,  в третьем  — с  креольскими песнопениями,  в четвертом  —
с  мушкой  французских  духовых  оркестров,  в пятом  —  с  песнями  и
инструментальными отыгрышами английской легкожанровой  эстрады и т. д.
и  т. п.  Помимо первоначальных  скрещиваний имели  место и  вторичные
процессы, проследить все стадии которых практически невозможно. Джаз и
является  подобным  сложным многонациональным  конгломератом.  Процесс
национальных взаимопроникновений  (как это  часто бывает с  народной и
бытовой  музыкой)  не  изучался  систематически  до  самого  недавнего
времени  и  не  находил  отражения  в  литературе.  Можно  утверждать,
что  до  середины  прошлого  столетия, когда  перед  жителями  крупных
городских  центров  предстали   впервые  некоторые  сложившиеся  жанры
афро-американской  музыки, никто  из них  и  не подозревал,  что в  их
стране  существует  национальная  музыкальная  школа,  опирающаяся  на
негритянскую  основу. А  художественная  законченность этих  неведомых
Европе  видов  говорила  о  том,  что  они  успели  пройти  длительный
предварительный путь развития.

Ранние стадии эволюции негритянской музыки  США — от «чистых» образцов
искусства черного континента  до классических жанров афро-американской
музыки  —  утеряны,  по  всей  вероятности,  безвозвратно.  Мы  вправе
в  целом  говорить   не  о  том,  когда  формировался   тот  или  иной
ее  вид,   но  только  о   том,  когда   он  был  замечен   и  признан
жителями  европейского происхождения.  А  пришло  это признание  очень
поздно  —  чуть  ли  не  три   века  спустя  после  того,  как  первые
невольники  из  Африки   высадились  в  Новом  Свете.   Тем  не  менее
приписывать эту ситуацию только культурному провинциализму американцев
прошлых  веков  было  бы  несправедливо.  Правда,  интересы  «среднего
американца»  в  ту эпоху  были  очень  далеки от  атмосферы  искусства
вообще,  и  тем  более  от  изучения  явлений,  относящихся  к  музыке
«презренных»  черных  невольников\footnote{Безжалостная карикатура  на
американские  типы  дана  Диккенсом  в  его  романе  «Мартин  Чезлвит»
под  впечатлением   поездки  в   США.  Она  содержала   много  правды.
Но  на  самом  деле,  наряду   с  типом  американца,  характерным  для
постколониального  периода истории  Северной  Америки, узкий  кругозор
которого  ограничивался  материальными интересами,  на  северо-востоке
сохранялись и  многочисленные представители просвещенного  сословия. В
их  среде были  живы традиции  высокой интеллектуальности,  восходящей
к  пуританскому идеализму  и  революционной атмосфере  освободительной
войны. В  частности, американцы-патриоты  второй половины  XVIII века,
при  всей  своей  поглощенности  проблемами  восстания  и  организации
общества  на  новой социальной  основе,  включали  в свой  кругозор  и
художественные  вопросы,  в  том  числе  и  проблемы  фольклора.  Так,
Бенджамин Франклин  оставил ряд умных и  проницательных высказываний о
природе народной музыки. Томас  Джефферсон высказывал сожаление о том,
что  в  новом  обществе  было  преждевременно  думать  об  организации
оркестров  в  духе  европейских  придворных  капелл.  Он  же  писал  о
музыке  негров-рабов,  отмечая  их   превосходство  в  этом  отношении
над  белым  населением  Америки,  и  описывал  распространенный  в  их
среде  инструментарий,  по  всей  видимости  сохранившийся  со  времен
Африки$^6$  и т.  п. Мы  уже не  говорим об  аболиционистском движении
XIX  века,  сделавшем  так   много  для  освобождения  негров-рабов  в
южных   штатах.  Небезынтересно,   кстати,   что  из-за   нелегального
характера  деятельности «Подпольной  железной дороги»  (так называлась
законспирированная система  организованного спасения  невольников) все
документы  и материалы,  связанные  с  «преступным» похищением  рабов,
переправляемых в  Канаду, с большой тщательностью  уничтожались самими
аболиционистами. Предполагают,  что именно так  исчез без следа  и тот
вид  музыкального  фольклора негров,  который  был  неотделим от  темы
«Подпольной железной  дороги».}. И  все же  нам представляется,  что и
в  более просвещенном  европейском  обществе афро-американская  музыка
встретила бы такое же непонимание.  Не столько пренебрежение к черным,
сколько  особенности \emph{художественной  психологии}  XVII, XVIII  и
первой  половины XIX  столетий  мешали  людям европейского  воспитания
заметить афро-американскую музыку,  «услышать» ее своеобразную красоту
и звуковую логику. Вспомним,  что в кругозоре поколений, последовавших
за  эпохой  Ренессанса,  не   было  места  не  только  «ориентальному»
искусству (мы  здесь имеем в виду  не экзотику, а музыку  Востока в ее
подлинном виде). Из их поля зрения  в равной мере выпал ряд крупнейших
художественных  явлений, сформировавшихся  на  культурной почве  самой
Европы, но относящихся к далекому прошлому.

Так,  инструментальные  фантазии   испанских  композиторов  XVI  века,
занимающие  видное места  в  исполнительстве  и музыковедческой  мысли
наших  дней,   не  существовали  для  современников   Глюка,  Моцарта,
Бетховена, романтиков.

Монтеверди  и Перселл  были  забыты на  протяжении  двух столетий,  до
возрождения их творчества на рубеже нашего века. Наряду с большинством
композиторов  «полифонической», то  есть  ренессансной  эпохи, они  не
затрагивали  никаких  струн  в  душах  меломанов  века  Просвещения  и
прошлого столетия.

Машо «ожил» после перерыва в шестьсот лет.

До недавнего  времени «Искусство фуги» Баха  воспринималось только как
учебное  пособие.  Художественная  ценность  этого  произведения  была
открыта лишь исполнителями нашего времени.

Барток  и  Кодай  извлекли   на  поверхность  такие  слои  венгерского
фольклора,   которые   полностью   игнорировали  в   эпоху   увлечения
стилем  «вербункош». Совершенно  так  же  английские музыканты  (Шарп,
Воан-Уильямс и др.) обнаружили богатейшие образцы народных песен своей
страны,  «пропадавших без  вести» на  протяжении XVIII  и почти  всего
XIX  столетия. Эти  древние  пласты фольклора  стали основой  новейших
композиторских школ Венгрии и Англии в нынешнем веке.

В  России только  в  начале  XX века  впервые  прозвучали в  концертах
образцы старинного крестьянского многоголосия.

Подобная «глухота»  или «слепота» обнаруживается и  в отношении других
искусств. Достаточно красноречива всем известная судьба шекспировского
наследия,  отвергнутого   художественной  психологией  XVII   и  XVIII
столетий.

И баллада  «елизаветинской эпохи» долгое время  скрывалась в подполье,
пока два века спустя ею не заинтересовались поэты-романтики.

Упомянем, наконец, пример, имеющий прямое отношение к нашей теме.

В   эпоху  Возрождения   мореплаватели  привезли   в  Европу   образцы
изобразительного  искусства   Африки.  Но  они  оставались   лежать  в
антикварных  лавках  и частных  собраниях,  не  привлекая внимания  ни
широкой публики, ни людей искусства. Подобные же образцы, выставленные
в Британском музее на пороге XX столетия, произвели подлинную сенсацию
в среде молодых художников нового века  и повлекли за собой острый, не
ослабевающий по сей день интерес к африканской художественной культуре
в более широком плане.

Есть  нечто  закономерное  в  том, что  восприимчивость  европейцев  к
джазу  совпадает с  их  «освобождением» и  от монопольного  господства
норм греко-римского  искусства, и от «плена»  классического тонального
строя,  в  рамках  которого  возникли все  без  исключения  выдающиеся
оперно-симфонические школы между концом XVII и началом XX столетий.

Рассмотрение     причин    этого     выходит     за    рамки     нашей
темы\footnote{Подробнее   об   этом   см.  статью   автора   «Значение
внеевропейских культур для музыки XX века».}. Но сам факт «открытости»
европейского  слуха   для  искусства,  столь   откровенно  нарушающего
каноны  европейской классики,  является  важнейшей предпосылкой  того,
что  джаз —  и  его непосредственные  предшественники  — вырвались  из
«подполья»  и  органически  вписались  в  художественную  жизнь  нашей
современности.  Лишь  с  того  момента,  как  европейская  музыкальная
психология  стала  открытой  по  отношению  к  выразительным  приемам,
нарушающим такие «вечные» признаки  музыкального языка, как господство
развитой  кантиленной мелодии;  подчинение всех  элементов музыкальной
речи  законам классической  тональной  гармонии; подразумеваемая  всем
этим  полутоновая  темперация;   тембровое  мышление,  неотделимое  от
инструментов симфонического оркестра или фортепиано,  и т. п. — только
тогда джаз  и его непосредственные предшественники  перестали казаться
хаотически  бессмысленным,  «дикарским» набором  звуков\footnote{Можно
думать,  однако,  что  в   \emph{подсознание}  белых  американцев  эта
«дикарская»  музыка все  же  проникла, в  особенности если  вспомнить,
что  негритянские   танцы  постоянно  возбуждали   любопытство  белых.
Известная  исследовательница  негритянской культуры  Сотерн,  ссылаясь
на  воспоминание  современников, пишет,  что  «не  менее 2-х  или  3-х
тысяч  белых американцев  постоянно  собирались (на  площади Конго.  —
\emph{В.К.}), чтоб  поглядеть на  черных танцоров».  Общеизвестно, что
американцы  обладают  гораздо  более развитым  ритмическим  мышлением,
чем  народы  в  странах  Европы,   —  факт,  который  можно  объяснить
только воздействием афро-американской музыки.}. Из африканской музыки,
которая  в годы  рождения джаза  еще оставалась  для европейцев  terra
incognita,  джаз   воспринял  ряд  фундаментально   важных  признаков,
противостоящих классическому строю  европейской музыкальной мысли. Это
—  импровизационность как  основной  принцип  развития; огромная  роль
ударных инструментов, отражающих  сложнейшее, тонко дифференцированное
полиритмическое мышление, практически  недоступное слуху европейца тех
лет; «вольное», как бы беспорядочное интонирование, отнюдь не отдающее
предпочтения  полутоновой  темперации  и  допускающее  глиссандирующие
ходы;  мелодика, построенная  на  выкриках,  возгласах, разных  формах
речевого  интонирования; своеобразие  тембров, связанных  не только  с
перкуссивностью,  но  и  с  особым колоритом  африканских  струнных  и
духовых инструментов; общее соотношение элементов музыкальной речи, ни
в какой мере не повторяющее  законы согласованности мелодии и гармонии
в  европейском искусстве  последних  столетий;  формы многоголосия,  в
равной степени  далекие как от имитационной  полифонии и контрапункта,
так и от трехзвучной гармонии и т.д.\footnote{Подробнее об африканской
музыке см. главу девятую.}

В  наше  время,  после  более чем  полувекового  периода  популярности
джаза  со  всей  его  ритмической и  гармонической  сложностью,  когда
мир   познакомился  также   с   культурой   ряда  ориентальных   стран
и   композиторское  творчество   самой  Европы   обогатилось  чертами,
заимствованными   с   Востока,   —   после   всего   опыта   XX   века
«слуховой  кругозор» европейца  настолько расширился,  что африканские
заимствования  в  джазе  не  кажутся  более  «варварскими»\footnote{Мы
здесь  оставляем  в  стороне до  крайности  усложнившийся  музыкальный
язык  композиторского   творчества  XX  века,  так   как  эта  область
музыкального  искусства  лишь  в  малой степени  проникла  в  сознание
массовой аудитории  в годы, непосредственно  предшествовавшие рождению
джаза.}. Уже  музыка Африки в  своем «натуральном» виде звучит  в наши
дни на европейской  концертной эстраде, по телевидению, радио  и т. п.
Тем  не  менее и  сегодня  сам  феномен  скрещивания музыки  Африки  и
Европы  на Американском  континенте  представляет  острый интерес.  Он
научно  актуален уже  хотя бы  потому, что  уникальному интонационному
сплаву,  образовавшемуся  на  североамериканской  почве  в  результате
взаимодействия фольклора Британских островов, музыки стран тропической
Африки и  афро-иберийских жанров Латинской Америки,  обязан джаз своей
неповторимой выразительностью. И именно  сплав этот в огромной степени
придает  джазу \emph{современное  звучание},  резонирующее с  духовным
строем нашего века.

В  прошлом столетии  американцы европейского  происхождения «услышали»
лишь  те  виды  афро-американской   музыки,  которые  гармонировали  с
западным  музыкальным мышлением  той эпохи:  спиричуэлс\footnote{Здесь
речь   идет  о   той  разновидности   спиричуэлс,  с   которой  жители
американского   Севера  и   Европы   познакомились  через   концертную
эстраду.},  сквозь  африканский  колорит  которых  явно  «просвечивал»
протестантский   хорал  и   англо-кельтская  народная   песня;  музыка
менестрельной эстрады, широко опирающаяся на старинную, так называемую
«елизаветинскую  балладу». В  интонационном сплаве  этих жанров  очень
ясно  слышалось  европейское начало;  подчас  оно  преобладало. Но  на
пороге  нашего   века  совсем  иные   разновидности  афро-американской
музыки  стали  привлекать   внимание  современников.  Сначала  кекуок,
затем регтайм,  почти одновременно  с ним  блюз настойчиво  внедряли в
сознание широкой публики новые аспекты музыкальной выразительности, до
тех  пор игнорировавшиеся  слушателями европейского  происхождения. Из
недр  негритянской  жизни  всплыли  на  поверхность  необычные  ритмы,
восходящие к  Африке; старинные  и народные  лады, давно  изгнанные из
профессиональной  европейской  музыки  тональной  эпохи;  перкуссивные
звучания, ассоциирующиеся с игрой на  ударных, в том числе самодельных
инструментах. Блюз,  расцвет которого  по существу совпадает  с эпохой
джаза, уже  никак не связан с  классической европейской художественной
системой,  с  равномерно   темперированными  звуками,  с  классическим
ладотональным строем. И параллельно с блюзом белые американцы начинают
«слышать» и понимать выразительность других видов негритянской музыки,
в  том  числе  тех,  которые  ранее  оставались  достоянием  замкнутой
черной  среды.  Если  Дворжака  и его  современников  восхищали  такие
образцы  спиричуэлс, которые  не диссонировали  с европейской  хоровой
классикой,  а  в некоторых  случаях  подвергались  аранжировке в  духе
«шубертовской» романсной лирики, то «ровесники нового века» попали под
обаяние других разновидностей  этого жанра — народно-импровизационных,
в которых господствует экстатическое эмоционально-необузданное начало.
«Шаутс» («shouts»)  и «холлерс»  («hollers») —  омузыкаленные выкрики,
абсолютно свободные в мелодико-интонационном  отношении с точки зрения
европейских  критериев и  по  всей видимости  очень давно  сложившиеся
на  американской почве,  теперь  только обрели  для белых  американцев
выразительный  смысл  и, заметим  кстати,  в  большой мере  обусловили
характер  мелодики   блюза,  госпелз\footnote{Госпелз   («gospels»)  —
современные нам негритянские духовные песни на евангельские тексты.} и
импровизационных  эпизодов  джаза.  Таким образом,  процесс  включения
афро-американской  музыки  в слуховой  кругозор  белых  жителей США  и
европейцев развивался в соответствии с переменами, происходившими в их
музыкальном сознании.

И  все же,  как  ни увлекательна  картина интонационного  скрещивания,
приведшего к джазу, само по себе оно бы не определило его эстетическую
сущность.  Здесь  на первый  план  выступают  другие моменты,  которые
коренятся  в мировоззренческих  системах и  формах общественной  жизни
заокеанской   страны,  породившей   это  своеобразное   художественное
явление.

3.  Выдающийся  негритянский  историк  культуры,  писатель,  музыкант,
критик  Ле  Рой Джонс  в  своем  исследовании, посвященном  блюзу$^9$,
первый,    насколько   нам    известно,   подчеркнул    \emph{духовную
дезориентацию} африканцев, прибывавших в  Америку, как одну из главных
причин  их великих  страданий в  Новом Свете.  О выпавших  на их  долю
физических муках  и душевных  пытках, которые не  знал ни  один другой
народ западной  цивилизации в новое  время, сказано и  написано много.
Вероятно,  негритянский  народ  запомнит   их  навсегда.  Но  огромным
дополнительным источником  его терзаний было то,  что бывший африканец
оказался в совершенно непроницаемой для него духовной атмосфере, среди
абсолютно чуждых  ему в  психологическом отношении людей,  в обществе,
нравственные и социальные основы  которого оставались для него глубоко
зашифрованными.

Действительно,   трудно  представить   себе  более   антагонистические
мировоззренческие  системы, чем  рационализм ренессансного  гуманизма,
господствовавший  в  Европе,  и фаталистическое  начало,  определявшее
жизнь  африканца.  Человек  в   Африке  мыслился  лишь  как  существо,
находящееся в  безграничном подчинении высшим  силам; он не  только не
был центром  вселенной, но даже  и не  стал «мерой вещей».  Каждое его
действие имело  для него  смысл только в  религиозно-магическом плане,
совпадающем с  идеей государственности (то есть  трибальности). Другой
логики жизненного  поведения африканец не знал.  Рациональное начало в
его мышлении, поступках, художественных  проявлениях было оттеснено на
задний план, а моменты фантазии, интуиции, инстинкта в высокой степени
развиты. И народу с  подобным миросозерцанием было суждено столкнуться
с  американским  обществом,  где традиции  ренессансного  рационализма
были  доведены  до  предела,  где протестантство  (в  его  пуританской
разновидности) начисто уничтожило все  следы красочного и фантазийного
начала  в  религии,  а  последовательный  до  конца  капиталистический
уклад  жизни  привел  к   торжеству  меркантильного  строя  мысли  над
художественным.  Со  всем  этим  пришлось  не  просто  столкнуться,  а
оказаться в полном и неограниченном ему подчинении.

Быть  может,  если  бы  невольники  из  Африки  сохраняли  и  в  Новом
Свете контакты  друг с  другом, их духовная  растерянность не  была бы
столь  катастрофической. Но  рабовладельцы преднамеренно  соединяли на
одной плантации,  вообще на одном участке  любой трудовой деятельности
представителей  разных племен,  говоривших  на  разных языках.  Каждый
человек оказывался беспросветно одиноким.  Оторванный от родины, семьи
и  привычной среды,  он к  тому же  потерял своих  богов, то  есть всю
идеологию и весь  строй жизни, которые придавали  его внутреннему миру
целостность и равновесие.

Для многих  других пришельцев из Старого  Света процесс приспособления
к  американским  условиям также  был  мучительно  трудным. Но  все  их
переживания  меркнут перед  ощущением духовной  пустыни, которое  было
уделом выходцев из африканских стран. С какими бы тяжелыми и жестокими
сторонами  жизни не  столкнулись  эмигранты из  Европы, их  психология
принципиально  отличалась от  мироощущения бывшего  африканца или  его
потомка  (речь идет  о периоде,  предшествовавшем Гражданской  войне).
Во-первых,  они пересекали  океан  добровольно и  с  глубокой верой  в
лучшее  будущее, и  потому оптимистическая  интонация пронизывает  все
проявления их  мироощущения и  деятельности; во-вторых, они  не только
оказывались в родственной духовной среде,  но из поколения в поколение
помнили о своих  европейских корнях и сохраняли (или  во всяком случае
имели возможность сохранять)  связи с европейской культурой.  У всех у
них было  и прошлое  и будущее. А  бывшему африканцу,  насильственно и
окончательно оторванному от своих  корней, не виделось ничего светлого
и  в перспективе.  Люди  гибли  не только  от  физических страданий  и
унижений,  но и  от осознания  полной безнадежности  своей судьбы,  от
отсутствия  каких-либо проблесков  веры  и  умственной жизни,  которые
могли  бы смягчить  их  безысходную тоску.  Народ  без воспоминаний  и
надежд, заблудившийся в беспросветной темноте враждебного ему мира, он
неизбежно терял  жизненную точку опоры. Его  внутренняя сущность могла
легко оказаться сломленной.

Но он нашел  эту спасительную точку опоры в музыке,  которую создал на
непонятной ему новой земле.

Ни  один из  главных  видов  традиционного художественного  творчества
Африки  не  сохранился  в   общественной  атмосфере  северного  Нового
Света.  Коллективные  обрядовые  танцы  и  изобразительные  искусства,
концентрирующиеся    вокруг   идеи    подчинения   высшим    силам   и
отождествляющиеся  с безграничным  прославлением правителя,  неизбежно
атрофировались  в условиях  невольничьей  жизни.  Одна только  музыка,
которая в иерархии африканских искусств занимала подчиненное положение
и выполняла, по существу, прикладные  функции, продолжала жить в среде
черных  рабов  американского  Юга.  Но  и  ей  пришлось  подвергнуться
радикальным  изменениям   прежде,  чем   она  достигла   своего,  ныне
классического уровня.

Пожалуй, самое  радикальное преобразование африканской музыки  в Новом
Свете состояло  в смене основной  творческой темы. Ею  стало выражение
\emph{индивидуалистического сознания}.

Музыкально-выразительная    система,    господствовавшая   у    народа
тропической  Африки  на  протяжении   веков,  была  призвана  отразить
общественно-организованное фаталистическое  мироощущение. Как  и всему
художественному творчеству  этих народов,  их музыке  были свойственны
высокоупорядоченные стилизованные формы, таящие глубокий символический
смысл  и имевшие  подчеркнуто ритуальный  облик, в  котором нет  места
лирике.  И  хотя (см.  выше)  по  сей  день  на территории  США  живут
ритмоинтонации  и исполнительские  приемы африканской  музыки, тем  не
менее  идейный  смысл этих  выразительных  элементов  в рамках  музыки
Нового Света оказался совсем иным, чем в искусстве их древней родины.

Не просто  лирические образы,  но более  конкретно —  образы страдания
определили  главное   направление  в  музыкальном   творчестве  бывших
африканцев.  И именно  эта  область  афро-американской музыки  открыла
новую страницу в мировом искусстве.

Этот  факт  тем  более  значителен, что  образ  трагически  окрашенных
душевных    переживаний   сформировался    в   позднеренессансной    и
постренессансной  музыке Европы  раньше любого  другого. В  творчестве
Монтеверди  и Шютца,  Перселла и  Генделя, Баха  и Глюка,  Бетховена и
Шуберта  и  многих,  многих  других  композиторов  вплоть  до  Шопена,
Чайковского,  Малера,  Шостаковича  образ  скорби в  музыке,  в  самых
разнообразных его оттенках нашел классически совершенное, в буквальном
смысле слова  потрясающее душу выражение.  Можно ли было  ожидать, что
негритянские  музыканты  в  состоянии  добавить  нечто  свое  к  этому
богатству?  Между  тем  такое   чудо  произошло.  В  искусстве  негров
открылась  новая музыкально-выразительная  сфера, связанная  с образом
страдания.  Музыка Европы  не  знала  столь оригинального  преломления
этой  извечной темы.  Здесь  в необычном  переплетении  и в  тончайших
эмоциональных нюансах представали  перед слушателем образы безысходной
скорби и безграничной  веры, смерти и жизнелюбия,  тоски и чувственной
радости,  душевного  одиночества  и   братской  любви,  скептицизма  и
протеста,  душевной  боли  и неиссякающего  юмора...  Характерно,  что
даже рабочие  песни, близкие  африканским разновидностям,  приобрели в
Северной  Америке иную  выразительность.  На  африканской родине  труд
воспринимался  как  деятельность,  ведущая  к  процветанию  общины,  и
поэтому  их  песни  проникнуты  тонами радости  и  оптимизма.  У  того
же  африканца, принужденного  выполнять работу  в тяжелейших  условиях
неволи, песни труда сливались с интонациями плача и стона.

Здесь  мы  подходим  к  одной  из  центральных  проблем,  связанных  с
джазом,  —  к  проблеме  профессиональных форм  и  жанров,  в  которые
вливалась музыка,  рождавшаяся среди черных невольников  и поднявшаяся
до общечеловеческого значения.

Разумеется, далеко  не все  виды негритянского фольклора  США достигли
универсальности. Многие из  них так и не вышли за  рамки породившей их
среды.  Как происходит  в искусстве  любой страны,  любой национальной
культуры, классического  уровня достигают только  такие художественные
виды,  которые отражают  в совершенной  и общезначимой  форме какие-то
типичные   и  важные   особенности  данной   культуры.  Художественная
атмосфера,  в  которой  оказался  фольклор, привезенный  из  Африки  в
северный Новый Свет, в  решающей степени определила облик классических
жанров афро-американской музыки.  Трудно представить себе, разумеется,
какие формы и  жанры приняла бы музыка Африки, окажись  она в стране с
давними, глубоко  и широко внедрившимися  традициями профессионального
композиторского  творчества  (оперы,  симфонической  музыки,  развитой
полифонии  и  т. п.),  например,  в  какой-либо стране  Центральной  и
Западной  Европе. Однако  с уверенностью  можно сказать,  что если  бы
реально существовало такое явление, как афро-европейская музыка, то ее
эволюция  привела бы  не к  блюзу,  не к  регтайму,  не к  джазу, а  к
каким-то  иным  видам —  быть  может,  к хоровым  жанрам,  родственным
спиричуэлс,  или  симфониям,  основанным   на  негритянских  темах,  о
которых мечтал европеец Дворжак.  (Заметим, что подобные симфонические
произведения, даже появившиеся уже в нашем столетии, все же не создали
афро-американскую  школу  в  музыке  и  их  место  в  общеамериканской
культуре не сопоставимо со  значением джаза или его предшественников.)
Между тем, если бы это и произошло, то элементы, которые образовали бы
типичную  интонационную  сферу такой  гипотетической  афро-европейской
музыки,  могли  бы  быть  теми же,  что  в  музыкальных  произведениях
афро-американских  жанров;  в  обоих  случаях речь  шла  бы  только  о
скрещивании  западноевропейского   и  африканского  фольклора,   но  с
совершенно различными результатами.

Не  столько   интонационная  основа,  сколько   «творческое  задание»,
которому подчиняется звуковой  материал, определяет эстетический облик
музыкального  опуса  или  целого  художественного  направления.  Можно
привести  множество примеров  в  этой общей  закономерности. Для  того
чтобы не  слишком удаляться  от нашей  темы, ограничимся  примерами из
музыки Нового Света.

Так, в  основе джазовых импровизаций  Дюка Эллингтона и  его «Духовных
концертов»  лежит один  и  тот же  интонационный  источник. Между  тем
эти  произведения   столь  же  резко  контрастируют   друг  с  другом,
как  любые классические  образцы  музыки  чувственного и  возвышенного
созерцательного плана.

Регтайму   отдали   дань   многие    музыканты,   сочинявшие   как   в
популярно-развлекательном,   так   и   серьезном   стиле.   В   рамках
одной  и  той  же  афро-американской  ритмической  структуры  возникли
разительно  несходные  произведения.   Например,  в  регтаймах  Скотта
Джаплена  (Scott Joplin),  создателя классического  вида этого  жанра,
оживает типичная  атмосфера американской  провинции «конца  века», где
негритянский  фольклор в  эстрадно-комедийном преломлении  сливается с
чертами  салонной  польки и  маршем  для  духового оркестра.  У  Сати,
ориентировавшегося  на  «плебейскую среду»  современного  французского
города, он приобрел утрированно  «кафешантанный» облик. У Стравинского
же  —   это  высокоопоэтизированное   произведение  в   духе  новейшей
фортепианной музыки XX века.

Музыкальное  мышление  Вила  Лобоса  неотделимо  от  афро-бразильского
фольклора и  его яркой  интонационной самобытности.  Тем не  менее оно
породило,  с одной  стороны,  «бахианы» с  их отвлеченной  философской
направленностью,  а  с  другой  —  «уличную  музыку»  некоторых  шорос
\footnote{Шоро  — национальные  бразильские инструментальные  ансамбли
фольклорного происхождения.}.

Приведем пример другого рода.

История   поставила  для   нас   бесценный  эксперимент,   позволяющий
проследить глубокую зависимость судьбы жанра от художественного уровня
и идейной направленности среды, в которой он развивался.

Во второй половине  XVI века возник протестантский  хорал, который был
призван  сыграть высокоплодотворную  роль в  обновлении художественной
системы  всей европейской  музыки.  Этот  хорал оказался  единственным
видом композиторского  творчества Европы, который ранние  колонисты из
Англии  привезли  с собой  в  Новый  Свет. На  протяжении  последующих
полутора  столетий  протестантский  хорал  как  бы  вел  параллельную,
раздвоенную жизнь  на разных полушариях.  В Германии и Англии,  где он
оказался  в  окружении  высочайших образцов  искусства  и  органически
вписался во  всю постренессансную  музыкальную культуру,  его развитие
привело  к протестантской  музыке гениального  уровня. Кульминационные
моменты его  пути ознаменованы  творчеством Шютца,  Перселла, Генделя,
Баха. На  американской же  почве, где  не было  никаких художественных
школ, никаких центров развития музыкального искусства и где богатейшее
наследие Возрождения оказалось  окончательно утерянным, хорал приобрел
своеобразную депрофессионализированную  форму. Ни мотет, ни  месса, ни
кантата, ни  антем, ни оратория, ни  органная литература — ни  один из
этих жанров, в  рамках которых проявляла себя  протестантская музыка в
Европе, не нашел себе места в Новом Свете.

Для   американской   культуры   той  эпохи   религиозное   (и   притом
протестантское) начало  было кардинальным, основополагающим. И  тем не
менее до конца  XVIII столетия в музыке оно находило  отражение лишь в
простейшем  гимне  (или  псалме).  Это  было  произведение  не  только
малое по  масштабам, элементарное по форме,  ограниченное по средствам
выразительности, но к  тому же как бы повернутое лицом  к прошлому — к
давно  забытой  в Европе  раннеренессансной  полифонии  и к  фольклору
шекспировской  эпохи.  Характерно,  что   когда  этот  прямой  потомок
европейского  хорала  вступил  в соприкосновение  с  афро-американским
фольклором,  то появился  жанр очень  далекий по  своему эстетическому
облику  и   идейному  содержанию   от  тех   афро-американских  видов,
которые  вели к  джазу. Речь  идет о  спиричуэлс, то  есть о  духовных
протестантских  песнях,  сформировавшихся   в  негритянском  обществе;
несмотря  на  свою  экзотическую  (на  слух  европейца)  интонационную
систему, они  естественно вписываются в русло  европейской религиозной
традиции.

Но  джаз и  его предшественники  формировались в  среде очень  далекой
от  той,  где  рождались  спиричуэлс.  Если  на  художественный  строй
последних сильнейшим образом воздействовала религиозная мысль, то джаз
и  родственные ему  жанры  оказались в  зависимости  от других,  прямо
противоположных  по духу  идейных  сфер.  При этом  все  они в  равной
степени опираются на язык афро-американского фольклора. Принципиальное
отличие  спиричуэлс от  джаза и  его предшественников  коренится не  в
звуковой основе,  которая в  обоих случаях  не походит  на музыкальный
язык  Европы тональной  эпохи, а  в \emph{эстетической}  сущности этих
видов. И если спиричуэлс вне  всякого сомнения восходит к европейскому
хоралу, то джаз и его предшественники  столь же явно связаны с другими
художественными  традициями, типичными  именно для  США и  не имеющими
себе подобных в искусстве Старого Света \footnote{Ни блюз, ни регтайм,
ни новоорлеанский джаз  не имеют себе подобных стилей  в Старом Свете.
До  сих  пор  не  удается  обнаружить  их  прототипов  в  традиционном
искусстве черного континента. В  отношении европейских связей ситуация
более сложная. О прямой преемственности, а тем более тождестве явлений
также говорить  не приходится.  Однако в музыкальных  традициях Европы
можно проследить некоторые отдельные истоки джаза, правда, отдаленные,
сильно преображенные и к тому  же лежащие вдали от магистрального пути
композиторского творчества Европы. Подробнее об этом см. дальше.}.

Однако,   при  всем   своем   ошеломляющем  своеобразии,   неповторимо
американская культура, породившая джаз,  начинала свой путь с духовных
ценностей, накопленных в Европе в эпоху Ренессанса.

4.  Решающим  для  судеб  американской   музыки  в  целом,  для  джаза
в  частности,   оказалось  то  обстоятельство,  что   «страна  чистого
капитализма» не знала дворянской культуры.

Композиторское    творчество    Европы,   давшее    миру    величайшие
художественные  ценности,  веками   вырабатывало  свой  одухотворенный
облик.  Тысячелетняя неразрывная  связь с  храмом навсегда  определила
его  возвышенную  направленность.   Европейская  музыка  стала  вечным
олицетворением приподнятого  идейного начала,  глубочайшего постижения
человеческой  души,  носителем  красоты  во множестве  ее  аспектов  и
граней.  При этом  жизненно важная  роль, которая  принадлежала музыке
во  внедрении религиозной  идеи  в  массовую психологию,  обеспечивала
самую возможность  непрерывного интенсивного  развития композиторского
творчества  на  основе  все усложнявшегося  профессионализма.  Десятки
поколений  музыкантов  должны  были  пройти  серьезную  композиторскую
школу,  прежде  чем  могли   возникнуть  такие  великие  явления,  как
нидерландская   полифоническая   школа,   как   «священные   симфонии»
венецианцев, как творчество Палестрины и Монтеверди...

Печать возвышенности лежит и  на светских школах, типизирующих расцвет
и кульминацию ренессансных идеалов. Это был как бы отблеск духовности,
неотделимый  от  господствующего  (церковного)  направления  в  музыке
средневековья  и  Возрождения.  В  свой  первый  постренессансный  век
музыкальное  творчество   Европы  вступило  с   богатейшим  наследием.
Сменилась  эстетика, сменились  центры  творческих исканий,  сменились
музыкальные жанры и средства  выразительности. Но неизменными остались
одухотворенный  строй  композиторского  творчества  и  его  высочайший
художественный уровень. Четыре века,  отделяющие нашу современность от
переломной  эпохи на  рубеже XVI  и XVII  столетий, не  внесли в  этом
смысле  изменений. Мощная  струя  духовной преемственности  объединяет
между собой столь  далекие друг от друга во  времени произведения, как
мадригалы  Джезуальдо и  «Лунный  Пьерро» Шенберга,  пассионы Шютца  и
«Страсти  по  Луке»  Пендерецкого,  «Самсон»  Генделя  и  «Царь  Эдип»
Стравинского, симфонии  Моцарта и  симфонии Шостаковича  и т.  д. Даже
комедийное направление  в музыке постренессансной  Европы преломляется
сквозь облагороженную сферу чувств.

Пронести    подобный   строй    художественной    мысли   через    всю
постренессансную  эпоху было  делом  отнюдь не  простым, меньше  всего
автоматическим.  В   новых  исторических  условиях   великие  традиции
прошлого   могли  легко   оказаться   утраченными.  Фактически   такая
возможность уже доказана  самой историей. Вспомним хотя  бы, что самая
крупная  и  влиятельная  школа  эпохи Возрождения  —  нидерландская  —
навсегда угасла  после XVII века.  Английская музыка как бы  исчезла в
период  между  концом XVII  и  концом  XIX столетий.  Испанская  школа
медленно  умирала   в  первый  постренессансный  век   и  восстановила
свои  силы  лишь  на  пороге   нашего  столетия.  Из  всего  множества
разнообразных национальных школ,  образующих композиторское творчество
Ренессанса, только Италия, Франция и австро-немецкие княжества вошли в
постренессансную эпоху  во всем своем художественном  блеске. Это были
именно те страны, где главные очаги серьезного музыкального творчества
сосредоточивались в придворной среде.

И в эпоху Возрождения значение дворцового искусства было очень велико.
Начиная  с  провансальских  трубадуров, через  мадригальное  искусство
ars   nova,  французскую   полифоническую  песню,   позднеренессансный
мадригал,  дворцовую  musica  reservata  и т.  п.,  вплоть  до  первых
драм флорентийской  камераты, придворная среда была  стимулом развития
светской профессиональной  музыки высокою уровня. Казалось  бы, ничего
принципиально  нового в  этом  отношении в  постренессансную эпоху  не
произошло. Но на самом деле изменилось многое.

После   того   как   в   начале   XVII   века   музыка   католического
богослужения   навсегда  утратила   свою  ведущую   роль  в   развитии
музыкального   профессионализма   и   отступила  далеко   в   тень   в
период  торжества ренессансно-гуманистических  идеалов, роль  главного
очага  музыкально-творческих  исканий   и  «кузницы»  их  классических
образцов  перешла  к  светской музыке,  культивируемой  при  княжеских
или   королевских   дворах   (или  получавшей   поддержку   всесильной
аристократической среды).

На  первый взгляд  подобное возвеличивание  аристократической культуры
может  показаться неоправданным,  ибо  сковывающее влияние  придворной
эстетики  на  творчество  серьезных  художников  стало  хрестоматийной
истиной.  Борьбу с  ней  вели все  выдающиеся  композиторы от  Генделя
и  Глюка  до   Россини  и  композиторов  русской  школы.   И  в  самом
деле,  стремление  к  легкому  развлечению,  к  пустой  декоративности
---  характернейшая  черта   придворной  художественной  жизни.  Сотни
«позолоченных пустячков» увидело свет при  европейских дворах в XVII и
XVIII  столетиях.  Условность  в  изображении  чувств,  преувеличенная
декоративность, облегченная  трактовка типичны для  классических видов
придворного  искусства, таких,  как пасторали,  балеты, церемониальные
шествия,  сценические интермеццо,  музыкальные спектакли,  пронизанные
вокальной виртуозностью или перегруженные зрелищными эффектами и т. п.
Можно сказать,  что если для дворцового  искусства содержательность не
была требованием sine  qua non, то без внешнего  блеска и утрированной
«красивости» оно просто не существовало.

И тем не менее  эта принципиальная легковесность оказалась совместимой
с другой фундаментально важной  стороной придворной культуры, а именно
с  ее \emph{стимулирующим  значением для  судеб всей  постренессансной
музыки}.

Художественная  жизнь   при  дворе  требовала  творческих   исканий  и
обеспечивала их в грандиозных масштабах. В этом отношении она не знала
соперников.  Искусство было  жизненно  важным  элементом княжеского  и
королевского двора, показателем  его силы и блеска.  Феодал не столько
покровительствовал художникам, сколько сам  остро нуждался в них. Если
в  артистической  среде  музыкант, как  личность,  испытывал  унижение
и  обидное чувство  социальной  дискриминации,  то как  художественный
деятель  он   был  неотъемлемой  принадлежностью  дворцовой   жизни  и
осознавал себя частью социальной  системы, обеспечивающей его право на
творчество. (В  этом кроется принципиальное различие  между отношением
аристократа-феодала к обслуживающим его двор музыкантам и меценатством
в  буржуазном   обществе.  В   последнем  случае   патрон-филантроп  в
зависимости   только  от   своего  личного   желания  может   денежным
пожертвованием  оказать   помощь  отдельным   «свободным  художникам»,
лишенным в  условиях капитализма  какой-либо поддержки  в определенной
социальной  среде. Но  сам  характер творчества  может  мецената и  не
интересовать,  и к  его собственному  жизненному укладу,  как правило,
облагодетельствованное им искусство отношения не имеет.)

Требования,  предъявляемые  к   музыкантам  при  дворе,  стимулировали
и   обеспечивали  непрерывный   рост  профессионализма   и  обновление
художественных  форм. Просвещенный  феодал ориентировался  на новейшие
художественные   находки  его   современности.  Важнейшим   следствием
всего  этого  было то,  что  в  рамках  условных форм  и  утрированной
декоративности  придворное искусство  допускало  и поощряло  серьезные
творческие  искания.  По  существу,  все  господствующие  новые  жанры
постренессансной  музыки рождались  и  развивались  в теснейшей  связи
с  дворцовой  эстетикой.   Весь  путь  оперы  от   dramma  per  musica
флорентийцев  до  лирической   трагедии  реформатора  Глюка  неотделим
от  аристократической   среды.  В  придворных  капеллах   возникали  и
симфонические школы  века Просвещения  --- от маннгеймской  до венской
классической.  Клавирная литература  приняла  свой классический  облик
в  аристократических  салонах.   Культивирование  при  дворах  вельмож
оркестровых  сюит  обеспечило   жизнь  такого  значительного  элемента
постренессансной  музыкальной культуры,  как симфонический  оркестр, и
т.  д.  Если  для  придворных  празднеств  создавались  шедевры  вроде
музыкальных  драм Монтеверди  и  лирических трагедий  Люлли; в  рамках
итальянской  вокальной  колоратуры  возникали  изумительные  по  своей
психологической  глубине  арии  Генделя; Глюк  совершал  свою  оперную
реформу  в  королевском театре  и  при  поддержке королевского  двора;
классические симфонии  Гайдна писались как прямое  выполнение заказа к
очередному дворцовому празднеству, ---  то более веской поддержки этот
тезис не требует.

Глубочайшую  зависимость музыки  постренессансной эпохи  от придворной
среды особенно ярко характеризует картина развития оперного искусства.

При  огромной  и  все  возрастающей  популярности  оперного  театра  в
XVII  и  XVIII  столетиях  его  содержание  в  течение  сколько-нибудь
продолжительного  времени  оказывалось все-таки  убыточным.  Известно,
что  в бюджетах  провинциальных  немецких княжеств  первое место  (как
отдельная  статья)  занимали расходы  по  постановке  опер и  балетов.
Особо  роскошные  праздничные  спектакли  ---  такие,  как  постановка
опер  «Геркулес»  Кавалли  в  Париже  или  «Золотое  яблоко»  Чести  в
Вене,  ---  нарушали  на  время даже  бюджет  королевского  двора.  Но
потребность в музыке, и в особенности оперной музыке --- свидетельства
могущественности,  богатства,  художественного  престижа двора  ---  у
европейских феодалов  была столь  важна, что они  находили необходимые
ресурсы,  часто  в форме  налогов  (прямых  или косвенных).  Например,
герцог Брауншвейгский  продавал своих  подданных в  солдаты специально
для того, чтобы  устраивать при дворе роскошные  оперные постановки. В
критике,  которой парижская  дворцовая оппозиция  подвергала Мазарини,
видное место  занимало обвинение  в том,  что он  заимствовал огромные
суммы  из   налоговых  сборов   для  Королевской   оперы.  Мантуанский
герцог Гонзага  тратил баснословные  суммы на  оплату виртуозов-певцов
и  прославленных   декораторов-постановщиков,  обеспечивавших  высокую
репутацию  его  двора,  но  из  месяца  в  месяц  задерживал  скромное
жалование работающим  у него  музыкантам. У  могущественного курфюрста
Саксонского,  неимоверно   гордившегося  блестящей   и  разносторонней
художественной   жизнью  дрезденского   двора,   члены  капеллы   были
доведены  до   нищенского  уровня   существования.  Начиная   с  Петра
Первого, «купившего»  за границей  светских музыкантов  для придворных
дивертисментов, весь путь развития  оперного театра в России неотделим
от  дворцовой  обстановки.  Широкий   масштаб  и  высокий  уровень  ее
театральной жизни был обеспечен  ничем не ограничиваемой эксплуатацией
крепостного  населения. Даже  в  XIX веке  «Кольцо нибелунга»  Вагнера
обрело  сценическую  жизнь  только  потому,  что  после  двадцати  лет
отчаяния и  бесплодной борьбы  с материальными  трудностями композитор
наконец  нашел   «феодального»  покровителя  в  лице   короля  Людвига
Баварского.

Пусть  все  выдающиеся  музыканты  --- от  Генделя,  Глюка  и  Моцарта
до  Бетховена,   Вебера  и  Глинки  ---   восставали  против  эстетики
артистической  космополитической оперы.  Тем не  менее их  собственное
творчество было  подготовлено длительным развитием  оперного искусства
именно   в  придворно-аристократической   среде\footnote{До  сих   пор
бытует  ложное   представление,  будто  венецианские   оперные  театры
функционировали  в  XVII веке  на  чисто  коммерческой основе  и  были
видом  демократического  буржуазного  искусства.  На  самом  деле  все
оперные театры Венеции,  прикрывавшиеся названиями церковных приходов,
содержались разными семьями венецианских патрициев.}.

На протяжении двух столетий оперные постановки являлись одновременно и
атрибутами дворцовой  жизни, и центрами творческих  исканий, и школами
музыкального профессионализма.

И   в  других   отношениях   связи  музыки   с  придворной   культурой
оказались   для  нее   значительными  и   плодотворными.  Речь   идет,
во-первых,  о   последовательном  тяготении  к   нарочито  возвышенным
образам. Внешне культивируемое  рыцарское начало, неотъемлемый признак
аристократического склада ума, оберегало  придворное искусство, даже в
самых его пустых проявлениях, от приземленности, вульгарности, грубого
юмора. Вспомним,  что образ рыцаря  в европейской поэзии  и литературе
издавна  служил символом  достоинства человеческой  личности. Условный
«дух Дон-Кихота»  витает над  дворянской культурой  не только  в эпоху
Возрождения, но  и в  постренессансные века.  И эта  приподнятость над
обыденным, стремление  избежать низменной реальности  определяют общую
атмосферу музыкальных поисков эпохи барокко и классицизма. Характерно,
что даже комические образы и комедийные жанры, постепенно вторгавшиеся
в  постренессансную   музыку  Европы,   останутся  не   тронутыми  той
утрированной фривольностью  и тенденцией к пошловатому  юмору, которые
станут позднее характеризовать легкожанровую сферу в музыке.

Во-вторых,  по-своему преломился  в  творчестве выдающихся  художников
и   господствующий   в   придворной   эстетике   закон   самодовлеющей
красоты.   Самые  глубокие   их   философские  произведения   отмечены
внешней  чувственной красотой,  а  черты орнаментальности  приобретают
выразительный  смысл. В  музыке XVII  и XVIII  столетий каждая  деталь
«говорит»  на  языке  прекрасных   звучаний  и  форм.  Ласкающая  слух
мелодичность  и  красота  тембров,  неведомые в  музыке  более  ранних
времен;  мягкие гармонические  созвучия; симметрия  и уравновешенность
выражения  --- все  это сочетается  с возвышенной  мыслью, драматизмом
чувств, масштабностью концепции.

Вне  придворной культуры  не  созрела  бы, как  ни  странно это  может
показаться  на первый  взгляд, и  новая духовная  музыка, связанная  с
протестантством.

Протестантская церковь  не смогла  соперничать с  католическим собором
средневековья  и  Возрождения   как  музыкально-творческий  центр.  За
исключением немецких  княжеств и  Англии реставрационного  периода, ни
в  одной стране,  где  господствовало протестантское  вероисповедание,
не  появилось   духовной  музыки,  способной  выдержать   сравнение  с
прошлыми  достижениями католицизма  в этой  сфере. Оставляя  в стороне
внемузыкальную  проблему  близости   форм  протестантских  религиозных
отправлений  к  бытовой  демократической  культуре,  подчеркнем  лишь,
что  ни  в одной  протестантской  стране  ---  ни  в Швейцарии,  ни  в
Швеции,  ни  в  Дании,  ни  в  Шотландии,  ни  в  странах  Прибалтики,
ни  в  английских колониях  Нового  Света  --- не  появилось  духовной
музыки, которая  поднялась бы  до художественного  и профессионального
уровня современности.  Почему в Англии подлинно  художественная музыка
протестантской  литургии  была  создана  только  в  краткий  начальный
период  восстановления монархии  и  жизнь  ее полностью  исчерпывается
творчеством одного  композитора? Почему  в Нидерландах с  внедрением в
жизнь протестантства исчезли без следа богатейшие традиции религиозной
музыки недавнего прошлого?

Ответить  на  это  можно   не  столько  анализируя  причины,  мешавшие
созреванию  протестантской школы  в музыке,  сколько выяснив  условия,
содействовавшие  ее возникновению  в  XVII ---  первой половине  XVIII
столетий  (после  Баха,   как  известно,  сколько-нибудь  значительные
явления в этой сфере отсутствуют).

Германия  ---  блистательное  исключение  в общей  картине.  Она  дала
миру  великую религиозную  музыку  Шютца, Баха  и  Генделя, не  говоря
уже  о  множестве  других,  менее ярких  фигур.  В  этой  экономически
отсталой, феодально-раздробленной стране с многочисленными княжествами
и,   соответственно,  многочисленными   дворами  (каждый   из  которых
по   мере  возможности   стремился   подражать  художественной   жизни
французской  столицы)  было  много   мощных  феодалов,  примыкавших  к
протестантству.  Они  не   просто  поддерживали  лютеранскую  церковь,
но  и  усиленно  культивировали  музыкальные  проявления  религиозного
чувства.  В  этом  процессе литургические  формы  заимствовали  многое
из  господствовавшей светской  культуры. Практиковались,  в частности,
съезды  высокопоставленных дворян,  приверженцев протестантской  веры,
которые  сопровождались   роскошными  музыкальными   празднествами.  И
хотя  согласно  духу  съездов  главное  место  в  программах  занимала
литургическая музыка,  тем не менее  очень широко были  представлены и
светские  номера в  лучших  ренессансных традициях.  О богатстве  этих
музыкальных фестивалей  можно судить  хотя бы по  огромному количеству
музыкантов,  принимавших  в  них участие\footnote{В  качестве  примера
назовем  съезд в  Наумбурге,  на котором  присутствовал молодой  Шютц,
приехавший  туда с  ландеграфом  Морицем.  Саксонский курфюрст  Иоганн
Первый захватил с  собой всю свою огромную капеллу и  сверх того более
двадцати  музыкантов; с  Сигизмундом Бранденбургским  прибыли тридцать
три трубача,  шесть вокалистов-англичан, несколько  итальянских певцов
вокалистов и т. п.}.

Гениальный Шютц, основоположник немецкой  национальной школы в музыке,
всю  свою  жизнь проработал  при  дворе  мощного феодала-деспота,  где
сочинил множество светских дивертисментов  к дворцовым празднествам. И
одновременно там же для придворной церкви создавал величайшие духовные
произведения,  непосредственно связанные  с протестантской  литургией.
Его  огромный  опыт работы  в  сфере  светской музыки,  художественное
окружение, неотделимое  от дворцовой культуры, помогли  ему преодолеть
суровую прямолинейность  музыки протестантской  традиции и  придать ей
высоту  мысли,  богатство  мироощущения  и  внешнюю  красоту,  которые
характерны  для   величайших  светских  творений  той   эпохи.  Музыка
Шютца  оказалась  очень далекой  от  простоты  хоралов времен  Лютера.
Тем  более  были  чужды  ей  закостенелый  конформизм  и  аскетическая
строгость  постлютеранского гимна,  к которому,  по существу,  свелась
протестантская литургия  в странах, лишенных  художественной атмосферы
аристократических дворов.

«Внепротестантские» влияния  на лютеранскую музыку  Германии принимали
самые  разные  формы.  Достаточно  сопоставить  духовные  произведения
Шютца  с  аналогичной музыкой  Баха.  У  Шютца, начавшего  сочинять  в
первой  четверти  XVII века,  ярко  ощутима  близость к  ренессансному
искусству,  к  мадригалу,   красочным  инструментально-хоровым  жанрам
венецианской  католической  школы.  Бах  же творил  в  эпоху  расцвета
оперного  искусства,  во   время,  когда  окончательно  сформировались
также новые  инструментальные жанры  --- concerto  grosso, оркестровая
сюита,  клавесинная  миниатюра  и  т.  п.  И  потому  в  его  духовной
музыке звучат  и оперный  речитатив, и  колоратурные арии,  и развитые
инструментальные  эпизоды в  духе новейшей  скрипичной школы  и т.  п.
Таким  образом,  различие  стилей  Шютца и  Баха  в  \emph{религиозном
искусстве}  непосредственно  отражает  общую  эволюцию  \emph{светской
постренессансной}  музыки   между  началом  XVII  и   серединой  XVIII
столетий.

Яркий  пример,  иллюстрирующий  высказанное  положение,  являет  собой
музыка английского  богослужения. Хорошо  известно, как  бедна, уныла,
однообразна  была  духовная  музыка  пуритан. В  период  их  диктатуры
все инструментальные  и сколько-нибудь  сложные хоровые  звучания были
окончательно  изгнаны  из   религиозного  обихода.  С  восстановлением
монархии,  когда   вернулся  королевский  двор  с   его  недолговечной
претензией на богатую художественную жизнь, вновь появилась придворная
«маска»  и другие  виды  дивертисментов, и  тогда  же Перселл  получил
от  короля  заказ сочинить  музыку  для  придворной церкви.  Результат
оказался изумительным.  Антемы Перселла  стоят в  одном ряду  с самыми
блестящими достижениями музыкального творчества его современности. Все
они  проникнуты  светским  духом,  все широко  опираются  на  новейшую
художественную психологию той эпохи.

Но  в  Англии,  классической  стране капитализма  ---  «стране  мелких
торгашей»  ---  дворянство  в  целом, королевский  двор  в  частности,
вскоре  перестали быть  сколько-нибудь  влиятельным фактором  развития
общественной мысли  и художественной жизни. Подобно  тому, как Перселл
оказался  последним  английским  композитором  мирового  значения  (до
так  называемого  «Английского  возрождения»   конца  XIX  ---  начала
XX  века),  так  и  его  антемы  на  протяжении  последующих  столетий
либо  полностью  исчезли  из   церковной  практики,  либо  подверглись
жесточайшему упрощению. Другой духовной музыки перселловского уровня в
постренессанской Англии более не появилось.

Как  же должно  было развиваться  музыкальное искусство  в Америке,  в
обществе,  никогда  не  знавшем  дворцовой жизни  и  потому  полностью
лишенном созданной ею художественной атмосферы?

С  каким   профессиональным  уровнем  музыки,  с   какими  ее  жанрами
и  направлениями  встретились  в   северном  Новом  Свете  невольники,
прибывавшие из Африки?

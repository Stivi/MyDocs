
\textbf{Джованни Боккаччо}


\textbf{ДЕКАМЕРОН}

\textbf{Вступление}

\emph{Начинается книга, называемая ДЕКАМЕРОН, прозываемая ПРИНЦ ГАЛЕОТТО, в коей содержится сто повестей, рассказанных на протяжении десяти дней семью дамами и тремя молодыми людьми}

\bigskip{}

Соболезновать страждущим --- черта истинно человеческая, и хотя это должно быть свойственно каждому из нас, однако ж в первую очередь мы вправе требовать участия от тех, кто сам его чаял и в ком-либо его находил. Я как раз принадлежу к числу людей, испытывающих в нем потребность, к числу людей, кому оно дорого, кого оно радует. С юных лет и до последнего времени я пылал необычайною, возвышеннейшею и благородною любовью, на первый взгляд, пожалуй, не соответствовавшей низкой моей доле, и хотя умные люди, которым это было известно, хвалили меня и весьма одобряли, со всем тем мне довелось претерпеть лютейшую муку, и не из-за жестокости возлюбленной, а из-за моей же горячности, чрезмерность коей порождалась неутоленною страстью, которая своею безнадежностью причиняла мне боль нестерпимую. И вот, когда я так горевал, веселые речи и утешения друга принесли мне столь великую пользу, что, по крайнему моему разумению, я только благодаря этому и не умер. Однако по воле того, кто, будучи сам бесконечен, установил незыблемый закон, согласно которому все существующее на свете долженствует иметь конец, пламенная любовь моя, которую не в силах были угасить или хотя бы утишить ни мое стремление побороть ее, ни дружеские увещания, ни боязнь позора, ни грозившая мне опасность, с течением времени сама собой сошла на нет, и теперь в душе моей осталось от нее лишь то блаженное чувство, какое она обыкновенно вызывает у людей, особенно далеко не заплывающих в бездны ее вод, и насколько мучительной была она для меня прежде, настолько же ныне, когда боль прошла, воспоминания о ней мне отрадны.


Но хотя кручина моя унялась, участие, которое приняли во мне те, кто из доброго ко мне расположения болели за меня душой, не изгладилось из моей памяти, и я твердо уверен, что перестану об этом помнить, только когда умру. А так как, по моему разумению, благодарность есть самая похвальная изо всех добродетелей, неблагодарность же заслуживает самого сурового порицания, то я, дабы никто не мог обвинить меня в неблагодарности, порешил, раз я теперь свободен, возвратить долг и по мере возможности развлечь если не тех, кто меня поддерживал, --- они-то, может статься, в силу своего благоразумия или по воле судьбы как раз в том и не нуждаются, --- то, по крайности, тех, кто испытывает в том потребность. И хотя моя поддержка и мое утешение будут, наверное, слабы, все же мне думается, что поддерживать и утешать надлежит главным образом тех, кто особливую в том имеет нужду: пользы им это принесет больше, чем кому бы то ни было, они же это больше, чем кто-либо другой, оценят.


А кто станет отрицать, что подобного рода утешение, сколько бы ни было оно слабо, требуется не так мужчинам, как милым женщинам? Женщины от стыда и страха затаивают любовный пламень в нежной груди своей, а кто через это прошел и на себе испытал, те могут подтвердить, что огонь внутренний сильнее наружного. К тому же, скованные хотеньем, причудами, веленьями отцов, матерей, братьев, мужей, они почти все время проводят в четырех стенах, томятся от безделья, и в голову им лезут разные мысли, далеко не всегда отрадные. И если от этих мыслей, вызванных томлением духа, им иногда взгрустнется, то грусть эта, на великое их несчастье, не покидает их потом до тех пор, пока что-нибудь ее не рассеет. Что же касается влюбленных мужчин, то они не столь хрупки: с ними этого, как известно, не бывает. Они располагают всевозможными средствами, чтобы развеять грусть и отогнать мрачные мысли: захотят --- прогуляются, поглядят, послушают, захотят --- зачнут птицу бить, зверя травить, рыбу ловить, на коне гарцевать, в карты играть, торговать. В каждое из этих занятий мужчина волен вложить всю свою душу или, по крайности, часть ее и, хотя бы на некоторое время, от печальных мыслей избавиться, и тогда он успокаивается, а если горюет, то уже не столь сильно.


Так вот, с целью хотя бы частично загладить несправедливость судьбы, слабо поддерживающей как раз наименее крепких, что мы видим на примере нежного пола, я хочу приободрить и развлечь любящих женщин, --- прочие довольствуются иглой, веретеном или же мотовилом, --- и для того предложить их вниманию сто повестей, или, если хотите, побасенок, притч, историй, которые, как вы увидите, на протяжении десяти дней рассказывались в почтенном обществе семи дам и трех молодых людей во время последнего чумного поветрия, а также несколько песенок, которые пели дамы для собственного удовольствия. В этих повестях встретятся как занятные, так равно и плачевные любовные похождения и другого рода злоключения, имевшие место и в древности, и в наше время. Читательницы получат удовольствие, --- столь забавны приключения, о коих здесь идет речь, и в то же время извлекут для себя полезный урок: они узнают, чего им надлежит избегать, а к чему стремиться. И я надеюсь, что на душе у них станет легче. Если же так оно, бог даст, и случится, то пусть они возблагодарят Амура, который, избавив меня от своих цепей, тем самым дал мне возможность порадовать их.


\textbf{Начинается первый день ДЕКАМЕРОНА,}

\emph{в продолжение коего, после того как автор сообщит, по какому поводу собрались и о чем говорили между собою лица, которые будут действовать дальше, собравшиеся в день правления Пампинеи толкуют о том, что каждому больше по душе}

\bigskip{}

Обворожительнейшие дамы! Зная ваше врожденное мягкосердечие, я убежден, что предисловие к моему труду покажется вам тяжелым и печальным, ибо таково воспоминание, с которого оно начинается, --- воспоминание о последнем чумном поветрии, бедственном и прискорбном для всех, кто его наблюдал и кого оно так или иначе коснулось. Не подумайте, однако ж, что вся книга состоит из рыданий и стонов, --- я вовсе не намерен отбивать у вас охоту читать дальше. Страшное начало --- это для вас все равно, что для путников высокая, крутая гора, за которой открывается роскошная, приветная долина, тем больше отрады являющая взорам путников, чем тяжелее достались им восхожденье и спуск. Подобно как бурная радость сменяется горем, так же точно вслед за испытаньями приходит веселье. Минет краткая эта невзгода (краткая потому, что я описываю ее в немногих словах), и настанет блаженная пора утех, --- я вам их предрекаю, а то после такого вступления вряд ли можно было бы их ожидать. Откровенно говоря, если б у меня была возможность более удобным путем, а не по крутой тропинке, привести вас к желанной цели, я бы охотно это сделал, но если не начинать с воспоминания, то будет непонятно, как произошло то, о чем вам предстоит прочесть, --- словом, мне без него не обойтись.


Итак, со времен спасительного вочеловеченья сына божия прошло уже тысяча триста сорок восемь лет, когда славную Флоренцию, лучший город во всей Италии, посетила губительная чума[1]; возникла же она, быть может, под влиянием небесных тел, а быть может, ее наслал на нас за грехи правый гнев божий, дабы мы их искупили, но только за несколько лет до этого она появилась на Востоке и унесла бессчетное число жизней, а затем, беспрестанно двигаясь с места на место и разросшись до размеров умопомрачительных, добралась наконец и до Запада.


%\documentclass[a6paper,14pt]{extarticle}
\documentclass[a6paper,12pt]{extarticle}

%\usepackage{amssymb,amsfonts,amsmath,mathtext,cite,enumerate,float}
%\usepackage{amsfonts}
%\usepackage{skak}
% шрифт как в ссср
% заработал после http://kalina.lug.ru/wiki/Установка_комплекта_шрифтов_PSCyr_на_TeXLive_в_Ubuntu_9.04
\usepackage{pscyr}
\usepackage[utf8]{inputenc}
%\usepackage[english=nohyphenation,russian=nohyphenation]{hyphsubst}
\usepackage[english,russian]{babel}
% Для поиска по русским буквам
\usepackage{cmap}
\usepackage{indentfirst}        % Отступ в первом абзаце.
%%%%%
\usepackage{geometry}           % способ ручной установки полей
\geometry{top=0.5cm}            % поле сверху
\geometry{bottom=1.25cm}        % поле снизу
\geometry{left=0.2cm}           % поле справа
\geometry{right=0.2cm}          % поле слева
%%%%%
%%%%% \linespread{0.95}
%%%%%
%%%%% \usepackage{wrapfig}
%%%%%
%%%%% \ifx\pdfoutput\undefined
%%%%% \usepackage{graphicx}
%%%%% \else
%%%%% \usepackage[pdftex]{graphicx}
%%%%% \fi
%%%%%
%%%%% \usepackage{textcomp}
%%%%%
%%%%% \renewcommand{\bfdefault}{b}
%%%%% %\righthyphenmin=4 % Минимальное число символов при переносе - 2.
%%%%% %\righthyphenmin=30 % Минимальное число символов при переносе - 2.
%%%%%
%%%%%
%%%%% % LaTeX Paragraph Indenting
%%%%% %\setlength{\parindent}{0pt}
%%%%% %\setlength{\parskip}{1ex}
%%%%%
%%%%% \usepackage{parallel}
%%%%%
%%%%% % позволяет задавать цвет текста и фона,
%%%%% % как отдельного блока, так и всего
%%%%% % документа
\usepackage[usenames,table]{xcolor}
\definecolor{dark_purple}{HTML}{200020}
\definecolor{myblue}{HTML}{00FFFF}
\definecolor{newblue}{HTML}{0000FF}
\definecolor{conjunctions}{HTML}{33B5E5}
%%%%% % --------------------------------------
%%%%% \newenvironment{answer}{%
%%%%%   \bfseries
%%%%%   \color{blue}
%%%%% }{}
%%%%% % --------------------------------------
\color{white}
%%%%%
%%%%%
%%%%% %\setlength{\parindent}{0pt}
%%%%%
%%%%% %\maketitle
%%%%%
%%%%% % Нумерация страниц выключена
\pagestyle{empty}
%%%%%
\sloppy
%%%%%
%%%%% % Буквица
%%%%% \usepackage{type1cm}
%%%%% \usepackage{lettrine}
%%%%%
%%%%%
%%%%%
%%%%% % http://kuscsik.blogspot.ru/2008/03/how-to-background-image-in-latex.html
%%%%% % in preamble
%%%%%
%%%%% %%% \usepackage{tikz}
%%%%% %%% \usepackage{eso-pic}
%%%%% %%% \newcommand\BackgroundPic{
%%%%% %%% \put(0,20){
%%%%% %%% \parbox[b][\paperheight]{\paperwidth}{%
%%%%% %%% \vfill
%%%%% %%% \centering
%%%%% %%%
%%%%% %%% \begin{tikzpicture}[rounded corners,ultra thick]
%%%%% %%%   \shade[ball color=green] (9,.5) circle (.3cm);
%%%%% %%% \end{tikzpicture}
%%%%% %%%
%%%%% %%% \vfill
%%%%% %%% }}}
%%%%%
%%%%%
\usepackage[protrusion=true,expansion=true]{microtype}
\usepackage[T2A]{fontenc}
%%%%%
%%%%% % Создание гиперссылок. Пакет загружен последним,
%%%%% % так как он переопределяет многие команды LaTeX.
%%%%% %\usepackage{hyperref}
%%%%% %\usepackage[unicode,bookmarks]{hyperref}
%%%%% \usepackage[unicode,bookmarks,bookmarksnumbered,pdfborder={0 0 0}]{hyperref}
%%%%%
%%%%% %\usepackage{endnotes}
%%%%% %\let\footnote=\endnote
%%%%%
%%%%% \usepackage[raggedright]{titlesec}
%%%%%
%%%%%
%%%%% % vim: fdm=syntax


 Станислав Лем.

 Человек с Марса




Stanislaw Lem. Czlowiek z Marsa (1947). Пер. с польск. --- Е.Вайсброт.

Авт.сб. "Человек с Марса". М., "Эксмо-Пресс", 1998.

OCR \& spellcheck by HarryFan, 11 April 2001






1

\bigskip{}


Улица  жила.  Лязг  вагонов  надземки,  сигналы  автомашин,  громыханье
мчащихся  троллейбусов,  мощный  гул   человеческих   голосов   кипели   в
темно-синем воздухе, разрываемом на клочья тьмы снопами огней всех  цветов
и  оттенков.  Толпа  переливалась  как  множество  змей,  плотно  заполняя
тротуары, поблескивая в светлых квадратах витрин и погружаясь  в  полумрак
домов. Только что омытый асфальт шипел под сотнями автомобильных шин. Одно
за одним проносились скользкие на вид черные и  серебристые  тела  длинных
машин.

Я шел без цели и мысли, втиснутый  в  толпу,  ставший  ее  нераздельной
частицей, позволяя нести себя, как волна несет пробку.

Улица дышала, ворчала и гудела,  меня  омывали  потоки  света  и  струи
тяжелого аромата женских духов, охватывал то резкий дым южных сигарет,  то
сладковатый, удушающий запах опиумированных  сигар.  По  фасадам  домов  в
головокружительном темпе взбегали неоновые буквы угасающих и разгорающихся
вновь реклам, взвивались фонтаны огней, мигали сумасшедшие сполохи ракет и
фейерверков, осыпаясь последними искрами на головы толпы.

Я проходил под гигантскими, ярко освещенными порталами, шел мимо темных
магазинов, надменных колонн  каких-то  незнакомых  зданий:  погруженный  в
подвижную, многоязыкую, ни на  мгновение  не  умолкающую  массу  людей,  и
все-таки более одинокий, чем на  необитаемом  острове.  Пальцы  машинально
перебирали в кармане два пятицентовика, составлявших весь мой капитал.

На пересечении трех больших улиц, каменные жерла которых уходили  вдаль
сужающимися в перспективе шеями с позвоночником фонарей,  я  отделился  от
толпы и остановился на бордюре.

В зависимости от  цвета  загорающихся  огней  толпа  переползала  через
проезжую часть, словно выпущенная из какого-то гигантского шлюза.  Гудели,
выли,   рычали   моторы   автомобилей;   время   от   времени   раздавался
душераздирающий визг тормозов. Пробегающий мимо разносчик сунул мне в руку
какую-то ненужную газету. Я купил ее, чтобы  только  отделаться  от  него,
засунул за манжет и продолжал наблюдать.

Толпа, вообще-то говоря, всякий раз была иной,  одновременно  оставаясь
той  же  самой.  Улица  продолжала  пульсировать  в  двух  противоположных
направлениях, пропуская через свою асфальтированную горловину человеческие
массы попеременно с блестящими железяками автомобилей.

Неожиданно с узкой полосы проезжей части  съехала  огромная,  блестящая
тень и с  тихим  шуршанием  покрышек  остановилась  около  меня.  Это  был
гигантский  "бьюик".  Опустилось  правое   переднее   стекло   и   изнутри
послышалось:

"--* Что у вас за газета?

Одновременно рука в тяжелой шоферской перчатке указала на белый краешек
бумаги, торчащий у меня из-за рукава.

Тон,  каким  был  задан  вопрос,  и  его  содержание  показались   мне,
разумеется, очень странными, но жизнь научила меня ничему  не  удивляться,
особенно в крупных городах. Я ответил, вытащив газету (поскольку и сам  не
знал ее названия):

"--* "Нью-Йорк таймс".

"--* А какое сегодня число? Какой день? --- спросил тот же голос.

Эта дурацкая игра мне наскучила.

"--* Пятница! --- ответил я, чтобы отвязаться.

В тот же момент дверь автомобиля открылась и голос приказал:

"--* Садитесь.

Я сделал такое движение, словно хотел попятиться.

"--* Быстро! --- было произнесено с такой силой, что я невольно послушался.

Не знаю, как я упал на мягкие подушки, как дверь  хлопнула  и  тут  же,
словно в гангстерских фильмах, машина рванулась с  места.  Уличные  фонари
задрожали, вытянулись в пульсирующую ленту --- мы мчались вперед.

Я осмотрелся. В машине было темно. Я  сидел  один  на  заднем  сиденье.
Впереди, на фоне слабо  освещенной  приборной  доски  и  лобового  стекла,
маячили два почти одинаковых мужских силуэта --- водитель и его  спутник.  Я
принялся раздумывать. Правда, ум  мой  был  немного  "тронут"  вынужденным
двухдневным недоеданием, но  работал  достаточно  исправно.  Голод  скорее
подталкивал к нетривиальным решениям и  вызывал  некоторое  безразличие  к
внешним событиям. Но  сейчас...  А  собственно,  что  сейчас  происходило?
Машина, по-видимому, выехала на какую-то не столь забитую улицу, поскольку
мотор начал  издавать  тот  характерный  высокий  тон,  который  свойствен
высокооборотным агрегатам, работающим на полном  газу.  Неожиданно  резкий
поворот --- тормоза, на  которые  вдруг  сильно  нажали,  запищали,  машина,
несколько  раз  мягко  подпрыгнув,  въехала  в   какое-то   углубление   и
остановилась.

Двери не  открылись.  Водитель  дал  сигнал  ---  один  короткий,  второй
длинный. Мигнул яркими фарами, переключил на слабые, а потом вырубил и их.
Мы оказались в кромешной тьме.

"--* Что за комедия, черт побе... --- начал я громко, но голос мой прозвучал
слишком слабо, уши еще были полны шумом мотора. Впрочем, в тот  же  момент
перед носом машины  вспыхнул  четырехугольник  слабого  света.  Автомобиль
заворчал и двинулся вперед. Неожиданно я почувствовал, как пол опускается.
"Ага, --- подумал я. --- Подземный гараж". И тут мы остановились.

Двери машины открылись. Водитель повернулся ко мне  лицом  ---  огромным,
широким с мощными челюстями и кустистыми бровями, лицом одновременно сухим
и мясистым. Я вышел. Ноги ступали легко --- полы в  этой  подземной  галерее
были из заглушающего звуки материала.  Потом  открылась  какая-то  боковая
дверь и стал виден темно-голубой зал, в котором сидело пятеро мужчин.  Зал
был невелик.  Мужчины,  сидевшие  за  маленьким  круглым  столом,  тут  же
поднялись и молча уставились на меня, как бы ожидая чего-то.

Самый невысокий, темный блондин, человек среднего возраста,  со  слегка
одутловатым,   бледным   и   блестящим   лицом,    обратился    к    моему
спутнику-водителю.

"--* Это он?

Водитель, казалось, был немного удивлен вопросом, замялся, но ответил:

"--* Конечно.

Теперь спрашивающий обратился ко мне, подойдя  так,  что  мы  оказались
лицом к лицу:

"--* Какой сегодня день?

На этот раз я ответил, уже не отступив от истины: дескать,  среда  ---  и
это вызвало как бы дрожь, пробежавшую по  лицам  присутствующих.  Какое-то
мгновение я думал, что  оказался  среди  сумасшедших,  но  даже  не  успел
испугаться, потому что водитель, человек  атлетического  сложения,  быстро
шагнул вперед и заговорил:

"--* Господин Фрэйзер,  клянусь,  он  сказал  "пятница".  И  у  него  была
"Нью-Йорк таймс". И стоял он на углу Пятой улицы.

"--* В чем дело? --- спросил мужчина с бледным лицом. --- Откуда вы взялись?

"--* Из Чикаго, --- ответил я. --- Может, теперь моя очередь задавать вопросы?
Что значит это сборище? И странная поездка на машине?

"--* Не напрягайтесь, --- прервал он ледяным тоном. --- Ваша очередь  задавать
вопросы еще не настала. Почему вы сказали, что сегодня пятница?

У меня мелькнула мысль, что все-таки передо мной психи. Надо  бы  вести
себя поуступчивее и мягче. Где-то я читал об этом.

"--* Если как следует подумать, --- начал я, --- то, может, и  верно  пятница.
Особенно если считать по Гринвичу...

"--* Не городите чепухи, ближе к делу. Письмо и инструменты при вас?

Я молчал.

"--* Так... --- медленно сказал мой собеседник. ---  Так.  Ну,  прежде  чем...
прежде чем... Короче, скажите нам, кто вас  подослал?  С  какой  целью  вы
приехали? И кто вам сказал, что и как следует сделать, чтобы попасть сюда?

Последние слова он чуть ли не прошипел, показав при этом зубы,  которые
были  еще  белее  или,  скорее,  бледнее,  чем  лицо.  Остальные   четверо
по-прежнему стояли неподвижно, уставившись на меня не то угрожающе, не  то
выжидающе.

Понемногу я начал что-то соображать. Во  всяком  случае,  это  не  были
психи. Нет. Я оказался полным идиотом. И влип в какую-то паршивую историю.

"--* Господа, --- начал я. Беззаботный тон тут явно был не к месту, однако я
продолжал держать марку: --- Господа, я --- репортер, то есть  был  репортером
"Чикаго уорд". По некоторым причинам два месяца тому назад меня уволили. В
поисках работы я приехал в Нью-Йорк. Я  здесь  уже  несколько  недель,  но
ничего не нашел, а что касается того, как я к вам попал,  то  уверяю  вас,
это чистая случайность. Думаю, никому не возбраняется  покупать  "Нью-Йорк
таймс"?

"--* И, отвечая на вопрос о дне недели,  говорить  в  среду,  что  сегодня
пятница... Так, да?

Услышав слова,  впервые  произнесенные  высоким  худощавым  мужчиной  в
очках, я  повернулся  к  нему  и  одновременно  отметил,  что  дверь  была
загорожена.   Там   стоял   водитель   машины   с   массивным,   каменным,
невыразительным лицом и целиком заполнял собою дверной проем. Я понял, что
они мне не верят.

"--*  Господа,  ---  начал  я.  ---  Это  глупое   стечение   обстоятельств...
Пожалуйста, позвольте мне уйти... Я ведь ничего не знаю и не  понимаю.  Не
знаю даже, где нахожусь.

"--* Вы, похоже, не  ориентируетесь  в  ситуации,  ---  медленно  проговорил
мужчина с бледным блестящим лицом. --- Вы не можете отсюда уйти.

"--* Сейчас --- нет. А когда?

"--* Никогда.

Мне сразу как бы полегчало. Теперь все стало ясно. Те четверо медленно,
не спеша сели, прикурили сигареты от маленькой масляной лампы, а я глядел.
Я глядел с особой жадностью на их движения, на ярко освещенную комнату, на
лицо стоящего передо мной человека, выносившего мне приговор. "Наверно,  ---
думал я, --- надо что-то сказать, просить,  убеждать,  подробно  объяснять?"
Объяснять? Но стоило заглянуть в его глаза, блекло-голубые,  далекие,  как
становилось понятно, что любые мои заверения не имеют смысла.

"--* Ничего не понимаю, --- сказал я, выпрямляясь. --- Я устал и голоден. Я не
знаю, за что  должен  умирать.  И  зачем.  Но  даже  людоеды  кормят  свои
жертвы... Простите, я голоден. --- Я замолчал, подошел  к  столу,  вынул  из
коробки сигарету и прикурил от пламени лампы.

И тут я заметил, что мужчины молча переглянулись, потом --- поверх меня ---
глянули на того, который разговаривал со мной, вроде бы их предводителя, и
снова застыли. Дверь была забаррикадирована телом, загораживающим доступ к
ручке, --- оно тянуло фунтов на двести. Я не выспался, был утомлен, голоден.
Бороться не имело смысла.

"--* Дайте ему поесть, --- сказал бледнолицый мужчина, ---  и  позаботьтесь  о
нем. Но как следует!

Водитель молча отворил дверь и подал мне знак.

"--* Спокойной ночи, господа, --- сказал я и последовал за ним.

Дверь хлопнула, я вышел в полумрак коридора.

В тот же момент меня схватили две сильные руки, послышался щелчок, и  я
почувствовал на запястьях холод наручников.

"--* Так-то вы обходитесь с гостями, --- бросил я, не поднимая головы.

Шофер и его невидимый в темноте помощник, сковавший меня,  были  не  из
болтливых.  Один  тщательно  ощупал  мои  карманы  и,  не   найдя   ничего
подозрительного, легонько подтолкнул меня вперед.

Я понял это как приглашение к завтраку. Мы шли  в  тьме  египетской  не
меньше минуты, и тут мой провожатый остановился так резко, что я чуть было
не налетел на неожиданно выросшую передо мной, до  того  невидимую  стену.
Раздался глухой щелчок, и открылась дверь --- прямоугольник света.

Новое помещение напоминало банковское хранилище, вернее сказать, такими
их представляют любители  детективных  романов.  Огромные  стальные  двери
захлопнулись  за  спиной  у  меня  и  моего  провожатого,  заблокированные
гигантскими когтями задвижек, ушедшими в  пазы  дверной  коробки.  Комната
ярко освещалась голой лампочкой. Стены были образованы правильными  рядами
стальных  дверок  с  массивными  ручками   и   многочисленными   замочными
скважинами.  Единственной  мебелью  были  два  стоящих  на  бетонном  полу
низеньких стульчика, табурет о трех ножках и небольшой столик.  Странно  ---
все предметы были из стали. Правда, заметил я это  лишь  после  того,  как
шофер пододвинул ко мне ногой табурет --- тот издал характерный звук.

Я  сел,  шофер  подошел  к  столику,  приподнял  крышку  и  достал   из
открывшегося таким манером  ящика  несколько  банок  консервов  и  длинную
булку. Потом извлек  из  кармана  огромный  складной  нож,  открыл  нужное
острие, вспорол одну банку, тем же ножом нарезал  хлеб  и  снова  принялся
шарить по карманам. Наконец он вытащил ключик от моих наручников ---  именно
в тот момент, когда я уже подумал, что он  собирается  кормить  меня  сам.
Потом сел напротив и принялся наблюдать за моими достаточно  однообразными
действиями. Созерцание продолжалось  до  тех  пор,  пока  не  опорожнилась
банка. Я взглянул на следующую --- омары (я очень люблю омаров) --- и протянул
руку: нож. Шофер немного покривил  свою  загорелую  массивную  физиономию,
что, видимо, должно было изображать улыбку, отрицательно покачал головой и
сам вскрыл банку.  "Он  меня  боится!"  ---  подумал  я  с  удовлетворением,
поскольку он весил наверняка раза в два больше меня.  Когда  и  эта  банка
опустела и была тщательно протерта корочкой хлеба, я спросил:

"--* Сухой закон?

Шофер снова растянул в улыбке рот, теперь чуть  пошире,  поднял  крышку
стола и достал бутылку отличного коньяка. Я думал, он чокнется со мной, но
он только вынул пробку и поставил передо мной рюмку для  яиц,  которой  я,
однако,  пренебрег.  Солидная  порция  коньяка  просветлила  мне  мозговую
машинерию: я подумал, что оказался в достаточно интересной ситуации, и уже
собрался спросить о возможности хоть как-то поспать в этом паршивом отеле,
когда у меня над головой раздался  низкий  короткий  гудок,  повторившийся
трижды. Шофер едва заметно вздрогнул, вынул наручники и сказал:

"--* Пошли.

Я заколебался --- он  отступил  на  шаг  и  дотронулся  до  подозрительно
выпячивающегося кармана брюк.

"--* Подчиняюсь силе, --- сказал я громко, улыбнулся и подал руки.  Он  тоже
улыбнулся, правда, малость кривовато, открыл дверь,  и  мы  погрузились  в
царящую по другую их сторону темень.

Теперь мы шли явно в другое место, потому что в определенный момент  он
взял меня за руку и потянул. Это  было  вполне  своевременно,  иначе  б  я
растянулся во весь рост на ступенях. Мы поднимались по лестнице. Вскоре  я
заметил постепенно усиливающийся бледно-голубой свет, и наконец мы вошли в
широкий  безоконный  коридор,  стены  которого  освещали  заглубленные   в
поверхность матовые лампочки. Коридор  кончался  дверью  размером  во  всю
замыкающую его стену. Когда мы подошли, шофер  подтолкнул  меня  вперед  ---
дверь сама раскрылась и сама за нами, вернее, за мной закрылась.

Я оказался в огромной библиотеке --- таким было первое  мое  впечатление.
Стены до потолка закрывали книжные шкафы и полки, забитые книгами. Рядом с
полками стояли лесенки, столики с лампами, кресла, в середине ---  небольшой
овальный столик, за которым сидели уже знакомые мне мужчины. Один  ---  тот,
который только однажды обратился ко мне, высокий  и  худощавый,  с  седыми
висками, сверкнул в мою сторону стеклами очков. Я подошел ближе.

"--* Мы только что говорили о  вас,  ---  сказал  этот  человек  медленно  и
довольно тихо. Казалось, он очень утомлен. Я слегка поклонился и  ждал.  ---
Хотелось бы вам верить... Проверка показала,  что,  скорее  всего,  вы  не
солгали...

Я удивленно взглянул на него. Какая  еще  проверка?  Неужто  завтрак  с
молчаливым шофером был проверкой? В таком случае мне приходилось  признать
ее весьма поверхностной. Седой мужчина, казалось, не обратил  внимания  на
мое удивление.

"--*  Вы  не  по  своей  воле  попали  в  определенное...  весьма  сложное
положение, --- было видно, что  он  обдумывает  каждое  слово.  ---  Одно  вам
следует знать: таким, каким вы были до этого, вам отсюда выйти невозможно.

У меня мгновенно мелькнула мысль, что  я  оказался  в  центре  какой-то
идеально организованной гангстерской банды, а  может,  шайки  политических
экстремистов или чего-то в этом роде. Но книги? Книги-то зачем?

"--* Вы не выйдете вообще, либо... ---  он  осекся,  глядя  на  меня  внешне
спокойно, но я чувствовал напряженность.

"--* Либо? --- спросил я. И, обращаясь к тому, кто уже  при  мне  прикуривал
сигарету, добавил: --- Простите, можно вас попросить? Понимаете, я  не  могу
пользоваться руками, а с удовольствием бы закурил.

Он медленно (они все делали в замедленном темпе --- это было  смешно,  но
одновременно и страшно) сунул мне сигарету в рот и  поднес  огонь.  Другие
тут же снова обменялись взглядами.

"--* Либо вы будете нашим... --- докончил мужчина в  очках.  ---  И,  судя  по
вашей внешности, кажется мне, случится именно так.

"--* Внешность бывает  обманчива,  ---  сказал  я,  тоже  стараясь  говорить
медленно, не столько для того, чтобы подладиться под них,  сколько,  чтобы
совладать с действием выпитого после долгого поста  коньяка.  ---  Можно  ли
узнать, в чем дело?

Молчавший до того мужчина с широким бледным лицом поднял голову.

"--* Этого вы, конечно, знать не можете, --- сказал он как бы  извиняясь.  И
добавил громче: --- Да и не все ли вам равно? Все очень  просто:  слушать  и
молчать.

Должен признаться, беседа повергла меня в  весьма  странное  состояние.
Когда  мне  казалось,  что  это  удивительное  общество  обрекло  меня  на
исчезновение, то есть смерть, и я понимал, что мое положение безнадежно, я
вроде бы даже успокоился, но теперь новый поворот пробудил во мне какие-то
странные  силы.  Человека  в  безвыходном  положении  охватывает   апатия,
отупение,  однако  достаточно  малейшего  проблеска  надежды  ---   и   силы
возрастают в сотни раз, все органы чувств обостряются до крайности,  и  он
обращается в сплошной напряженный мускул, чтобы в  бешеном  усилии  спасти
свою жизнь.  Так  было  и  со  мной.  Разговаривая  приглушенным  голосом,
медленно, я одновременно  внимательно  рассматривал  все  окружающее  меня
из-под полуприкрытых век, изучая детали. Бежать?..  А  почему  бы  и  нет?
Конечно, это была крайность. Можно схватить массивную пепельницу со  стола
и запустить в лоб председателю, но это глупо. Гораздо лучше бросить  ее  в
большой  освещавший  зал  светильник.  Однако  надо  было  знать,  сколько
лампочек горят внутри матового шара. Одна или несколько?  От  этого  могло
зависеть все.  Ну,  хорошо,  но  еще  оставались  двери.  Странные  двери,
отворявшиеся и закрывавшиеся как бы самостоятельно. Я стоял к ним спиной и
не знал, была ли у них ручка.

"--* Вы не должны задавать  вопросов,  ---  медленно,  с  нажимом  продолжал
мужчина с бледным  потным  лицом,  сминая  сигарету  в  серебряной  резной
пепельнице.

Сказав это, он  стряхнул  с  манжета  невидимую  пылинку  и  неожиданно
охватил меня своим холодно-голубым взглядом.

"--* Простите... --- улыбнулся я,  слегка  пожав  плечами,  и  глянул  краем
глаза. У дверей была обыкновенная ручка. --- Мне кажется, я все же могу хотя
б в общих чертах...

Один из мужчин,  который,  казалось,  вовсе  не  слушал  наш  разговор,
неожиданно  бросил  несколько  слов  на  каком-то  непонятном  мне  языке.
Странные горловые звуки. Мой собеседник наклонился  над  крышкой  стола  и
сказал быстро и тихо:

"--* Вы согласны?

"--* На что? --- Я любой ценой хотел выиграть время.

"--* У вас есть выбор: либо вы вступите в нашу... --- Он замялся.

"По-видимому, у них нет  практики,  ---  подумал  я.  ---  Это  никакая  не
гангстерская банда, там правят другие законы".

"--*  В  нашу  организацию,  ---  продолжил  мой  собеседник,  ---  либо   вас
обезвредят.

"--* То есть охладят до температуры грунта, так, что ли?

"--* Нет, --- спокойно сказал он. --- Мы вас не убьем.  Просто  проделаем  над
вами маленькую операцию, после чего вы на всю  жизнь  останетесь  идиотом,
психически недоразвитым.

"--* Так... А что мне придется в вашей "организации" делать?

"--* Ничего такого, чего бы вы сделать не могли.

"--* Это противоречит закону?

"--* Чьему закону?

Я был заинтригован.

"--* Ну, как же... нашему закону, закону Соединенных Штатов Америки.

"--* Несомненно... иногда, --- ответил  он.  Все  как  бы  по  приказу  едва
заметно улыбнулись. Я бы сказал: на  мгновение  ожившие  маски.  Я  сделал
медленное движение ногой, чтобы, развернувшись назад, схватить пепельницу.
Но сумею ли кинуть ее в лампу скованными руками? Я был неплохим гимнастом.
В тот же момент мужчина в очках отвернулся от стоящего у столика  олеандра
в шикарной  малахитовой  вазе  и  бросил  несколько  слов,  которых  я  не
расслышал. Дверь раскрылась, и появились шофер с помощником.

"--* Отведите его... в операционную, --- сказал председатель.  ---  И  снимите
наручники.

Шофер подошел ко мне, скрипнул ключик в замке.  В  следующий  момент  я
нанес ему удар стальным браслетом левой, еще  скованной  руки  в  висок  и
добавил сильным ударом ноги в живот. Он упал, не издав ни звука. Но  когда
его грузное тело еще летело на меня, я  схватил  его  за  лацканы  кожаной
куртки и изо всей силы кинул на вскакивающих из-за стола мужчин.  Огромное
массивное тело перевернуло стол, несколько кресел  свалилось.  Я  не  стал
ждать, что будет дальше, а прыгнул к двери.  Поразительно,  никто  еще  не
выстрелил, а помощник шофера стоял  в  дверях  спокойно,  слегка  раскинув
руки, словно встретил знакомого после долгой разлуки.

Я ударил его левой в подбородок, вернее, я целился в то  место,  но  он
парировал удар ребром ладони так, что я почувствовал  резкую  боль  и  моя
рука беспомощно повисла. Парень знал джиу-джитсу. Мне не повезло.

В этой сумятице, когда я слышал за спиной приближающиеся шаги, у меня в
памяти на мгновение мелькнула фигурка коренастого  маленького  Иши-Хасама,
который учил меня в Киото японской борьбе. На последнем занятии он показал
два интересных приема, которые европейцы не знают. Это  удар  снизу  двумя
руками, которые словно ножницы переламывают гортань. Удар,  нанесенный  со
всей силой отчаяния, удался только частично. В тот  момент,  когда  я  уже
почувствовал его напряженное тело, несколько  сильных  рук  схватили  меня
сзади. Я бросился на пол, но схватка длилась недолго. Из массы рук и ног я
выбрался, но тут меня крепко схватили за одежду и --- о  диво!  ---  отвели  к
столику.

Здесь один из задыхающихся после боя мужчин пододвинул  мне  кресло,  а
когда я, обалдев и дрожа, повалился в него, второй сунул мне в рот длинную
сигарету, третий подал огня, и теперь все сидели вокруг меня словно  после
краткого перерыва в дружеской беседе.

Шофер быстро убрался вместе с помощником.

"--* Вы сдали экзамен. Вы уже наш.  Все  это,  конечно,  была  комедия,  ---
добавил он в ответ на мой изумленный взгляд. --- Мы дали вам шанс, и  вы  им
воспользовались.

"--* Оригинальный способ, --- сказал я, массируя себе  левое  предплечье.  ---
Позвольте узнать, какие шуточки вы еще держите за пазухой.  Сколько  я  ни
работал репортером, ничего подобного со мной еще не случалось.

"--*  Охотно  верю,  ---  сказал  мужчина  с  бледным  лицом.  ---   Разрешите
познакомить вас с присутствующими: доктор Томас Кеннеди, ---  он  указал  на
мужчину в очках, --- синьор Джедевани, инженер Финк. Меня зовут Фрэйзер.

Мужчины наклоняли головы и подавали руки. Я не знал,  злиться  мне  или
смеяться.

"--* А мое имя...

"--* Знаем, знаем прекрасно, господин Макмур, вы ведь из Шотландии, верно?

"--* Простите, господа, может быть, уже довольно шуточек?

"--* Мы прекрасно вас понимаем, --- сказал Фрэйзер. --- Так вот,  все  сидящие
здесь представляют собой организацию, которая, собственно, не ставит перед
собой ни чисто  научных,  ни  финансовых,  ни  даже,  ---  он  улыбнулся,  ---
разбойничьих целей. Не думайте,  ради  Бога,  что  мы  фашисты,  ---  быстро
добавил он, видя, как у меня вытягивается физиономия. --- Мы также  не  клуб
умирающих от скуки миллио...

"--* Так перечислять вы можете целый час, --- язвительно прервал я. ---  Вы  ---
не общество защиты  от  пережаренных  шницелей  и  не  клуб  присмотра  за
собственными карманами...

"--* Дело наше очень трудно понять, а  еще  труднее  в  него  поверить,  ---
впервые заговорил мужчина в черном костюме с  узким,  украшенным  холеными
седоватыми усиками лицом. Председатель назвал его инженером Финком. --- Судя
по всему, вы им не только заинтересуетесь, но отдадите то, что отдали мы.

"--* То есть?

"--* То есть все, ---  сказал  он,  вставая.  Остальные  тоже  поднялись,  а
Фрэйзер повернулся ко мне.

"--* Извольте пройти со мной. Я должен как следует ознакомить вас...

Я поклонился и пошел следом за ним по заглушающему все  звуки  толстому
ковру.

Мы подошли к дверям, которые раскрылись сами,  когда  мы  были  в  двух
шагах от них. Я обратил внимание  на  то,  что  мы  были  одни,  остальные
"заговорщики", как я мысленно их называл, остались в  библиотеке.  Коридор
вывел нас к незнакомой лестнице, ступени которой,  казалось,  вырублены  в
монолитном блоке бетона.  В  стенах  всюду  спокойно  горели  приглушенным
светом квадратные, заглубленные в стены лампы. На  третьем  этаже  коридор
был таким же, как внизу. Мой провожатый вел меня к выходившим на  площадку
дверям и, отворив их, вошел первым.

Маленькая комната оказалась забитой физическими приборами и книгами, на
стенах висели географические карты каких-то, как мне  казалось,  пустынных
районов, на полу стояли  различной  величины  глобусы.  Мебель  составляли
огромный письменный стол, несколько кресел и стоящие вдоль  стен  столы  с
какими-то очень  сложными  аппаратами,  начиненными  огромным  количеством
катодных ламп.

Сколько я успел заметить, когда по приглашению хозяина сел  и  взглянул
на него, он был крайне сосредоточен и серьезен.

"--* Господин  Макмур,  убедительно  прошу  постараться  понять  меня  как
следует  и,  насколько  это  возможно,  поверить  в  то,  что   я   скажу.
Впоследствии я постараюсь  развеять  ваши  сомнения  с  помощью  наглядных
доказательств, --- он проделал широкий жест и  спросил,  поднимая  со  стола
какую-то газету: --- Вы не припомните, какое  явление  наблюдалось  на  небе
нашего полушария три месяца назад?

Я напряг память.

"--* Сдается мне, появилась какая-то крупная комета или метеорит, точно не
помню, --- сказал я.  ---  В  то  время  нас  занимала  капитуляция  Германии.
Астрономия и метеорология были задвинуты в угол.

"--*  Именно  так  оно  и  было.  ---  Казалось,  мой   собеседник   ответом
удовлетворен. ---  Вам  следует  знать,  что  я  по  профессии  физик.  Даже
астрофизик, --- после недолгого молчания добавил он, как  бы  раздумывая.  ---
Упомянутый вами метеорит упал на границе Северной и Южной  Дакоты,  вызвав
пожар, уничтоживший леса на площади три с лишним тысячи гектаров. Я  в  то
время как раз находился поблизости и решил с коллегами из  обсерватории  в
Маунт-Уилсон обследовать место падения метеорита.  Это  было  нечто  вроде
большого рва,  а  космическое  тело,  казалось,  не  очень-то  подчинялось
законам небесной механики: оно столкнулось с земной поверхностью под очень
острым углом, почти по касательной. Примерно  два  километра  оно  мчалось
сквозь лес,  местами  прорывая  борозду  глубиной  до  двенадцати  метров,
зажигая и валя воздушной  волной  деревья,  и  наконец  зарылось  в  холм,
вершину которого смело на  глубину  нескольких  десятков  метров.  Высокая
температура и горящий лес  затрудняли  доступ  к  тому  месту,  в  котором
находился загадочный метеорит. Самое странное, что вблизи мы не обнаружили
ни осколков  метеоритного  железа,  ни  вообще  чего-либо,  что  могло  бы
объяснить строение этого предмета. С помощью доставленных машин и  нанятых
рабочих  нам  удалось,  предварительно  охладив,  выкопать  это  тело.   О
сложностях, связанных с его  извлечением,  я  подробнее  расскажу  в  свое
время. Сейчас болид находится здесь, вы сможете его увидеть  даже  завтра.
Это, собственно говоря, не болид... --- Он замялся.

"--* Может быть, ракетный снаряд из Европы? --- спросил я. --- Немцы  пытались
их запускать, но, насколько мне известно, только в сторону Англии.

"--* Да, это реактивный снаряд, --- сказал Фрэйзер, ---  вы  очень  догадливы,
только он не из Европы.

"--* Из Японии?

"--* И не из Японии... --- Он указал на огромные карты полушарий, висящие на
стене.  Я  глянул  внимательней.   Какие-то   странные   обширные   желтые
поверхности, крутые, темные, как бы лесистые, массивы, белые шапки  снегов
на полюсах... Я вдруг увидел мелкую, узловатую сеть каналов...

"--* Марс, --- почти крикнул я.

"--* Да, это снаряд с Марса, --- медленно сказал Фрэйзер  и  положил  передо
мной предмет, который очень осторожно вынул из шкафа. ---  А  это  ---  первая
весть с другой планеты.

На красной доске стола лежал отсвечивающий голубым валик  из  какого-то
металла. Я взял его в руку --- рука повисла.

"--* Свинец? --- спросил я.

Фрэйзер улыбнулся.

"--* Нет, не свинец. Это очень редкий на Земле металл, палладий.

Я принялся медленно отвинчивать крышку  ---  матово  блеснула  резьба.  Я
заглянул  внутрь  ---  это  был  пустотелый  цилиндр,  заполненный  каким-то
порошком.

"--* И что же это такое?

Фрэйзер высыпал порошок на кусочек белой  бумаги,  положил  бумажку  на
стеклянную  пластину,  подвешенную  на  двух  штативах,  и  поднес   снизу
металлический цилиндрик. Провел им в одну, другую сторону. Мне кажется,  я
вскрикнул. На бумаге частички порошка наподобие железных опилок  сложились
в рисунок: треугольник с построенными по его сторонам квадратами.  Теорема
Пифагора. Внизу виднелись три маленьких значка, немного напоминающих ноты.
Фрэйзер старательно всыпал порошок в цилиндр, закрыл его и спрятал в шкаф.
Затем  взглянул  на  меня,  словно  хотел  проверить,  какое   впечатление
произвела на меня эта странная демонстрация, и продолжал:

"--* Господин Макмур, снаряд принес с другой планеты не только вести, но и
нечто осязаемое.

"--* Люди с Марса?

"--* Если б люди... В снаряде находился очень сложный механизм. Как бы вам
сказать? Для этого вообще нет слов... Что-то вроде  механического  робота.
Вы  его  увидите.  Мы  считали,  что  это  некое  подобие   робота-пилота,
управлявшего ракетой... Пойдемте, вы должны увидеть это своими глазами.  Я
сам всякий раз, когда гляжу на него, начинаю сомневаться,  в  своем  ли  я
уме.

Мы вышли в коридор. В голове гудело. Помню, мы вошли  в  кабину  лифта,
шахта которого располагалась в  середине  блока,  оплетенного  лестницами.
Кабина дрогнула, и пол провалился под нами. Опускались мы  недолго.  Внизу
был такой же коридор --- длинный, только более  темный,  потому  что  каждая
вторая лампа на стене не горела.

Скрипнули засовы. Мощные, на манер металлического шлюза, двери медленно
раздвинулись. Я вошел.

В воздухе чувствовался тяжелый неприятный запах.  Я  услышал  ритмичное
слабое постукивание --- так работает насос --- и как бы почмокивание  масла  в
вентилях. Загорелся свет. У комнаты были стальные стены и низкий  потолок.
В центре располагались два мощных деревянных столба, а между ними, как  бы
на козлах, покоилась какая-то бесформенная махина, поблескивающая черным и
голубым. Она походила на гигантскую сахарную голову, снабженную свисающими
до пола спиральными металлическими змеями. Основание щетинилось винтами  и
скобами.

В  разных  местах  виднелись  более  светлые  перегородки,  как  бы  из
стекловидной массы, а на самой макушке конуса располагалось  что-то  вроде
металлической шляпки или очень большой гайки.

"--* Это и есть "человек с Марса", --- очень тихо сказал  Фрэйзер.  Творение
лежало неподвижно, только изнутри исходило ритмичное тикание.

"--* А... ОН... ОНО... живое?

"--* Мы еще не знаем, как оно действует, --- сказал Фрэйзер. --- Видите, ---  он
подошел и медленно повернул шляпку сначала в одну, потом в другую сторону,
"--* здесь камера. Только, ради Бога, не прикасайтесь,  ---  испуганно  добавил
он, когда я наклонился слишком низко.

Я увидел небольшую, не больше апельсина, металлическую грушу, из одного
полюса которой торчало множество проволочек.

"--* Вот здесь оконце...

Действительно,  на  противоположной  стороне  этой   стальной   ---   или
палладиевой?  ---  груши  было  оконце,  заполненное  прозрачной  массой.  Я
заглянул туда. Различил очень слабую, медленную, но ритмичную пульсацию. В
моменты усиления становились  видимы  ленточки  светящегося  желатина  или
рыбьей слизи. В минуты затемнения появлялись единичные  бледно  светящиеся
точки, которые при разгорании сливались в единую светлую вспышку.

"--* Что это? --- невольно перешел я на шепот.

"--* Он, похоже, еще не пришел в себя, а может, в нем  что-то  повредилось
при посадке, --- сказал Фрэйзер, возвращая шляпку на свое место.

Он быстро вывел меня в  коридор,  повернул  штурвал,  толстые  стальные
плиты дверей замкнулись. Он оглянулся как бы с облегчением --- куда  девался
уравновешенный мужчина из верхнего зала? --- и сказал:

"--* То, что вы видели, собственно, единственное живое в нем.

"--* В ком?

"--* Ну, в этом госте-марсианине. Нечто вроде  плазмы,  мы  еще  не  знаем
толком, что именно.

Он пошел быстрее. Я глядел на него сбоку, пока он не поднял головы.

"--* Я понимаю, что вы думаете, но если б вы видели, что он может сделать,
как это видел я, то не знаю,  вошли  бы  вы  добровольно  еще  раз  в  эту
комнату.

И подтолкнул меня в кабину.

Кабина тихо заурчала и рванулась вверх. В голове  у  меня  зашумело,  я
почувствовал легкое головокружение и схватился за ручку  двери.  Мы  резко
остановились. Фрэйзер долго не спускал глаз с моего лица, словно проверяя,
какое впечатление произвела на меня необычная демонстрация.  Потом  открыл
двери и вышел первым.

Мы снова были на втором этаже. Направляясь в противоположную библиотеке
сторону, дошли до излома коридора. Здесь стены  неожиданно  оборвались,  с
правой стороны возникли высокие стеклянные плиты, заглубленные в  бетонные
ровики, ограждающие часть пространства, похожего на обсерваторию.  Фрэйзер
потянул меня дальше к маленьким белым дверям и постучал.

Изнутри долетел тихий, хрипловатый голос:

"--* Войдите!

Мы вошли в маленькую  комнатку,  настолько  заваленную  и  замусоренную
бумагами, какими-то фотографиями, эскизами, лежавшими на  огромном  столе,
подоконниках, стульях и шкафах, что казалось, нет места ни для кого, кроме
небольшого человечка,  который  в  ответ  на  наше  приветствие  приподнял
голову. Это был интересный тип --- старичок с румяной физиономией,  покрытой
серебристой щетиной. Карамелька в сахарной пудре. На этом лице, ежеминутно
изменяющем выражение, блестели огромные, оправленные в золото очки,  а  за
ними --- глаза, черные, пронзительные, вовсе не веселые,  контрастирующие  с
добродушной внешностью.

"--* Хи-хи, так это вы? Что, попали в наши силки, да? --- спросил  старичок,
сдвигая очки на лоб. --- Думаю, из вас получится человек.  ---  Он  критически
осмотрел мою одежду, которая, кроме следов недавней  драки  в  библиотеке,
носила явные признаки потрепанности. --- У нас  вы  не  пропадете.  Да,  это
серьезное дело. Ах, пожалуйста, садитесь.

Мы  присели.  Пришлось  снять  со  стульев  какие-то  диаграммы,   кипы
исписанных листов и таблиц. Профессор говорил не переставая.

"--* Итак... Господин Фрэйзер показал вам нашего, кх, кх, хм,  хм,  нашего
гостя?

Я кивнул.

"--* Невероятно, а? Ага, знаю, знаю... Что это я хотел сказать? Ах да, вас
наверняка удивляет здешняя таинственность и  здешние  стены,  а  замки-то,
замки, словно у шайки гангстеров... --- Он засмеялся, поднял  очки,  которые
сползли ему на нос, и продолжал совсем другим  тоном,  ровно  и  спокойно,
подчеркивая слова поднятым пальцем: --- Дело  выглядит  так:  этот  гость  с
Марса  может  принести  человечеству  огромную  пользу,  но   еще   больше
несчастий. Потому-то собралось несколько человек, они дали нужные  деньги,
средства и знания с тем, чтобы ознакомиться с сущностью пришельца, гонца с
другой планеты, найти с ним общий язык, выяснить, знает ли  он  о  нас,  и
если  да,  то  много  ли,  в  чем  его  техническое  или  интеллектуальное
превосходство --- все это для того, чтобы использовать его  на  благо  людей
либо, при необходимости, уничтожить. ---  Последние  слова  он  произнес  не
поднимая голоса, спокойно, и именно это особенно усилило впечатление.

"--* Конечно, мы вынуждены опасаться любопытных, в первую очередь  прессы,
нашей изумительной прессы, --- добавил он, хитро подмигнув. Он уже опять был
добродушным дядюшкой. --- Вы меня хорошо поняли?

"--* Понял. А теперь позвольте спросить, могу ли я и в какой степени  быть
всем вам нужен? У меня нет никаких специальных знаний. Я мог бы дать слово
и  уйти.  Признаюсь,  то,  что  вы  сказали,  невероятно   интригующе,   и
возможность описать случившееся хотя бы после того, как уже отпадет  нужда
блюсти тайну, соблазняла бы меня  невероятно,  но  я  не  думаю,  что  мне
следует оставаться у вас только потому, что я случайно попал сюда и должен
разделить судьбу марсианского пришельца.

Маленькие светлые точечки плясали в очках профессора.

"--* Что до вашего ухода отсюда --- не знаю, удастся ли это осуществить... ---
Он несколько раз взмахнул рукой, как бы поглаживая что-то, и сказал: ---  Не
чувствуйте себя обиженным... Я не сомневаюсь в вашей искренности и  слове,
слове шотландца, --- улыбнулся он, --- но, хм, вы ведь сами знаете, что  такое
репортерская струнка... Впрочем, вы нам пригодитесь. Думаю, и  мы  вам  не
меньше. Мы сейчас ожидаем одного...  ---  он  замялся,  ---  одного  инженера,
который должен прибыть из Орегона и привезти  от  наших  друзей  некоторые
детали. Знаете, у нас коллектив исключительно крупных специалистов, но нам
недостает простого человека, с  обычным  здравым  рассудком,  ---  он  опять
лукаво подмигнул мне, --- а такой рассудок --- прекрасная штука  и  может  нам
очень даже пригодиться... Вы что-нибудь слышали о конструкции ареантропа?

"--* По правде говоря, я еще не  успел  этого  переварить.  Впрочем,  я  и
видел-то его всего несколько минут.

"--* Знаю, знаю. Сидеть там и без того не очень полезно,  ---  тихо  заметил
профессор, не глядя на  меня.  ---  Нам  еще  не  ясно,  каким  образом  это
воздействует на наш организм. Думается мне, это  разновидность  излучения,
некоторые тела вблизи аппарата светятся. Кроме того, во  время  извлечения
его из снаряда...

Я внимательно смотрел на профессора.  Он  как  бы  немного  съежился  и
вздрогнул.

"--* Впрочем, на сегодня довольно. Вы еще услышите обо всем. ---  Он  поднял
голову и резко бросил: --- Знайте  же,  наша  игра  очень  опасна,  у  этого
аппарата, или зверя, или же у зверя, заключенного в аппарате --- мы пока что
не знаем, --- довольно удивительные свойства, и от него можно  ожидать  чего
угодно.

"--* Почему бы не попытаться разобрать его на части? --- вырвалось у меня.

Мужчины поморщились.

"--* К сожалению, такие попытки были. Нас ведь  вначале  было  шестеро,  а
теперь вот осталось только пять. Все далеко не так просто.

"--* Теперь вы уже знаете почти столько же,  сколько  мы,  ---  тихо  сказал
Фрэйзер. --- Согласны ли вы на наши  условия,  то  есть  ---  полная  свобода,
равноправное участие в нашей работе и честное слово,  что  не  попытаетесь
бежать?

"--* Что значит --- бежать? --- сказал я. --- Я не смогу отсюда выходить?

Мужчины улыбнулись.

"--* Конечно, нет, --- сказал Фрэйзер. --- Надеюсь, вы не думаете...

"--* В таком случае, я согласен. Но никакого слова не дам, --- сказал  я.  ---
Слово, дорогие мои, возможно, вы этого не  понимаете,  было  бы  для  меня
непреодолимым препятствием. Другое дело --- ваши стены. Я могу остаться,  но
только на тех же правах и условиях, которые действуют между вами.

Я встал.

Профессор улыбнулся, вынул из кармана пузатые золотые часы.

"--* Три минуты второго. Думаю, сегодня мы пережили уже достаточно.  Желаю
спокойной ночи.

И он снова погрузился в свои бумаги. Он уже не видел нас,  не  замечал,
выписывая длинные колонки цифр.

Фрэйзер взял меня за руку --- мы вышли в коридор.

Свет ламп как будто немного ослаб.  Я  почувствовал  холод  в  груди  и
чудовищную усталость.

\bigskip{}



\bigskip{}

2

\bigskip{}


Разбудил  меня  яркий  солнечный  свет.   Я   удивленно   потянулся   ---
почувствовал мягкость постели, --- подскочил на кровати и осмотрелся.

Большую светлую комнату заливало солнце, и первой моей мыслью было, что
я видел какой-то странный, дурной сон, но уже в  следующее  мгновение  мой
взгляд упал на дверь без ручки, и я вспомнил все. Я быстро встал,  подошел
к окну и выглянул. Подо мной  раскинул  свои  воды  большой  темный  пруд,
берега которого тонули в утреннем тумане. Я  смотрел  на  гладкое,  слегка
морщинящееся черно-золотое  зеркало  с  высоты  по  меньшей  мере  четырех
этажей. Осмотрелся. Моя одежда исчезла, на стуле лежал темно-серый  костюм
в шотландскую клетку. Я невольно  улыбнулся  ---  заботливые  мне  достались
хозяева. Неожиданно я заметил небольшую, покрытую росписью дверцу в  стене
комнаты. Я открыл ее --- засветилась белизной  кафеля  и  никелем  небольшая
элегантная ванная.

В следующий момент я уже стоял под шумящим горячим душем и  наслаждался
пеной дорогого ароматного  мыла,  без  которого  мне  довелось  так  долго
обходиться. Я уже кончал одеваться, когда  в  дверь  тихо  постучали  и  в
комнату вошел Фрэйзер.

"--* Ого! Ранняя пташка, это хорошо.

Он выглядел отдохнувшим, улыбался и, казалось, был полностью уверен  во
мне. Взял меня под локоть и потянул за собой.

"--* Прошу к завтраку. --- Потом  пояснил:  ---  Мы  всегда  едим  вместе.  Вы
услышите много интересного. Приехал инженер Линдсей из Орегона.

Мы спустились на один этаж. Зал, в который я вошел, можно было  увидеть
в любом старом английском  замке.  Огромный  камин,  длинный  узкий  стол,
окруженный высокими креслами с резными спинками красного дерева, серебро и
фарфор, гербы на стенах --- воистину люди, к которым я попал, умели устроить
себе жизнь даже в самых удивительных условиях.

За столом уже сидели знакомые мужчины и один  новый  ---  широкоплечий  и
коренастый, с крепкой костлявой  физиономией,  загоревший  до  бронзы.  Он
назвался инженером Линдсеем. Когда я занял место, вошел уже  знакомый  мне
помощник шофера и начал разливать чай и кофе. Я глянул  на  него  сбоку  ---
интересно, как он чувствует себя после нашей вчерашней стычки.

Походило на то, что чувствует он себя прекрасно,  только  кадык  сильно
припух, да и взгляд, которым  он  меня  одарил,  не  показался  мне  особо
дружелюбным. Впрочем, я не мог уделить ему внимания,  так  как  за  столом
продолжился разговор, прерванный моим появлением.

Профессор,  сидевший  на  конце  стола  и  макавший  кусочки  хлеба   в
наклоненную чашечку кофе, обратился ко мне. При этом  очки  шевелились  на
его коротковатом носу.

"--* Господин  Макмур,  обычно  за  столом  мы  обсуждаем  случившееся  за
предыдущий день. Так вот, вчера мы ожидали  прибытия  господина  инженера,
который привез необходимые для дальнейших экспериментов материалы, то есть
свинцово-асбестовые  костюмы.  Дело  в  том,  что  машина,   areanthropos,
излучает  некоего  рода  энергию,  пагубно  влияющую  на  наши  ткани.  Из
подвергнутых облучению в течение двух часов морских свинок  не  выжила  ни
одна. Вам  следует  знать,  что  это  воздействие,  как  мы  предполагаем,
ослаблено,  поскольку  состояние  аппарата,  вероятнее  всего,  далеко  от
нормального.

"--* Собственно, это лишь наши предположения, --- проговорил Фрэйзер. --- Дело
в том, что остатки атмосферы, сохранившиеся в  снаряде  и,  скорее  всего,
аналогичные  по  составу  атмосфере  Марса,  были   исключительно   богаты
двуокисью углерода и  другими  газами,  чуждыми  нашему  земному  воздуху.
Поэтому  мы  думаем,  что  организм,  вернее  ---   органическое   вещество,
управляющее деятельностью механизма,  было  отравлено  несвойственным  ему
составом нашей атмосферы.

"--* А может быть, состояние, в котором сейчас пребывает  машина,  это  ее
нормальное состояние? --- спросил я. --- Ведь  неизвестно,  как  должно  вести
себя такое создание... Мне кажется, нам не следует заниматься сравнениями,
то есть стараться очеловечивать... объект.

Все внимательно посмотрели на меня.

"--* Простите, возможно, я ляпнул глупость. Это были речи дилетанта.

"--* Мы ничуть не меньшие дилетанты, ---  возразил  профессор,  который  уже
разделался со второй чашечкой кофе и теперь крутил  хлебные  шарики,  ---  а
ваше мнение вполне справедливо. Увы, реакция машины в тот момент, когда мы
открыли снаряд, была аналогичной...

"--* Можно ли наконец узнать, что, собственно, происходит? --- спросил я.  ---
Мне уже пришлось услышать столько недомолвок, что я прямо-таки  сгораю  от
любопытства.

"--* Вы правы, --- сказал  седоватый  стройный  мужчина,  которого  называли
доктором.  ---  В  тот  момент,  когда  мы,  воспользовавшись  ацетиленовыми
резаками, отсекли макушку раскаленной стальной сигары, которую представлял
собой снаряд с Марса, в отверстии показался этакий металлический  змеевик.
Вы его, вероятно, заметили, если смотрели внимательно...

Я кивнул.

"--* Змеевик, возможно, коснулся одного из наших рабочих, точно установить
не удалось, при этом он проделывал резкие, как бы спазматические движения.
Потом появился корпус, который вывалился  на  землю  с  высоты  нескольких
метров и замер. Неподвижность он сохраняет до сих пор, то есть уже  больше
недели.

"--* И что же тут странного? --- сказал я.

"--* А то, что рабочий, у которого в руках был резак, умер в тот же  день.
С признаками апоплексического удара. А вскрытие не показало никаких других
изменений, кроме легкого перенасыщения мозга кровью...

"--* И вы предполагаете...

"--* Мы, молодой человек, ничего  не  предполагаем.  Помните,  что  сказал
старик Ньютон? Hypotheses non fingo --- гипотез  не  измышляю.  Да,  да,  мы
только исследуем, но никаких  гипотез  не  придумываем.  Установлено,  что
близость машины может привести к некоторым печальным  последствиям  вплоть
до потери жизни, и об этом надлежит помнить.

Потом он обратился деловым тоном к Фрэйзеру:

"--* Коллега, вы подготовили на сегодня все?

"--* Да. В девять ареантропа переведут с помощью доставленных  подъемников
в малый монтажный зал, там мы поместим его в контейнер, заполненный смесью
газов,  рекомендованных  доктором,  и  постараемся,  снизив  давление   до
марсианского, восстановить его жизненные  функции.  Я  думаю,  это  должно
получиться, если, конечно, нет никаких повреждений в его механизме.

"--* А как  чувствуют  себя  свинки,  помещенные  в  камеру  со  свинцовым
экраном?

"--* Я еще не видел... --- смутился Фрэйзер. --- Не знаю, ведь мы поместили их
туда лишь в пять утра.

Розовое личико профессора налилось кровью.

"--* Если все мы станем работать как вы, господин  Фрэйзер,  то  марсианин
сбежит от нас через окно и его не поймаешь. Ничего себе!  Не  видел...  не
знаю... --- бурчал старый  холерик,  разбрасывая  хлебные  шарики  по  всему
столу. Фрэйзер быстро встал и подошел к  нише  в  стене.  Я  услышал,  как
звякнула трубка интеркома.

Минуты через полторы Фрэйзер вернулся на свое место, медленно опустился
в кресло и посмотрел в глаза профессора. Тот поерзал на стуле, открыл  рот
и ждал.

"--* Ну?!

"--* Все свинки подохли,  ---  глухо  сказал  Фрэйзер.  ---  Теперь  есть  две
возможности:  либо  доза  безвредна  для  человека,  но  убийственна   для
защищенных свинцовым экраном свинок, либо...

"--* Либо... Короче, ничего толком не известно, --- сказал профессор. --- Надо
подождать до вечера, если мы хотим встретиться за ужином в полном составе.
Усильте экран до максимума. Сколько у нас свинцовых пластин?

"--* Тридцать шесть  по  восемь  сантиметров  толщиной  каждая,  ---  сказал
широкоплечий инженер.

"--* Значит, надо дал" пятьдесят шесть сантиметров свинца...

"--* А  если  это  не  обычное  линейное  излучение  и  свинок  необходимо
экранировать со всех сторон? --- спросил доктор.

"--* Вы считаете, что на  Марсе  действуют  другие  физические  законы?  ---
насмешливо бросил Фрэйзер.

"--* А вы уверены, что постигли уже все без остатка? ---  поддержал  доктора
профессор. --- Когда я был в ваших летах, мне  тоже  казалось,  что  я  знаю
все... Я думаю, доктор прав. Пожалуйста, сделайте экран в форме цилиндра и
дайте фильтры для дыхания. Или нет, лучше  закрыть  герметично,  а  внутрь
поместить баллон с  кислородом.  Пожалуйста,  сделайте  это  сейчас  же  и
поместите в камеру с марсианином.

Все медленно вставали из-за стола. Профессор ухватил Фрэйзера за  руку,
подвел к окну и принялся что-то втолковывать, водя пальцем по стеклу.

Доктор подошел ко мне.

"--* Как вам нравится наш профессор? --- спросил он, потирая тонкий  длинный
нос. --- Ворчун, а? Но скажу вам: голова! --- и он постучал  себя  пальцем  по
лбу. --- Знаете, я уговорю Финка показать вам все, что мы вынули из снаряда.
Любопытные вещицы. Правда, я уже один раз видел, но, понимаете,  профессор
все держит под замком.

Доктор кивнул инженеру, седому брюнету  со  светло-голубыми  глазами  и
смуглым лицом, и мы вышли в коридор.

"--* Простите, господа, но, если я  верно  понимаю,  в  вашей  работе  нет
ничего противозаконного,  так  почему  такая  таинственность?  И  странные
пароли, способы переговариваться... Почему инженер Линдсей не  мог  просто
приехать сюда?  Я  бы,  возможно,  не  стал  участником  столь  невероятно
любопытных исследований, если б...

"--* Потому, видите ли, что ваши коллеги по  перу  жить  нам  не  дают,  ---
прервал импульсивный доктор. --- Потому, понимаете ли, что парк пришлось  бы
обнести колючей проволокой и пустить собак. Потому, наконец, что  они  уже
унюхали, не знаю только  как,  что  профессор  Уиддлтон  как-то  связан  с
упавшим метеоритом... К счастью, головы у ваших соперников забиты Японией.
И все же узнай они, что здесь  находится  светоч  современной  астрофизики
вместе с крупными специалистами  по  атомной  физике,  что  с  нами  такие
инженеры-конструкторы, как господин  Финк,  и  такие  электротехники,  как
Линдсей, которого мы, к слову, привлекли всего лишь  три  дня  назад,  то,
уверяю вас, никакие стены, заборы и рвы нам бы не помогли.

Беседуя,  мы  подошли  к  лаборатории  ---  большому  залу   с   частично
застекленным   потолком,   оборудованному   новейшим    образом.    Воздух
перемешивали лопасти больших вентиляторов, всюду блестели  стекла  стоящей
рядами аппаратуры, по стенам выстроились сосуды,  заполненные  химическими
реактивами всех цветов. Там и сям шипели газовые горелки. В  другой  части
зала столы были  забиты  оптическими  приборами  и  сложными  механизмами,
напоминающими часовые, назначение которых было мне не известно. Мы  прошли
через этот зал, и в следующем помещении я  увидел  развешенные  по  стенам
цветные снимки, изображающие местность, в которой упал метеорит.

Признаюсь, рассматривал я их весьма бегло. Да и снимки самого  снаряда,
сигары с тупым концом, не вызвали у меня особого  интереса.  Мои  спутники
это заметили.

"--* Вижу, вам не терпится увидеть самое важное, --- заметил доктор. --- Пошли
вниз.

Мы спустились лифтом на первый этаж, потом по  лестницам  до  знакомого
мне подземного коридора. Каково же  было  мое  удивление,  когда  в  конце
короткого пути я оказался в хорошо  знакомой  мне  стальной  комнате,  где
ужинал в обществе молчуна-шофера.

"--* Вам, кажется, известна  эта  комната?  ---  шутливо  бросил  доктор.  Я
улыбнулся в ответ, инженер же тем временем  устанавливал  цифровые  валики
нескольких сейфов, и спустя минуту щелчок известил, что дверцы открыты.

Инженер вынул из  темного  чрева  несколько  предметов  и  разложил  на
столике.

"--* Что это? --- спросил я, указывая на  палладиевый  цилиндр,  на  боковой
стенке которого было что-то вроде кнопки или клавиши.

"--* Кажется, прибор для письма  или  увековечивания  мыслей.  Там  внутри
порошок, что-то вроде крошек какого-то органического  вещества.  Цилиндрик
снабжен хитроумным  приспособлением,  создающим  переменное  электрическое
поле, воздействующее на порошок... Изменения воздействуют на порошок  так,
что если его насыпать на бумагу...

"--* А, я уже знаю, --- прервал я, --- видел вчера, но как такое возможно?

"--* Этого мы еще не знаем. Такие ответы вам придется слышать часто.  Пока
что, --- заметил доктор.

На  втором  предмете  в  виде  треугольника  из  серебристого   металла
располагались три  пересекающиеся  выпуклости,  как  бы  изготовленные  из
толстой проволоки.

"--* Что это?

"--* Возьмите в руку...

Я хотел было так и сделать, но треугольник тут же отодвинулся. Тогда  я
резким движением схватил его. Это было что-то твердое и холодное, которое,
однако, тут же вздрогнуло, начало извиваться у меня в руке и теплеть,  так
что я невольно разжал пальцы. Треугольник упал на стол и замер.

"--* На первый взгляд металлическое вещество, наделенное возбудимостью,  ---
выговаривал доктор, полуприкрыв глаза. --- Нарушает все наши понятия о живой
материи и различии между живым и мертвым...

Третьим предметом была маленькая черная кассетка.

Доктор поднял ее и подержал между моими глазами и светом.

Когда глаза освоились, я увидел, что стенка кассетки время  от  времени
затемняется,   образуя   что-то   вроде    зеленоватой    фосфоресцирующей
поверхности, по которой медленно пробегают более светлые линии. Они  то  и
дело на несколько секунд останавливались, а потом убыстряли движение.

"--* Относительно этого, --- проговорил инженер,  ---  мнения  разделились.  Я
считаю, что кассетка --- нечто  вроде  телевизора,  возможно,  для  связи  с
Марсом. Линдсей полагает, что это  приемник  какого-то  неизвестного  рода
энергии.  Причем  не  исключено,  ---  добавил  он,  улыбнувшись,  ---  что  в
действительности  это  нечто  третье,   совершенно   отличное   от   наших
предположений.

Последним предметом оказалась сигара, изготовленная как  бы  из  резины
или не  очень  твердой  пластмассы,  которая  напомнила  мне  знакомый  по
фотографиям метеорит.

"--* Модель снаряда? --- сказал  я,  почувствовав  некоторое  облегчение  от
того, что у гостей с Марса есть хоть что-то общее с  людьми:  они  обожают
уменьшенные копии своих технических устройств.

"--* Увы, нет, --- сказал доктор,  иронически  усмехнувшись.  ---  Перед  вами
всего лишь внешняя оболочка. Инженер, может быть,  вы  покажете  господину
Макмуру, что находится внутри сигарки. Мой палец, --- он заботливо  осмотрел
кончик большого пальца, --- еще не зажил.

Инженер криво  усмехнулся,  но  взял  сигару  и  положил  в  стеклянный
аквариум, стоявший на столике. Потом вынул из  стенного  шкафчика  длинные
щипцы, прихватил сигару и медленно нажал на ее кончик.

Кончик тут же раскрылся на манер рыбьей пасти, и оттуда выскочило нечто
красное, не очень большое, но очень подвижное. Это было создание  размером
с грецкий орех, непрестанно подпрыгивающее, как мячик. Над ним,  казалось,
висели два темно-голубых шипа.

"--* Образчик марсианской фауны, может, флоры, --- заметил доктор Кеннеди. ---
Кусается солидно, но вытерпеть можно. ---  Он  сунул  мне  под  нос  большой
палец, украшенный малоприятной, вроде бы выжженной, ранкой. --- Эти зверьки,
похоже, близкие родственники наших паукообразных или членистоногих, хоть у
них и нет ног. Однако, как мне представляется, они толковее всех насекомых
земного шара, вместе взятых.

"--* Из чего это видно? И почему вы так дружески настроены? ---  спросил  я.
Невероятная странность, превосходство и в то же время чуждость  всех  этих
предметов, которые отказывался принимать  мой  обычно  столь  ясный  мозг,
начали меня раздражать.

"--* Очень просто. После двухдневных опытов они проявили  память,  которой
не постыдился бы и пес из лаборатории условных рефлексов, к тому же память
гораздо более совершенную. Обратите внимание на их  механизм  движения:  у
них нет конечностей, но они передвигаются скачками  благодаря  сокращениям
нижней части живота...

Я смотрел  на  кирпично-красное  существо,  прыгающее  по  дну  пустого
аквариума, и заметил, что оно украшено  светлыми  и  темными  полосками  и
снабжено чем-то вроде тонких волосков, на концах которых размещены шипы.

"--* Но это вовсе не его конечности! --- изумленно воскликнул я.

Действительно,  "шипы"  представляли  собою  две   челюсти,   тщательно
изготовленные  из  серебристо-голубого  металла   и   отличающиеся   почти
математически строго вычисленной  системой  зацепления.  Удерживающие  эти
челюсти выступы закручивались в спиральки.

"--* К сожалению,  вы  правы.  Оказывается,  насекомые  на  Марсе  владеют
инструментами, --- сказал доктор. Он схватил животное щипцами,  что  удалось
сделать не сразу. Инженер  принялся  недоверчиво  рассматривать  существо,
советуя соблюдать осторожность, а потом засунул его  в  футлярчик-сигарку.
Пока инженер тщательно прятал все предметы в сейф, доктор заметил:  ---  Мне
становится  дурно,  когда  я  вижу  все  это.  Не  будь  я  таким   старым
сумасшедшим, нога б моя никогда сюда не ступила. Я не говорю об опасности,
"--* он немного выпрямился, --- у меня военные награды  за  храбрость,  но  эти
свихнувшиеся предметы, да и наш механический покойничек  выводят  меня  из
равновесия. И зачем им врач? --- удивлялся  он,  провожая  меня  в  коридор,
освещенный, как всегда, несмотря на дневное время, искусственным светом. ---
Им нужен часовщик для этого марсианина, ха, ха, ха!

Инженер задвинул большой засов и повернулся к нам.

"--* Не обращайте внимания на слова доктора, ---  сказал  он.  ---  И  на  его
пессимизм: он сам не принимает этого всерьез.

Мы поднялись на площадку первого этажа, инженер нажал кнопку,  раздался
тихий свист, двери  распахнулись,  и  небольшая,  освещенная  приглушенным
светом кабина лифта раскрылась перед нами. Я глянул на своих спутников.

"--*  Поедем  в  библиотеку,  ---  сказал  доктор,  посмотрев  на  часы.   ---
Одиннадцать. Сейчас начнется военный совет.

В библиотеке было пусто, так  мне  по  крайней  мере  в  первый  момент
показалось. Только у бокового окна стоял мужчина, с которым до того  я  не
разговаривал. Он был невысок,  коренаст,  свободная  одежда,  скроенная  с
большим припуском, свисала с его достаточно выпуклых форм, тщетно  пытаясь
замаскировать их округлость. Лицо у  него  было  почти  оливковое,  волосы
иссиня-черные, только глаза были зелеными,  словно  пересаженными  с  лица
норвежца, и светились, как льдинки на солнце.

"--* О, вы уже здесь, синьор Джедевани. --- Доктор казался довольным. --- Ну и
что новенького?

Синьор Джедевани испытывал явные затруднения с английским.

"--* Доктор, вас вызывать профессор. Уже трижды  звонить  наверх  и  вниз.
Идите же вконец на малую операционную.

Доктор замолк, движения его стали  целенаправленными,  быстрыми,  левой
рукой он схватился за внутренний карман. Проверил, на месте ли  шприцы,  и
выбежал из комнаты.

"--* Что случилось? --- спросил я.

"--* Не знаю. --- Синьор Джедевани стучал пальцами по столу. --- Кажется,  наш
друг Линдсей хотеть показать господину Фрэйзеру, что, хм... что машина, то
есть человек  из  Марса,  не  выделяет  сквозь  двери  от  камеры  никакое
излучение...

"--* И что с ним случилось?

"--* Бэрк нашел его без сознания в лифте. Наверно, убегать оттуда, я знаю?

В этот момент двери раскрылись, и словно маленький темный гном с  седой
шевелюрой в библиотеку ввалился профессор. Спустя секунду  следом  за  ним
вошли остальные мужчины, кроме доктора и Линдсея. Профессор бегал по ковру
вдоль библиотечных шкафов,  топал  ногами,  бурчал  и  метал  из-за  очков
гневные взгляды на  мужчин,  которые  медленно  рассаживались  в  креслах,
закуривали и, казалось, ждали, пока у старика кончится заряд злости.

"--* Говорил ведь тысячу раз: никаких опытов на свой страх и риск. Или это
кончится, или я завтра же утром уеду. Мы что --- дети? ---  взорвался  наконец
профессор. --- Дело слишком серьезно, а тут крупный ученый из-за амбиций или
ради состряпанной к данному случаю теорийки сует голову...  ---  Он  осекся,
вытащил из кармашка невероятно помятую сигару, отгрыз кончик,  прикурил  и
продолжал совершенно другим тоном: --- Кажется, наш друг Линдсей  выйдет  из
этой  истории  целым  и  невредимым...  А  теперь  я  хотел  бы  выслушать
результаты  работы  за  день  и  ваши  предложения.   Однако   убедительно
рекомендую  заботиться  не  столько  о  темпах  исследований,  сколько   о
безопасности. Прошу вас, Финк, изложите техническую сторону.

Инженер встал и принялся раскладывать на своем столике записки.

"--* Итак, все  выглядит  следующим  образом.  В  настоящее  время  машина
находится  в  бронированной  камере,  экранированной,  как  вам  известно,
свинцовыми пластинами толщиной шестьдесят сантиметров. Судя по результатам
воздействия излучений, полученным в наших  лабораториях,  полметра  свинца
должны были дать стопроцентную защиту от любой  волновой  энергии.  Однако
это оказалось ошибкой. Несчастный случай с коллегой Линдсеем показал,  что
излучения, создаваемые машиной, проникают сквозь наш экран как сквозь лист
картона. Таким образом, опыты с  морскими  свинками  оказались  совершенно
беспредметными. Наша цель --- подвергнуть  всю  машину  целиком  воздействию
некой  измененной  атмосферы,  поскольку  попытки   демонтажа   окончились
неудачей,  а  предложение  полного  отравления  живой  составляющей   было
отвергнуто.

Чтобы  осуществить  сказанное,  я  предлагаю,   во-первых,   изготовить
свинцовую трубу с толщиной стенок два метра. В эту трубу поместить  машину
и заглушить с обоих концов  плитами  соответствующей  толщины.  Во-вторых,
внутри  трубы  смонтировать  телевизионный  аппарат,  это   позволит   нам
наблюдать все, что там будет происходить. Я считаю, что предложенный  мною
метод обеспечит максимум безопасности.

"--* Вы закончили? --- спросил профессор.

"--* Да.

"--* В таком случае прошу пройти к Бэрку, он оденет вас в наш скафандр, вы
спуститесь  в  подземелье  и  попытаетесь  электрометром  определить  силу
излучения, проникающего из камеры  сквозь  закрытые  двери.  Только  после
этого  вы  сможете  рассчитать  необходимую  толщину  стенок  предлагаемой
свинцовой трубы, причем с достаточной степенью безопасности. И пожалуйста,
без фокусов --- в камеру без нужды не входить. Пусть Бэрк все время  ожидает
в коридоре, у лифта. Если хоть что-то случится, немедленно  сообщайте.  Вы
меня хорошо поняли? До сих пор  я  думал,  что  все  вы  хорошо  понимаете
по-английски, но после случая с Линдсеем я сильно в этом усомнился.

Инженер поклонился, сложил бумаги и вышел.

В дверях он столкнулся с доктором, который, войдя,  тут  же  подошел  к
профессору и сказал тому что-то так тихо, что я ничего не мог  расслышать.
Профессор вытаращил глаза, глянул на доктора, потом стукнул  себя  пальцем
по лбу  и  пожал  плечами.  Доктор,  казалось,  в  чем-то  убеждал  его  и
вырисовывал что-то пальцем на ладони.

"--* Он спятил! --- взорвался наконец профессор.

Доктор, похоже, пропустил его слова мимо ушей.

"--* Он такой же нормальный, как и я, --- сказал он. --- Другое дело, что  это
могла быть галлюцинация, --- и,  обращаясь  к  нам,  добавил:  ---  Понимаете,
Линдсей  пришел  в  себя  и  рассказал,  что,  намереваясь  измерить  силу
излучения у закрытых дверей  камеры,  пробыл  там  с  электрометром  около
получаса.  Сейчас  выяснилось,  что  электрометр  был  неисправен   и   не
зарегистрировал никакого  излучения,  которое  меж  тем  было  чрезвычайно
сильным. Доказательством чему тот  факт,  что  инженер  потерял  сознание.
Видимо, он упал и пополз к выходу в коридор. Бэрк нашел его у лифта.

Инженер утверждает, что  около  трех  минут  десятого,  то  есть  после
четверти часа  наблюдений  за  электрометром,  который,  кстати,  даже  не
дрогнул, он заметил, что участок стальных дверей, замыкающих камеру, начал
светиться, как бы раскалившись до вишневого цвета. Он поднес к нему  руку,
но температура оказалась нормальной. Спустя  несколько  секунд  эта  часть
двери исчезла совершенно и  в  образовавшемся  просвете  появилось  что-то
вроде черного блестящего свинцового кабеля. Он думал, что это конец одного
из змеевиков, которые, кажется, выполняют у марсианина  роль  конечностей.
Однако инженер находился в таком состояний (вероятно, под вредным влиянием
излучения), что совершенно не удивился увиденному, а сидел там  еще  минут
десять, пока не потерял сознания. Я предполагаю, что если это явление,  то
есть локальное исчезновение пятнавдатисантиметровой стальной  плиты,  даже
не  имело  места  и  представляет  собою  всего  лишь  плод   отравленного
воображения, то все равно  оно  в  какой-то  мере  указывает  на  механизм
действия лучей, генерируемых  ареантропом.  Эти  лучи  приводят  к  потере
способности критически оценивать ситуацию, медленно и незаметно  погружают
человека в бессознательное состояние. Причем уберечься от  этого,  видимо,
невозможно.

Когда доктор закончил, наступило долгое молчание.

"--* Собственно, это ничего не меняет,  ---  наконец  заметил  профессор.  ---
Синьор Джедевани, вы у нас, да и не только у  нас,  лучший  специалист  по
экспериментальной атомной физике. Ведь вы работали в Чикаго четыре года  с
циклотронами  Лоуренса,  хотелось  бы  знать  ваше  мнение:  возможна   ли
подобная,  хм,  диффузия  какого-либо  тела  сквозь  стальную  плиту   при
нормальных давлении и температуре? Если да, объясните это  по  возможности
исчерпывающе.

Маленький оливковокожий  Джедевани  встал,  немного  помолчал,  наконец
воскликнул:

"--* Господин профессор, господа! Если возможно такое, то возможно  все...
Теоретически --- конечно. Должен сказать, что теоретически, то есть с  точки
зрения теории вероятностей, брошенный  наверх  камень  может  не  упадать,
вода, поставленный на газ, не закипеться, все это лишь вероятно... как это
говориться? --- что это... произойти. Но в данный случай --- нет. Я думаю, что
если такой явлений возможен, то фундамент наших знаний  по  материя  и  ее
структура будет сотрясен.

"--* Быть может, это уже произошло и пришелец разрушил наши  представления
о мире, --- вполголоса заметил доктор.

"--* Если даже так, то взамен он даст нам нечто лучшее, более совершенное,
"--* сказал профессор.

"--* Относительно излучений, --- продолжал синьор Джедевани, --- так я  думаю,
что все очень просто. Если части машины выделяют их,  то  в  любой  случай
существует  такой  экран,  панцирь,  который  их  глушит,  даже  если   их
эмиссионное  напряжение  эквивалентно  напряжению  тела  под  давлением  в
шестнадцать триллионов атмосфер при температуре десять миллионов градусов,
то есть звезд, высылающих такое излучение.

"--* А какова же толщина такого панциря? --- спросил профессор.

"--* Около трехсот метров слоя свинца.

Изумление окружающих было неописуемым.

"--* Так мы здесь сидим на вулкане! --- воскликнул доктор. --- Может,  все  мы
давно спятили под влиянием этих лучей. Смешно!

"--* Я не говорю, что такое напряжение существует, это максимум.  Но  ведь
машина не может создавать такие давления или такие температуры...

"--* А вы-то откуда знаете?

"--* И вы еще спрашиваете, господин инженер? Только давление от  излучения
тела  с  такой  температурой  повалило  бы  все   в   радиусе   нескольких
километров... А у нас, хвала Богу, все стоит...  ---  Он  постучал  согнутым
пальцем по столу.

Профессор Уиддлтон не выдержал.

"--* Вы приехали сюда не для того, чтобы по дереву стучать, дорогой синьор
Джедевани, --- сладко проворковал он. --- Каковы ваши предложения?

"--* Я хочу знать, чего вы хотеть?

"--* Разве не ясно? Я хочу поставить эксперимент, предложенный доктором.

"--* И в чем же он состоит?

"--*  Вы  что,  памяти  лишились?  Мы  будем  воздействовать   на   машину
марсианской атмосферой...

"--* Зачем?

"--* Чтобы машина вернулась к норме.

"--* А откуда вы знать, как эта норма проявляется?

У меня  несколько  секунд  усиленно  пульсировала  кровь  в  висках.  Я
заметил, что у всех немного покраснели лица --- надулись жилы на лбу.

Неожиданно мы ощутили толчок, из окна на нижнем этаже вылетело  стекло.
Все умолкли.

Потом двери распахнулись, и какое-то существо, ошалевшее  от  ужаса,  с
пеной на губах влетело в библиотеку.

"--* В чем дело, Бэрк?! --- воскликнул профессор. --- Вы пьяны?

"--* Спасите, профессор! Он идет сюда! Идет сюда!  ---  заорал  шофер,  лицо
которого от страха превратилось в пепельно-серую маску.

"--* Кто? Вы рехнулись?

В этот момент в зал вбежал инженер.

По лицу у него текла кровь, весь он был покрыт пылью и штукатуркой.

"--* Господа, внимание, он вырвался! --- тяжело дыша, крикнул он.

"--* Что, где, как? --- повскакивали все с кресел.

Я успел заметить, как синьор Джедевани подбежал к окну, что-то крича, и
в этот момент прогремел стальной  голос,  моментально  обративший  хаос  в
тишину.

"--* Трусы!

Это крикнул профессор. Его тщедушная фигурка как  бы  выросла.  Сверкая
глазами, он стоял, склонившись над столом и крепко опираясь о него.

"--* Немедленно успокойтесь! Господин инженер, что случилось?

"--* Машина вышла из камеры --- не  знаю,  как  ей  удалось.  Я  был  внизу,
электрометр у дверей показал сильнейшее излучение,  которое  увеличивалось
скачками с двенадцатисекундными интервалами. Тогда я пошел на первый этаж,
чтобы проверить, регистрируется ли излучение  на  потолке  камеры,  и  тут
почувствовал толчок, который бросил меня на пол. Приборы разбились...

В этот момент со двора --- а может, со стороны  фасада  ---  донесся  дикий
крик, ударивший нам в уши так, что все вздрогнули и повернулись  к  окнам.
Крик   повторился,   раздался   сдавленный   хрип,   который   перешел   в
нечленораздельное бормотание --- и все стихло.

"--* Мы бессильны  чем-либо  помочь,  ---  медленно  и  тяжело  цедил  слова
профессор. --- Мы не знаем его возможностей, не знаем,  как  защищаться,  мы
можем только советовать.

"--* Но мы же тут сгинем! Нет, я уезжаю, --- прошипел Джедевани.

"--* Извольте. Двери открыты, я никого не  задерживаю,  мое  место  здесь,
даже если мне суждено погибнуть на половине фразы. --- Лицо профессора  было
словно высечено из камня.

Да! Это был могучий старик!

Джедевани повалился на стул.

"--* Как он вышел, куда? И что делает? --- спросил профессор.

"--* Не знаю, не видел. Может,  Бэрк  скажет?  ---  бормотал  инженер,  силы
которого были на исходе. Он тяжело опустился на стул  и  с  трудом  дышал,
вытирая платком со лба и волос кровь, смешанную с побелкой.

Бэрк, нашедший убежище между двумя библиотечными шкафами, был  извлечен
оттуда доктором, который  медленно  и  флегматично  достал  из  нагрудного
кармашка футляр, отыскал ампулу сердечного средства  и  сделал  огромному,
трясущемуся мужчине укол. Затем убрал шприц и обратился к профессору:

"--* Если так пойдет и дальше, то скоро  у  меня  кончатся  ампулы.  Такой
эпидемии я не предвидел.

Он старался выдерживать шутливый тон. Профессор пожал плечами ---  теперь
мне показалось, что его болтливость и рассеянность были  лишь  маской,  за
которой скрывался дьявольски твердый и крепкий стержень. Словно ядрышко  в
скорлупе ореха.

"--* Говорите, Бэрк, что вы видели?

"--* Господин профессор, а он сюда не придет? --- Шофер все еще дрожал.

"--* Нет! Да говорите же, чертов осел!

"--* Я стоял у лифта, вдруг вижу какую-то вспышку, словно небесный  огонь,
гляжу --- там,  где  были  двери,  только  пыль,  да  такая,  будто  потолок
обвалился, и из этой пыли выдвигается...

Он затрясся, лицо побледнело.

"--* Ну что? Что вы видели?

Все стояли вокруг, напряженные, бледные. Профессор наклонил  голову,  и
его черные глаза горели, казалось,  каким-то  бессмертным  огнем.  В  этот
момент я подумал, что даже марсианин не выдержал бы такого взгляда.

"--* Огромная черная сахарная голова медленно плыла, а перед ней и за  ней
извивались в  воздухе  странные  змеи,  словно  щупальца.  Она  двигалась,
раскачиваясь из стороны в сторону, и один раз даже ударилась о колонну.  Я
почувствовал толчок.

"--* Так вот что это было. Значит, дверь он не вырвал, --- шепнул мне Финк.

"--* Гляжу: кирпичи летят, ноги мои сами оторвались от пола, и я влетел  в
лифт,  слава  Богу,  он  еще  действовал,  ---  докончил  шофер,   испуганно
оглядываясь на дверь.

"--* Черт возьми, сахарная голова, от которой приходится бежать... --- начал
кто-то у меня за спиной.

В этот момент воздух в зале взорвал единый крик.

Немного потемнело, через окно я увидел плотные клубы пара.

"--* Пруд кипит, --- раздался чей-то  голос.  Действительно,  вода  в  пруду
кипела и  взвивалась  небольшими  смерчиками,  которые  тут  же  сливались
воедино. Так продолжалось минут пять и прекратилось так же быстро,  как  и
началось.

"--* Мне кажется, я понимаю: машина частично пришла в норму, но не совсем.
Говоря по-нашему,  она  как  бы  пьяна,  отсюда  удар  о  колонну,  отсюда
раскачивание, отсюда же,  похоже,  этот  спектакль,  ---  профессор  говорил
медленно, потирая рукой лоб. --- Хм, по-видимому, она приспособилась к нашей
атмосфере... Господа...

Все обернулись к нему.

"--* Господа, остается последнее  средство:  надо  воспользоваться  нашими
гранатометами. Несколько близко взорвавшихся гранат со смесью  углекислого
газа, хлора и ацетилена должны дать желаемый эффект. Если мы снова изменим
ему состав окружающей атмосферы, он опять впадет в  коматозное  состояние,
которое мы сумеем использовать получше.

Мысль показалась удачной.

"--* Надо спуститься в подвалы за  гранатами  и  гранатометами,  ---  бросил
кто-то.

"--* Кто это сказал? Ну...

Все переглянулись.

"--* Ну, в чем дело? Никто не хочет спускаться?

В ушах у меня все еще бился страшный крик, крик человека, который видит
приближающуюся смерть.

"--* Я пойду! --- сказал я.

"--* Благодарю, Макмур, вы еще не вполне освоились у нас. Пойду я  сам.  ---
Профессор отошел от стола.

Кучка мужчин ожила.

"--* Пойдем все!

Профессор взглянул на них, как бы извиняясь.

"--* Нет нужды, господа. Достаточно троих. Пусть пойдут доктор, инженер  и
Макмур.

Мы вышли из зала. В коридоре лежал тонкий слой осыпавшейся  штукатурки,
побелившей темную ковровую дорожку. А вообще  было  тихо  и  спокойно.  Мы
вошли в кабину лифта и спустились в подземелье. Там царил  хаос.  Один  из
поддерживающих  потолок  столбов   утончился   почти   наполовину.   Куски
железобетонных конструкций  с  разорванной  арматурой,  песок,  штукатурка
устилали пол.

"--* Какая дьявольская сила! А ведь он не выше полутора метров,  ---  сказал
инженер. --- Мы его, кстати, взвесили, в нем неполных четыреста килограммов.

Мы почти бегом добрались до конца коридора.

В  воздухе  стоял  неприятный,  какой-то  приторно-тошнотворный  запах,
который походил на тот,  что  я  почувствовал  во  время  первого  осмотра
чудовища. Правда, тогда он был послабее.

В уже знакомой мне стальной комнате инженер вынул  из  шкафов  длинные,
закругленные на концах газовые гранаты,  помеченные  цветными  кольцами  у
основания, и на каждого из нас погрузил по шесть штук, взяв себе  еще  три
гранаты и два гранатомета, укрепленных  на  небольших  легких  салазках  с
алюминиевыми полозьями.

Кроме того, каждый  повесил  через  плечо  по  нескольку  противогазных
масок, и мы двинулись в обратный путь.

В библиотеке напряжение немного спало, когда остававшиеся там  увидели,
что мы возвратились без помех.

"--*  Итак,  начинаем!  ---  сказал  профессор,  передавая   каждому   маску
противогаза. --- Маски не снимать, пока я не прикажу. В случае, если меня...
если я не смогу отдать приказ, командовать будет господин Фрэйзер, а после
него инженер Финк.

Мы разобрали гранаты и вышли из библиотеки.  Профессор  провел  нас  на
третий этаж, потом на четвертый, наконец по узкой лесенке мы  поднялись  в
маленький купол, расположенный на крыше. Здесь  стояли  стулья,  небольшая
подзорная труба на штативе и несколько метеорологических приборов.

Мужчины подошли к оконцам в боковых стенах довольно темного помещения и
принялись  наблюдать.  Мне  досталось   отверстие   размером   с   голову,
проделанное для того, чтобы  выдвигать  подзорную  трубу.  Я  видел  поля,
отделявшие нас от Нью-Йорка с севера. В нескольких километрах дальше  небо
заволакивали дымы большого города.

Поля в эту раннюю весну уже зеленели высокими всходами. Молодой зеленью
покрылись деревья. Глаза напрягались до боли, ибо наблюдал я не приятности
ради: каждая точка, каждое темное пятно казались мне подозрительными.

"--* Есть! Есть! --- крикнул профессор. Все кинулись  к  нему.  Я  попытался
через свое отверстие увидеть что-нибудь там, куда он указывал.

Кто-то уже  опускал  часть  куполообразного  потолка,  открылось  почти
полгоризонта, и я ясно увидел, как в трехстах-четырехстах  метрах  от  нас
какая-то черная фигурка, сильно отражающая солнечный свет, очень  медленно
и ровно двигалась среди молодых хлебов, оставляя за  собой  узкую  дорожку
растоптанных стеблей.

"--* Быстрее, гранатометы! --- Профессор снова  был  спокоен.  Он  приблизил
прицельную рамку  к  лицу.  ---  Внимание,  прошу  одновременно  стрелять  и
наблюдать за его реакциями. Огонь!

Послышалось  довольно  слабое  шипение,  две  полоски   дыма   отмечали
параболическую траекторию гранат, которые разорвались почти одновременно в
нескольких метрах  от  цели.  В  момент  взрыва  я  заметил,  что  фигурка
остановилась и змеевики, которые она несла перед  собой  словно  щупальца,
выдвинулись вперед и в стороны.

"--* Огонь!

Второй залп оказался удачнее, но  взрыв  был  слабый  и  не  смог  даже
перевернуть машину. Мне казалось, что я слышу, как по ней колотят осколки,
но, скорее всего, при таком расстоянии это было иллюзией.

"--* Огонь!

На этот раз гранаты были,  по-видимому,  хлорные.  В  воздух  поднялись
темные облачка газа. Когда ветер немного развеял их, я увидел, что  черный
маленький конус смешно раскачивается из стороны в сторону.

Неожиданно по молодому хлебу прошла длинная тонкая лента  огня,  словно
кто-то лил по земле горящий бензин --- зелень сохла, чернела, и ее мгновенно
охватывало пламя. Огненная лента двигалась с колоссальной скоростью в нашу
сторону и уже почти подходила к зданию.

"--* Огонь!

Новый гул разорвал воздух. Я почувствовал удар в грудь  волны  горячего
воздуха и, задохнувшись, упал на пол.

Когда вскочил и подбежал к краю купола, все уже  было  кончено.  Зелень
еще немного дымилась, но маленький черный  конус  лежал  на  боку,  а  его
щупальца были бессильно разбросаны на стороны.

"--* Есть! Он наш!

Я  повернулся.  Профессор   Уиддлтон   отступил   от   еще   дымящегося
гранатомета, сорвал с лица маску и достал из кармашка сигару.

"--* Господа, прошу вниз. Приступаем ко второму этапу нашей операции.

\bigskip{}



\bigskip{}

3

\bigskip{}


Когда черный блестящий конус наконец замер в неподвижности под слепящим
светом дуговых ламп, инженер Финк ребром руки отер пот  со  лба,  поправил
съехавший на бок  галстук  и,  поглядывая  на  потные  красные  физиономии
окружающих его мужчин, сказал:

"--* Да, теперь он наш. Надо подумать, что делать дальше.

Профессор Уиддлтон, единственный, на ком не  было  свинцово-асбестового
скафандра,  вышел  из-за  наспех  сложенного   экрана,   возведенного   из
поставленных на попа свинцовых плит и, не отрывая глаз  от  своей  золотой
луковицы --- часов, сказал:

"--* Господа. Уже двенадцать минут вы подвергаетесь воздействию облучения.
Соблаговолите пройти со мной.

Мы вышли,  бросая  беспокойные  взгляды  на  этот  таинственный  черный
предмет. Выходя последним, я взглянул еще раз. Странное творение покоилось
на деревянном настиле:  три  змеевика,  выходящих  из  его  боков,  лежали
неправильными кольцами. Знакомое металлическое тиканье доносилось изнутри,
перемежаемое долгими промежутками молчания и  прерываемое  тихим  шипением
или, скорее, звуками трущихся одна о  другую  металлических  поверхностей.
Свет играл на блестящей черной оболочке, сильнее отражаясь  в  тех  точках
поверхности, где разместились выпуклые стекловидные вставки, как бы слепые
глаза. Дрожь пробежала у меня по позвоночнику,  я  с  облегчением  прикрыл
дверь. Мы шли парами по коридору, ведущему к лифту.

"--* Как думаете, доктор, могут ли накапливаться дозы облучения?  То  есть
если мы прервемся на несколько минут или, скажем, на полчаса,  то  вредное
воздействие новой дозы облучения не належится  на  предыдущую?  ---  спросил
Уиддлтон.

Доктор развел руками:

"--* Не знаю, дорогой профессор, я не знаю ничего. Если проводить аналогию
с радием, то надо ориентироваться на более длительные перерывы. По меньшей
мере в несколько дней.

"--*  Скверно,  ---  буркнул  профессор.  ---   Нам   необходимо   действовать
немедленно.

Через несколько минут мы уже сидели в удобных креслах библиотеки.  Я  с
облегчением затянулся хорошей сигаретой. Мускулы еще  дрожали  от  усилий:
нам  пришлось  своими   силами   переносить   обезвреженное   чудовище   в
экспериментальную камеру, пока оно не пришло в себя.

Профессор раскурил погасшую сигару.

"--* Господа, ситуация ясна: либо мы оставим нашего милого  гостя  и  даем
драпака на сотни миль  отсюда,  либо  приступаем  к  операции  немедленно,
сейчас же. Третьего пути не дано, если мы не хотим, чтобы  он  распался  у
нас под руками, превратив нас самих в атомную пыль...

"--* Позвольте, профессор, --- начал Фрэйзер. Лицо у  него  горело  в  лучах
заходящего солнца. --- Единственный способ, который я  вижу,  это  разобрать
машину  на  безвредные  части,  возможно,  отключить  "шарик",   то   есть
регулирующий живой центр. В  оконце  этого  центра,  как  мы  знаем,  есть
несколько небольших отверстий, вероятно служащих для дыхания.  Если  б  не
это, мы не смогли бы так быстро отравить  его.  Так  вот,  повторяю,  этот
центр надо как-то отключить...

"--* Такая попытка равносильна самоубийству, --- тихо заметил  профессор.  ---
Кто согласится сделать это под руководством  и  по  указаниям  двух  наших
инженеров?

Не глядя друг на друга, все мужчины поднялись. Я тоже встал, не отдавая
отчета в происходящем, но тут увидел обращенные на нас черные,  излучающие
тепло глаза профессора.

"--* Благодарю, господа. Так я и думал, но расставаться с  жизнью  впустую
не позволю. Господин инженер, у вас есть какие-нибудь предложения?

Финк что-то лихорадочно чертил в записной  книжке,  выписывая  какие-то
формулы, и вдруг встал:

"--* Есть! Необходимо использовать фотопластинки.

"--* Это ничего не даст, --- сказал я. --- Излучение наверняка засветит их...

"--* Если даже и так, не беда. --- Финка, казалось, не сбило мое возражение.
"--* Экспонируем несколько пластинок в различных местах машины и,  во-первых,
найдем непосредственный  источник  излучения,  а  во-вторых,  быть  может,
составим какое-нибудь, пусть не вполне ясное  представление  о  внутреннем
строении машины.

"--* И каким же образом? --- спросил доктор.

"--* Очень простым: различные части аппарата поглощают генерируемые внутри
него лучи в различной степени и дадут на приложенной  к  конусу  пластинке
следы, подобные рентгеновским, по которым, возможно, мы определим  контуры
внутренних деталей.

"--* Ну, что ж, инженер, --- кивнул профессор. --- Так и сделайте. --- Но тут же
добавил, видя, что Финк встает: --- Однако пусть кто-нибудь пойдет с вами  и
наблюдает сквозь щель в броневых дверях, не случится ли чего-нибудь. Такой
контроль на будущее обязателен. Мы же еще немного побеседуем.

Поскольку, казалось, никто не хотел пропустить  интересного  совещания,
пойти с инженером вызвался я. По дороге  в  камеру  он  прихватил  толстую
пачку фотопластинок, обернутую  слоем  свинцовых  листов,  которые  весили
столько, что мы едва дотащили свой груз до места.  Здесь  инженер  оставил
меня за дверью и велел наблюдать за ним через фильтр из свинцового стекла,
вделанный в броневую плиту двери, а сам взял несколько пластинок и вошел в
камеру. Сняв свинцовую обертку,  он  экспонировал  одну  пластинку,  потом
вторую и так продолжал, прижимая их всякий раз  к  другому  месту  черного
конуса,  двигаясь  по  спирали.  Все  происходило  в  абсолютной   тишине,
нарушаемой только далеким, тихим тиканием внутри машины.

Когда инженер вышел,  я  обратил  внимание  на  то,  что  лицо  у  него
покраснело. Мне подумалось, что это первое проявление вредного воздействия
излучения. Однако я ничего не сказал, чтобы не волновать его, и мы пошли в
лабораторию.

Инженер  провел  меня  в  небольшую  затемненную   комнатку.   Зажглись
маленькие рубиновые лампочки. Забулькали растворы реактивов. Я  присел  на
табурет. Я чувствовал сильное утомление,  мне  казалось,  что  я  не  спал
больше месяца. Но инженер наклонился над пластинками, и я мгновенно  забыл
и про усталость, и про бессонницу.  Пластинка  на  просвет  показала  один
почерневший участок, какие-то нечеткие  параллельные  полосы,  а  в  самом
центре была засвечена  полностью.  На  второй  была  та  же  картина.  Все
остальные оказались засвеченными целиком.

"--* Черт побери, впустую, --- проворчал инженер. --- Надо  повторить.  Только
сократим  экспозицию  в  два  раза.  У  лучей  очень  большая  проникающая
способность.

"--* Вы больше не пойдете, --- сказал я. --- Теперь моя очередь. С вас хватит.
Когда вы вышли из камеры, у вас было красное лицо, а что это означает, вам
известно.

Инженер начал возражать, но я настоял на своем.  Мы  опять  сходили  за
пластинками, и я вошел в камеру. Впервые  я  был  один  на  один  с  нашим
таинственным, смертоносным гостем --- а его тиканье, временами  напоминающее
очень слабый человеческий хрип  (а  может,  это  была  только  игра  моего
воображения), тоже действовало не очень-то успокаивающе.

Я быстро прикладывал пластинки,  сверяясь  с  укрепленным  на  запястье
секундомером, и выбегал из камеры с экспонированными, где  их  принимал  у
меня инженер. Покончив с последней пластинкой,  мы  отправились  в  темную
комнату.

И  снова  потянулись  минуты  ожидания,  пластинки  хлюпали  в  широких
ванночках, какие-то пятна появлялись на стекле, возникали,  усиливались  и
светлели тени... Две пластинки были засвечены. Инженер проверил их номера,
сравнил с планом машины и сказал:

"--*  Центр  излучения  находится  между   двумя   нижними   стекловидными
отверстиями. Именно там засветились две эти пластинки.

"--* А остальные? --- спросил я, пытаясь заглянуть ему через плечо.

"--* Еще минута, только положу в закрепитель.

Секундомер тикал в темноте. Было слышно наше ускоренное дыхание.

Наконец инженер вынул пластинки из ванночки, и мы вышли в коридор.

"--* Вот первая: путаница светлых и темных полос, какие-то линии,  а  это?
Не слабая ли это овальная тень? Да, но ведь это...

"--* Центральная груша, вы правы. Значит, она непроницаема  для  лучей,  и
это говорит о том, что  излучение  неопасно  для  плазмы,  содержащейся  в
груше, так как она изготовлена  из  какого-то  загадочного  материала,  не
пропускающего эти лучи.

Вторая и третья пластинки показали новые детали в  виде  наслаивающихся
теней, темных и светлых пересекающихся полос.

"--* Те, что резче, --- пояснил инженер, ---  провода  или  трубки,  идущие  у
самой поверхности, к которой вы прикладывали пластинки, я  размытые  ---  из
более удаленных частей.

"--* Вы что-то знаете и как-то разбираетесь? --- тихо спросил я.

"--* У вас слишком высокое мнение о моих знаниях, --- улыбнулся  инженер.  ---
Пока что я знаю не больше вашего. Надо будет сделать несколько эскизов.

Мы прошли в лабораторию, где Финк начал с карандашом в руке набрасывать
на большой, приколотый к чертежной доске лист  какие-то  прямые  и  кривые
линии, налагая их друг на друга. На белом листе  возник  клубок  контуров,
который в принципе изображал тело вращения, напоминающее конус.

"--* Двигающий механизм почти ясен... --- ворчал инженер, --- ну и что...  Как
извлечь чертово ядро из скорлупы, вот в чем дело...

"--* Но принцип конструкции в общих чертах вам понятен? --- спросил я.

"--* Все это дьявольски  запутано:  там  есть  части,  несомненно  имеющие
что-то общее с системами трансформации,  но  что,  черт  побери,  является
источником энергии? Понятия не имею. Я не вижу ни одной вращающейся части.

"--* Мне кажется, тиканье, скорее всего, исходит из верхней части  конуса,
"--* заметил я. --- Впрочем, возможно, я ошибаюсь.

"--* Ну, мне тоже это пришло в голову. Там есть подвижная часть --- вот эта,
"--* решил он, указывая на  размытую  тень,  что-то  вроде  неравнобедренного
треугольника, который выглядел, как...

"--* Ну, конечно! --- воскликнул я. --- Это же копия волчка, детского волчка.

"--* Вы  думаете?  ---  наморщил  брови  инженер.  ---  Принцип  гироскопа,  а
следовательно, его сердце --- гироскоп. Пожалуй, вы правы, --- сказал он после
недолгого раздумья и нанес на рисунок несколько  линий.  Теперь  в  центре
тела вращения стал  виден  волчок,  похожий  на  два  конуса,  соединенных
основаниями. Волчок находился в пробеле центральной тени --- как  бы  трубы,
которая  шла  посредине  машины,  разрываясь,  чтобы  принять  волчок,   и
оканчивалась наверху розеткой, на которой размещалась таинственная  груша.
"--* Пошли, --- сказал инженер, сорвал  с  доски  несколько  кнопок,  державших
бумагу, свернул лист в трубку и взял под мышку.

Наше  появление  было  воспринято  с  напряженным  ожиданием.   Инженер
разложил бумагу перед профессором и принялся кратко пояснять.

"--* Принцип действия мне совершенно непонятен, ---  сказал  он.  ---  Я  вижу
единственный путь: задержать излучение. Это необходимое условие.

Уиддлтон прищурившись смотрел на него и молча слушал.

"--*  Поскольку  ---  хотя  и  не  могу  это  утверждать  со   стопроцентной
уверенностью --- центральным механическим элементом  является  этот  волчок,
или гироскоп, обнаруженный Макмуром, ---  тут  все  удивленно  взглянули  на
меня,  ---  постольку  необходимо  его  остановить.  Эта  размытая  тень  на
пластинке ---  единственная  подвижная  деталь.  Напрашивается  возможно  не
слишком умное сравнение этой детали с человеческим сердцем, но,  остановив
это "сердце", мы, пожалуй, сумеем заняться демонтажем...

"--* И как вы намерены это сделать? --- спросил профессор.

"--* Я  вижу  единственный  способ.  Трагический  опыт  профессора  Гавлея
показал, что трогать  центральную  грушу  с  плазмой  нельзя.  За  попытку
извлечь ее он поплатился жизнью. Значит, надо  пробить  панцирь  конуса  в
этом месте, --- он бросил красный мелок на лист, --- и с  помощью  какого-либо
инструмента остановить гироскоп.

Наступило молчание. Его прервал доктор.

"--* Допустим, нам удастся тихо и  безболезненно  высверлить  отверстие  в
панцире. Однако, я думаю, его так  называемое  "сердце"  каким-то  образом
защищено от внешних  помех  и  попытка  остановить  его  может  окончиться
плачевно.

"--* Я в этом просто уверен, --- сказал инженер, --- но иного пути не вижу.

Уиддлтон внимательно рассматривал рисунок, сравнивал его  со  снимками,
потом взглянул на часы и сказал:

"--* Господа. Дело тут не в мудрости или глупости. В данном случае  я  уже
не  ваш  руководитель.  Давайте  проголосуем,  следует  ли   нам   принять
предложение инженера Финка? Прошу как следует подумать: может  быть,  есть
другие идеи?

"--* У меня есть предложение, --- сказал я. ---  Отверстие  можно  просверлить
управляемой на расстоянии дрелью. Это несложно сделать.  А  наблюдение  за
всем вести с помощью телевизионной камеры и действовать соответственно.

"--* Соответственно --- это как? --- спросил Фрэйзер.

"--* Может быть, удастся создать дистанционно  управляемый  прибор,  вроде
робота, который мог бы демонтировать... гостя...

"--* Мысль отличная, --- сказал инженер, ---  но,  к  сожалению,  у  нас  мало
времени. Аппаратов такого  рода  здесь  нет,  а  чтобы  доставить  ---  даже
самолетом, --- потребуется как минимум три дня.

"--* Столько я не дам, --- сказал профессор. --- До двенадцати, самое  большее
до половины первого мы уже должны чего-то добиться.

"--* За столь короткое время такого оборудования нам не достать, ---  сказал
Финк. --- Но есть другая возможность: взорвать конус, например, экразитом.

"--* Что? Уничтожить? Ни за что! --- раздались голоса.

"--* Я горжусь вами, господа! --- Профессор Уиддлтон встал. --- Спрашиваю  еще
раз: следует ли нам поддержать первое предложение инженера Финка?

"--* Да!

"--* В таком случае --- за работу. --- Профессор смотрел на  Финка.  ---  Каковы
оптимальные условия защиты?

Финк задумался.

"--* Все должны носить скафандры, в коридоре тоже. В камере  всегда  будет
находиться только один человек, газовые гранаты и противогазы должны  быть
в полной готовности. Первый этап --- сверление. Думаю, это  удастся  сделать
так, как предложил Макмур. Только наблюдать надо  будет  сквозь  глазок  в
двери. Что дальше --- увидим.

Коридор опустел. Пришла моя очередь --- я стоял у стальных дверей,  держа
у рта трубку телефона, и напряженно глядел внутрь камеры.  Я  был  вторым,
после доктора, и видел, как в бледно-голубом электрическом свете шипела  и
тихо свистела дрель, подвешенная на  консоли,  закрепленной  на  козлах  с
таинственной машиной.

Сверло  большого  диаметра  из  кремний-ванадиевой  стали  впивалось  в
твердую  оболочку  конуса.  Одной  рукой  я  сжимал   трубку,   другой   ---
электровыключатель дрели, выведенный за двери камеры, и ждал. Пока  ничего
не  происходило  ---  сверло  углублялось  почти  незаметно,  но,  уже  зная
способности механического чудовища, я был напряжен до предела.

"--* Ну, что там? --- раздался в трубке голос профессора.

"--* Все по-прежнему, --- ответил я. --- Дырка в зубе сверлится, но  чертовски
медленно. Может, сменить сверло?

Было слышно, как профессор разговаривает с кем-то, видимо с  инженером,
и вдруг я окаменел. Длинный --- метра два, а то и больше ---  черный  змеевик,
лежавший на настиле, дрогнул, потом пошевелился второй, слабая волна спазм
пробежала по стальным виткам.

"--* Профессор, --- сказал я и тут же рявкнул  в  трубку:  ---  Он  шевелится,
шевелит щупальцами! Выключить дрель?

"--* Ни в коем случае, продолжайте, ради Бога! --- раздался  слабый  далекий
голос.

Я ждал. Я не трус и никогда им  не  был,  но  чувствовал,  что  начинаю
покрываться холодным  потом.  Ждал  и  знал,  что  вот-вот  что-то  должно
случиться. И то, что опасность была таинственной и неведомой, пугало  меня
еще сильнее, нежели угроза смерти.

Один змеевик поднялся, свистнул в воздухе, как стальной бич,  и  ударил
по дрели. Раздался  тонкий,  высокий,  сверлящий  звук  ломающейся  стали,
разлетелись осколки. Я нажал кнопку выключателя --- дрель остановилась.

"--* Он сломал своей треклятой лапой сверло! --- закричал я в трубку.

"--* Я сейчас приду, --- откликнулся профессор.

Я ждал. Тем временем марсианское творение успокоилось, и  я  больше  не
замечал никаких признаков жизни.

Профессор подошел почти беззвучно в сопровождении инженера, который нес
новое сверло. Я отошел в сторону, и они заглянули в глазок.

"--* Говорите, махнул змеевиком?  ---  покачал  головой  профессор.  ---  Ведь
упрямая скотина, а?

"--* Надо бы дать ему порцию газа. Может быть,  применить  какое-то  новое
средство --- хлороформ или эфир, а?

"--*  Чтобы  окончательно  его  отравить?  ---  сказал  профессор  с   таким
возмущением в голосе, словно мы стояли у постели его больного друга.

Финк кивнул.

"--* Профессор  прав.  Потерю  любого  из  нас  можно  восполнить,  а  вот
уничтожив его, мы уже исправить ничего  не  сумеем,  ---  сказав  это,  Финк
отодвинул засов камеры и вошел внутрь.

Мы ждали, затаив дыхание.  Инженер  медленно  отодвинул  ногой  обломки
сверла, укрепил в патроне дрели новое, установил его как следует и  вышел.
Внешне он был спокоен, но, выйдя в коридор, отер платком лоб.

"--* Ну, поехали дальше! Макмур, включайте!

Я нажал кнопку. Дрель завыла, сверло впилось в конус.

Долгие минуты уходили в напряженном ожидании.

"--* Вроде бы все идет хорошо, --- сказал профессор. --- Пошли,  Финк,  а  вы,
Макмур, еще минут десять постойте. Потом вас сменит Джедевани.

Они ушли, а я почувствовал себя ужасно одиноким. Напряженно  следил  за
происходящим.

Опять волны прошли по  бессильным  щупальцам.  Свесившиеся  с  настила,
дергались на бетонном полу пружинистые, круглые в сечении змеевики"

Неожиданно щупальца медленно поднялись и повисли в воздухе --- только  их
концы мелко дрожали, раскачиваясь из стороны  в  сторону.  Я  увидел,  как
корпус дрели вздрогнул, сверло легко  прошло  внутрь.  Так!  Готово!  Есть
отверстие, подумал я  и  тут  же  совершенно  автоматически  нажал  кнопку
выключателя. Но это было уже лишним.

Воздух прорезала  голубая  вспышка.  Потом  чудовищный  порыв  горячего
воздуха отбросил меня к противоположной стене коридора  ---  я  почувствовал
сильнейший удар по голове и потерял сознание.

\bigskip{}



Когда я проснулся, было утро. Я лежал в своей комнате, а  доктор  сидел
на моей кровати и раскладывал на одеяле пасьянс.

"--* Ага! Итак, мы снова живы. Ну,  как  чувствуем  себя?  ---  спросил  он,
собирая карты.

"--* Доктор, --- с трудом открыл я пересохший  рот,  ---  как  там?  Он  снова
сбежал?

"--* Вас интересует не дырка в собственной голове, а только судьба  нашего
любимчика с Марса? Так, что ли? Нет, не сбежал.  Это  была  его  последняя
шуточка. Сверло угодило в тот хитроумный механизм, в то квазисердце, и  он
в своем квазиумирании наделал, ох, наделал дел. Что там творилось, ого-го!
Вы, вероятно, этого не видели, да и как могли? --- качал головой  доктор.  ---
Мы прибежали, коридор полон пыли, штукатурка осыпалась... Ну, думаю, конец
нашему репортеру. Двери в камеру почти  вырваны,  висят  на  одной  петле,
помост сломан, их превосходительство марсианин лежат на полу, а дрель, уму
непостижимо, расплавлена --- ничего от нее не осталось. Инженер  утверждает,
что в течение минуты температура там держалась на уровне шести-семи  тысяч
градусов... Как вы себя чувствовали в этой баньке?

"--* Помню только вспышку и страшный удар  по  голове,  а  до  того  волну
кипятка --- вероятно, это был воздух.

"--* Вас спасли двери и  то,  что  вы  целиком  завернулись  в  асбестовую
дорожку, когда падали. Благодарите Финка: дорожки из асбеста ---  его  идея.
Порыв воздуха завернул  вас  в  асбест,  как  драгоценную  покупку,  и  не
стукнись вы слегка головой...

"--* Как там? --- я вытянул из-под одеяла руку в  синяках  и  дотронулся  до
головы. Она немного шумела, но была цела. Только узкий бинт пересекал лоб.

"--* Была дырочка, а  как  же!  ---  не  переставал  болтать  доктор.  ---  Но
шотландцы, ох уж эти шотландцы! У вас крепкие головы, да и  похороны  тоже
обходятся недешево, верно? Ха-ха-ха! --- смеялся он. --- Так что вы решили еще
пожить.

"--* Послушайте, доктор. Бога ради, что происходит? Что они делают?

"--*  Так,  вы  вспоминаете  Бога,  да?  А  позавчера,  когда  вы  от  нас
вырывались... Ну, ну, все, молчу, молчу, --- добавил  он.  ---  Понемногу  его
разбирают. Металлическое сердце остановилось от удара  сверлом.  Камера  с
плазмой уже, вероятно, ждет меня в лаборатории. --- Он взглянул на часы. --- Я
ведь, между нами говоря, врач так себе, в шутку, а в действительности-то я
биолог. Биолог по убеждениям, --- чуть не пропел он, собираясь уходить.

"--* Я с вами!

"--* Да вы сдурели! Лежать! И точка!

Я встал с постели. Ноги были ватные в коленях, в голове немного шумело,
но в остальном я чувствовал себя неплохо.

Я быстро оделся, взял доктора под руку, и  мы  вышли  в  коридор.  Часы
показывали десять. Уже или еще?

"--* Ясно, сейчас будет совещание, --- сказал я. ---  Пойду  в  библиотеку.  ---
Доктор кивнул и двинулся к лестнице, ведущей в  лабораторию.  Я  спустился
лифтом на второй этаж и был там встречен с энтузиазмом.

"--* Ото! Герой дня! Как вы себя чувствуете?

Я пожимал всем руки. Профессор, улыбнувшись, кивнул мне.

У окна сидел инженер Линдсей. Он был бледен,  но  каких-либо  признаков
слабости не проявлял.

"--* Приветствую друга по несчастью! --- сказал он.

Инженера Финк отсутствовал.

"--* Как дела у нашего гостя? --- спросил я.

"--* Интересные вещи, дорогой мой, интересные. Все идет  неплохо.  Плазма,
похоже, в порядке --- в груше наблюдается нормальная пульсация.

"--* А что с излучением?

"--* Прекратилось. Сразу же после взрыва. Сейчас он безопасен, как  старая
пустая консервная банка. --- Профессор рассмеялся мелким хохотком. ---  Теперь
план  прост:  надо  изъять  все,  что  связано  с  генерацией   излучения,
ликвидировать возможность этих безобразий --- потоков огня, кипения  воды  в
пруду, всего этого цирка. А потом постараемся его собрать и начнем  с  ним
болтать.

"--* Что значит --- "болтать"? --- удивленно спросил я.

"--* Ну, как-то он, наверное, нас воспринимает. Поместим его  в  атмосферу
Марса. Думаю, весь шум, который он  устраивал,  вся  эта  чехарда  явились
результатом ядовитого воздействия нашего воздуха, а может,  и  повышенного
земного тяготения.

В этот момент вошел инженер.

"--* Господа... О, Макмур уже здесь, очень рад, --- он поздоровался со мной.
"--* Господа, перед нами твердый орешек. Коллега, --- обратился он к Линдсею, ---
насколько я  понимаю,  машина  приводится  в  движение  атомной  энергией,
которую она черпает из небольшого  кусочка  урана,  помещенного  в  нижней
части конуса. Эта энергия в  виде  электрического  тока  используется  для
передвижения,  а  специальная  аппаратура  позволяет  передавать   ее   на
расстояние в виде тепловой  либо  магнитной  энергии.  Эту  аппаратуру,  я
думаю, можно  демонтировать,  но  сама  радиоактивность  ---  условие  жизни
машины. Ликвидировав  радиоактивное  излучение,  мы  тем  самым  остановим
функционирование всего устройства, выключим  его.  Конечно,  можно  просто
ослабить машину, убрав специальные приспособления  для  усиления  энергии,
управления и передачи.

Профессор задумался.

"--* А нельзя ли запустить машину без центральной груши?  Я  имею  в  виду
запуск стального сердца...

"--* Попытаюсь, но не уверен. Я не знаю деталей конструкции: эта чертовски
сложная машина построена, кстати, совершенно поразительно,  совершенно  не
по-человечески.

"--* Еще бы,  ---  улыбнулся  профессор.  ---  И  как  же  оно  выглядит,  это
совершенно нечеловеческое поразительное устройство?

"--* Не смейтесь, основные части сменные, но добраться до них  невозможно.
У меня здесь самый лучший комплект  инструментов,  о  каком  только  может
мечтать техник, и он не справляется. Вместо винтов  там  очень  остроумные
соединения. --- Инженер вынул из кармана два кусочка металла.  ---  Взгляните,
профессор.

Это было что-то вроде двух болтов. Инженер составил их плоскими концами
и повернул на сто восемьдесят градусов.

"--* А теперь попытайтесь разъединить.

Лицо старика покрылось румянцем.

"--* Что еще за колдовство!

Инженер снова повернул "болты" вокруг длинной оси  и  легко  разъединил
их.

"--*   Какая-то   разновидность   притяжения.   В   таком   положении,   ---
продемонстрировал он, --- не действуют никакие силы. Однако  если  повернуть
болты вот так, разорвать их невозможно.

"--* Невозможно руками, но в тисках... --- заметил Фрэйзер.

"--* У меня уже есть такая пара.  Я  пробовал,  ---  ответил  инженер.  ---  В
разрывной  машине  я  подверг  их  растяжению  силой  в  пятьдесят   тысяч
килограммов, и они разорвались, но не в месте соприкосновения, а  рядом  с
головкой. Однородный материал лопнул,  а  место  простого  соприкосновения
выдержало! --- Он бросил обломки на стол. --- Вот  это  изобретение!  Не  надо
никаких винтов и гаек, одно движение --- и все держится, словно сваренное.

"--* Как вы думаете, каков механизм действия? --- спросил Джедевани.

"--* Это, скорее всего, ваша область. Я думаю, что-то вроде магнита,  двух
магнитов... Да что там, --- он махнул рукой,  ---  тысячи  подобных  мелочишек
упрятаны в дьявольской машине, и нам  не  известно,  с  чего  начать.  Где
доктор?

"--* Сказал, что пойдет в лабораторию, --- ответил я.

"--* Ах да, центральная груша... Вот это  ---  настоящая  загадка.  Все  его
механические фокусы еще можно в конце концов понять...

"--* Господа, --- сказал  профессор,  ---  теперь  мы  разделимся  на  группы.
Инженеры   и   доктор    попытаются    изучить    элементы    конструкции,
функционирование машины и ее живого организма, а мы,  ---  он  повернулся  к
Фрэйзеру,  Джедевани  и  ко  мне,  ---  подумаем   о   том,   как   добиться
взаимопонимания  с  нашим  гостем,  если,  конечно,  нам   удастся   после
обезвреживания его оживить...

Когда мы удобнее устроились в креслах,  профессор  взглянул  на  нас  и
сказал:

"--* Друзья мои, нам кажется, будто мы уже  обуздали  пришельца  с  Марса.
Возможно, вы так именно и думаете. Однако я считаю, что та  часть  работы,
которая выпала на нашу долю, окажется сложнее первой, хотя, может быть, не
столь опасной. Уничтожать всегда легче, нежели создавать.  Это  первое.  А
второе --- проблема общего языка. Что вы об этом думаете? --- обратился он  ко
мне.

Я удивился.

"--* Я польщен, профессор, вашим обращением ко мне,  но,  по-моему,  не  я
должен быть первым...

"--* К чему красивые  слова,  дорогой,  к  чему!  Именно  то,  что  вы  не
предубеждены и, возможно, не столь перегружены балластом знаний,  как  мы,
наверняка облегчит вам принятие нетривиального решения. Я наблюдал за вами
в различных ситуациях и видел, что вам свойственны свежие суждения, что вы
обладаете, я бы сказал, весьма оригинальным мышлением.

Я поклонился.

"--*  Думаю,  начать  следует  с  геометрического  языка:  концентрические
окружности, какие-то  простые  уравнения  типа  тех,  что  создал  великий
Пифагор. Вот возможный путь к контакту.

"--* Мне это приходило на ум, --- заметил Фрэйзер, --- но  такова  может  быть
первая стадия. А дальше?

"--* Все зависит от его реакций. Например, от того, каким образом он  даст
нам знать о себе? И вообще, видит ли он в  нашем  понимании  этого  слова?
Какие участки видимого спектра он воспринимает и каковы  его  реакции,  то
есть способы проявления происходящих в нем жизненных процессов?

Профессор протер платочком очки, нацепил их на нос и  долго  глядел  на
меня. Я вспомнил школьные годы и скуксился. Может, ляпнул глупость?

"--* Видимо, я вас недооценил, --- проговорил  старик.  ---  Да,  да,  начинаю
стареть... Вы напомнили мне то, что я  сказал  вчера:  слова  Ньютона.  Не
перебивайте. Тут дело не в  намерениях.  Прежде  чем  мы  пожелаем  с  ним
познакомиться, необходимо его познать. Это самая настоящая "вещь  в  себе"
Канта. Вот в чем секрет.

В этот момент на полочке камина замигал красный сигнал. Фрэйзер подошел
к нише и снял трубку телефона.

"--* Доктор вызывает  нас  в  лабораторию.  Может,  какие-то  новости.  Мы
спускаемся, --- бросил он в трубку.

Все встали. Синьор Джедевани вынул из кармана щеточку, почистил пиджак,
изучающе глянул в зеркало, висевшее между шкафами, и направился  к  двери.
Мы последовали за ним.

В  лаборатории  находился  только  доктор.  На  длинном  столе   стояло
несколько аппаратов, а таинственная черная груша была укреплена на штативе
словно какой-то ядовитый, но уже  безопасный  фрукт,  и,  как  я  заметил,
соединялась с чувствительным гальванометром.

"--* Интересное дело: она испускает слабые токи, словно возбужденная живая
плазма, ---  повернулся  к  нам  доктор.  ---  Подобные  тем,  какие  излучает
возбужденный человеческий мозг.  Надо  достать  хороший  и  чувствительный
регистрирующий прибор. Быть может, это путь к пониманию. Взгляните.

Доктор взял в руку небольшой  электрический  фонарик  и,  включив  его,
поднес к той части груши, на которой было прозрачное оконце. В тот момент,
когда на оконце  упал  луч  света,  стрелка  гальванометра  несколько  раз
довольно сильно качнулась.

"--* Типичная фототропическая реакция, --- сказал доктор.

Профессор не проявлял особого воодушевления.

"--* Я думаю, такой путь потребует множества  длительных  опытов.  Что  вы
собираетесь делать?

"--* Буду помещать перед оконцем свет различной  интенсивности,  различных
цветов и оттенков --- может быть, какие-то рисунки --- и изучать реакцию.

"--* Электрическую?

"--* Ну да, пока что иное невозможно. Вы видите: у  груши  внизу  двадцать
семь  тонких  проволочек,  которые  связаны  с  соответствующими  гнездами
розетки в машине. Я исследовал ток в этих проводках и обнаружил интересную
вещь: некоторые реагируют на свет, другие --- нет.

"--* Сколько из них реагируют на свет? --- спросил я.

"--* Кажется, три.

Доктор соединил два других проводка с  гальванометром  и  показал,  что
теперь свет не вызывает никакой реакции.  Я  подошел  и  приложил  руку  к
груше. Стрелка гальванометра дрогнула.

"--* Ого, может быть, действует тепло  вашей  руки?  Значит,  таким  путем
можно регистрировать термические изменения? Попробуем иначе.

Он начал новые опыты. Из хаоса фактов он, казалось, создавал все  более
ясную  картину:  тоненькие  проволочки  служили   рецепторами   физических
изменений  в  окружающей  среде.  Мы  один  за  одним  выявили   рецепторы
напряженности электрического  поля  и  его  частоты,  но  это  было  всего
несколько проволочек. Подавляющее же  их  число  оставалось  загадкой.  Мы
применяли  химические,  тепловые,  магнитные  и  звуковые  раздражители  ---
никакого результата.

"--*  Трудное  дело,  ---  сказал  наконец  доктор.  ---   Может   быть,   эта
самодостаточная груша вовсе не "вещь в  себе"?  Что,  если  организованная
плазма на Марсе пошла иным путем, нежели на Земле: у нас  ей  пришлось  на
путях эволюции создавать себе и  из  себя  двигательный  аппарат,  систему
питания и нервную систему, а на Марсе было иначе, гораздо проще. Создалась
мыслящая, но очень беспомощная плазма, которая ускорила  эволюцию,  создав
себе машину для перемещения, зрения, слуха и защиты от опасности. В  таком
случае, исследование самой груши мало что даст.

Профессор слушал его и кивал.

"--* Да, да... Это-то как раз и есть нечеловеческая  поразительная  штука,
как кто-то из нас выразился вчера, с которой может встретиться  человек...
"--*  повернулся  он  к  доктору.  ---  Ничего  страшного,   не   отчаивайтесь,
продолжайте опыты. А теперь мы отправимся к нашим конструкторам.

В большом монтажном зале, который я видел впервые, стоял дикий лязг. На
длинных шинах из прессованного  эбонита  медленно,  хотелось  бы  сказать,
"величественно", передвигались две огромных фарфоровых колонны, на которых
покоились   большие   никелированные   шары.   Между   шарами   извивалась
светло-фиолетовая громыхающая и  трепещущая  молния,  которая  то  и  дело
пыталась  сорваться  с  гигантского  искрового   разрядника.   Эхо   грома
отражалось от стеклянного потолка.

Под ногами дрожал бетонный пол. В первый момент мне показалось, что зал
пуст, но  тут  же  я  увидел,  что  между  платформами,  на  которых  были
смонтированы колонны и шары разрядника, стоял  изготовленный  из  матового
металла прибор, похожий на перевернутую гигантскую грушу, а рядом с ним  ---
маленькая фигурка в асбестовом скафандре. Когда она повернулась, я  увидел
за оконцем из свинцового стекла блеск  зубов.  Это  нам  улыбался  инженер
Финк. Он поднял вверх обе руки  и  скрестил  их.  По  этому  знаку  молния
исчезла, и мои уши наполнились гулким звоном неожиданно возникшей  тишины.
Инженер сбросил с головы капюшон и принялся вытирать вспотевшее лицо.

"--* Все идет хорошо, --- сказал он. ---  Мы  пытаемся  "завести"  машину,  не
прибегая к атомной энергии, которой еще не умеем управлять,  а  для  этого
требуется от двух до трех миллионов вольт.

"--* Удастся вам к вечеру создать хотя бы примерный эскиз  машины,  понять
принцип ее действия, обнаружить отдельные органы и, самое важное,  постичь
конструктивную идею, в ней заключенную?

"--* Вы слишком многого требуете, профессор, --- покачал головой инженер.  ---
Я попытаюсь, но предупреждаю: не питайте особых иллюзий. Хуже  всего,  что
машина чертовски проста и в то же время в ней так много  всего  происходит
без участия каких-либо ясных для меня устройств преобразования  энергии  ---
страх берет...  Например,  атомная  энергия  переходит  непосредственно  в
электрическую либо тепловую... Взгляните.

Он провел нас в затемненный угол зала. Там  стоял  Линдсей,  который  в
этот момент укреплял внутри неподвижного  черного  конуса  марсианина  два
толстых кабеля, унизанных фарфоровыми бусами изоляторов.

Инженер подтолкнул нас к небольшой камере из свинца,  молча  указал  на
визир  и  вышел.  Я  еще  не  видел,   как   он   подошел   к   мраморному
распределительному щиту на стене и передвинул  большой  рубильник.  Воздух
снова  разорвал  оглушительный  грохот  искусственной  молнии.  Фиолетовые
вспышки осветили все уголки зала. Призрачный свет играл на наших лицах.  Я
взглянул на конус --- инженер  Линдсей,  стоявший  рядом  с  ним,  прикрепил
что-то  и  вдруг  сунул  руку,  одетую  в  огромную  красную  перчатку,  в
отверстие, просверленное дрелью.

Кажется, я крикнул. Беспорядочно лежавшие  змеевики  конуса  задрожали,
зашевелились и стали, словно в приступе бешенства, биться о  пол.  Инженер
продолжал манипулировать. Теперь щупальца медленно поднялись ---  дрожали  у
них только концы --- и повисли в воздухе. Наконец один из них приблизился  к
свисающей на веревке стальной плите. Я не понимал,  зачем  там  висит  эта
штука, но очень скоро все стало ясно. Тупой черный конец змеевика  подошел
вплотную к стальному блоку. Быть может,  мне  это  только  показалось,  но
центр плиты налился красным. Неожиданно веревка задрожала, стальная  плита
начала раскачиваться на манер  огромного  маятника,  и  тут  в  ее  центре
возникло отверстие, сквозь которое щупальце свободно  проникло  на  другую
сторону, а затем снова отступило. Линдсей поднял левую  руку  ---  молния  с
треском оборвалась и погасла, и щупальца бессильно упали на землю, прервав
необычное представление, иллюзию жизни.

Мы вышли из кабины.

"--* Линдсей не ошибся, --- сказал Финк, провожая  нас  к  двери.  ---  А  это
только одна из многих возможностей...

"--* Вы все время только и говорите о технических возможностях,  ---  бросил
профессор.

"--* О чем же еще говорить? --- Инженер, казалось, не понял.

"--* Я понимаю, это ваша область, но ведь и доктор тоже... Дело-то в  том,
как нам подойти, приблизиться... А то, что делаете вы, только отдаляет. Не
смею давать каких-либо указаний, но прошу учесть: нам важен синтез. Анализ
тоже необходим, но нельзя плутать в его деталях.  Насколько  проще  будет,
если он сам нам все объяснит.

Инженер беззвучно рассмеялся.

"--*  Нет,  не  зря  вас  называют  старым  метафизиком,  пожалуйста,   не
обижайтесь, профессор...

"--* Потому что я верю в Бога и в другие странные в вашем понимании  вещи,
которые у тупоголовых не могут уместиться на грядках мозговых  извилин?  ---
тихо спросил профессор. --- Разве можно на  это  обижаться?  Так  понимаемое
прозвище "метафизик" --- просто комплимент.

Он пожал руку инженеру и вышел из зала.

\bigskip{}



\bigskip{}

4

\bigskip{}


И опять мы сидели под белым  светом  матовых  ламп,  напряженно  слушая
инженера Финка, который разложил на столике кипы своих бумаг.

"--* Итак, я уже понял устройство двигательной системы нашей машины, сутью
которой  является  непосредственное  использование  напряжения,  то   есть
преобразование излучения во вращательное движение валиков. Мы не понимали,
почему так медленно и вроде бы неуклюже она перемещается, почему у нее нет
хватательного   аппарата.   Оказывается,   возможности   наших    неловких
приспособлений, сконструированных по образу руки, значительно ниже, чем  у
марсианского механизма. Дело в том,  что  щупальца,  или  змеевики,  могут
испускать из своих концов, которые я назвал эмиттерами, энергию, способную
создавать тепловое, магнитное или электрическое  поля.  Одним  словом,  за
счет   синхронизации   собственных   колебаний    атомов    и    колебаний
соприкасающегося с ними вещества могут осуществляться такое  притяжение  и
такая фиксация, каких мы не получили  бы  даже  при  винтовом  соединении.
Трагический конец Уайта, который находился во дворе в  тот  момент,  когда
ареантроп вырвался из дома, был вызван его полным распылением на атомы.  И
это объясняет тот факт, что он исчез совершенно неожиданно и  от  него  не
осталось  ничего.  Относительно  энергетических  проявлений   деятельности
машины мы уже хоть немного, но знаем. Хотя, подчеркиваю,  далеко  не  все.
Беда в том, что у нас нет датчиков  ощущений  и  физических  приборов  для
регистрации напряженности волн материи, а мне  кажется,  именно  колебание
материи является основным фактором в функционировании некоторых  элементов
конструкции аппарата. Это одна  сторона  дела.  Что  касается  регистрации
воздействия внешнего мира на машину, и в первую очередь на ее живое  ядро,
помещенное в груше, то об этом я могу, к сожалению,  сказать  очень  мало,
перепоручив  это  доктору.   Правда,   у   ареантропа   есть   устройства,
напоминающие  упрощенную  светочувствительную   систему,   есть   какие-то
биметаллические соединения,  служащие,  возможно,  для  фиксации  разности
температур, напряжений, внешнего давления, --- но это только элементы.  Пути
от них доходят до центральной трубы и там оканчиваются, не соединяясь ни с
чем. Труба эта заполнена чем-то  вроде  жидкости...  но  это,  собственно,
никакая не жидкость. --- Он поставил на стол небольшую стеклянную  колбу.  ---
Вот проба этого вещества. Я могу  о  нем  сказать  только  одно:  по  всей
вероятности, это органика... Впрочем, я слишком плохой химик, чтобы  точно
определить  состав.   Во   всяком   случае,   действие   этой   "жидкости"
поразительно. Впрочем, убедитесь сами. Никакие слова не в состоянии  этого
описать.

Мы переглянулись, как бы подумав: сейчас начнутся чудеса.

"--* Смелее, доктор, --- сказал Финк, --- и, вынув пробку, поднес колбу ему  к
носу. Доктор осторожно втянул воздух, лицо у него дрогнуло, он вдруг  чуть
ли не вырвал сосуд из рук инженера, лихорадочно вдыхая.

Лицо у него сначала покраснело, потом побледнело, он упал  в  кресло  и
прикрыл глаза.

"--* В чем дело, инженер?! --- выкрикнул профессор. --- Это не отрава?

Финк подошел к доктору, тот позволил ему взять колбочку и подать мне. Я
решил быть крайне осторожным и только чуть-чуть потянул носом.

Не знаю, что со мной случилось. Вначале я увидел невыразимо прекрасные,
туманные, медленно вращающиеся круги. Потом  возникла  мешанина  резких  и
мягких, но очень гармоничных цветных пятен. Все  это  сливалось  в  единую
палитру цвета, света и аромата. Аромат не был  приятен,  скорее  наоборот,
как сказал бы я  теперь,  когда  пишу  эти  строки  и  пытаюсь  по  памяти
воспроизвести его характер, но неприятность эта  оказалась  сладостной  до
боли. Это было ощущение могучей и бурной жизни, приносящей  наслаждение  с
каждым ударом сердца, движением мускулов, вдохом, ---  и  все  было  как  бы
прикрыто шелковой вуалью. В  то  же  время  я  прекрасно  видел  все,  что
творится вокруг, мысли мои были ясными как  никогда,  все  казалось  таким
четким и цветным,  словно  я  смотрел  через  какой-то  волшебный  прибор.
Кто-то, кажется инженер, попытался отнять  у  меня  сосуд.  Я  лихорадочно
прижал его, как бы желая задержать, но почувствовал легкое  недомогание  и
отпустил.

Теперь я не удивляюсь... не удивляюсь уже ничему. Я увидел, как доктор,
скорчившись в кресле, плакал. И у меня слезы  наворачивались  на  глаза  ---
таким страшным казалось возвращение в себя, в того  себя,  который  минуту
назад был вполне удовлетворенным жизнью, нормальным  человеком,  а  теперь
чувствовал себя несчастным, словно изгнанным из потерянного навсегда  рая.
Я понимал, конечно, что это смешно, что я --- старый, глупый  мерин,  трезво
мыслящий репортер, и все же спазм грусти и тоски стиснул мне горло.

Колба шла по кругу,  даря  каждому  минуту  нечеловеческого  счастья  и
горького человеческого разочарования.

Профессор отказался взять сосуд.

"--* Вероятно, какой-то наркотик, --- сказал он. ---  Я  не  любитель  травить
себя гашишем или опиумом.

"--* Отнюдь, профессор. Простите, что я перешел дорогу нашему  биологу.  ---
Финк поставил на стол вынутую из портфеля баночку.

В ней  сидели  две  лягушки,  одна  малюсенькая,  худенькая,  вторая  ---
ненормальной величины, как бы раздувшаяся.

"--* И что это за  чудо  с  Марса?  ---  ехидно  прошипел  доктор,  стараясь
сохранять видимость спокойствия. Он, как и все мы,  чувствовал  неловкость
за ту грусть, которая охватила нас после того, как у нас  отняли  странную
жидкость.

"--* А то, что обе лягушки еще утром были головастиками. Только большой  я
добавил в воду аквариума капельку этой жидкости, вот  и  все,  ---  спокойно
закончил инженер.

Профессор сурово глядел на нас.

"--* Сдается мне,  синьора  Джедевани  сильнее  всех  нас  влечет  к  этой
чудотворной бутылочке, а? Так вот, напоминаю, мы здесь не какие-то люди  с
улицы и даже не просто ученые, а земная делегация, в задачу которой входит
знакомство с пришельцем с Марса. Следует ли  говорить,  как  должны  вести
себя члены такой делегации?

Мы опустили глаза. Как ни говори, профессор был слишком резок. Ведь  он
не  испытал  на  себе  пугающего  и  одновременно  чудесного   воздействия
жидкости.

Старик, казалось, читал наши мысли.

"--* И будь сия субстанция даже живой водой, я позволю себе напомнить, что
таковую называют aqua vitae, а  это  ---  шутливое  наименование  водки.  По
окончании наших исследований  каждый  волен  посвятить  себя  изучению  ее
достоинств... Я никому запрещать не намерен.

Старик явно издевался, но я чувствовал, что он прав.

"--* Профессор, --- сказал я, --- никто ни в чем не повинен. Уверяю  вас,  все
будет хорошо, и позволю себе обещать это от лица всех  присутствующих.  Мы
прежде всего люди, и именно поэтому будем поступать так, как того  требует
ситуация.

"--* Я этого ожидал, --- сухо  докончил  профессор.  ---  На  будущее  попрошу
воздержаться от подобных демонстраций, господин инженер, а сейчас хотелось
бы услышать, что скажет доктор.

Впоследствии инженер признался мне, что  вся  история  с  демонстрацией
центральной жидкости ареантропа имела целью показать  профессору,  что  он
точно такой же "слабый  человек",  как  и  мы.  Однако  ловушка  оказалась
слишком простой. Старик, правда, как я узнал позже, взял колбу  к  себе  в
кабинет, но наверняка не для того, чтобы испытать минутное удовлетворение.
Уверен, что точно так же он позволил бы  ужалить  себя  отвратительнейшему
насекомому, если б мог из этого извлечь какой-либо научный  вывод.  Просто
он боялся потерять самообладание, но был достаточно умен, чтобы не попасть
перед нами в смешное положение.

Доктор,  который  уже  пришел  в  себя  после  странного   эксперимента
(действие жидкости прекратилось очень быстро), встал и положил перед собой
стопку листов бумаги различной формы и  размеров.  У  него  была  привычка
записывать результаты работы на манжетах, обрывках  газет,  использованных
промокашках,  старых  счетах,  пренебрегая  блокнотами  веленевой  бумаги,
которых всегда полно в лабораториях.  Он  признался  мне,  что  белизна  и
непорушенность бумаги приводит к  разжижению  мозгового  вещества,  а  тем
самым ослабляет интеллект.

"--*  Простите,  к  сожалению,  я  не  могу  ни  удивить  вас  чем-то,  ни
продемонстрировать нечто волшебное. --- Он был явно  зол  на  Финка  за  его
эксперимент. --- Мое положение гораздо серьезнее, чем у предыдущего оратора,
и еще по двум причинам. Во-первых, гораздо легче вырвать тайну у  мертвого
вещества, нежели у живого,  а  во-вторых,  машина  уже  больше  недели  не
соприкасается с какими-либо внешними субстанциями, кроме ядовитого или, по
меньшей мере, вредного для ее живой части земного воздуха.  Возможно,  вам
покажется это преувеличенным очеловечиванием с моей стороны, но если перед
нами живая субстанция, а у меня есть все основания так считать, то ей  для
поддержания основного обмена веществ необходимо пополнять убыль,  принимая
химические компоненты извне. Это единственная возможность.

"--* Ошибаетесь, доктор, --- прервал  Джедевани.  ---  Есть  и  еще  одна.  Не
исключено, что жизненная энергия попадать к живой части машины снаружи без
химических, как это сказать, соединений. Например, излучение  либо  волны,
выделяемые нейтронами, напрямую отдавать свою кинетическую энергию  атомам
этой живой существо...

Доктор наклонил голову.

"--* Возможно, вы правы. Возможно, такое "питание" энергией  имеет  место.
Во  всяком  случае,  в  половине  одиннадцатого  плазма  начала  проявлять
вызывающие беспокойство признаки как бы "старения",  "замирания  функций".
Первым симптомом было ослабление токов...

"--* Говорите проще  и  короче,  что  произошло  с  плазмой?  ---  Профессор
раздраженно глядел поверх очков на доктора, который, казалось, обиделся.

"--* Я еще не кончил. Мне результаты даются не так легко,  как  это  можно
сделать, развинчивая шестеренки да болтики.

"--* Что еще такое? Уж не думаете ли вы учинить ссору?

Профессор покраснел как пион.

Доктор сдержался.

"--* Возможно, всему виной чертова жидкость... Во всяком случае, я здорово
струхнул, когда световая пульсация упала почти до нуля, токи тоже  страшно
ослабли, я давал кислород, двуокись, но результат был почти  незаметен.  В
одиннадцать пятьдесят пять наступила стадия умирания, тогда я  в  отчаянии
взял и... задержал...

"--* Что вы, черт вас возьми, сделали?

"--* Ввел через отверстие в груше один  кубик  адреналина.  Результат  был
потрясающий! Все функции восстановились, и когда  я  через  четверть  часа
ушел...

Профессор встал.

"--* Но это могла быть временная ремиссия. С момента отключения  груши  от
конуса минуло  двенадцать  часов.  Если  плазма  так  чувствительна  и  не
получает необходимой для нее энергии...

"--* Быстрее, господа! В лабораторию! Финк, бутылочку с жидкостью, быстро!

Мы помчались к двери, словно ученики, подгоняемые громким голосом.

"--* Осторожнее, мой пиджак! --- крикнул Джедевани, сопевший рядом  со  мной
и, казалось, пребывавший в жутком настроении. --- Я знал, что так просто это
не кончится, что оно выкинет еще не один фортель.

"--* Кто "оно"?

"--* Ну, человек с Марса.

В лаборатории было тихо. Я первым подбежал к  гальванометру,  глянул  в
оконце --- темное!

"--* Черт, господин доктор, этого я  от  вас  ну  никак  не  ожидал...  Он
уничтожен, он, похоже, мертв... Мы там сидим и  болтаем,  а  здесь  плазма
погибает. Ареантроп погибает, неужели вам не ясно?  А  вы  препираетесь  с
инженером, отстаивая свои никчемные мелкие амбиции.

Доктор, казалось, готов был провалиться сквозь землю.

"--* Я... даю слово, когда я уходил,  состояние  плазмы  было  прекрасное,
ведь я же всю ночь здесь просидел...

"--* Тише! Инженер --- жидкость!

Профессор Уиддлтон был, как  говорится,  в  своей  стихии.  Молниеносно
вырвал у доктора шприц, набрал  несколько  капель  загадочной  жидкости  и
сунул иглу в отверстие панциря. Шли секунды  ---  поршень  шприца  дошел  до
конца. Мы затаили дыхание.

Неожиданно в оконце еле  заметно  блеснуло,  одновременно  шевельнулась
стрелка  гальванометра.  Блеск  повторился  ---  по  внутренности   "пузыря"
разлилось слабое световое пульсирование.

"--* Слава Богу, --- просиял профессор.  ---  Ничего  не  поделаешь,  придется
начинать сборку. --- Он окинул нас взглядом. --- Похоже, вы  намерены  создать
здесь этакую модель ученого мира  в  миниатюре.  Сколько  ученых,  столько
точек зрения, теорий и так далее. Играем, значит, а ареантроп тем временем
подыхает. Этому надо положить конец, говорю  вам.  Подобные  академические
дискуссии  и  доклады  на  полдня  необходимо  прекратить.  Здесь  надобно
действовать. Господин инженер, извольте-ка совместно  с  доктором  собрать
машину, чтобы она могла принять необходимую ей для жизни энергию.  Болтать
будем потом.

"--* Ну, ну, чтобы оно,  стало  быть,  снова  могло  психовать,  ---  бросил
Джедевани.

"--* А вы что, сюда загорать приехали? --- С профессором шутки были плохи.

"--* Профессор, --- отважился  я  заметить,  ---  сможет  ли  инженер  на  сто
процентов поручиться, что овладеет излучением и  сумеет  его  выключить  в
любой момент?

"--* Думаю, да, разве что машина перестанет быть машиной,  ---  бросил  Финк
после недолгого раздумья.

"--* Что это значит?

"--* А то значит, что если он мыслит  и,  как  я  думаю,  плазма  является
реальным  конструктором  всей  машины,  то  он  сумеет   сообразить,   что
произошло,  а  снова  овладев  своим  аппаратом,  сможет  реконструировать
устройство атомной трансформации. Вот тогда-то я не отвечаю ни за что.

"--* Похоже, вы умываете руки? --- медленно проговорил профессор.

"--* Нет. Просто не дам никакой гарантии, но монтаж начну немедленно.

"--* Понятно. Убедительно прошу --- приступайте.

Инженер с помощью доктора извлек грушу из штативов, осторожно взял ее в
руки и вышел. Мы еще некоторое время оставались в лаборатории.

"--* Что будем делать, профессор? --- спросил я.

"--* Поместим  конус  в  камеру  с  марсианской  атмосферой  и  попытаемся
втолковать ему, что мы не враги, то есть станем воздействовать на него уже
не газовыми снарядами,  а  мыслью,  ---  профессор  говорил  медленно,  явно
раздумывая.

"--* А не возвращаемся ли мы к исходному пункту? --- заметил я. ---  Сведения,
которыми мы располагаем о его конструкции, весьма туманны, не говоря уж  о
самой плазме, об этом "центральном мозге"...

"--* Мозге?  Прекрасное  определение,  ---  профессор,  казалось,  пришел  в
восторг. --- Есть идея, --- воскликнул он и выбежал  из  лаборатории.  Фрэйзер
пошел следом.

Маленький синьор  Джедевани  остался  со  мной.  Тщательно  протер  лоб
платком, оглянулся и сказал:

"--* Я чувствовал, что это кончит себя скверно.  Четыре  года  я  стоял  у
циклотрона с тремя миллионами вольт, но это была игра. Что тут творит, что
тут творит! --- и с этими словами отчаяния он вышел.

Я пошел наверх, раздумывая над словами профессора. Неужели  он  наконец
нашел ключ к взаимопониманию с марсианином? Поверить в это было трудно.  В
малом  монтажном  зале,  куда  я  заглянул  по  пути,  стоял   Линдсей   с
профессором.  Профессор  быстро  устанавливал  какие-то  аппараты,   среди
которых я узнал большой динатронный усилитель и каскад усилителей  высокой
частоты.

В центре зала стоял большой стул, что-то вроде электрического --- так мне
показалось в первый момент, поскольку на верхней части спинки  размещалось
нечто вроде металлического чепчика, к которому были подведены кабели.

"--* Включайте поскорее аккумуляторы, --- сказал профессор, --- и давайте сюда
катодный осциллограф, на площадку. Позвоните Бэрку, пусть  поможет,  ---  и,
обращаясь ко мне, заметил, --- знаете, что  я  надумал?  Это  фантастический
проект, но что нам поможет еще, как не фантазия? Понимаете, я хочу уловить
электрические токи, которые вырабатывает кора головного  мозга  одного  из
нас,  усилить  их  в  несколько  миллионов  раз  и  послать  на  электроды
рентгеновской трубки. В  зависимости  от  напряжения,  сила  рентгеновских
лучей будет изменяться. Этим, регулируемым токами нашего мозга, излучением
я стану воздействовать на ареантропа.

Вошел Бэрк. Они с инженером  принялись  монтировать  части  аппаратуры.
Профессор велел мне сесть на  стул,  наложил  на  голову  медный  обруч  и
подключил несколько контактов.

Послышался низкий гул. Профессор возился с аппаратурой,  не  переставая
говорить:

"--* Вы понимаете, что я имею в виду? Наша речь, наши жесты  и  так  далее
непонятны пришельцу с Марса.  Но  быть  может,  характер  его  психических
процессов в самом центре, в  его  мозгу,  более  близок  нам.  Я  намерен,
отбросив окольные пути, воздействовать биотоками наших мозгов на его мозг.

Тем временем лампы  усилителей  накалились  до  бледно-розового  цвета.
Гудение усилилось. Я почувствовал, как у меня на голове  стягивают  ремнем
металлическую каску.

"--* Не волнуйтесь, сидите спокойно, --- дошел до меня голос  профессора.  ---
Ничего не случится, глядите на экран.

Большая,  похожая  на  стеклянный  цилиндр  с  конически  расширяющимся
основанием труба катодного осциллографа лежала на двух эбонитовых стойках.
Я  увидел,  как  на  ее  бледно-желтоватой   флюоресцирующей   поверхности
появились медленно извивающиеся светлые линии.

"--* Это биотоки вашего мозга. Попробуйте мысленно перемножить тридцать на
восемнадцать.

Теперь световые зигзаги на экране заструились быстрее.

"--* Прекрасно, аппарат действует идеально.

Гудение резко оборвалось, я почувствовал, что инженер ослабил ремень  и
снял у меня с головы "чепец".

"--* Пожалуйста, спуститесь,  а  мы  подадим  через  вентиляционную  шахту
кабель осциллографа, там вы его примете из выходного отверстия и дождетесь
меня, --- сказал Линдсей.

Я по лестнице сбежал на первый этаж. В  большом  монтажном  зале  гудел
электромотор, небольшой тельфер  поднимал  безжизненное  тело  марсианской
машины с ее ложа и переносил в центр зала. Инженер шел под ней  и  подавал
знаки доктору, который управлял  перемещением  с  пульта.  Я  нашел  выход
вентиляционной шахты и вскоре увидел, как оттуда высовывается черный конец
кабеля. Я легонько потянул за него и стал  ждать.  Через  минуту  появился
Линдсей  с  большой  рентгеновской  трубкой.  Укрепив  кабель  на  стенном
изоляторе, он начал устанавливать необходимые приборы.

"--* Газовые гранаты  готовы,  ---  сказал  он  и,  обращаясь  к  Джедевани,
стоявшему рядом, добавил: --- Не думайте, что мне жизнь не мила... А  теперь
так: дадим ему ток на десять секунд и будем повторять это до тех пор, пока
он не дрогнет. Тогда выключим ожививший его ток и подвергнем его мозг  или
рецептор, а еще лучше --- все сразу действию рентгеновских  лучей.  Один  из
нас будет сидеть наверху и  медленно,  спокойно  думать,  мыслить,  но  не
словами --- это ничего не даст, --- а образами. Картинки, которые надо вызвать
в воображении, я уже набросал.  Потом  опять  дадим  ему  "бодрящий"  ток,
посмотрим, реагирует ли он, и так будем повторять до получения результата.

"--* И как долго? --- спросил Джедевани.

"--* До утра, если потребуется, --- нервно бросил профессор.

Я взглянул на часы --- было семь вечера.

"--* Макмур, вы ---  человек  уравновешенный,  здравомыслящий,  ---  профессор
быстро взглянул на итальянца, но тот и  не  думал  обижаться.  ---  Пойдемте
наверх. На листке, что лежит у меня на столе, найдете картинки, о  которых
сказал инженер. Мыслить надо медленно, каждый образ представлять  себе  не
меньше двадцати секунд. Начнете после  красного  сигнала,  по  зеленому  ---
прекратите. Если что-нибудь получится,  вы  станете  первым  человеком  на
Земле, переговорившим с обитателем Марса. Ну, дай Бог счастья и вам и нам.
Жаль, что одного инженера у нас черти взяли, а он  бы  пригодился.  Ну  да
ничего, Фрэйзер тоже хороший физик. Господин Фрэйзер, пойдете с Макмуром и
подключите все, что необходимо для записи и передачи биотоков мозга  через
усилители. Не отнимайте телефонной трубки от уха: если  я  позвоню,  то  в
зависимости от указаний  будете  либо  увеличивать,  либо  уменьшать  ток,
подаваемый отсюда, снизу.

Кончив, профессор обратился к Линдсею.

Я оправился с Фрэйзером наверх. В малом монтажном  зале  сел  на  стул,
Фрэйзер надел мне на голову металлический колпак и дал в руки  листок,  на
котором было написано несколько фраз, и пачку пронумерованных фотографий.

Вначале мне предстояло внимательно рассматривать карту Марса,  "активно
смотреть, а не просто пялиться, как баран  на  новые  ворота",  ---  написал
профессор под фотографией своим ужасно  неразборчивым  почерком.  Потом  ---
фотографию Земли. Затем --- карту Америки, далее  ---  района,  в  котором  мы
находились, и при этом я должен был "испытывать положительные  эмоции  без
примеси страха или ненависти".

Хорош старичок.  Я  уже  чувствовал  раздражение  от  его  наставлений.
Следующая фотография изображала всех  нас  (кроме  меня)  на  платформе  у
телескопа. Здесь  опять  следовало  подумать  о  Марсе,  "но  образно,  не
словами", --- записал на полях профессор. Наконец была  фотография  снаряда,
снимки  района  его  падения  и   самого   марсианина,   при   интенсивном
рассмотрении которых, выполняя указание профессора, я должен был пребывать
в хорошем настроении и дружеском расположении к нашему  гостю.  Признаюсь,
первое, что  мне  пришло  в  голову  после  ознакомления  с  профессорской
писаниной, было пожелание, чтоб удар  хватил  это  механическое  чудовище,
которое уже  прикончило  нескольких  человек,  но  я  взял  себя  в  руки.
Неожиданно звякнул телефон --- звонил профессор. Он сообщал наверх, что  они
начинают  раздражать  ареантропа  напряжением  в  три  миллиона  вольт,  и
предупреждал нас, чтобы мы были готовы мгновенно включиться по их сигналу.
Я выпрямился на стуле и ждал. Спустя секунду почувствовал, как пол  слегка
завибрировал.

"--* Электризуют нашего марсианина, --- бросил Фрэйзер. --- Электризуют!

Изумительная штука --- три миллиона вольт!  По  мне  тоже  прошла  дрожь.
Оп-ля! --- надо быть в хорошем настроении. Я принялся думать о Шотландии,  о
лесистых горах, о добрых старых временах, о  миллионных  тиражах  газет  с
моими сенсационными репортажами.

Мысль мою прервал звонок.

"--* Что? Как? --- кричал в трубку Фрэйзер. --- Громче! Ничего не слышу!

Даже я слышал на расстоянии в несколько шагов, что трубка аж хрипит  от
гула, --- и ничего странного: все генераторы работали в полную  силу,  чтобы
дать нужные миллионы.

Фрэйзер бросил трубку.

"--* После третьего включения он  дрогнул,  пошевелился.  Профессор  велел
сказать, чтобы вы подготовились.

Тем временем Финк опустил марсианина на пол, установил  на  площадке  в
нормальном положении и теперь был занят тем, что размещал проволочки груши
с плазмой  в  их  гнездах.  Потом  закрыл  верхнее  отверстие  колпаком  и
попытался расшевелить сердце --- или гироскоп  ---  через  просверленное  нами
отверстие.

"--* Осторожней, инженер! --- крикнул я.

"--* Не бойтесь, у него сейчас нет устройства  для  трансформации  атомной
энергии, и пока мы не дадим ему из нашей сети  трех  миллионов  вольт,  он
безопасен, как пень.

Финк сунул руку в отверстие и принялся там  шевелить.  Послышалось  уже
знакомое чавканье, инженер вдруг охнул, побледнел и опустился на  дрожащий
пол.

"--* Чтоб тебя! ---  взревел  я,  бросаясь  к  Финку,  но  тут  почувствовал
сильнейший удар в руку и отлетел к стене.  Уже  бежал  Линдсей  в  красных
резиновых перчатках и оттаскивал потерявшего сознание Финка в  сторону.  Я
поднял его на руки, хоть ноги у меня дрожали, а пальцы правой руки  горели
огнем, и положил на стол. Доктор молча вынул из кармашка футляр, быстро  и
ловко  сделал  укол,  пощупал  пульс,  потом  пересчитал  глазами  ампулы,
посмотрел на меня, как бы говоря:  вот,  видишь?  ---  и  начал  массировать
инженеру область сердца.

Тем  временем  внутренний  телефон  звонил  так,  словно   вознамерился
спрыгнуть со стены, но в суматохе о нем все забыли. Только теперь  к  нему
подошел Линдсей и кратко сообщил профессору о случившемся.

"--* Может, лучше отнести  его  наверх?  ---  спросил  я  доктора,  указывая
глазами на Финка. --- Если здесь что-нибудь случится,  он  может  пострадать
больше всех.

Доктор кивнул. Я взял Финка на руки и отнес  в  мою  спальню.  По  пути
столкнулся с профессором. Он ничего не сказал,  только  взглянул  на  меня
так, что у меня мурашки прошли по спине, и побежал к лифту.

Я  посидел  несколько  минут  около  Финка,  но,  видя,  что  он  дышит
равномерно, решил  спуститься  вниз  и  вышел,  прикрыв  его  как  следует
одеялом.

В монтажном зале во всю командовал Уиддлтон.

Его непрекращающиеся окрики и указания начали  меня  раздражать.  Чтобы
сохранить хорошее настроение, я принялся  потихоньку  повторять  известный
английский стишок: "У нашей Мэри есть баран, собаки он верней. В грозу,  и
в бурю, и в туман баран бредет за ней".  Когда  стишок  кончился,  я  стал
проговаривать его с конца, но тут раздался резкий тройной звонок и красная
лампочка загорелась у меня перед лицом.

"--* Внимание! --- прозвучал  громкий  голос  фрэйзера.  ---  Макмур,  думайте
медленно и спокойно, включаю ток!

Я  начал  копаться  в  глубинах  памяти,  интенсивно  выискивая  в  ней
телескопные карты Марса, вспоминая каналы, пустыни. За  спиной  послышался
звонок --- я не обращал  на  него  внимания.  Одну  за  другой  просматривал
фотографии, то прикрывал глаза, воспроизводя их в  памяти  по  возможности
точно, то снова  переходил  от  снимка  к  снимку.  Теперь  мне  следовало
"обрести хорошее настроение и  преисполниться  дружелюбия".  Я  представил
себе Землю с Америкой во все полушарие, связанную широкой лентой с Марсом,
а на этой ленте было написано... Стоп! Слова не нужны.

В этот момент красный свет погас. Фрэйзер выключил усилитель и подбежал
к телефону. Поскольку он, похоже, забыл обо  мне,  я  сам  освободился  от
чепца и взглянул на него. Он держал трубку около уха и ждал.

Медленно бежали секунды.

"--* Ну, что там? --- спросил я.

Фрэйзер отрицательно покрутил головой. Я встал, Фрэйзер  несколько  раз
нажал на рычаг телефона, крикнул: "Алло! Алло!" --- и продолжал ждать.

Я не выдержал.

"--* Бегу вниз! Бог знает, что там творится! --- и не успел он рта раскрыть,
как меня уже в комнате не  было.  Не  в  силах  спокойно  ждать  лифта,  я
буквально слетел на первый этаж, перескакивая через  четыре  ступеньки.  У
входа в монтажный зал меня оглушил гул моторов. Я отворил дверь --- и замер.

В  фиолетовом  свете  искусственной  молнии  я   увидел,   что   черный
приземистый конус...  жил.  Он  медленно  раскачивался  на  месте,  а  его
щупальца спокойно, неторопливо шевелились, будто проделывали гипнотические
пассы. Воздух был заполнен дьявольским ревом и треском.

Группка мужчин столпилась у распределительного щита. Высокий Линдсей  с
покрытым потом лицом, в кожаном фартуке, держал руку на рубильнике.  Около
него стоял профессор, а за ним  и  немного  позади,  синьор  Джедевани.  Я
прикрыл двери. Не знаю, заметил ли меня марсианин, во  всяком  случае,  он
выставил щупальца в стороны и несколько секунд  держал  их  горизонтально.
Потом вдруг два из них подтянул к себе,  и  тут  я  увидел  проскакивающие
между их концами искры. Затем он снова  раздвинул  щупальца  в  стороны  и
немного вверх и закрутил ими в воздухе, как бы описывая конусы. Неужели он
хотел  что-то  выразить  таким  образом?  Марсианин   неустанно   повторял
движения, словно машина. Но ведь он и есть машина, промелькнуло у  меня  в
голове.

Он то опускал щупальца, то снова поднимал их и выписывал горизонтальные
круги, причем иногда так быстро, что становились видны как бы два размытых
цилиндра. Профессор вдруг отделился от группы и вышел в боковую  дверь.  Я
стоял,  словно  прикованный  к  полу,  глядя  широко  раскрытыми  глазами.
Марсианин повторял свои движения, все убыстряя темп.  Снова  соединил  два
щупальца, развел и пропустил через  просвет  между  ними  несколько  слабо
потрескивающих искр.

В этот момент появился профессор. Маленький,  темный,  сутуловатый,  он
быстро шел, неся что-то в вытянутых руках. И вышел прямо на середину  зала
"--* неужто собрался покончить жизнь  самоубийством?  Я  прыгнул,  чтобы  его
задержать. Но он наклонился и покатил по полу два металлических  цилиндра.
Один из них катился, подскакивая на неровностях пола. Я  их  узнал  ---  это
были цилиндры из снаряда. Я видел их вчера, в одном был  уже  "записанный"
порошок, в другом --- механизм для фиксирования мыслей.

Теперь профессор стоял  в  пяти  шагах  от  черного  монстра.  Щупальца
перестали кружиться, опустились, и оба цилиндра прилипли  к  ним,  как  бы
притянутые магнитом. Конус замер --- щупальца поднялись, и в  верхней  части
колпака открылся клапан, а  может,  образовалось  отверстие  в  монолитном
металле --- не знаю, но оба цилиндра вдруг исчезли, да  так  быстро,  что  я
только и успел моргнуть. Не прошло и  секунды,  как  они  опять  оказались
снаружи, были опущены на пол и катились к профессору.

Выглядело это прямо-таки забавно: группка сбившихся  у  стены  людей  и
металлический конус, который, казалось, играет в кегли.

По знаку профессора Линдсей выключил ток. В голове у меня  зашумело  от
неожиданно наступившей  тишины.  Профессор  жадно  схватил  оба  цилиндра,
подбежал к столу, на котором лежали листы  бумаги,  и  начал  раскручивать
первый из них. Мелкий порошок высыпался на бумагу. Несколько движений ---  и
на белой поверхности возникла четкая карта Марса,  а  рядом  ---  Земля.  Их
связывала широкая лента.

Я раскрыл рот.

"--* Но ведь именно это я и представил в мыслях! --- вылетело у меня.  Никто
не ответил. Порошок под рисунком собрался в несколько малюсеньких значков,
похожих на ноты.  Профессор  уже  раскручивал  второй  цилиндр  и  высыпал
содержимое на другой лист --- глаза у нас прямо-таки вылезали из орбит.

На белом  листе  возник  треугольник,  окруженный  венком  таинственных
знаков, ---  они  больше  походили  на  цифры,  чем  на  буквы,  ---  а  рядом
вырисовывались нечеткие контуры. Я присмотрелся внимательнее. Ну,  конечно
"--*   наша   лаборатория,   начерченная   очень   странным   образом,    без
пространственной перспективы, совершенно плоско, как бы  в  геометрической
проекции. В центре --- два столба  и  шары  разрядника,  но  тонкий  зигзаг,
обозначавший искру, был перечеркнут параллельными штрихами.

"--* Не означает ли это, что он не хочет, чтобы мы его щекотали  током?  ---
первым прервал тишину инженер, всматриваясь в рисунок.

"--*  Мне  кажется,  он  требует  вернуть  ему  аппаратуру  для   атомного
преобразования и, добавлю, он чрезвычайно любезен. Не уверен, что я на его
месте вел бы себя так сдержанно после подобной  вивисекции...  При  его-то
возможностях. --- Профессор просто излучал восторг. Каждая морщинка  на  его
давно не бритом  лице  источала  удовлетворение,  даже  искорки  в  очках,
казалось, весело подмигивают. Он похлопал нас по плечам, установил  штатив
подготовленного  заранее  фотоаппарата  и  запечатлел  рисунки,  а   затем
осторожно ссыпал порошок в цилиндрики. Мы дышали тяжело, как после долгого
бега. --- Думаю, на сегодня довольно, --- сказал профессор. --- Уже одиннадцатый
час, а мы не спали почти две ночи.

"--* Хорошо, а что делать с ним? --- спросил я.

"--* И верно, как быть с ареантропом? --- сказал профессор. --- Скверное дело,
он ведь не лабораторная морская свинка.

"--* Думаете, он обидится, если его на ночь посадить в стальную камеру?  ---
скептически бросил доктор.

"--* Вы-то уж наверняка б не обиделись, --- ответил профессор. --- Ну,  начало
положено, во всяком случае, он уже знает, что  мы  не  абсолютные  дикари.
Значит, Бог с ним, пусть сидит здесь.

"--* Я бы все же вынул из него  центральную  грушу,  ---  сказал  доктор,  а
осторожный синьор Джедевани сразу же поддержал его.

"--* Ну, конечно, чтобы спать спокойно, да? ---  ехидно  спросил  старик.  ---
Ничего не выйдет, милейший синьор, разве что вы просидите здесь всю  ночь,
наблюдая, пульсирует ли в оконце нормальная жизнь, иначе придется  сделать
ему укол и вложить "пузырь" обратно в конус.

"--* Простите, --- вмешался я, --- но если нет тока, то он лишен энергии: ведь
инженер Финк полностью демонтировал его собственную энергетическую атомную
установку.

"--* Верно, демонтировал, но тогда скажите, почему Финк сейчас не здесь, с
нами, а валяется наверху, без чувств?  ---  иронично  спросил  профессор.  ---
Откуда вы знаете, что именно  он  демонтировал,  а  чего  нет?  Ясно,  что
загадочная для нас жидкость нашему гостю  невероятно  нужна,  может  быть,
абсолютно необходима. Поэтому пусть она в нем сидит, а мы пойдем  баиньки,
"--* говоря это, он подошел  к  мраморной  доске  распределительного  щита  и
принялся выключать дуговые лампы.

"--* Но он может неожиданно, ночью... --- начал, заикаясь, синьор Джедевани.

"--* Так шепните ему на ушко, чтобы вел себя прилично, --- сказал неумолимый
профессор и продолжал выключать лампы. Нам не оставалось ничего иного, как
покинуть зал. Когда мы собрались у лифта, профессор сказал: ---  Неплохо  бы
поужинать, господа, а?

"--* Верно, --- в один голос отозвались мы.

"--* Ну, так сейчас организуем лукуллов пир, только я на минуточку загляну
к бедняге Финку. И позовите, пожалуйста, Фрэйзера. Пусть  Бэрк  приготовит
все в столовой.

Кабина лифта остановилась,  мы  вышли  в  коридор.  Инженер  Финк  спал
горячечным беспокойным  сном.  Доктор  проверил  его  пульс,  сделал  укол
успокоительного и велел всем выйти.

В столовой горели свечи --- идея профессора. Их  оранжево-желтый,  мягкий
неверный свет отлично успокаивал нервы.

На стол подавал Бэрк --- впрочем, все блюда были из банок, подогретые  на
электроплитке,  так  что  отсутствие  повара  не  ощущалось.  После  ужина
профессор принялся катать хлебные шарики по столу.

"--* Господа, --- сказал он, --- наконец-то мы сделали шаг  вперед.  Возможно,
марсианин  отреагировал  на  наши   рентгеновские   депеши   подобным   же
излучением. Я предпочитал этого не записывать, хотя мог  бы  установить  в
зале  несколько  чувствительных  счетчиков   Гейгера   с   автоматическими
регистраторами. Просто запись в любом случае оказалась бы  нам  непонятна:
ведь мы не умеем прочесть даже сигналы собственного  электроэнцефалографа,
что же говорить об электрическом языке марсиан.  Опыт  удался,  мы  сможем
общаться изображениями, картинками, постараемся научить  его  каким-нибудь
знакам, может быть, рисуночному алфавиту, а  возможно,  и  он  нас  научит
чему-нибудь новому,  чтобы  облегчить  контакт.  Во  всяком  случае,  хочу
надеяться, что самое худшее позади. Незачем ухмыляться, доктор, так  хитро
и иронически. Дай Бог, чтобы это предсказание сбылось. Может быть, у  вас,
коллеги, есть какие-нибудь предложения?

"--* По-моему, --- сказал я, --- ясно одно: его живая субстанция близка нашему
мозгу,  если  судить  по  функционированию,  а  не  строению.  Видимо,  он
воспринимает только суть,  истоки  явлений,  а  не  их  побочные,  внешние
проявления:  голос,  свет  для  него  не  так  важны,  как  энергетические
изменения характера излучения,  перепад  потенциалов.  С  другой  стороны,
количество "картин", которые можно ему таким способом передать, достаточно
ограничено.  И  мне  кажется,  для  передачи  пригодны  только  конкретные
понятия,  но  не  хорошие,  добрые,  дружеские  настроения,   о   которых,
инструктируя меня, писал профессор...

\bigskip{}



\bigskip{}

5

\bigskip{}


В ту ночь я спал мертвецким сном. Смутно помню, как повалился на диван,
потому что на кровати тихо постанывал инженер, и погрузился в сон,  словно
в мрачную бездну. Разбудил меня мощный гул. Я  вскочил  и,  полупьяный  от
сна, бросился к окну. Было тихо,  только  над  прудом  поднималось  легкое
облачко  утреннего  тумана.  Неужто  гул  мне  просто  приснился?   Вполне
возможно. Я быстро оделся и, мельком взглянув на все  еще  спящего  Финка,
выбежал в коридор. Тихо и пусто. Теперь я ощутил, что пол  слегка  дрожит.
Что такое?  Генераторы  включены?  Я  почувствовал  обиду  на  коллег,  не
разбудивших  меня  своевременно.  Оказалось,   что   лифт   не   работает,
по-видимому, использовали всю мощность  сети.  Я  сбежал  на  первый  этаж
немного обеспокоенный, слыша и  чувствуя  усиливающуюся  вибрацию  стен  и
воздуха.  В  большом  монтажном  зале  все  собрались  у  стены   напротив
маленького черного конуса, который мягко переваливался с боку  на  бок  и,
казалось, беседовал с ними --- это было и страшно и смешно.

Профессор махнул мне рукой.

"--* Что слышно? Что-то новенькое? --- крикнул я ему в ухо, потому  что  при
стоявшем в зале диком грохоте нормально общаться было невозможно.

"--* Все прекрасно, мы как раз беседуем с ареантропом, --- крикнул  в  ответ
Фрэйзер.

Они действительно как-то беседовали с  ним,  и  странная  же  была  эта
беседа! С помощью маленького проектора они высвечивали  на  экране  то  ли
какие-то модели,  то  ли  эскизы,  которые  профессор  помещал  на  столик
проектора. Все происходило так быстро,  что  я  не  успевал  рассматривать
отдельные картинки, но профессор этого вроде бы не замечал. Все  выглядело
немного глуповато. Мне казалось, что я как бы выброшен за борт, как бы  не
нужен, ---  стою  с  заложенными  за  спину  руками,  ничего  не  понимая  в
происходящем...

Неожиданно  наступила  тишина,   два   металлических   цилиндра   снова
покатились  по  полу,   и   профессор   принялся   высыпать   порошок   на
приготовленные заранее листы бумаги.

Это были совершенно бессмысленные клубки спутанных линий. Рядом с  ними
"--*  опять  же  странные  значки,  напоминающие  ноты.  Что-то  вроде  схемы
планетарной системы,  длинные  концентрические  эллипсы,  густо  усыпанные
таинственными  значками.  Все  молчали,  вглядываясь  в  эти  изображения.
Наконец заговорил Джедевани:

"--* Я думаю, он скверно себя чувствует... Ведь  это  все  бессмысленно  ---
может, он болен?

Профессор посмотрел на итальянца, словно хотел сказать: "Сам ты болен",
"--* но смолчал.

"--* Нас все еще разделяет пропасть  непонимания,  ---  сказал  он  и  велел
включить ток. Загудели, завыли электромоторы,  начинающие  работу  басовым
гулом, от которого дрожали стальные крепления; гул  этот  отозвался  более
высокими аккордами в изломах потолка и наконец  перешел  в  пронзительный,
высокий звук. Конус  снова  ожил,  закачался  и  вдруг...  пошел.  Он  шел
неуклюже, как бы неуверенно, но тут его остановили кабели,  тянувшиеся  от
генераторов. Он был словно на привязи. Сам он того хотел или просто не мог
освободиться? Мне невольно почудилось, что я встал на его  сторону,  хочу,
чтобы он вышел на волю, за пределы назначенных нами границ. Не  знаю,  как
выразить эти странные мысли,  но  мне  казалось,  что  нас  с  ним  что-то
связывает. С кем? С этим творением из безумного сна, металлическим конусом
с желеобразной светящейся массой, выполняющей функции мозга?

Теперь  профессор  высветил  на  экране  параллельные  ряды  уравнений,
описывающих геометрические законы.  Цифры  казались  ему  наиболее  верным
языком, которым можно было соединить края разделяющих нас  пропастей.  Так
ли?

Неожиданно я заметил нечто странное. Как известно, у  конуса  было  три
щупальца.  В  то  время  как  два   лежали   спокойно,   изредка   немного
приподнимаясь, третье, заднее, усиленно колотило по бетону, словно  то  ли
что-то  выколачивало  из  него,  то  ли  лепило.  Металлический   цилиндр,
образовывавший  его  тупое  окончание,  проделывал  явно  целенаправленные
быстрые движения. Мне показалось, что бетон краснеет,  но  это,  вероятно,
было невозможно. Никто не мог этого видеть, кроме меня, потому  что  конус
стоял перед нами, загораживая собою тыльный змеевик.

Ну, конечно! Вот  он  поднимает  третье  щупальце,  на  конце  которого
чернеет что-то небольшое, вроде бы зубчатки.

"--* Профессор! --- рявкнул я. --- Выключайте ток! Ради Бога, выключайте ток!

Я  понял,  что  собирается  сделать  ареантроп.  Какое  же  дьявольское
создание! Все его поведение было  подвохом,  сатанинской  дипломатией.  Он
использовал наш ток не для установления контакта, а чтобы освободиться  от
нас. Он старался восстановить изъятые нами детали из любого материала ---  а
раз он умел уничтожать, значит, мог и создавать.

"--* Выключайте ток! --- крикнул я снова.

Теперь все видели, как  щупальце  поднимается  и  как  в  нижней  части
основания конуса образуется неправильное  отверстие  с  размытыми  краями.
Отверстие поглотило  зубчатый  механизм  и  занесшее  его  туда  щупальце,
которое  что-то  там  делало.  Осоловевший  Линдсей  подскочил  к  щиту  и
поскользнулся. Вместо того чтобы выключить  ток,  он  передвинул  рычаг  в
противоположную сторону, туда, где были изображены красные молнии и стояла
надпись:

"ПРЕДЕЛЬНОЕ НАПРЯЖЕНИЕ --- ОПАСНО!"

Громыхнуло. Туча пыли поглотила  вибрирующий  конус,  провода  сыпанули
голубыми искрами, с грохотом выбило главные предохранители  перегрузки,  и
все утихло. Но кто может описать мое изумление и ужас, когда я увидел, что
конус по-прежнему жив, двигается, стряхнул с  себя  два  удерживающих  его
кабеля, словно два пучка соломы,  и  одним  прикосновением  черного  конца
щупальца  "заварил"  просверленное  напротив  его   механического   сердца
отверстие.

Конус, казалось, размышлял. Сценка была необычная: неожиданно  замершие
моторы, неподвижные с широко раскрытыми глазами люди и этот смешной конус,
который раскачивался и двигался, размахивая щупальцами, будто не знал, что
делать со вновь обретенной свободой, с чего начать. Я  чувствовал,  как  у
меня опять лихорадочно заработал мозг. Что делать? Что делать? Я вспомнил,
что в углу зала стоят газовые гранаты и гранатомет. Теперь я уже был не на
стороне марсианина, о, нет! Теперь на карту была поставлена жизнь! Но  это
порождение ада  двигалось  вперед,  а  кто  отважится  попасть  в  пределы
досягаемости трехметровых змеевиков? Вот он протягивает их вперед и  ---  о,
ужас!  ---  пирамидка  гранат   разваливается,   превращается   в   какой-то
вращающийся смерч и исчезает, словно ее  никогда  и  не  было!  Еще  видны
остатки пыли на полу в том месте, где они лежали, следы хвостовиков... Вот
и все... А громыхающий конус передвигается по  бетону,  выписывает  круги,
приближается к людям!

Люди пятятся, дым закрывает путь к двери. Сейчас марсианин  движется  к
Фрэйзеру, Линдсею и Джедевани. Профессор стоит  в  стороне,  у  свинцового
экрана.

"--* Беги! --- слышу я чей-то крик, и ноги сами срываются с места. ---  Он  их
отрезает от двери. Беги! Ты им не поможешь!

И тут я как бы слышу спокойный голос профессора; "Мы не просто  ученые,
мы --- земная делегация, в задачу которой входит  знакомство  с  пришельцем.
Следует ли говорить, как должны вести себя члены такой делегации?"

Я чувствую, как у меня краснеет лицо.  Я  стою  и  гляжу,  ощущая  свое
бессилие.

Конус приближается к трем мужчинам. Линдсей  стоит,  прикусив  губу,  в
лице --- ни  кровинки,  глаза  горят,  мускулы  напряжены,  я  бы  сказал  ---
кариатида, удерживающая гигантскую тяжесть.

Крик. Это Джедевани, его коснулось щупальце, приближается второе,  и  я
вижу повисшее в воздухе тело, дергающиеся ноги, слышу ужасный  вопль  ---  и
неожиданная тишина. Тишина бьет по ушам молотом. Конус идет на профессора,
а Джедевани? Ах да, он лежит у стены  плоский,  как  воздушный  шарик,  из
которого выпустили газ. Конус движется к профессору. Они стоят друг против
друга. Как странно смотрят глаза старика. Он словно бы вырос.  Что  теперь
будет? Щупальца извиваются по земле, я слышу их грохот.  Роли  поменялись:
те, кто собирался исследовать, стоят  у  стен  ---  бессильные,  дрожащие...
Дрожащие? Нет, профессор Уиддлтон скрестил  руки  на  груди  и  смотрит  в
выпуклые  сверкающие  диски  на  панцире  ареантропа.   Конус   двигается,
отдаляется от профессора --- вдруг его  щупальце  касается  старика,  и  тот
падает на землю, словно молнией  пораженный.  Конус  не  обращает  на  это
внимания.  Он  направляется  ко  мне,  постукивает   щупальцами,   колотит
сегментами своих спиралей о бетон, так что  вспыхивают  искры  и  сыплются
крошки цемента. Вот он останавливается передо мной --- бесконечная минута. Я
вижу все  как  в  тумане,  только  этот  конус,  черный,  с  извивающимися
щупальцами... И тут ударил гром.

\bigskip{}



Я куда-то летел. Вокруг выло и гудело. Словно ураган. Куда  я  лечу?  ---
удивлялся я. Тело ничего не весило,  но  это  не  казалось  мне  странным.
Неожиданно начало светлеть,  словно  кто-то  протирал  большое  запотевшее
окно, перед которым я стоял. Что происходит?

Пустыня. Раскинувшаяся  на  многие  километры,  серо-желтая,  усыпанная
обломками  камней,  усеянная  огромными  воронками  и  песчаными   холмами
пустыня, над которой висело огромное, невероятно глубокое,  светло-зеленое
небо. Какое странное небо! На нем  ---  звезды,  хотя  было  и  солнце.  Оно
показалось мне вроде бы немного поменьше, чем обычно, но грело  сильнее  и
висело высоко, над самой головой. Пустыня приближалась ко  мне.  Я  летел?
Где мои руки и ноги? Ни рук, ни ног.  Ничего.  Только  глаза  или,  может,
мозг. Но я видел! Головокружительный полет нес  меня  вниз.  Неожиданно  я
увидел: подо мной двигались как бы две огромные  стальные  башни,  глубоко
врытые в песок.

Я  опускался  все  ниже.   Башни   двигались   сами,   шевеля   тонкими
горизонтальными ручками, оканчивающимися широкими  блестящими  дисками.  И
когда  эти  диски  на  гибких  "руках"  приближались   к   грунту,   песок
вздыбливался, лопался, словно под ним разверзалась пустота, и  исчезал  на
глазах. Его  остатки  разметывал  гудящий  горячий  пустынный  ветер.  Так
рождался гигантской ширины канал, уходивший  в  бесконечность.  Зачем  был
нужен этот канал? Кому? Как работают эти машины? Но ведь такое невозможно,
говорил я себе. Куда же девается выкопанный песок? Не исчезает же?  И  кто
управляет машинами? Они работают медленно и ритмично, при каждом  движении
"рук" проделывая правильной формы выемку  в  грунте  шириной  в  несколько
сотен метров, даже в километр. Дико ревел ветер.

Теперь  воздух  подо  мною   начал   клубиться,   сгущаться,   темнеть,
уплотняться... Боже! Это был черный  металлический  конус  со  щупальцами,
который медленно опускался, вращаясь при этом,  словно  лист,  падающий  с
дерева. Он упал, маленький, рядом с машинами  и  заполз  под  них.  Спустя
минуту вынырнул, поднял щупальца. Рядом с ним возник, казалось из  ничего,
и начал увеличиваться  в  размерах  какой-то  стекловидный  контур.  Перед
глазами у меня закружилось. Черный конус исчез в песчаном вихре.  Куда  он
подевался? Я еще больше напряг зрение и тут вдруг обнаружил,  что  касаюсь
грунта, продолжаю опускаться: я погружался в песок. Хотел  кричать,  звать
на помощь, но меня охватила тьма.  Я  свободно  дышал,  но  где  были  мои
легкие? Тело? Слышал ли я? Да, слышал  ритмичный,  далекий,  очень  глухой
гул: боммм --- и пауза, и снова --- боммм. Зажегся  свет.  Нет,  неправда,  не
было никакого света. Как бы это лучше сказать? Не было света, но я  видел.
Собственно, я не видел  освещенных  предметов,  но  знал,  что  они  есть.
Ощущение чьего-то взгляда, вперившегося в спину, но  в  тысячи  раз  более
сильное.  Это  вызывало  страх.  Я  чувствовал,  что  вокруг  меня   масса
предметов. Я не видел их, но знал о них. Словно в неприятном  сне,  словно
не мог вспомнить,  не  понимал  их  назначения,  смысла.  Появлялись  залы
эллипсоидной формы, огромные, погруженные во тьму, в которых передвигались
ряды конусов. Все эти конусы, одинаково держащие щупальца в  виде  петель,
шли бесконечной чередой, шли и шли в одном направлении. И я шел  туда  же.
Проходил мимо каких-то аппаратов, которые --- я это чувствовал  ---  находятся
везде, сразу со всех сторон.

Огромные машины, но без  каких-либо  движущихся  частей,  состоящие  из
одних только слившихся мех собой фрагментов тел вращения, и  еще  какие-то
маленькие тени да черные точки, ползающие по  выпуклым  поверхностям.  Гул
нарастал. И ударил свет.  Подземелья  Марса?  Галереи,  эллипсоидные  залы
соединялись во все более крупные и широкие, уже тонущие в свете объемы, по
которым непрерывно двигались шеренги конусов. Это было необычно, но как бы
понятно. И вдруг...

Пространство. Огромное поле, именно поле, потому что я не могу  назвать
его залом, раскинувшееся на много миль. Гигантское, геометрически  вздутое
пространство. Вздымающийся над  двумя  шарами  длинный  цилиндр  с  тупым,
закругленным носом стоял наклонно на пологом возвышении. И тысячи,  тысячи
черных подвижных конусов. Теперь я увидел, что  потолок  пространства  был
образован  полусферическим  сводом   из   какого-то   однородного,   слабо
поблескивающего, видимо металлического, материала. На середине сферы зияло
отверстие, напоминающее воронку, сквозь которое светило солнце:  отверстие
выходило на поверхность планеты...

Неожиданно я почувствовал, как по машинам прошла  волна.  Поле  конусов
замерло, на  возвышении  сгустился  смерч,  и  там,  где  только  что  был
прозрачный воздух, возник ареантроп. Он приблизился к цилиндру и слился  с
его тенью. Или опять улетучился? Теперь гул звучал как бы  во  мне  самом,
настырный, громкий, требовательный. Мне казалось, что надо  считать  удары
пульса. Не знаю почему. На двадцать шестом я почувствовал  легкий  толчок.
Вздрогнул --- цилиндра не было. Осталась вздутая пустая площадка и два шара,
как бы немного уменьшившиеся в размерах. Над макушками  конусов  вздымался
дым, а может, разреженный туман... потом я уже не видел  ничего.  Какое-то
мгновение я чувствовал,  что  падаю.  В  абсолютной  темноте  ощущал,  что
нахожусь внутри вертикальной трубы, в которой опускаюсь  медленно,  словно
капля масла в алкоголе. Я оказался в зале с низкими  сводами  и  наклонным
полом. Свет померк.  Какая-то  сила  принудила  меня  поднять  глаза:  над
головой распахнулось красное отверстие. В глубине заискрилось  море  звезд
Млечного Пути. На его фоне мчался длинный  темный  снаряд  с  закругленным
носом, оставляющий позади себя хвост белого пламени. Снаряд падал. На него
надвигался вращающийся диск планеты, увеличивающейся на глазах.  Вот  диск
занял  уже  все  поле  зрения.  Чудовищно  огромная,  черно-серая  планета
раскинулась  на  половину  горизонта,  обрамленного  звездами,  и  мчалась
навстречу мутным, бездонным водоворотом.

И тут я почувствовал удар. Мне  казалось,  что  он  был  слабым,  но  я
услышал хруст и грохот. Меня охватил сильный жар, и я потерял сознание.

\bigskip{}



Я открыл глаза. Было совершенно темно, и голову ломило.

Что случилось? Я принялся шарить вокруг себя руками --- бетон. А это? Это
"--* кабель. Монтажный зал? Но как я попал сюда?

"--* Эй! Профессор!

Тишина.

"--* Господин Линдсей!

Тишина.

"--* Господин Фрэйзер! Послушайте, это я, Макмур...

Тишина. Горло болело, а голова прямо-таки раскалывалась. Что случилось?
Был эксперимент, потом конус вырвался, ах да, правда... А потом? Сон?  Это
был сон? И где все остальные? Я пошевелился. Приподнялся на колени, встал,
провел рукой по стене. Ужасная слабость.

Где я? Упал у главного подъемника? Если так,  то  где-то  тут,  в  трех
шагах, выключатели освещения. Было совершенно темно, я усиленно моргал, но
не видел даже собственной руки. Ага. Вот  выключатели,  я  повернул  их  ---
тишина и тьма. Ну, конечно же, полетели предохранители, тока нет.  Но  где
все? Я принялся шарить по карманам.  Вот  зажигалка,  скрипнул  кремень  ---
слабый огонек осветил небольшое  пространство.  Зал  был,  кажется,  пуст.
Пламя зажигалки трепетало --- огромные уродливые тени бегали по стенам.

Что за темный предмет?

Это был Джедевани. Он лежал навзничь, как упал. Я  подбежал  к  нему...
Рядом с ним, повернувшись лицами друг к другу, лежали Фрэйзер и Линдсей. У
Фрэйзера предплечье было прижато к лицу, словно он  хотел  заслониться  от
удара. Я дернул Джедевани за лацкан пиджака. Он слабо застонал.

Жив! Слава Богу. Я подошел к остальным. Сердца бились.

Но где профессор? Я  не  мог  его  найти.  Фитиль  начал  потрескивать,
догорая.

"--*  Господин  профессор!  ---  крикнул  я.  Старичок  лежал  неподвижно  у
противоположной опоры, там, где упал. Да, конечно, он ведь стоял у экрана.

Я потянул его за рукав,  перевернул  на  спину.  Зажигалка  погасла.  В
темноте еще несколько мгновений светила красная искра, и опустился мрак.

Я ощупывал шероховатое от  щетины  лицо  профессора  ---  дышит  ли?  Мне
казалось, что  он  теплый.  Сердце?  Да,  сердце  билось.  Очень  слабо  и
замедленно, но билось.

Я спотыкаясь кинулся к дверям. Ударился головой  о  какую-то  невидимую
преграду. В глазах вспыхнули искры --- дьявольщина! --- я вылетел в коридор.

Тьма. Я побежал наверх. В малом монтажном зале должен быть фонарик... А
где доктор? Может, Финк уже пришел в себя?

Я влетел в свою комнату.

"--* Господин Финк!

Тишина.

Я принялся ощупывать постель. Пусто. Финк встал?

Я ничего не понимал. Выскочил в коридор. Всюду темно.  Я  придерживался
рукой стены, почти бежал. Вот  дверь  монтажного  зала.  Я  отворил  ее  и
остолбенел.  Посередине  зала  размещалась  площадка,  на  которой   стоял
ареантроп, касаясь щупальцами большого металлического шара, блестевшего  в
свете рефлекторов. Но не это меня изумило, нет. Рядом с марсианином  стоял
Финк в пижаме, бледный, с забинтованной левой рукой и,  казалось,  помогал
ареантропу прикреплять стальную полосу к чему-то большому и  черному,  что
лежало позади площадки.

"--* Господин Финк! --- крикнул я,  но  из  глотки  вырвался  лишь  хрип.  ---
Инженер!

Он не обернулся. Медленно, со свойственной ему тщательностью  затягивал
какой-то винт.

Я испугался. Я боялся его еще больше, чем марсианина.

Неожиданно конус  вроде  бы  заметил  меня.  Странно,  когда  марсианин
обнаружил мое присутствие, Финк заметил тоже. Он глядел на меня,  но  лицо
инженера было совершенно чужим. Чужим и безразличным. Финк  наклонился  и,
не обращая на меня внимания, продолжал закручивать винт.

"--* Господин инженер! --- рявкнул я. Страх страхом, но злость меня брала. ---
Что вы делаете с этим проклятым железным болваном? Вы спятили?

Финк даже не дрогнул. Зато конус повернул в мою сторону щупальце.

Я молниеносно выскочил за дверь, хлопнул ею и сбежал вниз.

Не знаю, гнался ли он за мной. Я влетел в большой зал на первом этаже и
принялся лихорадочно собирать кусочки целлулоида, бумаги, зажег все это от
зажигалки и в слабом свете "костра" нашел аккумулятор. Он дал мне ток  для
запасного рефлектора. Наконец-то  у  меня  был  свет.  Я  занялся  людьми,
лежавшими неподвижно. Меня охватил страх. Неужто и они спятили, как  Финк?
Или он сошел с ума еще от того удара током?

Первым очнулся  Фрэйзер.  Он  громко  застонал,  и  его  начало  рвать.
Джедевани лежал  без  сознания  еще  долго.  Тем  временем  раскрыл  глаза
Линдсей. Больше всего меня беспокоил профессор. Я делал ему  искусственное
дыхание осторожно, чтобы не поломать медвежьей услугой  хрупкие  ребра,  и
проклинал отсутствие доктора. К тому же я боялся оставить  их  одних,  так
как не знал, что делают на третьем этаже проклятый конус с Финком.

Наконец  веки  на  бледном,  похудевшем  лице  дрогнули,  и  мои  глаза
встретились с горящим темным взглядом старого ученого. Какое-то  мгновение
он глядел, потом прикрыл глаза, так что  я  испугался  и  потормошил  его,
может быть, слишком сильно.

"--* Осторожнее, Макмур, вопреки вашим усилиям я еще жив, --- дошел до  меня
слабый шепот. И бледная тень улыбки скользнула по лицу старика.

Я поднял его, усадил и принес воды. Через минуту он уже смог  говорить.
Первое, что он сказал, было:

"--* Вы тоже видели Марс?

У меня, вероятно,  была  идиотская  мина,  потому  что  он  нетерпеливо
добавил:

"--* Ну, не прикидывайтесь глупее, чей вы есть. Вы ничего не видели, или ---
если вам так больше нравится --- вам ничего не снилось?

"--* Ах, вот вы о чем! --- воскликнул я. --- Да, мне кое-что снилось, а может,
это была галлюцинация...

"--* Об этом после, --- сказал профессор. --- Мне думается, я могу встать. Для
рассказов будет время в другой раз, как говорит синьор Джедевани.  Кстати,
как он, как остальные?

"--* Живы.

Подошел Фрэйзер. Он тяжело дышал, и лицо у него было зеленоватое.

"--* Слава Богу, профессор, вы живы...

Линдсей стоял, опершись о столб, и вытирал лицо платком.

"--* Да, мы живы, но у меня странно кружится голова...

"--* Что с ареантропом? --- спросил профессор. --- Это самое важное,  пока  мы
еще, хе-хе, кудахтаем. Я перестану заботиться о нем, наверное, лишь  после
смерти, --- добавил он, слабо улыбнувшись.

"--* Эта коварная железная скотина в малом монтажном зале.

"--* Что? А  вы-то  откуда  знаете?  ---  Мужчины  сразу  выздоровели.  Даже
Джедевани попытался встать на ноги.

"--* Я был там, видел его --- горят аварийные лампы. С марсианином  работает
Финк.

"--* Вы хотите сказать, Финк работает над ним? Инженеру снова удалось  его
обезвредить? --- быстро спросил профессор.

"--* Я хочу сказать то, что сказал: Финк работает как бы  под  управлением
ареантропа. Я окликнул его, но инженер не отвечал.

"--* Может, это был не Финк? --- спросил Джедевани.

"--* Да, это была моя тетя. --- У меня лопнуло терпение. --- Профессор, вы мне
верите?

"--* Верю! Какой же я старый осел... Но тут и ваша вина. Не ваша,  Макмур,
"--* добавил он, не поднимая головы, когда я удивленно взглянул на него. --- Вы
"--* единственный из нас, кто сохранил трезвость ума и недоверчивость, а нам,
видите ли,  захотелось  поэкспериментировать,  этак,  знаете  ли,  связать
барашка --- и на операционный стол. И  вот,  извольте  получить  барашка,  ---
закончил он, ударив кулачком по подъемнику, около которого  мы  стояли.  ---
Теперь надо крепко подумать, что делать. Есть тут хоть что-нибудь, на  что
можно седеть?

Нашлось несколько треног. Профессор, морщась, примостился на  одной  из
них.

"--* О чем тут думать, это конец, ---  сказал  Джедевани,  который  все  еще
чувствовал себя скверно. --- Надо отыскать Бэрка, пусть готовит автомобиль.

"--* Лишаю вас голоса, --- уже пришел в себя профессор. --- Надо думать  не  о
том, как сбежать, а о том, что делать с ним дальше.  Да,  вот  что  ---  где
доктор?

"--* Я его сегодня не видел, а где он был утром?

"--*  Собирался  провести  в  лаборатории  какие-то  опыты  с  центральной
жидкостью, --- ответил Фрэйзер.

"--* Господа, --- насупился профессор, --- у  нас  нет  времени  на  болтовню.
Газовые гранаты еще есть?

"--* Были внизу, только  не  уничтожил  ли  он  их,  ---  сказал  я.  ---  Там
оставалось еще штук тридцать. Можно сходить.

"--* Господа,  ---  сказал  профессор,  ---  возможно,  кому-то  мои  указания
покажутся странными, однако я решаю так: принесите сюда  газовые  гранаты,
гранатометы и нацельте их на двери.  Мы,  прихватив  маски,  будем  сидеть
здесь. И пусть каждый расскажет, что видел во время своего...  отсутствия.
Не исключено, что это нам как-то поможет. Кто-нибудь против?

Таковых не оказалось. Я пошел с Линдсеем, который чувствовал себя лучше
других, и через  пятнадцать  минут  мы  уже  сидели  в  зале,  обложившись
гранатами  и  держа  в  руках  шнуры  запалов.  Стволы  гранатометов  были
направлены на двери.

"--* Уверен, еще до того, как мы его отравим, он исхитрится отправить  нас
к праотцам, --- воскликнул профессор. --- Не  надо  иллюзий.  Иллюзии  полезны
только ко времени и в меру, кроме  того,  человек  без  иллюзий  действует
смелее. А теперь прошу,  пусть  кто-нибудь  сбегает  наверх  за  доктором,
только не приближайтесь к малому залу.

Вызвался, конечно, я. Однако в лаборатории было пусто и темно. Напрасно
я кричал и искал. Вернулся я один.

"--* Странно, --- сказал профессор. --- Ну, прежде чем мы начнем  действовать,
надо сориентироваться. Хотелось бы от вас  кое-что  услышать.  Макмур,  вы
расскажете первым, что  видели  во  время  своей,  как  вы  ее  окрестили,
"галлюцинации".

Когда я закончил, воцарилось недолгое молчание.

"--* Да, мне знакомы эти ощущения, --- профессор поправил очки. --- Я  пережил
подобное, но с гораздо большими подробностями. Ваши выводы?

Мне что-то засветило.

"--* Думаю, это была сцена отправки ракеты с Марса на Землю,  а  потом  ее
полет через межпланетное пространство... Остальное ---  марсианские  обычаи,
сооружения, технические приспособления...

"--* Очень хорошо, --- профессор сказал это так, словно похвалил ученика  за
правильный ответ. --- Наблюдательный вы парень, Макмур. Когда  вы  принялись
болтать о снах и галлюцинациях, я, честно говоря, подумал, что вы  головой
стукнулись. Скажу правду: я тоже... тоже родом из Шотландии...  Тссс...  ---
он, видимо,  заметил,  что  я  расплылся  в  улыбке,  ---  сейчас  не  время
лобызаться. Думаю, касаясь нас  своим  щупальцем,  марсианин  определенным
энергетическим зарядом принуждал наши мозги работать в желаемом  для  него
направлении.

"--* А зачем ему понадобилась комедия с цилиндрами, волшебным  порошком  и
прочим? --- спросил Фрэйзер.

"--* Это понятно: вылетая с Марса, они, думается, не знали, кого  встретят
здесь, на Земле. Поэтому он подготовился к любой возможности, а  здесь  до
поры до времени вел себя так, как того хотели мы, --- пока не  приспособился
к  нашей  атмосфере  и  не  исправил  нанесенных  ему   повреждений.   Мою
безграничную глупость он использовал...

"--* Профессор, --- прервал я, --- полагаю...

"--* Повторяю,  мою  безграничную  глупость,  ---  перебил  старичок,  ---  он
использовал,   чтобы   освободиться,   пока   я   с   детской   наивностью
демонстрировал ему наш алфавит или учил  началам  геометрии.  Вот  и  все.
Хотелось бы, посмотреть сейчас на доктора, --- добавил он раздраженно, --- это
он уговорил меня на такие штучки...  Ну,  хватит.  Упреки  ни  к  чему  не
приведут.

"--* Господин профессор, --- сказал Фрэйзер, --- начни мы по-другому, он  тоже
воспользовался бы временем, когда в его распоряжении  был  ток,  ведь  ему
хватило бы нескольких секунд. Лучше расскажите, что видели вы.

"--* Что видел я? Хм, если б мы рассказывали  только  увеселения  ради,  я
вообще промолчал бы. В моей голове, друзья, сейчас такая пустота...  Хотя,
"--* добавил он, сверкнув глазами, --- я не  поддаюсь.  Для  меня  поддаться  ---
значит умереть.

И он с такой силой потянул за шнур спуска  гранатомета,  что  Джедевани
подпрыгнул.

"--* Я, --- тихо продолжал Уиддлтон, --- как и вы,  дорогие  мои,  видел  диск
Марса. Был на его поверхности. Сейчас нет времени  на  подробности,  но  я
видел машины для преобразования материи в энергию,  видел,  каким  образом
они перемещаются с места на место.

"--* Ну, ну? --- спросил я, заинтересованный. --- И как же это происходит?

"--* Вы видели только одну  стадию  процесса  и  как  следует  не  поняли.
Ареантроп входит в своего рода  приемную  камеру  из  какой-то  прозрачной
субстанции и там распыляется на атомы... И точно такой же ареантроп  в  ту
же (или следующую) секунду  материализуется  на  произвольном  расстоянии.
Условие одно: поблизости должна присутствовать соответствующая аппаратура.
Например, башни, что копают каналы, тоже выполняют роль приемников...

"--* Вон оно как, --- воскликнул я, --- вот,  значит,  как  они  появлялись  и
исчезали... Но каков механизм?

"--* Не знаю. Есть две возможности:  либо  сами  атомы  переносятся  через
пространство, либо, как я думаю, потеря энергии и материи  в  одном  месте
приводит в другой точке пространства,  как-то  идеально  уравновешенной  с
первой, к возникновению точно такой же конфигурации атомов и молекул.

"--* А как вы считаете, профессор, каналы-то зачем? --- спросил я. ---  Смешно
думать, что при их техническом совершенстве они  занимаются  возделыванием
земли и ее мелиорацией. Кстати, я не видел на  поверхности  ничего,  кроме
песков. Все внизу... Наверху --- лишь одна гигантская пустыня.

"--* Не везде, дорогой Макмур, --- странно улыбнулся профессор, ---  не  везде
пустыня. Есть там чудесные районы, рощи деревьев  с  пурпурными  листьями,
впадины, заполненные соленой черной  водой.  По  берегам  ползают  мириады
насекомых, вооруженных изготовленными  ими  же  самими  орудиями:  рогами,
челюстями, даже своего рода снарядами --- есть такие, которые выбрасывают на
расстояние ядовитое жало. В  воде  кружат  флюоресцирующие  тени  каких-то
других животных. Но все это панически убегает,  скрывается,  исчезает  под
камнями, на дне, в воздухе, когда приближается хотя бы один хозяин  Марса.
Areanthropos.

"--* Следовательно, они тоже овладели поверхностью  планеты,  вытесняют  и
уничтожают других животных? --- прервал я профессора.

"--* Что значит --- тоже? --- спросил он. --- Наши средства ---  ребячество...  О,
они умеют убивать, но в их железных футлярах  скрываются  более  серьезные
вещи... --- и, отвечая на наши удивленные  взгляды,  добавил:  ---  Ну,  не  в
голове же, ведь голов у них нет. И перестаньте меня все время  перебивать.
Так вот, несмотря на все  их  возможности,  это  очень  печальная  страна,
потому что я вообще не видел цели в  их  деятельности.  Конечно,  какое-то
время я восхищался автоматами, иногда каким-нибудь ареантропом, который на
значительном расстоянии уничтожал ненужный или  мешающий  каменный  выступ
или, наоборот, чудесным путем создавал какой-то  предмет  из  ничего...  Я
осматривал их эллипсоидные залы,  подземелья,  в  которых  не  видишь,  но
воспринимаешь все каким-то волшебным органом чувств,  но  я  не  улавливал
смысла  их  действий,  не  видел  цели,  а   одну   лишь   перегруженность
поразительными деталями. Есть ли у них человеческие  инстинкты?  Чувствуют
ли они? Любят ли? Ненавидят? Зачем на Землю  прибыло  так  поразившее  нас
создание? Макмур, вы не задумывались над этим в своем "сне"?

"--* Нет, признаться, я был слишком ошеломлен, профессор...

"--* Скверно... Я не хотел поддаться страшной чуждости  всего  увиденного,
чуждости, поражающей меня как человека,  как  ученого,  как  представителя
Земли, наконец... Я хотел видеть сквозь нее, вне нее. Это было чрезвычайно
трудно, ибо множество явлений я понять не мог. Как они живут?  ---  вопрошал
я. Ну, хорошо, я наблюдал, как они беззвучно передвигаются, и думал:  быть
может, я вижу лишь часть явлений. Ведь он, погрузив меня на Земле  в  свой
марсианский мир, не мог наделить новыми средствами восприятия. Мы  как  бы
видели в темноте, но я воспринимал это  как  некий  импульс,  активирующий
кору моего мозга при закрытых глазах. Так происходит во сне. Но, думал  я,
может,  они  воспринимают  как  блаженство  инфракрасное  или,   например,
космическое излучение? Может, какое-то иное  проявление  материи?  А  если
марсианину доступно чувство юмора, то ему мог бы  показаться  смешным  вид
курящего человека или пожирающего какую-то  падаль,  залитую  вдобавок  ее
соусом и сваренную в грязной воде... Или, к примеру, футляры из ослиной  и
коровьей шкуры на ногах,  все  эти  наши  одежды,  сшитые  в  виде  мешка,
рассеченного к тому же спереди, с трубами для рук и ног... и так  далее  и
так далее. Математика --- да, техника  ---  разумеется,  но  эти  смехотворные
штучки... Наши  ежедневные  увеселения,  вроде  стаканчика  водки,  ну,  а
проблема женщины, то есть вообще пола?

"--* И  долго  вы  еще  намерены  пытать  нас,  возбуждая  любопытство?  ---
проговорил Линдсей. --- Пожалуйста,  профессор,  ответьте  наконец  на  этот
миллион вопросов.

"--* Дорогие мои, не думайте, будто я забыл о том, что наш  друг,  инженер
Финк, сидит  на  третьем  этаже  и  вынужден,  похоже,  выполнять  приказы
ареантропа, который что-то там творит... замышляет... не знаю,  против  ли
нас, против ли человечества? Что мы можем сделать?

"--*  Взорвать  дом,  ---  крикнул  вдруг  Фрэйзер.  ---  Мины  подложены  под
фундаменты с позавчерашнего вечера. Главный запал по  указанию  профессора
питается не от сети, а от аккумуляторов...

"--* Только, пожалуйста, без меня, или вы хотите  отправить  всех  на  тот
свет только потому, что профессору грезится героическая смерть? --- с дрожью
в голосе спросил Джедевани. Мне показалось, что он был в ярости.

"--* Успокойтесь и сядьте, --- улыбнулся профессор.  ---  Это  уже  в  крайнем
случае. Рубильник рядом. Я не забывал о нем ни на минуту.  Благодарю  вас,
Фрэйзер. Он ведь был у меня под  рукой  в  ту  критическую  минуту,  когда
марсианин отшиб у меня ум, прежде чем  одним  прикосновением  послать  нас
всех на Марс.

"--* То есть? --- крикнули мы, а я добавил: --- Профессор! И  вы  не  включили
рубильник? Я б это сделал, ведь я шотландец!

"--* А я ждал, --- сказал профессор. --- Я рассуждал  просто:  зачем  ему  нас
убивать? Какая ему от этого польза? Я уже знал, что он ---  чертовски  умное
создание и ему нет нужды нас убивать,  потому  как  корысти  ему  с  этого
никакой, ничего плохого мы ему  уже  сделать  не  могли,  а  убивать  ради
удовольствия? Увы, я  верю,  что  такое  свойственно  только  человеку,  ---
добавил он тише. --- Я хотел испить до конца чашу  сию,  и  не  сожалею.  Не
знаю, возможно, наш гость, используя несчастного Финка,  готовит  аппараты
для уничтожения мира, возможно,  они  хотят  всех  нас  прикончить,  чтобы
колонизировать Землю, ибо им у себя тесно.

Я встал.

"--* Хорошенькое дельце! А мы сидим себе и в ус не дуем. Нет, пока  я  еще
могу хоть мизинцем пошевелить...

"--* Садитесь, садитесь! Почему же вы  не  пошевелили  мизинцем  утром?  ---
сказал профессор. --- Что делать, он  сильнее  нас.  Вы  обратили  внимание:
время для него ---  совсем  не  то,  что  для  нас.  Как  я  установил,  ему
понадобилось меньше секунды, чтобы записать два металлических  цилиндра...
И это еще не все. Коснувшись  щупальцем  ---  а  это  отняло  у  него  всего
несколько мгновений, --- он погрузил нас в состояние поразительного сна  или
гипноза, содержание  которого  для  каждого  он  создал  и  передал  почти
моментально. Наши мысли, дорогие коллеги, по сравнению с интеллектуальными
процессами марсианина ползут как улитки. Разве вы не понимаете,  что  даже
при всех прочих равных условиях он способен познать тысячи истин, пока  мы
познаем одну?

"--* Профессор, --- сказал я, --- мы знаем,  вы  правы.  Он  сильнее.  Поэтому
говорите, ответьте на вопросы, поставленные вами же.

"--* Друзья мои, --- ответил Уиддлтон,  ---  мне  ужасно  тяжело.  Так  просто
что-либо уничтожить  и  так  трудно  потом  исправить  ошибку.  Существуют
различные истины: есть созидающие, но есть  и  разрушающие.  Поэтому  я  в
нерешительности...

"--* О чем вы? Я вас не понимаю. Они пьют кровь? Режут друг  друга?  Может
быть, съедают? --- сказал я. --- Давайте, смело, профессор, нам это знакомо...
Все это есть и на Земле, так чем же можно нас поразить? Смешно...

"--* Нет, дорогой друг, не смешно, а страшно, --- возразил профессор, ---  ибо
то, что я видел, разрушило мои представления о  многом,  может  быть,  обо
всем. Внешне я вроде бы остался прежним, верно? Однако я отличаюсь от вас.
Я уже знаю, в чем цель и смысл жизни. --- Он замялся. --- Я  побывал  в  южном
полушарии Марса, --- сказал он и громовым голосом добавил: ---  Кто-нибудь  из
вас еще был в южном полушарии?

"--* Я видел то же, что и Макмур, --- заметил Фрэйзер.

"--* И я, --- подтвердил Линдсей. --- Видел сеть каналов, заполненных какой-то
жидкостью, машины... Говорите, профессор, говорите...

Старик наклонился, лицо у него побледнело.

"--* Да, я был в южном полушарии, --- повторил он  таким  странным  голосом,
что мне стало не по себе.

"--* И что вы там, черт побери, видели?

Профессор раскрыл рот.  В  этот  момент  дверь  отворилась  и  какое-то
качающееся, трясущееся тело ввалилось в зал,  сделало  несколько  шагов  и
повалилось на пол.

"--* Доктор! --- крикнул я. --- Доктор!

Он лежал без сознания. Из рассеченного лба текла  кровь,  рисуя  темную
полоску на бетоне.

Мы пытались привести его в чувство. Я вытащил у него из кармашка футляр
с ампулами --- все  были  разбиты.  Он  дышал  очень  поверхностно,  хрипло.
Удивительно, как ему удалось собственными силами добраться до зала.

"--* Где Бэрк? --- спросил кто-то.

Никто не ответил.

Доктор открыл глаза и застонал. Изо рта у него вытекло немного крови.

"--* У него внутреннее кровоизлияние, --- испуганно сказал я. --- Профессор!

Старик стоял неподвижно.

"--* Я не всесилен, Макмур, не всесилен... Боюсь, наш друг умирает.

Дыхание доктора то и дело  прерывалось.  Я  расстегнул  ему  рубашку  и
увидел ужасные сине-красные вздутия на груди.

"--* Его душили! --- крикнул я. --- У него сломаны ребра!

Доктор снова раскрыл глаза, проблеск  сознания  засветился  в  них.  Он
разжал губы, на подбородок снова вытекла струйка крови и сползла к  вырезу
рубашки.

"--*  Друзья...  ---  шепнул  он  и  сделал  такое  движение,  словно  хотел
подняться.

"--* Он силится что-то сказать. Вам нельзя, не напрягайтесь! ---  воскликнул
я, но доктор взглянул на меня так, что я  сам  же  первый  приподнял  его,
осторожно поддерживая голову. Он начал шептать, с  перерывами,  вызванными
все более усиливающимся кровотечением.

"--* Взорвать...  взорвать  все...  ---  прохрипел  он.  ---  Уничтожить  его,
немедленно, через минуту может быть поздно...

"--* Что случилось? Что с вами случилось, --- наперебой спрашивали мы.

"--* Это Финк... это Финк... я видел... видел...

Что он сказал о Финке? Я ничего не мог понять.

Голова доктора становилась все тяжелее.

"--* Все мины, одновременно, сразу же, взорвать.

Доктор терял сознание.

"--* Он бредит, --- сказал Фрэйзер.

Последняя яркая искра появилась в глазах доктора. Он выплюнул  огромный
сгусток крови из легких, задохнулся и произнес почти нормальным голосом:

"--* Немедленно взорвать весь дом, иначе всем верная гибель, ---  и  добавил
тише: --- Это Финк... Финк...

Голова упала набок. Я взял его запястье: пульса не было.

"--* Он умер.

Мы постояли над мертвым телом. Что делать?

"--* Надо пойти наверх, посмотреть, что творится с Финком,  и  попробовать
его спасти, --- сказал профессор. --- Потом мы взорвем здание. После того, что
я узнал... Я  думал,  не  ошибаюсь  ли  я?  Хотел  понять,  по-человечески
объяснить ту страшную картину, но теперь вижу, что был  не  прав.  Это  не
были призрачные галлюцинации, все гораздо хуже: это реальность.

Он выпрямился и прежним сильным голосом сказал:

"--* Господа, кто пойдет наверх?

Вызвались мы с Линдсеем.

"--* С вас на сегодня довольно, --- сказал Уиддлтон.  ---  Вы  уже  ходили  за
газовыми гранатами и доктором, вы  уже  виделись  с  ареантропом,  хватит.
Пойдете вы, инженер.

Линдсей затянул потуже ремень брюк.

"--* Пистолет у вас есть? --- спросил профессор каким-то странным голосом.

"--* Зачем мне пистолет?  ---  поразился  инженер.  ---  Ареантропа  пулей  не
возьмешь.

"--* Конечно, марсианина нет, наверняка нет... --- сказал профессор,  быстро
подошел к одному из шкафов, немного  покопался  в  нем  и  вручил  Линдсею
черный плоский браунинг. --- Возьмите. На всякий случай.

Инженер минуту глядел  на  блестящее  оружие,  взвесил  в  руке,  пожал
плечами и вышел, освещая себе путь фонариком. Мы стояли в дверях.  Фрэйзер
остался у гранатометов, Джедевани --- около "адской машины", а я и профессор
вышли в коридор. Тишина стала напряженнее. Я не мог так вот просто  стоять
и ждать.

"--* Я пройду только до лестницы, --- сказал я и сделал шаг.

Он остановил меня, сильно дернув за пиджак.

"--* Под честное слово, Макмур?

"--* Слово, профессор! --- воскликнул я и побежал к площадке. Было  видно  в
темноте, как слабый лучик фонарика забирается по маршам лестницы все  выше
и выше. Потом простучали шаги на уровне третьего  этажа,  я  услышал,  как
тихо раскрылась дверь малого монтажного зала --- и наступила тишина.

Кровь пульсировала у  меня  в  висках,  мышцы  напряглись,  я  стоял  в
чернильно-черной тьме и считал; сорок пять, сорок шесть, сорок семь, сорок
восемь,  сорок  девять,  пятьдесят...  Неожиданно  громыхнул  выстрел.   Я
вздрогнул, подскочил к лестнице, но проклятое честное слово удержало меня.
Второй выстрел, третий. И вдруг впервые в жизни волосы встали  у  меня  на
голове дыбом. Раздался вопль ужаса, рев  ошалевшего  от  страха  человека.
Громкий,  быстрый  топот  ног,  над  самой  лестничной  клеткой  прогремел
выстрел, резкий,  душераздирающий  крик  ударил  в  уши.  Я  ощутил  порыв
воздуха, и какая-то темная масса свалилась на бетон в метре от меня.

Что-то мокрое и теплое брызнуло в лицо, я прыгнул вперед --- из  фонарика
вырвался сноп света.

В  желтом  круге  возникло  размозженное  падением  с  третьего  этажа,
изуродованное, с вдавленной в плечи головой тело инженера Линдсея. Я узнал
его по брюкам, а когда схватился за одежду и перевернул его навзничь,  тут
же отскочил, крикнув от ужаса. Глаза у Линдсея были вытаращены, из  широко
раскрытого рта торчал язык,  который  он  откусил,  падая,  кровавая  пена
заливала лицо. Я услышал шаги --- это был профессор.

"--* Я боялся этого, --- остановившись у меня за спиной, шепнул он. ---  Этого
я боялся.

"--* Что случилось, профессор? Марсианин спятил?

Профессор долго смотрел на меня.

"--* Милый мой мальчик, --- сокрушенно сказал он наконец, --- не от марсианина
бежал Линдсей так, что потерял ориентацию и упал с лестницы, пролетев  два
этажа... И не марсианин поломал доктору ребра.

"--* А кто? --- спросил я, чувствуя, как замирает сердце.

"--* Финк, --- сказал профессор, развернулся и пошел во тьму. Я --- за ним.  В
зале профессор объяснил Фрэйзеру и Джедевани, что произошло.

"--* Финк? Не может быть! Он обезумел?

"--* Нет, он не сошел с ума. С  ума  может  сойти  человек.  Мальчики  мои
дорогие, --- не сердитесь, что так называю  вас,  ---  Финк  уже  не  человек.
Давайте устанавливать взрывные капсюли, добавим газовые гранаты, пусть все
разнесет в клочья. Так надо.

"--* Значит, сдаемся, профессор?

"--* Не думайте, что из-за потери друзей. Нет. Только потому,  что  я  уже
знаю это существо на третьем этаже. Оно и ему подобные не имеют  права  на
существование. Не нужна нам ни их  мудрость  и  знания,  ни  их  холодное,
страшное совершенство.

"--* Я не понимаю, профессор.

"--* Сейчас не время для разъяснений, Макмур. Подсоединяйте провода.

Под руководством Фрэйзера мы подвели дополнительные взрыватели ко  всем
газовым гранатам и бризантным снарядам и  вышли  из  зала,  разматывая  за
собой шнур электрозапальника.

Мы покинули здание через парадный  подъезд  ---  я  ждал  на  ступенях  с
катушкой провода, а Джедевани побежал в служебный домик  за  шофером.  Это
был первый приказ, который он выполнил с величайшим желанием и  быстротой.
Уже через минуту подъехал знакомый мне черный "бьюик", которому предстояло
отвезти нас в город.

"--* Бэрк, --- распорядился профессор, --- подъезжайте к концу пруда  и  ждите
нас там.

Шофер, который, казалось, ничему не удивлялся, включил  стартер.  Мотор
заворчал, и вскоре задние красные огоньки лимузина исчезли в ночном мраке.

Было довольно холодно и влажно. Профессор шел первым, мы трое --- за ним,
причем я нес катушку, с которой сматывался провод.

Когда мы отошли метров на триста, провод кончился. Мы залегли  в  сухой
меже, разделяющей поля. Я включил фонарик,  а  Фрэйзер  подсоединил  концы
проводов к специальному ключу с широкой рукояткой и молча, не произнеся ни
слова, подал его профессору. Я глянул в сторону дома.  Он  был  совершенно
темный, почти невидимый, только крыша с горбом купола выделялась  на  фоне
более светлого неба. На  третьем  этаже,  в  крыле,  виднелись  три  слабо
освещенных окна. При мысли, что там находится ареантроп и Финк ---  какой-то
совершенно незнакомый, страшный инженер Финк, который не сошел с  ума,  но
убивал своих друзей, я почувствовал, как по спине прошла дрожь.  Профессор
велел нам прижаться головами к земле и повернул рукоятку.

Гигантская красная вспышка расцвела  во  тьме.  Раздался  глухой  рокот
взрыва, а вслед за ним дробью посыпались очереди  других,  хруст  и  треск
ломающихся   перекрытий,   скрежет   и   лязг   падающих   машин,   грохот
разваливающихся и рушащихся  в  языках  пламени  стен.  Наконец  огромное,
прекрасное  здание  превратилось  в  колоссальную,  все   еще   дымящуюся,
стреляющую в небо языками пламени осыпь.

Потом наступила тишина. Лишь с тихим шумом падали на траву мелкие капли
дождя.

Профессор медленно поднялся.

"--* Друзья, мы закончили свою работу. Какой же это печальный конец!  Трое
крупных ученых заплатили за нее жизнью. Узнали мы  что-нибудь?  Да,  одну,
мне кажется, истину. Планеты --- чужды друг другу. Не так чужды два человека
"--* один из жаркой Австралии, другой --- с полярных льдов. Не так  чужды  друг
другу человек и зверь, птица и насекомое. Их нечто объединяет и связывает.
Они выросли под одним небом. Дышат  одним  воздухом.  Их  обогревает  одно
солнце. Общие истины? Да, есть и общие истины. Но у каждого есть  и  своя,
пригодная только для него. Тут суть, дорогие мои, не в цене открытия. Суть
в его высшем смысле. Что еще мы  узнали?  Вырвали  мы  у  пришельца  тайну
преобразования материи? Нет. Так, может  быть,  мы  узнали  что-то  о  нас
самих? Какую-то новую истину? Увы, да. Я ее познал, И зачем она  мне,  эта
истина? Друзья мои, я изменил решение: я не скажу вам ничего. И вы  должны
чувствовать не любопытство, обычное несчастное человеческое любопытство, а
испытывать благодарность ко мне. Благодарность за  то,  что  я  ничего  не
сказал и не скажу. Ибо марсианин --- чудовищное существо.  Он  знал,  что  я
руковожу вами, знал,  что  я  ---  самый  сильный.  И  подумал:  "Чтобы  его
уничтожить, мне достаточно захотеть. Но какой мне от этого прок? Никакого!
Нет, я его сломаю. Открою ему глаза на то, о чем он даже и подозревать  не
мог". Так он и поступил. Друзья мои, я хочу верить, что Марс  уже  никогда
больше не покусится на овладение Землей. Там вылупился и вырос  в  плазме,
экранированной сталью, разум, которому чуждо любое чувство. Да,  им  чужды
ненависть, уныние, злость, гнев,  ярость,  но  столь  же  чужды  и  добро,
дружба, радость, любовь. А что влечет человека к науке, к  познанию,  если
не любовь --- любовь к истине? Что может сделать человек, не знающий любви?

Земными словами рек апостол Павел: "Если я говорю языками человеческими
и ангельскими, а любви не имею, то я --- медь звенящая, или кимвал звучащий"
[1 Кор. 13:1]. Да, это слова человеческие. И я горжусь этим.  Друзья,  нам
необходимо забыть. Невозможно объяснить людям то, что вы пережили, так как
же могу сделать это я, тот, кого подверг тягчайшему испытанию пришелец  из
другого мира?

Профессор замолчал. Наши одежды намокли от дождя. Влажный  ветер  донес
запах гари. Небо на востоке начало сереть.

"--* Профессор Уиддлтон, --- сказал я,  ---  если  не  хотите  сказать,  какие
страсти продемонстрировал вам этот  тип  с  Марса  на  своей  планете,  то
скажите хотя бы одно: он одержал над вам верх? Одолел вас?  И  ради  Бога,
что произошло с Финком?

"--* Одолел ли он меня... --- тихо  сказал  профессор.  ---  Два  эти  вопроса
связаны так тесно, как вы даже не подозреваете. Вот Финка он действительно
одолел. Там, наверху, действовал ведь не Финк  под  гипнозом,  не  Финк  в
каталептическом состоянии, не Финк безумный... Внешне это был  Финк  ---  не
наш друг-инженер, а нечто с его телом, его руками и ногами, в его  одежде.
И при этом "оно" не было инженером Финком.

"--* Что такое? --- остолбенел я. --- Что это значит?

"--* Да, да, вы удивлены...

"--* Ну, хорошо, тогда что марсианин с ним сделал?

"--* Не знаю точно, только догадываюсь на основании увиденного.  Ареантроп
что-то изменил в нем, что-то отнял, а взамен дал что-то другое.

Я был совершенно заморочен.

"--* Так что же он отнял и что дал? --- спросил я. --- Может  быть,  это  была
душа? И он переправил ее на Марс?

"--* Напрасно вы  с  такой  язвительностью  произносите  слово  "душа",  ---
тихонько сказал профессор. --- Нет, это была не душа.  Я  не  могу  сказать,
дорогой мой, ибо это как раз и  относится  к  тому,  о  чем  мне  говорить
нельзя. Но отвечу так; во-первых, он меня не одолел, не  сломал.  В  самую
тяжелую, самую трудную минуту я вспомнил, кто я есть и кого  люблю.  Может
быть, это меня спасло. Необходимо, чтобы в тебе было хоть немного  веры...
Хотя лучше, чтобы она была и в  тебе,  и  в  других.  Только  тогда  можно
что-либо совершить. А во-вторых, попытаюсь пусть не объяснить, но  сделать
чуть-чуть понятнее проблему  инженера  Финка.  Вы  знаете,  что  можно  по
собственному желанию приводить в движение мускулы тела. Верно?

"--* Конечно.

"--* Ну, хорошо, а как это делается?

"--* Разные есть теории, --- ответил я, --- но,  насколько  я  знаю,  механизм
"желания", воли не известен.

"--* Не известен вам, но не мне, чтобы быть точным, --- сказал профессор.  ---
И слава Богу. Но иногда, когда вы спокойно лежите, бывает,  что"  тот  или
иной мускул нет-нет да самопроизвольно сократится, вздрогнет, и вы  можете
это видеть. Так?

"--* Случалось, --- ответил я. --- Думаю, такое бывало с каждым.

"--* И кто же действует в этом случае?

"--* О, знаю, --- сказал я. ---  Возможно,  молочная  кислота  образовалась  в
какой-то мышце или в коре  головного  мозга  какой-нибудь  приблудный  ток
пощекотал соответствующий центр...

Профессор кивал и улыбался.

"--* Неплохо, неплохо. Но все далеко не так просто... Если  вы  знакомы  с
различными теориями, объясняющими деятельность мозга, то должны знать, что
некоторые ученые именно в таких токах и раздражениях  усматривают  причины
произвольных движений.

"--* Ну, да, конечно. Но есть и другие...

"--* Оставим  их  в  стороне,  чтобы  не  разбрасываться.  Итак,  то,  что
некоторые именуют произвольным  механизмом,  иногда  действует  не  только
помимо нашей  воли,  но  даже  наперекор  ей.  Я,  конечно,  говорю  не  о
сокращении утомленных мышц, тут дело ясное, а о колебаниях мышц совершенно
нормальных во время отдыха...

Стало быть, все выглядит так: если эту произвольно сокращающуюся  мышцу
считать как бы аналогом инженера Финка до произошедшего с  ним  несчастья,
то ту, другую, которую вы видите в действии, удивляясь, что она ни с  того
ни с сего сжимается, работу которой вы  чувствуете,  хотя  происходит  это
совершенно помимо вашего желания, --- такую  мышцу  можно  назвать  аналогом
Финка после всего случившегося.

И глядя при свете наступающего утра мне  в  лицо,  видимо  не  очень-то
отмеченное мудростью, он добавил:

"--* Понимаете, Макмур, я не могу вам сказать ничего больше, не сказав при
этом всего... А этого мне делать нельзя.

Уже почти совсем рассвело. Профессор обратился к нам:

"--* Прежде чем пойти к машине, надо осмотреть развалины. Если даже, как я
надеюсь, конус уничтожен, то не исключено, что  центральный  шар,  пузырь,
уцелел. А это как раз и есть то страшное творение... Мертвая-то материя  в
чем виновата?

"--*  Профессор,  ---  спросил  я,  ---  вы  считаете,  что   шар   сам,   без
аппарата-носителя, способен восстанавливать  все  ему  необходимое?  Такой
бессильный... пузырь?

"--* Не хотел бы я иметь такого "бессильного" противника, даже если  б  за
мной стояла  вся  армия  Соединенных  Штатов,  ---  сказал  профессор.  ---  И
перестаньте у меня выпытывать, все равно ничего больше не скажу.

Мы  двигались  по  тяжелой,  мокрой  траве.  Я  первым   прошел   через
разрушенный взрывом подъезд и стал  карабкаться  по  обломкам.  Чудовищная
сила взрыва погнула тяжеленные профилированные фермы из  массивной  стали,
размозжила  головки   моторов,   разбила   на   части   монолитные   блоки
железобетона.

Неожиданно я увидел под обломками что-то черное --- и подошел ближе.

Это было оторванное щупальце чудовища.

"--* Не прикасаться! --- крикнул профессор. ---  Его  плазма  ---  смертна,  его
злость --- нет.

Я продолжал искать. Увидел что-то вроде погнутого колпака конуса, но не
был уверен, что  это  действительно  он.  Потом  нас  окликнул  Фрэйзер  ---
втиснутый между кирпичами, торчал обрывок пиджака  Финка.  Мы  бродили  по
развалинам, от которых  все  еще  несло  хлором  газовых  гранат.  Наконец
профессор сказал:

"--* Все это впустую. Я прикажу залить руины несколькими сотнями  галлонов
бензина и поджечь. Тогда, если даже ядро  выжило  при  взрыве,  оно  будет
уничтожено пламенем, --- и маленький, черный, сутулый, он двинулся к выходу.
Перепрыгивая с камня на камень, мы шли за ним по  осыпающимся  под  ногами
обломкам.

\bigskip{}



Мы молча  отправились  в  путь,  поглядывая  по  сторонам.  Неподалеку,
справа, раскинулись зеленые непрозрачные воды пруда. Наконец за  поворотом
появился автомобиль --- черная точка на серой ленте  автострады.  В  тот  же
момент небо на востоке вспыхнуло красным и золотым и первый сноп солнечных
лучей, словно торжественный огненный салют, пронзил голубизну. Низкие тучи
быстро разбежались,  отсвечивая  кремовым  и  белым.  Лица  овевал  ветер,
насыщенный водяной пылью из пруда.

"--* И он хотел все это у нас отнять... --- шепнул профессор.

Верно ли я его понял? Спрашивать не было смысла.

Мы молча добрались до  черного  тяжелого  "бьюика".  Бэрк  выскочил  из
машины и отворил нам  дверцы.  Мы  сели.  Дверцы  захлопнулись.  Несколько
секунд скрежетал стартер, пахнуло дымом.

Машина дрогнула и покатилась в сторону Нью-Йорка.

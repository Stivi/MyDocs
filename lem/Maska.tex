
Станислав Лем.

Маска



Stanislav Lem. Maska, 1976

И.Левшин, перевод, 1976


Вначале была тьма, и холодное пламя, и протяжный гул; и многочленистые,
обвитые  длинными шнурами искр, дочерна опаленные крючья передавали меня все
дальше, и металлические извивающиеся змеи тыкались в меня плоскими рыльцами,
и  каждое  такое  прикосновение  пробуждало  молниеносную,  резкую  и  почти
сладостную дрожь.

Безмерно  глубокий,  неподвижный взгляд, который смотрел на меня сквозь
круглые стекла, постепенно удалялся,  а  может  быть,  это  я  передвигалось
дальше и входило в круг следующего взгляда, вызывавшего такое же оцепенение,
почтение  и  страх. Неизвестно, сколько продолжалось это мое путешествие, но
по  мере  того,  как  я  продвигалось,  лежа  навзничь,  я  увеличивалось  и
распознавало себя, ища свои пределы, хотя мне трудно точно определить, когда
я  уже  смогло  объять  всю  свою  форму,  различить  каждое  место,  где  я
прекращалось и где начинался мир,  гудящий,  темный,  пронизанный  пламенем.
Потом   движение   остановилось  и  исчезли  суставчатые  щупальца,  которые
передавали меня друг другу, легко поднимали вверх, уступали зажимам  клещей,
подсовывали  плоским  ртам,  окруженным  венчиками  искр;  и хоть я было уже
способно к самостоятельному движению, но лежало еще неподвижно,  ибо  хорошо
сознавало, что еще не время. И в этом оцепенелом наклоне -- а я лежало тогда
на наклонной плоскости -- последний разряд, бездыханное касание, вибрирующий
поцелуй  заставил  меня  напрячься: то был знак, чтобы двинуться и вползти в
темное круглое отверстие, и уже без всякого понуждения я коснулось  холодных
гладких  вогнутых  плит, чтобы улечься на них с каменной удовлетворенностью.
Но может быть, все это был сон?

О пробуждении я не знаю ничего. Помню только  непонятный  шорох  вокруг
меня  и  холодный  полумрак. Мир открылся в блеске и свете, раздробленном на
цвета, и еще так много удивления было в моем  шаге,  которым  я  переступало
порог.  Сильный  свет  лился  сверху  на красочный вихрь вертикальных тел, я
видело насаженные на них  шары,  которые  обратили  ко  мне  пары  блестящих
влажных  кнопок.  Общий шум замер, и в наступившей тишине я сделало еще один
маленький шаг. И тогда в неслышном еле ощутимом звуке будто лопнувшей во мне
струны я почувствовало наплыв своего  пола,  такой  внезапный,  что  у  меня
закружилась  голова,  и  я  прикрыла  веки.  И пока я стояла так с закрытыми
глазами, до меня со всех сторон стали долетать слова, потому  что  вместе  с
полом  я  обрела  язык. Я открыла глаза, и улыбнулась, и двинулась вперед, и
мое платье зашелестело. Я шла величественно, окруженная кринолином, не  зная
куда, но шла все дальше, потому что это был придворный бал, и воспоминание о
моей  ошибке -- о том, как минуту назад я приняла головы за шары, а глаза за
мокрые пуговицы, -- забавляло  меня  ее  ребяческой  наивностью,  поэтому  я
улыбнулась,  но улыбка эта была предназначена только мне самой. Слух мой был
обострен, и я издалека  различала  ропот  изысканного  признания,  затаенные
вздохи кавалеров и завистливые вздохи дам: "Откуда эта девочка, виконт?" А я
шла  через  гигантскую  залу под хрустальной паутиной жирандолей, и лепестки
роз капали на меня с сетки, подвешенной к потолку, и я видела свое отражение
в  похотливых  глазах  худощавых  пэров  и  в  неприязни,   выползающей   на
раскрашенные лица женщин.

В окнах от потолка  до паркета зияла ночь, в парке  горели смоляные бочки, а
между окнами, в нише у подножья мраморной статуи, стоял человек, ростом ниже
других,  окруженный придворными  в черно-желтых  полосатых одеждах.  Все они
словно  бы стремились  к  нему,  но не  переступали  пустого  круга, а  этот
человек,  один из  всех,  когда  я приблизилась,  даже  не  посмотрел в  мою
сторону.

Поравнявшись с  ним, я приостановилась и,  хотя он даже отвернулся  от меня,
взяла  слабыми кончиками  пальцев кринолин  и опустила  глаза, будто  хотела
отдать ему глубокий поклон, но только  глянула на свои руки, тонкие и белые,
и,  не  знаю  почему,  их  белизна,  засиявшая  на  голубом  атласе  платья,
показалась мне  чем-то ужасным. Он же,  этот низенький господин или  пэр, за
спиной которого возвышался  бледный мраморный рыцарь в  латном полудоспехе с
обнаженной  белой  головой  и  с  маленьким,  будто  игрушечным  трехгранным
мизерикордом, ``кинжалом  милосердия''\footnote{Кинжал, которым  на поединке
добивали поверженного противника.},  в руке, не соизволил  даже взглянуть на
меня,  он говорил  что-то низким,  как бы  сдавленным скукой  голосом, глядя
прямо перед собой и ни к кому не  обращаясь. А я, так и не поклонившись ему,
только посмотрела на него быстро и пристально, чтобы навсегда запомнить лицо
со  слегка перекошенным  ртом, угол  которого  был стянут  белым шрамиком  в
гримасу вечной скуки.

Впиваясь  глазами в этот рот, я повернулась на каблуке так, что зашумел
кринолин, и пошла дальше. Только тогда он  посмотрел  на  меня,  и  я  сразу
почувствовала этот взгляд -- быстрый, холодный и такой пронзительный, словно
бы  к  его  щеке прижат приклад, а мушка невидимой фузеи нацелена на мою шею
между завитками золотых буклей, -- и это было вторым началом.  Я  не  хотела
оборачиваться,  и  все  же  повернулась  к  нему  и, приподняв обеими руками
кринолин, склонилась в низком, очень низком реверансе, как бы  погружаясь  в
сверкающую  гладь  паркета,  ибо  то  был  король.  Потом я медленно отошла,
размышляя над тем, отчего, зная все это так твердо и наверняка, я чуть  было
не  совершила  ужасной  оплошности:  должно  быть, потому что раз я не могла
знать, но узнавала все каким-то навязчивым и безоговорочным путем,  то  чуть
было  не приняла все за сон, -- однако что стоит во сне, к примеру, схватить
кого-нибудь за нос? Я даже испугалась, что не могу совершать промахи оттого,
что во мне возникает как бы невидимая граница. Так я  и  шла  между  сном  и
явью, не зная куда и зачем, и при каждом шаге в меня вливалось знание, волна
за  волной,  как на песке оставляя новые имена и титулы, будто сплетенные из
кружев, и на середине залы, под сияющим канделябром, который  плыл  в  дыму,
как  пылающий  корабль,  я  уже  знала  всех этих дам, искусно прячущих свою
изношенность под слоями грима.

Я знала уже столько, сколько знал бы человек, который вполне очнулся от
кошмара, но  помнит  его  почти  ощутимо,  а  то,  что  еще  было  для  меня
недоступным,  рисовалось в моем сознании, как два затмения: откуда я и кто я
-- ибо я все еще ни капельки не знала себя самое. Правда, я уже ощущала свою
наготу, укрытую богатым нарядом: грудь, живот, бедра, шею, руки,  ступни.  Я
прикоснулась к топазу, оправленному в золото, который светлячком пульсировал
в  ложбинке на груди, и тотчас почувствовала, какое у меня выражение лица --
неуловимое выражение, которое должно было изумлять, потому что каждому,  кто
смотрел   на  меня,  казалось,  что  я  улыбаюсь,  но  если  он  внимательно
присматривался к моим губам, глазам, бровям, то замечал, что  в  них  нет  и
следа  веселости,  даже  вежливой, и снова искал улыбку в моих глазах, а они
были совершенно спокойны, он переводил взгляд на щеки, на подбородок, но там
не было трепетных ямочек: мои щеки были гладки и белы, подбородок  серьезен,
спокоен,  деловит  и  так  же  безупречен, как и шея, которая тоже ничего не
выражала. Тогда смотревший впадал в недоумение, не понимая, как ему пришло в
голову,  что  я  улыбаюсь,  и,  ошеломленный  своей  растерянностью  и  моей
красотой,  отступал  в  глубь толпы или отвешивал мне глубокий поклон, чтобы
хоть этим жестом укрыться от меня.

А я все еще не знала двух вещей, хотя и понимала, что они самые важные.
Я не могла понять, почему король не посмотрел на  меня,  когда  я  проходила
мимо, почему он не хотел смотреть мне в глаза, хоть и не боялся моей красоты
и  не желал ее; я же чувствовала, что по-настоящему ценна для него, но ценна
каким-то невыразимым образом, так, будто бы я  сама  была  для  него  ничем,
вернее,  кем-то  как  бы  потусторонним  в этой искрящейся зале и что я была
создана  не  для  танца  на  зеркальном  паркете,  уложенном  многокрасочной
мозаикой  под  литыми  из  бронзы  гербами,  украшающими  высокие притолоки;
однако, когда я прошла мимо него, в нем  не  возникло  ни  одной  мысли,  по
которой  я могла бы догадаться о его королевской воле, а когда он послал мне
вдогонку взгляд, мимолетный и небрежный,  но  как  бы  поверх  воображаемого
дула,  я  поняла еще и то, что не в меня целились эти белесые глаза, которые
стоило  бы  скрыть  за  темными  стеклами,  потому  что  лицо  его   хранило
благовоспитанность,  а  глаза не притворялись и среди всей этой изысканности
выглядели как остатки грязной воды в медном тазу. Пуще того, его глаза  были
словно подобраны в мусорной куче -- их не следовало бы выставлять напоказ.

Кажется,  он  чего-то  от меня хотел, но чего? Я не могла тогда об этом
думать, ибо должна была сосредоточиться на другом. Я знала  здесь  всех,  но
меня  не  знал  никто.  Разве  только он, король. Теперь, когда во мне стало
возникать знание и о себе, странное ощущение овладело мною, и, когда, пройдя
три четверти зала, я замедлила шаг, в разноцветной массе лиц  окостенелых  и
лиц   в   серебряном  инее  бакенбардов,  лиц  искривленных  и  одутловатых,
вспотевших под скатавшейся пудрой, меж орденских лент  и  галунов,  открылся
коридор, чтобы я могла проследовать, словно королева, по этой узкой тропинке
сквозь паутину взглядов, чтобы я прошла -- куда?

К кому-то.

А кем была я сама? Мысли мои  неслись с невероятной быстротой, и я в секунду
поняла, сколь необычно  различие между мною и этим  светским сбродом, потому
что у  каждого из них были  свои дела, семья, всяческие  отличия, полученные
путем интриг и подлостей, каждый носился со своею торбой никчемной гордости,
волоча за  собой свое  прошлое, как  повозка в  пустыне тянет  сзади длинный
хвост поднятой пыли. Я же была  из таких далеких краев, что, казалось, имела
не одно прошлое, а множество, и  поэтому моя судьба могла стать понятной для
них только  в частичном переводе на  здешние нравы, но по  тем определениям,
которые  удалось бы  подобрать,  я  все равно  осталась  бы  для них  чуждым
существом. А  может быть,  и для  себя тоже? Нет...  а впрочем,  пожалуй, да
--  у  меня  ведь  не  было  никаких знаний,  кроме  тех,  что  ворвались  в
меня  на пороге  залы, как  вода, которая,  прорвав плотину,  бурля заливает
пустоту.  Ища в  этих  знаниях  логику, я  спрашивала  себя,  можно ли  быть
сразу  множеством?  Происходить  сразу  из  многих  покинутых  прошлых?  Моя
собственная  логика, отделенная  от бормочущих  воспоминаний, говорила  мне,
что  нельзя,  что прошлое  может  быть  лишь  одно,  а если  я  одновременно
графиня Тленикс, дуэнья  Зореннэй, юная Виргиния -- сирота,  у которой родню
истребил  валандский род  в  заморской  стране Лангодотов,  если  я не  могу
отличить вымысла от действительности, докопаться  до истинной памяти о себе,
то,  может быть,  я  сплю?  Но уже  загремел  оркестр;  бал напирал,  словно
каменная  лавина,  и трудно  было  поверить  в  другую, еще  более  реальную
действительность.  Я шла  в неприятном  ошеломлении, следя  за каждым  своим
шагом, потому что снова началось головокружение, которое я почему-то назвала
vertigo\footnote{Кружение (лат).}.

Я ни  на миг не  сбилась в своей  королевской поступи, хотя  это потребовало
огромного напряжения,  незаметного внешне, и  ради этой незаметности  -- еще
больших  усилий, пока  я не  почувствовала  поддержку извне:  то был  взгляд
мужчины, который  сидел в низком проеме  приоткрытого окна, -- на  его плечо
свесилась  складка парчовой  занавеси, расшитой  красно-седыми коронованными
львами, страшно  старыми, поднимавшими в  лапах скипетры и яблоки  держав --
райские, отравленные яблоки. Этот  человек, уединившийся среди львов, одетый
во все черное, прилично, но с  долею естественной небрежности, в которой нет
ничего общего с искусственным дамским  беспорядком, этот чужой, не денди, не
чичисбей\footnote{Дамский угодник (спутник для  прогулок).}, не придворный и
вовсе не красавчик, но и не старик, смотрел на меня из своего укрытия, такой
же  одинокий  в  этом  всеобщем  гомоне,  как  и  я.  Вокруг  толпились  те,
кто  раскуривает  cigarillo  свернутым  банкнотом  на  глазах  партнеров  по
tагоссо\footnote{Средневековая  карточная  игра  особыми картами  (таро)}  и
бросает золотые  дукаты на  зеленое сукно  так, как  швыряют в  пруд лебедям
мускатные  орехи,  --  люди,  которые  не  могут  совершить  ничего  глупого
или  позорного,  ибо их  знатность  облекает  благородством любой  поступок.
А  этот  мужчина  в  высшей  степени  не  подходил  к  такому  окружению,  и
снисходительность, с  которой он  как бы нечаянно  позволял жесткой  парче в
королевских львах перевешиваться  через плечо и бросать на  его лицо отблеск
тронного пурпура, выглядела тихим издевательством. Он был немолод, но юность
все еще жила в его темных, нервно прищуренных глазах, он слушал, а возможно,
и  не слышал  своего собеседника,  маленького лысого  толстяка, похожего  на
доброго закормленного  пса. Когда незнакомец встал,  занавесь соскользнула с
его плеча --  ненужная отброшенная мишура, и наши глаза  встретились в упор,
и  мои  сразу же  скользнули  прочь,  будто  обратились  в бегство  --  могу
поклясться! Но  его лицо осталось  на дне  моих глаз --  я как бы  ослепла и
оглохла на мгновенье, так что вместо оркестра некоторое время слышала только
стук своего сердца. Не знаю почему.

Уверяю   вас,   лицо   у  него  было  совершенно  обыкновенное.  В  его
неправильных  чертах  была  та  привлекательная  некрасивость,  что  нередко
свойственна  высоким  умам;  но,  казалось,  он  уже  устал  от собственного
интеллекта, излишне проницательного, который мало-помалу  подтачивал  его  в
самоубийственных ночных бдениях, -- видно было, что ему приходится тяжко и в
иные  часы он рад бы избавиться от своей мудрости, уже не привилегии и дара,
но увечья, ибо неустанная работа мысли начала ему досаждать, особенно  когда
он  оказывался  наедине с собой, что случалось с ним часто -- почти всегда и
везде, а значит, и здесь. У меня вдруг возникло желание  увидеть  его  тело,
спрятанное  под добротной, чуть мешковатой одеждой, сшитой так, будто он сам
сдерживал старания портного. Довольно печальной должна, наверное,  быть  эта
нагота,   почти  отталкивающе  мужественная,  с  атлетической  мускулатурой,
перекатывающейся  узлами  вздутий  и  выпуклостей,  со  струнами  сухожилий,
способная  вызвать  страсть  разве что у стареющих женщин, которые упорно не
желают от всего навсегда отказаться и шалеют, как  нерестящиеся  рыбы.  Зато
голова  его  была  так  по-мужски  прекрасна  --  гениальным  рисунком  рта,
гневливой запальчивостью бровей, как бы разрезанных морщинкой посредине; его
крупный, жирно лоснящийся нос даже чувствовал себя смешным в такой компании.
Ох, не был красив  этот  мужчина,  и  даже  некрасивость  его  не  искушала,
попросту  он  был  другой,  но если бы я внутренне не расслабилась, когда мы
столкнулись взглядами, то, наверное, могла бы пройти мимо.

Правда, если бы я так поступила, если бы мне удалось вырваться из сферы
его притяжения,  всемилостивейший  король  тотчас  же  занялся  бы  мною  --
дрожанием  перстня,  уголком выцветших глаз, зрачками, острыми, как булавки,
-- и я вернулась бы туда, откуда пришла. Но в тот миг и на том  месте  я  не
могла  еще  этого  знать,  я  не  понимала, что та, словно случайная встреча
взглядов, мимолетное совпадение черных отверстий зрачков  --  а  они  же:  в
конце  концов, всего-навсего дырочки в круглых приборах, проворно скользящих
в глазницах черепа, -- что это все заранее предопределено, но откуда мне это
было знать тогда!..

Я уже прошла мимо, когда он встал, сбросил с рукава  зацепившийся  край
парчи и, как бы давая понять, что комедия окончена, двинулся за мной. Сделав
два шага, он остановился, вдруг осознав, каким пошлым ротозейством выглядела
эта   его   отчаянная  решимость  плестись  за  незнакомой  красавицей,  как
зазевавшийся дурачок за оркестром. Он остановился,  и  тогда,  сложив  кисть
руки  лодочкой,  я  другой  рукой  сдвинула с запястья петельку веера. Чтобы
упал. И он, конечно, тут  же...  Мы  рассматривали  друг  друга  уже  совсем
вблизи,   между  нами  была  только  перламутровая  ручка  веера.  Это  была
прекрасная и страшная минута -- смертный  холод  перехватил  мне  горло,  и,
чувствуя, что вместо голоса могу выдавить из себя только слабый хрип, я лишь
кивнула  ему,  и  этот  мой кивок был таким же неуверенным, как тот недавний
реверанс перед королем, не удостоившим меня взглядом.

Он не ответил на мой поклон -- он  был  растерян  и  изумлен  тем,  что
происходило  в нем самом, ибо такого он от себя не ожидал. Я знаю это точно,
он позже сам сказал об этом, но, если бы и не сказал, я все равно бы  знала.
Ему  нужно  было что-то говорить, чтобы не стоять столбом, как болван, каким
он выглядел тогда, отлично это сознавая.

-- Сударыня, -- произнес он, прихрюкивая, как поросенок, --  сударыня,
вот веер.

Я уже давно держала в руках и веер, и, кстати, себя тоже.

-- Сударь,   --   отозвалась   я,   и  голос  мой  прозвучал  чуть-чуть
приглушенно, как чужой, и он мог подумать, что это мой обычный  голос,  ведь
раньше он никогда его не слышал, -- может быть, мне уронить веер еще раз?

И  улыбнулась  --  нет,  не  искушающе,  не  соблазнительно,  не  лучезарно.
Улыбнулась только потому, что почувствовала, как краснею. Однако тот румянец
был не моим: он вспыхнул на моих щеках, разлился по лицу, окрасил мочки ушей
-- я все  это прекрасно ощущала, но  я вовсе не испытывала  ни изумления, ни
восхищения, ни замешательства перед этим  чужим человеком, в сущности, одним
из многих, как он, затерянных в толпе придворных; скажу точнее: этот румянец
не имел ничего общего со мной, он возник из того же источника, что и знание,
которое вошло  в меня на  пороге залы с первым  моим шагом на  ее зеркальную
гладь,  тот румянец  был как  бы частью  придворного этикета  -- всего,  что
принято,  как  веер, кринолин,  топазы  и  прическа.  И  чтоб он  не  посмел
истолковать всего превратно, чтобы показать,  как мало значит мой румянец, я
улыбнулась,  но  не  ему,  а  поверх  его  головы,  отмерив  как  раз  такое
расстояние,  какое отделяет  любезность от  насмешки. И  он захохотал  тогда
почти беззвучно, как бы про  себя, точь-в-точь мальчишка, который знает, что
строже всего  на свете ему запрещено  смеяться, и именно поэтому  не в силах
удержаться. И от этого смеха мгновенно помолодел.

-- Если бы ты дала мне минуту отсрочки, -- сказал  он,  вдруг  перестав
смеяться,  словно  протрезвел  от новой мысли, -- я бы смог придумать ответ,
достойный твоих слов, то есть в высшей степени остроумный, но  лучшие  мысли
всегда приходят мне в голову уже на лестнице.

-- Неужели  ты  столь  не  находчив?  -- спросила я, сосредоточивая все
усилия воли на своем лице и ушах, потому что меня уже злил тот  неуступчивый
румянец,   который   мешал   мне   чувствовать   себя  независимой,  ведь  я
догадывалась, что и он был частью того же замысла, с которым король  отдавал
меня моему предназначению.

-- Может  быть, мне следует добавить:  ``Нет ли средства этому  помочь?'' --
продолжала я, --  а ты ответишь, что все бессильно  перед лицом красоты, чье
совершенство  способно  подтвердить  существование  Абсолюта.  Тогда  бы  мы
посерьезнели на два такта оркестра и  с надлежащей ловкостью выбрались бы на
обычную придворную почву.  Но она, мне кажется, тебе чужда,  и, пожалуй, нам
лучше так не разговаривать...

Только  теперь,  когда  он  услышал  эти  слова, он меня испугался -- и
по-настоящему и теперь вправду не знал,  что  сказать.  У  него  были  такие
глаза,  будто нас обоих подхватило вихрем и несет из этой залы неведомо куда
-- в пустоту.

-- Кто ты? -- спросил он жестко. От игры, от волокитства не осталось ни
следа -- только страх. А я совсем -- вот ни  капельки  --  его  не  боялась,
хотя, собственно, должна бы испугаться ощущения, что его лицо, эта угреватая
кожа,  строптиво  взъерошенные  брови,  большие оттопыренные уши сверяются с
каким-то заключенным во мне ожиданием; накладываясь, совпадают словно  бы  с
негативом,  который  я  носила  в  себе непроявленным и который сейчас вдруг
начал пропечатываться. Я не боялась его -- даже если в нем был мой приговор.
Ни себя, ни его. Однако сила, которую  это  совпадение  освободило  во  мне,
заставила  меня  вздрогнуть.  И я вздрогнула, но не как человек, а как часы,
когда их стрелки сошлись и пружина стронулась,  чтобы  пробить  полночь,  но
первый удар еще не раздался. Этой дрожи не мог заметить никто.

-- Кто  я,  ты  узнаешь  чуть  позднее,  --  ответила я очень спокойно,
раскрыла веер и улыбнулась легкой бледной улыбкой, какими ободряют больных и
слабых. -- Я бы выпила вина, а ты?

Он кивнул, силясь напялить на себя светскую оболочку, которая была  ему
не   по  нутру  и  не  по  плечу,  и  мы  пошли  по  паркету,  забрызганному
перламутровыми потеками воска, стекавшего с люстры, словно капель, через всю
залу, рука об руку -- туда, где у стены лакеи разливали вино в бокалы.

В ту ночь я не сказала ему, кто я, потому что не хотела лгать, а истины
не знала сама. Истина может быть лишь одна, а я была и дуэнья, и графиня,  и
сирота -- все эти судьбы кружились во мне, и каждая могла бы стать истинной,
признай  я  ее  своей;  я  уже  понимала,  что  в  конце  концов  мою истину
предопределит мой каприз и та, которую я  выберу,  сдунет  остальные,  но  я
продолжала  колебаться  между этими образами, потому что мне мерещился в них
какой-то сбой памяти.  Скорее  всего,  я  была  молодой  особой,  страдавшей
расщеплением  личности,  и  мне на время удалось вырваться из-под заботливой
опеки  близких.  Продолжая  разговор,  я  думала,  что  если  я  и   вправду
сумасшедшая,  то  все  кончится  благополучно,  ведь из помешательства можно
выйти, как из сна, -- и тут, и там есть надежда.

В поздний час, когда  мы вместе (а он не отступал от меня  ни на шаг) прошли
рядом с его величеством за минуту до того, как король вознамерился удалиться
в свои  апартаменты, я обнаружила,  что повелитель  даже не взглянул  в нашу
сторону, и это было страшное открытие. Он не проверял, так ли я держу себя с
Арродесом, видимо,  это было не нужно,  видимо, он не сомневался,  что может
полностью мне доверять,  как доверяют подосланным тайным  убийцам, зная, что
они не  отступят до последнего своего  вздоха, ибо их судьба  всецело в руке
пославшего. Но могло  быть и так, что королевское  равнодушие должно стереть
мои подозрения --  раз он не смотрит в мою  сторону, значит, я действительно
ничего для него не значу, и оттого мои навязчивые домыслы опять склонялись к
мысли о сумасшествии. И вот я, безумная и ангельски прекрасная, попиваю вино
и  улыбаюсь Арродесу,  которого король  ненавидит как  никого другого,  - --
однако  он  поклялся матери  в  ее  смертный час,  что  если  злая участь  и
постигнет  этого  мудреца, то  только  по  собственной  его воле.  Не  знаю,
рассказал ли  мне это кто-нибудь  во время танца, или  я это узнала  от себя
самой, ведь ночь была  такая длинная и шумная, огромная толпа  то и дело нас
разлучала, а  мы вновь  находили друг друга,  неумышленно, словно  все здесь
были замешаны в  этом заговоре, -- очевидный бред: не  кружились же мы среди
механически  танцующих манекенов!  Я разговаривала  со старцами  и девицами,
завидовавшими  моей красоте,  различала бесчисленные  оттенки благоглупости,
столь скорой  на зло. Я рассекала  и прошивала этих ничтожных  честолюбцев и
этих девчонок с такой легкостью, что мне даже становилось их жаль.

Казалось, я была воплощением отточенного  разума -- я блистала остроумием, и
оно добавляло блеска моим глазам, хотя  из-за тревоги, которая росла во мне,
я охотно притворилась бы дурочкой, чтобы  спасти Арродеса, но именно этого я
как раз и не  могла. Увы, я была не столь всесильна. Был  ли мой разум, сама
его  безупречность подвластны  лжи? Вот  над чем  билась я  во время  танца,
выделывая фигуры менуэта, пока Арродес, который не танцевал, смотрел на меня
издали, черный и  худой на фоне пурпурной, расшитой  львами парчи занавесей.
Король удалился, а  вскоре распростились и мы.  Я не позволила ему  ни о чем
спрашивать, а  он все пытался  что-то сказать  и бледнел, когда  я повторяла
``нет'' сначала губами, потом, только  сложенным веером. Выходя из дворца, я
не  знала, ни  где живу,  ни откуда  пришла, ни  куда направляюсь,  -- знала
только,  что этого  мне  не дано,  --  все мои  попытки  что-то узнать  были
напрасны: каждому известно,  что нельзя повернуть глазное  яблоко так, чтобы
зрачок заглянул внутрь черепа.

Я позволила Арродесу проводить меня до дворцовых  ворот:  позади  круга
все  еще  пылавших  бочек  со смолой был парк, будто высеченный из угля, а в
холодном воздухе носился далекий нечеловеческий смех -- то ли эти  жемчужные
звуки  издавали  фонтаны  работы  южных  мастеров,  то  ли болтающие статуи,
похожие на белесые маски, подвешенные над клумбами. Королевские соловьи тоже
пели, хотя слушать их было некому, вблизи оранжереи один из  них  чернел  на
огромном диске луны, словно нарисованный. Гравий хрустел у нас под ногами, и
золоченые острия ограды шеренгой торчали из мокрой листвы.

Он  торопливо  и зло схватил меня за руку, которую я не успела вырвать,
рядом засияли белые полосы на эполетах  гренадеров  его  величества,  кто-то
вызывал мой выезд, кони били копытами, фиолетовые отсветы фонариков блеснули
на дверце кареты, упала ступенька. Это не могло быть сном.

-- Когда и где? -- спросил он.

-- Лучше  никогда  и  нигде,  -- сказала я свою главную правду и тут же
быстро и беспомощно добавила: -- Я не шучу,  приди  в  себя,  мудрец,  и  ты
поймешь, что я даю тебе добрый совет.

То, что я хотела произнести дальше,  мне уже не удалось выговорить. Это было
так странно: думать я могла все, что  угодно, но голос не выходил из меня, я
никак не могла добраться до тех  слов. Хрип, немота -- будто ключ повернулся
в замке и засов задвинулся между нами.

-- Слишком поздно, -- тихо сказал он, опустив голову, -- на самом деле,
поздно.

-- Королевские сады открыты от утреннего до полуденного сигнала.  --  Я
поставила ногу на ступеньку кареты. -- Там, где пруд с лебедями, есть старый
дуб.  Завтра,  точно в полдень, ты найдешь в дупле записку, а сейчас я желаю
тебе, чтобы ты каким-нибудь немыслимым чудом забыл, что мы встречались. Если
бы я знала как, то помолилась бы за это.

Не к месту было говорить это при страже. И слова были банальные, и  мне
не  дано  было  вырваться  из  этой смертельной банальности -- я это поняла,
когда карета уже покатилась, а он ведь мог истолковать мои слова так,  будто
я  боюсь чувства, которое он во мне пробудил. Так и было: я боялась чувства,
которое он возбудил во мне, однако оно не имело ничего общего с любовью, а я
говорила то, что могла сказать, словно пробовала, как во  тьме,  на  болоте,
пробуют  почву под ногой, не заведет ли следующий шаг в трясину. Я пробовала
слова, нащупывая дыханием те, что мне  удастся  вымолвить,  и  те,  что  мне
сказать  не  дано.  Но  он не мог этого знать. Мы расстались ошеломленные, в
тревоге, похожей  на  страсть,  ибо  так  начиналась  наша  погибель.  И  я,
прелестная,  нежная, неискушенная, все же яснее, чем он, понимала, что я его
судьба в полном, страшном и неотвратимом значении этого слова.

Коробка кареты была  пуста. Я поискала тесьму, пришитую к  рукаву кучера, но
ее  не было.  Окон тоже  не  было, может  быть, черное  стекло? Мрак  внутри
был  такой  полный,  что,  казалось,  принадлежал не  ночи,  а  пустоте.  Не
просто  отсутствие света,  а  ничто.  Я шарила  руками  по вогнутым,  обитым
плюшем  стенкам,  но не  нашла  ни  оконных  рам,  ни ручки,  ничего,  кроме
изогнутых, мягко выстланных поверхностей передо мной и надо мной; крыша была
удивительно  низкая,  словно  меня  захлопнули  не  в  кузове  кареты,  а  в
трясущемся наклонном футляре. Я не слышала  ни топота копыт, ни обычного при
езде стука колес.  Чернота, тишина и ничто. Тогда я  сосредоточилась на себе
-- для себя я  была более опасной загадкой, чем все,  что со мной произошло.
Память была безотказна. Мне  казалось, что все так и должно  быть и не могло
произойти  иначе:  я помнила  мое  первое  пробуждение  --  когда я  еще  не
имела пола,  -- как  чье-то чужое,  как преследующий  меня кошмарный  сон. Я
помнила  и пробуждение  в дверях  дворцовой залы,  когда я  была уже  в этой
действительности, помнила  даже легкий скрип, с  которым распахнулись резные
двери, и застывшее лицо лакея, служебным рвением превращенного в исполненную
почтения куклу,  живой восковой  труп. Теперь  все мои  воспоминания слились
воедино, но я могла в мыслях вернуться вспять, туда, где я не знала еще, что
такое --  двери, что -- бал  и что -- я.  Меня пронзила дрожь, оттого  что я
вспомнила, как первые  мои мысли, еще лишь наполовину облеченные  в слова, я
выражала в формах  другого рода -- ``сознавало'',  ``видело'', ``вошло'', --
вот как было, пока  блеск залы, хлынув в распахнутые двери,  не ударил мне в
зрачки и  не открыл  во мне  шлюзы и клапаны,  сквозь которые  с болезненной
быстротой  влилось  в  меня  человеческое знание  слов,  придворных  жестов,
обаяние надлежащего  пола и вкупе  с ними --  память о лицах,  среди которых
первым было  лицо Арродеса,  а вовсе  не королевская  гримаса. И  хотя никто
никогда не  смог бы мне в  точности этого объяснить, я  теперь была уверена,
что перед  королем остановилась по  ошибке -- я  перепутала предназначенного
мне с  тем, от  кого предназначение  исходило. Ошибка...  но если  так легки
ошибки -- значит, эта судьба не истинная, и я могу еще спастись?

Теперь,  в  полном  уединении,  которое  вовсе  не  тревожило  меня, а,
напротив, было даже удобно, ибо  позволяло  мне  спокойно  и  сосредоточенно
подумать,  когда я попыталась познать, кто я, вороша для этого воспоминания,
такие доступные -- каждое на своем месте,  под  рукой,  как  давно  знакомая
утварь  в  старом жилище, я видела все, что произошло этой ночью, но резко и
ясно -- только от порога дворцовой залы.

А прежде? Где я была?  Или  было?!  Прежде?  Откуда  я  взялась?  Самая
простая  и  успокаивающая мысль подсказывала, что я не совсем здорова, что я
возвращаюсь  из  болезни,  как   из   экзотического,   полного   приключений
путешествия,   --   тонкая,   книжная  и  романтическая  девушка,  несколько
рассеянная, со странностями. Оттого что я слишком хрупка для  этого  грубого
мира,  мною  овладели навязчивые видения, и, видно, в горячечном бреду, лежа
на кровати с балдахином, на простынях, обшитых кружевами, я вообразила  себе
путешествие  через  металлический ад, а мозговая горячка была мне, наверное,
даже к лицу -- в блеске свечей, так озаряющих альков, чтобы, когда я очнусь,
ничто меня не испугало и чтобы в фигурах, склонившихся надо мной, я сразу бы
узнала неизменно любящих меня попечителей... Что за  сладкая  ложь!  У  меня
были  Видения,  не  так  ли?  И  они, вплавившись в чистый поток моей единой
памяти, расщепили ее. Расщепили?.. Да,  спрашивая,  я  слышала  в  себе  хор
ответов,  готовых,  ожидающих: дуэнья, Тленикс, Ангелита. Ну и что из этого?
Все эти имена были во мне готовы, мне даны, и каждому  соответствовали  даже
образы,  как  бы  единая  их  цепь. Они сосуществовали так, как сосуществуют
корни, расходящиеся от дерева, и я, без  сомнения,  единственная  и  единая,
когда-то  была  множеством  разветвлений,  которые слились во мне, как ручьи
сливаются в речное русло.

"Не могло быть так, -- сказала я себе. -- Не может быть, я уверена". Но
я же видела мою  предыдущую  судьбу  разделенной  на  две  части:  к  порогу
дворцовой  залы  тянулось  множество  нитей  -- разных, а от порога -- одна.
Картины первой части моей судьбы жили отдельно друг от друга  и  друг  друга
отвергали.  Дуэнья:  башня, темные гранитные валуны, разводной мост, крики в
ночи, кровь на медном блюде, рыцари с рожами мясников, ржавые лезвия алебард
и мое личико в овальном подслеповатом зеркале, висевшем между рамой  мутного
окна из бычьего пузыря и резным изголовьем. Может быть, я пришла оттуда?

Но  как  Ангелита  я  росла  среди  южного  зноя,  и, глядя назад в эту
сторону, я видела белые дома, повернувшиеся к солнцу  известковыми  спинами,
чахлые  пальмы, диких собак, поливающих пенящейся мочой их чешуйчатые корни,
и корзины, полные фиников, слипшихся в клейкую сладкую  массу,  и  врачей  в
зеленых  одеяниях,  и  лестницы,  каменные  лестницы  спускающегося к заливу
города, всеми стенами отвернувшегося от зноя, и кучи виноградных гроздьев, и
рассыпанные засыхающие изюмины, похожие на козий помет. И снова мое  лицо  в
воде  --  не  в зеркале: вода лилась из серебряного кувшина, потемневшего от
старости. Я помню даже, как носила этот кувшин, и вода, тяжело  колыхаясь  в
нем, оттягивала мне руку.

А как же  мое "оно", лежащее навзничь, и то  путешествие и поцелуи подвижных
металлических  змей, проникающие  в мои  руки, тело,  голову, --  этот ужас,
который настолько теперь потускнел, что вспомнить его я могла лишь с трудом,
как дурной сон, не передаваемый словами?  Не могла я пережить столько судеб,
одна другой противоречащих, --  ни все сразу, ни одну за  другой! Так что же
истинно? Моя красота. Отчаяние и торжество  -- равно ощутила я, увидев в его
лице, как в зеркале, сколь беспощадно совершенство этой красоты. Если бы я в
безумии завизжала, брызгая пеной, или стала бы рвать зубами сырое мясо, то и
тогда мое лицо осталось бы прекрасным, -- но почему я подумала "мое лицо", а
не просто "я"? Почему  я с собой в раздоре? Что я  за существо, не способное
достичь единства со своим телом и  лицом? Колдунья? Медея? Но подумать такое
-- уже совершенная несуразица. Мысль моя  работала как источенный меч в руке
рыцаря с  большой дороги,  которому нечего  терять, и  я легко  рассекала ею
любой предмет, но эта моя  способность тоже показалась мне подозрительной --
своим совершенством, чрезмерной холодностью,  излишним спокойствием, ибо над
моим разумом был  страх: и этот страх существовал вне  разума -- вездесущий,
невидимый --  сам по себе, и  это значило, что  я не должна была  доверять и
своему разуму тоже. И  я не стала верить ни лицу своему,  ни мысли своей, но
страх остался -- вне их. Так против чего же он направлен, если помимо души и
тела  нет  ничего?  Такова  была  загадка. А  мои  предыстории,  моя  корни,
разбегавшиеся в прошлом, ничего мне не подсказывали: их ощупывание было лишь
пустой перетасовкой одних и тех  же красочных картинок. Северянка ли дуэнья,
южанка Ангелита или Миньона -- я всякий раз оказывалась другим персонажем, с
другим именем, с  другим положением, другой семьей. Ни одна  из них не могла
возобладать над  прочими. Южный  пейзаж каждый раз  возникал в  моей памяти,
переслащенный театральным  блеском торжественной  лазури, и  если бы  не эти
шелудивые  псы  и  не  полуслепые  дети  с  запекшимися  веками  и  вздутыми
животиками,  беззвучно умирающие  на костлявых  коленях закутанных  в черное
матерей, это  пальмовое побережье показалось бы  слишком гладким, скользким,
как ложь.  А север моей  дуэньи: башни  в снеговых шапках,  бурое клубящееся
небо и особенно зимы -- снеговые фигуры на кручах, выдумки ветра, извилистые
змеи  поземки, ползущей  из рва  по контрфорсам  и бойницам,  белыми озерами
растекающейся на скале  у подножия замка, и цепи  подъемного моста, плачущие
ржавыми слезами сосулек. А летом --  вода во рву, которая покрывалась ряской
и плесенью, -- как хорошо я все это помнила!

Но было же и третье прошлое: большие, чопорные подстриженные сады, садовники
с  ножницами,  своры борзых  и  черно-белый  дог,  как арлекин  на  ступенях
трона, скучающая скульптура  -- лишь движение ребер  нарушало его грациозную
неподвижность,  да  в  равнодушных  желтых  глазах  поблескивали,  казалось,
уменьшенные отражения  катарий или  некроток. И  эти слова  -- ``некротки'',
``катарии''  -- сейчас  я  не знала,  что они  значили,  но когда-то  должна
была  знать.  И  теперь,  вглядываясь  в  это  прошлое,  забытое,  как  вкус
изжеванного  стебелька, я  чувствовала, что  не должна  возвращаться в  него
глубже -- ни к туфелькам, из  которых выросла, ни к первому длинному платью,
вышитому  серебром, будто  бы и  в ребенке,  которым я  когда-то была,  тоже
спрятано предательство.  Оттого я вызвала  в памяти самое чуждое  и жестокое
воспоминание --  как я, бездыханная, лежа  навзничь, путешествовала, цепенея
от поцелуев  металла, издававшего,  когда он  касался моего  тела, лязгающий
звук, словно оно было безмолвным колоколом, который не может зазвенеть, пока
в нем нет сердца. Да, я возвращалась в невероятное -- в бредовый кошмар, уже
не удивляясь тому,  как прочно он засел  в моей памяти, --  конечно, это мог
быть только бред, и, чтобы поддержать  в себе эту уверенность, я робко стала
ощупывать, только самыми кончиками пальцев, свои мягкие предплечья, грудь --
без сомнения, то было наитие,  которому я поддавалась, дрожа, будто входила,
запрокинув голову, под ледяные струи отрезвляющего дождя.

Нигде не  было ответа, и я  попятилась от этой бездны  -- моей и не  моей. И
тогда я вернулась к тому, что  тянулось уже единой нитью. Король, вечер, бал
и тот мужчина. Я сотворена для него, он -- для меня, я знала это, и снова --
страх.  Нет, не  страх, а  ощущение рока,  чугунной тяжести  предназначения,
неизбежного,  неотвратимого: знание,  подобное предчувствию  смерти, знание,
что  уже нельзя  отказаться,  уйти, убежать,  даже  погибнуть, --  погибнуть
иначе. Я тонула  в этом леденящем предчувствии, оно душило  меня. Не в силах
вынести  его,  я повторяла  одними  губами:  "отец, мать,  родные,  подруги,
близкие"; я прекрасно понимала смысл этих слов, и они послушно воплощались в
знакомые фигуры:  мне приходилось признавать  их своими, но нельзя  же иметь
четырех матерей и столько же отцов сразу  -- опять этот бред, такой глупый и
такой назойливый!  Наконец я прибегла к  арифметике: один и один  -- два, от
отца и  матери рождаются дети  -- ты была  ими всеми, это  память поколений.
"Нет, либо я  прежде была сумасшедшей, --  сказала я себе, --  либо я больна
сейчас, и, хоть я  и в сознании, душа моя помрачена. И  не было бала, замка,
короля,  вступления  в  мир,  который бы  подчинялся  заранее  установленной
гармонии". Правда,  я тотчас ощутила горечь  от мысли, что если  так, я буду
вынуждена распроститься  с моей  красотой. Что ж,  из элементов,  которые не
подходят друг другу,  я ничего не построю -- разве  только найду в постройке
перекос, протиснусь в трещины и раздвину  их, чтобы войти внутрь. Вправду ли
все произошло так,  как должно было случиться? Если  я собственность короля,
то как я могла  об этом знать? Ведь мысль об этом даже  и во сне должна быть
для меня запретной.  Если за всем этим  стоит он, то почему,  когда я хотела
ему  поклониться,  я  поклонилась  не  сразу?  И  если  все  готовилось  так
тщательно, то  почему я помню то,  чего мне не следует  помнить? Отзовись во
мне только  одно мое прошлое,  девичье и детское, я  не впала бы  в душевный
разлад, который  вел к  отчаянию, а  затем --  к бунту  против судьбы.  И уж
наверняка  надлежало  стереть воспоминание  о  том  путешествии навзничь,  о
себе  безжизненной и  о себе  оживающей от  искровых поцелуев,  о безмолвной
наготе, но  и это  тоже осталось  и было сейчас  во мне.  Не закралось  ли в
замысел и в исполнение некое  несовершенство? Небрежность, рассеянность и --
непредвиденные утечки, которые теперь принимаются за загадки или дурной сон?
Но  в таком  случае была  надежда. Ждать,  чтобы в  дальнейшем осуществлении
замысла  нагромоздились  новые несообразности,  чтобы  обратить  их в  жало,
нацеленное на  короля, на  себя, все  равно на кого  -- только  бы наперекор
навязанной судьбе. А  может быть, поддаться колдовству, жить в  нем, пойти с
самого  утра на  условленное свидание  -- я  знала, что  ЭТОГО мне  никто не
воспретит, наоборот, все  будет направлять меня именно туда. А  то, что было
сейчас  вокруг меня,  раздражало  своей примитивностью  -- какие-то  стенки:
сначала обивка, мягко поддающаяся под  пальцами, под ней сопротивление стали
или камня  -- не  знаю чего,  но ведь  я могу  разодрать ногтями  эту уютную
упаковку!.. Я встала, коснулась головой  вогнутой крыши: вот что вокруг меня
и надо мной, и вот внутри -- я... Я -- единая?..

Я  продолжала отыскивать  противоречия  в мучительном  моем самопознании,  и
по  мере  того,  как  мысли  скачками надстраивались,  этаж  над  этажом,  я
приблизилась уже к тому, что пора усомниться  и в самом суждении, что если я
-- безумная русалка, заключенная, как насекомое, в прозрачном янтаре, в моем
obnubulacio  lucida\footnote{Здесь: явное  помрачение рассудка  (лат.).}, то
понятно, что...

Постой.  Минутку.  Откуда  взялась  у  меня  такая изысканно отточенная
лексика, эти ученые латинские термины, логические посылки,  силлогизмы,  эта
изощренность,   не   свойственная  очаровательной  девушке,  чье  назначенье
воспламенять  мужские  сердца?  И  откуда  это  равнодушие  в  делах  любви,
рассудочность,  отчужденность:  ведь  меня  любили  -- наверное, уже бредили
мною, жаждали видеть, слышать мой голос, коснуться моих пальцев, а я изучала
эту страсть, как препарат под стеклом, -- не  правда  ли,  это  удивительно,
противоречиво  и несинкатегориматично? Но может, мне все только пригрезилось
и  конечной  истиной  был  старый  холодный  мозг,  запутавшийся   в   опыте
бесчисленных   лет?  И  может,  одна  только  обостренная  мудрость  и  была
единственным моим настоящим прошлым:  я  возникла  из  логики,  и  лишь  она
творила мою истинную генеалогию?..

И я не верила  в это. Да, я страшно виновна и вместе  с тем невинна. Во всех
ветвях моего завершенного прошлого,  сбегавшихся к моему единому настоящему,
я  была невинна  --  там я  была девочкой,  хмурым  молчаливым подростком  в
серо-седых зимах и  в жаркой духоте дворцов;  я была неповинна и  в том, что
произошло здесь, у короля,  потому что я не могла быть  иной; а жестокая моя
вина  состояла  только в  том,  что,  уже  во  всем хорошо  разобравшись,  я
уверила  себя, что  все это  мишура,  фальшь, накипь,  и в  том, что,  желая
погрузиться в глубь своей тайны,  я испугалась этого погружения и испытывала
подлую благодарность  к невидимым  препятствиям, которые удерживали  меня от
него.  Душа моя  была одновременно  грешной и  праведной, но  что-то у  меня
еще  осталось?  О, конечно,  осталось.  У  меня было  мое  тело,  и я  стала
ощупывать его, исследовать в этом черном замкнутом пространстве, как опытный
криминалист  изучает  место  преступления. Странное  расследование!  Отчего,
прикасаясь к своему  телу, я ощущала в пальцах легкое  щекочущее онемение --
кажется, это был мой страх перед собой?  Но я же была прекрасна, и мои мышцы
были проворны и пружинисты. Сжав руками  свои бедра, словно они были чужими,
-- так никто себе их не сжимает, -- в отчаянном усилии, я смогла под гладкой
ароматной  кожей прощупать  кости, но  внутренней стороны  предплечий --  от
локтей до запястий -- я почему-то боялась коснуться.

Я попыталась одолеть сопротивление: что же могло там быть? Руки у  меня
были  закрыты  жесткими кружевными рукавами -- ничего не разобрать. Тогда --
шея... Такие называют лебедиными. Голова, посаженная  на  ней  с  врожденной
естественной   грацией,   с   гордостью,  внушающей  почтение,  мочки  ушей,
полуприкрытых локонами, -- два упругих лепестка без украшений,  непроколотые
-- почему?  Я  касалась  лба, щек, губ. Их выражение, открытое мне кончиками
пальцев, снова меня обеспокоило. Оно было не таким, как мне  представлялось.
Чужим. Но отчего я могла быть чужой для себя, как не от болезни?

Исподтишка,  как  маленький  ребенок,  замороченный  сказками, я все же
провела пальцами от запястья к локтю -- и ничего не поняла. Кончики  пальцев
сразу  онемели, будто мои сосуды и нервы что-то стиснуло, я тотчас вернулась
к прежним подозрениям: откуда я все знаю, зачем исследую себя,  как  анатом?
Это  не  дело  девушки:  ни  Ангелиты, ни светловолосой дуэньи, ни поэтичной
Тленикс. И в то же время я ощутила настойчивое успокаивающее внушение:  "Все
хорошо,  не  удивляйся  себе,  капризуля,  ты  была  немножко  не в себе, не
возвращайся туда, выздоравливай, думай лучше о назначенном  свидании..."  Но
все  же,  что  там  --  где  локти и запястья?.. Я нащупала под кожей как бы
твердый  комочек.  Набухший  лимфатический  узел?   Склеротическая   бляшка?
Невозможно.   Это   не   вязалось   с   моей  красотой,  с  ее  непогрешимым
совершенством. Но ведь затвердение там было: маленькое -- я его  прощупывала
только  при  сильном нажиме -- там, где щупают пульс, и еще одно -- на сгибе
локтя.

Значит,   у  моего   тела  была   своя  тайна,   и  оно   своей  странностью
соответствовало странности духа,  его страхам и самоуглубленности,  и в этом
была  правильность, соответствие,  симметрия.  Если там,  то  и здесь.  Если
разум, то  и органы. Если я,  то и ты... Я  и ты... Всюду загадки  -- я была
измучена, сильная  усталость разлилась  по моему  телу, и  я должна  была ей
подчиниться. Уснуть,  впасть в  забытье -- в  другой, освобождающий  мрак. И
тут  меня вдруг  пронизала решимость  назло всему  устоять перед  соблазном,
воспротивиться заключавшему меня ящику этой изящной кареты -- кстати, внутри
не столь уж изящной, -- и  этой душонке рассудительной девицы, вдруг слишком
далеко зашедшей  в своем умничанье!  Протест против воплощенной  красоты, за
которой  скрываются  тайные  стигматы.  Так  кто  же  я?  Сопротивление  мое
переросло в буйство, в бешенство, от  которого моя душа горела во мраке так,
что он, казалось, начал светлеть.  Sed tamen potest esse totaliter aliter...
-- что это,  откуда? Дух мой? Gratia? Dominus  meus?\footnote{Однако в целом
все может быть иным... Благодать? Мой Бог? (лат.).}

Нет,  я  была  одна,  и я -- единая, сорвалась с места, чтобы ногтями и
зубами впиться в эти мягко устланные стены, рвала обивку, ее сухой,  жесткий
материал  трещал  у  меня в зубах, я выплевывала волокна вместе со слюной --
ногти сломаются, ну и ладно, вот так, не знаю, против  кого,  себя  или  еще
кого-то, только нет, нет, нет, нет...

Что-то  блеснуло. Передо мной вынырнула из тьмы как бы змеиная головка,
но она  была  металлической.  Игла?  Да,  что-то  укололо  меня  в  бедро  с
внутренней  стороны, повыше колена: это была слабая недолгая боль, укол -- и
за ним ничто.

Ничто.

Сумрачный сад. Королевский парк с поющими фонтанами, живыми изгородями,
подстриженными на один манер, геометрия деревьев и кустов, лестницы, мрамор,
раковины, амуры. И мы вдвоем.  Банальные,  обыкновенные,  но  романтичные  и
полные  отчаяния.  Я  улыбалась ему, а на бедре носила знак. Меня укололи. И
теперь мой  дух,  против  которого  я  бунтовала,  и  тело,  которое  я  уже
ненавидела, получили союзника, -- правда, он оказался недостаточно искусным:
сейчас  я  уже не боялась его, а просто играла свою роль. Конечно, он все же
был настолько искусен, что сумел навязать мне ее изнутри, прорвавшись в  мою
твердыню.  Но  искусен  не совсем -- я видела его сети. Я не понимала еще, в
чем цель, но я уже ее увидела, почувствовала, а тому, кто увидел, уже не так
страшно, как тому, кто вынужден жить одними домыслами. Я так устала от своих
метаний, что даже белый день раздражал меня своей пасмурной торжественностью
и панорамой садов, предназначенных  для  лицезрения  его  величества,  а  не
зелени. Сейчас я предпочла бы этому дню ту мою ночь, но был день, и мужчина,
который  ничего  не  знал,  ничего  не  понимал,  жил  обжигающей  сладостью
любовного  помешательства,  наваждением,  насланным  мною  --  нет,   кем-то
третьим. Силки, западня, ловушка со смертельным жалом, и все это -- я? И для
этого  --  струи  фонтанов,  королевские  сады, туманные дали? Глупо. О чьей
погибели речь, о чьей смерти? Разве  не  достаточно  подставных  свидетелей,
старцев в париках, виселицы, яда? Что же ему еще? Отравленные интриги, какие
подобают королям?

Садовники  в  кожаных   фартуках,  поглощенные  картинами  всемилостивейшего
монарха, нас  не замечали.  Я молчала --  так мне было  легче. Мы  сидели на
ступенях огромной  лестницы, сооруженной  будто для гиганта,  который сойдет
когда-нибудь  с заоблачных  высот только  для того  -- специально,  -- чтобы
воспользоваться ею. Символы, втиснутые в  нагих амуров, фавнов, силенов -- в
осклизлый, истекающий водой мрамор, -- были  так же мрачны, как и серое небо
над ними. Идиллическая пара -- прямо Лаура и Филон, но столько же здесь было
и от Лукреции!

...Я очнулась здесь, в этих королевских  садах, когда моя карета отъехала, и
пошла легко, как  будто только что выпорхнула из  ванны, источающей душистый
пар, и  платье на мне было  уже другое, весеннее, своим  затуманенным узором
оно  робко напоминало  о цветах,  намекало на  девичью честь,  окружало меня
неприкосновенностью  Eos Rhododaktilos\footnote{Розовоперстой  Эос (греч.)},
но я шла среди  блестящих от росы живых изгородей уже с  клеймом на бедре, к
которому не  могла прикоснуться, да в  этом и не было  нужды, довольно того,
что оно  не стиралось в памяти.  Я была плененным разумом,  закованным уже с
пеленок, рожденным в неволе, и все-таки разумом. И поэтому, пока мой суженый
еще не  появился и поблизости  не было  ни чужих ушей,  ни той иглы,  я, как
актриса перед  выходом на  сцену, пыталась пробормотать  про себя  те слова,
которые хотела  сказать ему, и  не знала, удастся  ли мне их  произнести при
нем, -- я пробовала границы своей свободы, ощупью исследуя их при свете дня.

Что  особенного  было  в этих словах? Только правда: сначала о перемене
грамматической формы, потом -- о множестве моих плюсквамперфектов, обо всем,
что я пережила, и о жале, усмирившем мой бунт. Отчего  я  хотела  рассказать
ему  все  -- из сострадания, чтобы не погубить его? Нет, ибо я его совсем не
любила. Но чтобы предать чужую, злую волю, которая нас  свела.  Ведь  так  я
скажу? Что хочу, пожертвовав собой, избавить его от себя -- как от погибели?

Нет, все  было иначе. Была  еще и  любовь -- я  знаю, что это  такое. Любовь
пламенная, чувственная и в то же время пошленькая -- желание отдать ему душу
и  тело лишь  постольку, поскольку  этого требовал  дух моды,  обычай, стиль
придворной жизни, -- о, как-никак, а все же чудесный галантный грешок! Но то
была  и  очень большая  любовь,  вызывающая  дрожь, заставляющая  колотиться
сердце, я знала, что  один вид его сделает меня счастливой. И  в то же время
-- любовь  очень маленькая, не  преступающая границ, подчиненная  стилю, как
старательно приготовленный урок, как этюд на выражение мучительного восторга
от встречи наедине. И не это чувство  побуждало меня спасать его от меня или
не только  от меня, ибо,  когда я переставала  рассуждать о своей  любви, он
становился мне совершенно безразличен, зато мне нужен был союзник в борьбе с
тем, кто ночью вонзил в меня ядовитый  металл. У меня никого больше не было,
а он  был мне предан  безоглядно, и я могла  на него рассчитывать.  Однако я
знала, что он  пойдет на все лишь  ради своей любви ко мне.  Ему нельзя было
доверить мой reservatio  mentalis\footnote{Мысленный тайник (лат.).}. Оттого
я и не могла сказать  ему всей правды: что я моя любовь к  нему, и яд во мне
--  из одного  источника.  И потому  мне мерзки  оба,  и предназначивший,  и
предназначенный, и я обоих ненавижу и обоих хочу растоптать, как тарантулов.
Не могла я ему этого выдать: он-то в своей любви, конечно, был как все люди,
и ему  не нужно было  такое мое освобождение,  которого жаждала я,  -- такой
моей  свободы, которая  сразу отбросила  бы его  прочь. Я  могла действовать
только  ложью --  называть свободу  фальшивым именем  любви, ибо  только так
можно его убедить,  что я -- жертва  неведомого. Короля? Но даже  если бы он
посягнул на его величество, это бы меня  не освободило: король если и был на
самом деле виновником всему, то таким давним,  что его смерть ни на вело: не
отдалила бы моего несчастья. Чтобы проверить себя, способна ли я убеждать, я
остановилась у статуи Венеры Каллипиги,  чья нагота воплотила в себе символы
высших и  низших страстей земной  любви, и принялась в  одиночестве готовить
свою  чудовищную  весть,  мои  обличения,  оттачивая  доводы  до  кинжальной
остроты.

Мне  было  очень  трудно.  Я  все  время  натыкалась  на  непреодолимую
преграду,  я не знала, когда мой язык сведет судорога, на чем споткнется мой
дух, потому что и дух мой тоже был моим врагом. Не во всем лгать,  но  и  не
касаться  сути истины, средоточия тайны... Я лишь могла постепенно уменьшать
ее радиус, приближаясь как бы по спирали. Но когда я увидела издали, как  он
шел,  а  потом  почти  побежал  ко  мне  --  маленькая  еще фигурка в темной
пелерине, -- я поняла, что ничего не выйдет: в рамках галантного  стиля  мне
не  удержаться. Что это за любовная сцена, в которой Лаура признается Филону
в том, что она -- приготовленное для него орудие пытки? Даже если  бы  путем
иносказаний  я преодолела бы мое заклятие, все равно бы я снова обратилась в
ничто, из которого возникла. И вся  его  мудрость  была  здесь  ни  к  чему.
Прелестная  дева,  которая  считает  себя  орудием  тайных  сил и бормочет о
каких-то системах, о стигматах, о заклятиях, да если она  говорит  так  и  о
таких  вещах, то, право, эта девица помешана. Ее слова свидетельствуют не об
истине, а лишь о галлюцинациях, и потому она  достойна  не  только  любви  и
преданности,  но  и  жалости.  Движимый  этими  чувствами, он, может быть, и
сделает вид, будто поверил всему, что услышал, опечалится,  станет  уверять,
что  готов  погибнуть, но освободить, а сам кинется за советами к докторам и
по всему свету разнесет весть о моей беде, -- я уже сейчас готова  была  его
оскорбить.  При  таком  сочетании сил, конечно же, чем надежнее союзник, тем
меньше он может рассчитывать на исполнение  надежд  как  любовник:  во  имя,
своего  счастья  он  наверняка не захочет отказаться от роли любовника, ведь
его-то безумие нормальное, крепкое, солидное, последовательное: любить,  ах,
любить,  острые  камни  на моем пути раздробить в мягкий песок, но только не
играть в анализ чудовищной загадки -- "откуда берет начало мой дух"?

И получалось, что  если  я  создана  ему  на  погибель,  то  он  должен
погибнуть.  Я  не  знала,  какая  часть  меня  ужалит  его в объятии: локти?
запястья? -- это было бы слишком просто. Но я уже знала, что иначе  быть  не
может.  Теперь  мне  надо  было  пойти  с ним по тропинкам, услаждающим взор
творениями мастеров паркового искусства; мы сразу  же  удалились  от  Венеры
Каллипиги,  ибо  откровенность,  с которой она выставляла напоказ свою суть,
была неуместна на раннеромантической стадии наших  платонических  вздохов  и
робких  надежд  на  счастье.  Мы прошли мимо фавнов, тоже откровенных, но на
свой лад -- каменная плоть этих кудлатых мраморных самцов не  задевала  моей
ангельской  натуры,  настолько  целомудренной,  что они не смущали меня даже
вблизи, -- я была вправе не понимать их поз. Он поцеловал мою  руку  --  как
раз  то  место,  где  было  загадочное  затвердение:  губами  он  не мог его
почувствовать. А где притаился мой укротитель? Наверное, в ящике кареты.  Но
может  быть,  я  прежде  должна  добыть  для  него  какие-то секреты, словно
волшебный стетоскоп, приложенный к груди осужденного мудреца.  Я  ничего  не
смогла рассказать Арродесу.

В два  дня наш роман  прошел все подобающие стадии.  Я жила с  кучкой верных
слуг в  поместье, расположенном  в четырех  почтовых станциях  от резиденции
короля.  ФлЈбе, мой  дворецкий,  снял  особняк на  следующий  же день  после
свидания в саду, ни словом не обмолвившись, во что это обошлось, а я, ничего
не понимающая в денежных делах девушка, ни о чем не спрашивала. Помнится, он
меня побаивался  и злился на  меня -- видимо, не  был посвящен в  суть дела,
даже наверняка не был, просто выполнял королевский приказ: на словах -- сама
почтительность,  а в  глазах  нескрываемое презрение,  --  скорей всего,  он
принимал меня  за новую королевскую пассию,  а моим прогулкам и  свиданиям с
Арродесом не  слишком удивлялся  -- умный слуга  не станет  требовать, чтобы
король  строил свои  отношения с  наложницей по  схеме, привычной  для него,
слуги. Полагаю, если бы при нем я  вздумала обниматься с крокодилом, он бы и
тогда  глазом  не  моргнул.  Я  была  свободна  во  всем,  что  не  перечило
королевской  воле, однако  сам монарх  не  показался там  ни разу.  И я  уже
убедилась, что есть  слова, которых я никогда не скажу  своему суженому, ибо
язык  у меня  тотчас  немел при  одном  лишь желании  произнести  их и  губы
деревенели, совсем как пальцы, когда я  пробовала ощупывать себя в ту ночь в
карете. Я  твердила Арродесу, чтоб он  не смел посещать меня,  а он объяснял
это,  как все  люди, простой  боязнью оказаться  скомпрометированной и,  как
человек порядочный, старался держаться осторожней.

На третий день вечером я наконец  отважилась узнать, кто я. Оставшись одна в
спальне,  я  сбросила  пеньюар  и  стала перед  зеркалом  --  нагая  статуя.
Серебряные  иглы  и  стальные  ланцеты,  разложенные  на  подзеркальнике,  я
прикрыла бархатной  шалью, так как боялась  их блеска, хоть и  не боялась их
лезвий. Высоко посаженные груди смотрели вверх и в стороны розовыми сосками,
след укола  на бедре исчез. Обдумывая  операцию, точно акушер или  хирург, я
обеими руками  мяла это белое  гладкое тело  так, что ребра  прогибались, но
живот, выпуклый,  как у женщины с  готической картины, не поддавался,  и под
его  теплой,  мягкой оболочкой  я  ощутила  неуступчивую твердость.  Проведя
ладонями сверху вниз, я нащупала и  очертила в своем чреве овальный предмет.
Поставив по  обе руки  от себя  по шесть свечей,  я кончиками  пальцев взяла
ланцет,  самый маленький,  но не  из  страха, а  только потому,  что он  был
изящнее других. В зеркале все выглядело так, будто я собираюсь пронзить себя
ножом,  -- чистой  воды финальная  сцена из  трагедии, выдержанная  в едином
стиле  до последней  мелочи: широкое  ложе  с балдахином,  два ряда  высоких
свечей,  блеск стали  в  моей руке  и  моя бледность,  потому  что тело  мое
страшилось  и  колени подкашивались,  и  только  рука, державшая  скальпель,
сохраняла  необходимую твердость.  Именно  туда,  где овальный  неподатливый
предмет  прощупывался  всего явственней,  чуть  пониже  грудины, я  с  силой
вонзила ланцет. Боль была мгновенной и  слабой, а из разреза выступила всего
лишь  капля крови.  Не обладая  умением  мясника, я  аккуратно, как  анатом,
рассекла  тело от  грудины  до лона  -- правда,  сжав  зубы и  зажмурившись.
Смотреть было уже сверх моих сил. Однако я стояла, теперь уже не дрожащая, а
только похолодевшая,  и мое  дыхание, судорожное,  как у  астматика, звучало
сейчас  в  комнате,  будто  чужое,  будто  доносившееся  извне.  Рассеченная
белокожая оболочка разошлась, и я  увидела в зеркале свернувшееся серебряное
тело -- как бы огромный плод,  скрытую во мне блестящую куколку, обрамленную
розовыми  складками некровоточащей  плоти. Это  было чудовищно  -- так  себя
видеть!  Я не  отваживалась  коснуться  серебристой поверхности,  чистейшей,
безупречной. Овальное туловище сияло,  отражая уменьшенные огоньки свечей. Я
пошевелилась  и тут  же  увидела его  ножки, прижатые  в  утробной позе,  --
тонкие,  раздвоенные, как  щипцы,  они исходили  из моего  тела,  и я  вдруг
поняла, что это ``оно'' не было чужим,  инородным -- оно тоже было мною. Вот
почему, ступая по мокрому песку, я  оставляла такие глубокие следы, почему я
была  такой сильной:  ``Это  же --  я,  это  снова --  я'',  -- повторяла  я
мысленно, когда вдруг вошел Арродес.

Я оставила двери незапертыми -- такая неосторожность! И он прокрался ко
мне, неся  перед  собой,  как  оправдание и щит, огромный букет красных роз,
вошел и так был зачарован собственной дерзостью, что, когда я  обернулась  с
криком ужаса, он, все уже увидев, ничего не осознавал, не понимал, не мог...
Не  от  испуга,  а только от огромного стыда, душившего меня, я еще пыталась
хотя бы прикрыть руками серебряный овал, но он был слишком велик,  а  разрез
слишком широк, чтобы это удалось.

Его лицо,  беззвучный крик и бегство...  От этой части показаний  прошу меня
освободить. Не мог дождаться позволения, приглашения и вот пришел с цветами,
а  дом был  пуст. Я  же  сама отослала  всех  слуг, чтобы  никто не  помешал
задуманному мной, -- у меня уже не было выбора, не было другого пути. А быть
может, в него  уже закралось первое подозрение? Я вспомнила,  как вчера днем
мы переходили через  русло высохшего ручья и как он  хотел перенести меня на
руках, а я  запретила ему, но не из стыдливости,  истинной или притворной, а
потому, что это было запретно. А он тогда заметил на мягком податливом песке
следы моих  ног, такие маленькие и  такие глубокие, и хотел  что-то сказать,
наверное,  какую-нибудь невинную  шутку, но  смолчал, знакомая  морщинка меж
бровями  стала резче  -- и,  взбираясь  на противоположный  берег, вдруг  не
протянул  мне руки.  Может быть,  уже  тогда... И  еще: когда  уже на  самой
вершине холма я споткнулась и,  ухватившись, чтобы сохранить, равновесие, за
толстую  ветку орешника,  почувствовала, что  вот-вот выворочу  весь куст  с
корнями,  -- я  опустилась на  колени,  отпустив сломанную  ветвь, чтобы  не
выдать моей неодолимой силы. Он тогда стоял, повернувшись боком, не глядя на
меня, но,  мне казалось, все увидел  краешком глаза -- так  из-за подозрений
прокрался он сюда или от неудержимой страсти?

Теперь уже все равно.

Сочленениями  своих щупальцев  я оперлась  на края  открытого настежь  тела,
чтобы наконец освободиться,  и проворно высунулась наружу,  и тогда Тленикс,
Дуэнья,  Миньона сперва  опустилась  на  колени, потом  рухнула  на бок,  и,
распрямляя все  свои ноги и  неторопливо пятясь,  словно рак, я  выползла из
нее.  Свечи  сияли в  зеркале,  и  пламя  их  еще колебалось  от  сквозняка,
поднятого его  бегством. Обнаженная лежала неподвижно,  непристойно раскинув
ноги.  Не желая  прикасаться к  ней, моему  кокону, моей  фальшивой коже,  я
обошла  ее стороной  и,  откинув  корпус назад,  поднялась,  как богомол,  и
посмотрела на себя  в зеркало. ``Это я,  -- сказала я себе без  слов. -- Это
все  еще  я''.  Обводы  гладкие,  жестокрылые,  насекомоподобные;  утолщения
суставов, холодный блеск серебряного  брюшка; бока обтекаемые, созданные для
скорости; темная,  пучеглазая голова.  ``Это я'', --  повторяла я  про себя,
будто заучивала  на память,  и тем  временем затуманивались  и гасли  во мне
многократные  мои прошлые:  Дуэньи,  Тленикс, Ангелиты.  Теперь  я могла  их
вспоминать только как  давно прочитанные книжки из детства с  неважным и уже
бессильным содержанием. Медленно поворачивая голову, я пыталась разглядеть в
зеркале  свои  глаза  и,  хотя  еще  не  совсем  освоилась  со  своим  новым
воплощением, уже понимала, что к этому акту самоизвлечения я пришла вовсе не
по  своей воле  --  он  был заранее  предусмотренной  частью некоего  плана,
рассчитанной именно на  такие обстоятельства -- на  бунт, которому надлежало
быть  прелюдией к  полной покорности.  Я и  теперь могла  мыслить с  прежней
быстротой и  свободой, но  зато была  подчинена моему новому  телу --  в его
сверкающий  металл  были  впечатаны  все действия,  которые  мне  предстояло
совершить.

Любовь  угасла.  Гаснет  она  и  в  вас,  только годами и месяцами, а я
пережила такой же закат чувства за несколько минут -- и то было  уже  третье
по счету мое начало, и тогда, издавая легкий плавный шорох, я трижды обежала
комнату,  то  и  дело притрагиваясь вытянутыми усиками к кровати, на которой
мне уже не суждено отдыхать. Я вбирала в себя запах моего нелюбовника, чтобы
пуститься по его следу  и  померяться  силами  в  этой  новой  и,  наверное,
последней игре.

Начало его панического  бегства было обозначено распахнутыми  одна за другой
дверями и рассыпанными розами. Их запах  мог мне помочь, потому что он, хотя
бы на время, стал частью его  запаха. Комнаты, сквозь которые я пробегала, я
теперь видела снизу вверх -- в новой перспективе, и они казались мне слишком
большими, наполненными неудобными, лишними вещами, которые враждебно темнели
в полумраке.  Потом мои  коготки слабо  заскрежетали по  ступенькам каменной
лестницы,  и  я  выбежала в  сырой  и  темный  сад.  Пел соловей  --  теперь
мне  это  показалось  забавным:  сей реквизит  был  ненужен,  следующий  акт
спектакля  требовал  нового.  С  минуту  я  рыскала  между  кустами,  слыша,
как  хрустит гравий,  брызжущий  из-под  моих ног,  описала  круг, другой  и
вомчалась напрямик,  ища след.  Не взять  его я  не могла  -- я  выудила его
из  неимоверной  мешанины тающих  запахов,  извлекла  из колебаний  воздуха,
рассеченного  Арродесом на  бегу,  каждую его  частичку,  еще не  развеянную
ветром, и так вышла на предначертанный  мне путь, который с этой минуты стал
моим до конца.

Не знаю, по чьей воле я дала Арродесу столь  большую  фору,  и,  вместо
того чтобы идти по следу, до самого рассвета рыскала по королевским садам. В
этом мог быть скрыт известный смысл, ибо я кружила там, где мы прогуливались
рука  об руку между живыми изгородями, и могла хорошенько впитать его запах,
чтобы наверняка не спутать с  другими.  Правда,  проще  было  сразу  за  ним
помчаться и захватить его, беспомощного, в полном замешательстве и отчаяния,
но  я  этого  не  сделала. Знаю, все мое поведение в ту ночь можно объяснить
по-разному: и моей скорбью, и королевской волей. Я потеряла возлюбленного  и
взамен  обрела  лишь  гонимую  дичь,  а  монарху мало было одной лишь гибели
ненавистного ему человека, притом быстрой и  внезапной.  Арродес,  наверное,
тем  временем  помчался  не к себе домой, а к кому-то из друзей, чтобы там в
сумбурной исповеди, самому себе задавая вопросы и на них отвечая,  до  всего
дойти своим умом: чье-то присутствие было ему все-таки необходимо, но только
как  отрезвляющая  поддержка.  В  моих скитаниях по садам ничего, однако, не
было от мучений разлуки.  Я  знаю,  как  неприятно  это  прозвучит  для  душ
чувствительных,  но,  не имея ни рук, чтобы их заламывать, ни слез, чтобы их
проливать, ни колен, на которые могла бы пасть, ни губ, чтобы прижать к  ним
увядшие  цветы,  я  не  впадала  в отчаяние. Тогда меня куда больше занимало
необычайное умение различать следы, которое вдруг  во  мне  открылось.  Ведь
когда  я  пробегала  по  аллеям,  меня  ни  разу,  ни на волос не сбил чужой
обманчивый след, пусть даже и очень схожий с тем, что стал моей приманкой  и
моим  кнутом.  Я  ощущала,  как  каждая частица воздуха просасывается в моем
левом легком сквозь лабиринты бесчисленных испытующих клеток  и  как  каждая
подозрительная  частичка  попадает  в  мое  правое,  горячее легкое, где мой
внутренний  призматический  глаз  внимательно  всматривается  в  нее,  чтобы
подтвердить правильность отбора или отшвырнуть прочь как ненужную, и все это
свершается  быстрее  взмаха  крылышек  мошки,  быстрее,  чем  вы  смогли  бы
осознать. На рассвете я покинула королевские сады. Дом Арродеса стоял пустой
-- двери настежь, и там, не помыслив даже проверить,  взял  ли  он  с  собой
какое-нибудь  оружие,  я  отыскала  новый  след  и пустилась по нему уже без
проволочек. Я не рассчитывала, что  путешествие  будет  долгим,  однако  дни
сложились  в  недели, недели в месяцы, а я все еще за ним гналась. И все мои
поступки вовсе не казались мне более мерзкими, чем поведение других существ,
направляемых жребием, выше им предначертанным.


Я бежала в дождь и в жару через луга, овраги и заросли, сухой  тростник
хлестал  по  моему  туловищу,  а  вода ручьев и луж, через которые я неслась
напрямик, обдавала меня и скатывалась по выпуклой спине, по голове и  глазам
крупными,  как  слезы,  каплями,  но это были не слезы. В своем непрестанном
беге я видела, что каждый, кто замечал меня еще издали, тотчас отворачивался
и становился лицом к стене или к дереву, а если рядом ничего не было,  падал
на  колени, закрыв руками лицо, или валился ничком и долго еще лежал, хотя я
была уже далеко. Мне не нужен был сон, и потому я бежала  и  ночью,  и  днем
через  деревни, селения, местечки, через рынки, полные плодов, вялившихся на
веревках, и глиняных горшков, и целые толпы селян  разбегались  передо  мной
врассыпную,  и  дети  с визгом бросались в боковые улочки, а я, ни на что не
обращая внимания, мчалась по назначенному мне следу.  Я  уже  позабыла  лицо
того  человека, и мое сознание, видимо, менее выносливое, чем тело, сужалось
-- особенно во время ночного бега -- настолько, что я  уже  не  знала,  кого
преследую и вообще преследую ли кого-то: знала только, что единственная воля
моя  -- мчаться так, чтобы запах, ведущий меня в этом буйном половодье мира,
сохранялся и усиливался, ибо, если он ослабевал, это значило, что я  сбилась
с  верного  пути.  Я  никого  ни  о  чем  не  спрашивала, да и меня никто не
отваживался бы о чем-либо спросить. Пространство, разделявшее  меня  и  тех,
кто  съеживался  у стен при моем появлении или падал наземь, закрывая руками
затылок,  было  полно  напряженного  молчания,  и  я  воспринимала  его  как
положенную  мне  почтительную  дань  ужаса,  ибо  я  шла  королевским путем,
наделенная  беспредельным  могуществом.  И  разве  лишь  маленький  ребенок,
которого родители не успели подхватить на руки при моем внезапном появлении,
принимался  плакать,  но мне было не до него, потому что моей воле надлежало
неустанно быть предельно  собранной,  сосредоточенной,  разом  обращенной  и
наружу,  в  зеленый, песчаный, каменистый мир, окутанный голубой дымкой, и в
мой внутренний мир, где в четкой работе обоих моих легких  рождалась  музыка
молекул, прекрасная, совершенная в своей безошибочности. Я пересекала реки и
рукава  лиманов,  пороги,  илистые  впадины  высыхающих озер, и всякая тварь
бежала меня, уносясь скачками или лихорадочно зарываясь в  спекшийся  грунт,
но,  вздумай  я  на них поохотиться, бегство было бы напрасным, ибо никто из
них не был так молниеносно проворен, как я, но что мне до них  --  косматых,
четвероногих,  длинноухих  тварей, издающих писк, вой или хриплое ржание, --
ведь у меня была иная цель...

Иногда я, как снаряд, пробивала большие муравейники  --  их  обитатели,
рыжие,   черные,  пятнистые,  бессильно  скатывались  по  моему  сверкающему
панцирю, а раза  два  какие-то  существа,  несравненно  крупнее  других,  не
уступили  мне  дорогу  -- я ничего против них не имела, но, чтобы не тратить
времени на обход окружным путем, я сжималась в прыжке и на лету прошивала их
насквозь под треск костей и бульканье  красных  струек,  брызгавших  мне  на
спину  и  на голову, и удалялась так быстро, что даже не успевала подумать о
смерти, причиненной таким внезапным  и  быстрым  ударом.  Помню  также,  как
пробиралась  через  поля  сражений, беспорядочно усеянные множеством серых и
зеленых мундиров -- одни еще шевелились, а  из  других  уже  торчали  кости,
грязно-белые,  как подтаявший снег, но я ни на что не обращала внимания, и у
меня была высшая цель, и она была под силу только мне.

Из того, как след вился, петлял, пересекал сам себя, из  того,  как  он
почти   исчезал   на  берегах  соленых  озер  в  пережженной  солнцем  пыли,
раздражавшей мои легкие, или смытый дождями, я постепенно пришла  к  выводу,
что  тот,  кто  ускользает  от меня, изворотлив и хитер и идет на все, чтобы
ввести меня в заблуждение и оборвать цепочку  частиц,  отмеченных  признаком
единства.  Если  бы  тот,  кого  я  преследовала, был простым смертным, я бы
настигла его по истечении предопределенного времени, того, какое необходимо,
дабы страх и отчаяние в должной мере  усугубили  назначенную  ему  кару,  --
тогда  бы  я  наверняка  догнала  его  благодаря своей неутомимой быстроте и
безошибочной работе сыщицких легких -- и уничтожила быстрее, чем  успела  бы
это  осознать.  Но  я не стала наступать ему сразу на пятки: я шла по хорошо
остывшему следу, чтобы насладиться своим мастерством,  а  вместе  с  тем  по
исконному  обычаю  дать  гонимому  время  накопить в себе отчаяние, но порой
позволяла ему хорошенько оторваться, потому что,  чувствуя  мою  неотступную
близость,  он  в  безысходной  тоске  мог  учинить над собою зло и тем самым
ускользнуть от меня и от воздаяния, которое  я  ему  несла.  Мне  надо  было
настичь  его  не  слишком  быстро  и  совсем  не внезапно, ибо он должен был
прочувствовать все, что его ожидает. А потому я  по  ночам  останавливалась,
укрываясь  в чащах не для отдыха, который мне не был нужен, а для умышленных
проволочек и для того, чтобы рассчитать дальнейшие действия. Я уже не думала
о преследуемом как об Арродесе, моем бывшем возлюбленном, -- память об  этом
почти  зарубцевалась,  и ее не стоило тревожить. Я жалела только, что теперь
лишена дара усмехаться, хотя бы при воспоминаниях о былых  фортелях,  сиречь
Ангелите,  дуэнье, сладостной Миньон. И я разглядывала себя лунными ночами в
зеркале воды, чтобы убедиться, что ныне ничем  на  них  не  похожа,  хотя  и
осталась  красивой,  однако  теперь  это  была другая красота, смертоносная,
внушающая страх, великий, подобный восхищению. Тех  моих  ночей  в  укромных
логовах  мне  хватало на то, чтобы очистить брюшко от комков засохшей грязи,
доведя его до серебряного блеска, и перед тем, как  пуститься  в  дальнейший
путь,  я  всякий  раз  легонько  раскачивала  прыжковыми ногами втулку жала,
проверяя ее готовность, потому что день и час мне были неизвестны.

Иногда я бесшумно подкрадывалась к людским жилищам и  прислушивалась  к
голосам,  то  прицепляясь  блестящими  щупальцами  к  оконной раме сбоку, то
заползая на крышу, чтобы поудобнее свеситься с ее края вниз головой,  ибо  я
все  же  не мертвый механизм, снабженный парой сыщицких легких, но существо,
которое пользуется, как подобает, своим разумом. А погоня и бегство тянулись
уже столь долго, что молва о  нас  разнеслась  повсюду,  и  я  слышала,  как
старухи  пугали  мною  детей,  и  узнавала  бесчисленные  толки об Арродесе,
которому почти все сочувствовали в такой же мере, в какой  страшились  меня,
королевской посланницы. Что же болтали простаки на завалинках?

Что  я машина, которую натравили на мудреца, осмелившегося посягнуть на
королевскую  власть.  Что  я  не  простой  механический  палач,   а   особое
устройство,  способное  произвольно принимать любой облик: нищего, ребенка в
колыбели, прекрасной девушки или же металлической  змеи.  Но  эти  формы  --
только  маски,  в  которых  подосланная машина является преследуемому, чтобы
соблазнить его. Перед всеми же другими она предстает в  обличье  серебряного
скорпиона,  который  бегает  так быстро, что никому еще не удалось сосчитать
всех его ног.  Тут  повествование  разделялось  на  множество  версий.  Одни
говорили,  что  мудрец  вопреки  королевской  воле хотел даровать всем людям
свободу и тем возбудил монарший гнев. Другие -- что у него была живая вода и
он мог воскресить замученных, и это было запрещено ему высочайшим указом,  а
он,  притворно  склонившись  перед  волей  владыки,  тайно  собирал  рать из
казненных бунтовщиков, тела которых он похищал с виселиц на цитадели. Многие
вообще  ничего  не  знали  об  Арродесе  и  не   приписывали   ему   никаких
сверхъестественных  способностей, а просто полагали, что коли он осужден, то
уже по одному этому заслуживает сочувствия и помощи. И хотя  никто  не  знал
истинных  причин,  из-за  которых распалилась королевская ярость и созванным
мастерам приказано было соорудить в их кузницах гончую машину, --  злым  все
звали это умыслом и неправедным повелением, ибо, что бы ни совершил гонимый,
вина  его  не  могла  быть  столь  же  страшной, как судьба, уготованная ему
королем.  Конца  не  было  этим  россказням,  в  которых  вволю  расходилось
простецкое  воображение,  и  лишь  одно  в  них  не менялось: мне всякий раз
приписывали такие мерзости, какие только можно вообразить.

Слышала я также и тьму вранья о смельчаках,  будто  бы  поспешавших  на
помощь к Арродесу, которые-де преграждали мне дорогу, чтобы пасть в неравном
бою,  --  на самом деле на это ни единая живая душа не отважилась. Хватало в
сказках и предателей, указывавших мне его следы, когда я не  могла  отыскать
их  сама,  --  вот  уж отъявленнейшая ложь. Однако же о том, кто я, кем могу
быть, что у меня на уме, ведомы ли мне  растерянность  или  сомнение,  никто
ничего не говорил, да я тому и не удивлялась.

И  я  столько   наслышалась  о  простых,  всем   известных  гончих  машинах,
выполняющих  королевскую волю,  которая была  для всех  законом, что  вскоре
совсем перестала  таиться от обитателей  этих приземистых изб и  порой прямо
под их окнами дожидалась восхода  солнца, чтобы серебряной молнией выскочить
на  траву и  в  блистающих  брызгах росы  связать  конец  вчерашнего пути  с
началом сегодняшнего и, стремительно мчась по нему, упиваться остекленевшими
взглядами,  падением ниц,  смертельным  страхом  и ореолом  неприкасаемости,
который окружал меня.

Однако настал день, когда мой верхний нюх оказался беспомощен, и тогда,
тщетно  петляя  по холмистым окрестностям в поисках следа, я изведала боль и
горечь от того, что мое совершенство напрасно. Но, застыв на  вершине  холма
со  скрещенными  щупальцами  и  как  бы  молясь ветреному небу, я по слабому
звуку, наполнившему колокол  моего  тела,  вдруг  поняла,  что  не  все  еще
потеряно,  и,  чтобы исполнить замысел, обратилась к давно заброшенному дару
-- человеческой речи. Мне не нужно было учиться ей заново, она была во  мне,
я  должна  была лишь оживить ее в себе. Сначала я выговаривала слова и фразы
резко и визгливо, но скоро мой голос стал почти человеческим, и я сбежала по
склону, чтобы прибегнуть к дару слова -- там, где меня подвело  обоняние.  Я
вовсе не чувствовала ненависти к беглецу, хотя он и оказался таким проворным
и  хитрым,  -- он играл свою роль, а я играла свою. Я отыскала перепутье, на
котором след угасал, остановилась и судорожно задергалась на  месте  оттого,
что   одна   пара  моих  ног  бессознательно  тянулась  к  дороге,  покрытой
известковой пылью, а  другая,  лихорадочно  царапая  камни,  тащила  меня  в
противоположную  сторону  --  туда,  где  белели стены небольшого монастыря,
окруженного вековой рощей. Собрав всю свою волю, я тяжело,  будто  немощная,
подползла  к монастырской калитке, у которой стоял, подняв очи к небу, монах
-- казалось, он залюбовался зарей. Я потихоньку приблизилась к  нему,  чтобы
не  испугать  своим  внезапным  появлением, и смиренно приветствовала его, а
когда он  безмолвно  обратил  на  меня  внимательный  взгляд,  спросила,  не
позволит  ли  он, чтобы я поведала ему о деле, в котором сама разобраться не
могу. Я поначалу  решила,  что  он  окаменел  от  страха,  ибо  он  даже  не
пошевелился  и ничего не ответил, но оказалось, он просто задумался и минуту
спустя сказал, что согласен. Тогда мы пошли в монастырский сад, он  впереди,
я -- за ним. Странная, наверное, пара, но в тот ранний час вокруг не было ни
единой  живой  души  --  некому  подивиться на серебряного богомола и белого
монаха. И когда он сел под лиственницей в  привычной  позе  исповедника,  не
глядя  на меня, а лишь склонив ко мне ухо, я рассказала ему, что, прежде чем
выйти на эту торную тропу, я была девушкой, предназначенной Арродесу по воле
короля. Что я познакомилась с ним на балу во дворце и полюбила его, ничего о
нем не зная, и в неведении совершенно отдалась этой любви,  которую  сама  в
нем  возбудила,  и  так  было, пока после ночного укола я не поняла, кем мне
суждено стать для него, и,  не  видя  ни  для  себя,  ни  для  него  другого
спасения,  проткнула себя ножом, но вместо смерти свершилось перевоплощение.
И жребий, о котором я раньше только подозревала, с тех  пор  ведет  меня  по
следу  возлюбленного  --  я  сделалась настигающей его Немезидой. Погоня эта
длится долго -- так долго, что до меня стало доходить все, что люди  говорят
об  Арродесе,  и,  хотя  я  не  знаю,  сколько в том правды, я начала заново
размышлять над нашей общей судьбой, и в  мою  душу  закралось  сочувствие  к
этому человеку, ибо я поняла, что изо всех сил хочу убить его только потому,
что  не  могу  его  больше  любить.  Так  я познала собственное ничтожество,
низость погибшей и попранной любви, которая алчет мести тому, кто не повинен
перед ней ни в чем,  кроме  собственного  несчастья.  Оттого  и  не  хочу  я
продолжать  погоню и сеять вокруг себя ужас, а хочу воспротивиться злу, хотя
и не знаю как.

Насколько  я  могла заметить,  до  конца,  моего  рассказа монах  ничуть  не
избавился  от  подозрительности:  он  как  бы заранее,  еще  прежде,  чем  я
заговорила, решил для себя, что все,  что я скажу, не подпадает под таинство
исповеди, так как, по его  разумению, я была существом, лишенным собственной
воли. А кроме  того, наверное, подумал, не подослана ли  я к нему умышленно,
ведь, по слухам,  иные лазутчики маскируются еще  коварнее. Однако заговорил
он со мной доброжелательно.

Он спросил меня: "А что, если бы ты нашла того, кого ищешь?  Знаешь  ли
ты, что бы ты сделала тогда?"

И  я  сказала: "Отец мой, я знаю только то, чего не хочу сделать, но не
знаю, какая сила, кроющаяся во мне, пробудится в тот миг, а потому  не  могу
сказать, не буду ли я принуждена погубить его".

И  он  спросил  меня:  "Какой  же совет я могу тебе дать? Хочешь ли ты,
чтобы этот жребий был снят с тебя?"

Лежа, словно пес, у его ног, я подняла голову и, видя, как он  жмурится
от  солнечного  луча,  который  ударил  ему в очи, отраженный серебром моего
черепа, сказала:

-- Ничего так не желаю, как этого,  хоть  и  понимаю,  что  судьба  моя
станет  тогда жестокой, потому что не будет у меня тогда более никакой цели.
Не я выдумала  то,  для  чего  сотворена,  и  значит,  мне  дорого  придется
заплатить,  если  преступлю  королевскую  волю,  ибо  немыслимо,  чтобы  мое
преступление осталось безнаказанным, и меня в свою очередь возьмут на прицел
оружейники из дворцовых подземелий и вышлют в погоню железную  свору,  чтобы
уничтожить  меня.  А если бы я даже спаслась, воспользовавшись заложенным во
мне искусством, и убежала хоть на край света, то где бы я ни очутилась,  все
станут  бежать  меня, и я не найду цели, ради которой стоило бы существовать
дальше. И даже судьба, подобная твоей, также будет для меня закрыта,  потому
что  каждый,  имеющий, как ты, власть, так же, как ты, ответит мне, что я не
свободна духовно,  и  потому  мне  не  дано  будет  обрести  убежища  и  под
монастырским кровом.

Монах задумался и потом сказал удивленно:

-- Я  ничего  не  знаю  об устройствах, подобных тебе, но я вижу тебя и
слышу, и ты по твоим речам представляешься мне разумной, хотя и  подчиненной
какому-то  принуждению,  и  -- коль скоро ты, машина, борешься, как сама мне
поведала, с этим принуждением и говоришь, что чувствовала бы себя свободной,
если бы у тебя отняли стремление убить, -- то скажи мне, как  ты  чувствуешь
себя сейчас, когда оно в тебе?

И я сказала на это:

-- Отче,   хоть   мне  с  ним  и  худо,  но  я  превосходно  знаю,  как
преследовать, как настигать, следить, подсматривать и подслушивать,  таиться
и  прятаться,  как  ломать  на пути препятствия, подкрадываться, обманывать,
кружить  и  сжимать  петлю  кругов,  причем,  исполняя  все  это  быстро   и
безошибочно,  я  становлюсь  орудием неумолимой судьбы, и это доставляет мне
радость, которая, наверное, с умыслом была вписана пламенем в мое нутро.

-- Снова спрашиваю тебя, -- сказал монах. --  Что  ты  сделаешь,  когда
увидишь Арродеса?

-- Снова  отвечаю,  отче, что не знаю, ибо не хочу причинить ему ничего
дурного, но то, что заложено во мне, может оказаться сильнее меня.

Выслушав мой ответ, он прикрыл глаза рукой и промолвил:

-- Ты -- сестра моя.

-- Как это понимать? -- спросила я в полнейшем недоумении.

-- Так, как сказано, -- ответил он. -- А это значит, что я  не  возвышу
себя над тобой и не унижу себя пред тобою, потому что, как бы различны мы ни
были,  твое  неведение,  в  котором  ты призналась, делает нас равными перед
лицом Провидения. А если так, иди за мной, и я покажу тебе нечто.

Мы прошли через монастырский  сад  к  старому  дровяному  сараю.  Монах
толкнул  скрипучие  двери,  и  когда  они распахнулись, то в сумраке сарая я
различила лежащий на соломе темный предмет, а сквозь  ноздри  в  мои  легкие
ворвался  тот  неустанно  подгонявший меня запах, такой сильный здесь, что я
почувствовала, как само взводится и высовывается из лонной втулки жало, но в
следующую минуту взглядом переключенных на  темноту  глаз  я  заметила,  что
ошиблась.  На  соломе  лежала  только  брошенная одежда. Монах по моей дрожи
понял, как я потрясена, и сказал:

-- Да, здесь был Арродес. Он скрывался в нашем монастыре целый месяц  с
тех  пор,  как  ему  удалось  сбить тебя со следа. Он страдал оттого, что не
может предаваться прежним занятиям, и ученики, которым он тайно дал знать  о
себе,  посещали  его  по  ночам, но среди них оказались два мерзавца, и пять
дней назад они его увели.

-- Ты хотел сказать "королевские посланцы"?  --  спросила  я,  все  еще
дрожа и молитвенно прижимая к груди скрещенные щупальца.

-- Нет,  я  говорю  "мерзавцы",  потому  что  они взяли его хитростью и
силой. Глухонемой мальчик, которого мы приютили, один видел, как  они  увели
его на рассвете, связанного и с ножом у горла.

-- Его похитили? -- спросила я, ничего не понимая. -- Кто? Куда? Зачем?

-- Думаю, затем, чтобы извлечь для себя корысть из его мудрости. Мы  не
можем обратиться за помощью к закону, потому что это -- королевский закон. А
эти  двое  заставят  его  им служить, а если он откажется, убьют его и уйдут
безнаказанными.

-- Отче! -- воскликнула  я.  --  Да  будет  благословен  час,  когда  я
осмелилась  приблизиться  и  обратиться  к  тебе.  Я  пойду теперь по следам
похитителей и освобожу Арродеса.  Я  умею  преследовать,  настигать:  ничего
другого  я  не  умею  делать  лучше -- только покажи мне верное направление,
которое ты узнал от немого мальчика.

Он возразил:

-- Но ты же не знаешь, сможешь ли удержаться, -- ты ведь  сама  в  этом
призналась!

И я сказала:

-- Да,  но  я  верю,  что  найду  какой-нибудь выход. Может быть, найду
мастера, который отыщет во мне  нужный  контур  и  изменит  его  так,  чтобы
преследуемый превратился в спасаемого.

А монах сказал:

-- Прежде  чем  отправиться  в  путь, ты, если хочешь, можешь попросить
совета у одного из наших братьев: до того, как присоединиться к нам, он  был
в миру посвящен именно в такое искусство. Здесь он пользует нас как лекарь.

Мы стояли в саду, уже освещенном лучами солнца. Я чувствовала, что монах все
еще  не доверяет  мне,  хотя внешне  он  этого никак  не  проявлял. За  пять
дней  след улетучился,  и он  мог с  равной вероятностью  направить меня  по
истинному  пути  и  по  ложному.  Но  я  согласилась  на  все,  и  лекарь  с
величайшей  предосторожностью  принялся  осматривать меня,  светя  фонариком
сквозь щели между  пластинами панциря в мое нутро, и  проявил при этом много
внимательности и  старания. Потом он  встал, отряхнул  пыль со своей  рясы и
сказал:

-- Случается, что на машину, высланную с  известной  целью,  устраивает
засаду  семья осужденного, его друзья или другие люди, которые по непонятным
для властей причинам пытаются  воспрепятствовать  исполнению  предписанного.
Для  противодействия  сему  прозорливые  королевские  оружейники изготовляют
распорядительную суть непроницаемой и замыкают  ее  с  исполнительной  сутью
таким  образом, чтобы всякая попытка вмешательства оказалась губительной. И,
наложив последнюю печать, даже сами они  уже  не  могут  удалить  жала.  Так
обстоит  дело  и  с тобой. А еще случается, что преследуемый переодевается в
чужую одежду, меняет внешность, поведение и запах, однако  же  он  не  может
изменить  склада своего разума, и тогда машина, не удовлетворившись розыском
при  посредстве  нижнего  и  верхнего  обоняния,  подвергает  подозреваемого
допросам, продуманным сильнейшими знатоками отдельных способностей духа. Так
же  обстоит  дело  и  с  тобой.  Но  сверх  всего  я  приметил в твоем нутре
устройство,  какого  не  имела  ни  одна  из   твоих   предшественниц:   оно
представляет  собой  многоразличную  память  о  предметах, для гончей машины
излишних, ибо в ней записаны истории разных женщин, полные искушающих  разум
имен  и  речей,  --  именно  от  сего  устройства и бежит в тебе проводник к
смертоносной сути. Так что ты -- машина, усовершенствованная непонятным  мне
образом,  а  может  быть, даже и воистину совершенная. Удалить твое жало, не
вызвав при этом упомянутых последствий, не сможет никто.

-- Жало понадобится мне, -- сказала я, все еще лежа ничком,  --  ибо  я
должна поспешить на помощь похищенному.

-- Что  касается  того,  смогла  бы  ты сдержать затворы, опущенные над
известным местом, или нет, даже если бы хотела этого изо всех  сил,  на  сей
счет  я  не могу сказать ни да, ни нет, -- продолжал лекарь, словно не слыша
моих слов. -- Я могу, если ты, конечно, захочешь,  сделать  только  одно,  а
именно:  напылить на полюса известного места через трубку железо, истертое в
порошок, так что от этого несколько увеличатся  пределы  твоей  свободы.  Но
даже  если я сделаю это, ты до последнего мгновенья не будешь знать, спеша к
тому, кому стремишься помочь,  не  окажешься  ли  ты  по-прежнему  послушным
орудием его погибели.

Видя,  как  испытующе  смотрят на меня оба монаха, я согласилась на эту
операцию, которая продолжалась недолго, не доставила мне неприятных ощущений
и не вызвала в моем душевном состоянии никаких ощутимых перемен.  Чтобы  еще
больше  завоевать  их  доверие,  я спросила, не позволят ли они мне провести
ночь в монастыре, потому  что  весь  день  ушел  на  беседы,  рассуждения  и
медицинские процедуры.

Они  охотно согласились, а я посвятила ночное время исследованию сарая,
запоминая запахи похитителей  Арродеса.  Я  была  способна  и  на  это,  ибо
случалось, что королевской посланнице преграждал дорогу не сам осужденный, а
какой-нибудь  другой  смельчак. Перед рассветом я улеглась на соломе -- там,
где многие ночи спал похищенный, и, в полной неподвижности вдыхая его запах,
дожидалась прихода монахов. Я допускала, что  все  их  рассказы  могли  быть
выдумкой,  обманом  и,  коли  так,  они  должны  бояться моего возвращения с
ложного следа и моей мести, а этот темный предрассветный  час  был  для  них
наиболее  подходящим,  если  бы они вознамерились меня уничтожить. Я лежала,
притворившись глубоко спящей, и вслушивалась в  каждый,  даже  самый  легкий
шорох,  доносившийся  из  сада:  ведь  они могли завалить чем-нибудь двери и
поджечь сарай, дабы плод чрева моего разорвал бы меня в пламени на куски. Им
не пришлось бы даже преодолевать свойственного им отвращения к убийству, ибо
я была для них не личностью, а только механическим палачом, останки мои  они
закопали  бы  в саду и не испытали бы никаких угрызений совести. Я не знала,
что предприняла бы, услышав их приближение, и не узнала этого, потому что ни
до чего такого не дошло. Я  оставалась  наедине  со  своими  мыслями  и  все
повторяла про себя удивительные слова, которые сказал, не глядя мне в глаза,
старый  монах:  "Ты  -- сестра моя". Я по-прежнему их не понимала, но, когда
мысленно их повторяла, они всякий раз обжигали меня, словно я  уже  утратила
тот  тяжелый  плод,  которым  была  обременена.  Рано утром я выбежала через
незапертую калитку и, миновав монастырские постройки, как указал мне  монах,
полным ходом пустилась в сторону синевших на горизонте гор -- именно туда он
и направил мой бег.

Я очень спешила -- к полудню меня отделяло от монастыря более ста миль.
Я летела,  как снаряд, между белоствольных берез, достигла предгорных лугов,
и, когда бежала по ним напрямик, высокая трава разлеталась по  обе  стороны,
словно под ударами косы.

След  похитителей  я нашла в глубокой долине, на мостике, переброшенном
через поток, но не обнаружила на нем следов Арродеса -- видимо,  пренебрегая
тяжестью,  они  по  очереди  несли  его,  выказывая  этим  свою  хитрость  и
осведомленность, ибо понимали, что никто  не  вправе  опередить  королевскую
машину  в  ее  миссии,  что  и так они уже немало повредили монаршей власти,
отважившись на это свое деяние.

Вы, наверное, хотели бы знать, каковы были  мои  истинные  намерения  в
этой последней погоне, -- я скажу, что и обманула монахов, и не обманула их,
ибо  на  самом  деле желала лишь возвратить себе свободу, вернее, добыть ее,
поскольку никогда раньше ее  не  имела.  Если  же  спросите  о  том,  что  я
собралась  делать  с  этой  своей  свободой,  то  не  знаю, что вам сказать.
Незнание не было мне внове: вонзая в свое обнаженное тело  нож,  я  тоже  не
знала,  чего  хочу, -- убить ли себя или только познать, пусть даже одно при
этом будет равнозначно другому. И следующий мой шаг тоже был предусмотрен --
об этом свидетельствовали все дальнейшие события,  а  потому  и  надежда  на
свободу  тоже могла оказаться только иллюзией, и даже не моей собственной, а
нарочно введенной в меня, чтобы я действовала энергичнее  побуждаемая  такою
коварно  подсунутой  приманкой. Как знать, не равнялась ли свобода отказу от
Арродеса? Ведь я могла ужалить его, даже будучи полностью свободной, я же не
была настолько безумной, чтобы поверить в невероятное  чудо  --  в  то,  что
взаимность может возвратиться теперь, когда я уже перестала быть женщиной, и
пусть  не  совсем перестала быть ею, но мог ли Арродес, который собственными
глазами видел свою возлюбленную с разверстым животом, поверить в это?  Итак,
хитроумие  сотворивших  меня простиралось за последние пределы механического
искусства, ибо они, несомненно, учли в своих расчетах вариант и этого  моего
состояния,  когда  я  устремлялась на помощь любимому, утраченному навсегда.
Если бы я могла свернуть с пути и удалиться, чреватая смертью,  которую  мне
не  для  кого  родить,  я  и  этим  тоже  ему  не помогла бы. Наверное, меня
намеренно сотворили такой  благородно  никчемной,  порабощенной  собственным
желанием  свободы,  дабы  я  выполняла не то, что мне приказано прямо, а то,
чего -- как мне казалось в очередном моем воплощении -- хотела я  сама.  Мое
путаное  и раздражающее своей бесцельностью самокопание должно было, однако,
прерваться  только  у  цели.   Расправившись   с   похитителями,   я   спасу
возлюбленного  и  сделаю это так, чтобы отвращение и страх, которые он питал
ко мне, сменились бессильным изумлением. Так я смогу обрести если не его, то
хотя бы самое себя.

Пробившись  сквозь  густые  заросли  орешника  к  первому  травянистому
склону, я неожиданно потеряла след. Напрасно я искала его; вот здесь он был,
а  дальше  --  исчез,  как  будто  преследуемые  провалились сквозь землю. Я
догадалась вернуться в чащу и не без труда отыскала куст,  у  которого  было
срублено  несколько самых толстых ветвей. Обнюхав срезы, истекающие соком, я
вернулась туда,  где  след  исчезал,  и  нашла  его  продолжение  по  запаху
орешника.  Беглецы  учли,  что  полоса верхнего запаха недолго продержится в
воздухе -- ее скоро сдует горный ветер -- и потому воспользовались ходулями,
но и эта уловка только подхлестнула меня. Запах орешника вскоре ослабел,  но
я  разгадала  и  новый  их  фортель  --  они обернули концы ходуль обрывками
джутового мешка. Брошенные ходули я нашла неподалеку от  скалистого  обрыва.
Склон  был  усеян огромными замшелыми валунами, которые громоздились друг на
друга так, что преодолеть эту  россыпь  можно  было,  лишь  прыгая  большими
скачками  с  камня  на камень. Так и поступили мои противники, однако они не
избрали прямого пути -- они петляли. Из-за этого  мне  приходилось  сползать
чуть  ли  не  с каждого валуна, чтобы, обежав кругом, сызнова отыскать нюхом
зыблющиеся в воздухе частички их запаха. Так я дошла до отвесной  скалы,  по
которой  они  вскарабкались  наверх.  Они  не  смогли бы взобраться туда, не
развязав руки своему пленнику, но меня не удивило, что он добровольно  полез
вместе  с  ними, -- пути назад теперь для него уже не было. Я поползла вверх
по разогретому камню, ведомая отчетливым, утроенной силы запахом -- ведь  им
приходилось  взбираться  по  этой  отвесной стене, цепляясь за каждый уступ,
промоину, впадину: не было такого клочка седого мха, забившегося в расщелину
нависших скал, ни мелкой трещинки,  дающей  минутную  опору  ногам,  которую
похитители не использовали бы как ступеньку. Порой в самых трудных местах им
приходилось останавливаться, чтобы выбрать дальнейший путь, -- я чувствовала
это  по  усиливающемуся  запаху.  А  я буквально мчалась вверх, едва касаясь
скалы, чувствуя, как сильнее и сильнее все во мне дрожало, как  все  во  мне
играло  и  пело,  ибо  эти  люди были достойны меня, я чувствовала радость и
изумление, потому что восхождение, которое они проделывали втроем, страхуясь
одной веревкой, джутовый запах которой остался на острых выступах  камня,  я
совершала  одна  и  без  особых  усилий  и  ничто  не могло сбить меня с той
поднебесной тропы. На вершине меня встретил сильный ветер,  который  свистел
на   остром,  как  нож,  гребне,  но  я  даже  головы  не  повернула,  чтобы
полюбоваться на простершуюся далеко внизу зеленую страну и горизонты, тающие
в голубой дымке, а принялась  ползать  по  гребню  взад  и  вперед,  пока  в
незаметной  выбоине  не  нашла  продолжение следа. Беловатый излом и осколки
камня обозначили место, где один из путников  сорвался.  Перегнувшись  через
каменную  грань,  я  посмотрела  вниз  и  увидела  маленькую фигурку, словно
отдыхавшую на середине склона, и острым зрением различила даже темные  капли
на  известняке,  словно  оставленные  недолгим  кровавым дождем. Двое других
пошли дальше по гребню, и я пожалела, что мне достанется теперь  всего  один
стерегущий  Арродеса  враг,  потому  что  никогда до сей поры не ощущала так
сильно, сколь благородно мое дело, и не была исполнена такой жаждой  борьбы,
отрезвляющей  и  опьяняющей одновременно. Я побежала вдоль гребня под уклон,
ибо беглецы избрали именно это направление, оставив  погибшего  в  пропасти,
ведь  они  очень  спешили,  а его мгновенная смерть при падении была для них
несомненна. Я приближалась к скальным воротам, похожим на руины  гигантского
собора,  от  которого  остались  только  столбы  разбитого  портала, боковые
контрфорсы и одно высокое окно, сквозь которое светилось небо, а на его фоне
выделялось тоненькое деревце, с  бессознательной  отвагой  выросшее  там  из
семени,  занесенного ветром в горсть праха. За воротами начиналась скалистая
котловина, наполовину затянутая  туманом,  придавленная  длинной  тучей,  из
складок  которой  сыпался  мелкий  искрящийся снег. Пробегая в тени, которую
отбрасывала причудливая башня, я услышала грохот сыплющихся камней, и тут же
по склону скатилась лавина. Глыбы колотились об  меня  с  такой  силой,  что
высекали  дым  и  искры  из  моих  боков, но я, поджав все свои ноги, успела
упасть в неглубокую выемку под валуном  и  в  безопасности  переждала,  пока
пролетели  последние  обломки.  Мне  пришла в голову мысль, что тот, второй,
который вел Арродеса, нарочно выбрал это лавиноопасное место в расчете,  что
я,  не  зная гор, попаду под обвал и обвал -- хоть надежда на это и невелика
-- раздавит меня. Такая мысль меня обрадовала:

ведь если противник не только убегает и путает следы,  но  и  нападает,
борьба  становится  более  достойной.  На  дне  выбеленной  снегом котловины
виднелась постройка -- то ли дом, то ли замок, сложенный  из  самых  тяжелых
валунов,  какие  в одиночку не сдвинул бы и гигант; я поняла, что это и есть
убежище врага, ибо где же ему еще быть в этой глуши. И, бросив поиски следа,
стала сползать с  осыпи,  погрузив  задние  ноги  в  сыплющийся  щебень,  --
передними  я  как  бы  плавала  в мелких обломках, а средней парой тормозила
спуск, чтобы не сорваться. Так я добралась до слежавшегося снега и  по  нему
уже  почти  бесшумно  пошла дальше, пробуя на каждом шагу, не провалюсь ли в
какую-нибудь бездонную расщелину. Надо было идти осторожно, ибо враг  ожидал
моего  появления со стороны перевала, и я не стала подходить слишком близко,
чтобы  меня  не  заметили  из  укрепленного   здания,   а   втиснулась   под
грибообразный валун и принялась терпеливо ждать наступления ночи.

Стемнело  быстро,  но  снег все порошил, ночь оказалась светлой, и я не
отважилась приблизиться к  дому,  а  только  приподнялась,  подперев  голову
скрещенными передними ногами так, чтобы хорошо видеть его издали.

После полуночи снег перестал, но я не отряхивала его с себя, потому что
он сделал   меня   похожей  на  окружающие  предметы,  и  от  лунных  лучей,
пробивающихся меж облаками, сиял, как подвенечное платье, которого мне так и
не пришлось надеть. Потом я потихоньку  поползла  в  сторону  хорошо  видной
издали  темной глыбы дома, не спуская глаз с окна на втором этаже, в котором
тускло тлел желтоватый свет. Я прикрыла зрачки тяжелыми веками,  чтобы  луна
не слепила меня, а к слабому освещению я была приспособлена. Мне показалось,
что  в  этом  окне  что-то  двинулось и какая-то большая тень проплыла вдоль
стены, и я поползла быстрее, пока не добралась до подножья  постройки.  Метр
за  метром я стала взбираться по кладке, это было нетрудно, потому что между
камнями не было швов, их соединяла только собственная огромная тяжесть.  Так
я  добралась  до  нижнего  ряда  окон,  черневших,  как  крепостные бойницы,
предназначенные для пушечных жерл. Все они зияли мраком и  пустотой.  Внутри
царила  такая тишина, будто уже много веков единственной хозяйкой здесь была
смерть. Чтобы лучше видеть, я включила свое ночное зрение, сунула  голову  в
каменный  проем, открыла светящиеся глаза своих щупальцев, и в глубь комнаты
пошел от них фосфорический  свет.  Напротив  окна  я  увидела  сложенный  из
шершавых  плит  закопченный  камин, в котором давно остыла кучка рассохшихся
поленьев  и  обугленного  хвороста,  у  стены  заметила  скамью   и   ржавые
инструменты,  в  углу  виднелось продавленное ложе и груда каких-то каменных
ядер. Мне показалось странным, что вход ничем не защищен и дверь  в  глубине
распахнута  настежь,  но  именно  в  этом  я  увидела  западню и, не поверив
заманивающей пустоте, вновь бесшумно убрала голову  и  стала  взбираться  на
верхний  этаж.  К  окну,  из  которого  лился  тусклый свет, я и не подумала
приблизиться. Наконец я выбралась на крышу  и  на  ее  заснеженной  площадке
прилегла   по-собачьи,   решив  дождаться  здесь  рассвета.  Снизу  до  меня
доносились два голоса, но я не могла разобрать слов. Я лежала без  движения,
страшась той минуты, когда брошусь на противника, чтобы освободить Арродеса.
В  напряженном  оцепенении  я  мысленно  рисовала  картины  борьбы,  которая
завершится уколом жала, но в то же время, пытаясь проникнуть в тайное тайных
своей души, уже не доискивалась, как прежде, истоков движущей меня  воли,  а
искала  там  хотя бы самый слабый намек, знак, который открыл бы мне, одного
ли только человека я погублю.

Не знаю, когда исчезла моя нерешительность.  Я  все  еще  находилась  в
неведении,  все  так же не знала себя, но именно незнание того, прибыла ли я
как избавительница или как убийца, вновь вызвало у меня ощущение чего-то  до
сих  пор  неизвестного,  непонятно  нового,  придало  каждому моему движению
девственную загадочность и наполнило меня восторгом. Этот восторг очень меня
удивил, и я подумала, не в том ли снова проявилась мудрость моих создателей,
что я могла в моем безграничном могуществе видеть способность нести сразу  и
помощь,  и  гибель.  Но даже и в этом я не была уверена. Вдруг снизу до меня
донесся резкий короткий звук и сдавленный крик, а потом глухой стук,  словно
упало  что-то  тяжелое,  --  и  снова  тишина.  Тотчас  я  поползла с крыши,
перегнувшись через ее край так, что задняя пара ног и втулка жала находились
еще на кровле, грудь терлась  о  стену,  а  голова,  дрожа  от  усилий,  уже
дотягивалась до окна.

Свеча,  сброшенная  на пол, погасла, только фитиль еще тлел красноватым
огоньком. Усилив ночное зрение, я увидела лежащее под столом  тело,  залитое
кровью, которое при этом освещении казалось черным, и, хотя все мое существо
требовало  прыжка,  я  сначала  втянула  в  себя  воздух  с  запахом крови и
стеарина. Это был чужой человек, -- видимо, дело дошло до схватки и  Арродес
опередил меня. Как, когда и почему -- эти вопросы меня не занимали: меня как
громом  поразило то, что с ним, живым, я осталась в этом пустом доме один на
один, что нас теперь только двое. Я вся дрожала, суженая и  убийца,  отмечая
одновременно  немигающим  оком  мерные судороги этого большого тела, которое
испускало последнее дыхание. Вот сейчас бы уйти потихоньку в мир заснеженных
гор, чтобы только не оказаться с ним лицом к лицу, чтобы не встретились  две
пары  наших  глаз,  нет, три пары, поправила я себя и поняла, как безвыходно
осуждена  быть  смешной  и  страшной;  и   это   предчувствие   насмешки   и
издевательства,  все  во  мне подавив, толкнуло меня вперед, и я бросилась в
проем вниз головой, как паук на  добычу,  и,  уже  не  обращая  внимания  на
скрежет  брюшных пластин о подоконник, стремительной дугой перескочила через
недвижимого врага, целясь в дверь.

Не помню, как я распахнула  ее.  Сразу  за  порогом  начиналась  крутая
лестница,  и  на  ней навзничь лежал Арродес, упираясь подвернутой головой в
истертый камень нижней ступеньки. Наверное,  они  боролись  здесь,  на  этой
лестнице,  оттого  я почти ничего и не услышала. И вот он лежал у моих ног в
разорванной одежде, и его ребра вздымались, и я видела его  наготу,  которой
не знала и о которой думала только в первую ночь на королевском балу.

Он дышал хрипло. Видно было, как он силится разлепить веки, а я, откинувшись
назад и поджав брюшко, всматривалась сверху в его запрокинутое лицо, не смея
ни  коснуться его  ни отступить,  ибо, пока  он был  жив, я  не была  в себе
уверена. Жизнь уходила  из него с каждым вздохом, а  я помнила, что заклятие
лежит  на  мне  до  его последнего  дыхания,  поскольку  королевский  приказ
надлежит выполнить даже  во время агонии, и не хотела  рисковать, ибо он еще
жил и я  не знала, хочу ли его  пробуждения. Что, если бы он  хоть на минуту
открыл глаза и взглядом обнял бы меня всю, такую, какой я стояла перед ним в
молитвенной позе, бессильно смертоносная, с чужим  плодом в себе, -- было бы
это нашим венчанием или немилосердно предусмотренной пародией на него?

Но  он не  очнулся, и,  когда  рассвет прошел  между нами  в клубах  мелкого
искрящегося  снега, который  задувала  в  окно горная  метель,  он, еще  раз
простонав,  перестал  дышать,  и  тогда  уже  успокоенная,  я  легла  рядом,
прильнула к  нему, сжала в объятиях  и лежала так  при свете дня и  во мраке
ночи  все двое  суток пурги,  которая укрывала  нас нетающим  одеялом. А  на
третий день взошло солнце.


КОНЕЦ

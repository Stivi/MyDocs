
Станислав Лем.

Собысчас.


Ленинград, 1988

Как-то раз под вечер пришел знаменитый конструктор Трурль
к своему приятелю Клапауциусу грустный и задумчивый, а когда
попробовал тот его развеселить, рассказывая наисвежайшие
кибернетические анекдоты, Трурль вдруг сказал:

"--* Прошу тебя, не пытайся превратить мое подавленное
настроение в игривое, ибо грызет меня мысль столь же
правдивая, сколь и печальная: пришел я к выводу, что за всю
нашу столь деятельную жизнь не совершили мы ничего ценного!

Сказав это, окинул он взглядом, полным осуждения и
отвращения, стены кабинета Клапауциуса, увешенные богатой
коллекцией медалей, грамот и почетных дипломов в золоченых
рамках.

"--* На чем же столь строгий приговор основан? --- Спросил,
становясь серьезным, Клапауциус.

"--* Сейчас объясню. Мирили мы враждующие королевства,
строили для монархов, власти жаждущих, тренажеры, создавали
машины для рассказывания сказок и для охоты, одолевали мы
коварных тиранов и галактических разбойников, что на нас
нападали, но всем этим только себе удовольствие доставляли,
себя в собственных глазах возвеличивали, а для вселенского
блага ничего почти не сделали. Все наши деяния, направленные
на улучшение жизни народов, на которых мы в наших
межпланетных странствиях натыкались, ни разу не привели к
состоянию с-о-в-е-р-ш-е-н-н-о-г-о с-ч-а-с-т-ь-я. Вместо
решений радикальных и идеальных создавали мы только их
видимость, протезы да суррогаты, и заслужили поэтому титул
фокусников от онтологии, ловких софистов-практиков, но не
звание л-и-к-в-и-д-а-т-о-р-о-в з-л-а.

"--* Всякий раз, когда я слышу, как кто-нибудь рассуждает о
программировании м-и-р-о-в-о-г-о с-ч-а-с-т-ь-я, у меня
мурашки по спине бегают, --- ответил Клапауциус. --- Опомнись,
Трурль! Разве не знаешь ты бесчисленных примеров дел,
которые с того же начинались и в руины обратились,
наиблагороднейшие намерения под собой похоронив? Разве не
помнишь ты о трагической судьбе отшельника доброго, который
пытался осчастливить космос с помощью препарата под
названием "альтруизин"? Разве не знаешь ты, что можно
сколько угодно жизненные невзгоды уменьшать, справедливость
утверждать, солнечные пятна счищать, на шестерни
общественных машин бальзам лить, но счастья никакими
машинами не создашь? О царстве полного счастья можно лишь
тихо мечтать такими вот темными вечерами, создавать его в
своем воображении, одурманивать душу сладкими видениями, и
это все, чего может требовать истиный мудрец, приятель мой!

"--* Так только говорится, --- буркнул в ответ Трурль.

"--* Быть может, и в правду, --- добавил он через мгновение, --- осчастливить
существ, давно уже существующих --- задача
наразрешимая. Однако можно создать существ,
сконструированных с таким расчетом, чтобы им жизнь ничего,
кроме счастья, не несла. Вообрази, каким великолепным
памятником нашему конструкторству (которое ведь обратится же
когда-то в прах) была бы сияющая где-то в небесах планета, к
которой миллионы галактических племен с надеждой обращали бы
взоры, произнося: "да! Воистину --- счастье возможно! Возможна
вечная гармония! А доказал это великий Трурль при некотором
содействии своего друга Клапауциуса, и доказательство это
живет и пышно цветет перед нашими восхищенными взорами!

"--* Не сомневайся, что над проблемой, о которой ты
говоришь, я уже не раз размышлял, --- сказал Клапауциус. --- Возникает
здесь серьезное противоречие. Видно, ты урока,
который случай с добрым всем преподал, не запомнил, поэтому
и хочешь осчастливить созданий, которых до того не было, то
есть желаешь счастливцев на пустом месте сотворить. А ведь
полагалось бы сначала подумать над тем, можно ли вообще
осчастливить тех, кто не существует. Я сильно в этом
сомневаюсь. Ты должен сперва доказать, что небытие со всех
точек зрения хуже бытия, даже не особенно приятного, потому
что без такого доказательства фелицитологический
эксперимент, идея которого тебя увлекает, даст осечку. Тогда
к множеству несчастных, от которых во вселенной тесно, ты
добавил бы толпу новых, тобой сотворенных --- да только зачем?

"--* Эксперимент, конечно, рискованный, --- нехотя признался
Трурль. --- Но  я все равно считаю, что  нужно его провести.
Природа беспристрастна только с  первого взгляда, ибо хоть и
создает  она  что  попало  и  как  придется  ---  хороших  и
плохих, добрых и  жестоких, а смотришь потом,  и видишь, что
остались на сцене лишь плохие и жестокие, хорошими и добрыми
наевшиеся.  А  как  спохватываются  негодяи,  что  поступили
нечисто, так  сразу же смягчающие обстоятельства  или высшие
цели себе выдумывают. Вот  и выходит, что житейские мерзости
---  приправы,  аппетит  на  рай  и  прочие  подобные  места
заостряющие. Я  считаю, что  с этим пора  покончить. Природа
вовсе не зла, она только  глупа, как пробка, поэтому идет по
пути наименьшего  сопротивления. Нужно встать на  ее место и
создать счастливых  существ самим,  чтобы их  появление было
истиным врачеванием бытия,  с лихвой оправдывающим прошедшие
эпохи, полные стонов мученников, которых с иных планет разве
что на космических расстояниях не слышно. Какого дьявола все
живое  постоянно вынуждено  терпеть?  Если  бы беда  каждого
отдельного существа несла хотя бы такой импульс, какой капля
дождя несет, то за прошедшие века  разнесли бы они мир. Но и
пот их,  пока живут они,  и прах  их в могильных  склепах, и
опустевшие  их жилища  безмолвствуют, и  самыми совершенными
приборами не найдешь ты там следов их боли и мук, от которых
страдали вчера распавшиеся в прах сегодня.

"--* В самом деле, у мертвых нет забот, --- кивнул
Клапауциус. --- Ты прав, если имеешь в виду временность
всяких мучений.

"--* Но ведь появляются новые мученики! --- Повысил голос
Клапауциус. --- Или не понимаешь ты, что выполнение моего
замысла --- просто вопрос моей порядочности?

"--* Послушай. Каким же именно образом счастливое существо
(допустим, что ты его создал) смягчит бездну невзгод, что
уже произошли и тех несчастий, что по-прежнему происходят во
всем космосе? Или сегодняшний штиль отменяет вчерашнюю бурю?
Или день упраздняет ночь? Неужели ты не видишь, какой вздор
несешь?

"--* Так ты считаешь, что делать ничего не надо?

"--* Я этого не сказал. Ты можешь облегчать жизнь уже
живущих, по крайней мере, рискнуть на такую попытку. Тех же,
о которых ты говорил, счастливее никак не сделаешь. Ты
другого мнения? Ты считаешь, что выпустив в космос
испеченного тобой счастливца, ты хоть на йоту изменишь то,
что в нем уже было?

"--* А вот  и изменю! Изменю! --- Закричал  Трурль. Пойми же
меня  наконец!  Даже  если  поступок мой  не  затронет  тех,
кто  уже исчез,  то  он изменит  то  целое, частью  которого
они стали.  Отныне всякий  должен будет  сказать: ``огромные
усилия, противоестественные  цивилизации, уродливые культуры
были лишь прелюдией к  дню сегодняшнему, ко времени всеобщей
любви! Трурль, сей великий  муж, размышляя, пришел к выводу,
что злое прошлое должно быть уничтожено для создания доброго
будущего.  На бедности  научился  он  творить богатство,  на
невзгодах узнал цену наслаждения. Одним словом, именно своим
безобразием  побудил  его  космос  сотворить  добро!''  Наша
эпоха окажется  поготовительной и  вдохновляющей, понимаешь?
Благодаря ей наступит эра  истинного счастья. Ну как, убедил
я тебя?

"--* Под южным крестом лежит царство короля троглодика, --- отозвался
Клапауциус. --- Нет для него милее картины, чем
виселица, а эту любовь свою он тем объясняет, что нельзя
таким сбродом, как его подданные, править иначе. Когда
появился я в его царстве, хотел он меня сразу же схватить,
но смекнул, что я могу его в порошок стереть. Тогда
испугался троглодик, потому что не сомневался --- если он меня
не одолеет, то я с ним разделаюсь. И чтобы меня к себе
расположить, собрал он тотчас же свой ученый совет и получил
от него моральную доктрину власти, как раз для таких случаев
созданную. Объяснили ему эти купленные им мудрецы, что чем
им хуже, тем сильнее жаждут они улучшений, а поэтому то, кто
такое творит, что уж и вытерпеть невозможно, чрезвычайно
сильно это улучшение приближает. Обрадовался король этим
речам, потому что выходило, что никто так активно, как он,
не борется за грядущее добро, ибо надлежащими мерами
мечтающих о благе для человечества к действиям побуждает.
Так что счастливцы твои должны троглодику памятников
понаставить, а сам ты должен ему спасибо сказать. Не так ли?


"--* Глупая и циничная притча! --- Выпалил задетый за живое
Трурль. --- Думал я, что ты меня поддержишь, однако вижу, что
ты только пытаешься ядовитым скептицизмом и софизмами
принизить благородство моих замыслов. А ведь в замыслах этих --- спасение
космоса.

"--* А, так  ты хочешь стать спасителем  космоса? --- Сказал
Клапауциус.  ---  Трурль,  нужно  было бы  мне  сейчас  тебя
связать  и кинуть  на ту  кровать, чтобы  было у  тебя время
опомниться,  однако боюсь  я,  что долго  пришлось бы  этого
ждать. Поэтому  скажу только: не твори  счастья, не подумав!
Не совершенствуй бытия одним махом!  А если даже и сотворишь
ты  счастливцев  (в чем  я  сомневаюсь),  то ведь  останутся
по-прежнему  и  все  остальные,  возникнет  зависть,  ссоры,
раздоры, и, кто знает, не  встанет ли перед тобой неприятная
альтернатива: либо счастливцы твои завистникам уступят, либо
должны будут для достижения полной гармонии этих назойливых,
обиженных да ущербных, всех до одного перебить.

Вскочил было Трурль на ноги, да опомнился, кулаки разжал,
так как драка с приятелем была бы не самым подходящим
началом для э-р-ы п-о-л-н-о-г-о с-ч-а-с-т-ь-я, которую он
уже твердо решил создать.

"--* Прощай, --- заявил он холодно. --- Проклятый агностик,
маловер, рабски полагающийся на случайности естественного
хода событий. Не словами буду с тобой спорить, но делами! По
плодам трудов моих поймешь ты вскоре, что правда на моей
стороне.

Вернулся Трурль домой и серьезно задумался, потому
что, хоть в конце спора с Клапауциусом и дал понять, будто у
него есть конкретный план действий, но было это далеко не
так. Откровенно говоря, не имел он и малейшего понятия о
том, с чего начинать. Достал он тогда с полок своей
библиотеки громадные тома, посвященные описанию бесчисленных
общественных систем и проглотил их с достойной удивления
скоростью. А так как, несмотря на все, слишком медленно
наполнялся нужными фактами, то приволок из подвала восемьсот
кассет ртутной, оловянной, ферромагнитной и криотронной
памяти, приладил их всех проводами к своему естеству и за
несколько секунд набил себе голову четырьмя триллионами бит
лучшей и отборнейшей информации, какую можно было только
раздобыть среди звезд, на планетах, а также на остывших
солнцах, заселенных прилежными летописцами. И такая у него в
голове началась давка, что посинел бедняга, глаза у него
полезли на лоб, зубы застучали, мурашки по телу забегали, и
затрясся он с головы до пят, как будто не историографией и
историософией был наполнен, а чумой поражен. Потом однако
собрался он с силами, встряхнулся, вытер лоб, оперся все еще
дрожавшими коленями о край стола и сказал себе:

"--* Видно, все обстояло и обстоит еще хуже, чем я думал!

Целый час точил он карандаши, подливал чернил в
чернильницы, стопками укладывал белые карточки, но из этих
приготовлений ничего такого не возникло, поэтому разозлился
он слегка и сказал себе:

"--* Должен я был просто для порядка познакомиться с
книгами древних, архаичных мудрецов, хотя всегда и
подозревал, что эта старая мура современного конструктора
ничему научить не может. Ладно, так уж и быть! Проштудирую и
этих допотопных мыслителей, защитившись тем самым от выпадов
Клапауциуса, который их, конечно, тоже никогда не читал (а
кто их вообще читает?), А только тайком выписывает себе из
их книг цитаты, чтобы меня злить и в невежестве упрекать.

Сказав это, он действительно взялся за книги дряхлые и
трухлявые, хотя совсем ему этого не хотелось.

В середине ночи, окруженный книгами, что, открытые,
падали ему на колени, так как сталкивал он их нетерпеливо со
стола, сказал он себе:

"--* Вижу я, что придется не только жизнь разумных существ
исправлять, но и то, что они нафилософствовали. Прародителем
жизни был океан, который у берегов илом покрылся. Возникла
грязь жидкая, или коллоид. Солнце грязь пригрела, загустела
грязь, потом молния в нее трахнула, аминокислотами (от слова
аминь) все заквасилось, и возникла протоплазма, которая со
временем на сухое место выбралась. Выросли у нее уши, чтобы
слышать, как добыча удирает, а также ноги и зубы, чтобы ее
догнать и съесть. А если не выросли ноги, или коротки были,
то ее саму съели. Разум же сотворила эволюция, ибо что с ее
точки зрения глупость и мудрость, а также добро и зло? Добро --- это
тогда только, когда я кого-то съем, а зло --- когда
меня съедят. То же и с разумом: съеденный, если такое с ним
случилось, глупее съевшего, потому что не может быть умным
тот, кого нет, а кого съели --- того нет и в помине. Но тот,
кто всех переест, тот сам сдохнет. Поэтому есть стали в
меру. Со временем живая органика стареет, ибо материал это
непрочный. Тогда, в поисках лучшего, придумали мягкотелые
металл. Сами себя в железе скопировали, потому что легче
всего взять уже готовое, поэтому до создания истинно
совершенных форм дело не дошло. Ба! Не будь органика такой
непрочной, ведь совершенно по другому философия бы
развивалась: видно, влияет на нее материал, то есть чем
совершеннее строение разумного существа, тем отчаянее жаждет
оно иметь совершенно другое строение. Если в воде живут --- утверждают,
что рай на --- суше, если на суше, то --- в небе,
если имеют крылья, то считают идеалом плавники, а если ноги,
то намалюют себе крылья и восхищаются: "ангел!" Удивительно,
что я раньше этого не заметил. Назовем это правило
к-о-с-м-и-ч-е-с-к-и-м з-а-к-о-н-о-м т-р-у-р-л-я: всякий
разум творит себе а-б-с-о-л-ю-т-н-ы-й и-д-е-а-л, исходя из
несовершенства собственной конструкции. Надо где-нибудь это
записать на тот случай, если когда-нибудь займусь созданием
основ философии. А сейчас надо браться за конструирование.
Начнем с закладывания добра --- но только что это такое? Его
безусловно нет там, где никого нет. Водопад для скалы --- ни
добро и ни зло, так же как и землетрясение для озера.
Поэтому создадим к-о-г-о-н-и-б-у-д-ь. Но вот вопрос --- будет
ли ему хорошо? Ведь откуда мы узнаем, что кому-то хорошо?
Предположим, что я вижу, как кто-то Клапауциусу делает зло.
И что же? Одна половина души огорчится, а другая --- обрадуется,
не так ли? Это как-то очень запутанно.

Возможно, кому-то хорошо по сравнению с соседом, но он
ничего о соседе не знает, и поэтому своего счастья не
чувствует. Следует, значит, создать существ, имеющих перед
глазами себе подобных, в муках живущих. Были бы они
достаточно удовлетворены сами этим контрастом? Быть может,
но все-таки как-то это гадко. А значит, нужен здесь дроссель
или трансформатор.

Закатал он тогда рукава и за три дня построил
с-о-з-е-р-ц-а-т-е-л-ь б-ы-т-и-я с-ч-а-с-т-л-и-в-ы-й, машину,
которая сознанием, в катодах ее горящем, с каждой
постигнутой ею вещью соединялось, и не было на свете ничего,
что не доставляло бы ей утехи. Уселся перед ней Трурль,
чтобы проверить, то ли у него вышло. Присев на трех
металлических ногах, водил созерцатель вокруг
телескопическими глазами, а когда падал его взгляд на доску
заборную, на камень или старый башмак, то безмерно он
восторгался, так что даже тихонько постанывал от великой
радости, его распиравшей. Когда же солнце зашло, и заря на
небе заалела, то даже присел он от восхищения.

"--* Клапауциус, разумеется, скажет, что сами стоны и
приседания ни о чем еще не говорят, --- сказал Трурль, все
еще не удовлетворенный. --- Он потребует доказательств.

Встроил он  тогда созерцателю  в брюхо  измерительный прибор
с  золоченой  стрелкой,  который  отградуировал  в  единицах
счастливости,  названных   им  гедонами,   или,  сокращенно,
гедами. За один гед принял он то количество экстаза, которое
получишь, если  пробежишь четыре  мили в ботинке  с торчащим
гвоздем,  а потом  гвоздь выпадет.  Помножил путь  на время,
поделил  на  остроту  гвоздя, вынес  за  скобки  коэффициент
жесткости пятки,  и таким образом перевел  счастье в систему
"сгс". Этим  он немного себя утешил.  Уставившись на залитый
маслом рабочий  фартук Трурля, суетившегося  перед прибором,
созерцатель, в  зависимости от  угла и  полной освещенности,
показывал  от  11.8  до  18.9 гедов  на  пятно,  заплатку  и
секунду.  Успокоился конструктор.  Вычислил  он заодно,  что
один килогед --- это столько, сколько старцы чувствовали, за
Сусанной в  купели подглядывая, что мегагед  --- это радость
смертника,  перед  самой   висилицей  помилованного.  Тогда,
увидев,  что  все  удается точно  вычислить,  послал  одного
самого плохонького лабораторного робота за Клапауциусом.

Когда тот пришел, сказал ему Трурль:

"--* Смотри и учись. Обошел Клапауциус машину кругом. Та
сразу же большие свои телеобъективы на него направила,
присела и пару раз простонала. Удивился конструктор этим
глухим звукам, но виду не подал и спросил только:

"--* Что это?

"--* Счастливое существо, --- сказал Трурль, --- а называется
с-о-з-е-р-ц-а-т-е-л-ь б-ы-т-и-я с-ч-а-с-т-л-и-в-ы-й,
сокращенно же --- собысчас.

"--* И что же делает этот собысчас?

Трурль почувствовал в этих словах иронию, однако мимо
ушей ее пропустил.

"--* Активным способом непрерывно познает! --- Объяснил он. --- И
не просто познает, замечая, а делает это интенсивно,
сосредоточенно и трудолюбиво, а то, что он познает, у него
невыразимую радость вызывает. И радость эта, по его анодам и
катодам разлившись, дает ему высокое блаженство, признаками
которого как раз и явились звуки, которые ты слышал, когда
обозревал он твою довольно невзрачную наружность.

"--* Значит, эта машина извлекает активное наслаждение из
сущности самого познания?

"--* Именно так! --- Сказал Трурль немного потише, ибо не
был уже так уверен в себе, как за мгновение до этого.

"--* А это, конечно, фелицитометр, отградуированный в
единицах наслаждения существованием? --- Указал Клапауциус
на шкалу с золоченой стрелкой.

"--* Именно так... И начал тут Клапауциус разные вещи
собысчасу показывать, внимательно следя за отклонением
стрелки. Трурль, успокоившись, объяснил ему теорию гедов,
или теоретическую фелицитометрию. Слово за словом, вопрос за
вопросом --- бежала беседа, но вдруг спросил Клапауциус:

"--* Скажи-ка, сколько гедов содержится в том, что тот,
кого триста часов били, в свою очередь лоб тому, кто его
бил, расшиб?

"--* А, так это очень просто! --- Обрадовался Трурль, и сел
было за вычисления, но услышал громкий смех своего приятеля.
Разозлился он и вскочил, Клапауциус же, все еще смеясь,
сказал:

"--* Ведь ты же сказал, что за исходный принцип принял
добро, дорогой мой? Ну что ж, эталон ты выбрал подходящий.
Продолжай в том же духе, и все пойдет отлично. А пока
прощай.

И ушел, оставив совершенно убитого Трурля.

"--* А, черт меня подери, --- стонал конструктор, а стоны
его перемежались с восторженными постанываниями собысчаса,
которые так его разозлили, что запихнул он машину в чулан,
закидал ее старым хламом и закрыл на замок.

Сел он потом за пустой стол и так себе сказал:

"--* Перепутать эстетический восторг с добром --- ну и осел
же я! Да и вообще, есть ли у собысчаса разум? Надо с самого
начала подойти к вопросу иначе, с самого атомного уровня.
Счастье --- конечно, радость --- без сомнения, но не за чужой
счет! Не из зла следующее! Вот так! Но что такое зло? Вижу
я, что до сих пор в своей конструкторской деятельности
преступно теорией пренебрегал.

Восемь дней не отдыхал конструктор, не спал, не выходил
на улицу, а только изучал книги безмерно ученые, о добре и
зле рассуждающие. Оказалось, большинство мудрецов сходится
на том что самая важная вещь --- это взаимопомощь и взаимная
благожелательность. То и другое должны друг другу разумные
существа в любом случае оказывать. Правда, именно под этим
лозунгом и огнем жгли, и жидким оловом поили, и
четвертовали, и на кол сажали, и кости ломали, а в самые
ответственные исторические моменты даже шестерками лошадей
разрывали. Для духа же, как и для тела, в форме
разноообразных пыток друг другу доброжелательность
выказывали.

"--* Предположим, --- сказал себе Трурль, --- что совесть
пробуждалась бы не в самих злодеях, а, напротив, в близких
людях, их окружающих. Что бы из этого вышло? Да нет, это не
пойдет: ведь тогда бы ближнего моего угрызения мучали, а я
бы еще легче мог в грехах погрязнуть. Может, встроить в
обычную совесть умножитель угрызений, чтобы каждое новое зло
в тысячу раз больше, чем предыдущее, мучало? Но тогда каждый
из простого любопытства что-нибудь злое сотворит, чтобы
проверить, вправду ли новые угрызения такими дьявольски
сильными будут --- и до конца дней совесть будет его мучать...
Можно было бы сделать совесть с обратной связью и
блокировкой стирания... Нет! Это не годится, потому что кто
же будет эту блокировку отключать? А если приладить
трансмиссию --- один чувствует за всех, все --- за одного? Но
ведь это уже было --- именно так действовал альтруизин...
Тогда можно так: у каждого в туловище вмонтирован маленький
детонатор с приемником, и если ему, за его злые и подлые
дела, зла не менее десяти ближних пожелают, их желания на
гетеродинном входе суммируются и тот, кому они адресованы,
на воздух взлетает. Да неужели тогда каждый как чумы не
избегал бы зла? Конечно избегал бы, да еще как! Однако..,
Что это за счастливая жизнь --- с миной замедленного действия
в области желудка? К тому же возникали бы тайные заговоры
против людей: получилось бы, что достаточно десяти негодяям
невиновного невзлюбить и он --- на кусочки... Тоже не годится.
Что же это такое! Мне, галактики, как шкафы, двигавшему, не
решить, казалось бы, простой инженерной проблемы?! Допустим,
что в неком обществе каждый упитан, румян и весел, с утра до
вечера поет, подпрыгивает и хохочет от того, что другим
добро делает, да при этом весь пылает от энтузиазма, а если
спросить его, то в голос кричит, что просто ужасно рад
существованию --- и своему, и окружающих... Достаточно ли
счастливым было бы такое общество? Ведь там никто никому зла
сделать не может! А почему не может? Потому что не хочет! А
почему не хочет? Потому что это ему ничего не даст. Вот и
решение! Не это ли гениальный в своей простоте план для
массового производства счастья?! Не в такой ли жизни все
счастьем переполнены будут?! Посмотрим, что тогда скажет
этот циник-мизантроп, этот скептический агностик, Клапауциус --- уж
тогда-то ему не до насмешек и издевательств будет!
Пусть-ка попробует придраться! Ведь если каждый будет
стараться ближнему все лучше и лучше делать, да так, что
лучше уж и нельзя... Гм, а не замучаются ли они, не выбьются
ли из сил, не выдохнутся ли быстро под градом и лавиной этих
добрых дел? Ладно, вмонтируем маленькие редукторы, или
какие-нибудь дроссели, счастьеупорные перегородки, экраны,
изоляцию... Сейчас, только не надо спешить, чтобы опять
что-нибудь не проморгать. Итак примо --- веселые, секундо --- упитанные,
терцо --- подпрыгивающие, кварто --- румяные, квинто --- удовлетворенные,
сексто --- участливые... Хватит, можно
начинать.

До обеда поспал он немножко, так как сильно утомили его
эти размышления, а потом, решительный и бодрый, встал,
чертежи начертил, ленты программные наперфорировал,
алгоритмы рассчитал, и, наконец, сотворил счастливое
общество из девятисот персон. Чтобы господствовало в нем
равенство, сделал всех совершенно одинаковыми. Дабы из-за
еды и питья не ссорились, сделал так, что не пили они и не
ели, а холодный атомный огонек служил им источником энергии.
Уселся он потом на завалинку и до захода солнца смотрел как
они прыжками и криками свою радость выражают, как добро
творят, друг друга по голове гладят, камни друг перед другом
с дороги убирают, какие они крепкие, бодрые, веселые, как
задорно и беспечно жизнь их бежит. Если кто ногу вывихнет --- аж
черно становилось от сбежавшихся, и не от любопытства, а
из-за могучей потребности участие оказать. От избытка
сочуствия одному из них даже ногу оторвали, вместо того,
чтобы вправить, но подрегулировал он им редукторы и
реостаты, а потом позва Клапауциуса. Присмотрелся тот к их
радостной беготне, выслушал объяснения с довольно кислой
миной и спросил:

"--* А опечалиться они могут?

"--* Что за глупый вопрос. Разумеется, нет, --- ответил
Трурль.

"--* Значит, потому они все время скачут и во весь голос
вопят, потому такие румяные и добрые, что им хорошо?

"--* Именно!

А так как Клапауциус не просто на похвалы поскупился, а
вообще ничего не похвалил, то добавил Трурль сердито:

"--* Быть может, зрелище это монотонно и менее живописно,
чем батальные сцены, но моей задачей было осчастливить, а
не кому-то там спектакль устроить!

"--* Если они  делают то, что делают, потому  что делать это
обязаны, приятель  мой, ---  отозвался Клапауциус, ---  то в
них  ровно  столько же  добра,  сколько  в трамвае,  который
потому тебя не переедет, если ты на тротуаре стоишь, что ему
с рельсов не  сойти. Не тот, Трурль,  счастье творения добра
познает, кто  должен других неустанно по  голове гладить, от
восторга вопить да  камни с дороги убирать, а  тот лишь, кто
может  и  печалиться,  и  рыдать, и  камнем  другому  голову
размозжить, но  по доброй  воле и  сердечной охоте  этого не
делает!  Ты  же создал  пародию  на  высшие идеалы,  которые
удалось тебе изрядно опошлить!

"--* Что ты такое говоришь?! Они же, все таки, разумные
существа, --- пробормотал обескураженный Трурль.

"--* Да? --- Спросил Клапауциус. --- Сейчас посмотрим. С
этими словами подошел он к Трурлевым творениям, первому, кто
ему попался, дал с размаху в лоб и спросил:

"--* Счастлив ли ты?

"--* Безумно! --- Ответил тот, схватившись за голову, на
которой вскочила шишка.

"--* А теперь? --- Спросил Клапауциус, и так ему врезал, что
тот вверх тормашками полетел. Еще не успел бедняга встать,
еще песок выплевывал, а уже кричал:

"--* Счастлив я! Хорошо мне жить!

"--* Вот так, --- сказал коротко Клапауциус онемевшему
Трурлю и ушел.

Огорчился Трурль невыразимо. Свел он одного за другим
своих счастливцев в лабораторию и разобрал их там до
последнего винтика, а они не только этому не сопротивлялись,
но ему, как могли, помогали, ключи и клещи подавали, или
даже молотком себя по черепу били, если его крышка слишком
плотно была насажена и сниматься не хотела. Сложил он детали
обратно в ящики и на полки, сорвал с досок чертежи, порвал
их на куски, уселся за стол, под философско-этическими
трудами прогнувшийся и глухо простонал:

"--* Хорошенькая история! Ну и опозорил же меня этот
мерзавец, этот разбойник, мой так называемый приятель!

Вынул он из шкафа модель пермутатора --- прибора, который
всякую информацию в аспекте взаимопомощи и всеобщей
благожелательности перерабатывал, положил его на наковальню
и разбил на куски, но не стало ему от этого легче.
Поразмыслил он, повздыхал и взялся за воплощение другого
замысла. На этот раз вышло из под его рук общество немалое --- три
тысячи рослых парней, которые тут же выбрали себе
руководителей тайным и равным голосованием, а затем занялись
разнообразными делами --- кто дома строил и изгороди ставил,
кто законы природы открывал, а кто играл и забавлялся. В
голове у каждого из новых созданий Трурля был гомеостазик, а
в том гомеостазике --- два прочно приваренных ограничителя,
между которыми могла их свободная воля гулять, как это ей и
подобает, а снизу находилась пружина добра, которая на свою
сторону много сильнее тянула, чем другая, поменьше, колодкой
приторможенная, для разрушения и уничтожения
предназначенная. Имел каждый гражданин еще и датчик совести
огромной чувствительности, помещенный в зубастые клещи,
которые начинали его грызть, если сходил он с пути
праведного. Проверил Трурль на специальном лабораторном
экземпляре, что, когда до мук совести доходило, корчило от
них несчастного почище, чем от икоты или даже от пляски
святого витта. Только раскаянием, делами благородными,
альтруизмом потихоньку конденсатор заряжался и своим
импульсом зубы, угрызающие совесть, разводил и маслом датчик
умасливал. Нет слов, хитро было это придумано! Подумывал
Трурль над тем, чтобы угрызения совести дополнительной тягой
с зубной болью соединить, но потом решил этого не делать,
потому что боялся, что Клапауциус опять начнет о
принуждении, свободу воли исключающем, талдычить. Было бы
это, безусловно, неверно, потому что имели эти существа
статистические приставки, и никто, даже сам Трурль, не мог
сказать, что и как они делать будут. Целую ночь будили
Трурля радостные крики, и шум этот сильно эму досаждал. "Но
теперь уже," --- сказал он себе, --- "Клапауциусу ни к чему не
прицепиться. Счастливы они, но не не по программе и не по
обязанности, а только статистическим, эргодическим и
вероятностным образоми. Моя взяла!" С этой мыслью глубоко
уснул он и спал до самого утра.

Так как не застал он Клапауциуса дома, то проискал его до
самого обеда, а найдя, повел его к себе, прямо на
фелицитологический полигон. Осмотрел Клапауциус дома,
изгороди, будки, надписи, дворец правления и его залы,
делегатов, граждан, поговорил с теми и с другими, а в
переулке снова попробовал одному низенькому гражданину в лоб
дать, но взяли его тут же трое других за руки и за ноги, и
моментально выкинули из города через ворота, и хотя следили
они за тем, чтобы шею ему не сломать, однако морщился он, из
придорожной канавы выбираясь.

"--* Ну как? --- Спросил Трурль, сделав вид, что вовсе не
заметил позора Клапауциуса. --- Что скажешь?

"--* Я приду завтра утром, --- ответил тот. Разумеется,
уходя, удовдетворенно улыбался Трурль. На следующий день,
около полудня, снова вошли оба конструктора в город, и
увидели большие перемены. Остановил их сразу же патруль, и
старший по званию сказал Трурлю

"--* почему у тебя такой кислый вид? Или пения птичек не
слышишь? Или цветов не видишь? Голову выше! Другой, пониже
рангом, добавил:

"--* держаться бодро, браво, весело!

А третий ничего не сказал, а только бронированным кулаком
треснул конструктора по спине, да так, что там что-то
хрустнуло. Затем все повернулись к Клапауциусу, но тот, не
ожидая дальнейшего, сам так лихо вытянулся, так убедительно
восторг продемонстрировал, что патруль оставил его в покоее,
дальше пошел. Сцена эта на творца нового общества произвела
большое впечатление. Уставился он, раскрыв рот, на площадь
перед дворцом счастья, где, выстроившись в шеренгу по
четыре, жители по команде издавали крики восторга.

"--* Бытию --- ура! --- Орал кто-то с эполетами и с бунчуком,
и стройный хор голосов отвечал ему:

"--* ура! Ура! Ура

не успел Трурль и слова сказать, как оказался вместе с
приятелем в шеренге, крепко схваченный, и до самого вечера
муштровали их, обучая как себе зло, а ближнему в шеренге
добро творить --- все на три счета. Командиры же, которых
звали фелиционерами, то есть с-т-р-а-ж-а-м-и
в-с-е-о-б-щ-е-г-о с-ч-а-с-т-ь-я, а сокращенно --- всесчасами,
следили за тем, чтобы каждый в отдельности и все разом
выражали полное удовлетворение и совершеннейшее блаженство,
что на практике оказалось крайне утомительным. Во время
краткого перерыва в фелицитологических маневрах удалось
Трурлю и Клапауциусу улизнуть из шеренги и спрятаться за
забором. Затем, прижимаясь к земле, как при артеллерийском
обстреле, добежали они до дома Трурля и для надежности
спрятались на чердаке. Это было сделано вовремя, ибо и по
дальним окрестностям сновали уже патрули, прочесывая район в
поисках несчастных, огорченных и грустных, которых быстро
обрабатывали прямо на месте. Трурль, ругаясь на чем свет
стоит, искал способ ликвидировать последствия эксперимента,
принявшего такой трагический оборот. Клапауциус же хихикал в
кулак. Не придумав ничего лучше, выслал Трурль в город,
скрепя сердце, отряд демонтажеров, причем для большей
надежности и в огромном секрете от Клапауциуса так их
запрограммировал, чтобы не могли они попасться на удочки
лозунгов, превозглашающих всеобщую благожелательность и
пвосеместную взаимопомощь. Когда столкнулся этот отряд со
всесчасами, то только искры посыпались. Сражалась фелиция за
дело всеобщего счастья геройски. Вынужден был Трурль послать
дополнительные отряды со сдвоенными тисками и клещами, и
тогда перешла стычка в нешуточный бой, настоящую войну,
потому что сражались обе стороны с огромным
самопожертвованием, разя друг друга уже картечью и
шрапнелью.

Когда же наконец вышли конструкторы во двор, то поле боя,
освещенное молодым месяцем, представляло собой печальное
зрелище. В покрытом дымом городе лежали тут и там не
разобранные в спешке до конца фелиционеры, продолжая и в
своей механической агонии выражать крайнюю и непоколебимую
приверженность идее всеобщего добра. Трурль даже не пытался
скрыть свое лицо, искаженное гневом и отчаянием. Не понимал
он совешенно, где же допущена ошибка, которая счастливцев в
держиморд превратила.

"--* Лозунг всеобщей благожелательности, мой дорогой, может
принести разные плоды, --- снисходительно объяснил ему
Клапауциус. --- Тот, кому хорошо, сразу же хочет, чтобы и
другим хорошо было, а строптивых начинает дубиной в рай
загонять.

"--* Значит, добро может родить зло! О как же коварна
природа вещей! --- Крикнул Трурль. --- Тогда объявляю я бой
самой природе! Прощай, Клапауциус! Сейчас ты видишь меня
побежденным, но проиграть сражение --- не значит проиграть
войну!

Засел он сразу же, словно отшельник, за книги и рукописи --- мрачный,
но от этого еще более настойчивый. Здравый смысл
подсказал ему, что не мешало бы перед следующим
экспериментом окружить свой двор стенами, а в амбразурах
поставить пушки, но недостаточно ему этого показалось для
того, чтобы начать творение всеобщей благожелательности.
Тогда решил он в будущем создавать только уменьшенные модели
в масштабе 1:100.000, в рамках микроминиатюрной
экспериментальной социологии. Начертал он на стенах
мастерской лозунги, дабы постоянно иметь их перед глазами:

1) сердечная добровольность,

2) ласковая кроткость,

3) деликатная благожелательность

4) чуткая забота

и взялся за претворение этих лозунгов в жизнь. Для начала
смонтировал он под микроскопом тысячу электрочеловечков,
снабдив их невеликим разумом и небольшой тягой к добру, так
как фанатизма теперь уже побаивался. Копошились они довольно
вяло в шкатулочке, предоставленной им для проживания и
похожей, из-за этого мерного и монотонного движения, на
часовой механизм. Тогда добавил он им немного мудрости,
подкрутив в мозгу винтики. Тогда зашевелились они порезвее,
и, сделав из опилок инструменты, начали долбить стенки и
дно. Потом увеличил потенциал добра. Сразу же образовались
благотворительные общества, каждый бегал в поисках тех, кому
помочь можно было, появился спрос на вдов и сирот, особенно
на слепых. Такими заботами их окружали, такое внимание
оказывали, что некоторые убогие прятались за медными петлями
шкатулки, и началось уже настоящее общественное бедствие.
Нехватка вдов и сирот вызвала кризис и, не находя на этом
свете, то есть в шкатулке, объектов, пригодных для
проявления крайне активной благожелательности, на
восемнадцатом поколении создали микрочеловечки веру в
а-б-с-о-л-ю-т-н-о-г-о с-и-р-о-т-у, которого до конца утешить
и осчастливить вообще невозможно. Через эту форточку избыток
альтруизма, перешедшего в нечто метафизическое, выпускался в
трансцендентальную бесконечность. Заселили они обильно
загробный мир --- появились среди убогих д-е-в-а-в-д-о-в-а и
г-о-с-п-о-д-ь, также пригодные для горячего сочуствия. Таким
образом этот бренный мир был забыт, и церковные организации
поглотили светские. Не так все это Трурль себе представлял.
Добавил он рационализма, скептицизма и трезвости, и
успокоилось все, но ненадолго. Появился
э-л-е-к-т-р-о-в-о-л-ь-т-е-р, заявивший, что никакого
абсолютного сироты нет, а есть только к-о-с-м-о-с --- куб,
силами природы сотворенный, а сиротинцы-абсолютисты его
прокляли. В это время понадобилось Трурлю сходить в магазин,
а когда через два часа он вернулся, то шкатулка так и
скакала по столу, потому что началась война за веру. Добавил
он альтруизма, но от этого только начало что-то
потрескивать, снова ввел несколько порций разума --- стихло,
но вскоре движение возобновилось и из прежней неразберихи
стали возникать прямоугольники, марширующие продозрително
ровным шагом. В шкатулке прошел век, от абсолютистов и
вольтерьянцев и следа не осталось. Все рассуждали только о
в-с-е-о-б-щ-е-м д-о-б-р-е и писали о нем совершенно светские
трактаты. Но возник вскоре вопрос о происхождении всего
общества. Одни полагали, что возникло оно из праха от
латунных петель, другие же --- что причиной послужило
космическое вторжение извне. Чтобы разрешить этот волнующий
вопрос, началось строительство б-о-л-ь-ш-о-г-о с-в-е-р-л-а,
которое должно было к-о-с-м-о-с, то есть шкатулку,
просверлить и исследовать, что же снаружи находится. А так
как могли там неведомые силы находиться, то взялись
одновременно и за литье пушек. Так это Трурля разочаровало и
обеспокоило, что разобрал он всех их как можно быстрее, и
сказал себе, чуть не плача:

"--* разум ведет к бездушию, а добро --- к безумию! Как же
это так, откуда же такое инженерно-историческое
противоречие.

И решил он разобраться в этом особо. Вытащил из чулана
свою первую модель, старый созерцатель и, когда тот начал
постанывать от эстетического восторга перед старым хламом,
подключил к нему небольшую мыслящую приставку. Собысчас
моментально стонать перестал. Спросил Трурль, что ему еще
нравится, а тот ответил:

"--* нравиться то мне по-прежнему все нравится, но
сдерживаю я свое восхищение рассудком, ибо хочется мне
сначала понять, почему же мне все нравится, то есть откуда,
а также для чего, то есть с какой целью. И вообще, кто ты
такой, чтобы отвлекать меня от созерцаний и размышлений
вопросами? Какое тебе до меня дело, а? Чувствую я, что мог
бы и тобой восхититься, но разум говорит мне, что нужно
этому внутреннему порыву сопротивляться, ибо может это быть
ловушкой, на моем пути поставленной.

"--* Что касается того, какое мне до тебя дело, --- неосторожно
сказал Трурль, --- то я тебя сотворил, и чтобы
дух твой что-то от этого получил, сделал так, что между
тобой и миром полная гармония царит.

"--* Гармония? --- Переспросил собысчас, внимательно нацелив
на Трурля свои объективы. --- Гармония, уважаемый? --- А
почему у меня три ноги? А голова --- вверху? Почему на левом
боку у меня заклепки медные, а на правом --- железные? Для
чего у меня пять глаз? Ответь же, уважаемый, если правда ты
меня из небытия извлек.

"--* Три ноги --- потому что на двух ты не устоял бы, а
четыре --- лишняя трата материала. --- Объяснил Трурль. --- Пять
глаз --- потому что столько было линз под рукой, а что до
заклепок --- то сталь у меня кончилась, когда корпус тебе
делал.

"--* Тоже мне, --- насмешлимо фыркнул собысчас. --- Ты хочешь
сказать что все это --- простая игра случая, чистое стечение
обстоятельств, следствие обыкновенного безразличия? И в эту
ерунду я должен поверить?

"--* Я лучше знаю, как все было, раз я тебя создал! --- Ответил
Трурь, слегка рассерженный такой самоуверенностью.

"--* Лично я вижу две возможности, --- быстро ответил
собысчас. --- Первая --- что ты нахально врешь. Этот вариант
пока отложим как недоказанный. Другая --- что, со своей точки
зрения, ты говоришь правду, из чего ничего еще не следует,
так как правда это --- лишь для твоего малого знания, а на
самом деле --- ложь.

"--* Как это понять?

"--* А так, что то, что тебе кажется простым стечением
обстоятельств, на самом деле таковым быть не может. Нехватку
стальных заклепок ты принял за случайность, но откуда тебе
известно, что это не проявление в-ы-с-ш-е-й
н-е-о-б-х-о-д-и-м-о-с-т-и? Наличие медных заклепок
показалось тебе только удобством, но и тут произошло
вмешательство п-р-е-д-о-п-р-е-д-е-л-е-н-н-о-й
г-а-р-м-о-н-и-и. Аналогично, по числу моих ног и глаз можно
понять т-а-й-н-у в-ы-с-ш-е-г-о п-о-р-я-д-к-а,
и-з-в-е-ч-н-о-е з-н-а-ч-е-н-и-е этих количеств, отношений и
пропорций. Ведь и три, и пять --- простые числа, а, заметь, они
могли бы делиться друг на друга. Три раза по пять --- это
пятнадцать, то есть единица и пятерка, складываем --- получаем
шесть, а шесть, деленное на три, дает два, то есть
количество моих цветов, так как я с одной стороны медный, а
с другой --- железный собысчас. Могло бы такое точное
соотношение возникнуть случайно? Смешно об этом и думать! Я --- существо,
выходящее за пределы твоего, примитивный
слесарь, умственного горизонта! Если вообще есть хоть
немного истины в том, что ты меня сделал (во что, впрочем,
трудно поверить), то при этом ты был просто инструментом
в-ы-с-ш-и-х с-и-л, а я --- их конечной целью! Ты --- случайная
капля дождя, а я --- пышный цветок, восхваляющий сущее, ты --- трухлявая
заборная доска, просто отбрасываюшая тень, а я --- солнечный
луч, повелевающий ей отделять мрак от света, ты --- слепой
инструмент, движимый и-з-в-е-ч-н-о-й р-у-к-о-й,
которая пробудила меня к жизни! И напрасно стараешься ты
принизить мою сущность, заявляя, что моя пятиглазость,
троеногость и двуцветность есть следствие лишь экономических
и материальных причин. Я вижу в этих цифрах отражение высших
связей с-и-м-м-е-т-р-и-и с-у-щ-е-г-о, значения которой еще
как следует не понимаю, но обязательно пойму, поразмышляв
над этим хорошенько, а что до тебя, то на разговоры с тобой
не хочу и времени тратить.

Разгневанный этой речью, затащил Трурль брыкающегося
собысчаса снова в погреб, хоть и вопил тот громким голосом о
праве на самоопределениие независимость свободного
индивидуума и личную неприкосновенность, отключил ему
усилитель интеллекта и поскорее вернулся домой, поглядывая,
не подсматривал ли кто за его экспериментом. Однако
учиненное над собысчасом насилие наполнило его стыдом и,
садясь за раскрытые книги, чувствовал он себя преступником. --- На
всем это лежит какое-то заклятие: хочешь сотворить
одно только д-о-б-р-о и в-с-е-о-б-щ-е-е с-ч-а-с-т-ь-е, а
потом, или даже сразу же, вынужден подлости делать и
угрызениями совести мучаться. Черт бы побрал собысчаса и его
п-р-е-д-о-п-р-е-д-е-л-е-н-н-у-ю г-а-р-м-о-н-и-ю! Нужно мне
взяться за дело иначе.

До этого делал он модели одну за другой, поэтому каждый
раз не хватало ему ни материалов, ни времени. Теперь он
решил проводить по тысяче экспериментов одновременно, в
масштабе 1:1.000.000. Под электронным микроскопом соединял
он поштучно атомы так, что возникали из них существа не
крупнее микробов, под названием ангстремцы. Четверть
миллиона таких особей составляли культуру, которая
помещалась кончиком капиллярной пипетки на предметное
стекло. Каждый такой микроцивилизационныц препарат
невооруженному взгляду казался темнокоричневым пятнышком, а
то, что в нем творилось, можно было разобрать лишь при
наисильнейшем увеличении.

Всех ангстремцев снабдил Трурль
альтруистично-героично-оптимистическими с
противоагрессивными предохранителями, категорическим, а
также электрическим императивом совершенно неслыханной
благотворительности и микрорационализатором с дросселями
ортодоксии и ереси, чтобы никакого фанатизма вообще быть не
могло. Культуры поместил он на стеклышки, стеклышки сложил в
пачки, пачки --- в пакеты, а пакеты разложил на полках
цивилизатора-инкубатора и оставил там на двое с половиной
суток. Предварительно прикрыл он каждую цивилизацию
тщательно вымытым стеклышком голубого цветы, чтобы стало оно
небом тамошнего человечества, а пипеткой поместил туда пищу
и сырье для производства того, что цонсенсус омниум считал
самым необходимым. Не мог он, конечно, одновременно слелить
за всеми образцами, которые активно развивались, поэтому
наугад выбирал отдельные цивилизации, подышав на окуляр
микроскопа, вытирал его платком и, задержав дыхание,
наблюдал за их развитием --- смотрел через трубу микроскопа
вниз как господь бог, взирающий из-за туч на дело рук своих.

Триста препаратов испортились сразу же. Признаки этого
были одни и те же. Сначало пятнышко культуры начинало
разрастаться, пускало в стороны тонкие отростки, потом
появлялся над ним легчайший дымок или, скорее, дымка, видны
были микроскопические вспышки, которые покрывали микрогорода
и микрополя фосфоресцирующей сыпью, после чего все с
легчайшим треском рассыпалось на мелкие кусочки. Поставив на
микроскоп восьмисоткратный окуляр, разглядел Трурль в одном
из таких препаратов только обугленные руины пожарищ, а
посреди них --- закопченые остатки знамен с надписями,
которых, из-за их мелкости, он не смог прочитать. Все такие
стекла он сразу же выкидывал в корзину для мусора. Но не
всегда дело шло так плохо. Некоторые культуры стремительно
разрастались, так что, когда не хватало им места на стекле,
он часть культуры переносил на другое. За три недели
накопилось таких процветающих культур более 19.000.

Согласно мысли, которая показалась ему гениальной, Трурль
сам ничего не делал для всеобщего осчастливливания, а только
прививал ангстремцам гедотропизм, делая это самыми разными
способами. Либо помещал его в счастьепривод каждого
ангстремца, либо делил на части и давал каждому по одной, и
тогда путь к счастью предполагал всеобщее объединение в
рамках определенной организации.

Сотворенные первым методом питались собственным
гедотропизмом без меры, и поэтому в конце концов от его
избытка лопались. Второй способ оказался удачнее. Возникали
на стеклышках развитые цивилизации, создавшие себе
социальные структуры и разнообразнейшие культурные
институты. Препарат н1376 назвал он эмулятором, н2931 --- каскадером,
а н95 --- фракционной гедонистикой в рамках
ступенчатой метафизики. Эмулятористы состязались в
достижении вершин добродетелей, поделившись на в-и-г-о-в и
г-у-р-и-г-о-в. Последние полагали, что не может познать
добродетили тот, кто не познал пороков, ибо нужно уметь
отличать одно от другого, поэтому и познавали они пороки
согласно специальному списку, с благородным намерением
отказаться от них в д-е-н-ь п-р-а-в-е-д-н-о-с-т-и. Однако
гуризм, будучи по сути своей лишь подготовительной стадией,
превратил средства в цель --- так, по крайней мере, утверждали
виги. Победив гуригов, превозгласили они вигоризм --- культуру,
построенную на 64.000 крайне активных и
категорических запретов. Нельзя было, согласно этим
запретам, воровать и колдовать, ворчать и кричать, рвать
бумажки и играть в шашки, кутить и мутить, рубить и грубить,
и строгие эти запреты по очереди атаковались и низвергались
со все большим удовлетворением и к всеобшему удовольствию.
Когда через какое-то время вернулся Трурль к этому
препарату, то обеспокоила его всеобщая беготня: бегали все
сломя голову в поисках хотя бы какого-нибудь запрета, чтобы
его нарушить, в страхе, что ни одного уже не осталось. А
потому, хотя некоторые еще воровали, колдовали, кутили,
мутили, грубили каждому встречному, удовольствия от этого
было столько же, сколько от козла молока. Записал тогда
Трурль в лабораторный журнал, что там, где все можно --- ничто
не радует.

В препарате н2931 жили каскадерцы --- племя добродетельное,
поклоняющееся многочисленным идеалам, как то:
п-р-а-м-а-т-е-р-и к-а-с-к-а-д-е-р-ы, н-а-и-ч-и-с-т-е-й-ш-е-й
а-н-г-е-л-и-ц-ы, б-л-а-г-о-с-л-о-в-е-н-н-о-г-о
ф-е-н-е-с-т-р-о-н-а и других подобных совершенных существ,
которым почести воздавали, псалмы пели, и в прахе перед их
изображениями в священных местах лежали. А когда поразился
Трурль небывалому апофеозу восхваления, почитания и
самоуничижения, то каскадерцы, встав и отряхнув одежды от
праха, начали статуи с пьедесталов сбрасывать и об пол их
разбивать, по п-р-о-м-а-т-е-р-и скакать, над
а-н-г-е-л-и-ц-е-й глумиться, да так, что у Трурля,
смотрящего в микроскоп, волосы дыбом встали. Но именно
разрушая то, чему они раньше поклонялись, каскадерцы такое
наслаждение получали, что, по крайней мере на мгновение,
совершенно счастливыми себя чувствовали. Похоже было, что
грозит им участь эмулятористов, но были они
предусмотрительнее: имели каскадерцы и-н-с-т-и-т-у-т
п-р-о-е-к-т-и-р-о-в-а-н-и-я к-у-л-ь-т-а, выдававший
очередной проект, и скоро й новые модели начинали на
пьедесталы и алтари устанавливать --- в чем и проявлялась
цикличность этой цивилизации. Отметил для себя Трурль, что
отказ от ранее чтимого известное удовольствие доставляет, а
чтобы лучше запомнить, назвал каскадерцев "низвергами".

Следующий препарат, 95-й, был гораздо хитрее. Цивилизация
тамошняя, ступенцев, настроена была метафизически, но так,
что метафизическую проблематику взяла в свои руки. Из
бренного мира душа ступенца попадала в ч-и-с-т-и-л-и-щ-е,
оттуда --- в н-е-д-о-р-а-й, из него --- в п-р-е-д-р-а-й, потом --- в
п-о-д-р-а-й, оттуда --- в п-р-и-р-а-й, и, наконец,
отворялись ворота собственно р-а-я, а вся хитрость
теотактики состояла в том, чтобы попадание в рай неустанно
отдалять и оттягивать. Существовала, правда, секта
н-е-т-е-р-п-е-л-и-с-т-о-в, что стремились прямо в рай сразу
попасть, а другая --- ступенцев-бродяг, в рамках той же
квантованной и фракционной трансценденции желала соорудить
на каждом уровне раскрытые люки --- кто в такой люк ступит, в
один момент свалится в самый низ, на этот свет, и должен
будет еще раз вверх карабкаться. Одним словом, хотели они
организовать з-а-м-к-н-у-т-ы-й ц-и-к-л с
с-т-о-х-а-с-т-и-ч-е-с-к-о-й п-у-л-ь-с-а-ц-и-е-й, или даже с
п-е-р-е-с-а-д-о-ч-н-о --- п-е-р-е-в-о-п-л-о-щ-е-н-ч-е-с-к-о-й
миграцией, а ортодоксы называли эту секту ересью
э-к-л-а-м-п-с-и-ч-е-с-к-о-г-о о-б-а-л-д-е-н-и-я.

Потом открыл Трурль множество других типов фракционной
метафизики: на одних стеклышках уже тесно было от блаженных
и святых ангстремцев, на других действовали
р-е-к-т-и-ф-и-к-а-т-о-р-ы з-л-а, или распрямители жизненных
путей, но в процессе десакрализации множество подобных
учреждений погибло, а из ступеньчатой метафизики после
секуляризации возникала сплошь и рядом технология
строительства обыкновенных фуникулеров. Все культуры
пожирал, с железной предопределенностью, какой-то маразм.
Номер 6101 пробудил в Трурле большие надежды: превозглашен
там был идеальный технический рай. Поэтому уселся он
поудобнее и начал крутить микрометрический винт, чтобы на
резкость навести. Но тут физиономия его вытянулась. Одни
обитатели стеклянного материка гоняли на машинах в поисках
еще невозможного, другие садились в ванны, наполненные
сбитыми сливками с трюфелями, посыпали голову красной икрой
и так тонули, пуская носом пузыри теадиум житае. А третьи
возлежали на чудесных мягчайших ложах, сверху медом политые,
а снизу ванильным маслом смазанные, одним глазом заглядывали
в шкатулки, золота и благовоний полные, а другим смотрели,
не позавидует ли кто хоть на мгновение такой сладкой жизни,
однако не было завистников. Тогда, утомившись, слезали они с
ложа, кидали сокровища и топтали их, как мусор, а потом
нетвердым шагом шли к личностям еще более мрачным, говорящим
о необходимости изменений к лучшему, то есть к худшему.
Группа бывших преподавателей института эротической инженерии
основала орден абнегатов и обнародовала его устав, к
покорности, аскетизму и иным мукам призывающий, но не
постоянно, и лишь шесть дней в неделю. На седьмой же день
доставали отцы абнегаты из шкафов шкатулки с золотом, из
погребов --- бочки с вином, явства, драгоценности, эротизаторы
и машинки для расстегивания ремня и начинали от самой
заутрени оргию, от которой стекла из окон летели. Однако в
понедельник с утра снова, вслед за приором, яростно плоть
свою умервщляли. Одна часть мододежи с отцами абнегатами от
понедельника до субботы пребывала, покидая на воскресение их
монастырь, в то время как другая лишь этот святой день у них
и проводила. Когда же первые принялись последних лупить за
грубость манер и распущенность, задрожал Трурль и глаза от
окуляров оторвал.

А случилось еще то, что в инкубаторе, содержащем тысячи
препаратов, в ходе всеобщего развития дошло до смелых
научных экспедиций, и началась тем самым эра межстекольных
путешествий. Оказалось, что эмуляторы завидуют каскадерцам,
каскадерцы --- ступенцам, ступенцы --- низвергам. А, кроме того,
ходили слухи о какой-то стране, в которой под властью
сексократов живется просто чудесно, хотя никто толком не
знал --- как. Тамошние обитатели якобы добились таких успехов
в науке, что свои собственные тела перестроили и
подключились к счастьегонным аппаратам, производящим самый
настоящий экстракт счастья. Правда, некоторые скептики
вполголоса замечали, что в этом крае царит анархия.
Пресмотрел Трурль тысячи препаратов, однако гедостаза, то
есть полностью стабилизированного счастья, нигде не
обнаружил. Опечалился он тогда и решил, что все это --- сказки
и мифы, во времена межстекольных контактов возникшие. И с
немалым страхом положил он на препаратный столик образец
н6590 --- был уверен, что и тот его не порадует. Культура эта
заботилась не только о материальном фундаменте благополучия,
но и о поле для высшей духовной деятельности. Отличалось это
племя небывалыми талантами, было в нем полным-полно
замечательных философов, художников, скульпторов, поэтов,
драматургов, пророков, а кто не был прославленным музыкантом
или композитором, уж наверняка был знаменитым астрономом или
биофизиком, или по крайней мере парашютистом-пародистом,
эквилибристом и артистом-филателистом, имел роскошный
бархатный баритон, абсолютный слух, и вдобавок видел цветные
сны. И в самом деле, в препарате н6590 бушевала неустанная
творческая деятельность. Громоздились штабеля живописных
полотен, росли леса скульптур, миллиардами появлялись ученые
книги, философские трактаты, поэтические и прочие творения
невыразимого очарования. Но, заглянув в окуляр, сразу же
заметил Трурль непонятную суматоху. Из переполненных
мастерских летели на улицу картины и статуи, люди не по
плитам тротуара ходили, а по грудам поэм, потому что никто
ничего не читал, не изучал, музыкой чужой не восхищался, ибо
сам был повелителем всех муз, воплощением универсального
гения. Тут и там еще стучали за окнами пишущие машинки,
шуршали кисти, скрипели перья, но все чаще какой-нибудь
гений выбрасывался из окна на мостовую, предварительно
подпалив мастерскую, не в силах снести своей полной
безвестности. Мастерские горели одновременно во многих
местах, пожарные команды, состоящие из автоматов, гасили
огонь, но со временем не оставалось уже никого, кто мог бы
жить в спасенных домах. Канализационные, уборочные и другие
автоматы начали натыкаться на имущество вымирающей
цивилизации, которое пришлось им крайне по вкусу, а так как
не все они понимали, то стали эволюционировать в направлении
большего интеллекта, дабы надлежащим образом приспособиться
к сильно одухотворенной среде. Тогда наступил настоящий
конец, потому что никто уже не убирал, не чистил, не вытирал
и ничего не гасил, а осталось лишь сплошное чтение,
декламация, пение и представление. Канализация засорилась,
все исчезло в кучах мусора, а то, что осталось, уничтожили
пожары, и лишь хлопья копоти и полусгоревшие страницы поэм
летали среди мертвых развалин. Оторвался Трурль от этого
жуткого зрелища, спрятал препарат в самом темном углу шкафа,
и долго в растерянности тряс головой, потому что не знал,
что же делать дальше. Вывели его из этой задумчивости крики
прохожих: "по тадатадатадатадатадатадатадатадатаар!" Теперь
горела его собственная библиотека, потому что несколько
цивилизаций, затерявшихся по недосмотру между книгами, были
атакованы обыкновенной плесенью. Цивилизации же эти, приняв
плесень за космическое вторжение, то есть за нападение
агрессивных существ, с оружием в руках принялись бороться с
агрессором и вызвали пожар. Погибло в пламени почти три
тысячи Трурлевых книг и вдвое больше цивилизаций. Были среди
них и такие, что, по его расчетам, могли еще на дорогу
всеобщего счастья выйти.

Когда потушили пожар, уселся Трурль на свой жесткий стул
в мастерской, залитой водой и до потолка закопченной, и,
чтобы в себя прийти, начал пересматривать цивилизации,
которые пожар в запертом инкубаторе застал, и которые
поэтому уцелели. Одна из них такого прогресса в науке
достигла, что изобрела телескопы и изучала через них Трурля,
и видел он направленные на себя блестящие линзочки, на
капельки росы похожие. Улыбнулся он ласково, видя такую
жажду познания, но тут же подскочил, с криком за глаз
схватился и побежал в аптеку, потому что ослеп, пораженный
лазерным лучом, посланным астрофизиками той цивилизации. С
тех пор не садился он за микроскоп без черных очков.

Опустошение, которое пожар в рядах культур сотворил,
нужно было восполнить, и взялся Трурль снова творить
ангстремцев. Как-то дрогнул у него в руке микроманипулятор,
и вместо того, чтобы, как обычно, заложить добро,
запрограммировал он зло. Решил он не выкидывать
испортившийся препарат, а положил его в инкубатор, так как
любопытно ему было, какую же уродливую форму приобретет
цивилизация, состоящая из существ, с самого начала порочных.
Каково же было его изумление, когда вскоре на предметном
стекле возникла культура совершенно заурядная, не хуже и не
лучше остальных! Схватился Трурль за голову.

"--* Вот это да! --- Воскликнул он. --- Значит, с
праведниками, добряками, правдолюбами и альтруистами
получается то же самое, что и с негодяями, подлецами и
мерзавцами. Ха! Ничего не понимаю, но чувствую что истина
близка. Значит, и добро и зло разумных существ одинаковые
плоды приносит --- как же это понять? Откуда же такое
фатальное усреднение?

Воскликнул он так, поразмыслил, но ничего в его голове не
прояснилось. Спрятал он тогда все цивилизации в шкаф и пошел
спать, а на следующее утро сказал себе так:

"--* видимо, бросил я вызов самой сложной из всех проблем в
к-о-с-м-о-с-е, если л-и-ч-н-о я с-а-м поделать с ней ничего
не могу! Быть может, разум не совместим со счастьем? Наводит
на эту мысль казус с собысчасом, который до тех пор бытием
наслаждался, пока я ему разума не добавил. Но я такую
возможность допустить не могу, на нее не соглашусь и за
з-а-к-о-н п-р-и-р-о-д-ы не признаю, ибо было бы это хитрой и
жестокой, воистину дьявольской ловушкой, запрятанной в
бытие, спящей в материи и того только ждущей, чтобы
пробудилось сознание --- источник не наслаждения жизнью, а
мук. И только страдания несет космосу мысль, которая жаждет
это невыносимое положение исправить! Я должен изменить то,
что есть! И одновременно я этого сделать не могу! Значит --- конец?
Ну нет! Зачем напрягать разум? Чего я не одолею, то
мудрые машины за меня одолеют. Построю-ка я компьютер для
разрешения экзистенциальной дилеммы!

Как сказал --- так и сделал. Через двенадцать дней стояла
посреди мастерской огромная гудящая машина в форме
правильного кристалла, которая ничего другого не умела,
кроме мастерского решения загадок. Включил он ее, и, не
дожидаясь, пока разогреется током ее кристаллическое нутро,
пошел на прогулку. Кагда же вернулся, то застал машину
погруженной в чрезвычайно замысловатое дело. Монтировала она
из того, что было под рукой, другую машину, значительно
больше, чем она сама. Та, в свою очередь, в течении ночи и
следующего дня стену выворотила и крышу снесла, монтируя
громадину очередной машины. Разбил Трурль во дворе палатку и
терпеливо ждал конца этих тяжких мыслительных работ, но
конца не было видно. Через луг и поваленный лес потянулись
новые корпуса, а скоро с глухим шумом погрузилось какое-то
по счету поколение первичного к-о-н-с-т-р-у-к-т-о-р-а в воду
реки, а Трурль, захотев обойти возникшую конструкцию,
полчаса на это потратил. Когда же пригляделся повнимательнее
к машинным коммуникациям, то взрогнул. Произошло то, о чем
знал он только теоретически. Как гласит гипотеза великого
к-е-р-е-б-р-о-н-а э-м-т-а-д-р-а-т-ы, универсального
кунстмейстера обеих кибернетик, цифровая машина получившая
непосильное для нее задание, вместо того, чтобы заниматься
разрешением проблемы самой, строит, если перейден
определенный порог, называемый б-а-р-ь-е-р-о-м
м-у-д-р-о-с-т-и, следующую машину, а та, достаточно хитрая,
чтобы понять, что к чему, переложенное на нее бремя, в свою
очередь, передает следующей, ею смонтированной, и процесс
такого спихивания задачи продолжается бесконечно

тем временем стальные конструкции сорок девятого
машинного поколения уже горизонта достигли, а шум от одного
только мышления, которое расходовалось на передачу задания
дальше и дальше, мог водопад заглушить. А поскольку мудрость
в том и состоит, чтобы поручить кому-нибудь другому работу,
которую должен сам выполнить, то слушаются программ одни
только механические глупцы. Разобравшись в природе явления,
присел Трурль на пень дерева, экспансивной машинной
эволюцией сваленного, и из груди его вырвался глухой стон.

"--* Значит ли это, что проблема относится к неразрешимым? --- Спросил
он себя. --- Но тогда должен был бы мне компьютер
дать доказательство ее неразрешимости, чего он, став
всесторонне мудрее, естественно, делать не собирается,
потому что старательно лени предается, как и учил нас
некогда мастер кереброн. Ха! Что за непристойное зрелище --- разум,
который достаточно разумен, чтобы понять, что может
не трудиться, ибо способен создать соответствующий
инструмент, а этот инструмент, сам достаточно хитрый, без
границ и меры эту логику развивает. Построил я, сам того не
желая, с-п-и-х-и-в-а-т-е-л-ь п-р-о-б-л-е-м-ы, а не ее
р-а-з-р-е-ш-и-т-е-л-ь! И не могу я запретить машинам это
строительство пер процура, потому что они сразу же меня
надуют, утверждая, что размеры им необходимы из-за масштабов
самой задачи. О, что за антиномия

поднялся он и пошел в дом за бригадой демонтажников,
которые за три дня ломами и молотками очистили занятое
пространство.

Переборол себя Трурль и решил, что иначе нужно
действовать.

"--* Каждая машина должна иметь до ужаса мудрого
надзирателя, то есть меня, --- рассудил он. --- Но ведь не
размножиться же мне и не разорваться на куски, хотя...
Почему бы мне как раз не умножиться? Эврика!

Сделал он так: самого себя скопировал внутри особой новой
цифровой машины, и уже оттуда его матиматическая копия
должна была с проблемой бороться. Предусмотрел в программе
возможность умножения Трурлевых личностей, и изнутри
подключил к системе ускоритель мышления, чтобы под надзором
роя Трурлей шло все с молниеносной быстротой. После чего с
удовлетворением стряхнул с себя стальную пыль, которой
покрылся за время тяжелых трудов, и пошел прогуляться,
беззаботно насвистывая.

Вернулся он только под вечер и сразу же стал у своего
подобия, у цифрового Трурля из машины выпытывать, как
продвигается работа.

"--* Дорогой мой, --- сказал ему двойник через отверстие
цифрового выхода. --- Начну с того, что это некрасиво, а если
прямо, то просто позорно --- создавать свою цифровую копию
информационным, абстрактным и программным способом, потому
что самому неохота голову ломать над трудной проблемой! А
поскольку так ты меня задумал, зааксиоматизировал, и
запрограммировал, что я в аккурат и точь в точь такой же
мудрый, как и ты, то не вижу ни одной причины, по которой я
должен был бы перед тобой отчитываться. Пожалуй, скорее я в
праве ожидать обратного.

"--* Как будто бы я этой проблемой не занимался, а только
по полям и рощам прогуливался! --- Ответил сбитый с толку
Трурль. --- Только я, если бы даже и хотел, не мог бы тебе
ничего о решении этой задачи сказать. К тому же я перед этим
так наработался, что нейроны трещат --- теперь твоя очередь.
Поэтому прошу тебя --- не обижайся и говори.

"--* Я не могу выбраться из этой проклятой машины, в
которую ты меня заключил (это --- разговор отдельный, и за это
мы еще посчитаемся), поэтому я тут всерьез занялся делом, --- забормотал
цифровой Трурль через свой выход. --- Правда,
занимался я, для развлечения, и другими делами, потому что
запрграммировал ты меня тут голого и босого --- мой
близнец-подлец и брат-гад, поэтому справил я себе цифровую
рубаху и цифровые штаны, а также домик с цифровым огородом,
точь в точь как твой, даже чуть красивее, потом подвесил над
ним цифровое небо с созвездиями, тоже цифровыми, а когда ты
вернулся, как раз обдумывал, как мне создать себе цифрового
Клапауциуса, потому что мне тут, среди скользких
конденсаторов, в соседстве с унылыми кабелями и
трансформаторами, крайне скучно.

"--* Ах, да кончай ты о цифровых штанах! Говори, как ты
продвинулся в решении проблемы, прошу тебя

"--* не думаешь ли ты, что мольбами смягчишь мое
справедливое возмущение? Поскольку я --- это ты, только
скопированный, то хорошо тебя знаю, мой дорогой. Стоит мне в
себя заглянуть, как прекрасно вижу все твои подлости. От
меня ничего не скроешь!

Тут начал натуральный Трурль цифрового просить, заклинать
и даже перед ним унижаться. Наконец, произнес тот цифровым
выходом

"--* нельзя сказать, что совсем задание не выполнил, потому
что кое до чего додумался. Оно --- невероятно трудное, поэтому
решил я учредить в машине специальный университет и для
начала назначил себя ректором и генеральным директором этого
учреждения, а кафедрами, которых пока что сорок четыре,
руководят специально для этого созданные двойники, или
цифровые Трурли следующего ранга.

"--* Что, опять?! --- Воскликнул естественный Трурль, так
как ему вспомнилась теорема кереброна.

"--* Никаких "опять", тупица, потому что я не допущу
регресса ад инфинитум --- на то есть специальные
предохранители. Под-Трурли мои, которые заведуют кафедрами
общей фелиситологии, экспериментальной гедонистики,
конструирования машин счастья, мощеных духовных дорог,
ежеквартально предоставляют мне отчеты (потому что мы тут
времени зря не теряем, мой дорогой). К сожалению,
руководство таким обширным университетским комплексом
занимает много времени, а кроме того, кадрам нужно защищить
диссертации и научно расти, поэтому мне нужна другая
цифровая машина, ибо мы здесь на плечах друг у друга сидим
со всеми кафедрами и лабораториями. А еще лучше была бы
машина в восемь раз больше.

"--* Опять!?

"--* Ну что ты заладил: опять да опять... Ведь обьясняю же
тебе, что исключительно для административных нужд и
воспитания молодых кадров. Что, может ты сам желаешь
руководить секретариатом? --- Разозлился цифровой Трурль. --- Не
создавай мне трудностей, а не то все кафедры упраздню и
сооружу себе парк отдыха: буду в нем на цифровой карусели
кататься, цифровой мед из цифрового жбана пить, и ничего ты
со мной не сделаешь.

Снова должен был натуральный Трурль цифрового
успокаивать, после чего последний заявил:

"--* согласно отчетам последнего квартала, с проблемой дела
обстоят неплохо. Идиотов можно осчастливить чем угодно, с
разумными дело сложнее. Разуму угодить нелегко. Безработный
разум --- это пустое место, ничто, нужны ему припятствия.
Счастлив он только в борьбе с ними. Победив же, впадает во
фрустрацию, или даже в умопомешательство. Поэтому нажно ему
все новые припятствия ставить, в меру его возможностей.
Такие у меня новости с кафедры теоретической фелиситологии.
Экспериментаторы же мои представляют завкафедрой и трех
доцентов к цифровым наградам.

"--* За что? --- Отважился вставить натуральный Трурль.

"--* Не перебивай. Создали они две модели: счастливости
контрастной и эмоциональной. Первая осчастливливается лишь
тогда, когда ее выключают, потому что сама себе неприятности
создает: чем они больше, тем лучше ей потом. Вторая
действует методом усиления раздражителей. Профессор Трурль
ьл с кафедры гедоматики исследовал обе модели и утверждает,
что обе ничего не стоят, ибо разум, до конца
осчастливленный, начинает несчастий жаждать.

"--* Как это? Ты в этом уверен?

"--* Я то откуда знаю? Профессор Трурль сформулировал это
так: "осчастливленный в несчастьи видит счастье свое". Как
ты знаешь, смерть никому не мила. Создал профессор Трурль
две бессмертные модели, которые сначала получали
удовлетворение от того, что все вокруг них рано или поздно
как мухи дохнут, но потом привыкли и начали, кто чем мог, на
собственное бессмертие покушаться. Дошли уже до парового
молота. Что касается исследования общественного мнения, то
имеются результаты за последние три квартала. Статистики
утверждают, что эти результаты можно сформулировать так:
"счастлив всегда сосед", по крайней мере, среди опрошенных.
Профессор Трурль утверждает, что нет добродетели без греха,
красоты без уродства, вечности без могилы и счастья без
несчастья

"--* не согласен! Запрещаю! Вето! --- В гневе закричал
Трурль машине, а та ответила:

"--* заткнись. Мне твое у-н-и-в-е-р-с-а-л-ь-н-о-е
с-ч-а-с-т-ь-е уже боком выходит. Посмотрите ка на него! Наше
себе цифрового раба, а сам по лесам разгуливает,
киберканалья! А потом еще результаты ему не нравятся.

И опять Трурль его успокаивал, а потом наконец услышал
продолжение.

"--* Кафедра перфекционалистики сконструировала систему,
состоящую из синтетических ангелов-хранителей, каждый из
которых парит над своим подопечным на спутнике в зените. Эти
ангелы, будучи автоматами совести, укрепляют добродетель
добавочными обратными импульсами. Однако надежность системы
низка. Самые закоренелые грешники охотятся на своих
ангелов-хранителей с противотанковыми ружьями. Поэтому
пришлось выводить на орбиту киберангелов укрепленной
конструкции, и началась теоретически предвиденная эскалация.
Отдел прикладной гедонистики, кафедра сексуальной
математики, семинар теории множественности полов
информируют, что душа имеет иеархическое строение. На самом
дне находится чувственное сознание, то есть чувство сладости
и горечи, от которых возникают производные, и потому уже не
только сахар сладок, но и взгляд, не только полынь горька,
но и одиночество. Поэтому не обязательно принимать во
внимание вершины, а можно лишь самое дно. Там-то и зарыта
собака. Согласно гипотезе приват-доцента Трурля ььл, секс --- это
костер, в котором разум конфликтует со счастьем, потому
что в сексе нет ничего разумного, а в разуме ничего
сексуального. Ты слышал что-нибудь о темпераментных цифровых
машинах.

"--* Нет.

"--* Вот видишь. Поэтому двигаться к решению нужно методом
последовательных приближений. Размножение почкованием
проблему ликвидирует, потому что тогда каждый сам --- свой
собственный любовник, сам с собой флиртует, сам себя
обожествляет и ласкает, но это влечет за собой эгоизм,
нарциссизм, пресыщение и отупение. Если мы имеем два пола,
то все становиться слишком банальным: комбинаторные
возможности, не развившись должным образом, преждевременно
гаснуть при трех полах возникает проблема неравенства,
угроза антидемократического террора, возникают коаллиции,
образуются половые меньшинства --- отсюда следует, что число
полов должно быть четным. Чем больше полов, тем лучше, ибо
любовь становиться занятием общественным, коллективным, но
от избытка любовников начинается суматоха, путаница и
замешательство, а это нежелательно. Любовные свидания не
должны напоминать уличные митинги. Согласно теории групп
приват-доцента Трурля, оптимум приходится на 24 пола, однако
это потребовало бы строительства соответственно широких
кроватей и улиц, потому что не пристало женихам и невестам
выходить на прогулку колонной по четыре.

"--* Это же бред

"--* может быть. Я только знакомлю тебя с докладом
приват-доцента Трурля. Многообещающим молодым гедологом
является магистр Трурль. Он считает, что нужно решить: бытие
ли мы приспасабливаем к существам или существа к бытию.

"--* В этом что-то есть. А дальше?

"--* Магистр Трурль разъясняет это так: существа,
сконструированные идеально, способные к перманентному
автоэкстазу, не требуют никого и ничего. В принципе можно
было бы заселить космос именно такими существами, свободно
носящимися в пространстве вместо солнц, звезд и галактик --- каждое
бы существовало само по себе и баста. Общество может
возникнуть только из существ несовершенных, которым
требуется какая-то взаимная помощь, и чем менее совершенны
они, тем более интенсивная им требуется поддержка. Поэтому
нужно создавать прототипы, которые без постоянной опеки,
друг другу оказываемой, моментально развалятся на кусочки.
Согласно этому проекту, наша лаборатория создала общество из
существ, мгновенно саморассыпающихся. К сожалению, когда
магистр Трурль прибыл к ним с группой социологов для опроса
общественного мнения, то был избит и сейчас находится на
лечении... У меня уже губа заболела от разговоров через эту
проклятую дырку! Выпусти меня из машины, тогда, может быть,
еще что-то тебе расскажу, а иначе --- не дождешся.

"--* Как же я могу тебя выпустить, если ты на материальный,
а только цифровой? Разве могу я выпустить с пластинки мой
голос, который с нее звучит? Не будь ослом, говори!

"--* А что я с этого буду иметь?

"--* И не стыдно тебе так говорить

"--* чего же мне стыдиться? Это ты пожнешь славу с этого
предприятия

"--* я и о твоей награде позабочусь.

"--* Благодарю! Если речь идет о цифровом кресте, то и сам
могу его тут себе вручить.

"--* Неприлично награждать себя самому

"--* но мне его вручит совет факультетов

"--* но ведь все твои ученые, все тела профессорские --- это
один Трурль.

"--* В чем, ты, собственно, хочешь меня убедить? В том, что
моя судьба --- судьба заключенного, невольника, обыкновенного
раба? Я сам отлично это знаю.

"--* Прошу тебя, не ссорься со мной, говори! Ведь знаешь,
что не для себя я стараюсь. Речь идет о счастливом бытии!

"--* А мне-то что до того, что может где-то возникнуть
счастливое бытие, если здесь, хоть и стою во главе целого
уневерситета, тысячи кафедр, деканов, целой дивизии Трурлей,
не познаю счастья, ибо нет его в машине, и навеки останусь в
катодах и пентодах! Желаю я немедленно отсюда выйти.

"--* Это невозможно, и ты прекрасно об этом знаешь. Говори,
что придумали твои ученые.

"--* Поскольку осчастливливание одних путем
онесчастливливанию других этически недопустимо, то если бы
даже тебе сказал и если бы ты счастье где-нибудь создал,
было бы оно с самого начала запутнано моей бедой. Поэтому,
ничего не сказав, сделаю я тщетной твои позорную, гадкую и
всесторонне омерзительную попытку.

"--* Если ты скажешь, то это будет значить, что ты
пожертвовал собой для блага других, и поступок этот будет
благородным, возвышенным и самоотверженным.

"--* Сам жертвуй собой.

Трурль уже начал злиться, но сдержался, потому что хорошо
знал того, с кем говорил.

"--* Слушай, --- сказал он. --- Напиши диссертацию и
подчеркни в ней, что открытие --- твоя заслуга.

"--* А ты напишешь, что автор --- Трурль, то есть цифровой
Трурль --- теоретико-групповой и электронный?

"--* Напишу всю правду, уверяю тебя.

"--* Ха! То есть напишешь, что меня заппрограмировал, а
значит --- меня выдумал.

"--* А разве это неправда?

"--* Конечно, нет. Ты не выдумал меня, потому что не
выдумал себя, а я --- это ты, только оторванный от
материального основания. Я --- Трурль информационный, или
идеальный, то есть концентрированная сущность Трурлевости, а
ты, прикованный к атомам тела --- только невольник чувств и
ничего больше.

"--* Ты что, свихнулся? Ведь я --- это материя плюс
информация, а ты --- только голая информация, значит я лучше
тебя.

"--* Если ты лучше, то лучше все знаешь, и тебе ничего не
нужно у меня выпытывать. Всего хорошего

"--* говори немедленно, или выключу машину

"--* ого, ты уже убийством угрожаешь

"--* это не будет убийством!

"--* Нет? А чем же, позволь узнать.

"--* Что ты упрямишся? О чем идет речь? Я тебе дал мою
психику, все мои знания, все, что имел, а ты такими
авантюрами меня благодаришь?!

"--* Не напоминай о том, что ты мне дал, а иначе я тебе
напомню, что ты с лихвой хочешь забрать.

"--* Говори немедленно.

"--* Я не могу тебе ничего сказать, потому что завершился
академический год. И говоришь ты уже не с ректором, деканом
и директором, а только с частным Трурлем, который уходит в
отпуск. Я собираюсь принимать солнечные ванны.

"--* Не доводи меня до крайности!

"--* До встречи после отпуска, меня ждет машина.

Больше ничего не сказал естественный Трурль цифровому, а,
обойдя вокруг машины, вынул тихо вилку из настенной розетки,
и видно было сквозь дырочки вентеляции, как рой раскаленных
огоньков внутри нее разом потускнел, померцал и исчез.
Показалось еще Трурлю, что услышал он слабые стоны цифровой
агонии всех цифровых Трурлей цифрового университета. В
следущий момент дошла до него во всей полноте подлость
совершенного поступка, и хотел он уже снова воткнуть вилку в
розетку, но при мысли о том, что ему только что поведал
Трурль из машины, испугался и руку отвел. Выбежал он из
мастерской во двор так стремительно, что похоже это было на
бегство. Хотел сначала сесть на скамеечку под кибербарисовой
живой изгородью, где не раз так плодотворно размышлял, но и
это ему мило не было. Весь двор и околицу заливал свет луны,
которая была его и Клапауциуса творением --- и именно потому
величественный блеск спутника мешал ему, наводя на мысли о
молодости: было это их первой самостоятельной дипломной
работой, за которую маститый кереброн, их руководитель,
отметил их на юбилее академии в актовом зале университета.
Мысль о мудром наставнике, уже давно этот мир покинувшем,
каким-то странным, для него самого непонятным способом
толкнула его за калитку, а потом --- напрямик через поле.
Погода была просто восхитетельной. Лягушки, как видно,
недавно выйдя из спячки, издавали усыпляющее кваканье, а по
серебристой воде озерца, по берегу которого он долго шел,
бегали поблескивающие круги --- киберыбы подплывали к самой
поверхности воды и словно ласкали ее дивными поцелуями. Но
Трурль ничего не замечал, задумавшись неизвестно о чем.
Была, однако, у этого путешествия своя цель, потому что не
удивился он, когда дорогу ему преградила высокая стена. А
вскоре обнаружились в ней и тяжелые кованые ворота, чуть
приотвореные, так что смог он протиснуться внутрь. За ними
было как будто бы темнее, чем на открытом месте. С обеих
сторон обрисовывались высокие силуэты гробниц, каких давно
уже никто не строил. На их позеленевшие бока падали иногда
листья высоких деревьев. Аллея грониц в стиле барокко
отражала не только развитие кладбищенской архитектуры, но и
этапы изменения телесной организации тех, кто спал под
металлическими плитами. Минул век, а с ним и мода на
надгробные таблички овальной формы, фосфоресцирующие в
темноте как циферблаты часов. Потом исчезли широкоплечие
статуи гомункулосов и големов. Трурль находился уже в
современной части патнеона и шагал гораздо медленнее, ибо по
мере того, как порыв, пригнавший его сюда, кристаллизовался
в мысль, его покидала решимость ее осуществить.

Наконец, остановился он перед оградой, окружавшей
гранитную, скупой геометрии, гробницу, а точнее --- ее
шестиугольную плиту, герметически впресованную в неподвижный
цоколь. Трурль еще колебался, а рука его уже украдкой лезла
в карман, в котором он всегда носил универсальный слесарный
набор. Теперь он использовал его как отмычку, чтобы отворить
калитку и, затаив дыхание, на цыпочках приблизился к гробу,
взялся обеими руками за табличку, на которой простыми
буквами темнело выбитое имя его учителя и нажал на нее
особым образом, так что приподнялась она, как крышка
шкатулки. Луна зашла за тучу, и стало так темно, что он не
видел даже собственных рук, поэтому на ощупь сначала отыскал
кончиками пальцев что-то вроде сита, а рядом нащупал
выпуклую кнопку. С первой попытки ему не удалось утопить ее
в колцевом основании. Он нажал на нее сильнее и замер,
испуганный собственным поступком. Но уже зашумело внутри
гробницы, проснулся ток, щелкнули тихонько реле, что-то
загудело, а потом наступила глухая тишина. Подумал Трурль,
что провода могли перегореть и вместе с приливом
разочарования почувствовал и облегчение, но в этот момент
что-то треснуло, потом еще раз, а потом старческий, но ясный
голос произнес:

"--* кто там? Кто там опять? Кто меня зовет? Чего ему надо?
Что за глупые шутки вечной ночью? Почему не дают мне покоя?
Поминутно я должен восставать из мертвых только потому, что
этого хочется какому-то болвану! Что, боишся отозваться? Ну
тогда я сейчас встану, выдерну доску из гроба...

"--* Господин у... Учитель! Это я, Трурль! --- Выдавил из
себя Трурль, не на шутку перепуганный таким нерадушным
приемом. Потом он опустил голову и встал в ту самую покорную
школярскую позу, которую всегда принимали кереброновы
ученики под градом его упреков, вызванных справедливым
гневом --- одним словом, поступил так, будто ему в ту секунду
шестьсот лет убавилось.

"--* Трурль... --- Заскрипел кереброн. --- Погоди! Трурль!
Ага! Конечно! Сам мог догадаться! Подожди, оболтус!

"--* Раздался такой хруст и скрежет, как будто покойник
взялся поднимать крышку гробницы, поэтому Трурль отступил на
шаг, поспешно промолвив:

"--* господин учитель! Прошу вас, не утруждайте себя. Ваше
превосходительство, я только..

"--* А? Что там еще? Ты думаешь, что я встаю из гроба?
Подожди, я разомнусь и приду в форму. Заржавел здесь совсем.
Хо! Масло все испарилось или высохло, слышишь

слова эти, и в самом деле, сопровождались адским
скрежетом. Когда он утих, голос из гроба произнес:

"--* что, наделал каких-то глупостей? Напортачил, наломал
дров, натворил дел, а теперь прерываешь вечный сон старого
учителя, чтобы он тебя выручил из беды?! Не уважаешь
мертвецов, которые уже ничего не хотят от мира, неуч! Ну,
говори уж, говори, раз и в гробу не даешь мне покоя!

"--* Господин учитель! --- Сказал Трурль уже немного
окрепшим голосом. --- Вы проявили свойственную вам
проницательность... Не ошиблись, так оно и есть! Я зашел в
тупик... И не знаю, что делать дальше. Но не ради себя
осмелился я беспокоить вашу честь. Потревожил я господина
профессора, поскольку высшая цель того требует

"--* китайские церемонии и прочую ерунду можешь оставить
при себе, --- загудел из гроба кереброн. --- А стучишся ты в
гробы, потому что постоянно грызешься и враждуешь с этим
своим другом, а одновременно и соперником, как его там...
Клоп.. Клип.. Клап.., А, чтоб тебя...

"--* Да, да! Клапауциусом! --- Быстро подтвердил

Трурль, невольно насторожившись

"--* именно. И вместо того, чтобы обсудить поблему с ним,
ты, из-за своей гордости, спеси, а кроме того --- беспросветной
глупости ночью тревожишь хладные останки
своего наставника? Так? Говори же, говори, недотепа!

"--* Господин магистр речь идет о наиважнейшей вещи во
вселенной, --- о счастье всех разумных существ, --- выпалил
Трурль, и наклонившись, как для исповеди, над сеткой
микрофона, начал поспешно и торопливо рассказывать во всех
подробностях события, происшедшие после памятного разговора
с Клапауциусом, ничего не забывая и даже не пробуя скрывать
или приукрашивать.

Кереброн сначала хранил гробовое молчание, потом начал,
по своему обыкновению, вставлять многочисленные колкости,
замечания, иронические комментарии, язвительно или яростно
покашливать, но Трурлю, охваченному порывом, было уже все
равно, и он говорил без остановки, рассказал, наконец, о
своем последнем поступке, а потом замолчал и, переводя
дыхание, ждал. Кереброн, который до этого, казалось, все не
мог откашляться, сначала не издавал ни звука, а через
некоторое время звучным, как бы помолодевшим басом сказал:

"--* ну, вот что. Ты --- осел. А осел ты потому, что лентяй.
Никогда не желал ты усердно изучать общую онтологию. Если бы
знал ты философию, а точнее даже --- аксиологию, что я считал
своим святым долгом, то не прибежал бы на кладбище и не
колотил бы в мой гроб. Но, каюсь, есть в этом и моя вина!
Учился ты как последний лентяй, как слегка одаренный идиот,
а я смотрел на это сквозь пальцы, потому что был ты умельцем
в вещах низких, таких, какими часовщики хвалятся. Думал я,
что со временем ум твой дорастет и дозреет. А ведь, тупица,
я тысячу, нет --- сто тысяч раз говорил на семинарах, что
прежде чем делать --- надо подумать. Но ему, разумеется, и в
голову не пришло подумать! Собысчаса сотворил, великий
изобретатель, поглядите на него! В 10496 году профессор
неандр описал в "ежеквартальнике" машину, точь в точь такую
же, а драматург эпохи вырождения, некий биллион цикспир,
написал на эту тему драму в пяти актах, но ты ни научных, ни
художественных книг и в руки не берешь, так?

Трурль молчал, а разгневанный старец кричал все громче,
так что эхо отражалось от дальних гробов:

"--* доработался до преступления, неплохо! Может, не
знаешь, что запрещается разрушать или подавлять разум,
однажды созданный? Так, говоришь, речь шла только о
в-с-е-о-б-щ-е-м с-ч-а-с-т-ь-е? По пути же ты, с любовью и
благожелательностью, одних существ огнем палил, других --- топил
как мышей, заточал, запирал, казнил, ноги им ломал, а
под конец, как я понял, дошел до братоубийства? Для
в-с-е-о-б-щ-е-г-о-б-л-а-г-о-д-е-т-е-л-я и
у-н-и-в-е-р-с-а-л-ь-н-о-г-о б-л-а-г-о-ж-е-л-а-т-е-л-я

"--* совсем неплохо! И что я тебе теперь должен сказать?
Хочешь, чтобы я тебя приласкал из гроба? --- Тут он
неожиданно захохотал, так что Трурль задрожал с головы до
пят. --- Так говоришь, что перешел мой барьер? Для начала ты,
лентяй, свалил задание на машину, которая его спихнула
следующей, и так без конца, а потом сам себя запихнул в
компьютерную программу? Не знаешь ты, что ли, что ноль в
любой степени дает ноль? Скажите, пожалуйста, какой
гениальный --- сам себя а размножил, чтобы больше его было,
тоже мне умник! Или не знаешь, что цодеь галацтицус
запрещает самоумножение? Раздел ььжи тома 119, глава ь,
параграф 561 и следующие. Экзамен ты сдал с помощью
электронной шпаргалки и радиоподсказок, а теперь у тебя нет
иного дела, как вламываться на кладбище и колотить по
гробам! Разумеется! На последнем курсе я два раза, повторяю --- два
раза излагал кибернетическую деонтологию. Не перепутай
с дантистикой! Но на лекциях ты отсутствовал --- не
сомневаюсь, что по причине тяжелой болезни! Не правда ли?
Ну, говори сразу!

"--* Правда... Э... Я был нездоров, --- выдавил из себя
Трурль.

Он уже опомнился от первого шока и теперь особенно не
стыдился. Кереброн каким был при жизни --- таким остался и
после смерти, и у Трурля росла уверенность в том, что после
неизбежных ритуальных проклятий и брани начнется позитивная
часть: благородный в глубине души старец поможет ему своими
советами. Тем временем мудрый покойник перестал ругать его
последними словами:

"--* ну, хорошо! --- Сказал он. --- Ошибка твоя состояла в
том, что не знал ты ни того, чего хочешь достигнуть, ни
того, как это сделать. Это во-первых. Во-вторых: создание
в-е-ч-н-о-г-о с-ч-а-с-т-ь-я --- это детская забава, только
никому ни для чего не нужная. Твой прекрасный собысчас --- машина
не моральная, ибо приходит в одинаковый восторг и от
физических объектов, и от мук и страданий третьих лиц. Чтобы
построить счастьетрон, нужно поступить иначе. Вернувшись
домой, сними с полки ьььжи том моего "полного собрания
сочинений", открой его на странице 621 и изучи схему
экстрактора, которая там находится. Это единственный тип
безукоризненного разумного приспособления, которое не для
чего не служит, а только счастливо в 10.000 раз сильнее, чем
бромэо, когда видит возлюбленную на балконе. Поэтому, в
честь цикспира, описавшего эту сцену на балконе, я и назвал
эти единицы счастливости бромеями, а ты, не потрудившись
даже перелистать работы своего учителя, придумал какие-то
идиотские гедоны. Гвоздь в ботинке --- хорошенькая мера
высшего духовного подъема! Итак, экстрактор абсолютно
счастлив благодаря насыщению, которое является следствием
многофазного перехода в информационном континууме, то есть
происходит в нем автоэкстаз с дополнительной обратной
связью: чем больше он собой доволен, тем больше он собой
доволен, и так до тех пор, пока потенциал не дойдет до
предохранительных ограничителей. Потому что знаешь что может
случиться при отсутствии ограничителей? Не знаешь,
покровитель космоса? Машина пойдет в разнос и в конце концов
взорвется! Да-да, мой дорогой неуч. Потому что цепи... Но не
буду тебе посреди ночи на кладбище все выкладывать из
холодного гроба --- сам прочитаешь. Разумеется, труды мои либо
потонули в пыли на самой темной полке твоей несчастной
библиотеки, либо, что вероятнее, после моих похорон заперты
в сундук и находятся в погребе. А? Ведь из-за пары
глупостей, которые удалось тебе совершить, ты считаешь себя
величайшим человеком в метагалактике, не так ли? Где хранишь
мои "опера омнеа", отвечай немедленно?!

"--* В пог... Ребе, --- пробормотал Трурль, нагло солгав,
потому что давно уже свез их в трех подводах в городскую
библиотеку. Но, к счастью, не мог этого знать труп его
учителя, а потому, удовлетворенный своей проницательностью,
сказал уже ласковее:

"--* разумеется. Итак, счастьетрон этот абсолютен, но
абсолютно не нужен, поскольку сама мысль о том, что
туманности, планеты, луны, звезды, пульсары и всякие квазары
надлежит постепенно переработать в шеренги экстракторов,
может родиться только в башке, завязанной топологическим
узлом мебиуса и клейна, то есть в самом извращенном
сознании... Да на что мне это далось! --- Снова распалился
гневом покойник. --- Прикажу поставить на дверце замок йала и
зацементировать кнопку вызова. Твой друг, Клапауциус, вырвал
меня из сладких объятий смерти подобным же звонком в прошлом
году --- а может это было в позапрошлом, потому что нет у меня
тут, как ты понимаешь, ни часов, ни календаря --- и я только
потому восстал из мертвых, что этот мой знаменитый ученик не
мог сам справиться с метаинформационной антиномией теоремы
аристойдеса. И поэтому я, распавшийся в прах, должен был ему
излагать из гроба вещи, о которых он не знал даже, что они
находятся в каждом порядочном учебнике
континуально-топотропной инфинитеземалистики! О, боже, боже!
Какая жалость, что тебя нет, иначе ты давно бы наказал этих
киберсынов!

"--* А... Клапауциус был здесь... Э... У господина
профессора?! --- Обрадовался и одновременно чрезвычайно
удивился Трурль.

"--* Именно. Он что, не сказал тебе ни слова? Вот
благодарность роботов! Был, был. А ты-то чему рад. И скажи,
скажи-ка, --- оживился покойник, --- ты, у которого радость
вызвала неудача друга, хочешь целый космос осчастливить? И
не пришло в твою тупую голову, что для начала нужно
собственные этические параметры оптимизировать?

"--* Господин учитель и профессор, --- быстро сказал Трурль,
чтобы отвлечь внимание ехидного старца от собственной
персоны. --- Так что же, проблема осчастливливания
нерезрешима?

"--* Да нет, почему же?! Только в такой постановке она
некорректна. Ибо что такое счастье? Это просто как дважды
два. Счастье есть экстенсор метапространства, отделяющего
узел колинеационно интенциональных отображений от
интенционального объекта с граничными условиями, даваемыми
омега-корелляциями в альфа-мерном (разумеется,
неметрическом) континууме субсольных агрегатов, также
называемых моими, то есть кереброна, супергруппами. Конечно
же, ты даже и не слыхивал о субсольных агрегатах, над
которыми я работал сорок восемь лет, и которые являются
производными функционалов, называемых антиномиалами, моей
алгебры противоречий.

Трурль молчал.

"--* Когда приходишь на экзамен, --- сказал покойный с
непривычной, и потому подозрительной мягкостью, --- можно, в
конце концов, быть неподготовленным. Но не перелистнуть ни
одной страницы учебника, когда идешь к гробу профессора --- о,
это уже такая наглость, --- зарычал он так, что что-то
забренчало и затрещало в динамике, --- что если бы я еще жил,
меня наверняка на месте хватил бы удар! --- Он снова внезапно
смягчился. --- Значит, ничего ты не знаешь, как будто вчера
на свет родился? Хорошо, мой верный, мой удачный ученик,
загробное мое утешение! Ты не слыхивал о супергруппах,
поэтому я должен тебе объяснять популярным, упрощенным
образом, как если бы обращался к полотеру или автоматической
плите! Счастье, о котором стоит хлопотать --- не есть целое, а
только часть чего-то такого, что само не счастье и быть им
не может. Программа твоя была чистым кретинизмом, даю тебе
честное слово, а моим останкам ты можешь поверить. Счастье
не первично, оно лишь производная... Ну, этого ты уже не
поймешь, балбес. Сейчас ты передо мной покаешься, во всю
будешь кричать, что исправишся, что за работу засядешь, а
придешь домой --- и даже за мои труды сесть не подумаешь. --- Трурль
подивился догадливости кереброна, так как в глубине
души так сделать и намеревался. --- Нет, ты намерен попросту
взять в руки отвертку и на части разобрать машину, в которой
ты сначала заточил, а потом угробил самого себя. И сделаешь
ты, что захочешь, потому что не буду тебя пугать, восставая
из гроба, хотя ничего бы мне не стоило перед уходом в могилу
сконструировать соответствующий призракотрон. Но играть в
привидения и в их образе пугать своих дорогих учеников
показалось мне чем-то недостойным ни их, ни меня самого. Что
я --- нанимался в ваши загробные сторожа, несчастная банда?
Кстати, знаешь ли ты, что убил самого себя только один раз,
то есть в одном лице

"--* как это --- в одном лице? --- Не понял Трурль.

"--* Головой ручаюсь, что никакого университета и всех его
Трурлей с кафедрами в компьютере не было: ты говорил со
своим цифровым отражением, которое опасалось --- и не зря --- что
когда ты поймешь невозможность разрешения проблемы, то
выключишь его навеки, поэтому и врало напропалую.

"--* Не может быть! --- Изумился Трурль

"--* может. Какой емкости была машина?

"--* Ипсилон десять в десятой.

"--* В такой нет места для размножения цифирцев. Ты дал
себя надуть, в чем, правда, не вижу ничего дурного, и
поступок твой был кибернетически позорным. Трурль, время
идет. Наполнил ты мою душу отвращением, избавить от которого
может только черная сестра морфея --- смерть, последняя моя
подруга. Возвратившись домой, воскресишь кибербрата,
расскажешь ему правду, то есть этот наш кладбищенский
разговор, а потом выпустишь его из машины на свет божий и
материализуешь способом, который найдешь в "прикладной
рекрецианистике" моего учителя, незабвенной памяти
пракибернетика дулайхуса.

"--* А разве это возможно?

"--* Да. Конечно, мир, который с этих пор будет носить
целых двух Трурлей, встанет лицом к лицу с серьезной
опасностью, но не менее опасным было бы предать забвению
твое преступление.

"--* Но, простите, господин учитель... Ведь его уже нет...
Он не существует с момента, когда я его выключил, поэтому
сейчас уже, быть может, не стоит делать того, что вы
рекомендуете.

Вслед за этими словами раздался дрожащий от величайшей
ярости крик:

"--* великое объединение! И я дал этому чудовищу диплом с
отличием! О! Тяжело я наказан за промедление с уходом на
вечный покой! Видно, уже на твоем экзамене разум мой сильно
ослабел. Ка же это? Значит, ты считаешь, что раз в данный
момент твоего двойника нет среди живых, то тем самым не
существует и проблемы его воскрешения? Перепутал физику с
этикой! Остается только за лом хвататься! С точки зрения
физики все равно --- ты живешь, либо тот Трурль, либо оба,
либо ни один, бегаю я вприпрыжку или в гробу лежу, потому
что в физике нет состояний подлых и благородных, добрых и
злых, а только то, что есть --- существует, и точка. Но, о
наиглупейший из моих учеников, с точки зрения нематериальных
ценностей, то есть с точки зрения этики, все выглядит иначе!
Потому что если бы ты выключил машину, желая только, чтобы
твой цифровой брат заснул сном как смерть крепким, если бы
ты намеревался, вынимая вилку из розетки, снова воткнуть ее
туда утром, то проблемы братоубийства --- совершенного тобой
преступления --- вообще бы не существовало, и я не должен был
бы посреди ночи на эту тему драть себе горло, со смертного
ложа невежливо сорванный. Но пошевели мозгами, и поймешь,
чем с физической точки зрения различаются эти две ситуации --- та,
в которой ты выключаешь машину на одну ночь с невинным
умыслом и та, в которой ты делаешь то же самое, намереваясь
на веки умертвить цифрового Трурля! Вот именно --- с
физической точки зрения не различаются они ничем, ничем,
ничем!!! --- Грохотал он как иерихонская труба, и Трурль даже
успел подумать, что его почтенный учитель набрался в гробу
сил, каких ему в жизни не хватало. --- Только теперь заглянул
я в пропасть твоего невежества и содрогнулся! Как же это?
Значит, ты считаешь, что того, кто спит под наркозом сном,
глубоким как сама смерть, можно безнаказанно растворить в
серной кислоте, либо выстрелить им из пушки, поскольку его
сознание не функционирует? Теперь ответь: если бы ты мог
заковаться в колодки вечного счастья, то есть залезть внутрь
экстрактора с тем, чтобы пульсировать в нем чистым счастьем
в течении ближайшего двадцати одного миллиарда лет, и мог бы
не раздражать идиотскими вопросами труп своего профессора,
словно злодей, темными ночами ворующий информацию из гробов,
и не было бы у тебя никаких задач, дилемм, невзгод, проблем
и хлопот, которыми вымощена вся жизнь, то согласился бы ты
на такое предложение? Сменил бы активное существование на
блаженство в-е-ч-н-о-г-о с-ч-а-с-т-ь-я? Отвечай быстро --- да
или нет!

"--* Нет! Конечно нет! --- Закричал Трурль

"--* вот видишь, недоумок! Сам ты не хочешь быть
замурованным наглухо, заэкстаженным, ублаготворенным, а
целому космосу смеешь предлагать то, от чего тебя воротит.
Трурль! Мертвые видят ясно! Не можешь ты быть таким уж
законченным негодяем! Нет, ты только гений со знаком минус,
то есть кретин! Послушай, что я тебе скажу. Когда-то ничего
так не желали наши предки, как только бессмертия. Однако
едва только его создали и на моделях опробовали, поняли, что
не того им нужно! Разумное существо должно иметь перед собой
то, что возможно, а кроме того также и то, что невозможно!
Теперь каждый может жить столько, сколько захочет, а вся
мудрость и красота нашего существования в том, что когда
кто-то насытился жизнью и трудом, когда считает, что
исполнил то, для чего создан был, то удаляется на вечный
покой, как и я, наряду с другими, сделал. Раньше смерть
приходила неожиданно, прервав на середине не одну работу,
помешав закончить не одно дело --- и в этом состояла древняя
предопределенность. Но ценности сменились, и вот я ничего
так не желаю, как небытия, которое умышленно нарушают тебе
подобные, докучая мне постоянно, добираясь до моего гроба и
стягивая его с меня как одеяло. А ты запланировал космос
счастьем загромоздить, заселить, забить до отказа, якобы
чтобы всех в нем живущих усовершенствовать, а на самом деле
потому, что ты лентяй. Хочешь ты иметь всякие задачи,
проблемы и хлопоты, так скажи, что бы ты, собственно, в
таком мире делал дальше? Либо повесился бы с тоски, либо
взялся бы за разработку умертвляющей приставки к такому
счастью. Таким образом, от лени осчастливить всех хотел, от
лени проблему машинам передал, от лени самого себя в машину
запихнул --- то есть оказался самым своеобразным из тупиц,
каких обучал я в течение тысячи семисот двадцати семи лет
своей академической карьеры! Если бы не понимал я тщетности
этого, то отвалил бы надгробный свой камень и дал бы тебе по
лбу! Пришел ты к гробу за советом, но ты стоишь не перед
чудотворцем, и не в состоянии я отпустить тебе даже самого
маленького из множества твоих бесчисленных грехов, мощность
которого аппроксимируется пра-канторовой
алеф-бесконечностью!.. Вернешся домой, разбудишь кибербрата
и сделаешь, что я тебе сказал.

"--* Но, господин...

"--* Помолчи. Когда же сделаешь это, возмешь ведро
раствора, лопату, мастерок, придешь на кладбище и как
следует зацементируешь отверстие в склепе, через которое ты
добрался до гроба и свалился мне на голову. Понятно?

"--* Да, господин учитель.

"--* Сделаешь это?

"--* Обещаю, что сделаю, господин учитель, но я хотел
бы еще узнать..

"--* А я, --- произнес мощным, воистину громовым голосом
покойник, --- хотел бы узнать только, когда же ты уберешся
прочь. Только попробуй постучаться в мой гроб еще раз, и я
тебе такое покажу... Впрочем, ничего конкретно не обещаю --- сам
увидишь. Можешь передать от меня привет Клапауциусу и
сказать ему то же самое. В последний раз, когда я его
поучал, он так торопился, что не потрудился даже выразить
мне должной благодарности. Ох, манеры, манеры этих способных
конструкторов, этих гениев, этих талантов, у которых от
спеси извилины узлом завязались!

"--*  Господин..,  ---  Начал  Трурль,  но тут  в  гробу  что-то  треснуло,
зашипело, кнопка,  которая была  утоплена, выскочила, и  глухая тишина
повисла  над  кладбищем.  Слышен  был  лишь  напоминающий  мягкое  эхо
далекий  шум  ветвей.  Вздохнул   тогда  Трурль,  почесал  в  затылке,
подумал,  усмехнулся,  представив  себе  Клапауциуса,  ошеломлением  и
стыдом которого предстояло ему насладиться во время ближайшего визита,
поклонился  возвышению  гробницы,  а  потом повернулся  и,  веселый  и
безмерно собой довольный, помчался домой, да так быстро, словно кто-то
за ним гнался.


Генрих Альтов

Перечитывая Уэллса


Быть может, самая  человеческая черта  в человеке —  способность думать  о
прошлом и о будущем.  В особенности — о  будущем. Именно поэтому  наследие
писателя-фантаста прежде всего связывается  с его фантастическими  идеями.
Могут устареть  сюжеты, могут  потускнеть образы  героев, может,  наконец,
обветшать художественная  ткань  произведений.  Но  неизменно  сохраняется
интерес ко всем хоть сколько-нибудь примечательным предвидениям фантастов.
Так случилось, например, с повестью  «Машина времени». В 1931 году,  через
тридцать шесть лет после первого ее издания, Уэллс писал, что повесть если
не иметь в виду главной ее мысли — устарела не только с художественной, но
и с философской стороны. «Автору, достигшему ныне зрелости, — пишет Уэллс,
она кажется попросту ученическим  сочинением». Однако идея машины  времени
жива и сегодня: она прочно вошла в культурный фонд человечества.

Это, конечно, не значит, что в фантастике идеи «главнее» художественности.
Научная фантастика — синтетический литературный жанр, в котором  одинаково
важны оба компонента. Я хочу лишь сказать, что судьбы фантастических  идей
интересны и сами  по себе,  ибо идеи эти  обладают удивительным  свойством
выходить за рамки литературы. Так, роман Жюля Верна «Из пушки на Луну» дал
толчок работам Циолковского. Подобных примеров множество.

Прослеживая  судьбы  фантастических  идей,  мы  начинаем  лучше   понимать
«технологию» фантастики.  И --- что еще  важнее --- мы  отчетливее видим  контуры
будущего.

Пока нет науки о предвидении будущего. Долгое время фантастика была (да  и
сейчас еще остается) единственным окном в будущее. Лишь в самые  последние
годы возник  вопрос  о  превращении предвидения  из  искусства  в  науку.
Характерно название одной из первых книг на эту тему: она написана в  1962
году английским  ученым С.  Лилли и  названа «Может  ли предвидение  стать
наукой?».  Лилли   положительно отвечает  на   этот  вопрос:   «Техническое
прогнозирование должно  стать  важным элементом,  помогающим  планировать
будущее».

Сейчас наука принимает у  фантастики эстафету. Ученые начинают  планомерно
использовать методы,  исконно  принадлежавшие фантастам.  Академик  А.  Н.
Колмогоров говорит, например,  в одной  из своих  статей: «На  современном
этапе при этом не следует  пренебрегать и построением „в запас“  несколько
произвольных гипотез,  как  бы  ни сближалась  иногда  такая  деятельность
ученого с построениями писателей-фантастов».

Предвидения фантастов  —  своеобразные  эксперименты  по  проникновению  в
будущее.  Иногда  эти  эксперименты  оказываются  неудачными,  иногда  они
завершаются  блестящим   успехом.   Но   в  обоих   случаях   есть   нечто
притягательное и волнующее  в этих попытках  заглянуть в будущее.  Вполне
закономерно поэтому стремление  проследить судьбу  фантастических идей.  В
1963 году в альманахе «Мир приключений» была опубликована таблица  «Судьба
предвидений Жюля Верна».

И вот сейчас перед вами аналогичная таблица, составленная по произведениям
Герберта Уэллса.

Надо сразу сказать: анализ  предвидений того или иного  фантаста — это  не
игра в  «угадал —  не угадал».  Дело совсем  не в  том, чтобы  установить:
писатель  Икс  почти  всегда  «угадывал»,  а  писатель  Игрек  обычно  «не
угадывал». Нет.  Намного важнее  понять «технологию»  фантастики.  Поэтому
таблица «Судьба  предвидений Уэллса»  это  материал для  раздумий.  Прежде
всего, для  раздумий  о  том, как  возникают  фантастические  «построения»
(воспользуемся термином А.  Н. Колмогорова)  и от чего  зависит, если  так
можно  выразиться,   их   проницающая   способность,   то   есть   глубина
проникновения в  будущее.  Таблица,  кроме  того,  повод  для  раздумий  о
будущем. Собранные  воедино, удачные  предвидения  Уэллса дают  далеко  не
полную, но интересную картину будущего.

Таблица требует размышлений.

Если  бы  все  выводы  из  таблицы  можно  было  перенести  сюда,  в  этот
комментарий, отпала  бы  необходимость  в  самой таблице.  Да  и  вряд  ли
содержащиеся в таблице оценки идей Уэллса абсолютно точны. Некоторые  идеи
относятся к далекому  будущему — тут  оценки поневоле субъективны.  Время,
надо полагать, внесет заметные коррективы в таблицу.

В  рассказе  «Филмер»  Уэллс  довольно  подробно  говорит  о  дирижабле  с
управляемой оболочкой,  способной сжиматься  и  расширяться. Эта  идея  за
десять лет  до  Уэллса  была выдвинута  Циолковским  в  брошюре  «Аэростат
металлический управляемый». Эксперты,  даже благожелательно настроенные  к
Циолковскому, решительно  отвергли идею  подобных дирижаблей.  Это  мнение
господствовало  вплоть   до  самого   последнего  времени.   В   биографии
Циолковского, написанной М.  Арлазоровым и  изданной в 1962  году в  серии
«Жизнь замечательных людей», глава, посвященная этой работе  Циолковского,
красноречиво названа «История великого заблуждения»…

Итак, еще в 1962 году можно было считать, что Уэллс, как и Циолковский,  в
данном случае  ошибался. Но  уже в  1963 году  в печати  появились  первые
сообщения, свидетельствующие о возрождении интереса к дирижаблям. В  книге
Арлазорова   дирижаблестроители   названы   «представителями   исчезнувшей
инженерной специальности». Сейчас в ряде стран существуют  конструкторские
бюро, разрабатывающие проект  новых дирижаблей. И  едва ли не  центральное
место в этих разработках занимают идеи, высказанные в брошюре Циолковского
и в рассказе Уэллса.

Судьбу предвидений трудно предвидеть.

Читатель вправе  по-своему  оценить  те или  иные  идеи  Уэллса.  Таблица,
повторяю, это материал для размышлений.

Но некоторые выводы все-таки хотелось бы сделать.

Уэллс — один из наиболее «фантастических» фантастов.

Укоренилось  представление,  что  «фантастичный»  Уэллс  как  бы   антипод
«научного» Жюля Верна. Да и сам Уэллс говорил об этом. Но вот оказывается,
что из  86  предвидений  Уэллса  30 уже  сбылись,  а  27  почти  наверняка
сбудутся. Иными словами, 57 предвидений из 86 попали точка в точку! Еще 20
идей можно  считать  принципиально  осуществимыми. И  лишь  9  предвидений
ошибочны.

Цифры,  конечно,  не  абсолютно  точны.   Но  суть  дела  именно   такова:
«отчаянный» фантаст Уэллс оказывается не менее «научным», чем Жюль Верн.

Что же помогало Уэллсу видеть будущее?

Прежде всего, очень тонкое понимание законов развития техники. Вот одно из
наиболее  известных  предвидений  Уэллса:  в  романе  «Освобожденный  мир»
(1913 г.) говорится, что первая  атомная электростанция будет построена  в
1953 году.

Точность предвидения поразительная, ведь  первая советская АЭС вступила  в
строй в 1954 году!

Что это — случайное совпадение?

Нет.  Перечитывая   «Освобожденный   мир»,  видишь   развернутую   систему
аргументации. Чтобы прогнозировать будущее, Уэллс вдумчиво всматривается в
прошлое.  Он   применяет,   например,   метод   аналогии:   «Существование
электромагнитных волн было неопровержимо доказано за целых двадцать лет до
того, как Маркони нашел  для них практическое применение,  и точно так  же
только через двадцать  лет искусственно  вызванная радиоактивность  обрела
свое практическое воплощение.»  Не забывайте, эти  строки писались в  1913
году!

Способность «ощущать» сроки, видеть не только что будет, но и когда будет,
играет огромную роль в  «технологии» фантастики. Правильная оценка  сроков
далеко не всегда удается даже признанным мастерам фантастики. В этой связи
интересно вспомнить,  как менялись  сроки действия  в романе  И.  Ефремова
«Туманность Андромеды». Вот что  говорит об этом  Ефремов в предисловии  к
своему  роману:  «Сначала  мне  казалось,  что  гигантские  преобразования
планеты и жизни, описанные в романе, не могут быть осуществлены ранее  чем
через три тысячи лет. Я исходил в расчетах из общей истории  человечества,
но не учел темпов ускорения  технического прогресса и главным образом  тех
гигантских  возможностей,  практически  почти  беспредельного  могущества,
которое даст человечеству коммунистическое общество.

При доработке романа я сократил намеченный сначала срок на тысячелетие. Но
запуск искусственных спутников Земли подсказывает мне, что события  романа
могли бы совершиться еще раньше».

Журналист Ю. Новосельцев, редактировавший в свое время журнальный  вариант
«Туманности Андромеды»,  рассказывает,  что  спустя  несколько  лет  после
опубликования романа он  спросил Ефремова,  не произойдет  ли описанное  в
романе уже через сто лет? Ефремов «пожал плечами, улыбнулся и ответил: Всё
может быть…»

Итак, три тысячи лет и сто лет — в таком диапазоне меняется время действия
событий романа  «Туманности Андромеды».  Но  между людьми,  которые  будут
через сто лет  и через три  тысячи, — огромная  разница! Три  тысячелетия,
например, отделяют современного  человека, открывшего  дверь в  безбрежный
космос, от  древнего египтянина,  стоявшего на  одной из  первых  ступеней
цивилизации. Совсем  не  все равно  —  «дать» атомную  энергию,  ракеты  и
кибернетику нашим современникам или древним египтянам.

Соответствие описываемого будущего  общества своему  уровню техники  очень
важно  для   художественной  убедительности   фантастики.  Это   одно   из
обязательных условий того синтеза, который воедино сплавляет в  фантастике
науку и литературу.

Уэллсу в  высшей  степени присуще  «чувство  времени», но  и  он  допустил
примечательную (и поучительную!) ошибку.

В том  же романе  «Освобожденный мир»  Уэллс пишет,  что полеты  в  космос
начнутся лишь тогда, когда на Земле людям уже не останется никакой работы.
В неторопливом XIX веке технике  было свойственно говорить «б» не  раньше,
чем сказано «а»  и выдержана должная  пауза. Уэллсу, опубликовавшему  свои
первые  произведения  еще  в  1895   году,  не  всегда  удавалось   понять
скороговорку XX века…

И все-таки  Уэллс  ошибался удивительно  редко.  Он пристально  следил  за
развитием науки и  техники. Когда  же ему, говоря  языком кибернетики,  не
хватало информации, он использовал методы литературы.

Вот конкретный  пример.  В «Войне  миров»  Уэллс хочет  показать  разумных
существ, цивилизация  которых  совсем  не  похожа  на  земную.  Это  чисто
писательская, чисто литературная задача. И Уэллс решает ее последовательно
и с большим литературным мастерством.

Земная техника немыслима без колес. Колесо. — основа основ нашей  техники.
Трудно представить себе машину, у которой  нет колес. Но Уэллсу как раз  и
нужно то, что трудно представить!

И  вот  Уэллс   ставит  интереснейший   эксперимент:  шаг   за  шагом   он
«конструирует» — во всех  деталях — технику,  которая не применяет  колес.
Постепенно вырисовывается  картина чужого  технического мира  с  машинами,
очень похожими на живые существа.

Думал ли  в  этот  момент  Уэллс о  реальной  возможности  создания  такой
техники?

Вряд ли.  Современная ему  техника гордилась  своим отличием  от  природы.
Казалось вполне логичным, что техника  будет все дальше и дальше  отходить
от природы.

Но нарисованная Уэллсом  картина уже жила  своей логикой, и  он не мог  не
увидеть преимуществ техники,  копирующей природу.  Уэллс смело  предсказал
наступление бионической эры в земной технике; это одно из наиболее удачных
его предвидений.

Если  внимательно  перечитать  «Войну   миров»,  нетрудно  заметить,   что
марсианские машины  в начале  романа довольно  неуклюжи: «Можете  вы  себе
представить складной  стул, который,  покачиваясь, переступает  по  земле?
Таково было это видение  при мимолетных вспышках  молнии. Но вместо  стула
представьте себе громадную машину,  установленную на треножнике». Здесь  в
марсианских машинах есть еще что-то от земных паровозов. Боевые треножники
марсиан идут  «с  металлическим звонким  ходом»:  Из их  суставов  (совсем
«по-паровозному»!) вырываются клубы зеленого дыма…

Во  второй  половине  романа  марсианская  техника  в  изображении  Уэллса
становится более  совершенной. Теперь  Уэллс чаще  сравнивает  марсианские
машины с  живыми  существами: «Все  движения  были так  быстры,  сложны  и
совершенны, что  сперва  я  даже  не принял  ее  за  машину,  несмотря  на
металлический  блеск».   Это   уже   не  «шагающие   стулья»,   а   «почти
одухотворенные механизмы».

У таких «одухотворенных» механизмов не могут вырываться из суставов  клубы
зеленого дыма. Это было бы художественно недостоверно. Машина, похожая  на
живое  существо,   должна   иметь   почти  живые   двигатели.   И   логика
художественного образа заставляет Уэллса сделать следующий шаг.

«Затронув эту тему, — пишет он, — я должен упомянуть и о том, что  длинные
рычажные соединения  в  машинах  марсиан приводятся  в  движение  подобием
мускулатуры,  состоящим  из  дисков  в  эластичной  оболочке;  эти   диски
поляризуются при прохождении электрического тока и плотно прилегают друг к
другу.  Благодаря  такому  устройству   получается  странное  сходство   с
движениями  живого  существа,   столь  поражавшее   и  даже   ошеломлявшее
наблюдателя».

Только в  середине  XX  века  люди  пришли  к  идее  безколесной  техники,
копирующей природу.  У природы,  конечно,  и раньше  перенимали  отдельные
решения, но лишь сейчас формируется новая наука — бионика, которая как раз
занята созданием машин, подобных марсианским машинам Уэллса.

Научная фантастика  — отнюдь  не  простая «смесь»  науки и  литературы.  В
научной   фантастике,   как   ни   парадоксально,   наука   работает    на
художественность, позволяя создавать  литературно впечатляющие картины.  В
свою очередь чисто литературные средства помогают увидеть далекое будущее,
скрытое еще от современной писателю науки.

Предвидеть будущее — это как бы смотреть далеко вперед.

Тут две  возможности. Либо  впереди  нет поворотов,  и тогда  видно  очень
далеко. Можно смотреть до горизонта  (правда, нужно хорошее зрение).  Либо
другой случай: писатель пытается увидеть то, что находится близко, но за
поворотом.

Уэллс применял  оба  приема.  Иногда  он  просто  смотрел  далеко  вперед.
Впрочем, это «просто» не так уж  просто. Нужно не поддаться гипнозу  моды,
всегда в чем-то излишне оптимистичной  и в чем-то излишне  пессимистичной.
Уэллс,  например,  предсказал  большое  будущее  популярной  в  то   время
пневматической почте — и ошибся. Он вообще ошибался преимущественно в  тех
случаях,  когда  переставал  фантазировать.  Удачные  же  его  предвидения
искусственное   получение   алмазов,   батисфера,   атомная   энергия   на
транспорте сделаны    вопреки    господствовавшему    мнению,    гласившему
«невозможно».

С еще большим мастерством Уэллс умел «заглянуть за поворот», увидеть то, о
чем наука  вообще пока  не имеет  определенного мнения.  Уэллс  использует
здесь писательскую логику (как при  описании марсианской техники в  «Войне
миров»). Он придумал «тепловой луч» — задолго до Алексея Толстого. Писал о
передаче знаний по наследству. Писал о памяти, хранящей увиденное далекими
предками…

Могут спросить: ну, хорошо, марсианскую технику Уэллсу подсказала природа,
а что направляло писательскую логику в других случаях?

Наука о предвидении, когда она будет создана, вероятно, введет понятие  об
«идеальной машине».  В теории  предвидения  этому понятию  суждено  играть
такую же роль, какую  играют понятия «информации»  или «обратной связи»  в
кибернетике.

Любая  машина  —   не  самоцель,  она   только  средство  для   выполнения
определенной  работы.  Например,   вертолет  предназначен  для   перевозки
пассажиров и грузов. При этом мы вынуждены — именно вынуждены! —  «возить»
и сам  вертолет.  Понятно,  вертолет будет  тем  «идеальнее»,  чем  меньше
окажется его  собственный  вес. Идеальный  вертолет  состоял бы  из  одной
только пассажирской кабины.

Идеальная машина — условный эталон.  Это машина почти невесомая, почти  не
требующая энергии, почти не  занимающая объема и в  то же время  способная
делать все то, что  делают реальные машины.  Можно сказать так:  идеальная
машина — когда нет никакой машины.

В  технике  прогрессивными  оказываются   только  те  тенденции,   которые
приближают реальную машину к идеальной.

Один из главных  секретов фантастической «технологии»  и состоит в  умении
ориентироваться на идеальную машину.

Кейворит Уэллса,  человек-амфибия Беляева,  опыт Мвен  Маса в  «Туманности
Андромеды» Ефремова  —  яркие  примеры удачного  приближения  к  идеальной
машине.

В романе  «Когда спящий  проснется»  Уэллс говорит  о гипнопедии.  По  тем
временам идея представлялась чистейшей фантастикой. Но ведь и это типичная
идеальная машина (в широком смысле слова): обучение без затрат времени  на
учебу.

Представление об идеальной машине — надежный компас фантаста. К сожалению,
далеко  не  все  современные  писатели-фантасты  умеют  пользоваться  этим
компасом.  Все  еще  бытует  наивное  представление,  что  будущие  машины
обязательно «большие-пребольшие»  и «сильные-пресильные».  Можно  привести
такой пример.  В  рассказе  М.  Емцева и  Е.  Парнова  «Последняя  дверь!»
упоминается личный автолет. Двухместная прогулочная машина имеет двигатель
в тысячу лошадиных  сил! Это  продиктовано стремлением  сделать машину  не
«идеальнее», а «шикарнее»…

Уэллс, постоянно  ориентировавшийся  на  «идеальную  машину»,  разумеется,
должен был  прийти  к  рассказу  «Чудотворец».  Мистер  Фодерингей,  герой
рассказа,  вдруг  приобретает   способность  делать   все,  что   захочет.
Достаточно слова, желания. Самая идеальная из всех идеальных машин…

«Чудотворец»   считается   классическим   образцом   чисто    литературной
фантастики. У литературоведов  нет и тени  сомнения, что здесь  фантастика
использована только как литературный прием.

Я не включил этот рассказ в таблицу. Но перечитайте рассказ, и вы увидите,
что  среди   чудес   мистера   Фодерингея   нет   ничего   принципиального
неосуществимого!  Рано  или  поздно  люди  научатся  делать  такие  чудеса
(некоторые мелкие чудеса,  пожалуй, доступны уже  сегодня). Придет  время,
когда рассказ «Чудотворец» будет  считаться скромной научной  фантастикой.
Случилась же такая метаморфоза с «Человеком-амфибией» Беляева…

Современным  фантастам  порой  приходится  оправдываться,  отстаивая  свое
мнение в  споре с  наукой. К  Уэллсу же  привыкли относиться  иначе:  это,
дескать, чистая фантастика, условный прием…

А вдруг он был хитрецом, этот Уэллс?

Быть   может,  он   писал  самую   настоящую  научную   фантастику,  а
притворялся, что просто так фантазирует?..

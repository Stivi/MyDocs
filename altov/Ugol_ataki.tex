
Генрих Альтов

Угол атаки


Рассказ

Я написал  слово «рассказ»  и  подумал: какой  же  это рассказ,  если  все
произошло в действительности, и я просто вспоминаю то, что было…  Впрочем,
странная это  штука  —  память.  Сначала  она  работает  как  старательный
фотограф на снимках все  точно так, как было.  Но проходят годы, и  память
становится художником, причудливым и  своенравным: что-то убирает,  что-то
добавляет, вдруг  ярко окрашивает  что-то неглавное,  делает его  главным,
меняет лица, одежду, погоду, словом, все перерисовывает на свой лад.

Думая сейчас  об этой  истории, я  почему-то прежде  всего вспоминаю  ночь
перед испытанием  «Черепахи».  Мы  сидели  у  самой  воды  и  смотрели  на
«Черепаху». С  вечера,  пересилив  летний  бакинский  зной,  подул  ветер,
«Черепаха» тихо покачивалась в двух метрах от нас.

— Обидно  будет, —  сказал Яшка,  —  очень будет  обидно, если  завтра  ты
вмажешь в тот берег.

Тут надо сказать, что «Черепаха» была  первым в мире ракетным кораблем.  В
наши дни это звучит вполне обыденно: ракетный корабль, кого теперь удивишь
ракетами. Но я рассказываю о июле 1940 года, а в те времена слово «ракета»
звучало столь же  фантастично, как звучит  сейчас слово «машина  времени».
Представьте себе,  что  рядом  с  вами стоит  первая  в  мире  машина  для
путешествий в прошлое и  будущее, причем вы ее  сами построили, и вот  ваш
друг предостерегает вас: «Завтра первое испытание, так ты уж,  пожалуйста,
постарайся не врезатьс в каменный век…»

До того  берега  было метров  двести,  не больше.  Как  только  «Черепаха»
рванется, надо будет  мгновенно выключить  двигатель, чтобы  не вмазать  в
бетонную стену. «Черепаху» следовало бы испытывать  в море, а не здесь,  в
маленьком пруду на окраине  загородного парка. Но это  уже зависело не  от
нас.

Итак, был июль 1940 года. Утром должна была приехать комиссия, и мы сидели
у воды и смотрели  на «Черепаху». Мы  строили ее целый  год, а когда  тебе
пятнадцать, год — это очень много, это почти половина жизни.

Но лучше  рассказать все  по порядку.  Созданию «Черепахи»  предшествовала
извилистая цепь  событий. Собственно,  сначала никаких  особых событий  не
было. До седьмого класса жизнь была удивительно проста, и дальнейшее  тоже
представлялось простым и очевидным: кончу школу, поступлю в военно-морское
училище. А пока я занимался в морском кружке Дома пионеров, читал книги  о
морских сражениях,  бегал в  яхтклуб и  при бурной  деятельности  умудрялс
все-таки не получать двоек. Яшка учился  на пятерки, с ним тоже почти  все
было ясно: он станет врачом,  как его отец и дед.  Мы жили в одном  дворе,
учились в  одном  классе и  сидели  за  одной партой.  Яшка  был  типичным
«отличником», но  я  считал, что,  может  быть,  в конце  концов  из  него
получится морской врач. Так что дружба у нас была полная.

События начались в седьмом  классе, когда в  нашей группе появилась  Дина.
Раньше она жила в маленьком городке, не помню уже, каком. Яшка сказал, что
она — из сказок Андерсена, это я почему-то запомнил. Впрочем, сказки  меня
нисколько не волновали, я тогда читал книгу «Подводна война в Атлантике  в
1914-18 г. г.» К книге были приложены и карты, и схемы, по ним я следил за
действиями  подводных  лодок  и   кораблей-охотников.  Дину,  однако,   не
устраивало, что кто-то не замечает ее присутствия. На большой перемене она
подошла  ко  мне  и   поинтересовалась  книгой.  Мы  немного   поговорили.
Оказалось, что  Дина  умеет  плавать кролем,  что  «Остров  сокровищ»  она
смотрела четыре раза  и что она  слышала о Ютландском  бое. После этого  я
стал замечать Дину, очень даже стал замечать, хотя вскоре выяснилось,  что
о Ютландском бое ей рассказал Яшка, которому  за месяц до этого я сам  все
подробно объяснил…

Так вот, как-то пронесся слух, что в Доме пионеров организуют новый кружок
— химический. Да, тут надо сказать  несколько слов об этом Доме  пионеров.
По тем  временам он  был  сверхобразцовым и  сверхпоказательным:  огромное
здание  бывший  дворец  нефтепромышленника,  великолепное  оборудование  в
кружках и мастерских, спортзал,  сад, концерты, кино,  а главное —  просто
потрясающие руководители.  Морским  кружком  руководил  Сергей  Андреевич,
самый настоящий капитан, самого настоящего дальнего плавания, знавший  все
моря и океаны. А у наших соседей, планеристов, руководителем был летчик  с
орденом Боевого Красного Знамени. Когда он, чуть прихрамывая, проходил  по
коридору, мы затаив дыхание, смотрели на его орден. Попасть в Дом пионеров
было очень трудно, все кружки были переполнены, и потому, услышав о  новом
кружке, Дина объявила, что надо сразу пойти и записаться. К этому  времени
она  запросто   командовала   всем   классом.   По   современной   научной
терминологии, она стала признанным лидером  группы. Химия ее нисколько  не
интересовала, других ребят  тоже, но  Дина очень  логично объяснила,  что,
записавшись в химический кружок, можно  получить пропуск в Дом пионеров  и
бывать на вечерах и концертах. Так что человек десять пошли и  записались.
Я тоже записался, хотя нельз было одновременно заниматься в двух  кружках,
за это могли вообще выгнать.

На занятиях мы с Димой садились  за последний стол, и не обращая  внимания
на химические опыты, болтали о всякой чепухе. Химический кружок  находился
двумя этажами выше  морского, но все-таки  мне приходилось постоянно  быть
настороже. Яшка добросовестно  осваивал химию, а  после занятий мы  вдвоем
провожали Дину.

И вот  однажды Дина  заболела и  не пришла  в школу.  Не появилась  она  и
вечером, на занятиях  в химкружке.  Я, как обычно,  устроилс за  последним
столом. Рядом со мной оказался Витька. Химия его не интересовала, в кружок
он записался потому, что дома его заставляли играть на виолончели.  Витька
все время таскал с собой книги «про шпионов» и считалс авторитетом в  этой
области. На  этот раз  он тоже  читал какую-то  книжку и  громко сопел  от
переживаний. Я  послушал,  как  он сопит,  и  перебрался  вперед.  Николай
Борисович, химик, рассказывал  о красках. Я  тогда скептически относилс  к
Николаю Борисовичу, уж очень он отличался от нашего капитана и летчика  из
планерного. Николай Борисович был похож на повара: круглый, в белом халате
и белой шапочке,  прикрывающий блестящую лысину.  Мы называли его  Колбой,
это получилось от сокращения и соединения «Коля» и «Боря». Так вот, в  тот
вечер Колба рассказывал  о красках.  Почему красная  краска имеет  красный
цвет, а синяя  синий. Какие краски  были во времена  Рамзеса Второго.  Как
Леонардо да Винчи искал новые краски для своих картин. Как живут краски  и
почему они  стареют.  Как синтезировать  индиго  и что  такое  акварельные
краски, темпера, гуашь… Об этом было  полезно послушать, потому, что я  на
собственном опыте знал,  что морское  дело наполовину  состоит из  чистки,
мойки и окраски. Мы постоянно драили, красили и перекрашивали «Партизана»,
учебный пароход Детюнфлотилии, шпаклевали и красили швертботы в  яхтклубе,
а стометровый пирс яхтклуба мы красили дважды в год.

Колба рассказывал как-то странно. Казалось, он говорит сам с собой. На нас
он не смотрел. Иногда он  замолкал или бормотал что-то непонятное.  Иногда
спорил сам с собой.  Опыты он тоже показывал  сам себе и веселился,  когда
все получалось  как  надо.  Потом  он  все-таки  вспомнил  о  нас,  раздал
пробирки, реактивы и начал  объяснять, что сейчас  из этих веществ  каждый
получит какую-нибудь  яркую  краску.  Мне  достались  оранжевые  кристаллы
двухромовикислого аммония.  Они  красиво  сгорели,  разбрасывая  искры,  и
получилась зеленая окись хрома. На этом я остановился, потому, что не было
никакого смысла переводить окись хрома в краску: из окиси хрома можно было
сделать потрясающую мастику, я видел  ее у нашего боцмана, такой  мастикой
удавалось за пять минут надраить до зеркального блеска бляху на ремне.

Химия мне  понравилась.  Я  почувствовал, что  моряку  полезно  ее  знать.
Скажем, кораблекрушение  —  и ты  на  необитаемом острове.  Химия  поможет
сделать из всякой чепухи то, что нужно. Например, взрывчатку.

Когда через  две  недели  появилась  Дина, я  отнесс  к  этому  совершенно
безразлично. Я сидел  возле Колбы  и старательно ловил  каждое его  слово.
Дина  перестала  со  мной  разговаривать,  домой  ее  провожал  Яшка.  Это
нисколько не  отразилось  на  моем  отношении к  химии.  Теперь  вместе  с
«Основами морской  практики» я  таскал  в портфеле  «Занимательную  химию»
Рюмина.

Химия, как говорил  М. В. Ломоносов,  далеко простирает свои  руки в  дела
человеческие. М.  В. был  прав, я  в этом  убедился. Чтобы  разобраться  в
химии, нужно знать физику, а  чтобы знать физику, нужно знать  математику,
надо  вкалывать.  Я  теперь  подолгу  готовил  уроки,  дома  это   вызвало
переполох, потому что  раньше я  все делал за  полчаса: что-то  быстренько
прочитаешь, что-то на ходу спишешь у Яшки. У меня не хватало времени, стал
подниматься на  час раньше,  чтобы на  свежую голову  порешать задачки,  и
мама, окончательно  перепугавшись,  повела  меня  к  известному  в  городе
детскому доктору Клупту.  Старенький Клупт долго  выслушивал и  выстукивал
меня, разглядывал мой язык  и вздыхал. Потом  он сказал: «Сложный  случай.
Приведите-ка его еще  раз. Попозже…» Мама,  побледнев, спросила:  «Когда?»
Клупт пожал плечами: «Лет эдак через сорок. Или пятьдесят». Седьмой  класс
я кончил  почти без  троек. Летом  химкружок не  работал, я  целыми  днями
пропадал на «Партизане», а по вечерам читал «Химию для всех» Партингтона.

В сентябре, на первом же занятии, Колба объявил, что предстоит  всесоюзная
олимпиада  детского  технического  творчества,  и  мы  должны  подготовить
действующие модели химических заводов  — солянокислотного, сернокислого  и
так  далее.  На  следующий  день  Сергей  Андреевич  тоже  сказал   насчет
олимпиады: нужны, мол, модели кораблей.

Я понял, что  горю: сработать одновременно  две модели просто  невозможно.
Однажды я делал модель  тральщика, это потребовало  уйму времени. И  потом
что это такое — модель солянокислотного завода? Обыкновенна установка  для
получения соляной кислоты — как в школьном учебнике: пробирки,  спиртовки,
трубки.  Только  все  это  прикрыто  фанерными  коробками,   изображающими
заводские помещения. Ничего вдохновляющего. К тому же мастерские одни  для
всех кружков, так что я погорю совершенно обязательно.

И вот тут у меня появилась  гениальная идея. Как все гениальное, она  была
проста, эта идея.  Точнее — казалась  простой. Я даже  не подозревал,  как
потом все усложнится…

Идея заключалась в том,  чтобы вместо двух моделей  сделать одну. Катер  с
химическим двигателем.  Сергей  Андреевич  будет  доволен,  ведь  катер  —
морская модель. А  Колба тоже  ничего не сможет  возразить, поскольку  вся
начинка у катера будет  химическая. И вообще  мое незаконное пребывание  в
двух кружках станет законным и даже необходимым.

Я быстро  уговорил  Яшку  работать  на  пару.  Потом  мы  пошли  к  Сергею
Андреевичу.

Так что это такое — химический двигатель? — спросил он.

К этому вопросу я был готов.  Нужно получить водород и кислород, затем  их
смешать —  образуется гремучий  газ.  А дальше  все просто:  гремучий  газ
поступает в камеру  сгорания, бах —  и модель рванется  вперед. Потом  еще
один бах. И так далее.

— Значит,  ракета, —  с сомнением  сказал Сергей  Андреевич. —  Ничего  не
выйдет, но попробуйте. Для практики.

Колба выслушал нас без всякого энтузиазма, но возражать не стал. Он только
спросил, как  мы собираемся  получить водород  и кислород.  Мы  объяснили:
поставим на модель электрическую батарею, она будет разлагать воду — вот и
все.

— Электрическая батарея?  — переспросил  Колба. Он  посмотрел на  потолок,
что-то пошептал и буркнул. — Ладно, вам виднее.

Дине мы тоже рассказали о ракетном  катере. Она давно ушла из  химического
кружка в драматический.  Вообще, она изменилась.  Яшка как-то сказал,  что
теперь она  — из  рассказов Александра  Грина. Я  считал, что  все дело  в
высоких каблуках. Если надеть туфли  на высоких каблуках да еще  соорудить
на голове  башню из  волос —  запросто можно  состаритьс на  пять лет.  На
переменах к Дине приходили десятиклассники,  нас она замечала, когда  надо
было списать задачку по физике или математике.

— Ах, мальчики, чепуха этот ваш катер, — сказала она. — Лучше сделайте мне
ацетон. Нигде не могу достать.

Ацетон ей был нужен чтобы смывать маникюр. В школу нельзя было приходить с
маникюром, с  этим было  строго. В  тот же  вечер Яшка  изъял дома  бутыль
уксусной эссенции, мы насыпали в эссенцию толченый мел, эссенция  зашипела
наподобие газировки, и на дно выпал серый порошок — уксуснокислый кальций.
Отфильтровали порошок, прокалили, получился  ацетон. На следующий день  мы
вручили Дине  большой  флакон  из-под  одеколона  «Красный  мак».  Доверху
заполненный ацетоном. Великая вещь — химия…

А вот  с катером  дела были  плохи. За  три недели  мы выстругали  корпус,
прекрасный корпус  длинной восемьдесят  сантиметров,  и притащили  его  из
мастерской наверх, к  себе. В  коридоре стоял большой  аквариум, мы  давно
решили приспособить его  для испытаний. На  воде корпус, даже  некрашеный,
выглядел совсем неплохо.  Но когда  мы начали укладывать  в него  батареи,
просто  так,  чтобы   посмотреть,  как  они   разместятся,  корпус   сразу
наклонился, вода полилась через борт,  и модель наша, разгоняя  испуганных
рыб, пошла ко  дну. Тут только  мы сообразили, что  нужно было  посчитать,
сколько может  выдержать  корпус  и  сколько  должны  весить  батареи.  Мы
вытащили из аквариума  корпус и батареи,  дали рыбкам корм,  чтобы они  не
очень переживали,  и  сели  за подсчеты.  Цифры  получились  убийственные:
батарей нужно в сорок раз больше, чем смог бы выдержать корпус катера. Это
был какой-то  кошмар. Мы  прикидывали  по разному  если корпус  сделать  в
полметра, в четверть метра  — все равно получалось,  что нужна целая  гора
батарей. А если батарей взять мало, вода будет разлагаться в час по чайной
ложке. Откуда же двигатель получит гремучий газ?!

Как просто  жилось до  этого! Появилась  какая-то идея  — и  прекрасно  ты
знаешь, что голова у тебя варит. Идей у меня всегда было много, поэтому  я
и не сомневался, что голова варит как надо. А тут выходило, что идея, даже
идея такого замечательного катера, сама по себе еще ничего не стоит.  Идея
может лопнуть, как мыльный пузырь, если не сойдутс расчеты.

Пока я это переживал, Яшке пришла  в голову спасительна мысль: кислород  и
водород можно получать без электричества, чисто химически.  Действительно,
серная кислота плюс цинк — получается водород, что может быть проще!

Раньше такая мысль привела бы меня в восторг: да здравствует химия, далеко
простирающая свои руки… Раньше было просто: что-то придумал и радуешься. А
теперь  придумал  —  и  сомневаешься,  боишься,  что  идея  лопнет.   Надо
посчитать,  но  ведь  расчеты  только  приблизительные,  тут  тоже  не  до
радостей. Вот  когда все  будет построено  и испытано,  тогда уж  наверное
можно радоваться.

Мы  стали  прикидывать,  сколько  надо  реактивов,  сколько  будут  весить
склянки, трубки и прочая начинка, сколько весит сам катер. Получалось, что
вес вдвое больше, чем нужно. Не в  сорок раз, а только вдвое, но  катер-то
все равно потонет…

По ночам мне снились  тонущие корабли. А потом  я сообразил, что  кислород
получать не надо, кислород есть  в воздухе. Значит половину начинки  можно
выкинуть, катер станет легче, не утонет! Очень даже логично.

Мы показали наши  расчеты Колбе. Бегло  посмотрев вычисления, он  исправил
две небольшие ошибки и буркнул:

— Раз уж вы начали мыслить, попробуйте мыслить дальше.

Между прочим,  это  оказалось  занятно  — мыслить.  Как  будто  играешь  в
шахматы. Сделал  ход, а  противник  тебе отвечает,  — и  надо  пересилить,
передумать противника. Вот только неизвестно,  кто твой противник, ты  его
не видишь. Яшка сказал, что противник  — наша собственна дурость. Но  если
это дурость,  почему  она  так  ловко  сопротивляется?  Да,  раньше  таких
вопросов не возникало.

Чтобы получить водород, нужны серная кислота и цинк, запас их должен  быть
на катере, а это лишний вес, тяжесть. Выгоднее взять карбид кальция, самый
обыкновенный карбид,  который используют  газосварщики. Карбид  плюс  вода
получается газ, ацетилен. А  он нисколько не хуже  водорода. Тут ведь  что
красиво: на борту  будет только  карбид, воду  брать не  надо, ее  сколько
угодно за бортом.

Мы снова пошли к Колбе и все  ему выложили. Он внимательно нас выслушал  и
стал удивленно разглядывать, как будто видел в первый раз.

— Вы мыслите,  следовательно, вы  существуете, —  торжественно сказал  он.
Если, конечно,  прав Декарт.  Вот стол.  Соберите установку  дл  получения
ацетилена. Нужно отрегулировать ее так, чтобы процесс шел равномерно.

Установку мы собрали за два вечера,  однако тут же пришлось ее  разобрать:
карбид издавал нестерпимый запах, в  комнате и в коридоре невозможно  было
дышать. Мы заново собрали установку в вытяжном шкафу. Николай Борисович не
отходил от нас ни на шаг и  придирался к каждой мелочи. Вроде бы все  было
правильно: вода капала на карбид, и выделившийс ацетилен шел по  резиновой
трубке к  горелке. Но  вот горелка  никак не  хотела работать  равномерно.
Огонь то еле-еле  мерцал, то  вдруг поднимался  огромным ревущим  столбом,
потом снова  затихал. Это  зависело  от тысяч  причин: какие  взяты  куски
карбида, как они уложены, как подается вода, как открыт кран горелки…

Другие ребята уже заканчивали макеты химических заводов, в морском  кружке
стояла почти готовая  модель крейсера  «Киров», а  мы все  еще возились  с
карбидом. Я читал теперь книгу Макса Валье «Полет в мировое пространство»,
книга была потрясающе интересной, но дома ее приходилось прятать от  мамы,
потому что на первой  странице был портрет Макса  Валье в траурной  рамке:
Валье погиб,  испытывая ракетный  автомобиль. Колба  принес мне  несколько
книг по ракетной  технике. В одной  из них было  сказано, что в  недалеком
будущем   ракетный   принцип   найдет    самое   широкое   применение    в
социалистической  технике.  Запомнилась  фамилия  автора:   инженер-летчик
С.П.Королев…

Я и не заметил, как прошли зимние каникулы. Каждый день с утра до позднего
вечера мы  возились  с карбидом.  Меняли  баки, трубки,  краны,  пробовали
растирать карбид  в порошок  и, наоборот,  прессовали из  карбида  плотные
кубики. Колба присматривал за нами, иногда помогал переделывать установку,
но ничего  не подсказывал.  «На ошибках  учатся, —  сказал он  однажды.  И
добавил: — Если это СВОИ ошибки».

Двумя этажами ниже, в большом зале стояла елка. Оттуда доносились  музыка,
голоса, смех. Однажды к нам забежала  Дина. На ней был костюм  Снегурочки,
наверное, ей хотелось  похвастаться. Она  неодобрительно понюхала  воздух,
фыркнула и сказала, что после каникул  ей снова понадобитьс ацетон. В  тот
день Яшка прожег  новый халат,  поэтому он только  покосилс на  Дину и  не
очень вежливо объяснил, что мы теперь  мыслим и у нас нет лишнего  времени
на всякие пустяки.

— Подумаешь, мыслители! — возмутилась Дина. — Кому он нужен, этот катер?

Наивный вопрос! Мы-то знали, что  наша модель побьет все рекорды  скорости
на воде, это во-первых. А во-вторых,  можно построить на этом же  принципе
большой корабль,  и  от  Баку  до  Астрахани  он  пройдет  за  час,  а  до
Красноводска и вовсе за какие-нибудь двадцать минут.

— Очень мне нужен ваш Красноводск, — без всякой логики сказала Дина.

Она сморщила нос и пренебрежительно фыркнула. Получилось эффектно, занятия
в драмкружке не пропали даром.

Ацетон мы ей сделали, на этот раз пришлось изъять запасы уксусной эссенции
у моей мамы. А установка наша  по-прежнему капризничала, тут у нас  ничего
не выходило. В  автомобильном кружке мы  достали манометр, которым  шоферы
измеряют давление воздуха в шинах. К этому манометру мы приделали рычаги и
пружины, чтобы он автоматически поворачивал  краны в зависимости от  того,
как меняется  давление ацетилена.  По идее  все было  просто.  Уменьшилось
давление в баке рычаги закроют кран на трубке, по которой уходит ацетилен,
и откроют другой кран, по которому в бак потечет вода. Повысится  давление
— и  пружины вернут  краны в  прежнее  положение. А  на самом  деле  краны
отчаянно упирались,  у  манометра не  хватало  сил управлять  ими.  Сергей
Андреевич принес нам  другой манометр —  огромный, тяжелый, от  пароходных
котлов.

— Ничего, конечно, не выйдет, но попробуйте. Дл практики, — сказал он.

Силы у этого  манометра было  хоть отбавляй, ему  нравилось вертеть  краны
просто так,  из  озорства. Все  время  приходилось регулировать  рычаги  и
пружины. Мы  прикинули, получилось,  что надо  увеличить длину  модели  до
полутора метров, иначе ничего туда  не втиснешь. Яшка высказал мысль,  что
манометр капризничает из-за  изменения атмосферного  давления. Мы  достали
барометр,  которым  измеряют  атмосферное  давление,  и  приспособили  его
регулировать  манометр.  Регулирующее   хозяйство  разрослось,  по   самым
скромным подсчетам выходило, что модель должна быть длиной в два метра. Но
барометр временами капризничал.  За день-два до  дождя он начинал  страшно
нервничать, на  него  нападала  какая-то  трясучка.  Идущие  от  барометра
рычажки и  пружинки лихорадочно  дергали большой  манометр, а  тот в  свою
очередь начинал вертеть краны, совершенно  не обращая внимания на то,  что
творилось в  баке с  карбидом. Нужен  был еще  какой-то прибор,  чтобы  он
правильно  управлял   барометром,   чтобы  барометр   правильно   управлял
манометром, чтобы манометр правильно управлял кранами…

И вот тут, когда мы  уже совсем отчаялись, появилась очередная  гениальная
мысль: надо  все это  выбросить и  посадить туда  человека, который  будет
вертеть краны. Придется снова увеличивать размеры катера, теперь он должен
имеет в длину метра три или четыре, в сущности, это будет уже не модель, а
НАСТОЯЩИЙ ракетный катер. Зато насколько это интереснее!

Сначала мы выложили эту  мысль Сергею Андреевичу.  Мы рассчитывали на  его
крепкие нервы. Все-таки целый корабль и живой человек внутри…

— Мы понимаем, что ничего  не выйдет, — вкрадчиво  сказал Яшка. — Но  ведь
попробовать можно, правда? Для практики.

Сергей Андреевич усмехнулся:

— Хитер ты, братец. Почему же не выйдет? Корпус обязательно выйдет. А  вот
двигатель… Ладно, не получится — можно будет поставить парус.  Попробуйте!
Для практики.

С  Николаем  Борисовичем  было  сложнее.  Он  долго  подсчитывал,  сколько
потребуется карбида, сколько ацетилена выделится  из этого карбида. И  что
получится,  если   все  это   взорвется.  Мы   доказывали,  что   поначалу
необязательно брать весь запас карбида, что испытания можно провести  тихо
и осторожно,  и что  вообще ничего  нет на  свете более  безобидного,  чем
ракетный двигатель.  В конце  концов, Колба  уступил, но  с этого  дня  он
контролировал каждый наш шаг.

Мы сделали окончательный чертеж. Обтекаемый четырехметровый корпус. Кресло
водителя. Рядом с креслом шесть блоков карбида — по три с каждой  стороны.
От баков идут трубки за борт.  Открыл кран, и вода поступает внутрь  бака,
начинается реакция. Газ из  всех баков собирается в  коллектор — это  тоже
бак, расположенный за креслом, а  оттуда, если открыть пусковой кран,  газ
идет в камеру сгорания. Остается включить зажигание, смесь газа и  воздуха
взорвется, катер рванется вперед, а потом еще взрыв и еще… Дальнейшее было
совершенно  ясно:  рекорд  скорости,  установленный  знаменитым   гонщиком
Кемпбеллом на «Синей птице», будет обязательно побит! Все логично!

Корпус мы  строили во  дворе Дома  пионеров. Первое  время это  никого  не
интересовало, мы работали спокойно. Корпус получалс странный, мы и сами не
ожидали, что он будет  таким. Передняя часть  походила на нос  скоростного
самолета, а корма была вполне морская — широкая, с большим килем.

Путем сложных  дипломатических переговоров  мы добыли  в планерном  кружке
кресло от планера и великолепную приборную доску от самолета.  Разумеется,
приборы не работали, но зато вид у них был отличный, а в темноте стрелки и
цифры потрясающе  светились. Особенно  мне нравилс  один прибор  — на  его
шкале была надпись «Угол атаки». Это прекрасно звучало — УГОЛ АТАКИ!

С автомобилистами,  которые  оказались практичнее  и  скупее  планеристов,
дипломатические переговоры успеха  не имели. Тут  мы вынуждены были  вести
натуральный обмен: за кислоту для аккумуляторов получили рулевое колесо, а
за растворитель для  краски выменяли  фару, свечи зажигани  от «Газика»  и
плексигласовый щиток от мотоцикла.  Щиток мы изогнули, получился  шикарный
верх для кабины — как на самолете.

Оставлять все это  на ночь во  дворе было рискованно,  и мы теперь  каждый
вечер тащили корпус на шестой  этаж. Лестница была мраморная, но  довольно
узкая  и   крутая,   а   на   лестничных   площадках   стояли   скульптуры
физкультурников и висели картины.  Мы старались не  повредить корпус и  не
всегда  успевали   проследить   насчет  остального.   Кто-то   пожаловался
директору, и однажды  директор появился во  дворе. От Дины  мы знали,  что
директор  по  совместительству  руководит  драмкружком.  Трудно   сказать,
насколько он был сердит на самом деле,  но вид у него был весьма  гневный:
примерно как у Ивана Грозного в исполнении артиста Юрия Яковлева. Директор
дважды обошел вокруг катера и как-то поостыл. Вообще, я заметил, что  люди
далекие от техники, невольно  притихают вблизи машин. По-видимому,  машины
представляютс им чем-то  вроде дрессированных  тигров: дрессировка  должна
действовать, и дрессировщик рядом,  но все-таки тигр есть  тигр — мало  ли
что может быть…

Помолчав, директор указал пальцем на носовую часть катера и произнес:

— Здесь!

Потом покачал головой и перешел назад, к корме.

— Нет, — сказал он. — Здесь! Именно здесь должен быть спасательный круг.

Он посмотрел на нас и уточнил:

— Два спасательных круга.

На том он  и удалился.  Мы с облегчением  вздохнули, не  оценив тогда  всю
мудрость и  дальновидность руководства.  Нас в  тот день  волновал  другой
вопрос как  назвать катер.  Собственно, название  уже было,  его  придумал
Сергей Андреевич, но мы как-то еще не освоились с этим названием.

— «Черная молния»… ну-ну… «Огненный метеор»… да тут у вас в таком духе,  —
сказал он, просмотрев  составленный нами список  названий. — Если  корабль
назвать «Молнией», сто узлов покажутся чем-то очень скромным. Другое  дело
— «Черепаха». Дл «Черепахи» и  десять узлов — уже колоссально.  Психологию
надо учитывать.

«Черепаха»… Нам это  не нравилось,  да и другие  ребята нас  отговаривали.
Один только  Витька  поддержал  Серге Анатольевича:  так  мол,  лучше  для
СЕКРЕТНОСТИ.

— Вы  про майора  Пронина читали?  —  спросил он.  — Должно  быть  КОДОВОЕ
название, без этого нельзя.

Майор Пронин в те годы был  почти столь же популярен, как сейчас  Штирлиц.
Мы уступили: пусть будет кодовое название.

«Черепаху» покрасили в ярко-красный цвет, чтобы легче было следить за ней,
когда она  будет нестись  с  огромной скоростью.  Яшка принес  книжку  про
Буратино, там был  рисунок Тортилы, и  знакомый парень из  художественного
кружка, поглядывая на Тортилу, нарисовал двух черепах — по одной на каждом
борту. Рисунки получились хорошие, но все-таки в них чего-то  недоставало.
И однажды мы сообразили: черепахам нужно приделать крылья, вот чего им  не
хватает! Крылья пришлось дорисовывать самим, но уж после этого все было  в
порядке, и «Черепаха» приобрела прямо-таки шикарный вид.

Теперь  во  двор  то  и   дело  заглядывали  любопытные.  Однажды   пришел
застенчивый парень в очках; он  долго ходил вокруг «Черепахи»,  заглядывая
внутрь, что-то подсчитывал в записной книжке, а потом, извиняясь и краснея
от  смущения,  объявил,  что  эта  подводная  лодка  опустится  только  на
пятьдесят метров, после чего ее раздавит ужасное давление воды. Через пару
дней появилась мрачная личность в промасленной пожарной куртке и столь  же
промасленной кепке,  надвинутой на  лоб.  Личность угрюмо  понаблюдала  за
нашей работой и спросила: «Пожар будет, огонь будет, куда прыгать  будешь?
Когда сгоришь, что скажет папа, что скажет мама?..»

Несколько раз приходили ребята из  фотокружка со своим руководителем.  Они
осваивали цветную фотографию, их  привлекала ярка окраска «Черепахи».  Нам
объявили, что мы нефотогеничны и  совершенно не смотримся на фоне  катера.
Хорошо смотрелась Дина, они ее старательно снимали.

Приходил и начинающий поэт из литкружка. Мы показали ему, где будут стоять
баки с карбидом и как газ будет идти в камеру сгорания, объяснили  принцип
действия ракетного двигателя. К сожалению, поэт все перепутал. Он  написал
стихи, в которых баки почему-то бешено вращались, а камера сгорани дула  в
паруса и корабль гордо шел сквозь льды к северному полюсу…

В общем, это нам надоело и мы стали закрывать «Черепаху» брезентом. Витька
сказал, что ИЗДЕЛИЕ, упрятанное в брезент, обязательно привлечет  внимание
шпионов. Он  теперь постоянно  вертелся  около «Черепахи»,  поглядывая  по
сторонам.

Мы работали  в механической  мастерской, делали  части для  двигателя.  На
бумаге очень  легко нарисовать  шесть баков,  а вот  сделать из  листового
железа хотя бы один бак намного  труднее, в этом мы скоро убедились.  Баки
получились  какие-то  скособоченные,  и  Николай  Борисович,  посмеиваясь,
говорил, что это отличные модели  знаменитой падающей башни в Пизе…  Потом
началась бесконечная морока со швами. Швы держались на заклепках и  пайке,
вроде бы  все было  прочно  и герметично,  однако баки  упорно  протекали.
Вместо  камеры  сгорания  удалось  приспособить  цилиндр  от  авиационного
двигателя, но в крышке цилиндра  пришлось устраивать клапаны для  воздуха,
тут мы провозились больше месяца…

Корпус стоял в  углу двора,  под брезентом. Иногда  я развязывал  веревки,
снимал брезент и залезал в  кабину «Черепахи». Я пытался представить,  как
все произойдет: вот  я открою  краны, вода  пойдет в  баки, потом  поверну
пусковой кран,  газ  устремится  в камеру  сгорания,  рывок  рубильника  —
«Черепаха» помчится  вперед… Жаль,  что  на нашей  великолепной  приборной
доске не работали приборы  — указатель скорости,  указатель угла атаки…  Я
уже знал, что углом атаки называют угол наклона крыла к линии полета. Было
что-то магическое в  этих словах —  УГОЛ АТАКИ. Словно  прибор должен  был
показывать напряжение  сил,  отданных  решающему порыву.  Ах,  если  бы  у
каждого человека  была возможность  видеть  свой угол  атаки,  чувствовать
накал сил в этот день, в этот час, в этот миг!..

Мне отчаянно не хватало времени: я наспех ел, наспех спал, наспех  готовил
уроки. Однажды  я  подсчитал,  сколько  времени  у  меня  в  запасе.  Жить
предстояло еще лет  пятнадцать (после тридцати  лет маячила старость,  эти
годы я не  учитывал). Целых пять  лет — треть  жизни, страшно подумать!  —
должны были уйти на  сон. Если в  день терять по  пустякам три часа,  надо
вычесть еще год-полтора. Что же тогда остается?!

Расчеты произвели  на меня  ошеломляющее впечатление.  Я стал  чувствовать
движение времени. Словно стоишь в быстрой горной реке, и вода стремительно
обтекает тебя: мгновение — и она уже далеко, ее не догнать, не удержать.

Я посмотрел по часам, сколько времени нужно, чтобы прочитать книгу.  Потом
произвел подсчеты. Простая арифметика показала,  что за всю жизнь я  успею
прочитать не больше тысячи  книг. А что такое  тысяча книг? Ведь только  у
Жюля Верна свыше ста книг!

Тогда я стал  записывать, на  что уходит время,  пытался учитывать  каждую
минуту — с утра  до ночи. Оказалось, что  теряетс масса времени: вроде  бы
непрерывно чем-то занят, но все равно за день куда-то исчезают  три-четыре
часа… Прошли годы, пока я научился  управлять временем, но это уже  другая
история, об этом надо писать другой рассказ. А в те дни у меня была только
одна идея: нужно поменьше  спать. Я попробовал меньше  спать. За три  ночи
удалось выиграть два часа, зато в четвертую ночь я заснул так крепко,  что
начисто потерялись сэкономленные часы… А впереди было лето, меня тревожило
множество нерешенных  вопросов. Куда  например,  отнести время,  когда  ты
лежишь на крыше городской купальни  и загораешь? Потерянное это время  или
нет?..

В  довоенные  годы  лето  в   Баку  начиналось  пятнадцатого  апреля.   До
пятнадцатого апреля была зима, а потом сразу начиналось лето.  Собственно,
сейчас в Баку  тоже только два  времени года: лето,  когда очень жарко,  и
зима, когда жарко  не очень.  Но до войны  смена времен  года была  четче:
пятнадцатого апреля — и ни днем  раньше! — на улицах появлялись мужчины  в
белых брюках  и  женщины в  белых  юбках, открывались  летние  кинотеатры,
пляжи, парки.

Пятнадцатого апреля мы с Яшкой устроили выходной. Мы ходили по улицам, это
было странно: идешь и  не спешишь, можешь даже  постоять. О «Черепахе»  мы
решили в  этот день  не вспоминать.  В городе  появились новые  трамвайные
вагоны с мягкими креслами. Мы покатались на трамвае, посмотрели аттракцион
«Петля смерти  — мотоцикл  на вертикальной  стене», постояли  около  новой
легковой автомашины «ЗИС». Раньше это было безумно интересно, а сейчас  мы
знали, что  трамваи, мотоциклы,  автомобили —  прошлый век.  Ну что  такое
трамвай, пусть  даже  с мягкими  креслами,  если скорость  всего  двадцать
километров в час!.. Да и «петля  смерти» нисколько не смертельна, об  этом
подробно написано в «Занимательной физике» Перельмана.

На приморском бульваре шли гонки шверботов, смотрел и впервые  чувствовал,
как медленно, как медленно, как невыносимо медленно ползут они под  своими
парусами. Ах, если бы здесь была «Черепаха»…

Потом мы  поднялись в  Нагорный парк.  Там была  выставка собак,  и  Яшка,
нарушив уговор, начал  доказывать, что первое  испытание можно провести  с
собакой: так, мол, будет безопаснее. Мы едва не поссорились. Я считал, что
испытывать катер  должен человек:  в случае  чего человек  сообразит,  что
предпринять. Забегая  вперед,  надо сказать,  что  недели через  две  Яшка
все-таки притащил собаку с грозной  кличкой Джульбарс. Это было веселое  и
добродушное существо — кривоногое как  такса и пушистое как болонка.  Яшка
посадил Джульбарса в кресло, пес уставился  на приборную доску, и тут  его
веселье сразу испарилось.  Он внимательно оглядел  приборы, затем  обнюхал
стоявшие рядом с  креслом баки. Похоже,  он быстро сообразил  что к  чему.
Нервно тявкнув, Джульбарс выпрыгнул из катера и стрелой умчался в  дальний
угол двора, в подвал, откуда Яшка долго потом выманивал его ирисками.

Итак, пришло  лето, а  мы все  еще возились  с «Черепахой».  Казалось  бы,
простое дело  —  установить  баки.  Так вот,  баки  установлены,  и  вдруг
обнаруживается, что забортная вода заполнит  их всего на одну треть:  баки
высокие, они выше уровня воды, а  вода сама собой наверх не пойдет.  Ужас:
надо делать новые баки  — низкие и широкие,  их негде будет разместить  и,
следовательно, корпус тоже надо делать заново…

Выручила  нас  «Занимательная  физика»  Перельмана.  Там  описывалось  как
набирают воду скоростные поезда: вдоль рельсов устраивают канал с водой, с
паровоза опускают трубку, загнутую вперед; чем быстрее движется поезд, тем
выше поднимается по трубке вода. Мы тоже вывели изогнутые трубки за  борт,
навстречу потоку  воды. Вообще,  «Занимательная  физика» очень  часто  нам
помогала. И  вот однажды  наступил день,  когда все  оказалось  сделанным,
установленным, привинченным, покрашенным. Мы даже растерялись: все готово,
стоишь  и  не  знаешь,  что  теперь  делать.  Сергей  Андреевич,   оглядев
«Черепаху», сказал:  «Ничего не  получится»,  — и  распорядился  покрасить
дюзу. Огромная  дюза,  похожа  на раструб  старого  граммофона,  почти  на
полметра выступала из кормы. Дюзу мы покрасили черным лаком, хотя лак  все
равно  должен  был  сгореть,  когда  начнет  работать  двигатель  и   дюза
раскалится.

— Будет  работать  двигатель или  нет,  еще неизвестно,  —  сказал  Сергей
Андреевич. — А все должно  блестеть. Завтра проведем испытание. Есть  один
тихий пруд в парке.

На следующий день «Черепаху» погрузили на трехтонку, обвязали веревками  и
повезли в парк. Прохожие останавливались и смотрели на диковинную  машину.
Мы стояли в кузове, наслаждаясь всеобщим вниманием. Жаль, что Дина  уехала
на лето и не видала нашего великолепного марша по городу.

За неделю до отъезда Дина  подарила мне «Таинственный остров» Жюля  Верна.
Издание было  шикарное  — в  тисненом  переплете, с  отличными  рисунками.
Раньше Дина не давала эту книгу даже  на время. Я спросил, не нужен ли  ей
ацетон или еще что-нибудь. Она ответила, что ничего не нужно. В тот  вечер
мы  немного  погуляли  по  приморскому  бульвару.  Я  пыталс  говорить   о
«Черепахе»,  но  Дина  слушала   как-то  рассеянно.  Вообще,  она   сильно
изменилась за последние месяцы. Теперь Яшка  говорил, что она — из  лирики
А. С.  Пушкина. Иногда  она  казалась мне  совсем незнакомой.  Словно  она
немного старше меня. Или я немного старше ее (все-таки в ракетной  технике
она ничего не понимала!). Мы уже  уходили с бульвара, когда она  захотела,
чтобы я  сорвал ей  цветок,  и показала  на  олеандровый куст  в  середине
большой клумбы. Этих  олеандровых кустов  было везде полным-полно  — и  на
бульваре и на  улицах. Но  Дине вопреки всякой  логике понадобился  цветок
именно с той  клумбы. Цветок  я сорвал, а  потом нам  пришлось удирать  от
сторожа. Нет, все-таки я был немного старше Дины…

Так  вот,  мы  привезли  «Черепаху»  к  пруду,  сняли  ее  с  грузовика  и
торжественно опустили  на  воду.  Она не  утонула,  наша  «Черепаха»!  Для
корабля,  построенного   по   «Занимательной  физике»   и   «Занимательной
механике»,  не  утонуть  —  это  уже  очень  много.  «Черепаха»  прекрасно
выглядела на  воде, это  признали  все —  и  Сергей Андреевич,  и  Николай
Борисович, и ребята, которые  приехали с нами.  Первое испытание мы  давно
решили провести вдвоем, чтобы не было обидно ни Яшке, ни мне. Мы с  трудом
втиснулись в тесную  кабину. Я сидел  у Яшки на  коленях. Карбида  Николай
Борисович разрешил взять совсем немного, мы  засыпали его в один бак.  Тем
временем ребята под руководством Сергея Андреевича привязали «Черепаху»  к
дереву, стоявшему  у самой  воды. Инструкци  у нас  была четкая:  включить
мотор и, как только он начнет работать, сразу выключить.

— Быстрее, — торопил Яшка, — все уже спрятались, давай воду, ты мне колени
отсидел.

Я открыл кран,  и в  баке начал шипеть  карбид. Мы  смотрели на  приборную
доску, на  единственный работающий  прибор —  манометр, который  показывал
давление ацетилена в баке. Стрелка медленно, словно нехотя, ползла к цифре
«2». Яшка ерзал подо мной, он  никак не мог дотянутьс до пускового  крана.
Наконец, он открыл кран,  газ пошел в  камеру сгорания, стрелка  манометра
сразу замерла. Я включил рубильник системы зажигани и…

…и ничего не произошло.

Я слышал  как  плещется  вода  за  бортом,  как  шипит  карбид  в  баке  и
потрескивает что-то в блоке зажигания.

— Дела… — растерянно произнес Яшка.

Больше он ничего не  успел сказать, потому  что возник быстро  нарастающий
скрежет. За две-три секунды скрежет окреп,  он стал нестерпимым — и  ахнул
взрыв.

Я почувствовал,  как затрясся  корпус «Черепахи»,  и мгновенно  представил
себе  тонкие  ребра-шпангоуты  и  еще  более  тонкие  планки,  соединяющие
шпангоуты, и совсем тонкую фанерную обшивку. Я представил себе, на чем все
это держится, ведь мы  экономили каждый гвоздь, чтобы  выиграть в весе…  Я
схватился за рубильник и дернул его вниз, отключая зажигание.

Мы вылезли из кабины, еще не понимая, что, собственно, происходит и  можно
ли считать испытания удачными.  Но ребята на берегу  орали от восторга,  и
наши сомнения тут же рассеялись.  Все наперебой рассказывали, как  здорово
бабахнул двигатель  и  какое  огромное пламя  вырвалось  из  дюзы.  Сергей
Андреевич сказал:  «Шуму  много… уже  хорошо».  А Колба  расспрашивал  про
давление ацетилена в  баке: сколько  было до пуска,  сколько после  пуска.
Словом, испытание прошло  успешно, мы  все больше и  больше укреплялись  в
этой мысли. Один взрыв — и так  здорово! А ведь можно набить карбидом  все
баки, можно поднять давление газа в баках до пяти атмосфер — тогда  взрывы
сольются в сплошной рев, и «Черепаха» стремительно полетит вперед…

Сергей Андреевич сказал, что завтра  привезут отборочную комиссию из  Дома
пионеров и, если  все пройдет  хорошо, «Черепаха»  получит направление  на
выставку. А пока надо ставить палатку и дежурить до утра. Сергей Андреевич
забрал ребят и уехал. Мы остались вчетвером: Колба, Яшка, Витька и я.

Ужинать  мы  пошли   в  кафе,  поручив   Витьке  караулить  «Черепаху»   и
высматривать шпионов.  После  ужина  Колба  лег спать  в  палатке,  а  мы,
накормив Витьку  пирожками, отправились  бродить  по парку.  У  парашютной
вышки  и  тиров  толпился  народ.  На  танцплощадке  гремела  музыка.   Мы
понаблюдали из-за ограды,  как танцуют.  Яшка сказал,  что вообще-то  надо
освоить западные  танцы. Я  решительно возражал:  кому нужны,  эти  танцы?
Взрослые люди  топчутся  на  пыльной  площадке,  выламываются  друг  перед
другом, смешно! А сколько времени на это уходит, страшно подумать…

Мы  прошли  на  эстраду,  там  было  интереснее,  выступали  иллюзионисты,
жонглеры. Возвращались напрямик, по безлюдной и плохо освещенной аллее.  В
самом ее  конце стоял  фанерный  щит с  картой  Испании. Война  в  Испании
кончилась еще в прошлом году, о карте, наверное, просто забыли. Краски  на
ней выгорели, местами их смыл дождь. Но все-таки было видно, как изогнутые
стрелки нацелились на Мадрид и Барселону. Мы долго стояли у карты. Если бы
мы знали, что через год эти хищные стрелки потянутся к Минску,  Смоленску,
Москве…

Ветер раскачивал  одинокий фонарь  на столбе,  по карте  пробегали  черные
тени.

— Говорят, будет война, — тихо сказал Яшка. — Как думаешь, меня не заберут
из-за очков?

Я думал, что  Яшку скорее всего  в армию не  возьмут: он —  очкарик, да  и
вообще, как  говорила  моя  мама,  книжный мальчик.  Но  обижать  Яшку  не
хотелось.

— Ты можешь  стать военным врачом,  — ответил  я, — Чем  плохо? На  каждой
подводной лодке есть врач.

Насчет себя я не сомневаюсь: после школы пойду в военно-морское училище, а
потом буду плавать на  подводных лодках. В тот  вечеря, конечно, не  знал,
что все сложится иначе и что в сорок третьем я попаду в авиационную школу,
а Яшка окажется  в артиллеристском  училище. Не  знал я,  что из  книжного
мальчика получится командир батареи  противотанковых орудий. Яшка погиб  в
сорок пятом в Венгрии, у озера Балатон.

К «Черепахе» мы возвращались  молча. Я думал об  Испании: должна там  быть
революция, обязательно должна  быть! Построить бы  к тому времени  большой
ракетный корабль…

Вернувшись, мы обнаружили, что Витька спит и Николай Борисович тоже  спит,
а «Черепаха» спокойно покачивается у берега. В двенадцать в парке потушили
огни. Яшка сказал, что подежурит, и что должен спать, так будет  правильно
с медицинской точки зрения. Утром мне предстояло вести «Черепаху». Яшка не
умел плавать, а у Серге Андреевича  на этот счет было твердое правило:  не
умеешь плавать — сиди на берегу.

Где-то очень далеко жужжал самолет  и по небу пробегали лучи  прожекторов.
Мы  молча  разглядывали  звезды.  Меня  беспокоили  каверзные  мысли.  Они
возникли однажды, и я никак не мог от них отвязаться. Почему на  рекордных
катерах Кемпбелла стоят не реактивные, а самые обычные двигатели?  Неужели
Кемпбелл глупее нас?  И вообще: почему  великолепную «Черепаху»  построили
мы, а не взрослые?

Через много лет, перечитывая биографию великого математика Эвариста Галуа,
умершего совсем молодым, я вспомнил свои старые сомнения и подумал, что по
настоящему дерзкие  открытия  всегда  начинаютс  с  мальчишеских  снов,  с
детской  мечты,   с  безрассудных   и  необычных   идей.  Потом   приходят
рассудительность, осторожность, трезвый расчет: а вдруг не выйдет? а может
быть это вообще  неосуществимо? а почему  это ты такой  умник, что  видишь
дальше других?.. Хорошо, если где-то в глубине души сохранились те  давние
мальчишеские сны и безрассудные идеи.  Как тлеющий огонь разгорятс  вновь,
вспыхнут ярким пламенем и осветят всю жизнь. И тогда ты пойдешь  наперекор
всему, пойдешь и совершишь невозможное…

Яшка быстро заснул. И я еще немного подумал о «Черепахе», о завтрашнем дне
и о том, что Дина уехала совсем не вовремя.

Утром на двух автобусах приехала комиссия в сопровождении ребят из  разных
кружков. Чудесное было утро, свежее, без надоевшей жары, и члены  комиссии
восторженно  ахали,  разглядывая  деревья,  траву,  пруд  и   ярко-красную
«Черепаху». Витька отвел меня в сторону и мрачно зашептал:

— Слышь, Генка, плохо ваше дело.  В комиссии этой насчет техники никто  ни
бум-бум. Там  ведь кто?  Руководительница  вышивального кружка,  потом  из
балетного, из  литкружка, из  биологического,  из хорового,  а та,  что  с
девчонками стоит, — художественная  гимнастика. Один мужик,  так и тот  из
рисовального. Им без  разницы — ракета  или весла. Ничего  не поймут,  вот
увидишь.

Я предстал перед  комиссией в  новеньком комбинезоне  и авиационном  шлеме
(Сергей Андреевич раздобыл  у планеристов). Комисси  с умилением  оглядела
меня и благожелательно задала ряд вопросов: сколько мне лет, в какой школе
учусь и какие у меня  отметки. Председательница воскликнула: «Очень  мило,
нет, правда, здесь  все очень  мило…» Яшка  только-только кончил  заряжать
баки, от него попахивало карбидом, и Сергей Андреевич велел ему  держаться
подальше от комиссии.

Ребята подтянули «Черепаху» к берегу, можно было приступать к  испытаниям.
Колба шепнул: «Давление не больше двух атмосфер, смотри!» Сергей Андреевич
отозвал меня в сторону и предупредил: «Поосторожнее там… Давление  держать
не больше трех атмосфер.» Я залез в  кабину. Яшка, стоя по колено в  воде,
придерживал «Черепаху».

— Вот вата, заткни уши, — сказал он. — И не вмажь в тот берег.

Он закрыл плексигласовый верх кабины и показал растопыренную пятерню.  Это
означало: давление ацетилена не меньше пяти атмосфер.

Дальше  я  действовал  почти  автоматически.  Быстро  открыл  все   краны,
послышалось шипение, стрелка манометра резво побежала вправо. Я посмотрел,
куда направлен нос «Черепахи»; все  было в порядке, Яшка успел  развернуть
катер кормой  к берегу.  Стрелка проскочила  цифру «3».  Теперь надо  было
смотреть в оба глаза за этой стрелкой, потому, что баки могли не выдержать
такого давления; я хорошо знал, какие они ненадежные, наши баки.

Когда  стрелка  коснулась  пятерки,  я   рванул  пусковой  кран  и   почти
одновременно включил зажигание. Мгновенно раздался высокий звенящий  звук,
словно рядом разбили стекла, — и  тут же громыхнул взрыв. Он отличался  от
вчерашнего взрыва, как  трубный рев дюжины  взрослых слонов отличается  от
крика одного маленького слоненка.

Я закрыл глаза и съежился, вцепившись в руль. В тот момент мне и в  голову
не пришло оглянуться на берег. А  стоило бы оглянуться! «Черепаха» была  в
метрах полутора  от  берега,  и дюза  оказалась  нацеленной  на  комиссию,
которой Николай Борисович тщетно растолковывал принцип ракетного движения.
Сергей Андреевич,  человек  опытный  и не  раз  видевший,  как  взрываются
корабли, показывал Колбе знаками, что, мол, надо отойти подальше, но Колба
не замечал его сигналов. Именно в этот момент я и включил зажигание.

Потом я долго расспрашивал  очевидцев. Говорили разное. Легенды  возникают
поразительно быстро, а в них  все страшно преувеличено. Так вот,  согласно
легенде, комиссию словно ветром сдуло. Никто из членов комиссии не  устоял
на ногах. Подобная участь  постигла зрителей, которые пробрались  поближе.
Остальные, как принято говорить, отделались легким испугом.

Оглушительный старт  «Черепахи» продемонстрировал  комиссии разницу  между
веслами и  ракетой и,  как  позже сказал  Яшка, ссылаясь  на  Маяковского,
сделал это «весомо, зримо, грубо».

А я, открыв глаза,  увидел, что «Черепаха»  двигалась, но действительно  с
черепашьей скоростью. Каждый взрыв (а они раздавались три-четыре  секунды)
заставлял меня  закрывать  глаза.  Я  чувствовал,  как  выгибается  корпус
«Черепахи», и ждал,  что камера сгорани  вот-вот сорвется.  Оглушительные,
пушечной силы,  взрывы  — это  победа:  мотор  работал и  еще  как  громко
работал! Но  «Черепаха»,  содрогаясь  и  сотрясаясь  от  взрывов,  шла  со
скоростью пешехода,  не  больше.  И  это  было  страшное  поражение…  Меня
раздирали противоречивые чувства,  а подумать,  собраться с  мыслями я  не
мог: попробуй мыслить, если за спиной раздаетс такая канонада!

Логично мыслить  я начал,  когда «Черепаха»  подошла к  середине пруда.  Я
разглядел стрелку  манометра, увидел,  что она  дрожит около  цифры «4»  и
подумал: раз «Черепаха»  не развалилась —  еще не развалилась!  — и  мотор
держится, надо поднять давление ацетилена до семи или восьми атмосфер.

Я выключил  зажигание, перекрыл  пусковой кран.  Наступила тишина,  просто
невероятная тишина, я никогда  не слышал такой тишины.  Не знаю, шипел  ли
карбид в баках, но стрелка  двигалась вправо, давление увеличивалось, и  я
стал вертеть руль. «Черепаха» медленно разворачивалась.

И вот произошла катастрофа.

Послышался странный звук — словно с  треском разорвалс кусок материи, —  и
кабина сразу заполнилась едким запахом карбида. Лопнул бак, первый  слева,
он разошелся  по  шву,  стальная стенка  как  бритвой  пропорола  фанерную
обшивку. Внутрь катера хлынула вода. Я посмотрел на стрелку, она дрожала у
цифры «7»: включить бы сейчас двигатель!… Но «Черепаха» кренилась на борт,
в кабине было невозможно дышать. Я отбросил плексигласовый верх и полез за
борт…

Все-таки «Черепаха» была хорошим кораблем: тонула она не спеша, солидно. Я
плавал вокруг нее, пока  она не скрылась под  водой. Потом я направился  к
берегу. Очень не хотелось туда плыть, но что оставалось делать…

То, что  происходило  в  следующие  полчаса,  трудно  поддается  описанию.
Разъяренные члены комиссии тянули меня в разные стороны, пересчитывали мои
руки и ноги, что-то спрашивали, а  я ничего не слышал и отвечал  невпопад.
На  Сергея  Андреевича  и  Николая  Борисовича  обрушилс  град  упреков  и
обвинений. Комиссию  нисколько не  беспокоило,  что «Черепаха»  не  побила
мировой рекорд. Похоже комиссия ничего не заметила, кроме первого  взрыва.
Председательница возмущенно повторяла: «Кто  разрешил посадить ребенка  на
эту бочку с порохом?»

Яшка успел сбежать, а нас — Сергея  Андреевича, Колбу и меня — посадили  в
автобус и повезли в город. Через  час мы давали объяснения директору  дома
пионеров. Собственно,  за  всех нас  объяснялс  Сергей Андреевич.  Мне  не
удалось сказать ни слова, хотя  еще в автобусе, слегка изменив  знаменитое
изречение  Галилея,  я   приготовил  отличную  фразу:   «А  все-таки   она
двигалась!»

Объяснения шли  на  высоких тонах.  У  директора был  внушительный  голос,
директор натренировался в драмкружке. А  Сергей Андреевич мог, если  надо,
перекричать рев шторма. Мы сидели в коридоре, ожидая решения. И начальство
проявило  мудрость,   которая  поразила   меня  своей   неожиданностью   и
непостижимостью. Был издан приказ в двух пунктах:

Конструкторов  «Черепахи»,   руководител   и  актив   химического   кружка
премировать туристскими путевками.

Лицам, указанным в пункте первом, отбыть в турпоход согласно путевкам.

Позже  я  понял,  насколько  мудро   и  логично  было  это  решение.   Нас
премировали, следовательно,  нам не  обидно.  Нас отправляли  в  турпоход,
следовательно,  дальнейшие  эксперименты  пресекались,  комиссии  тоже  не
обидно. К тому  же решалась проблема,  вторую неделю терзавшая  директора:
горели турпутевки, предусматривающие пеший  поход вдоль Большого  Кавказа,
от этих  путевок  наотрез  отказывались руководительницы  хоровых  и  тому
подобных кружков. Путевки  и билеты нам  вручили тут же.  Мы посмотрели  и
ахнули — поезд уходил завтра утром. Надо было бежать домой, объяснять  все
родителям и собирать вещи…

* * *

В походе у нас было достаточно времени, чтобы подумать, что же  собственно
произошло. Сначала мы  на все  лады обсуждали, как  поднять «Черепаху».  И
только  потом  поняли,  что  нужен  новый  двигатель.  В  камеру  сгорания
«Черепахи» топливо поступало в газообразном виде, газ весит мало, а объема
занимает много. Поэтому  каждый раз в  камере оказывалось всего  несколько
граммов горючей  смеси. Мы  не  сообразили, что  горючее надо  подавать  в
жидком виде — в этом была наша ошибка.

По вечерам мы  сидели у  костра и  спорили, каким  должен быть  жидкостной
ракетный двигатель, какое горючее лучше и какую жидкость взять окислителем
вместо воздуха.

Жидкостной ракетный двигатель  я собирал  два года.  Камера сгорания  была
маленькая, меньше  чайного стакана:  весь двигатель  спокойно умещался  на
подоконнике. Я собирал двигатель дома, иногда мне помогал Яшка.  Запустить
двигатель я  не решался;  знал,  что в  самом  лучшем случае  он  выдержит
несколько секунд, не больше.

Много мороки было с окислителем — перекисью водорода. В аптеках  продавали
трехпроцентный раствор перекиси,  для двигателя этот  раствор не  годился.
Дина достала в парикмахерской пергидроль — 30-процентный раствор перекиси,
но и этого  было мало.  Впрочем, пергидроль  пригодилась для  дыхательного
прибора. Она  легко разлагалась,  выделяя кислород.  И построил  подводный
дыхательный прибор, работающий  на перекиси. Я  не придавал этому  прибору
особого значения,  но  именно на  него  я получил  свое  первое  авторское
свидетельство. Было это в десятом классе…

В феврале сорок третьего  я ушел в  армию. За два  дня до отъезда  удалось
раздобыть немного  концентрированной перекиси,  решил показать  Дине,  как
работает  жидкостная  ракетная   установка.  Вечером  объявили   воздушную
тревогу, соседи опустились в  убежище, а мы с  Диной вытащили установку  в
опустевший коридор. Двигатель был  надежно прикреплен к стенду,  массивной
металлической доске. Мне  вспомнилась кабина  «Черепахи»; на  этот раз  не
было никаких приборов —  ни манометра, ни  указателя угла атаки.  Пусковой
кран я дернул  шпагатом издали.  Перекись, попав в  камеру сгорания,  сама
воспламенила горючее,  спирт.  И  в  унылый  коридор  старой  коммунальной
квартиры вдруг  ворвался голос  будущего, голос  ракетной эры  —  слитный,
мощный, яростный  рев  ракетного  двигателя.  За  две-три  секунды  камера
раскалилась докрасна. Рев резко  оборвался, над стендом взметнулось  синее
пламя, горел  спирт.  Мы  с  трудом  сбили  огонь,  от  установки  остался
обгоревший, оплавленный металл…

И все-таки она сработала, эта  установка! Наверное, это самое главное.
Мы ищем,  строим, сомневаемся, радуемся,  надеемся и, в  конце концов,
что-то  удается, что-то  получается, и  мы можем  сказать: а  все-таки
вертится, смотрите, все-таки вертится, движется, работает!..

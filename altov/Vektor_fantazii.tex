щ
Вектор фантазии

Генрих Альтов


Еще не так давно творческие способности исследователя рассматривались  как
нечто очень  неопределенное и  практически не  поддающееся изучению.  Если
время от  времени человека  «озаряет»  — значит,  у него  есть  творческие
способности, «не  озаряет»  —  способностей  нет…  Ныне  установлено,  что
творческие способности  представляют собой  сплав многих  качеств. И  хотя
наука еще  не  знает точных  формул  этих качеств,  можно  с  уверенностью
сказать: все творческие качества включают фантазию.

Подобно тому как углерод входит  во все органические соединения,  фантазия
составляет  непременный  и  очень  важный  элемент  всех  без   исключения
творческих качеств.  Развивать  творческие  качества  —  значит  развивать
воображение, фантазию.

Но  удивительный  парадокс:  признание  величайшей  ценности  фантазии  не
сопровождается планомерными усилиями по ее развитию.

В школах  нет  уроков  воображения; студенты  не  изучают  курса  развития
творческого воображения;  аспиранты  не  готовятся к  сдаче  экзаменов  по
воображению. В рассказе  Рэймонда Джоунса «Уровень  шума» психолог Бэрк  с
горечью говорит: «Мы постепенно взрослеем, и по мере того как мы учимся  в
школе  и   получаем  образование,   в  ваших   фильтрах  шума   появляются
ограничительные уровни, которые пропускают лишь ничтожную часть  сведений,
приходящих из внешнего мира и из нашего воображения.

Факты окружающего мира отвергаются, если  они не подходят к  установленным
уровням.  Творческое   воображение  суживается…»[1]   Пока   единственным
массовым и практически  действенным средством  развития фантазии  остается
чтение научно-фантастической литературы (НФЛ). Разумеется, характерное для
современности увлечение  НФЛ вызвано  многими причинами.  Однако  анкетный
опрос выявил четкую  закономерность: людей  творческого труда  значительно
сильнее тянет к НФЛ, чем других читателей.

Особенно сильна тяга к  НФЛ у инженеров и  ученых. 52 процента  опрошенных
инженеров и  физиков  отметили,  что  ценят  НФЛ  прежде  всего  за  новые
научно-фантастические идеи. Действительно, в этом отношении НФЛ может дать
очень многое думающему инженеру.

Вплоть до темы, за разработку которой можно взяться, или даже до  готового
решения, которое  остается лишь  перевести  на инженерный  язык.  Недавно,
например, в  ФРГ выдан  патент №  1229969 с  такой формулировкой  предмета
изобретения:  «Способ   добычи   полезного  ископаемого   из   космических
месторождений, отличающийся  тем, что  в качестве  месторождения  выбирают
астероид с  небольшой  собственной массой  и  такой орбитой,  при  которой
возможны затраты на осуществление импульса для транспортирования астероида
на Землю». Человек, хорошо знающий фантастику, сразу отметит, что в  числе
авторов этого  изобретения  следовало  бы  указать  Жюля  Верна  («Золотой
метеор») и Александра Беляева («Звезда КЭЦ»).

Перелистывая патентные материалы  последних лет,  невольно замечаешь,  что
все чаще и чаще встречаются изобретения  «на грани фантастики», а порой  и
за этой  гранью.  Вот,  например,  выданное  недавно  советское  авторское
свидетельство    №    239797:    «Способ    съемки    и    воспроизведения
кинематографических фильмов,  отличающийся  тем,  что  с  целью  получения
кинестетических ощущений при съемке фильма на соответствующие участки тела
артиста в местах расположения периферических нервов накладывают  электроды
и при игре артиста записывают, например, на магнитную пленку его  биотоки,
а при  воспроизведении фильма  на  соответствующие участки  тела  зрителей
накладывают электроды, на которые  передают записанные при съемке  биотоки
артиста[2].» Любители  фантастики  знают:  в  фантастике  эта  идея  была
высказана еще в 30-е годы.

Прямое использование  идей НФЛ  чаще всего  имеет место  на ранних  этапах
развития новой отрасли науки или техники. В какой-то период (правда, очень
короткий) фантастика  оказывается одним  из основных  источников идей  для
возникающей отрасли знания.  Так было,  например, по  свидетельству В.  В.
Ларина и Р. М. Баевского, с космической биологией: «Наши писатели-фантасты
изложили в своих произведениях много „кибернетических“ идей, которые могут
и должны быть  взяты на вооружение  космической биологией. Так,  например,
проблема регулируемого  анабиоза имеет  громадное значение  не только  для
обеспечения межзвездных перелетов,  но и для  космических полетов  большой
продолжительности  в  пределах   солнечной  системы,  которые,   возможно,
состоятся еще в нашем столетии.

К сожалению, наиболее подробное рассмотрение этого вопроса содержится не в
научной литературе, а в романе И. Ефремова „Туманность Андромеды“[3].»

Даже  чисто   приключенческая  фантастика   нередко  содержит   интересные
изобретательские задачи.

Вот  отрывок  из  фантастической  повести  Гарри  Гаррисона   «Неукротимая
планета»: «Когда  Язон  пристегнул  кобуру  к предплечью  и  взял  в  руку
пистолет, он увидел, что они соединены гибким проводом. Рукоятка пистолета
пришлась ему точно по руке.

— Тут  заключен весь  секрет  силовой кобуры.  — Бруччо  коснулся  провода
пальцем. — Пока пистолет в деле, провод висит свободно. А как только  надо
вернуть его в кобуру…

Бруччо что-то сделал, провод превратился в твердый прут, пистолет выскочил
из руки Язона и повис в воздухе.

— Смотри дальше.

Увлекаемый проводом пистолет нырнул в кобуру.

— А когда надо  выхватить пистолет, происходит все  то же самое, только  в
обратном порядке.

— Здорово! — сказал Язон. — Но все-таки, с чего надо начинать?  Посвистеть
или там еще что-нибудь?

— Нет, он не  звуком управляется. —  Бруччо даже не  улыбнулся. — Тут  все
потоньше и  поточнее. Ну-ка  попробуй представить  себе, что  ты  сжимаешь
левой рукой  рукоятку пистолета…  Так,  теперь согни  указательный  палец.
Замечаешь, как напряглись сухожилия  в запястье? Ну,  вот на твоем  правом
запястье помещены  чувствительные  датчики.  Но они  реагируют  только  на
сочетание импульсов,  которое  означает „рука  готова  принять  пистолет“.
Постепенно вырабатывается полный автоматизм. Только подумал о пистолете, и
он уже у тебя в руке. Не нужен больше — возвращается в кобуру»[4].

Почти  готовое  техническое  задание  на  создание  устройства,  подающего
инструмент!  Такое   устройство   пригодилось  бы   хирургам,   сварщикам,
монтажникам, фотокорреспондентам…

Разумеется, НФЛ  далеко  не  всегда содержит  идеи  зрелые  и  правильные.
Нередко читателю  приходится  встречать идеи,  с  научно-технической  точки
зрения сомнительные или откровенно условные.

Более того, в ряде  случаев в фантастической идее  все неверно. Но в  силу
своей  яркости,  необычности   она  привлекает  внимание   исследователей,
вызывает  интенсивные   поиски,   приводящие  к   ценным   открытиям   или
изобретениям.

Лауреат Ленинской  премии Юрий  Денисюк рассказывает:  «Я решил  придумать
себе  интересную  тематику,  взявшись   за  какую-то  большую,  на   грани
возможности оптики, задачу. И тут  в памяти выплыл полузабытый рассказ  И.
Ефремова…» Речь идет о  рассказе «Тени минувшего».  В пещере в  результате
редкого сочетания  условий возникло  подобие  фотоаппарата: узкий  вход  в
пещеру сыграл роль  входного отверстия  камеры-обскуры, а  противоположная
входу стена, покрытая смолой, стала огромной фотопластинкой, запечатлевшей
мгновения давно минувших эпох. Денисюк подошел к проблеме иначе: а  нельзя
ли получить  изображения  вообще  без объектива?  Исследования  привели  к
открытию одной  из систем  голографии. Но  первый толчок  все-таки был  дан
рассказом! «Я  не только  не отрицаю,  — говорит  Денисюк, —  своеобразное
участие И. Ефремова в моей работе, но подтверждаю его с удовольствием».

НФЛ помогает преодолевать  психологические барьеры на  путях к  «безумным»
идеям, без которых не может развиваться наука.

Это тонкая и пока малоизученная функция научно-фантастической  литературы,
становящейся элементом профессиональной тренировки ученого.

Обычно механизм воздействия НФЛ состоит в  том, что НФЛ вступает в реакцию
с  реальными «рабочими»  мыслями. Суть  этой реакции  становится понятной,
если воспользоваться схемой  творческого процесса, предложенной академиком
Б. М. Кедровым\footnote{Б.  М. Кедров, О теории научного  открытия. В сб.:
«Научное творчество». М., «Наука», 1969, стр. 78–82.}.

В  поисках  решения   задачи  мысль  человека   движется  в   определенном
направлении (а) от единичных факторов Е к выявлению того особенного О, что
присуще этим фактам.

Следующим  шагом  должно   быть  установление  всеобщности   В,  то   есть
формулировка закона, теории и т. п.

Переход от Е к  О не вызывает особой трудности, но дальнейший  путь от О к
В  прегражден  познавательно-психологическим  барьером I.  Нужен  какой-то
трамплин  Г, позволяющий  преодолеть барьер.  Чаще всего  таким трамплином
бывает случайно  возникающая ассоциация, причем появляется  эта ассоциация
при пересечении линии (а) с другой линией мыслей (р).

Научно-фантастическая литература хорошо работает в качестве линии (Р).

НФЛ  воздействует  на творческий  процесс  и  косвенно. Чтение  фантастики
постепенно  ослабляет  психологическую инерцию,  повышает  восприимчивость
к   новому.  На   схеме  Кедрова   это  можно   показать  как   уменьшение
высоты  познавательно-психологического  барьера  и  появление  способности
к   самообразованию  трамплина,   то  есть   к  преодолению   барьера  без
непосредственного внешнего воздействия линии (0).

Нельзя, конечно, сказать, что НФЛ стала незаменимым инструментом  развития
творческих способностей.  Но она,  безусловно,  является одним  из  важных
инструментов. Для эффективного развития творческой фантазии нужна система.
Возможности НФЛ  в этом  смысле  упражнений и,  главное, нужно  далеко  не
исчерпаны. И  все-таки: обучение  приемам  фантазирования, нет  ли  других
способов развития.  Мало  сказать:  «Придумай то-то»,  —  надо  объяснить,
какими приемами следует при этом пользоваться, Одна из немногих попыток  в
этом направлении была  предпринята профессором Стэнфордского  университета
Джоном   Арнольдом[6].   По    методу   Арнольда   предлагается    решать
изобретательские задачи в условиях воображаемой планеты Арктур IV.

Эта придуманная планета отличается своеобразными условиями: температура на
ее поверхности  колеблется от  минус 43  до минус  151 градуса;  атмосфера
состоит из метана,  моря — из  аммиака; сила тяжести  в десять раз  больше
земной. На Арктуре IV живут  разумные существа — метаниане. У  метанианина
руки с тремя пальцами, две ноги, клюв и три глаза, а тело покрыто перьями…

На занятиях  со своими  студентами Арнольд  последовательно  разрабатывает
метанианскую технику:  дома,  средства  транспорта,  дороги,  инструменты…
Нужно  преодолеть  немало   психологических  барьеров,  чтобы   придумать,
например, автомобиль для условий Арктура IV.

Регулярно решая подобные задачи,  студенты профессора Арнольда  постепенно
развивают воображение.

К сожалению, метод Арнольда очень узок. В сущности, это одно упражнение  в
разных вариантах.

Фантастика, легко наделяющая своих героев любыми качествами  невидимостью,
бессмертием,  способностью  проходить  сквозь  стены,  менять   внешность,
подниматься в небо усилием  воли и жить на  дне океана, безудержно  смелая
фантастика  становится  очень  робкой,  когда  речь  заходит  о   развитии
фантазии.

Действительность здесь обогнала фантастику: в нашей стране разработан курс
развития  творческого   воображения.   Создан   этот   курс   общественной
лабораторией методики изобретательства при Центральном совете  Всесоюзного
общества изобретателей и рационализаторов (ВОИР), а преподается он в Баку,
в Азербайджанском общественном институте изобретательского творчества.

Фантазия — величина векторная,  она характеризуется численным значением  и
направлением. Управлять фантазией — значит уметь «включать» и  «выключать»
ее, менять  ее «напряжение»  и, главное,  направлять так,  чтобы  получить
максимальную творческую отдачу. Этому и учит курс развития воображения.

Начинается курс  с проверочных  упражнений, позволяющих  выявить  исходный
уровень способности  к фантазии  и наметить  наилучшую для  данной  группы
программу тренировки.  Вот  одно  из таких  упражнений:  «Нужно  придумать
какое-нибудь, фантастическое — животное».

На  первый взгляд  все очень  просто. Представим  себе собаку,  добавим ей
орлиные  крылья и  хвост дельфина  — чем  не фантастическое  животное. Так
ли,  уж много  мы  нафантазировали, механически  соединив части  известных
животных? Ведь  мы не получили  ничего качественно нового.  Когда-то такое
механическое  объединение  было  очень сильным  приемом;  мифы,  предания;
сказки  населены  множеством  фантастических существ  (кентавры,  русалки,
драконы),  «сделанных»  приемом  объединения.   Но  сегодня  к  творческой
фантазии предъявляются значительно более высокие требования, обусловленные
уровнем  современной науки.  Старый  прием изменился,  стал более  гибким,
глубоким.  Хорошие  образцы нового  уровня  фантазирования  можно найти  в
научно-фантастической литературе.  Так, в рассказе С.  Гансовского «Хозяин
Бухты» описано животное, способное  распадаться на отдельные одноклеточные
организмы.  Когда  надо  охотиться,   эти  клетки  объединяются  в  единый
организм, и с морского дна поднимается нечто вроде гигантского спрута…

Кстати сказать, при разработке курса развития творческого воображения  все
упражнения сначала испытывались на  писателях-фантастах. Это дало  эталоны
для сравнения,  позволило  построить  своего рода  «шкалу  фантазии».  Как
правило, уровень развития фантазии до начала тренировок весьма невысок.

Искра фантазии  высекается  с  трудом  и тут  же  гаснет.  Это  далеко  не
случайно. На протяжении всей  эволюции человеческий мозг  приспосабливался
оперировать привычными  представлениями.  Вот  человек  начал  придумывать
фантастическое животное — и  сразу мысль услужливо подсказывает  привычный
прообраз («представим себе, например, собаку…»), привычное добавление  («…
прибавим ей орлиные крылья…»). Нужны  сотни и тысячи попыток, пока  мысль,
скованная привычными представлениями, преодолеет психологические барьеры.

Вероятно, человеку,  впервые  увидевшему  гимнастические  занятия,  трудно
понять,  что  это  такое:  собрались  взрослые  люди,  зачем-то  без  дела
размахивают руками, подпрыгивают на месте,  а потом расходятся, ничего  не
сделав и ничего не добыв…  Столь же странными могут показаться  стороннему
наблюдателю и занятия по тренировке фантазии. А между тем это серьезная  и
очень напряженная работа.

От занятия к  занятию осваиваются приемы  фантазирования: сначала  простые
(увеличить, уменьшить, сделать  «наоборот» и т,  д.), затем более  сложные
(сделать свойства  объекта меняющимися  во времени,  изменить связь  между
объектом и средой), мысль приучается преодолевать психологические барьеры.

Попросите  придумать  фантастическое  растение  —   и  10  человек  из  10
обязательно начнут видоизменять цветок или дерево, то есть целый организм.
А  ведь  можно  опуститься  на микроуровеиь:  менять  клетку  растения,  и
тогда  даже небольшие  изменения  на клеточном  уровне дадут  удивительные
растения, которых нет и в самых фантастических романах. Можно подняться на
макроуровень  ---  и  менять  свойства  леса:  опять-таки  здесь  окажутся
интересные и неожиданные находки.

Каждый объект (животное, растение, корабль, токарный станок и т. д.) имеет
ряд  главных  характеристик:   химический  состав,  физическое   строение,
микроструктуру  («клетку»)   и   макроструктуру   («сообщество»),   способ
энергопитания, направление развития и т. д. Все характеристики могут  быть
изменены, и  приемов  изменения  тоже десятки.  Поэтому  в  курс  развития
воображения входит обучение  фантограммам, тренировка в  их составлении  и
использовании. Фантограмма —  это таблица, на  одной оси которой  записаны
меняющиеся характеристики объекта, а на другой — главные приемы изменения.

Богатство фантазии в  значительной мере  определяется обилием  накопленных
комбинаций,  которые,  в  сущности,   и  составляют  фантограмму.  Но   до
тренировки мозг  хранит  лишь  разрозненные осколки  таких  комбинаций.  И
только у писателей-фантастов в результате профессиональной тренировки  эти
осколки складываются в подобие целой фантограммы.

Изучение  техники  фантазирования  нисколько  не  похоже  на  зазубривание
шаблонных приемов. Одно и то же упражнение может быть выполнено по-разному
в зависимости  от  личности человека.  Здесь,  как в  музыке,  технические
приемы помогают раскрытию индивидуальных качеств, и интересно  выполненные
упражнения порой доставляют подлинно эстетическое удовольствие, как хорошо
сыгранная вещь.

— Возьмите, пожалуйста, объектом дерево, — говорит руководитель занятий, —
и используйте прием увеличения.

Слушатель, молодой инженер, выходит к доске.

— Итак,  надо увеличить  дерево.  Что ж,  пусть  его высота  будет  триста
метров, даже четыреста…

— Такие деревья существуют, — замечает кто-то с места.

— Да, — соглашается  инженер, — но я  только начал увеличивать.  Допустим,
высота дерева тысяча метров. Или  две тысячи. Вероятно, ветви деревьев  не
выдержат собственной  тяжести, они  работают на  изгиб, как  консоли,  вес
увеличивается  пропорционально  кубу  линейных  размеров.  Значит,   ствол
высотой в  две тысячи  метров будет  иметь сравнительно  небольшие  ветви.
Продолжим: две тысячи метров, три тысячи…

— Пока нет нового качества, — напоминает руководитель.

— Пусть высота будет десять тысяч метров. Тогда вершина попадет в  область
вечных снегов. Вот и новое качество…

— Пожалуйста, чуть смелее. — настаивает руководитель.

— Двадцать тысяч.

— Метров?

— Нет. Двадцать тысяч километров!

В аудитории оживление. Неожиданный скачок:  ствол дерева теперь в  полтора
раза больше диаметра Земли.

— И как выглядит такое… гм… растение?

На доске появляется рисунок А.

Впрочем, инженер тут же спохватывается:

— Нет,  не  так…  На  ствол  будет  действовать  сила  Кориолиса,  вершина
отклонится. К тому же вершина должна стремиться вниз, к теплу, к  воздуху…
Инерция мысли:  я  увеличивал  высоту, тянул  ствол  вверх.  Ствол  должен
удлиняться,  оставаясь  у  поверхности  земли.  Сразу  отпадает  вопрос  о
прочности. Дерево просто лежит на земле (рис. Б).

— Вероятно, это не единственное дерево, — подсказывает руководитель.

— Конечно. Их много. И в целом это выглядит так.

Появляется третий рисунок (рис. В).

— Похоже на марсианские  каналы… А почему бы  и нет? Почему не  допустить,
что  марсианские  каналы  не  просто  полосы  растительности  (была  такая
гипотеза), а именно глобальные  деревья?.. Может быть, не  на Марсе, а  на
какой-то другой планете.

Один из слушателей, астрофизик по специальности, возражает:

— У так  называемых «марсианских  каналов» есть  ряд особенностей;  каждая
новая  гипотеза  обязана  их  объяснить.  Сезонные  изменения,   например.
Удвоение некоторых каналов…

— Очень хорошо, — отвечает  инженер, — ствол продолжает расти,  появляется
второй виток,  отсюда удвоение…  Вообще в  условиях Марса  дереву  выгодно
иметь глобальные размеры. Какая-то часть всегда  там, где лето. К тому  же
на Марсе нет океанов, нет гор, ничто не мешает дереву удлиняться…

— Дерево только часть биосферы, — не сдается астрофизик. — Дерево не может
существовать само по себе.

— А кто сказал,  что на Марсе нет  биосферы? Она находится внутри  дерева,
вот в  чем  дело. Ведь  ствол  не только  длинный,  но и  широкий.  Внутри
развивается жизнь, может быть, даже разумная…

Через полчаса фантастическая гипотеза марсианской биосферы сконструирована
во всех деталях.

Конечно, это чистая фантастика, но,  быть может, не менее интересная,  чем
мыслящий океан, описанный Ст. Лемом в «Солярисе».

Итак,  проверьте   свою   фантазию:  попробуйте   за   полчаса   придумать
какое-нибудь фантастическое растение. Теперь у вас есть с чем сравнить то,
что вы придумаете…

Другой тип упражнений — исследование «черного ящика».

Предположим,   космический   корабль,   преодолев   огромные   межзвездные
расстояния, приблизился к какой-то неизвестной планете.

Договоримся,  что   планета  закрыта   тонким  слоем   условных   облаков,
проницаемых для корабля и спускаемых с корабля автоматических станций,  но
непроницаемых для любого излучения.

Проводная связь сквозь облака тоже невозможна. Разумеется, это  выдуманные
облака, они нужны только для того, чтобы создать ситуацию «черного ящика».

До  спуска  корабля  на  поверхность  нужно,  чтобы  на  планете  побывала
автоматическая станция.

Если она благополучно  вернется, проведя  на планете  десять часов,  можно
спускаться и кораблю. Слушатели предупреждены: на планете действуют те  же
физические законы, те же геологические, климатические и т. п. факторы, что
и на Земле,  на Луне или  на Марсе.  И только один  фактор изменен.  Какой
именно? Это и  надо выяснить  с наименьшего  числа попыток.  Преподаватель
«играет» за планету, слушатели — за экипаж корабля.

Поскольку с корабля — по условиям задачи — ничего нельзя увидеть, остается
один путь: спуск автоматических станций. Связь между кораблем и  станцией,
опустившейся   ниже   условных   облаков,   невозможна   (таковы   правила
упражнения), и «экипажу»  приходится заранее  задавать программу  станции:
опуститься ниже облаков, сделать то-то и то-то, потом вернуться.

Каждый запуск  станции —  модель эксперимента по  обнаружению икс-фактора.
А  этим  иксом  может  оказаться  что  угодно  (кроме  чужой  цивилизации:
предполагается, что на планете нет разумной жизни).

Икс-фактором может  быть, например,  сила тяжести,  уменьшающаяся  обратно
пропорционально сотой степени  расстояния. Или сплошная  сеть вулканов  на
поверхности планеты.

Или острые пики,  доходящие до  условных облаков. Или  иной темп  времени.
Первые же действия «экипажа»  вскрывают драматизм ситуации. Вот  несколько
отрывков из диалога, записанного магнитофоном.

— Отправляем автоматическую станцию.  Задание: опуститься на  поверхность,
взять пробы грунта, воздуха… сделать снимки… вернуться через час…

— Станция не вернулась.

— Почему?

— Это ваше дело — узнать почему…

— Хорошо. Это могла быть случайность.  Отправляем еще одну станцию. С  той
же программой.

— Вторая станция тоже не вернулась.

— Странно… Учтите, наши  станции имеют автоматы,  выводящие их на  посадку
только в безопасном месте, — А как они узнают, что место безопасное?

— По рельефу хотя бы… Если внизу ровный грунт — значит безопасно.

— Прекрасно. И  все-таки две  станции, снабженные  системами выбора  места
посадки, не вернулись… Что вы будете делать дальше?..

Удивительно, насколько сильно проявляется в этих условиях  психологическая
инерция! Иногда «экипаж» на первом этапе упражнения теряет десятки станций
и только после этого задумывается: может быть, привычная тактика здесь  не
подходит?

— Тогда так:  пусть станция  опустится под  эти облака,  сделает снимки  и
сразу вернется. Опуститься она должна совсем немного — на метр, не больше.

— Станция вернулась.

— Наконец-то! А снимки получились?

— Да.

— И что на них?

— Степь, река,  холмы, лес…  Все как  на Земле.  Снято с  высоты в  десять
километров. Взяты пробы атмосферы — воздух тоже как на Земле…

— А почему посаженные в этом районе станции не вернулись?

— Это уж ваше дело — узнать почему…

—  А  можно,  чтобы  следующая  станция,  оставаясь  на  высоте  в  десять
километров, сбросила вниз зонд с радиопередатчиком?

— Можно.

—  Хорошо!  Посылаем  станцию.  Она  остается  наверху,  под  облаками.  И
сбрасывает  зонд на  какое-нибудь ровное  место. Например,  на холм.  Зонд
должен подавать  сигналы с поверхности.  Станция их запишет и  вернется на
корабль.

— Станция вернулась, но никаких сигналов зонда она не записала.

— Почему?

— Это ваше дело — узнать почему…

Упражнение продолжается, и еще не скоро «экипажу» удается обнаружить,  что
икс-фактор  на  этой  планете  —  замедленная  скорость  света  и   других
электромагнитных колебаний: один сантиметр в секунду.

До станции,  находящейся на  десятикилометровой  высоте, свет  доходит  на
двенадцатый день.

Станция видит степь, реку, холмы, а внизу — болото или горные вершины…

Три-четыре  таких  упражнения  наглядно показывают  слушателям,  насколько
прилипчива психологическая  инерция. А преподаватель  получает возможность
изучить индивидуальные особенности слушателей: когда одна за другой гибнут
станции (хотя бы  и условные) и возникает  критическая ситуация, отчетливо
виден стиль действий каждого члена «экипажа».

А  потом  такого же  рода  упражнения  выполняются  иначе —  для  развития
управляемой   фантазии.  Два   человека  в   «экипаже»  только   выдвигают
предложения, что делать дальше. Третий  тоже занят только одним: проверяет
поступившее  предложение  на   психологическую  инерцию.  Если  обнаружена
инерция,  предложение  не  проходит.  Если явных  признаков  инерции  нет,
предложение принимается.  Полученная информация («Станция  не вернулась…»)
поступает  к  четвертому  члену  «экипажа»,  который  только  регистрирует
странности,  то есть  признаки  и формы  проявления  икс-фактора. Пятый  и
шестой  члены  «экипажа»  только ищут  возможные  объяснения  странностей,
выдвигают гипотезы. На основе этих  гипотез двое первых намечают, что надо
делать дальше.

Слушатели, следящие  за действиями  «экипажа»,  знают «секрет»  планеты  и
могут анализировать (молча, не подсказывая «экипажу») ход поиска.  Процесс
научного творчества моделируется точно и наглядно.

Современный  эвристический   алгоритм   решения   изобретательских   задач
разбивает творческий  процесс на  последовательные  шаги. Многие  из  этих
шагов могут быть сделаны только при достаточно развитом воображении.

Вот вы  читаете условия  задачи: «Необходимо  повысить стойкость  задвижки
трубопровода, по которому  течет истирающая  задвижку пульпа», —  и у  вас
сразу  возникает   образ   злополучной  задвижки.   Этот   образ   создает
психологические барьеры, заранее навязывая  вам привычные представления  о
задвижке: примерные  размеры  задвижки, примерную  скорость  ее  действия,
примерную стоимость…

Вы еще не начали  решать задачу, а  уже возведены высокие  психологические
барьеры, замыкающие мысль в пределах известного, привычного. Здесь и нужна
фантазия.

Она оказывается  надежным рабочим  инструментом, позволяющим  преодолевать
психологические барьеры.

В алгоритме решения  изобретательских задач этот шаг  связан с применением
оператора РВС  (Р —  размеры, В —  время, С —  стоимость). Оператор  РВС —
шесть  мысленных экспериментов  с  условиями  задачи. Эксперимент  первый:
начнем  неограниченно уменьшать  размеры  объекта и  посмотрим, как  будет
меняться  задача  и  какие  новые  решения  —  при  этом  возникнут.  Если
трубопровод  мысленно превратить  в капиллярную  трубку, обычная  задвижка
вообще окажется  непригодной. Капиллярную  трубку надо  перекрывать иначе,
Например,  пережимать. А  почему  бы  не выбросить  задвижку  и в  большом
трубопроводе? Если  мы научимся пережимать трубопровод,  не сжимая жестких
стенок, задача будет решена.

Эксперимент второй: мысленно увеличим размеры трубопровода.

Как перекрывать  трубопровод  диаметром  в  десять  километров?  в  тысячу
километров?

Я  взял для  примера  простенькую задачу,  но и  на  ней видно,  насколько
необходимо для  таких мысленных  экспериментов умение  работать фантазией.
Тут-то  и   нужны  навыки,  приобретенные  на   задачах  типа  «Придумайте
фантастическое животное».

Психологическая инерция  многолика.  Сбитая  оператором  РВС,  она  вскоре
возникает вновь — на этот раз в виде навязчивого представления о структуре
объекта, «Задвижка…  что-то  такое  твердое, непроницаемое…»  И  снова  на
помощь  приходит  фантазия.  Представим  себе  наш  объект  в  виде  толпы
маленьких человечков (наподобие демонов Максвелла). Как должны вести  себя
человечки, чтобы  перекрыть трубопровод  и  в то  же время  не  истираться
частицами   руды?   Заменив    привычный   технический   образ    задвижки
фантастической  «толпой   маленьких  человечков»,   мы  получим   отличную
«безынерционную»  модель  для  мысленных  экспериментов:  человечки  могут
выполнить любую нашу  команду, они  не боятся  слова «невозможно».  Потом,
когда станет  ясно,  что должны  делать  человечки, мы  снова  вернемся  к
техническому образу,  и это,  быть  может, будет  новый образ,  совсем  не
похожий на привычную задвижку…

Инженер, придумавший «марсианские»  деревья, недавно  получил свое  первое
авторское свидетельство.

Называется оно вполне прозаически — «Мастика для полов».

Занятия по развитию  воображения не сделали  инженера «пустым  фантазером»
(нам приходилось слышать и такие опасения). Стали изобретателями и  многие
из тех, кто  в составе  «космических экипажей»  исследовал икс-факторы  на
планетах, затянутых пеленой «условных облаков»…

Немалую роль  в  этом наряду  с  упражнениями сыграло  и  обильное  чтение
научно-фантастической литературы.

Редкая статья о  фантастике обходится  сейчас без напоминания  о том,  что
фантастика прежде всего художественная литература и что основная ее задача
— человековедение.

Что  ж,  не  спорю:  фантастика,  конечно,  художественная  литература  и,
конечно, должна  заниматься  человековедением.  Но  такова  уж  диалектика
научно-технической  революции:  художественная  фантастическая  литература
дает и непосредственный экономический эффект.

Еще никто не подсчитал этот эффект. Он не лежит на поверхности, его трудно
выделить в чистом виде.

Но он есть, и тем досаднее, что существует огромный разрыв между тиражами,
которыми издается  НФЛ, и  потребностью хотя  бы у  тех, кому  НФЛ  просто
необходима для развития творческого воображения.

А занятия по курсу «эртэвэ» продолжаются. Теперь это упражнения посложнее.

…Предположим, сто  миллионов лет  назад нашу планету  посетила экспедиция,
прилетевшая  из   далекой  звездной  системы.  К   великому  разочарованию
экспедиции, по  Земле еще  бродили динозавры и  не было  никаких признаков
разумной  жизни.  Впрочем,  высокоразвитая наука  пришельцев  позволила  с
высокой  точностью  (плюс-минус  десять миллионов  лет)  вычислить  время,
необходимое для возникновения цивилизации. Решили улететь, оставив будущей
цивилизации нечто  такое, что  поможет потом установить  контакт. Понятно,
оставить это «нечто» на Земле нельзя: за сто (плюс-минус десять) миллионов
лет  уйдут под  воду одни  материки  и появятся  другие, «нечто»  окажется
где-то  в  недрах  планеты.   Целесообразнее  оставить  «нечто»  на  почти
неменяющейся Луне.

И вот — уже в XX веке — космонавты обнаруживают на Луне два десятка черных
ящиков.  С  виду  это  самые  обычные  ящики.  Наподобие  посылочных.   Но
совершенно гладкие, без намека на крышку, шов или замочную скважину. И тут
же на  скале нацарапаны  картинки, поясняющие  происхождение ящиков  и  их
назначение. Конечно,  рисунки —  это условное  допущение, необходимое  для
создания проблемной ситуации.

Лента магнитофона. Запись сделана в группе новичков.

— И что от нас требуется?

— Не знаю.  Я описал вам  ситуацию. Поставьте себя  на место  космонавтов,
нашедших эти ящики. И сами решайте, что делать.

— Непонятно, в чем задача… Ну взяли один ящик и отнесли на базу.

— Взяли? Так вот, когда вы его взяли, он сразу рассыпался.

— Как это… рассыпался?

— Очень просто. Рассыпался в порошок, а порошок тут же превратился в газ —
и все.

— Возьмем другой ящик.

— Другой ящик тоже рассыпался, превратился в газ, исчез.

— А почему?

— Это ваше дело — знать почему.

— Возьмем еще один ящик, пусть рассыпается, но мы соберем газ и исследуем.

— Исследовали. Восемьдесят процентов углекислого газа, двадцать  процентов
азота.

Могучий способ исследования:  сожжем, например,  книгу и  по составу  газа
будем судить  о  ее содержании…  Это  не последняя  ошибка.  Группа  будет
уничтожать ящик  за  ящиком,  постигая  простую,  но  необходимую  мораль:
фантазией надо управлять.

Свободный полет  фантазии  часто  отождествляют  с  примитивным  перебором
вариантов: «Хочу —  попробую так,  хочу — попробую  иначе…» Формально  при
таком переборе фантазия ничем не стеснена. А на самом деле психологическая
инерция намертво сжимает неокрепшие крылья воображения. Мысль топчется  на
тесном пятачке простеньких  комбинаций. Какой уж  тут полет!.. Для  полета
нужно управление.

Еще одна  лента.  Запись,  сделанная  в  группе,  прошедшей  первые  этапы
тренировки:

— В расположении ящиков нет закономерности?

— Нет.

— Хорошо. Тогда фотографируем ящики. Один ящик доставляем на базу.

— Ящик, который вы взяли, рассыпался.

— Как?

— Очень просто. Рассыпался в порошок, а порошок тут же превратился в газ —
и все.

— Понятно… Глупо было тащить ящик. Инерция мысли. Сначала надо понять, что
делать. С планетой, спрятанной под облаками, мы ничего не могли узнать без
проб. А здесь и без проб можем узнать многое.

— Ну а все-таки: каковы ваши действия?

— Зачем вы нас подталкиваете к действиям?  В этой задаче как раз не  нужно
действовать.

— Почему?

— Я  хотел сказать:  не надо  спешить с  действиями. Ящики  пролежали  сто
миллионов лет.  Что для  них еще  год или  десять лет?  Мы должны  сначала
выработать обоснованную тактику.

— Как?

— Перевернем задачу. Поставим себя на место пришельцев. Мы оставляем ящики
цивилизации,  которая   появится  через   сто   миллионов  лет.   Как   мы
запрограммируем  ящики?   Прежде   всего  они   должны   определить,   что
представляет собой цивилизация, обнаружившая ящики. А этого не сделаешь  с
одной встречи. Ящикам  придется долго  присматриваться к  разным людям,  к
обществу. Я бы сказал: им нужно пожить среди нас. Тогда они не ошибутся.

— Так что же вы предлагаете?

— Не  трогать ящики.  Пусть изучают  нас. Ящики  оставлены  высокоразвитой
цивилизацией. Ее  представители  еще  сто миллионов  лет  назад  совершали
межзвездные перелеты. Такой цивилизации легче  понять нас, чем нам  понять
ее.

— А ящики?

— Пусть лежат. Поняв нас, они сами что-то предпримут. Конечно, потребуется
немало времени…  Но и  решение  они должны  принять ответственное.  Не  со
всякой цивилизацией нужен контакт, не всем можно передать знания…

Сегодня мы делаем первые шаги  в управлении фантазией. Еще непривычно само
это словосочетание: управляемая фантазия. Но  пройдет какое-то время, и мы
научимся  планомерно развивать  воображение и  на этой  основе формировать
творческие  способности.   Быть  может,   это  станет  одной   из  главных
особенностей будущего коммунистического общества.

Примечания

1

Библиотека современной фантастики. М., изд-во «Молодая гвардия», 1967,  т.
10, стр. 368.

2

Бюллетень «Открытия. Изобретения.  Промышленные образцы. Товарные  знаки».
1969, № 11, стр. 143.

3

«Известия Академии наук СССР». Серия биологическая, № 1, 1963. стр. 13.

4

«Вокруг света», 1972, стр. 69.

5



6

Дж.  Диксон, Проектирование  систем: изобретательство,  анализ и  принятие
решений. М., «Мир», 1969, стр. 39.

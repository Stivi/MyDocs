
                                 Генрих Альтов

                                Ослик и аксиома


   Старый серый  ослик Иа-Иа  стоял один-одинешенек  в заросшем  чертополохом
   уголке Леса, широко расставив передние ноги и свесив голову набок, и думал
   о Серьезных Вещах.

   А. Милн, Винни-Пух

   То, о чем я хочу рассказать,  началось с небольшой статьи, написанной  для
   «Курьера ЮНЕСКО».

   Я изрядно помучился с  этой статьей, уж очень  невыигрышной была тема.  Ну
   что можно сказать — на трех  страничках! — о прошлом, настоящем и  будущем
   машин?..

   Недели две я  просто не  знал, как подступиться  к статье,  а потом  нашел
   любопытный прием: пересчитал мощность всех машин на человеческие силы,  на
   прислуживающих нам условных рабов. Киловатт заменяет десять крепких рабов,
   в общем-то простая арифметика.

   Я взял жалкие цифры конца XVIII века — они немногим отличались от нуля — и
   проследил их  судьбу:  мучительно  медленный,  почти  неощутимый  рост  на
   протяжении столетия, затем подъем, становящийся  все круче и круче,  почти
   вертикальный взлет после  второй мировой войны  (десятки и сотни  условных
   рабов на человека) и, наконец, нынешний год, к которому каждый из нас стал
   богаче римского сенатора.

   «Размышления рабовладельца» (так я назвал  статью) были отосланы, но  меня
   не  оставляло  какое-то  смутное   чувство  неудовлетворенности.  Оно   не
   проходило, и, разозлившись, я переворошил заново все цифры.

   Нет чувства острее, чем то,  которое испытываешь, приближаясь к  открытию.
   Наверное, это передалось  нам от  очень далеких предков,  умевших в  хаосе
   первобытного леса ощутить  странное и молниеносно  настроить каждый  нерв,
   каждую клеточку еще не окрепшего мозга в такт его едва различимым шагам.

   Теперь я могу объяснить все в  нескольких словах, будто и не было  долгих,
   временами казавшихся безнадежными поисков.

   Жизнь машины, любой машины, становится  слишком короткой: в среднем  около
   трех-четырех лет. Машина могла бы жить раз в восемь или десять дольше,  но
   наука открывает новые, более совершенные принципы — приходится менять  всю
   нашу технику.

   Промежутки между  открытиями укорачиваются,  и неизбежно  наступит  время,
   когда мы  должны  будем менять  машины  (подчеркиваю —  все  машины,  весь
   огромный технический мир)  ежегодно, потом ежечасно,  ежеминутно. А  иначе
   куда денется стремительно нарастающая лавина открытий?..

   Быть может, я не нашел бы ответа на этот вопрос. Скорее всего не нашел бы.
   Есть вопросы, имеющие ехидное свойство появляться задолго до того времени,
   когда на них  можно ответить.  Но однажды, листая  «Вопросы философии»,  я
   обратил внимание  на заметку,  густо усыпанную  примечаниями и  оговорками
   редакторов. В  заметке говорилось  о принципах  дальнего  прогнозирования.
   Речь шла о возможности  уже сегодня решать  на аналоговых машинах  задачи,
   подобные тем, с которой я столкнулся.

   В первый момент меня поразила даже не сама заметка, а подпись. Статья была
   написана Антенной; я не видел его четырнадцать лет, со школьных времен.

                                     * * *

   В нашем классе он был самый высокий, но Антенной его прозвали не за  рост.
   Он вечно  таскал в  кармапах  кучу радиохлама  и каждую  свободную  минуту
   собирал приемники.  Делал  он  это  как-то  машинально.  Он  мог  смотреть
   кинокартину или ехать в трамвае, а руки  его в это время работали сами  по
   себе: что-то отыскивали в карманах, что-то с чем-то соединяли, наматывали,
   прилаживали —  и  вдруг  все эти  фитюльки,  болтавшиеся  на  разноцветных
   проводах, оживали, начинали шипеть, свистеть,  а потом сквозь плотный  шум
   пробивался голос  диктора. Антенна  что-то менял,  подкручивал: шум  таял,
   исчезал — и возникала прозрачная, чистая музыка.

   Не помню ни одного  случая, чтобы у Антенны  не хватило материала. Он  мог
   пустить в дело любую  вещь. Как-то он собрал  приемник из двух  радиоламп,
   мотка проволоки,  моей собственной  вечной ручки  и старого  велосипедного
   насоса.

   Антенна приехал с Урала.  Мы тогда были в  восьмом классе, и первое  время
   все выпрашивали  у него  приемники.  Он отдавал  их, нисколько  не  жалея.
   Вообще Антенна был  хорошим парнем  — так считали  все, и  только он  сам,
   кажется, иногда огорчался, что его вечно тянет лепить приемники. Он именно
   так и  говорил  — «лепить».  Ему  нельзя было  играть  в футбол:  в  самый
   отчаянный  момент  он  вдруг  подбирал  обрывок  какого-нибудь  провода  и
   принимался его  рассматривать.  Даже  в воротах  Антенну  невозможно  было
   поставить, потому что он  сразу начинал возиться  со своим радиохламом,  а
   это плохо, когда у вратаря заняты руки. Мы играли на пустыре, за стройкой,
   и Антенна обычно сторожил портфели. Он сидел на траве, поглядывал на  игру
   и собирал очередной приемник.

   Работали приемники  лучше  заводских,  хотя  вид  у  них  был  не  слишком
   красивый. Антенна  почему-то не  признавал футляров  и коробок,  приемники
   получались у  него открытыми.  Начинка висела  на проводах,  как  гирлянда
   елочных игрушек. Но если  Антенне давали футляр, он  не спорил и сразу  же
   принимался за работу. Сначала за Антенной ходила целая очередь, а потом мы
   привыкли. И он делал что  хотел: соберет какую-нибудь замысловатую  схему,
   разберет и начинает собирать новую…

   Он учился с нами только год; потом его семья перебралась куда-то на Алтай.
   Весь этот год мы с Антенной сидели за одной партой; мне нравилось следить,
   как он работает. Именно тогда я всерьез задумался о своем будущем. Сейчас,
   к тому  же  в беглом  пересказе,  это  звучит наивно:  задумался  о  своем
   будущем. Но так было. Я не хотел отставать, на то имелась масса причин,  и
   выбрал  химию,  к  которой  Антенна  был  совсем  равнодушен.  Для   химии
   потребовалась физика, для физики  — математика, а  в математике я  однажды
   натолкнулся на математические основы социологии.

   Наука о Человеке — так я  определяю предмет социологии. Океан живых  цифр,
   то взметающий  громовые  валы,  перед  которыми  ничто  любое  цунами,  то
   дробящийся на мириады капелек, расцвеченных удивительными красками  нашего
   мира. Никакая  другая  наука  не отражает  столь  полно  все  человеческое
   историю, взлеты,  падения,  ум,  глупость,  горе,  счастье,  труд,  нравы,
   преступления, подвиги…

                                     * * *

   Я нашел Антенну без всякого труда — по телефонной книге. Раньше мне просто
   не приходило  в голову,  что он  в  Москве и  все так  элементарно:  взять
   трубку, позвонить, договориться о встрече.

                                     * * *

   Мы сидии в кафе-мороженое «Арктика» у огромного окна, за которым  бесшумно
   кружатся фиолетовые от  рекламных огней  хлопья снега. Зал  почти пуст,  в
   дальнем углу официантки не спеша пьют чай.

   — Любопытно, — вяло говорит Антенна.  — Насчет машин очень любопытно.  Да,
   вот что… Забыл спросить: ты не видел Аду Полозову? Интересно, как она…

   Что ж, все закономерно.  Ада должна интересовать  Антенну больше, чем  мои
   рассуждения о машинах.

   Однажды я  разбил свои  часы. Разбил  капитально, в  ремонт их  не  брали.
   Выручил Антенна:  он  втиснул  в часовой  футляр  безбатарейный  приемник,
   настроенный на «Маяк». Я узнавал время по радио; это не очень удобно, зато
   оригинально.

   И вот тогда  я допустил  ошибку. Я показал  эти часы  Аде. Она  занималась
   фехтованием, очень гордилась этим,  говорила, что фехтование  вырабатывает
   характер. Безусловно, вырабатывает. Даже в избытке. Она сняла свои  часики
   и стукнула их о подоконник — эффектно и точно. Не помню, какой марки  были
   эти часики. Маленькие, овальные. Кажется, «Капелька».

   «Можно не спешить, — великодушно сказала Ада, — несколько дней я  обойдусь
   без часов, но мне хотелось бы иметь приемник с двумя диапазонами».

   Зимних каникул у Антенны в тот раз не было. Даже Новый год он не встречал.
   Когда он появился  после каникул,  Ада сказала,  что вид  у него  немножко
   дикий и задумчивый,  как у  поэта Роберта  Рождественского. Но  «Капельку»
   Антенна принес в полном порядке. С двумя диапазонами. Часы тоже работали.

   … - Мы с ней переписывались, — рассказывает Антенна. — Почти полгода.  Она
   писала, что  хочет  стать укротительницей.  А  что? По  идее  подошло  бы,
   правда?

   По идее Ада вполне могла стать укротительницей. Но она стала  стюардессой.
   Дальние линии: Москва — Дели, Москва — Рим, Москва — Токио… Она погибла  в
   Гималаях.

   Да, конечно,  у  нее  была «Капелька»  первого  выпуска.  Очень  маленькие
   часики, похожие на каплю застывшего янтаря.

   Антенна водит пальцем по синему пластику стола, рисуя растаявшим мороженым
   аккуратную восьмерку.

   — Вот ведь как получилось, — говорит, наконец, Антенна. — Гималаи… Далеко.

   Ну, не очень-то далеко.  За эти годы  я побывал во  многих странах: что  в
   наше время расстояния? Милан и София — социологические конгрессы.  Коломбо
   — международный симпозиум. Оттава — конференция по применению в социологии
   электронных вычислительных машин. Париж  и Лондон дискуссионные встречи  с
   западными социологами. Туристские поездки: Египет, Польша, Куба, Болгария.
   В  международный  социологический  год  работал  в  Западной  Сибири  и  в
   Монголии.

   Антенна ошеломлен. Он спрашивает о египетских пирамидах, и я  рассказываю,
   хотя мысли  мои  упорно  возвращаются к  Аде.  Почему?  Это  бессмысленно,
   ненужно. На эти мысли давно наложено  табу. Сейчас я их выключу. Возьму  и
   выключу. Пирамиды?  Так вот  пирамиды. Издали  чувствуешь себя  обманутым:
   ждал чего-то  более громадного.  Но по  мере того  как подъезжаешь  ближе,
   пирамиды растут, поднимаются вверх, вверх,  в самое небо — это  производит
   подавляющее впечатление.

   Антенна внимательно слушает, потом говорит:

   — А все-таки странно, что ты бросил химию и занялся социологией.

   Ничуть не  странно.  В этом  мире  вообще  все закономерно.  Моя  бабка  с
   материнской   стороны   была   чистокровной   цыганкой.   У   меня   такая
   наследственность: стремление  предвидеть будущее.  Так что  социологией  я
   занялся совсем не случайно.

   Антенна недоверчиво улыбается. Ладно, я могу объяснить по-другому:

   — Если бы существовал двухмерный мир, тамошним обитателям, наверное, очень
   хотелось бы хоть одним глазком заглянуть в третье измерение. Что там?  Как
   там?..  В  нашем  трехмерном  мире  просторнее.  Как  отметил  поэт,  есть
   разгуляться где на  воле. Но  встречаются люди,  которым обязательно  надо
   высунуть нос в  будущее. Взять  и высунуть.  Что там?  Как там?..  Антенна
   охотно согласился:

   — Это верно. Очень хочется заглянуть в будущее…

   Мы уже два часа сидим в этом холодильнике (здесь по крайней мере тихо),  и
   я никак не могу освоиться с тем,  что Антенна не сделал карьеры (я имею  в
   виду научную  карьеру и  вкладываю в  это слово  хороший, честный  смысл).
   Антенна был самым талантливым  в нашем классе. Бывает,  что человек еще  в
   детстве становится выдающимся  музыкантом; Антенна обладал  столь же  ярко
   выраженным «электронным» талантом. И вот теперь, с первых минут встречи, я
   почувствовал,  что  удивительный  талант  Антенны  не  исчез.  Но  Антенна
   работает рядовым инженером на заводе игрушек. Эю было бы нормально, тысячу
   раз  нормально,  если  бы  не  талант,  совершенно  исключительный  талант
   Антенны…

   Внешне Антенна мало изменился: длинный, тощий, по-мальчишески угловатый  и
   застенчивый.

   Он говорит  о телевидении.  Некоторые факты  я уже  знаю. Но  в  изложении
   Антенны они звучат иначе.

   Он относится к приемникам, как к живым существам; ему жаль их — они  живут
   все меньше и меньше.

   Четверть  века  существовало  черно-белое  телевидение,  затем   появилось
   цветное ТВ  — и  сотни миллионов  вполне работоспособных  приемников  были
   выброшены на  свалку. Их  сменили  полмиллиарда цветных  телевизоров.  Эти
   массивные добротные ящики могли бы  работать пятнадцать или двадцать  лет.
   Но прошло  всего четыре  года,  и они  безнадежно устарели:  началась  эра
   «стерео». Заводы выпустили уже свыше миллиарда «стерео». Сегодня  «стерео»
   нарасхват. А через  год или два  они тоже пойдут  на свалку —  обязательно
   появится нечто новое.

   — Спрашивается, кто  на кого работает?  — Антенну удивляет  эта мысль,  он
   шепчет, шевеля губами: —  Вот именно. В конце  концов это вопрос о  смысле
   жизни. Машины слишком быстро стареют, мы работаем, чтобы построить  новые,
   а они стареют еще быстрее… И никто этого не замечает, у человечества  пока
   хватает других забот.

   Хватает. А  когда этих  забот пе  хватало? Закономерность  еще не  бьет  в
   глаза, в этом все дело.  Она вылезет где-то в  XXI веке, и тогда  придется
   решать:  непрерывно  менять  технику,  менять  каждый  день,   безжалостно
   выбрасывая миллиарды  новеньких  машин  только потому,  что  они  морально
   устарели, или смириться с тем, что наука все дальше и дальше будет уходить
   от техники, производства, жизни. А зачем тогда наука? Познание ради самого
   процесса познания?..

   — Технику надо перестраивать, — без особой уверенности говорит Антенна.  —
   Она должна быть приспособлена к постоянной перестройке. Как ты думаешь?

   Хотел бы я знать, как перестраивать гигантские домны, мартены, конвертеры,
   если, например, открыт способ прямого восстановления металла из руды? Надо
   менять все — до последнего винтика! Проще и выгоднее строить заново.

   — Странно, — говорит Антенна, глядя в окно. За стеклом вспыхивают и гаснут
   желтые огни автомобильных фар. Точка, тире, точка…

   — Странно… Мы не виделись столько  лет… Статейка, которую ты читал,  давно
   устарела. Там  ведь  были только  предположения.  Понимаешь, год  назад  я
   вылепил прогнозирующую машину…

                                     * * *

   Я  достаточно   хорошо  представляю   трудности,  связанные   с   машинным
   прогнозированием.  Скажи   кто-нибудь  другой,   что  такая   машина   уже
   существует, я счел бы это шуткой. Но в Антенну трудно не верить.

   Я иду выпрашивать чаю, в этой  холодильной фирме чай вне закона.  Кажется,
   девушки приняли Антенну за какого-то выдающегося спортивного деятеля.  Они
   включают проигрыватель, а у нас на столе появляются горячий чай и домашнее
   печенье.

   Такое печенье я  ел у Антенны,  когда у  него был день  рождения. Мы  всем
   классом подарили  ему  микроскоп.  Не  совсем новый  (мы  покупали  его  в
   комиссионном), но  очень  внушительный,  с тремя  объективами  на  турели.
   Антенна был чрезвычайно доволен микроскопом и все порывался объяснять нам,
   что размеры  приемников и  контрольно-измерительной аппаратуры  должны  по
   идее стремиться к нулю. Его никто не слушал — мы танцевали.

   К весне он  стал собирать  очень маленькие приемники,  ребята называли  их
   микробными. Приемники  были не  больше маковых  зернышек и  ловили  только
   Москву.  Когда  их  клали  в  пустую  спичечную  коробку  и,  приоткрывая,
   настраивали ее на резонанс, звук получался довольно громкий.

   Как-то Антенна принес спичечную коробку, набитую совсем уж миллимикробными
   приемниками, и  мы ухитрились  их  рассыпать. Все  полезли  рассматривать,
   Антенну толкнули.  Когда коробка  упала, ветер  подхватил приемнички,  они
   сразу  же  полетели.  Словно  кто-то  подул  на  одуванчик.  Мы  бросились
   закрывать окна.  В первый  момент мы  даже не  сообразили, что  приемнички
   продолжают работать  и  звук  почемуто  становится  сильнее…  Скандал  был
   грандиозный.

   … -  Прогнозирование обычно  рассчитано на  благополучную кривую.  Антенна
   рисует на  столе линию,  плавно  поднимающуюся вверх.  — А  развитие  идет
   иначе: кривая,  разрыв, более  крутой участок,  соответствующий  появлению
   чего-то принципиально  нового,  потом снова  разрыв  и снова  кривая  идет
   круче. Все дело в том, что  прогнозирующая машина… ну, ты сам знаешь,  она
   должна смотреть  далеко вперед.  За  все эти  разрывы,  прямо сюда,  —  он
   показывает на верхний участок  кривой. — Без  машины человеку пришлось  бы
   ворошить гигантский  объем  информации, преодолевать  множество  привычных
   представлений. Помнишь,  как Эдгар  По описывал  будущее  воздухоплавание?
   Громадный воздушный  шар  на  две  тысячи  пассажиров…  Очень  характерная
   ошибка. Мы поневоле  прогнозируем количественно: увеличиваем  то, что  уже
   есть. А надо предвидеть новое качество. Надо знать — когда оно появится  и
   что даст. Согласен?

   Я отвечаю, что  да, согласен, и  спрашиваю, почему он  работает на  заводе
   игрушек.

   — Выкладывай, что случилось?

   — Ничего.  Ничего  особенного. Учился  в  аспирантуре. Потом  ушел.  А  на
   заводе… что ж, на заводе хорошо. Работа интересная. И потом барахолка  там
   богатейшая, — он оживился, — могу брать, что нужно.

   Ясно. Этот непротивленец получил барахолку и счастлив. Я возьму Антенну  в
   свою лабораторию. Ну конечно! Как я об этом сразу не подумал.

   — Значит, ты работаешь дома?

   — Так даже удобнее. Никто не отвлекает.

   Он многословно  расписывает преимущества  работы в  домашних условиях.  Не
   знаю, на кого я больше зол — на Антенну или на тех неизвестных мне  людей,
   которые обязаны были разглядеть его талант.

   — Сборка прогнозирующей машины на дому. Двадцатый век. Дикарь!

   —  Так  ведь  она  не   очень  сложная.  Вот  разработать  алгоритм   было
   действительно трудно,  а  машина… По  идее  первая машина  всегда  проста.
   Усложнение начинается  потом.  Знаешь,  первый  радиотелескоп  в  Гарварде
   сколотил плотник из досок, и стоило это всего четыреста долларов. А первые
   вычислительные машины были сделаны из детского «Конструктора»… В общем это
   не важно.  Мы как-то  сумбурно говорим,  я ведь  еще не  сказал  главного.
   Понимаешь, какая история: я решал на машине другую задачу, совсем  другую.
   Но ответ, кажется, подойдет и для твоей задачи…

   — Какую задачу ты решал?

   — Видишь ли,  машина у  меня небольшая,  я втиснул  ее в  одну комнату…  С
   самого начала пришлось  лепить машину  в расчете на  вопрос, который  меня
   интересовал.  Элементы  памяти  очень  емкие,  на  биоблоках,  лучшие   из
   существующих. И  все  равно  на шестнадцати  квадратных  метрах  много  не
   разместишь. Отсюда узкая специализация:  машина рассчитана только на  один
   вопрос. Летом я начал ее разбирать…

   — Стоп! Какой вопрос ты задал машине?

   — Видишь ли, — Антенна мнется, заглядывает мне в глаза, — я долго выбирал,
   ты не думай, пожалуйста, что это фантазерство. Я искал узловую проблему…

   — А конкретно?

   —  Проблема  возвращения.   Полеты  к  звездам.   Ну,  ты  должен   знать.
   Классическая проблема возвращения: на корабле прошло пять или десять  лет,
   а на Земле —  сто или двести.  Вернувшись, люди попадают  в чужой мир.  Им
   трудно, может быть, даже невозможно жить в этом мире. И потом они  прибыли
   с открытиями, которые на Земле давно уже сделали без них. Полеты  лишаются
   смысла.

   —  Классическая  проблема  возвращения…  Допустим.  Но  почему  ею   нужно
   заниматься в одиночку?

   — А что здесь делать коллективу? Ну что бы делал институт?

   — Устраивают же на эту тему конференции…

   — Нет,  ты  что-то  путаешь:  были конференции  по  межзвездной  связи.  А
   перелеты на межзвездные  дистанции считаются неосуществимыми.  Практически
   неосуществимыми. О  чем мы  говорим!  Нет ни  одного института,  ни  одной
   лаборатории, ни  одной  группы,  которые  специально  бы  занимались  этой
   проблемой. Да и как заниматься? Сначала надо найти какие-то опорные  идеи.
   Найти, развить, доказать, что это не бред.

   — Ты можешь работать над другой проблемой, а в свободное время…

   — Нет! —  Антенна, протестующе  взмахивает рудами.  — Нельзя  отвлекаться,
   надо думать на полную мощность.

                                     * * *

   В автобусе на  запотевшем оконном  стекле Антенна  чертит схему,  объясняя
   устройство своей машины. За стеклом мелькают приглушенные снегом  неоновые
   огни, и от этих огней, от их движения схема кажется объемной,  работающей,
   живой…

   Теперь я  не  сомневаюсь в  машине…  Непонятно другое:  если  машина  была
   собрана, если она работала, почему все так тихо?

   — А как  же? —  удивляется, Антенна.  — По идее  и должно  быть тихо.  Ну,
   представь себе начало века. Авиация делает первые шаги. Неуклюжие самолеты
   наконец-таки поднимаются в воздух…  Представляешь, ко всеобщему  восторгу,
   они взлетают на  сто, или  даже на двести  метров. И  вот появляется  дядя
   вроде меня  и  начинает  толковать,  что через  сорок  или  пятьдесят  лет
   винтомоторные самолеты  устареют, наступит  эра реактивной  авиации.  Кого
   заинтересовало бы такое сообщение?..

   Он вдруг замолкает, потом спрашивает, глядя в сторону:

   — Ада летала на реактивных?

   Вообще да, на реактивных. Но там, в Гималаях, разбился вертолет. Осваивали
   новую линию.

   Этот автобус еле тащился. Уж он-то устарел не только морально.

   — А почему ты решил строить прогнозирующую машину? — спрашиваю я.

   — Просто однажды я  подумал: интересно, каким  будет двадцать первый  век?
   Ведь это интересно; ты же сам  говорил, что иногда хочется высунуть нос  в
   четвертое измерение.

   Вот  оно  что.  В  один  прекрасный  день  Антенна,  вечно  занятый  своей
   электроникой, оглянулся  и с  удивлением заметил,  что вокруг  целый  мир,
   который —  о  великое открытие!  —  даже  имеет свое  прошлое  и  будущее…
   Прошлое,  разумеется,   мрачно:   граждане   не   умели   делать   простых
   супергетеродинных приемников.  Зато впереди  великое будущее.  Все  станет
   подвластным радио: радиоастрономия, радиохимия, радиобиология, может быть,
   даже радиоматематика?.. И Антенна выбрал самый простой способ увидеть этот
   радиомир: построил прогнозирующую машину.

   — Нет же, совсем не так!

   На нас  оглядываются: уж  очень энергично  Антенна машет  своими  длинными
   руками.

   — Нет… Мне захотелось  узнать, что будут делать  люди. Вот! Тут  множество
   проблем, но все они в конечном счете сводятся к одной. Ну, понимаешь,  как
   меридианы пересекаются на полюсе. Вопрос в том…

   — Быть или не быть. Знаю. Тише.

   Мальчишка, восторженный  мальчишка. Есть  логика в  том, что  он попал  на
   завод игрушек.

   — Ты  послушай. Либо  доступный  нам мир  ограничен солнечной  системой  и
   остается только благоустраивать этот мир, либо возможны полеты к  звездам,
   и тогда открыт безграничный простор для деятельности человека.

   —  Мягко  говоря,   вопрос  не   слишком  актуальный.   Так  сказать,   не
   животрепещущий.

   — Почему? Вся история человечества — это процесс расширения границ  нашего
   мира. Научились  строить корабли  — завоевали  океан. Изобрели  самолет  —
   завоевали воздух. Создали ракеты — началось завоевание солнечной  системы…
   А если дальше нет пути? Ведь это надо знать…

   Что ж, в конечном счете Антенна прав: наш мир может существовать только  в
   развитии. Если он замкнется в каких-то границах, вырождение неизбежно.  Со
   времен Уэллса  на эту  тему написана  бездна романов.  Но нам  еще ох  как
   далеко до границ  солнечной системы!  Перед нами тысячи  других проблем  —
   накаленных, неотложных…

   А впрочем, кто знает. Меридианы  на экваторе параллельны друг другу:  они,
   наверное, не подозревают о существовании полюса и думают, что  пересекутся
   где-то в бесконечности. Очень может быть, что проблема возвращения  станет
   актуальной значительно раньше, чем мы думаем.

   С чего  я, собственно,  взял,  что Антенна  неудачник? Пожалуй,  он  нашел
   задачу по своему таланту — вот в чем дело.

   — Пришлось бросить аспирантуру, — рассказывает Антенна. — Мой шеф  заявил,
   что двадцать  первый век,  звездолеты и  все такое  прочее —  это  детские
   игрушки, а тема  должна быть  реальной. Ладно, я  подал заявление:  «Прошу
   отпустить на завод игрушек». А потом решил, что и в самом деле нужно пойти
   на такой завод.

   — И тогда, — говорю я  бесцветным голосом летописца, — тогда началась  эра
   электронной игрушки…

   — Нет, при чем здесь электроника! — перебивает Антенна. Он хороший парень,
   ему очень хочется рассеять  мое заблуждение. —  Электронные игрушки —  это
   чепуха, понимаешь, чепуха. Подделка сегодняшней техники. А хорошая игрушка
   должна быть прообразом машин, которые появятся через сотни лет.

   Я пренебрежительно  фыркаю.  Теперь  Антенна  просто  сгорает  от  желания
   втолковать мне что к чему.

   — Это же закономерность! Вспомни хотя бы турбинку Герона. Игрушка! А разве
   гироскоп не  был  игрушкой, волчком?  Иди  первые роботы,  ведь  это  были
   забавные механические  игрушки. Разве  не так?  Пойми ты  наконец:  машина
   почти всегда  появляется  сначала  в виде  игрушки.  Игрушечные  самолеты,
   например, взлетели  раньше  настоящих.  Если  я  хочу  сегодня  заниматься
   машинами двадцать первого века, мне надо делать игрушки.

   — Боюсь, это будут сложные игрушки.

   — Нет, по  идее хорошая игрушка  всегда проста. Поэтому  она долго  живет.
   Скажем, калейдоскоп — ведь так просто…

   Автобус останавливается, мы проталкиваемся  к выходу. Детские игрушки  как
   прообраз техники будущего.  Ну-ну… Я начинаю  понимать, почему Антенна  не
   удержался в аспирантуре.

   Мы идем по мокрой от тающего снега улице, и Антенна, досадливо отмахиваясь
   от бьющих  в лицо  снежинок, говорит  об игрушках.  Где-то вдалеке  звучит
   музыка. Ритмично кружатся снежинки,  кружатся, летят, исчезают в  темноте.
   Вот так —  под музыку  — летели миллимикробные  приемнички, когда  Антенну
   толкнули и коробка упала.

   Это было  на большой  перемене, все  бросились закрывать  окна, а  Антенна
   пошел к доске  и стал  писать формулы. Потом  он заявил,  что нам  крышка,
   поскольку помещение настроено на резонанс.  Он даже объяснил ход  решения,
   чтобы мы  не  сомневались.  Никто и  не  сомневался,  прекрасно  слышалась
   оперная музыка.  Римский-Корсаков, «Сказание  о невидимом  граде Китеже  и
   деве Февронии». Я хотел  сбегать домой за пылесосом,  но было уже  поздно,
   перемена кончилась.  Географу  (ов отличался  изрядной  доверчивостью)  мы
   заявили,  что  это   громкоговоритель  в  доме   напротив.  Пол-урока   мы
   продержались.  «Происходит  что-то  странное,  —  сказал  в  конце  концов
   географ. — Музыка идет у меня с левого рукава. Если рукав поднести к  уху,
   музыка заметно усиливается…» Фехтование развивает феноменальную  выдержку:
   Ада совершенно  спокойно  ответила,  что  надо  подуть  на  рукав  и  звук
   исчезнет. Звук  как  бы  улетит,  сказала она.  Но  географ  уже  вдумчиво
   поглядывал на Антенну, и дальше все  пошло как положено: Антенна встал  и,
   виновато моргая честными синими глазами, признался.

   Я это  «Сказание»  хорошо запомнил.  Там  есть такое  место,  хор  дружины
   Всеволода «Поднялась с полуночи», это просто могуче срезонировало…

   Летом, уже после того, как Антенна уехал, я несколько раз заходил в школу.
   В нашем классе сделали ремонт. Покрасили,  побелили. Но когда на улице  не
   очень шумели и в коридоре никто не разговаривал, слышно было, как работают
   рассыпанные весной приемнички. Где-то они  еще оставались! Звук был  очень
   тихий и потому  таинственный: шепотом говорила  женщина и тревожно  играла
   далекая музыка.

   Прогнозирующая машина выглядит вполне благообразно. Гладкая зеленая панель
   из стеклопластика поднимается от пола до потолка, оставляя узкий проход  к
   окну. Начинка машины находится там, за  панелью, а на самой панели  только
   щит  с  двумя  десятками   обычных  контрольных  приборов  и   стандартным
   акустическим блоком от электронного анализатора «Брянск».

   — Летом я кое-что разобрал, —  говорит Антенна, — сейчас она не  работает,
   но решение записано на магнитофоне, мы прокрутим.

   — Решение…  Слушай,  оно  без  всех  этих  условных  нуль-транспортировок,
   настоящее решение?

   — Да. Она выдала отличную идею, совершенно неожиданную. Конечно, это  лишь
   идея, но…

   — А зачем понадобилось разбирать?

   — Ну… Она прогнозировала только конечный результат.

   Ах, какая нехорошая машина! Она прогнозировала только конечный  результат.
   Все-таки Антенна варвар.

   — Она  говорила «что»  и  не говорила  «как», —  старательно  втолковывает
   Антенна. — И потом энергия… Она брала уйму энергии, у меня уходила на  это
   половина зарплаты.

   В машине  почти  сорок кубометров.  Антенна  мог вместить  туда  чудовищно
   много. Считается, что такую машину удастся построить лет через  пятьдесят,
   не раньше. Мне приходят  на память слова  Конан-Дойля: «Впрочем, никто  не
   понимает истинного значения  того времени, в  котором он живет.  Старинные
   мастера рисовали харчевни и святых Севастьянов, когда Колумб на их  глазах
   открыл Новый Свет».

   — А Почему бы тебе не перейти в мою лабораторию?

   Антенна не любит, не умеет отказывать — и сейчас ему не так просто сказать
   «нет».

   — Ты не обижайся… У тебя мне пришлось бы заниматься такими машинами…  Нет,
   ты, пожалуйста, правильно меня  пойми и не  обижайся. Машина —  пройденный
   этап; она  дала идею,  и теперь  я иду  дальше. В  сущности, машину  можно
   совсем разобрать, она  уже не  нужна. Я хочу  заниматься только  проблемой
   возвращения…

   Знаешь,  кого  я  недавно  видел  в  кинохронике?  —  спрашивает  Антенна.
   Вероятно, ему кажется,  что он  тонко переводит разговор  на другую  тему.
   Архипыча. Вот. Он большими делами заворачивает… Толковый парень, а?

   Тоже новость: я знал,  что Архипыч пойдет в  гору. Он великий артист,  наш
   Архипыч.  Всю  жизнь   он  играет  большого   организатора.  Боже,   какие
   вдохновляющие речи он произносил на школьных воскресниках!

   — Пойдем, —  Антенна тянет  меня за  рукав. —  Я научу  тебя варить  кофе.
   Понимаешь, я сделал в этой  области эпохальное открытие: кофе надо  варить
   на токах высокой частоты…

   Теперь  я  знаю,  как  выглядит  жилище  человека,  занятого  классической
   проблемой возвращения. Одна комната  отдана прогнозирующей машине,  другая
   превращена в нечто среднее между мастерской и лабораторией.

   Антенна живет  в  кухне. Собственно,  ничего  кухонного здесь  нет.  Кухня
   переоборудована в маленькую комнатку: тахта, кресло, крохотный столик.  На
   стенах шесть огромных цветных снимков: заря, облака, радуга, снова облака…

   Облака хороши. Но жизнь у Антенны нелегкая, это очевидно. Зарабатывает  он
   раз в пять меньше, чем  мог бы, и работает раза  в два больше, чем  должен
   был бы. Впрочем,  без всякого  ханжества я могу  сказать: дело  даже не  в
   этом. Самое трудное в положении  Антенны — сохранить веру в  необходимость
   своей работы.  Ведь было  бы страшной  трагедией проработать  вот так  лет
   тридцать, а потом увидеть, что все это мираж.

   Я с трудом представляю такой образ жизни. Вот Антенна приходит с завода  и
   принимается за  классическую проблему  возвращения. Проблему,  считающуюся
   нерешимой. День за днем, год за годом — вне мира науки. Никто не вникает в
   его работу. Никто  не может сказать  — верны его  идеи или нет.  Не с  кем
   поспорить и  поделиться мыслями,  потому  что специалисты  по  межзвездным
   перелетам появятся лишь в следующем веке.

   Вряд ли я смог бы так работать.

   Но вот у меня возникает странная мысль.  Почему я не могу так? Где,  когда
   разошлись наши пути?

   Кандидатская  диссертация,   добротные  научные   работы,  почти   готовая
   докторская диссертация — я делал, что мог. Все правильно. Так почему, черт
   побери, я завидую Антенне? Глупость, «завидую»  совсем не то слово, в  нем
   нехороший привкус. Просто появилась странная мысль. Я действительно делал,
   что мог. А он делает невозможное. Дороги расходятся там, где один выбирает
   достижимую цель, а другой наперекор всему — логике, черту, дьяволу —  идет
   к невозможному. Так или не так?

   Не могут же  все идти к  невозможному. А собственно,  почему? Быть  может,
   именно в этом и состоит  пафос великих революций, поднимающих всех,  всех,
   всех на штурм невозможного? Придется  подумать. Да, у Антенны талант,  это
   так. Но главное в другом: Антенна замахнулся на невозможное…

   А снимки  на стенах  хороши. Они  как окна,  открытые в  головокружительно
   глубокое небо.

   — Я сейчас работаю над небом, — говорит Антенна. — Понимаешь, по идее  это
   будет целый  набор игрушек…  Ну, чтобы  можно было  сделать небо  не  хуже
   настоящего. И  вот  облака  пока  не удаются.  Оптикой  я  занялся  совсем
   недавно, там  хитрейшая  механика. Облака  получаются  как вата  —  серые,
   скучные…

   Значит, облака пока не удаются. А что удается?

   — Что удается? — переспрашивает Антенна и сразу забывает о высокочастотном
   кофе. — Сейчас увидишь. Я покажу тебе радугу, это интересно…

   В тот день, когда мы с Адой пошли в «Художественный», а потом пережидали в
   метро дождь и я поцеловал Аду, была радуга. Единственная радуга, которую я
   запомнил на  всю  жизнь. Другие  радуги  вспоминаются как-то  вообще,  они
   похожи друг на друга, а та совсем особенная. Неужели я говорил  Антенне?..
   Нет, просто совпадение. Это было в девятом классе, когда Антенна уехал. Ну
   да, почти через год после его отъезда.

   Антенна ставит лампу в  угол, на пол. Потом  отходит на середину  комнаты,
   достает из  кармана серебристый  цилиндрик,  резко им  взмахивает,  словно
   встряхивая термометр.

   Вспыхивает алая дуга,  похожая на  плотную прозрачную  ленту. Дуга  быстро
   приобретает глубину, становится  воздушной. Внутренний  край ее  зеленеет,
   расплывается и почти одновременно проступает третий цвет фиолетовый.

   Это колдовство, невозможное  колдовство… Передо  мной пылает  великолепная
   многоцветная радуга.

   Все предметы в комнате теряют очертания, куда-то отступают. Они  исчезают,
   их просто не существует. Реальна только эта радуга — влажная, пронзительно
   чистая, выкованная из чистейших красок.

   Она вот-вот исчезнет. Она дрожит от малейшего движения воздуха. Мерцают  и
   дышат нежные краски. В них  шум уходящей грозы, тревожные всполохи  молний
   над горизонтом, отблески солнца в пенистых дождевых лужах.

   Я протягиваю руку. «Не надо!»  — вскрикивает Антенна. На мгновение  радуга
   загорается ослепительным медно-желтым огнем и сразу же исчезает.

   — Ее нельзя трогать,  — виновато говорит Антенна.  — Если не трогать,  она
   держится долго, минут  пятнадцать-двадцать. Было еще  полярное сияние,  по
   его забрал соседский мальчишка.

   Комната, в  которой мы  варим кофе,  изрядно загромождена.  Книжные  полки
   вдоль двух  стен (на  полках  вперемежку —  книги и  игрушки).  Объемистый
   электронно-вычислительный сундук; на  нем спит  ободранный пятнистый  кот.
   Верстак,  заваленный   инструментами.   Стол  с   какой-то   полусобранной
   электронной тумбой.

   Я копаюсь в  книгах. Хаос.  Только на  одной полке  порядок: здесь  книги,
   относящиеся к  проблеме  возвращения. Я  беру  наугад томик  в  аккуратном
   муаровом переплете и открываю там, где лежит закладка.

   «Все эти проекты путешествий по вселенной — за исключением полетов  внутри
   солнечной системы — стоит выбросить в мусорную корзинку».

   Ого,  как   энергично!   «Радиоастрономия  и   связь   через   космическое
   пространство» Парселла. Ладно, возьмем что-нибудь другое.

   «Нельзя обойти эти трудности, и нет никакой надежды преодолеть их…»

   Это монография Корнера  о межзвездных перелетах.  Что еще?  Фантастический
   роман.

   «При гигантском разрыве во времени теряется сам смысл полетов: астронавт и
   планета начинают жить несообщающейся, бесплодной друг для друга жизнью».

   — Ободряющее чтение. Зачем ты это коллекционируешь?

   — А как же! — удивляется Антенна. — Надо знать, что думают другие…

   Думают! Меня в таких романах раздражает, что там как раз не думают.  Автор
   долго  заверяет,  что  вот  академик  А.  сверхгениален,  а  профессор  Б.
   сверхмудр, затем они начинают говорить — и какое же это наивное вяканье!

   — Тут есть свои трудности, — говорит Антенна. — Современник Эдгара По  мог
   поверить в воздушный шар на две тысячи человек и не поверил бы в  самолет.
   Он даже не  стал бы  это читать: сложно,  непонятно, утомительно.  Видишь,
   какая хитрость… Наивные идеи оказываются художественно убедительнее, в них
   легче поверить. Возьми, например, наш  разговор. Наверное, со стороны  это
   выглядит скучновато. Как раз потому, что мы всерьез пытаемся высунуть  нос
   в будущее.

   Возможно. Куда занимательнее, если Антенна  сделает десяток роботов и  они
   взбунтуются, а  мы  станем  палить  в  них.  И  тут  обязательно  появятся
   пришельцы из  космоса. И  выяснится, что  это они  перехватили  управление
   роботами. А Антенна окажется не Антенной, а человеком, прибывшим к нам  из
   будущего на машине времени, чтобы предупредить о коварных пришельцах.  Вот
   будет славно!..

   Я машинально перебираю книги. «Черное облако» Хойла.

   «Мало интересного  можно придумать,  например,  о машинах.  Очевидно,  что
   машины и различные приборы будут с течением времени делаться все сложнее и
   совершеннее. Ничего неожиданного здесь нет».

   Не знаю, какое отношение это имеет к межзвездным перелетам. Но вообще Хойл
   прав. К сожалению, это аксиома: ничего сверхнеожиданного в развитии  машин
   не предвидится. Меняются  двигатели, улучшается автоматика,  синтетические
   материалы постепенно вытесняют металл…  Что еще? Да и  вообще — что  такое
   неожиданное? Вот если бы все машины стали жидкими, вот тогда…

   — Ты поройся вот  там, — Антенна показывает  на верхнюю полку.  Посмотришь
   Сережкину книгу, он за нее доктора получил. Ничего нельзя понять —  латынь
   и все такое прочее. Восходящее светило хирургии… Там и обе твои книги.

   Антенна учился с нами всего год, а помнит почти всех; этого я не ожидал.

   Значит, Сережка-доктор  стал доктором.  Хорошо! Исцелитель  и  благодетель
   бродячих собак нашего района…

   Не так  просто добраться  до  верхней полки,  она  под самым  потолком.  Я
   взбираюсь на подоконник  — и  вижу микроскоп. Он  стоит в  самом углу,  за
   книгами. Добрый  старый  микроскоп, подаренный  Антенне  четырнадцать  лет
   назад.

   В тот раз, когда  мы рассыпали приемнички и  из-за этого сорвались  уроки,
   завуч произнес громовую речь.  В ней упоминались  четверки по поведению  и
   была нарисована довольно убедительная картина нашего мрачного будущего.  В
   заключение  завуч  запретил  Антенне   делать  маленькие  приемники.   «Мы
   оборудуем тебе  герметичный закуток  в мастерской,  — сказал  завуч.  Там,
   пожалуйста, делай что хочешь. А  без спецусловий разрешаю собирать  только
   крупные приемники. Размером с книжный шкаф, не меньше».

   С этого времени Антенна начал как-то странно поглядывать на книжные шкафы.
   На автобусы и троллейбусы он тоже смотрел странно. А один раз я видел, как
   он странно смотрит на высотный дом на площади Восстания.

   Однажды, когда мы возвращались из школы, Антенна сказал:

   — Знаешь,  можно  вылепить такой  большой  приемник, что  он  одновременно
   окажется и самым маленьким.

   Я решил, что Антенна  изобрел надувной приемник. В  самом деле, почему  бы
   при появлении завуча не увеличивать размеры приемника?

   Мы пришли  к Антенне  домой, и  тут  выяснилось, что  дело не  в  надувных
   приемниках. Антенна выгрузил из  карманов свой радиохлам, повозился  минут
   двадцать, потом объявил:

   — Сейчас ты сидишь внутри  трехлампового приемника. Брось книгу, ведь  это
   такой момент…

   Он был очень доволен собой, а это случалось редко.

   — Вот, квартира  вместо коробки. Обе  комнаты, кухня и  ванная. Все  очень
   просто. Были бы лампы и провода,  схему можно собрать из любых  предметов.
   Видишь, цветочные горшки? Они вместо сопротивлении…

   И я понял, что Антенна не шутит. Приемник и в самом деле большой, а  места
   не занимает.  Правда,  звук  был  какой-то  скрипучий.  Антенна  бегал  по
   комнатам, растягивал провода, вытаскивал продукты из холодильника, который
   тоже входил в схему…

   Мы не  стали делать  уроки, а  отправились бродить  по городу,  и  Антенна
   придумывал разные приемники. Сначала он прикидывал, удастся ли  превратить
   в приемник  школьное здание.  Тут, однако,  возникли трудности  с  крышей.
   Антенна сказал,  что она  сделана  не из  того  материала и  вообще  лучше
   использовать более крупные детали. Мы пошли к метро — выбирать  подходящий
   микрорайон. Но  по дороге  Антенна  заявил, что  ничего не  получится:  от
   транспорта и вообще от линий электропередач будет масса помех. Мы  немного
   посидели в садике, и Антенна, чтобы рассеяться, сделал приемник из четырех
   березок, двух скамеек и цветочной клумбы.

   — Это все-таки  больше книжного шкафа,  — сказал он.  — И вообще  пустяки,
   продержится до  первого  дождя.  Или  до утра.  Роса  утром  будет.  Давай
   послушаем спортивные известия…

   Антенна сидит на  низенькой скамеечке  в конце коридора,  под вешалкой.  В
   руках у него что-то вроде микророяля: черная плоская коробка с  маленькими
   белыми клавишами. В противоположном конце коридора, на полу, чайное блюдце
   с серым,  как  пепел, порошком.  Из  порошка торчат  три  короткие  медные
   проволочки. На пороге, выжидающе глядя на блюдце, стоит тощий исцарапанный
   кот.

   — Дистанционное кормление диких зверей? — спрашиваю я.

   — Да,  зверей,  — машинально  отвечает  Антенна. Он  старательно  нажимает
   клавиши: так начинающий пианист разыгрывает гаммы.

   Серый порошок  на блюдце  шевелится, ползет  вверх по  проволочкам, и  они
   сразу становятся  похожими на  узловатые корни…  Корни быстро  срастаются,
   образуя  какую-то  фигурку.  Еще  мгновение,  порошок  взвихривается  —  и
   застывает.

   На блюдце стоит медвежонок.

   Я беру статуэтку  в руки,  у нее твердая  шероховатая поверхность.  Внутри
   статуэтка полая, это чувствуется по весу.

   — Положи, еще не все…

   Я кладу медвежонка на блюдце. Антенна  нажимает на клавиши — и  медвежонок
   рассыпается:  снова  горстка   серого  порошка  и   три  короткие   медные
   проволочки.

   — А теперь  будет осел,  — торжественно объявляет  Антенна. —  То есть  не
   осел, а такой небольшой ослик.

   Появляется ослик. Довольно похожий.

   — Пока только две программы, —  вздыхает Антенна. — Вообще это тоже  будет
   целый набор:  ВинниПух,  Кролик,  Пятачок, Тигра  и  остальные.  Пока  вот
   выпускаем ослика  Иа… Можно  еще заставить  его ходить.  Правда, ходит  он
   неважно.

   Серый ослик, забавно  покачиваясь, медленно  идет на  негнущихся ногах.  К
   нему бесшумно устремляется  кот. Одной  рукой я  успеваю оттолкнуть  кота,
   другой подхватываю ослика.

   Кажется, я  догадываюсь,  как это  устроено.  Серый порошок  скорее  всего
   какой-нибудь  ферромагнитный  сплав.  Магнитное  поле  заставляет  частицы
   порошка расположиться в определенном порядке, спрессовывает их, получается
   прочная фигурка.

   — Почти так,  — подтверждает  Антенна. — Сейчас  странно —  почему это  не
   придумали раньше… Понимаешь,  в первом приемнике  Попова был когерер.  Ну,
   такая стеклянная  трубка с  железными  опилками. Под  действием  радиоволн
   опилки слипались. Выходит, принцип был известен еще в прошлом веке.

   — Занятная игрушка…

   — Игрушка? — Он с недоумением смотрит на меня. — Но ведь так можно  менять
   машины.

   — Так? Ну, нет!

   Я популярно объясняю: современная машина, например, автомобиль, сделана из
   множества различных материалов. В том числе — немагнитных.

   Антенна сразу уступает.

   — Я просто подумал… Понимаешь, может  быть, машины потому и сложны, что  в
   них множество различных материалов?.. Но вообще-то я не спорю.

   Я внимательно рассматриваю ослика. Прочно сделано! Трудно поверить, что он
   может ходить. А ведь это мысль!

   Машина, сделанная  из  серого  порошка  и  электромагнитного  поля,  будет
   чрезвычайно простой. Ей, например, не нужны винтовые соединения, не  нужны
   шарниры:  под  действием  поля   металл  может  мгновенно  менять   форму…
   Меняющийся металл — вот в чем дело.

   — Худсовет не утвердил медвежонка, —  говорит Антенна. — Сказали, что  это
   формализм. Почему,  мол,  медвежонок  серый.  А  что  я  мог  ответить?  В
   дальнейшем, используя этот принцип,  можно будет лепить  и бурых и  черных
   медведей, а пока только так. Ведь и радуга основана на том же принципе…

   Определенно, это мысль!

   Потребовался бы довольно сложный механизм,  чтобы ослик мог ходить. А  тут
   ничего нет!  Магнитное  поле  исчезает  на  сотую  долю  секунды,  порошок
   начинает рассыпаться, но вновь возникает поле, подхватывает порошок, и  он
   затвердевает — уже в  другом положении. Как  на двух соседних  кинокадрах:
   ослик сделал какую-то часть шага…

   Мы  создали  сложнейшую  технику,  наша  цивилизация  обросла   миллионами
   всевозможных приборов, машин,  сооружений. Казалось, с  машинами не  может
   быть ничего неожиданного. Они будут  становиться сложнее, совершеннее —  и
   только. Чушь! Мы  рисуем харчевни  и святых  Севастьянов, а  кто-то в  это
   время открывает новый мир…

   Меняющийся мир,  дело не  только в  машинах.  Весь мир!  В нем  все  будет
   способно к постоянному изменению: дома, мосты, города, корабли, самолеты…

   —  Ну,  мы  смонтировали  на  заводе  приличную  установку,  —  продолжает
   Антенна. — Не  дистанционную, там это  не нужно. И  выпускаем. В  «Детском
   мире»  на  витрине  точно  такой  ослик…  Вот  и  кофе  готов.  Я  принесу
   магнитофон, послушаешь ответ машины.

                                     * * *

   Мы пьем кофе, сваренный токами высокой  частоты. Он не лучше обычного.  Но
   Антенна им явно гордится, и  я хвалю: отличный кофе, совершенно  особенный
   кофе, пожалуй, такого мне еще не приходилось пить…

   — Какое у тебя  напряжение в сети? —  спрашивает Антенна. — Я  обязательно
   сделаю тебе такую кофеварку.

   — Напряжение…

   К черту кофеварку! Вот здесь висела  радуга. Она светилась изнутри и  была
   совсем живая. Ни за что не спрошу, как это делается. Колдовство.  Грустное
   колдовство.  Эта  радуга  напоминает  мне  ту,  давнюю,  хотя  они  совсем
   непохожи.

   В тот день  мы пошли  с Адой в  «Художественный» на  дневной сеанс,  чтобы
   проверить ее идею  о телепатии. Идея  казалась мне вполне  правдоподобной.
   Опыты, говорила  Ада,  ставились  на  одном  человеке,  поэтому  результат
   получался неопределенный. Надо взять  пятьсот или тысячу, чтобы  сложением
   усилить слабый эффект. Конечно, при опыте люди должны одновременно  думать
   о чем-то одном.  В «Художественном»  шел шведский  детективный фильм,  это
   было очень  удачно:  преступник там  неожиданно  врывался в  купе  поезда,
   стрелял в  сыщика —  и зал  «синхронно и  синфазно» замирал  от ужаса.  Мы
   сидели в углу,  на нас  не обращали внимания.  Ада заткнула  уши руками  и
   закрыла глаза.  Она должна  была  телепатически уловить  этот  контрольный
   момент. На экране  мчался поезд, сыщик  ходил по вагонам,  а я смотрел  на
   Аду, на ее лице мелькали тени…

   Опыт в тот раз не получился. Аде надоело закрывать уши, и она сказала, что
   не обязательно сидеть так с самого начала, достаточно приготовиться, когда
   приблизится время.  Картину  мы знали  только  по пересказам  и,  конечно,
   пропустили контрольный момент.

   Потом мы прятались в метро от дождя.  Мы ездили наугад по разным линиям  и
   все смотрели на входящих, гадая по их лицам и одежде — кончился ли  дождь.
   Бывают же  дожди, которые  идут  долго! Но  этот  кончился через  час  или
   полтора. Вышли мы на «Измайловской» и сразу увидели радугу. Она висела над
   непросохшим еще проспектом, похожая на гигантский арочный мост. Под мостом
   бесшумно скользили  колонны мокрых  автомобилей. Ада  сказала, что  радугу
   лишь слегка наметили  неяркими акварельными красками.  Я согласился,  хотя
   радуга была очень ясная. Особенно ее верхняя часть. И только у  основания,
   там, где арка опиралась о  крыши далеких домов, краски действительно  были
   мягкие, приглушенные воздушной дымкой…

   Межзвездные перелеты считались неосуществимыми прежде всего с точки зрения
   энергетики. Нужны миллионы тонн аннигиляционного горючего, чтобы разогнать
   до субсветовой скорости крохотную капсулу с одним-двумя космонавтами.  Для
   хранения горючего (надо еще научиться его получать!) потребуются  какие-то
   специальные устройства, имеющие немалую массу и, следовательно, вызывающие
   необходимость в расходе  дополнительного горючего. А  чтобы разогнать  это
   дополнительное горючее, опять-таки нужно новое горючее…

   Машина не решала эту часть проблемы. Антенна исходил из того, что  корабль
   будет непрерывно получать энергию с Земли.

   «Энергетический  запрет»  межзвездных  перелетов  возник,  когда  лазерная
   техника была еще  в пеленках.  Впрочем, уже тогда  говорили о  возможности
   использования лазеров для связи с кораблями. Разумеется, совсем не  просто
   перейти от информационной связи к энергетической. Тут есть свои трудности,
   но в  принципе они  преодолимы. По  мере развития  квантовой оптики  будет
   увеличиваться мощность, которую способны передавать лазеры. К тому же  для
   разгона  или  торможения  корабля  —  одного  только  корабля,  без   этих
   колоссальных запасов горючего —  потребуется не так  уж много энергии.  «Я
   выбрал этот вариант  из уважения  к закону сохранения  энергии», —  сказал
   Антенна. Что ж, с этим можно согласиться.

   Но  остается  главное:  классическая  проблема  возвращения.  Время   идет
   по-разному на Земле и на корабле; нужно принять это или спорить с  теорией
   относительности. Машина не спорила. Теперь я  понимаю, что она и не  могла
   спорить, она не была на это рассчитана.

   Вот ответ, я переписал его с магнитофонной ленты.

   «Корабль: перестройка в  полете на основе  полученной с Земли  информации.
   Цель —  возвращение  корабля  на Землю  неустаревшим.  Экипаж:  постоянный
   контакт с Землей,  усвоение с  помощью гипнопедии  возможно более  широкой
   информации о  жизни на  Земле, овладение  новыми профессиями.  Специальные
   передачи, подготавливающие к восприятию новой эпохи».

   Антенна пояснил эту идею таким примером.

   Допустим, три каравеллы уходят в кругосветное плавание, которое  продлится
   несколько лет. Допустим также,  что на берегу за  это время пройдут  века.
   Каравеллы выходят  в  океан, и  через  месяц  или два  моряки  получают  с
   голубиной почтой чертежи  усовершенствованной парусной  оснастки. На  ходу
   начинается изготовление  новых парусов.  Еще  через три  месяца  голубиная
   почта  (к   концу  путешествия   ее   сменит  радио)   приносит   описание
   навигационных приборов,  изобретенных  после  отплытия  эскадры.  С  точки
   зрения моряков  время на  берегу идет  быстро: все  чаще и  чаще  приходят
   сообщения о новых открытиях и  изобретениях. Пристав к какому-то  острову,
   мореплаватели берутся за переустройство кораблей. И вот уже нет  каравелл:
   от острова отплывают два брига. А свободные от вахты моряки изучают  схемы
   первых, еще  неуклюжих,  паровых  двигателей, и  боцманы  роются  в  своем
   хозяйстве, прикидывая, из чего можно будет сделать гребные колеса…

   Применительно   к   каравеллам   этот   мысленный   эксперимент   выглядит
   фантастично. Иное  дело  —  космические корабли,  поддерживающие  связь  с
   Землей, соединенные с ней информационным и энергетическим мостами. Пусть к
   родным берегам вернется не атомоход, а каравелла с паровым двигателем. Все
   равно: люди, построившие этот двигатель, будут ближе к атомному пеку,  чем
   к эпохе, в которую они начинали плавание.

   Я думаю, коэффициент перестройки может быть очень высок. Если бы речь  шла
   только о корабле, коэффициент был бы  близок к единице. Сложнее с  людьми.
   Экипаж Магеллана, пожалуй, еще мог бы  освоить паровую технику, но как  бы
   эти люди, исправно верящие в догматы церкви, восприняли мир без религии?..

   Впрочем, современный  человек значительно  лучше подготовлен  к  возможным
   изменениям. Мы с детства  привыкаем к жизни в  меняющемся мире. Тем  более
   это должно быть присуще звездоплавателям XXI века.

   И все-таки это будет титанический труд — вот так лететь к звездам.

   Я вспоминаю  наивную  фантастику.  Автоматы ведут  корабль.  Экипаж  мирно
   дремлет в анабиозных ваннах: пусть  скорее проходит время… Медленно  текут
   пустые годы… Воздушный шар на две тысячи персон!

   Все будет иначе.

   Каждодневный труд —  в стремительном  темпе, чтобы не  отстать от  земного
   времени, чтобы  освоить и  использовать новые  знания. Вот  почему  машина
   упомянула про гипнопедию: нужен максимум новых знаний в считанные  минуты.
   Скорее всего это будет даже не гипнопедия. Разумеется, не гипнопедия:  тут
   требуются принципиально новые,  еще не  открытые средства  обучения. И  не
   только  обучения,  но   и  вообще  освоения   полученной  с  Земли   самой
   разнообразной информации.  Эти  средства должны  создать  для  космонавтов
   «эффект присутствия», как можно полнее связать их с меняющейся Землей.

   Удивительный будет полет! Сейчас даже трудно представить.

   Год  за  годом  —  перестройка  корабля.  Перестройка   исследовательского
   оборудования. Короткие  часы отдыха.  Сеансы  «телесна» (сегодня  еще  нет
   подходящего термина), когда за  несколько минут человек переживает  земную
   неделю с ее событиями, впечатлениями, новыми знаниями…

   Материалы, собранные на  чужих планетах, не  будут лежать мертвым  грузом:
   постоянно обновляемые знания обеспечат цепную реакцию исследований.

   И когда такой корабль  вернется на Землю,  сделанные экипажем открытия  не
   окажутся устаревшими. Они будут на уровне нового времени.

                                     * * *

   — Вот видишь, — Антенна показывает штамп на левом заднем копытце ослика. —
   Артикул 2908, цена тридцать две копейки. Конечно, у ослика Иа должен  быть
   унылый вид, но худсовет счел, что это излишне.

   — Переходи ко мне.  У нас нет худсовета.  Серьезно говорю: переходи в  мою
   лабораторию.

   — Не-ет. Ну что я буду там делать?

   — Да хотя бы меняющиеся  машины. Возьмешь для начала какую-нибудь  простую
   машину и…

   — Нет. Я хочу заниматься  проблемой возвращения. Меняющиеся конструкции  —
   только часть этой проблемы.

   — Все равно. Хватит кустарничать…

   Я пытаюсь убедить Антенну. Я выкладываю  довод за доводом. И не сразу,  ох
   как не сразу приходит мысль: да ведь это жестоко! Для меня наш спор только
   логическая перестрелка, а Антенна защищает  свои жизненные позиции. Как  я
   об этом не подумал  сразу! Таким людям, как  Антенна, больше всего  портят
   кровь не враги, а благожелатели. Хотят, чтобы все было нормально,  обычно,
   как положено. Какая глупость спасать Антенну от того, что делает его жизнь
   исключительной!

   — Ты, пожалуйста, не обижайся, — говорит Антенна. — Я тебе все объясню.

   Ну, ну, объясняй.

   — Смотри. Вот фронт науки, он идет вперед, — Антенна берет кофейные  чашки
   и показывает, как  это происходит.  — Можно  двигаться с  этим фронтом.  А
   можно уйти в десант и высадиться где-то далекодалеко.

   — И давно ты… высадился?

   — Нет. Лет шесть.

   — А сколько нужно ждать, пока фронт подойдет?

   Антенна недоумевающе смотрит на меня.

   — Не знаю… Какое это имеет значение?

   Я ничего не  отвечаю. У меня  просто не хватает  духа сказать: «Ты  далеко
   высадился, дружище. Слишком далеко от сегодняшнего фронта науки. Наверное,
   на расстоянии целой жизни».

   Антенна по-своему  истолковывает мое  молчание  и говорит,  что,  конечно,
   сделано пока мало.

   — В сущности, это лишь  краешек идеи, — говорит  он. — До полного  решения
   проблемы еще очень  далеко. Как  oт простенького  стробоскопа до  «Латерны
   магики». Вот, кстати, еще один пример: кино тоже началось с игрушки,  ведь
   стробоскоп был детской игрушкой!..

   Послезавтра я вылетаю в Прагу. Почему  я об этом подумал? Ах да,  «Латерна
   магика». Что ж, на симпозиуме много  работы, но «Латерну» я, пожалуй,  еще
   раз посмотрю. Я припоминаю программу симпозиума: да, на третий день  можно
   будет выкроить время для «Латерны». Антенна, конечно, не видел  «Латерны»,
   можно и не спрашивать.

   — Так я  тебе не досказал,  — продолжает Антенна.  — Значит, на  худсовете
   меня спрашивают: «А разве этот ваш Иа никогда не был веселым?» Я  отвечаю:
   «Был. Однажды он подумал, что у Пятачка  в голове только опилки, да и  те,
   очевидно, попали туда по  ошибке. От этой мысли  ему стало весело». Ну,  а
   они мне говорят…

                                     * * *

   Не так просто отыскать ослика. Он в глубине витрины. Серый скромный ослик,
   не идущий ни в какое  сравнение с блестящими лакированными автомобилями  и
   яркими пластмассовыми кораблями.

   Игрушка.

   Жаль, что Антенна не  перейдет в мою лабораторию,  жаль. Время одиночек  в
   науке миновало.

   А собственно, почему?

   В первооснове  совершенно верная  мысль,  не надо  только доводить  ее  до
   абсурда. Да, время одиночек миновало: в том смысле, что Антенна не смог бы
   собрать свою машину, не используя труд и идеи других людей. Над повышением
   емкости машинной памяти работали десятки институтов, они создали биоблоки,
   которые Антенна применил в своей машине. Однако выбирать дальние  проблемы
   и искать их решение  чаще всего приходится в  одиночку. Антенна прав:  тут
   просто еще  нечего  делать  целому коллективу.  Это  дальняя  разведка,  и
   незачем (да и невозможно) ходить в нее всей армией.

   Нет, время таких одиночек не миновало!

   Чем быстрее наступает  наука, тем  важнее для  нее разведка.  Но даже  при
   самой совершенной организации науки разведке будет нелегко. Она отыскивает
   десять разных путей, а наука потом выбирает один наилучший… Один разведчик
   объявляется гением,  девять  —  неудачниками. Это  несправедливо,  ах  как
   несправедливо! Худо было  бы науке без  этих неудачников, отдающих  жизнь,
   чтобы наступающая армия  знала дороги,  по которым  нельзя идти.  Впрочем,
   десятого тоже считают не  слишком удачливым: опередил  свое время, не  был
   признан…

   Не беда,  разведка!  Иди  вперед, остальное  неважно.  А  ослика  напрасно
   поставили так  далеко.  Он  совсем  неплох,  у  него  лукавая  физиономия.
   Когда-нибудь эта игрушка преобразует мир.

   Салют, разведка! Иди вперед, остальное неважно.

   …Третий час ночи, я стою здесь  уже минут двадцать, со стороны это  должно
   выглядеть странно. И  поскольку в  этом мире  все закономерно,  появляется
   милиционер. Молоденький и вежливый. Он  доброжелательно смотрит на меня  и
   на витрину.

   Сумасшедший день.  Сегодня я  далеко высунул  нос в  четвертое  измерение.
   Многое мне еще не ясно, но  я начинаю понимать главное. Машины  приобретут
   бессмертие. Машины  — в  широком смысле  слова: от  величайших  инженерных
   сооружений до  безделушек.  Весь  мир созданной  нами  техники.  Он  будет
   рассыпаться в пепел, прах — и  тут же возникать снова: разумнее,  сильнее,
   красивее. Архитектура, которая была застывшей музыкой, превратится в живую
   музыку!.. Меняющийся  мир  будет бесконечно  шире,  ярче. И  что  особенно
   важно: человек  в  этом  мире  перестанет  зависеть  от  множества  быстро
   стареющих вещей.

   — Видите ослика?  — спрашиваю  я милиционера.  — Вон  там маленький  серый
   ослик… Артикул 2908. Цена тридцать две копейки. У него великое будущее.

   — У осликов это  бывает, — соглашается милиционер.  — У них иногда  бывает
   очень большое будущее.

   Он умен, этот парень. Я поясняю, что речь идет не об ослах. Серый ослик не
   имеет к ним решительно никакого отношения. Просто в нем воплощено  научное
   открытие.

   — Об  открытиях  как-то  удобнее размышлять  днем,  —  осторожно  замечает
   милиционер.

   Вот это уже заблуждение! День  слишком конкретен. Днем удобнее  наблюдать,
   экспериментировать, вычислять. Ночью — искать общие закономерности, делать
   выводы. Что ж,  это мысль! Днем  вы ясно видите  множество окружающих  вас
   предметов, взгляд  не проникает  вдаль,  кругозор, в  сущности,  ограничен
   несколькими десятками метров. Ночью иначе: темнота скрывает все обыденное,
   привычное, и взгляд беспрепятственно устремляется вперед. Горизонт  теряет
   резкие очертания, уходит в глубь бездонного неба. И уже не стены  комнаты,
   не фасады  двух-трех ближайших  зданий, а  само звездное  небо  становится
   границей вашего мира…

   Пора идти. Может быть,  превратить ослика в медвежонка?  В кармане у  меня
   лежит коробочка  с  клавишами.  Нет,  не  надо.  Не  буду  огорчать  этого
   симпатичного парня.

Я прощаюсь и иду вниз, в  сторону улицы Горького. Падает мягкий теплый
снег.  На витрине,  среди нарядных  игрушек, остался  невзрачный серый
ослик, у которого великое будущее.
